\chapter{鄘柏舟詁訓傳第四}

\begin{quoting}\textbf{釋文}鄭云「紂都以南曰鄘」,王云「王城以西曰鄘也」。\end{quoting}

\section{柏舟}

%{\footnotesize 二章、章七句}

\textbf{柏舟,共姜自誓也。衛世子共伯蚤死,其妻守義,父母欲奪而嫁之,誓而弗許,故作是詩以絕之。}{\footnotesize 共伯,僖侯之世子。}

\textbf{汎彼柏舟,在彼中河。}{\footnotesize 興也。中河,河中。箋云舟在河中,猶婦人之在夫家,是其常處。}\textbf{髧彼兩髦,實維我儀。}{\footnotesize 髧,兩髦之貌。髦者,髮至眉,子事父母之飾。儀,匹也。箋云兩髦之人謂共伯也,實是我之匹,故我不嫁也。禮,世子昧爽而朝,亦櫛纚筓緫拂髦冠緌纓。}\textbf{之死矢靡它。}{\footnotesize 矢誓、靡無、之至也。至己之死,信無它心。}\textbf{母也天只,不諒人只。}{\footnotesize 諒,信也。母也天也,尚不信我。天,謂父也。}

\begin{quoting}髧 \texttt{dàn},齊詩韓詩作紞,玉篇「髧,垂髮貌」。實,同寔,是也。儀 \texttt{é},同偶。釋文「諒,本亦作亮」。\end{quoting}

\textbf{汎彼柏舟,在彼河側。髧彼兩髦,實維我特。}{\footnotesize 特,匹也。}\textbf{之死矢靡慝。}{\footnotesize 慝,邪也。}\textbf{母也天只,不諒人只。}

\begin{quoting}\textbf{馬瑞辰}特訓獨,又訓匹者,皆以相反為義也。慝 \texttt{tè},同忒,說文「忒,更也」。\end{quoting}

\section{牆有茨}

%{\footnotesize 三章、章六句}

\textbf{牆有茨,衛人刺其上也。公子頑通乎君母,國人疾之而不可道也。}{\footnotesize 宣公卒,惠公幼,其庶兄頑烝於惠公之母,生子五人,齊子、戴公、文公、宋桓夫人、許穆夫人。}

\begin{quoting}魯詩說「衛宣姜亂及三世,至戴公而後寧」。\end{quoting}

\textbf{牆有茨,不可埽也。}{\footnotesize 興也。牆,所以防非常。茨,蒺藜也。欲埽去之,反傷牆也。箋云國君以禮防制一國,今其宮內有淫昏之行者,猶牆之生蒺藜。}\textbf{中冓之言,不可道也。}{\footnotesize 中冓,內冓也。箋云內冓之言,謂宮中所冓成頑與夫人淫昏之語。}\textbf{所可道也,言之醜也。}{\footnotesize 於君醜也。}

\begin{quoting}\textbf{陳奐}中冓與牆對稱,牆為宮牆,則中冓當為宮中之室。\end{quoting}

\textbf{牆有茨,不可襄也。}{\footnotesize 襄,除也。}\textbf{中冓之言,不可詳也。}{\footnotesize 詳,審也。}\textbf{所可詳也,言之長也。}{\footnotesize 長,惡長也。}

\textbf{牆有茨,不可束也。}{\footnotesize 束而去之。}\textbf{中冓之言,不可讀也。}{\footnotesize 讀,抽也。箋云抽,猶出也。}\textbf{所可讀也,言之辱也。}{\footnotesize 辱,辱君也。}

\begin{quoting}\textbf{王先謙}束是總集之義,總聚而去之,言其淨盡也,較埽、襄義又進。\textbf{胡承珙}蓋道者約言之,詳者多言之,讀者反覆言之。\end{quoting}

\section{君子偕老}

%{\footnotesize 三章、一章七句、一章九句、一章八句}

\textbf{君子偕老,刺衛夫人也。夫人淫亂,失事君子之道,故陳人君之德、服飾之盛,宜與君子偕老也。}{\footnotesize 夫人,宣公夫人,惠公之母也。人君,小君也,或者小字誤作人耳。}

\textbf{君子偕老,副筓六珈。}{\footnotesize 能與君子俱老,乃宜居尊位,服盛服也。副者,后夫人之首飾,編髮為之。筓,衡筓、珈筓,飾之最盛者,所以別尊卑。箋云珈之言加也,副既筓而加飾,如今步搖上飾,古之制所有未聞。}\textbf{委委佗佗,如山如河,}{\footnotesize 委委者,行可委曲從迹也。佗佗者,德平易也。山無不容,河無不潤。}\textbf{象服是宜。}{\footnotesize 象服,尊者所以為飾。箋云象服者,謂榆翟、闕翟也,人君之象服,則舜所云「予欲觀古人之象,日月星辰」之屬。}\textbf{子之不淑,云如之何。}{\footnotesize 有子若是,何謂不善乎。箋云子乃服飾如是,而為不善之行,於禮當如之何,深疾之。}

\begin{quoting}說文「筓,簪也」。委委佗佗,應作委佗委佗。象服,褘衣,說文「褘,王后之服」。\end{quoting}

\textbf{玼兮玼兮,其之翟也。}{\footnotesize 玼,鮮盛貌。榆翟、闕翟,羽飾衣也。箋云侯伯夫人之服,自榆翟而下,如王后焉。}\textbf{鬒髮如雲,不屑髢也。}{\footnotesize 鬒,黑髮也。如雲,言美長也。屑,潔也。箋云髢,髲也。不潔者不用髲為善。}\textbf{玉之瑱也,象之揥也。}{\footnotesize 瑱,塞耳也。揥,所以摘髮也。}\textbf{揚且之晳也。}{\footnotesize 揚,眉上廣。皙,白皙。}\textbf{胡然而天也,胡然而帝也。}{\footnotesize 尊之如天,審諦如帝。箋云胡,何也。帝,五帝也。何由然女見尊敬如天帝乎,非由衣服之盛、顏色之莊與,反為淫昏之行。}

\begin{quoting}\textbf{朱熹}翟衣,祭服。鬒 \texttt{zhěn},說文引詩作㐱,稠髮也。髢 \texttt{dì},三家詩作鬄,說文「鬄,髲也,髲,益髮也」。瑱 \texttt{tiàn},說文「以玉充耳也」。揥 \texttt{tì},搔首、搔頭也。\textbf{馬瑞辰}按清揚皆美貌之稱,野有蔓草詩「清揚婉兮、婉如清揚」,此泛言貌之美也,猗嗟詩「美目揚兮、美目清兮」,此專言目之美也,此詩「揚且之晳也」,晳謂色白,又曰「子之清揚,揚且之顏也」,則顏色之美皆可曰清揚矣。\textbf{陳奐}古而、如通用。\end{quoting}

\textbf{瑳兮瑳兮,其之展也。蒙彼縐絺,是紲袢也。}{\footnotesize 禮有展衣者,以丹縠為衣。蒙,覆也。絺之靡者為縐,是當暑袢延之服也。箋云后妃六服之次展衣,宜白,縐絺,絺之蹙蹙者,展衣,夏則裏衣縐絺,此以禮見於君及賓客之盛服也。展衣字誤,禮記作襢衣。}\textbf{子之清揚,揚且之顏也。}{\footnotesize 清,視清明也。揚,廣揚而顏角豐滿。}\textbf{展如之人兮,邦之媛也。}{\footnotesize 展,誠也。美女為媛。箋云媛者,邦人所依倚以為媛助也。疾宣姜有此盛服而以淫昏亂國,故云然。}

\begin{quoting}說文段注「玼之或體作瑳 \texttt{cuō}」。紲袢 \texttt{xiè pàn},三家詩作褻,即內衣。\end{quoting}

\section{桑中}

%{\footnotesize 三章、章七句}

\textbf{桑中,刺奔也。衛之公室淫亂,男女相奔,至于世族在位相竊妻妾,期於幽遠,政散民流而不可止。}{\footnotesize 衛之公室淫亂,謂宣惠之世男女相奔,不待媒氏以禮會之也。世族在位,取姜氏、弋氏、庸氏者也。竊,盜也。幽遠,謂桑中之野。}

\textbf{爰采唐矣,沬之鄉矣。}{\footnotesize 爰,於也。唐,蒙菜名。沬,衛邑。箋云於何采唐,必沬之鄉,猶言欲為淫亂者,必之衛之都,惡衛為淫亂之主。}\textbf{云誰之思,美孟姜矣。}{\footnotesize 姜,姓也。言世族在位有是惡行。箋云淫亂之人誰思乎,乃思美孟姜。孟姜,列國之長女而思與淫亂,疾世族在位有是惡行。}\textbf{期我乎桑中,要我乎上宮,送我乎淇之上矣。}{\footnotesize 桑中上宮,所期之地。淇,水名也。箋云此思孟姜之愛厚己也,與我期於桑中而要見我於上宮,其送我則於淇水之上。}

\begin{quoting}沬,亦作湏,衛都朝歌,商代稱妹邦、牧野,見泉水注,在今淇縣北。期,會也。要,邀也。\textbf{馬瑞辰}「孟子之滕,館于上宮」,趙岐章句曰「上宮,樓也」,古者宮室通稱,此上宮亦即樓耳。\end{quoting}

\textbf{爰采麥矣,沬之北矣。云誰之思,美孟弋矣。}{\footnotesize 弋,姓也。}\textbf{期我乎桑中,要我乎上宮,送我乎淇之上矣。}

\begin{quoting}沬北,即邶地也。\textbf{朱熹}弋,春秋或作姒,蓋杞女夏后氏之後。\end{quoting}

\textbf{爰采葑矣,沬之東矣。}{\footnotesize 箋云葑,蔓菁。}\textbf{云誰之思,美孟庸矣。}{\footnotesize 庸,姓也。}\textbf{期我乎桑中,要我乎上宮,送我乎淇之上矣。}

\begin{quoting}沬東,即鄘地也。\end{quoting}

\section{鶉之奔奔}

%{\footnotesize 二章、章四句}

\textbf{鶉之奔奔,刺衛宣姜也。衛人以為宣姜鶉鵲之不若也。}{\footnotesize 刺宣姜者,刺其與公子頑為淫亂行,不如禽鳥。}

\begin{quoting}左傳襄二十七年「伯有賦鶉之賁賁,趙孟曰『牀笫之言不踰閾,況在野乎,非使人之所得聞也』」。\end{quoting}

\textbf{鶉之奔奔,鵲之彊彊。}{\footnotesize 鶉則奔奔,鵲則彊彊然。箋云奔奔彊彊,言其居有常匹,飛則相隨之貌,刺宣姜與頑非匹偶。}\textbf{人之無良,我以為兄。}{\footnotesize 良,善也。兄,謂君之兄。箋云人之行無一善者,我君反以為兄,君謂惠公。}

\begin{quoting}奔奔、彊彊,齊詩魯詩作賁賁、姜姜。\end{quoting}

\textbf{鵲之彊彊,鶉之奔奔。人之無良,我以為君。}{\footnotesize 君,國小君。箋云小君,謂宣姜。}

\section{定之方中}

%{\footnotesize 三章、章七句}

\textbf{定之方中,美衛文公也。衛為狄所滅,東徙渡河,野處漕邑,齊桓公攘戎狄而封之,文公徙居楚丘,始建城市而營宮室,得其時制,百姓說之,國家殷富焉。}{\footnotesize 春秋閔公二年冬,狄人入衛,衛懿公及狄人戰于熒澤而敗,宋桓公迎衛之遺民渡河,立戴公以廬於漕,戴公立一年而卒,魯僖公二年,齊桓公城楚丘而封衛,於是文公立而建國焉。}

\textbf{定之方中,作于楚宮。}{\footnotesize 定營、宮室也。方中,昏正四方。楚宮,楚丘之宮,仲梁子曰「初立楚宮也」。箋云楚宮,謂宗廟也。定星昏中而正,於是可以營制宮室,故謂之營室。定昏中而正,謂小雪時,其體與東壁連正四方。}\textbf{揆之以日,作于楚室。}{\footnotesize 揆,度也。度日出、日入以正東西,南視定、北準極以正南北。室,猶宮也。箋云楚室,居室也。君子將營宮室,宗廟為先,廐庫為次,居室為後。}\textbf{樹之榛栗,椅桐梓漆,爰伐琴瑟。}{\footnotesize 椅,梓屬。箋云爰,曰也。樹此六木於宮者,曰其長大可伐以為琴瑟,言豫備也。}

\begin{quoting}定之方中,營室黃昏見於正南也,春秋僖二年「春王正月,城楚丘」,周之正月,夏十一月也,楚丘,在今滑縣東。于,三家詩作為,儀禮士冠禮鄭注「于,猶為也」。\textbf{馬瑞辰}椅桐梓漆皆為琴瑟之用,若榛栗則無與於琴瑟也。\end{quoting}

\textbf{升彼虛矣,以望楚矣。望楚與堂,景山與京。}{\footnotesize 虛,漕虛也。楚丘有堂邑者。景山,大山。京,高丘也。箋云自河以東,夾於濟水,文公將徙,登漕之虛以望楚丘,觀其旁邑及其丘山,審其高下所依倚,乃後建國焉,慎之至也。}\textbf{降觀于桑。}{\footnotesize 地勢宜蠶,可以居民。}\textbf{卜云其吉,終然允臧。}{\footnotesize 龜曰卜。允信、臧善也。建國必卜之,故建邦能命龜、田能施命、作器能銘、使能造命、升高能賦、師旅能誓、山川能說、喪紀能誄、祭祀能語,君子能此九者,可謂有德音,可以為大夫也。}

\begin{quoting}\textbf{王先謙}景,當讀為憬,泮水傳「憬,遠行貌」,與上升望、下降觀相屬為義,毛訓大,於文不順。爾雅釋丘「絕高為之京」,郭注「為之者,人力所作也」。「其吉,終然允臧」卜辭也。\end{quoting}

\textbf{靈雨既零,命彼倌人。星言夙駕,說于桑田。}{\footnotesize 零,落也。倌人,主駕者。箋云靈,善也。星,雨止星見。夙,早也。文公於雨下,命主駕者「雨止,為我晨早駕」,欲往為辭說於桑田,教民稼穡,務農急也。}\textbf{匪直也人,}{\footnotesize 非徒庸君。}\textbf{秉心塞淵。}{\footnotesize 秉,操也。箋云塞,充實也。淵,深也。}\textbf{騋牝三千。}{\footnotesize 馬七尺曰騋。騋馬與牝馬也。箋云國馬之制,天子十有二閑,馬六種,三千四百五十六匹,邦國六閑,馬四種,千二百九十六匹,衛之先君兼邶鄘而有之而馬數過禮制,今文公滅而復興,徙而能富,馬有三千,雖非禮制,國人美之。}

\begin{quoting}釋文引韓詩「星,晴也」。說,同稅,史記索引「税駕,猶解駕,言休息也」。塞淵,見燕燕注。\end{quoting}

\section{蝃蝀}

%{\footnotesize 三章、章四句}

\textbf{蝃蝀,止奔也。衛文公能以道化其民,淫奔之耻,國人不齒也。}{\footnotesize 不齒者,不與相長稚。}

\textbf{蝃蝀在東,莫之敢指。}{\footnotesize 蝃蝀,虹也。夫婦過禮則虹氣盛,君子見戒而懼諱之,莫之敢指。箋云虹,天氣之戒,尚無敢指者,況淫奔之女,誰敢視之。}\textbf{女子有行,遠父母兄弟。}{\footnotesize 箋云行,道也。婦人生而有適人之道,何憂於不嫁而為淫奔之過乎,惡之甚。}

\begin{quoting}蝃蝀 \texttt{dì dòng},魯詩作蝃作螮。\end{quoting}

\textbf{朝隮于西,崇朝其雨。}{\footnotesize 隮升、崇終也。從旦至食時為終朝。箋云朝有升氣於西方,終其朝則雨,氣應自然,以言婦人生而有適人之道,亦性自然。}\textbf{女子有行,遠兄弟父母。}

\begin{quoting}隮 \texttt{jī},周禮注「隮,虹也,詩云『朝隮于西』」。\textbf{陳啟源}螮蝀在東,暮虹也,朝隮于西,朝虹也,暮虹截雨,朝虹行雨,屢騐皆然,雖兒童婦女皆知也。\end{quoting}

\textbf{乃如之人也,懷昬姻也。}{\footnotesize 乃如是淫奔之人也。箋云懷,思也。乃如是之人思昏姻之事乎,言其淫奔之過惡之大。}\textbf{大無信也,不知命也。}{\footnotesize 不待命也。箋云淫奔之女大無貞潔之信,又不知昏姻當待父母之命,惡之也。}

\begin{quoting}\textbf{王先謙}懷,蓋壞之借字,左襄十四年傳「王室之不壞」,釋文「壞,本作懷」,荀子禮論篇「諸侯不敢壞」,史記禮書作懷,是其證。\end{quoting}

\section{相鼠}

%{\footnotesize 三章、章四句}

\textbf{相鼠,刺無禮也。衛文公能正其群臣而刺在位,承先君之化無禮儀也。}

\textbf{相鼠有皮,人而無儀。}{\footnotesize 相,視也。無禮儀者,雖居尊位,猶為闇昧之行。箋云儀,威儀也。視鼠有皮,雖處高顯之處,偷食苟得,不知廉耻,亦與人無威儀者同。}\textbf{人而無儀,不死何為。}{\footnotesize 箋云人以有威儀為貴,今反無之,傷化敗俗,不如其死,無所害也。}

\begin{quoting}何,魯詩作胡。\end{quoting}

\textbf{相鼠有齒,人而無止。}{\footnotesize 止,所止息也。箋云止,容止,孝經曰「容止可觀」。}\textbf{人而無止,不死何俟。}{\footnotesize 俟,待也。}

\begin{quoting}\textbf{王先謙}韓說曰「止,節,無禮節也」,凡有所自處自禁者皆謂之止,禮大學「在止於至善」注「止,猶自處也」,淮南時則訓「止獄訟」注「止,猶禁也」,是其證,故止訓節,而無止即無禮節也。\end{quoting}

\textbf{相鼠有體,人而無禮。}{\footnotesize 體,支體也。}\textbf{人而無禮,胡不遄死。}{\footnotesize 遄,速也。}

\begin{quoting}禮記禮運鄭注「言鼠之有身體,如人而無禮者矣」。胡,三家詩作何。\end{quoting}

\section{干旄}

%{\footnotesize 三章、章六句}

\textbf{干旄,美好善也。衛文公臣子多好善,賢者樂告以善道也。}{\footnotesize 賢者,時處士也。}

\begin{quoting}干旄,所以招賢也,\textbf{崔述}吾讀干旄之篇,而知衛之所以久存,良有由也。\end{quoting}

\textbf{孑孑干旄,在浚之郊。}{\footnotesize 孑孑,干旄貌。注旄於干首,大夫之旃也。浚,衛邑。古者臣有大功,世其官邑。郊外曰野。箋云周禮「孤卿建旃,大夫建物」,首皆注旄焉。時有建此旄來至浚之郊,卿大夫好善也。}\textbf{素絲紕之,良馬四之。}{\footnotesize 紕,所以織組也,緫紕於此,成文於彼,願以素絲紕組之法御四馬也。箋云素絲者以為縷,以縫紕旌旗之旒縿,或以維持之。浚郊之賢者既識卿大夫建旄而來,又識其乘善馬。四之者,見之數也。}\textbf{彼姝者子,何以畀之。}{\footnotesize 姝,順貌。畀,予也。箋云時賢者既說此卿大夫有忠順之德,又欲以善道與之,心誠愛厚之至。}

\begin{quoting}說文「孑 \texttt{jié},無又臂也」,段注「引伸之,凡特立為孑」。干,三家詩作竿。\textbf{酈道元}浚城距楚丘只二十里。\textbf{郝懿行}是衣裳緣邊俱曰紕也。\textbf{王念孫}四馬,大夫以備贈遺者,下文或五或六,隨所見言之,不專是自乘,左昭十六年傳,鄭六卿餞韓宣子於郊,宣子皆獻馬焉,是以馬贈遺,古有是禮。\textbf{孔廣森}四之、五之、六之,不當以轡為解,乃聘賢者用馬為禮,轉益其庶且多也,左傳王賜虢公晉侯馬五匹、楚棄疾遺鄭子皮馬六匹,皆不必成乘,故或五或六也。\end{quoting}

\textbf{孑孑干旟,在浚之都。}{\footnotesize 鳥隼曰旟。下邑曰都。箋云周禮「州里建旟」,謂州長之屬。}\textbf{素絲組之,良馬五之。}{\footnotesize 緫以素絲而成組也。驂馬五轡。箋云以素絲縷縫組於旌旗以為之飾。五之者,亦為五見之也。}\textbf{彼姝者子,何以予之。}

\begin{quoting}爾雅釋天「錯革鳥曰旟」,孫炎「錯,置也,革,急也,畫急疾之鳥於縿也」。\textbf{陳奐}周制,鄉、遂之外置都、鄙,都為畿疆之境名。\end{quoting}

\textbf{孑孑干旌,在浚之城。}{\footnotesize 析羽為旌。城,都城也。}\textbf{素絲祝之,良馬六之。}{\footnotesize 祝,織也。四馬六轡。箋云祝,當作屬,屬,著也。六之者,亦謂六見之也。}\textbf{彼姝者子,何以告之。}

\begin{quoting}左傳定九年「竿旄『何以告之』,取其忠也」,杜注「取其中心願吿人以善道也」。\end{quoting}

\section{載馳}

%{\footnotesize 五章、一章六句、二章章四句、一章六句、一章八句}

\textbf{載馳,許穆夫人作也,閔其宗國顛覆,自傷不能救也。衛懿公為狄人所滅,國人分散,露於漕邑,許穆夫人閔衛之亡,傷許之小,力不能救,思歸唁其兄,又義不得,故賦是詩也。}{\footnotesize 滅者,懿公死也,君死於位曰滅。露於漕邑者,謂戴公也。懿公死,國人分散,宋桓公迎衛之遺民渡河,處之於漕邑而立戴公焉。戴公與許穆夫人俱公子頑烝於宣姜所生也。男子先生曰兄。}

\textbf{載馳載驅,歸唁衛侯。}{\footnotesize 載,辭也。弔失國曰唁。箋云載之言則也。衛侯,戴公也。}\textbf{驅馬悠悠,言至于漕。}{\footnotesize 悠悠,遠貌。漕,衛東邑。箋云夫人願御者驅馬悠悠乎,我欲至於漕。}\textbf{大夫跋涉,我心則憂。}{\footnotesize 草行曰跋,水行曰涉。箋云跋涉者,衛大夫來告難於許時。}

\textbf{既不我嘉,不能旋反。}{\footnotesize 不能旋反我思也。箋云既盡、嘉善也。言許人盡不善我欲歸唁兄。}\textbf{視爾不臧,我思不遠。}{\footnotesize 不能遠衛也。箋云爾,女,女許人也。臧,善也。視女不施善道救衛。}

\begin{quoting}視,比也、效也。爾,韓詩作我。\end{quoting}

\textbf{既不我嘉,不能旋濟。}{\footnotesize 濟,止也。}\textbf{視爾不臧,我思不閟。}{\footnotesize 閟,閉也。}

\begin{quoting}閟 \texttt{bì}。\end{quoting}

\textbf{陟彼阿丘,言采其蝱。}{\footnotesize 偏高曰阿丘。蝱,貝母也。升至偏高之丘采其蝱者,將以療疾。箋云升丘采貝母,猶婦人之適異國,欲得力助安宗國也。}\textbf{女子善懷,亦各有行。}{\footnotesize 行,道也。箋云善,猶多也。懷,思也。女子之多思者有道,猶升丘采蝱也。}\textbf{許人尤之,眾稺且狂。}{\footnotesize 尤,過也。是乃眾幼稺且狂進,取一槩之義。箋云許人,許大夫也。過之者,過夫人之欲歸唁其兄。}

\begin{quoting}蝱 \texttt{méng},魯詩作莔。論語「不尤人」,鄭注「尤,非也」。眾,\textbf{王引之}與終通用,既也。\end{quoting}

\textbf{我行其野,芃芃其麥。}{\footnotesize 願行衛之野,麥芃芃然方盛長。箋云麥芃芃者,言未收刈,民將困也。}\textbf{控于大邦,誰因誰極。}{\footnotesize 控引、極至也。箋云今衛侯之欲求援引之力助於大國之諸侯,亦誰因乎,由誰至乎。閔之,故欲歸問之。}\textbf{大夫君子,無我有尤。}{\footnotesize 箋云君子,國中賢者。無我有尤,無過我也。}\textbf{百爾所思,不如我所之。}{\footnotesize 不如我所思之篤厚也。箋云爾,女,女眾大夫君子也。}

\begin{quoting}芃 \texttt{péng}。\textbf{馬瑞辰}韓詩曰「控,赴也」是也,既夕注「赴,走吿也」,控于大邦即謂走吿于大邦耳,襄八年左傳云「無所控告」,今世興訟者猶稱控告,控告即赴吿也。\textbf{陳奐}至者,當讀如「申包胥以秦師至」。無,毋也。有,又也。\end{quoting}

%\begin{flushright}鄘國十篇、三十章、百七十六句\end{flushright}