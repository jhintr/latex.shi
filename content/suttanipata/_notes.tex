\chapter{凡例}

\begin{itemize}
    \item 底本。\textbf{经集}与其义注\textbf{第一义光}的底本均选自缅甸第六次结集本 CST,间或校以巴利圣典协会 PTS 本。
    \item 英译。参考了 K.R.Norman 的 \textit{The Group of Discourses} 和菩提比丘的 \textit{The Suttanipāta}。
    \item 偈颂编号。两种英译均按 PTS 本,这里为与义注保持一致,而按 CST 本。但 CST 本的\textbf{经集}有两处 \textbf{274} 颂,相应的\textbf{第一义光}则为 \textbf{274~275} 颂,所以这里采用第一义光的编号,似为合理。
    \item 行文。对正文中的偈颂,一般将两句置于一行,这样便于调整译文中词序,只有少数例外,才会更改行间的语序。随后给出义注,并对义注的段落编号,当然,分段是笔者自己的。
    \item 引用三藏。编号尽量按 CST 本,唯\textbf{清净道论}依 Kosambi 本分段,可与 Ñāṇamoli 的英译本一致。有关\textbf{法句}的引用,都来自叶均的译文。
    \item 名相。尽量与叶均所译的\textbf{清净道论}、\textbf{摄阿毗达摩义论}保持一致,人名地名也尽量从旧译中选择合适者,实在找不到对应的旧译,笔者更偏向意译,有些义注内的专名若是不大常见,则直书巴利。
\end{itemize}