\chapter{夙惠第十二}

\subsection*{1}

\textbf{賓客詣陳太丘宿,太丘使元方、季方炊,客與太丘論議,二人進火,俱委而竊聽,炊忘著箄,飯落釜中,太丘問:「炊何不餾?」元方、季方長跪曰:「大人與客語,乃俱竊聽,炊忘著箄,飯今成糜。」太丘曰:「爾頗有所識不?」對曰:「仿佛志之。」二子長跪俱說,更相易奪,言無遺失,太丘曰:「如此,但糜自可,何必飯也?」}

\subsection*{2}

\textbf{何晏七歲,明惠若神,魏武奇愛之,因晏在宮內,欲以為子,晏乃畫地令方,自處其中,人問其故,答曰:「何氏之廬也。」魏武知之,即遣還。}{\footnotesize \textbf{魏略}曰晏父蚤亡,太祖為司空時納晏母,其時秦宜祿阿鰾亦隨母在宮,並寵如子,常謂晏為假子也。}

\subsection*{3}

\textbf{晉明帝數歲,坐元帝厀上,有人從長安來,元帝問洛下消息,潸然流涕,明帝問何以致泣,具以東渡意告之,因問明帝:「汝意謂長安何如日遠?」答曰:「日遠,不聞人從日邊來,居然可知。」元帝異之,明日集群臣宴會,告以此意,更重問之,乃答曰:「日近。」元帝失色,曰:「爾何故異昨日之言邪?」答曰:「舉目見日,不見長安。」}

\subsection*{4}

\textbf{司空顧和與時賢共清言,張玄之、顧敷是中外孫,年並七歲,}{\footnotesize \textbf{顧愷之家傳}曰敷,字祖根,吳郡吳人,滔然有大成之量,仕至著作郎,二十三卒。}\textbf{在牀邊戲,于時聞語,神情如不相屬,瞑於燈下,二兒共敘客主之言,都無遺失,顧公越席而提其耳曰:「不意衰宗復生此寶。」}

\subsection*{5}

\textbf{韓康伯數歲,家酷貧,至大寒,止得襦,母殷夫人自成之,令康伯捉熨斗,謂康伯曰:「且著襦,尋作複褌。」兒云:「已足,不須複褌也。」母問其故,答曰:「火在熨斗中而柄熱,今既著襦,下亦當煗,故不須耳。」母甚異之,知為國器。}

\subsection*{6}

\textbf{晉孝武年十二,時冬天,晝日不著複衣,但著單練衫五六重,夜則累茵褥,謝公諫曰:「聖體宜令有常,陛下晝過冷,夜過熱,恐非攝養之術。」帝曰:「晝動夜靜。」}{\footnotesize \textbf{老子}曰躁勝寒,靜勝熱。此言夜靜寒,宜重肅也。}\textbf{謝公出,歎曰:「上理不減先帝。」}{\footnotesize 簡文帝善言理也。}

\subsection*{7}

\textbf{桓宣武薨,桓南郡年五歲,服始除,桓車騎與送故文武別,}{\footnotesize \textbf{桓沖別傳}曰沖,字玄叔,溫弟也,累遷車騎將軍、都督七州諸軍事。}\textbf{因指與南郡:「此皆汝家故吏佐。」玄應聲慟哭,酸感傍人,車騎每自目己坐曰:「靈寶成人,當以此坐還之。」}{\footnotesize 靈寶,玄小字也。}\textbf{鞠愛過於所生。}