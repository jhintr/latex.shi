\chapter{唐蟋蟀詁訓傳第十}

\begin{quoting}\textbf{釋文}唐者,周成王母弟叔虞所封也,其地帝堯夏禹所都之墟,漢曰太原郡,在古冀州太行恆山之西、太原大岳之野,其南有晉水,叔虞之子燮父因改為晉侯,至六世孫僖侯名司徒,習堯偷約遣化而不能以禮節之,今詩本其風俗,故云唐也。\end{quoting}

\section{蟋蟀}

%{\footnotesize 三章、章八句}

\textbf{蟋蟀,刺晉僖公也。儉不中禮,故作是詩以閔之,欲其及時以禮自虞樂也。此晉也而謂之唐,本其風俗憂深思遠、儉而用禮,乃有堯之遺風焉。}{\footnotesize 憂深思遠,謂「宛其死矣、百歲之後」之類也。}

\textbf{蟋蟀在堂,歲聿其莫。今我不樂,日月其除。}{\footnotesize 蟋蟀,蛩也,九月在堂。聿遂、除去也。箋云我,我僖公也。蛩在堂,歲時之候,是時農功畢,君可以自樂矣,今不自樂,日月且過去,不復暇為之,謂十二月當復命農計耦耕事。}\textbf{無已大康,職思其居。}{\footnotesize 已甚、康樂、職主也。箋云君雖當自樂,亦無甚大樂,欲其用禮為節也,又當主思於所居之事,謂國中政令。}\textbf{好樂無荒,良士瞿瞿。}{\footnotesize 荒,大也。瞿瞿然顧禮義也。箋云荒,廢亂也。良,善也。君之好樂不當至於廢亂政事,當如善士瞿瞿然顧禮義也。}

\begin{quoting}在堂,即詩七月之「九月在戶」,周以夏曆十一月為正月,故云歲暮。無,通毋。\textbf{馬瑞辰}爾雅釋詁「職,常也」,常從尚聲,故職又通作尚。\end{quoting}

\textbf{蟋蟀在堂,歲聿其逝。今我不樂,日月其邁。}{\footnotesize 邁,行也。}\textbf{無已大康,職思其外。}{\footnotesize 外,禮樂之外。箋云外,謂國外至四境。}\textbf{好樂無荒,良士蹶蹶。}{\footnotesize 蹶蹶,動而敏於事。}

\begin{quoting}\textbf{蘇轍}既思其職,又思其職之外。\end{quoting}

\textbf{蟋蟀在堂,役車其休。}{\footnotesize 箋云庶人乘役車。役車休,農功畢,無事也。}\textbf{今我不樂,日月其慆。}{\footnotesize 慆,過也。}\textbf{無已大康,職思其憂。}{\footnotesize 憂可憂也。箋云憂者,謂鄰國侵伐之憂。}\textbf{好樂無荒,良士休休。}{\footnotesize 休休,樂道之心。}

\begin{quoting}\textbf{馬瑞辰}古者役不踰時,月令「孟秋乃命將帥」則孟冬正當旋役之時,采薇詩「曰歸曰歸,歲亦陽止」,杕杜詩「日月陽止,女心傷止,征夫遑止」,皆古者歲莫還役之證。慆 \texttt{tāo},同滔,說文「滔,水漫漫大貌」。\textbf{方玉潤}休休,以安為念,亦懼意也。\end{quoting}

\section{山有樞}

%{\footnotesize 三章、章八句}

\textbf{山有樞,刺晉昭公也。不能修道以正其國,有財不能用,有鐘鼓不能以自樂,有朝廷不能洒掃,政荒民散,將以危亡,四鄰謀取其國家而不知,國人作詩以刺之也。}

\textbf{山有樞,隰有榆。}{\footnotesize 興也。樞,荎也。國君有財貨而不能用,如山隰不能自用其財。}\textbf{子有衣裳,弗曳弗婁。子有車馬,弗馳弗驅。}{\footnotesize 婁,亦曳也。}\textbf{宛其死矣,他人是愉。}{\footnotesize 宛,死貌。愉,樂也。箋云愉,讀曰偷,偷,取也。}

\begin{quoting}婁,韓詩作摟。宛,同苑,淮南子俶真訓「形苑而神壯」,高誘注「苑,枯病也」。\end{quoting}

\textbf{山有栲,隰有杻。}{\footnotesize 栲,山樗。杻,檍也。}\textbf{子有廷內,弗洒弗埽。子有鐘鼓,弗鼓弗考。}{\footnotesize 洒,灑也。考,擊也。}\textbf{宛其死矣,他人是保。}{\footnotesize 保,安也。箋云保,居也。}

\begin{quoting}\textbf{王引之}廷內,謂庭與堂室,非謂庭之內也。考,同攷,廣雅「攷,擊也」。\end{quoting}

\textbf{山有漆,隰有栗。子有酒食,何不日鼓瑟。}{\footnotesize 君子無故琴瑟不離於側。}\textbf{且以喜樂,且以永日。}{\footnotesize 永,引也。}\textbf{宛其死矣,他人入室。}

\section{揚之水}

%{\footnotesize 三章、二章章六句、一章四句}

\textbf{揚之水,刺晉昭公也。昭公分國以封沃,沃盛彊,昭公微弱,國人將叛而歸沃焉。}{\footnotesize 封沃者,封叔父桓叔于沃也。沃,曲沃,晉之邑也。}

\textbf{揚之水,白石鑿鑿。}{\footnotesize 興也。鑿鑿然鮮明貌。箋云激揚之水波流湍疾,洗去垢濁,使白石鑿鑿然,興者,喻桓叔盛強,除民所惡,民得以有禮義也。}\textbf{素衣朱襮,從子于沃。}{\footnotesize 襮,領也。諸侯繡黼丹朱中衣。沃,曲沃。箋云繡,當為綃,綃黼丹朱中衣,中衣以綃黼為領,丹朱為純。國人欲進此服,去從桓叔。}\textbf{既見君子,云何不樂。}{\footnotesize 箋云君子,謂桓叔。}

\begin{quoting}襮 \texttt{bó}。于,往也。\end{quoting}

\textbf{揚之水,白石皓皓。}{\footnotesize 皓皓,潔白也。}\textbf{素衣朱繡,從子于鵠。}{\footnotesize 繡,黼也。鵠,曲沃邑也。}\textbf{既見君子,云何其憂。}{\footnotesize 言無憂也。}

\begin{quoting}鵠,齊詩作皋,即曲沃,\textbf{馬瑞辰}鵠,古通作皋,澤也皋也沃也蓋析言則異,散言則通。\end{quoting}

\textbf{揚之水,白石粼粼。}{\footnotesize 粼粼,清澈也。}\textbf{我聞有命,不敢以告人。}{\footnotesize 聞曲沃有善政命,不敢以告人。箋云不敢以告人而去者,畏昭公謂己動民心。}

\begin{quoting}\textbf{嚴粲}言不敢吿人者,乃所以吿昭公。\end{quoting}

\section{椒聊}

%{\footnotesize 二章、章六句}

\textbf{椒聊,刺晉昭公也。君子見沃之盛彊,能脩其政,知其蕃衍盛大,子孫將有晉國焉。}

\begin{quoting}\textbf{應劭}風俗通曰漢官儀,皇后稱椒房,取其蕃實之義也,詩曰「椒聊之實,蕃衍盈升」。\end{quoting}

\textbf{椒聊之實,蕃衍盈升。}{\footnotesize 興也。椒聊,椒也。箋云椒之性芬香而少實,今一捄之實,蕃衍滿升,非其常也,興者,喻桓叔晉君之支別耳,今其子孫眾多,將日以盛也。}\textbf{彼其之子,碩大無朋。}{\footnotesize 朋,比也。箋云之子,是子也,謂桓叔也。碩,謂壯貌,佼好也。大,謂德美廣博也。無朋,平均不朋黨。}\textbf{椒聊且,遠條且。}{\footnotesize 條,長也。箋云椒之氣日益遠長,似桓叔之德彌廣博。}

\begin{quoting}聊,同莍,亦作朻、梂,草木之實聚生成叢。蕃衍,文選引詩作蔓延。且 \texttt{jū}。條,足利本作脩。\end{quoting}

\textbf{椒聊之實,蕃衍盈匊。}{\footnotesize 兩手曰匊。}\textbf{彼其之子,碩大且篤。}{\footnotesize 篤,厚也。}\textbf{椒聊且,遠條且。}{\footnotesize 言聲之遠聞也。}

\section{綢繆}

%{\footnotesize 三章、章六句}

\textbf{綢繆,刺晉亂也。國亂則昬姻不得其時焉。}{\footnotesize 不得其時,謂不及仲春之月。}

\textbf{綢繆束薪,三星在天。}{\footnotesize 興也。綢繆,猶纏綿也。三星,參也。在天,謂始見東方也。男女待禮而成,若薪芻待人事而後束也。三星在天,可以嫁取矣。箋云三星,謂心星也,心有尊卑夫婦父子之象,又為二月之合宿,故嫁取者以為候焉。昏而火星不見,嫁取之時也,今我束薪於野,乃見其在天,則三月之末四月之中見於東方矣,故云不得其時。}\textbf{今夕何夕,見此良人。}{\footnotesize 良人,美室也。箋云今夕何夕者,言此夕何月之夕乎,而女以見良人,言非其時。}\textbf{子兮子兮,如此良人何。}{\footnotesize 子兮者,嗟茲也。箋云子兮子兮者,斥嫁取者,子取後陰陽交會之月,當如此良人何。}

\begin{quoting}儀禮鄭注「婦人稱夫曰良」。\end{quoting}

\textbf{綢繆束芻,三星在隅。}{\footnotesize 隅,東南隅也。箋云心星在隅,謂四月之末五月之中。}\textbf{今夕何夕,見此邂逅。}{\footnotesize 邂逅,解說之貌。}\textbf{子兮子兮,如此邂逅何。}

\begin{quoting}\textbf{朱熹}昏現之星至此,則夜久矣。\textbf{胡承珙}邂逅,會合之意,淮南俶真訓「孰肯解構人間之事」,高注「解構,猶會合也」,凡君臣朋友男女之會合皆可言之,傳云「解悦之貌」,即因會合而心解意悦耳。\end{quoting}

\textbf{綢繆束楚,三星在戶。}{\footnotesize 參星正月中直戶也。箋云心星在戶,謂之五月之末六月之中。}\textbf{今夕何夕,見此粲者。}{\footnotesize 三女為粲,大夫一妻二妾。}\textbf{子兮子兮,如此粲者何。}

\begin{quoting}\textbf{朱熹}戶必南出,昏現之星至此,則夜分矣。\end{quoting}

\section{杕杜}

%{\footnotesize 二章、章九句}

\textbf{杕杜,刺時也。君不能親其宗族,骨肉離散,獨居而無兄弟,將為沃所并爾。}

\textbf{有杕之杜,其葉湑湑。}{\footnotesize 興也。杕,特生貌。杜,赤棠也。湑湑,枝葉不相比也。}\textbf{獨行踽踽,豈無他人,不如我同父。}{\footnotesize 踽踽,無所親也。箋云他人,謂異姓也。言昭公遠其宗族,獨行於國中踽踽然,此豈無異姓之臣乎,顧恩不如同姓親親也。}\textbf{嗟行之人,胡不比焉。}{\footnotesize 箋云君所與行之人,謂異姓卿大夫也。比,輔也。此人女何不輔君為政令。}\textbf{人無兄弟,胡不佽焉。}{\footnotesize 佽,助也。箋云異姓卿大夫,女見君無兄弟之親親者,何不相推佽而助之。}

\begin{quoting}杕 \texttt{dì}。杜,即甘棠也。踽 \texttt{jǔ}。佽 \texttt{cì}。\end{quoting}

\textbf{有杕之杜,其葉菁菁。}{\footnotesize 菁菁,葉盛也。箋云菁菁,希少之貌。}\textbf{獨行睘睘,豈無他人,不如我同姓。}{\footnotesize 睘睘,無所依也。同姓,同祖也。}\textbf{嗟行之人,胡不比焉。人無兄弟,胡不佽焉。}

\begin{quoting}睘睘 \texttt{qióng},魯詩作煢煢。\textbf{馬瑞辰}女生曰姓,此詩同姓,對前章同父者言,又據下文「人無兄弟」而言,同姓蓋謂同母生者。\end{quoting}

\section{羔裘}

%{\footnotesize 二章、章四句}

\textbf{羔裘,刺時也。晉人刺其在位不恤其民也。}{\footnotesize 恤,憂也。}

\textbf{羔裘豹袪,自我人居居。}{\footnotesize 袪,袂也。本末不同,在位與民異心自用也。居居,懷惡不相親比之貌。箋云羔裘豹袪,在位卿大夫之服也,其役使我之民人,其意居居然有悖惡之心,不恤我之困苦。}\textbf{豈無他人,維子之故。}{\footnotesize 箋云此民,卿大夫采邑之民也,故云豈無他人可歸往者乎,我不去者,乃念子故舊之人。}

\begin{quoting}袪 \texttt{qū}。自,用也,\textbf{胡承珙}自我人居居,猶言我人自居居。居,同倨。維,同惟。故,同婟,戀惜也。\end{quoting}

\textbf{羔裘豹褎,自我人究究。}{\footnotesize 褎,猶袪也。究究,猶居居也。}\textbf{豈無他人,維子之好。}{\footnotesize 箋云我不去而歸往他人者,乃念子而愛好之也。民之厚如此,亦唐之遺風。}

\begin{quoting}褎 \texttt{xiù},同袖。爾雅釋訓「居居、究究,惡也」。\end{quoting}

\section{鴇羽}

%{\footnotesize 三章、章七句}

\textbf{鴇羽,刺時也。昭公之後大亂五世,君子下從征役,不得養其父母而作是詩也。}{\footnotesize 大亂五世者,昭公、孝侯、鄂侯、哀侯、小子侯。}

\textbf{肅肅鴇羽,集于苞栩。}{\footnotesize 興也。肅肅,鴇羽聲也。集止、苞稹、栩杼也。鴇之性不樹止。箋云興者,喻君子當居安平之處,今下從征役,其為危苦如鴇之樹止然。稹者,根相迫迮梱致也。}\textbf{王事靡盬,不能蓺稷黍,父母何怙。}{\footnotesize 盬,不攻致也。怙,恃也。箋云蓺,樹也。我迫王事,無不攻致,故盡力焉,既則罷卷,不能播種五穀,今我父母將何怙乎。}\textbf{悠悠蒼天,曷其有所。}{\footnotesize 箋云曷,何也。何時我得其所哉。}

\begin{quoting}爾雅釋言「苞,稹」,孫炎注「物叢生曰苞,齊人名曰稹」。\textbf{馬瑞辰}爾雅釋詁「棲、憩、休、苦,息也」,苦即盬之假借。\end{quoting}

\textbf{肅肅鴇翼,集于苞棘。王事靡盬,不能蓺黍稷,父母何食。悠悠蒼天,曷其有極。}{\footnotesize 箋云極,已也。}

\textbf{肅肅鴇行,集于苞桑。}{\footnotesize 行,翮也。}\textbf{王事靡盬,不能蓺稻粱,父母何嘗。悠悠蒼天,曷其有常。}

\begin{quoting}\textbf{馬瑞辰}鴇行,猶雁行也。\end{quoting}

\section{無衣}

%{\footnotesize 二章、章三句}

\textbf{無衣,美晉武公也。武公始并晉國,其大夫為之請命乎天子之使而作是詩也。}{\footnotesize 天子之使,是時使來者。}

\textbf{豈曰無衣七兮,}{\footnotesize 侯伯之禮七命,冕服七章。箋云我豈無是七章之衣乎。晉舊有之,非新命之服。}\textbf{不如子之衣,安且吉兮。}{\footnotesize 諸侯不命於天子則不成為君。箋云武公初并晉國,心未自安,故以得命服為安。}

\textbf{豈曰無衣六兮,}{\footnotesize 天子之卿六命,車旗衣服以六為節。箋云變七言六者,謙也,不敢必當侯伯,得受六命之服,列於天子之卿,猶愈乎不。}\textbf{不如子之衣,安且燠兮。}{\footnotesize 燠,煖也。}

\begin{quoting}說文「燠,熱在中也」。\end{quoting}

\section{有杕之杜}

%{\footnotesize 二章、章六句}

\textbf{有杕之杜,刺晉武公也。武公寡特,兼其宗族而不求賢以自輔焉。}

\textbf{有杕之杜,生于道左。}{\footnotesize 興也。道左之陽,人所宜休息也。箋云道左,道東也。日之熱恆在日中之後,道東之杜,人所宜休息也,今人不休息者,以其特生陰寡也,興者,喻武公初兼其宗族,不求賢者與之在位,君子不歸,似乎特生之杜然。}\textbf{彼君子兮,噬肯適我。}{\footnotesize 噬,逮也。箋云肯可、適之也。彼君子之人至於此國,皆可來之我君所,君子之人義之與比,其不來者君不求之。}\textbf{中心好之,曷飲食之。}{\footnotesize 箋云曷,何也。言中心誠好之,何但飲食之,當盡禮極歡以待之。}

\begin{quoting}噬,韓詩作逝,語詞。曷,何不也。\end{quoting}

\textbf{有杕之杜,生于道周。}{\footnotesize 周,曲也。}\textbf{彼君子兮,噬肯來遊。}{\footnotesize 遊,觀也。}\textbf{中心好之,曷飲食之。}

\begin{quoting}韓詩「周,右也」。\end{quoting}

\section{葛生}

%{\footnotesize 五章、章四句}

\textbf{葛生,刺晉獻公也。好攻戰,則國人多喪矣。}{\footnotesize 喪,棄亡也。夫從征役棄亡不反,則其妻居家而怨思。}

\textbf{葛生蒙楚,蘞蔓于野。}{\footnotesize 興也。葛生延而蒙楚,蘞生蔓於野,喻婦人外成於他家。}\textbf{予美亡此,誰與獨處。}{\footnotesize 箋云予我、亡無也。言我所美之人無於此,謂其君子也,吾誰與居乎,獨處家耳,從軍未還,未知死生,其今無於此。}

\begin{quoting}\textbf{馬瑞辰}葛與蘞皆蔓草,延於松柏則得其所,今詩言蒙楚、蒙棘、蔓野、蔓域,蓋以喻婦人失其所依。\textbf{陳奐}婦人稱夫謂美,猶稱夫謂良。\end{quoting}

\textbf{葛生蒙棘,蘞蔓于域。}{\footnotesize 域,塋域也。}\textbf{予美亡此,誰與獨息。}{\footnotesize 息,止也。}

\textbf{角枕粲兮,錦衾爛兮。}{\footnotesize 齊則角枕錦衾。禮,夫不在,斂枕篋衾席,韣而藏之。箋云夫雖不在而不失其祭也,攝主,主婦猶自齊而行事。}\textbf{予美亡此,誰與獨旦。}{\footnotesize 箋云旦,明也。我君子無於此,吾誰與齊乎,獨自絜明。}

\textbf{夏之日,冬之夜。}{\footnotesize 言長也。箋云思者於晝夜之長時尤甚,故極言之以盡情。}\textbf{百歲之後,歸于其居。}{\footnotesize 箋云居,墳墓也。言此者,婦人專壹,義之至,情之盡。}

\textbf{冬之夜,夏之日。百歲之後,歸于其室。}{\footnotesize 室,猶居也。箋云室,猶塚壙。}

\section{采苓}

%{\footnotesize 三章、章八句}

\textbf{采苓,刺晉獻公也。獻公好聽讒焉。}

\textbf{采苓采苓,首陽之巔。}{\footnotesize 興也。苓,大苦也。首陽,山名也。采苓,細事也,首陽,幽辟也,細事喻小行也,幽辟喻無徵也。箋云采苓采苓者,言采苓之人眾多非一也,皆云采此苓於首陽山之上,首陽山之上信有苓矣,然而今之采者未必於此山,然而人必信之,興者,喻事有似而非。}\textbf{人之為言,苟亦無信。舍旃舍旃,苟亦無然。}{\footnotesize 苟,誠也。箋云苟,且也。為言,謂為人為善言以稱薦之,欲使見進用也。旃之言焉也。舍之焉舍之焉,謂謗訕人欲使見貶退也,此二者且無信,受之且無答然。}\textbf{人之為言,胡得焉。}{\footnotesize 箋云人以此言來,不信受之,不答然之,從後察之,或時見罪,何所得。}

\begin{quoting}\textbf{陳奐}為,讀作偽也,為言,即讒言,所謂小行無徵之言也,無然,無是也,無是者,無一是者也。\textbf{聞一多}得,取也,與「舍」對,言人之偽言不足取也。\end{quoting}

\textbf{采苦采苦,首陽之下。}{\footnotesize 苦,苦菜也。}\textbf{人之為言,苟亦無與。舍旃舍旃,苟亦無然。}{\footnotesize 無與,勿用也。}\textbf{人之為言,胡得焉。}

\textbf{采葑采葑,首陽之東。}{\footnotesize 葑,菜名也。}\textbf{人之為言,苟亦無從。舍旃舍旃,苟亦無然。人之為言,胡得焉。}

%\begin{flushright}唐國十二篇、三十三章、二百三句\end{flushright}