\chapter{秦車鄰詁訓傳第十一}

\begin{quoting}\textbf{釋文}秦者,隴西谷名也,在雍州鳥鼠山之東北,昔臯陶之子伯翳佐禹治水有功,舜命作虞,賜姓曰贏,其末孫非子為周孝王養馬於汧渭之間,封為附庸,邑于秦谷,及非子之曾孫秦仲,周宣王又命為大夫,仲之孫襄公討西戎救周,周室東遷,以岐豐之地賜之,始列為諸侯,春秋時稱秦伯,崔云「秦在虞夏商為諸侯,至周為附庸」。\end{quoting}

\section{車鄰}

%{\footnotesize 三章、一章四句、二章章六句}

\textbf{車鄰,美秦仲也。秦仲始大,有車馬禮樂侍御之好焉。}

\textbf{有車鄰鄰,有馬白顛。}{\footnotesize 鄰鄰,眾車聲也。白顛,的顙也。}\textbf{未見君子,寺人之令。}{\footnotesize 寺人,內小臣也。箋云欲見國君者,必先令寺人使傳告之,時秦仲又始有此臣。}

\begin{quoting}周官內小臣云「掌王后之命,掌王之陰事陰令」,鄭注「陰事,群妃御見之事」,寺人云「掌王之內人及女宮之戒令」。\end{quoting}

\textbf{阪有漆,隰有栗。}{\footnotesize 興也。陂者曰阪,下濕曰隰。箋云興者,喻秦仲之君臣所有各得其宜。}\textbf{既見君子,竝坐鼓瑟。}{\footnotesize 又見其禮樂焉。箋云既見,既見秦仲也。並坐鼓瑟,君臣以閒暇燕飲相安樂也。}\textbf{今者不樂,逝者其耋。}{\footnotesize 耋,老也,八十曰耋。箋云今者不於此君之朝自樂,謂仕焉而去仕他國,其徒自使老,言將後寵祿也。}

\textbf{阪有桑,隰有楊。既見君子,竝坐鼓簧。}{\footnotesize 簧,笙也。}\textbf{今者不樂,逝者其亡。}{\footnotesize 亡,喪棄也。}

\section{駟驖}

%{\footnotesize 三章、章四句}

\textbf{駟驖,美襄公也。始命有田狩之事、園囿之樂焉。}{\footnotesize 始命,命為諸侯也,秦始附庸也。}

\textbf{駟驖孔阜,六轡在手。}{\footnotesize 驖驪、阜大也。箋云四馬六轡,六轡在手,言馬之良也。}\textbf{公之媚子,從公于狩。}{\footnotesize 能以道媚于上下者。冬獵曰狩。箋云媚於上下,謂使君臣和合也,此人從公往狩,言襄公親賢也。}

\textbf{奉時辰牡,辰牡孔碩。}{\footnotesize 時是、辰時也。冬獻狼,夏獻麋,春秋獻鹿豕群獸。箋云奉是時牡者,謂虞人也。時牡甚肥大,言禽獸得其所也。}\textbf{公曰左之,舍拔則獲。}{\footnotesize 拔,矢末也。箋云左之者,從禽之左射之也。拔,括也。舍拔則獲,言公善射。}

\textbf{遊于北園,四馬既閑。}{\footnotesize 閑,習也。箋云公所以田則克獲者,乃遊于北園之時,時則已習其四種之馬。}\textbf{輶車鸞鑣,載獫歇驕。}{\footnotesize 輶,輕也。獫歇驕,田犬也,長喙曰獫,短喙曰歇驕。箋云輕車,驅逆之車也。置鸞於鑣,異於乘車也。載,始也。始田犬者,謂達其搏噬,始成之也。此皆遊於北園時所為也。}

\begin{quoting}輶 \texttt{yóu}。獫 \texttt{xiǎn}、歇驕皆狗也。\textbf{朱熹}以車載犬,蓋以休其足力也。\end{quoting}

\section{小戎}

%{\footnotesize 三章、章十句}

\textbf{小戎,美襄公也。備其兵甲以討西戎,西戎方彊而征伐不休,國人則矜其車甲,婦人能閔其君子焉。}{\footnotesize 矜,夸大也。國人夸大其車甲之盛,有樂之意也。婦人閔其君子,恩義之至也。作者敘外內之志,所以美君政教之功。}

\textbf{小戎俴收,五楘梁輈。}{\footnotesize 小戎,兵車也。俴淺、收軫也。五,五束也。楘,歷錄也。梁輈,輈上句衡也。一輈五束,束有歷錄。箋云此群臣之兵車,故曰小戎。}\textbf{游環脅驅,陰靷鋈續。}{\footnotesize 游環,靷環也,游在背上,所以禦出也。脅驅,慎駕具,所以止入也。陰,揜軓也。靷,所以引也。鋈,白金也。續,續靷也。箋云游環在背上無常處,貫驂之外轡以禁其出。脅驅者,著服馬之外脅以止驂之入。揜軓在軾前垂輈上。鋈續,白金飾續靷之環。}\textbf{文茵暢轂,駕我騏馵。}{\footnotesize 文茵,虎皮也。暢轂,長轂也。騏,騏文也。左足白曰馵。箋云此上六句者,國人所矜。}\textbf{言念君子,溫其如玉。}{\footnotesize 箋云言,我也。念居子之性溫然如玉。玉有五德。}\textbf{在其板屋,亂我心曲。}{\footnotesize 西戎板屋。箋云心曲,心之委曲也。憂則心亂也。此上四句者,婦人所用閔其君子。}

\begin{quoting}小戎,對「元戎」言。俴 \texttt{jiàn}。\textbf{陳奐}四面束輿之木謂之軫,詩則謂之收,收,聚也,謂聚眾材而收束之也。古人登車必自車後。楘 \texttt{mù},今曰箍。梁輈 \texttt{zhōu},即車轅,以其長易折故箍以五楘。\textbf{胡承珙}蓋靷從輿下而出於軓前,以繫於衡,其革不能如此之長,必須為環以接續之,故曰鋈 \texttt{wù} 續。\textbf{馬瑞辰}說文「曲,像器受物之形」,心之受事有如曲之受物,故稱心曲,猶水涯之受水處亦曰水曲也。\end{quoting}

\textbf{四牡孔阜,六轡在手。騏駵是中,騧驪是驂。}{\footnotesize 黃馬黑喙曰騧。箋云赤身黑鬣曰駵。中,中服也。驂,兩騑也。}\textbf{龍盾之合,鋈以觼軜。}{\footnotesize 龍盾,畫龍其盾也。合,合而載之。軜,驂內轡也。箋云鋈以觼軜,軜之觼以白金為飾也。軜繫於軾前。}\textbf{言念君子,溫其在邑。}{\footnotesize 在敵邑也。}\textbf{方何為期,胡然我念之。}{\footnotesize 箋云方今以何時為還期乎,何以然了不來,言望之也。}

\begin{quoting}駵 \texttt{liú}。騧 \texttt{guā}。觼 \texttt{jué},環之有舌者。軜 \texttt{nà}。\textbf{馬瑞辰}方之言將也。\end{quoting}

\textbf{俴駟孔群,厹矛鋈錞,蒙伐有苑。}{\footnotesize 俴駟,四介馬也。孔,甚也。厹,三隅矛也。錞,鐏也。蒙,討羽也。伐,中干也。苑,文貌。箋云俴,淺也,謂以薄金為介之札。介,甲也。甚群者,言和調也。蒙,厖也,討,雜也,畫雜羽之文於伐,故曰厖伐。}\textbf{虎韔鏤膺,交韔二弓,竹閉緄縢。}{\footnotesize 虎,虎皮也。韔,弓室也。膺,馬帶也。交韔,交二弓於韔中也。閉紲、緄繩、縢約也。箋云鏤膺,有刻金飾也。}\textbf{言念君子,載寢載興。厭厭良人,秩秩德音。}{\footnotesize 厭厭,安靜也。秩秩,有知也。箋云此既閔其君子寢起之勞,又思其性與德。}

\begin{quoting}\textbf{馬瑞辰}釋文「韓詩云,駟馬不著甲曰俴駟」,韓說是也,管子參患篇曰「甲不堅密與俴者同實,將徒人與俴者同實」,注「俴謂無甲單衣也」,又云「俴,單也,人雖眾,無兵甲則與單人同也」,今按人無甲謂之俴,馬無甲亦謂之俴,其義正同。\textbf{王先謙}韓則訓俴為單,謂馬不著甲,以示其驍勇。厹 \texttt{qiú} 矛,亦作仇矛、酋矛。錞 \texttt{duì},矛柄下端。伐,同瞂。苑 \texttt{yūn}。\textbf{嚴粲}鏤膺,鏤飾弓室之膺,弓以後為背,則以前為膺,故弓室之前亦為膺耳。閉,齊詩作柲,即檠,所以調弓也。緄縢 \texttt{gǔn téng}。厭 \texttt{yān}。\end{quoting}

\section{蒹葭}

%{\footnotesize 三章、章八句}

\textbf{蒹葭,刺襄公也。未能用周禮,將無以固其國焉。}{\footnotesize 秦處周之舊土,其人被周之德教日久矣,今襄公新為諸侯,未習周之禮法,故國人未服焉。}

\textbf{蒹葭蒼蒼,白露為霜。}{\footnotesize 興也。蒹薕、葭蘆也。蒼蒼,盛也。白露凝戾為霜然後歲事成,國家待禮然後興。箋云蒹葭在眾草之中蒼蒼然彊盛,至白露凝戾為霜則成而黃,興者,喻眾民之不從襄公政令者,得周禮以教之則服。}\textbf{所謂伊人,在水一方。}{\footnotesize 伊,維也。一方,難至矣。箋云伊,當作繄,繄,猶是也。所謂是知周禮之賢人乃在大水之一邊,假喻以言遠。}\textbf{遡洄從之,道阻且長。}{\footnotesize 逆流而上曰溯洄。逆禮則莫能以至也。箋云此言不以敬順往求之則不能得見。}\textbf{遡游從之,宛在水中央。}{\footnotesize 順流而涉曰溯游。順禮求濟,道來迎之。箋云宛,坐見貌。以敬順求之則近耳,易得見也。}

\begin{quoting}\textbf{馬瑞辰}一方,即一旁也。\end{quoting}

\textbf{蒹葭萋萋,白露未晞。}{\footnotesize 萋萋,猶蒼蒼也。晞,乾也。箋云未晞,未為霜。}\textbf{所謂伊人,在水之湄。}{\footnotesize 湄,水隒也。}\textbf{遡洄從之,道阻且躋。}{\footnotesize 躋,升也。箋云升者,言其難至,如升阪。}\textbf{遡游從之,宛在水中坻。}{\footnotesize 坻,小渚也。}

\begin{quoting}坻 \texttt{chí}。\end{quoting}

\textbf{蒹葭采采,白露未已。}{\footnotesize 采采,猶萋萋也。未已,猶未止也。}\textbf{所謂伊人,在水之涘。}{\footnotesize 涘,厓也。}\textbf{遡洄從之,道阻且右。}{\footnotesize 右,出其右也。箋云右者,言其迂迴也。}\textbf{遡游從之,宛在水中沚。}{\footnotesize 小渚曰沚。}

\begin{quoting}蜉蝣傳「采采,眾多也」。\end{quoting}

\section{終南}

%{\footnotesize 二章、章六句}

\textbf{終南,戒襄公也。能取周地,始為諸侯,受顯服,大夫美之,故作是詩以戒勸之。}

\textbf{終南何有,有條有梅。}{\footnotesize 興也。終南,周之名山中南也。條槄、梅柟也。宜以戒不宜也。箋云問何有者,意以為名山高大,宜有茂木也,興者,喻人君有盛德乃宜有顯服,猶山之木有大小也,此之謂戒勸。}\textbf{君子至止,錦衣狐裘。}{\footnotesize 錦衣,采色也。狐裘,朝廷之服也。箋云至止者,受命服於天子而來也。諸侯狐裘,錦衣以裼之。}\textbf{顏如渥丹,其君也哉。}{\footnotesize 箋云渥,厚漬也。顏色如厚漬之丹,言赤而澤也,其君也哉,儀貌尊嚴也。}

\textbf{終南何有,有紀有堂。}{\footnotesize 紀,基也。堂,畢道平如堂也。箋云畢也堂也亦高大之山所宜有也。畢,終南山之道名,邊如堂之牆然。}\textbf{君子至止,黻衣繡裳。}{\footnotesize 黑與青謂之黻,五色備謂之繡。}\textbf{佩玉將將,壽考不忘。}

\begin{quoting}紀,三家詩作杞。堂,三家詩作棠。將將,魯詩作鏘鏘。考,老也。\end{quoting}

\section{黃鳥}

%{\footnotesize 三章、章十二句}

\textbf{黃鳥,哀三良也。國人刺穆公以人從死而作是詩也。}{\footnotesize 三良,三善臣也,謂奄息、仲行、鍼虎也。從死,自殺以從死。}

\textbf{交交黃鳥,止于棘。}{\footnotesize 興也。交交,小貌。黃鳥以時往來得其所,人以壽命終亦得其所。箋云黃鳥止于棘以求安己也,此棘若不安則移,興者,喻臣之事君亦然,今穆公使臣從死,刺其不得黃鳥止于棘之本意。}\textbf{誰從穆公,子車奄息。}{\footnotesize 子車,氏。奄息,名。箋云言誰從穆公者,傷之。}\textbf{維此奄息,百夫之特。}{\footnotesize 乃特百夫之德。箋云百夫之中最雄俊也。}\textbf{臨其穴,惴惴其慄。}{\footnotesize 惴惴,懼也。箋云穴,謂塚壙中也。秦人哀傷此奄息之死,臨視其壙,皆為之悼慄。}\textbf{彼蒼者天,殲我良人。}{\footnotesize 殲盡、良善也。箋云言彼蒼者天,愬之。}\textbf{如可贖兮,人百其身。}{\footnotesize 箋云如此奄息之死可以他人贖之者,人皆百其身,謂一身百死猶為之,惜善人之甚。}

\begin{quoting}\textbf{馬瑞辰}柏舟詩「實維我特」傳「特,匹也」,匹之言敵也當也。\end{quoting}

\textbf{交交黃鳥,止于桑。誰從穆公,子車仲行。}{\footnotesize 箋云仲行,字也。}\textbf{維此仲行,百夫之防。}{\footnotesize 防,比也。箋云防,猶當也。言此一人當百夫。}\textbf{臨其穴,惴惴其慄。彼蒼者天,殲我良人。如可贖兮,人百其身。}

\begin{quoting}\textbf{陳奐}傳讀防為比方之方,徐邈云「毛音方」是也。\end{quoting}

\textbf{交交黃鳥,止于楚。誰從穆公,子車鍼虎。維此鍼虎,百夫之禦。}{\footnotesize 禦,當也。}\textbf{臨其穴,惴惴其慄。彼蒼者天,殲我良人。如可贖兮,人百其身。}

\begin{quoting}鍼 \texttt{qián}。\end{quoting}

\section{晨風}

%{\footnotesize 三章、章六句}

\textbf{晨風,刺康公也。忘穆公之業,始棄其賢臣焉。}

\textbf{鴥彼晨風,鬱彼北林。}{\footnotesize 興也。鴥,疾飛貌。晨風,鸇也。鬱,積也。北林,林名也。先君招賢人,賢人往之,駛疾如晨風之飛入北林。箋云先君,謂穆公。}\textbf{未見君子,憂心欽欽。}{\footnotesize 思望之,心中欽欽然。箋云言穆公始未見賢者之時,思望而憂之。}\textbf{如何如何,忘我實多。}{\footnotesize 今則忘之矣。箋云此以穆公之意責康公,如何如何乎,女忘我之事實多。}

\begin{quoting}鴥 \texttt{yù},韓詩作鷸,廣韻「鷸,鳥飛快也」。\end{quoting}

\textbf{山有苞櫟,隰有六駮。}{\footnotesize 櫟,木也。駮,如馬,倨牙,食虎豹。箋云山之櫟、隰之駮皆其所宜有也,以言賢者亦國家所宜有之。}\textbf{未見君子,憂心靡樂。如何如何,忘我實多。}

\begin{quoting}苞,叢生貌。六,同蓼,長貌。\end{quoting}

\textbf{山有苞棣,隰有樹檖。}{\footnotesize 棣,唐棣也。檖,赤羅也。}\textbf{未見君子,憂心如醉。如何如何,忘我實多。}

\begin{quoting}樹,直立貌。\end{quoting}

\section{無衣}

%{\footnotesize 三章、章五句}

\textbf{無衣,刺用兵也。秦人刺其君好攻戰,亟用兵而不與民同欲焉。}

\textbf{豈曰無衣,與子同袍。}{\footnotesize 興也。袍,襺也。上與百姓同欲,則百姓樂致其死。箋云此責康公之言也,君豈嘗曰「女無衣,我與女共袍乎」,言不與民同欲。}\textbf{王于興師,脩我戈矛,與子同仇。}{\footnotesize 戈長六尺六寸,矛長二丈。天下有道,則禮樂征伐自天子出。仇,匹也。箋云于,於也。怨耦曰仇。君不與我同欲,而於王興師,則云「脩我戈矛,與子同仇」,往伐之,刺其好攻戰。}

\textbf{豈曰無衣,與子同澤。}{\footnotesize 澤,潤澤也。箋云澤,褻衣,近汙垢。}\textbf{王于興師,脩我矛戟,與子偕作。}{\footnotesize 作,起也。箋云戟,車戟常也。}

\begin{quoting}澤,齊詩作襗。\end{quoting}

\textbf{豈曰無衣,與子同裳。王于興師,脩我甲兵,與子偕行。}{\footnotesize 行,往也。}

\begin{quoting}案曰袍、曰澤、曰裳,竟同於比丘之三衣矣。\end{quoting}

\section{渭陽}

%{\footnotesize 二章、章四句}

\textbf{渭陽,康公念母也。康公之母,晉獻公之女,文公遭麗姬之難,未反而秦姬卒,穆公納文公,康公時為大子,贈送文公于渭之陽,念母之不見也,我見舅氏,如母存焉,及其即位,思而作是詩也。}

\textbf{我送舅氏,曰至渭陽。}{\footnotesize 母之昆弟曰舅。箋云渭,水名也。秦是時都雍,至渭陽者,蓋東行送舅氏於咸陽之地。}\textbf{何以贈之,路車乘黃。}{\footnotesize 贈,送也。乘黃,四馬也。}

\begin{quoting}\textbf{陳奐}送舅氏至渭陽,不渡渭也。\end{quoting}

\textbf{我送舅氏,悠悠我思。何以贈之,瓊瑰玉佩。}{\footnotesize 瓊瑰,石次玉。}

\section{權輿}

%{\footnotesize 二章、章五句}

\textbf{權輿,刺康公也。忘先君之舊臣與賢者,有始而無終也。}

\textbf{於我乎,夏屋渠渠。}{\footnotesize 夏,大也。箋云屋,具也。渠渠,猶勤勤也。言君始於我厚,設禮食大具以食我,其意勤勤然。}\textbf{今也每食無餘。}{\footnotesize 箋云此言君今遇我薄,其食我纔足耳。}\textbf{于嗟乎,不承權輿。}{\footnotesize 承,繼也。權輿,始也。}

\begin{quoting}於,同烏,歎詞。渠渠,魯詩作蘧蘧,高也。大戴禮記「孟春,百草權輿」。\end{quoting}

\textbf{於我乎,每食四簋。}{\footnotesize 四簋,黍稷稻粱。}\textbf{今也每食不飽。于嗟乎,不承權輿。}

%\begin{flushright}秦國十篇、二十七章、百八十一句\end{flushright}