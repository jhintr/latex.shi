\chapter{Lesson 6}

\section*{Reading 1}

Pañca-sikkhāpadāni

Pāṇātipātā veramaṇī sikkhāpadaṃ samādiyāmi.

Adinnādānā veramaṇī sikkhāpadaṃ samādiyāmi.

Kāmesu micchācārā veramaṇī sikkhāpadaṃ samādiyāmi.

Musāvādā veramaṇī sikkhāpadaṃ samādiyāmi.

Surāmerayamajjapamādaṭṭhānā veramaṇī sikkhāpadaṃ samādiyāmi. (Khp 2, Dasasikkhāpadaṃ)

\section*{Reading 2}

Yathā pi cando vimalo,\\
gacchaṃ ākāsadhātuyā,\\
sabbe tārāgaṇe loke,\\
ābhāya atirocati.

Tath’eva sīlasampanno,\\
saddho purisapuggalo,\\
sabbe maccharino loke,\\
cāgena atirocati.

Yathāpi megho thanayaṃ,\\
vijjumālī satakkaku,\\
thalaṃ ninnaṃ ca pūreti,\\
abhivassaṃ vasundharaṃ.

Evaṃ dassanasampanno,\\
Sammāsambuddhasāvako,\\
macchariṃ adhigaṇhāti,\\
pañcaṭhānehi paṇḍito.

Āyunā yasasā ceva,\\
vaṇṇena ca sukhena ca,\\
sa ve bhogaparibyūḷho,\\
pecca sagge pamodati. (A 5.4.1)

\section*{Reading 3}

Atha kho Selo brāhmaṇo tīhi māṇavakasatehi parivuto… yena Keṇiyassa jaṭilassa assamo ten’upasaṅkami. Addasā kho Selo brāhmaṇo Keṇiyassa jaṭilassa assame app’ekacce uddhanāni khaṇante, app’ekacce kaṭṭhāni phālente, app’ekacce bhājanāni dhovante, app’ekacce udakamaṇikaṃ patiṭṭhāpente, app’ekacce āsanāni paññapente, Keṇiyaṃ pana jaṭilaṃ sāmaṃ yeva maṇḍalamāḷaṃ paṭiyādentaṃ.

Disvāna Keṇiyaṃ jaṭilaṃ etadavoca “Kiṃ nu kho bhoto Keṇiyassa āvāho vā bhavissati, vivāho vā bhavissati, mahāyañño vā paccupaṭṭhito, rājā vā Māgadho Seniyo Bimbisāro nimantito svātanāya saddhiṃ balakāyenā” ti?

“Na me, bho Sela, āvāho vā bhavissati vivāho vā, nāpi rājā Māgadho Seniyo Bimbisāro nimantito svātanāya saddhiṃ balakāyena, api ca kho me mahāyañño paccupaṭṭhito atthi. Samaṇo Gotamo Sakyaputto Sakyakulā pabbajito, Aṅguttarāpesu cārikaṃ caramāno mahatā bhikkhusaṅghena… Āpaṇaṃ anuppatto.… . So me nimantito svātanāya… saddhiṃ bhikkhusaṅghenā” ti.

“‘Buddho’ ti, bho Keṇiya, vadesi?”

“‘Buddho’ ti, bho Sela, vadāmi”.

“‘Buddho’ ti, bho Keṇiya, vadesi?”

“‘Buddho’ ti, bho Sela, vadāmī” ti.

“Ghoso pi kho eso dullabho lokasmiṃ yadidaṃ ‘buddho’” ti. (Sn 3.7)

\section*{Reading 4}

Dve’me, bhikkhave, puggalā loke uppajjamānā uppajjanti bahujanahitāya bahujanasukhāya, bahuno janassa atthāya hitāya sukhāya. Katame dve?

Tathāgato ca arahaṃ sammāsambuddho, rājā ca cakkavattī. Ime kho, bhikkhave, dve puggalā loke uppajjamānā uppajjanti bahujanahitāya bahujanasukhāya, bahuno janassa atthāya hitāya sukhāya ti.

Dve’me, bhikkhave, puggalā loke uppajjamānā uppajjanti acchariyamanussā. Katame dve?

Tathāgato ca arahaṃ sammāsambuddho, rājā ca cakkavattī. Ime kho, bhikkhave, dve puggalā loke uppajjamānā uppajjanti acchariyamanussā ti.

Dvinnaṃ, bhikkhave, puggalānaṃ kālakiriyā bahuno janassa anutappā hoti. Katamesaṃ dvinnaṃ?

Tathāgatassa ca arahato sammāsambuddhassa, rañño ca cakkavattissa. Imesaṃ kho, bhikkhave, dvinnaṃ puggalānaṃ kālakiriyā bahuno janassa anutappā hotī ti.

Dve’me, bhikkhave, thūpārahā. Katame dve?

Tathāgato ca arahaṃ sammāsambuddho, rājā ca cakkavattī. Ime kho, bhikkhave, dve thūpārahā ti. (A 2.5.6)

\section*{Reading 5}

Tam eva vācaṃ bhāseyya,\\
yāy’attānaṃ na tāpaye,\\
pare ca na vihiṃseyya,\\
sā ve vācā subhāsitā.

Piyavācam eva bhāseyya,\\
yā vācā paṭinanditā,\\
yaṃ anādāya pāpāni,\\
paresaṃ bhāsate piyaṃ.

‘Saccaṃ ve amatā vācā’,\\
esa dhammo sanantano,\\
‘sacce atthe ca dhamme ca’,\\
āhu ‘santo patiṭṭhitā.’ (Sn 3.3)

\section*{Further Reading 1}

“Nanu te, Soṇa, rahogatassa paṭisallīnassa evaṃ cetaso parivitakko udapādi ‘ye kho keci bhagavato sāvakā āraddhavīriyā viharanti, ahaṃ tesaṃ aññataro. Atha ca pana me na anupādāya āsavehi cittaṃ vimuccati, saṃvijjanti kho pana me kule bhogā, sakkā bhogā ca bhuñjituṃ puññāni ca kātuṃ. Yaṃ nūnāhaṃ sikkhaṃ paccakkhāya hīnāyāvattitvā bhoge ca bhuñjeyyaṃ puññāni ca kareyyaṃ’” ti?

“Evaṃ, bhante.”

“Taṃ kiṃ maññasi, Soṇa, kusalo tvaṃ pubbe agāriyabhūto vīṇāya tantissare” ti?

“Evaṃ, bhante.”

“Taṃ kiṃ maññasi, Soṇa, yadā te vīṇāya tantiyo accāyatā honti, api nu te vīṇā tasmiṃ samaye saravatī vā hoti kammaññā vā” ti?

“No h’etaṃ, bhante.”

“Taṃ kiṃ maññasi, Soṇa, yadā te vīṇāya tantiyo atisithilā honti, api nu te vīṇā tasmiṃ samaye saravatī vā hoti kammaññā vā” ti?

“No h’etaṃ, bhante.”

“Yadā pana te, Soṇa, vīṇāya tantiyo na accāyatā honti nātisithilā same guṇe patiṭṭhitā, api nu te vīṇā tasmiṃ samaye saravatī vā hoti kammaññā vā” ti?

“Evaṃ, bhante.”

“Evamevaṃ kho, Soṇa, accāraddhavīriyaṃ uddhaccāya saṃvattati, atisithilavīriyaṃ kosajjāya saṃvattati. Tasmātiha tvaṃ, Soṇa, vīriyasamataṃ adhiṭṭhaha, indriyānaṃ ca samataṃ paṭivijjha, tattha ca nimittaṃ gaṇhāhī” ti.

\section*{Further Reading 2}

Kodhano dubbaṇṇo hoti,\\
atho dukkhaṃ pi seti so,\\
atho atthaṃ gahetvāna,\\
anatthaṃ adhipajjati.

Tato kāyena vācāya,\\
vadhaṃ katvāna kodhano,\\
kodhābhibhūto puriso,\\
dhanajāniṃ nigacchati.

Kodhasammadasammatto,\\
āyasakyaṃ nigacchati,\\
ñātimittā suhajjā ca,\\
parivajjanti kodhanaṃ.

Anatthajanano kodho,\\
kodho cittappakopano,\\
bhayam antarato jātaṃ,\\
taṃ jano nāvabujjhati.

Kuddho atthaṃ na jānāti,\\
kuddho dhammaṃ na passati,\\
andhatamaṃ tadā hoti,\\
yaṃ kodho sahate naraṃ.

Nāssa hirī na ottappaṃ,\\
na vāco hoti gāravo,\\
kodhena abhibhūtassa,\\
na dīpaṃ hoti kiñcanaṃ.

\section*{Further Reading 3}

Rājā āha “Kiṃlakkhaṇo, bhante Nāgasena, manasikāro, kiṃlakkhaṇā paññā” ti?

“Ūhanalakkhaṇo kho, mahārāja, manasikāro, chedanalakkhaṇā paññā” ti.

“Kathaṃ ūhanalakkhaṇo manasikāro, kathaṃ chedanalakkhaṇā paññā, opammaṃ karohī” ti.

“Jānāsi, tvaṃ mahārāja, yavalāvake?” ti.

“Āma, bhante, jānāmī” ti.

“Kathaṃ, mahārāja, yavalāvakā yavaṃ lunantī” ti?

“Vāmena, bhante, hatthena yavakalāpaṃ gahetvā dakkhiṇena hatthena dāttaṃ gahetvā dāttena chindantī” ti.

“Yathā, mahārāja, yavalāvako vāmena hatthena yavakalāpaṃ gahetvā dakkhiṇena hatthena dāttaṃ gahetvā yavaṃ chindati, evameva kho, mahārāja, yogāvacaro manasikārena mānasaṃ gahetvā paññāya kilese chindati. Evaṃ kho, mahārāja, ūhanalakkhaṇo manasikāro, evaṃ chedanalakkhaṇā paññā” ti.

“Kallo’si, bhante Nāgasenā” ti.

\section*{Further Reading 4}

Atha kho aññataro brāhmaṇo yena bhagavā ten’upasaṅkami, upasaṅkamitvā bhagavatā saddhiṃ sammodi… ekamantaṃ nisīdi. Ekamantaṃ nisinno kho so brāhmaṇo bhagavantaṃ etadavoca “‘Sandiṭṭhiko dhammo, sandiṭṭhiko dhammo’ ti, bho gotama, vuccati. Kittāvatā nu kho, bho gotama, sandiṭṭhiko dhammo hoti” ti?

“Tena hi, brāhmaṇa, taññev’ettha paṭipucchissāmi. Yathā te khameyya tathā naṃ ākareyyāsi. Taṃ kiṃ maññasi, brāhmaṇa, santaṃ vā ajjhattaṃ rāgaṃ ‘atthi me ajjhattaṃ rāgo’ ti pajānāsi, asantaṃ vā ajjhattaṃ rāgaṃ ‘natthi me ajjhattaṃ rāgo’ ti pajānāsī” ti?

“Evaṃ, bho.”

“Yaṃ kho tvaṃ, brāhmaṇa, santaṃ vā ajjhattaṃ rāgaṃ ‘atthi me ajjhattaṃ rāgo’ ti pajānāsi, asantaṃ vā ajjhattaṃ rāgaṃ ‘natthi me ajjhattaṃ rāgo’ ti pajānāsi, evam pi kho,brāhmaṇa, sandiṭṭhiko dhammo hoti.”

“Taṃ kiṃ maññasi, brāhmaṇa, santaṃ vā ajjhattaṃ dosaṃ… pe … santaṃ vā ajjhattaṃ mohaṃ… pe… santaṃ vā ajjhattaṃ kāyasandosaṃ… pe… santaṃ vā ajjhattaṃ vacīsandosaṃ… pe… santaṃ vā ajjhattaṃ manosandosaṃ ‘atthi me ajjhattaṃ manosandoso’ ti pajānāsi, asantaṃ vā ajjhattaṃ manosandosaṃ ‘natthi me ajjhattaṃ manosandoso’ ti pajānāsī” ti?

“Evaṃ, bhante.”

“Yaṃ kho tvaṃ, brāhmaṇa, santaṃ vā ajjhattaṃ manosandosaṃ ‘atthi me ajjhattaṃ manosandoso’ ti pajānāsi, asantaṃ vā ajjhattaṃ manosandosaṃ ‘natthi me ajjhattaṃ manosandoso’ ti pajānāsi, evaṃ kho, brāhmaṇa, sandiṭṭhiko dhammo hoti” ti.

“Abhikkantaṃ, bho Gotama, abhikkantaṃ, bho Gotama… pe… upāsakaṃ maṃ bhavaṃ Gotamo dhāretu ajjatagge pāṇ’upetaṃ saraṇaṃ gataṃ” ti. (A 6.5.6 Dutiyasandiṭṭhikasuttaṃ)

\section*{Further Reading 5}

Manujassa pamattacārino,\\
taṇhā vaḍḍhati māluvā viya,\\
so palavati hurāhuraṃ,\\
phalam icchaṃ va vanasmi vānaro.

Yaṃ esā sahatī jammī,\\
taṇhā loke visattikā,\\
sokā tassa pavaḍḍhanti,\\
abhivaḍḍhaṃ va bīraṇaṃ.

Yo c’etaṃ sahatī jammiṃ,\\
taṇhaṃ loke duraccayaṃ,\\
sokā tamhā papatanti,\\
udabindu va pokkharā. (Dhp 24)