\section{有财者经}

\begin{center}Dhaniya Sutta\end{center}\vspace{1em}

\subsection\*{\textbf{18}}

\textbf{「我已煮好米饭,挤好牛奶,」有财者牛场主\footnote{此经旧译见杂阿含经第 1004 经、别译杂阿含经第 142 经。}说,「在摩醯沿岸聚居,\\}
\textbf{「屋已覆蔽,火已燃起,那么,若您愿意,下雨吧!天!」}

“Pakkodano duddhakhīro’ham asmi, \textit{(iti Dhaniyo gopo)} anutīre Mahiyā samānavāso;\\
channā kuṭi āhito gini, atha ce patthayasī pavassa deva”. %\hfill\textcolor{gray}{\footnotesize 1}

\begin{enumerate}\item 缘起为何?世尊住舍卫国。尔时,有财者牛场主在摩醯河岸边居住。其宿行为:在迦叶世尊的言教光耀之时,他每天供养僧团二十份行筹食\footnote{行筹食:菩提比丘注云,即供养给定数目的比丘、比丘尼而不指定受者,以行筹或轮流来决定。}达二万年之久\footnote{二万年:长阿含经云「迦叶佛时,人寿二万岁」。}。他死后投生于诸天,如是于诸天界虚掷了一佛的间隔,当我们的世尊之时,在毗提诃国中有山中之国,其中有城名达磨哥伦陀,他即投生于此城而为富家子,靠着牛群活命。因为他有三万头牛,二万七千头牛产奶。
\item 作为牛场主,并没有固定的住处。在雨季的四个月里,他们住在高地,而在其余的八个月,则逐水草易得之处而住,或是河边,或是天然的湖泊边。于是,在雨季,他从自己所住的村庄离开,为牛群的乐住寻觅地点。摩诃摩醯河在分叉后,一支称为黑摩醯,一支称为摩诃摩醯,经分流而又在入海前汇合。他进入这(河流)所形成的三角洲,教人建了牛棚和自己的居处,便住了下来。他有七个儿子、七个女儿、七个媳妇,还有一众工人。
\item 作为牛场主,他们识得雨相。当鸟儿在树梢筑巢,螃蟹封了水边的入口而使用高地边的入口时,他们便知道将有大雨,而当鸟儿在水边低处筑巢,螃蟹封了高地边的入口而使用水边的入口时,他们便知道雨水不多。于是,这有财者察得了丰雨之相,在雨期到来前,离开了三角洲,在摩诃摩醯的对岸、天降七七四十九日的大雨也不被水淹没之处,建了自己的居住之处,围好四周,教人造了牛棚,便住了下来。
\item 于是,当他收拾好柴火、草料等,为儿女、工人、仆人等一众人等准备好种种品类的硬食、软食,云团便从四面八方集起。他教人挤好牛奶,把牛犊赶回牛棚,从四面为牛群燃起烟,让所有仆从从吃完饭、干完活,教人在各处点了灯后,自己才吃了饭、喝了奶,躺在大床上,看到自己光彩的成就而心满意足,听到远处雷声大作,他便横卧而发了这慨叹。
\item 其释义为:\textbf{煮好米饭},即备好食物。\textbf{挤好牛奶},即挤完牛后,存好牛奶。\textbf{我},即指自己。\textbf{说} \textit{iti},即如是说之义,而在义释中如是解释其义:\begin{quoting}\textbf{iti},即词句的连接、词句的关联、词句的结束、音节的结合、字句的平顺、词句的次第。(小义释·阿耆多学童问义释)\end{quoting}当知也是就此而言。因为对任何以前句所说的句子,为显示是某人如是说之义,而以「说」字与后句相连接,如「世尊说、弥勒说」等,而非于它处。
\item \textbf{有财者牛场主},即此富家子的名称的组合。因为在不动等五种财产\footnote{五种财产:在\textbf{小诵}·伏藏经的义注中只列了四种,缺「可携带的财产」。}中,除布施、持戒等随行的财产外,较之田地、物品、园林等不动的财产,较之牛马等的可动的财产,较之货币、黄金等可携带的财产,较之技艺等等同肢体的财产,牛的财产,就其为世间供给五种奶制品\footnote{五种奶制品 \textit{gorasā}:即乳、凝乳 \textit{dadhi}、酪乳 \textit{takka}、生酥 \textit{navanīta}、熟酥 \textit{sappi}。}的诸多助益而言,\begin{quoting}无有财产与牛相等。(相应部·天相应第 1:13 经)\end{quoting}而如是殊胜,由具足此而为\textbf{有财者},由守牧牛群而为\textbf{牛场主}。因为若守牧自己的牛者,称为牛场主,若因别人的薪水而作雇工,则为牧牛人,而他只是为了自己,因而说是「牛场主」。
\item \textbf{沿岸},即在岸的附近。\textbf{摩醯},即名为摩诃摩醯的河。与行为随顺的仆从一起居住为聚居,而他即如此,因此说\textbf{聚居}。\textbf{覆蔽},即以茅草、树叶的屋盖使不漏雨。\textbf{屋},即居住之家的同义语。\textbf{天},即指云。以上是释词\footnote{义注于此区分了释词 \textit{padavaṇṇanā} 与释义 \textit{atthavaṇṇanā},前者即对语词的逐个解释,类似训诂,后者即对段落的串讲,类似章句。}。
\item 而其释义为:如是,这有财者牛场主横卧在自己卧室内的大床上,听到雷声后,说「我已煮好米饭」,是就止息身苦之法与身乐之因来阐明自身的安顿,而说「我已挤好牛奶」,是阐明止息心苦之法与心乐之因,「在摩醯沿岸」,即居处的成就,「聚居」,即当此之时无有以爱别离为足处的忧伤,「屋已覆蔽」,即驱除与防御身苦,「火已燃起」,即因为牧牛人燃起周围之火、烟火、薪火等三种火,且于其家这三种都已燃起,所以,就一切方位的周围之火而说「火已燃起」,阐明防止猛兽的到来,就牛群中间以牛粪等起的烟火,阐明牛群不为虻蚊等所患,就牧牛人卧处的薪火,阐明防御牧牛人的寒病。他如是阐明自己、牛群或仆从无有降雨之缘的任何病患,生起喜悦,便说「那么,若您愿意,下雨吧!天」。\end{enumerate}

\subsection\*{\textbf{19}}

\textbf{「我已无有忿怒,离于荒秽,」世尊说,「在摩醯沿岸住一夜,\\}
\textbf{「屋已敞开,火已熄灭,那么,若您愿意,下雨吧!天!」}

“Akkodhano vigatakhilo’ham asmi, \textit{(iti Bhagavā)} anutīre Mahiy’ekarattivāso;\\
vivaṭā kuṭi nibbuto gini, atha ce patthayasī pavassa deva”. %\hfill\textcolor{gray}{\footnotesize 2}

\begin{enumerate}\item 世尊住祇园大寺的香房中,以超越人类的清净天耳界听到了有财者如是所说的这偈颂,且听闻后,以佛眼观察世间,便看到有财者及其妻子,「这两人都具足因,如果我前去开示法,两人都会出家而圆满阿罗汉,若我不去,他们明天将会被暴流冲走」,即于此刻从舍卫国出发,从空中经七百由旬,到了有财者的住处,立于他屋舍的上方。有财者正反复不停地吟着这颂,即便当世尊到来还在吟。世尊听到后,为显示「不能只以这些就满足或是安心,而应如是」,便说了这相对的颂,字句相似而意义不相似。因为「煮好米饭、无有忿怒」等语词在意义上不同,好比大海的此岸彼岸,但字句于此却仿佛相同,故为字句相似\footnote{字句相似:即此颂「忿怒、荒秽」的巴利文分别与前颂的「米饭、牛奶」音近。}。
\item 这里,与前颂相同的语词的意义当知即如所说,而差别的语词的释词、释义为:\textbf{无有忿怒},即以不忿怒为自性。因为对某些人,任何从先前所说种类的嫌恨事生起的忿怒,即便极少量生起,烧灼心已而熄灭,则随后由更强力的生起,这些人便面容扭曲,随后,由更强力,这些人想说恶口而下巴蠢动,再随后,由更强力而说了恶口,再随后,由更强力而寻觅杖剑,四下张望,再随后,由更强力而摸索杖剑,再随后,由更强力而取了杖等追逐,再随后,由更强力而击打一二下,再随后,由更强力,甚至取了亲族、血亲的性命,随后,由更强力,这些人后便悔恨,甚至取了自己的性命,如在僧伽罗岛上迦罗村中居住的大臣一般,至此,忿怒便至最成满。世尊在菩提座上便已完全舍弃了这(忿怒),断其根本,如截多罗树头,所以世尊说「我已无有忿怒」。
\item \textbf{离于荒秽},即因为凡是以心的束缚的状态而说的五种心的荒秽\footnote{五种心的荒秽:即疑大师、疑法、疑僧、疑学、恼于同梵行者,见\textbf{中部}第 16 经。},以及因为在成为荒秽的心中,譬如在荒秽的土地上,即便天下了四个月的雨,谷物也不生长,如是,即便有听闻善法等的善因之雨落下,善也不得生长,而世尊在菩提座上便已完全舍弃之,所以世尊说「我已离于荒秽」。因为有财者将在雨季的四个月里固定居住于此,而世尊却不如是,因为世尊唯在是夜为了他的义利而居住于此,所以说「\textbf{住一夜}」。
\item \textbf{敞开},即除去屋盖。\textbf{屋},即自体。因为自体由种种用意而被称为身、洞窟、身体、自身、船、车、疮口、旗、蚁垤、屋、屋舍等,然而在此,如同名为家者由木材等得称为屋,(自体)由骨等得称为「屋」,如说:\begin{quoting}朋友!譬如由木材、藤蔓、粘土、草等围成的空间得称为家,如是,朋友!由骨、筋、肉、皮等围成的空间得称为色。(中部第 28 经)\end{quoting}或者,由心猿之所居而为屋,如说:\begin{quoting}这骨架所立之屋,即是猿猴之居室,\\猿猴从五门的屋舍出去,\\环绕着门,再再地拍击。(长老偈第 125 颂)\end{quoting}这屋由众生以爱、慢、见的屋盖所覆之故,再再漏入贪等烦恼之雨,如说:\begin{quoting}漏入覆蔽者,不漏敞开者,\\所以,应揭开覆蔽,如是便不漏入。(长老偈第 447 颂、附随第 339 段)\end{quoting}此颂见于犍度与长老偈等两处。在犍度中,是以「覆藏违犯者,其烦恼与再再的违犯便漏入,而不覆藏者则不漏入」之义而说,在长老偈中说:\begin{quoting}有贪等的屋盖者,再再于可意的所缘等,由从贪等的生起,漏入覆蔽,或者,若他容忍已生起的烦恼,则他的为所容忍烦恼的屋盖覆蔽的自体之屋便再再漏入烦恼之雨,然而,对于由以阿罗汉道智之风摧毁烦恼屋盖而敞开者,便不漏入。\end{quoting}此义是这里的旨趣。因为所说的屋盖已由世尊以所说的方法摧毁,所以说「屋已敞开」。\textbf{熄灭},即止息。\textbf{火},即由十一种火\footnote{十一种火:即\textbf{燃烧经}所说的「贪火、嗔火、痴火、生、老、死、忧、悲、苦、忧、恼」等十一种。}而使一切燃烧者,如「以贪火燃烧」等详说,对于世尊,这火在菩提树下便因浇以圣道之水而熄灭,所以说「火已熄灭」。
\item 且当如是说时,他是以寓意指责、教诫、教授有财者满足于不应满足者。如何?因为在说「无有忿怒」时,有财者!你满足于「我已煮好米饭」,而煮饭要终生以耗尽家产去做,耗尽家产则是获得、守护等苦的足处,在这样的情况下,你竟是满足于苦,而我满足于「我已无有忿怒」,即显示我满足于无有现世、来世之苦。
\item 在说「离于荒秽」时,你满足于「我已挤好牛奶」,未作所应作,却认为「我已作所应作」而满足,而我满足于「我已离于荒秽」,即显示我满足于已作所应作。
\item 在说「在摩醯沿岸住一夜」时,你满足于「在摩醯沿岸聚居」,以四个月固定的居住而满足。然而固定的居住即是执著于住处,这便是苦,在这样的情况下,你竟是满足于苦,而我满足于「住一夜」,以无固定的居住而满足,无固定的居住即无对住处的执著,无住处的执著即是乐,即显示我唯满足于乐。
\item 在说「屋已敞开」时,你满足于「屋已覆蔽」,以覆蔽之家而满足,但你的家即使已覆蔽,烦恼之雨也会漏入自体之屋舍,为由之所生的四大暴流所漂没,你便厄于不幸,在这样的情况下,你竟是满足于不应满足者,而我满足于「屋已敞开」,以自体之屋舍无有烦恼之屋盖而满足,如是,烦恼之雨便不漏入我的敞开之屋,不为由之所生的四大暴流所漂没,我便不厄于不幸,在这样的情况下,即显示我唯满足于应满足者。
\item 在说「火已熄灭」时,你满足于「火已燃起」,未防御灾祸,却认为「我已防御灾祸」而满足,而我满足于「火已熄灭」,即显示由无有十一种火的热恼、由已防御灾祸而满足。
\item 在说「那么,若您愿意,下雨吧!天」时,即显示这话对如是已离苦得乐、已作一切应作的我们这样的人才有意义,但在你而言,下雨或不下雨便有增损,你为什么如是说「那么,若您愿意,下雨吧!天」呢?所以,「且当如是说时,他是以寓意指责、教诫、教授有财者满足于不应满足者」作如是解。\end{enumerate}

\subsection\*{\textbf{20}}

\textbf{「没有苍蝇、蚊子,」有财者牛场主说,「牛群在丰草的泽边游荡,\\}
\textbf{「它们堪能忍受来临的雨,那么,若您愿意,下雨吧!天!」}

“Andhakamakasā na vijjare, \textit{(iti Dhaniyo gopo)} kacche rūḷhatiṇe caranti gāvo;\\
vuṭṭhim pi saheyyum āgataṃ, atha ce patthayasī pavassa deva”. %\hfill\textcolor{gray}{\footnotesize 3}

\begin{enumerate}\item 如是,有财者牛场主听到世尊所说的这颂后,虽未说「谁在说这颂」,却满意于这善说,还想再听这样的(话语),便说了这后一颂。
\item 这里,\textbf{苍蝇},即黑蝇的同义语,有些也说是褐蝇。\textbf{泽边},有两种沼泽,河泽与山泽,这里指河泽。\textbf{游荡},即在吃草。\textbf{雨},即风之雨等的多种雨,我们将在旷野经中说明,而这里是就雨季的雨而言。余皆自明。
\item 此中,苍蝇、蚊子聚集后吸血,顷刻间可致牛群于危难,所以那些牧牛人在刚开始下雨时便要以尘土、枝条扑杀它们,有财者以没有这些(苍蝇、蚊子)而说牛群之安稳,以游荡在丰草的泽边而说无行路之疲累,并显示无有饥疲,「别人的牛群为蝇蚊所触,为行路所累,为饥饿所竭,甚至不堪忍受一场雨,而我的牛群不如是,我的牛群以无如上所说的品类,堪能忍受两场或三场雨」。\end{enumerate}

\subsection\*{\textbf{21}}

\textbf{「筏已扎结实,」世尊说,「已度,已到彼岸,堪能调伏暴流,\\}
\textbf{「筏已没有意义,那么,若您愿意,下雨吧!天!」}

“Baddhāsi\footnote{PTS as \textit{Baddhā hi}。} bhisī susaṅkhatā, \textit{(iti Bhagavā)} tiṇṇo pāragato vineyya oghaṃ;\\
attho bhisiyā na vijjati, atha ce patthayasī pavassa deva”. %\hfill\textcolor{gray}{\footnotesize 4}

\begin{enumerate}\item 随后,因为有财者看到住在三角洲上的危险,扎好筏,度过摩诃摩醯,到了这泽边,认为「我已平安到达,立于无怖畏之处」而如是说,而他却立于具怖畏之处,所以世尊为说明自己所到之处较之他所到之处更高、更胜,说了此颂,意义相似而字句不相似\footnote{意义相似而字句不相似:对此,Norman 和菩提比丘都认为意义、字句都不相似,认为第 21 颂并不是对第 20 颂的回应,对第 20 颂的回应已经遗失,而第 21 颂的第一句(或第一、二句)应是有财者所说,其余的才是佛陀的回应,因为两者貌似矛盾。总结起来,共遗失了八句,即对第 20 颂的回应,计一颂四句,而现有的第 21 颂应为有财者与世尊各说四句的省并。}。
\item 这里,\textbf{筏}在世间是说摆平后成广大状的束好的筏,但在圣法律中,则是圣道的同义语。因为圣道即\begin{quoting}道、迹、路、途、径、路径,\\与船、梁、槎、筏、桥,(小义释·彼岸道赞颂义释)\\以及旅途、源头,于处处被说明。\end{quoting}当知在此颂中,世尊仍以先前的方法教诫,说了此义:有财者!你扎好了筏,度过了摩醯,来到此地,还应再扎筏、再度河,此非安稳之地,然而,我已将众道支并入一心,以智之束缚扎好了筏,且它以圆满三十七菩提分法、由混成一味的状态而不彼此违越、无须再予绑扎、不能被任何天或人解开而为\textbf{结实}。
\item 且我由之\textbf{已度},到达先前希求的岸边,也并非如须陀洹等只去到某个地方,而是\textbf{已到彼岸},到达一切漏尽、一切法之彼岸、最上安稳的涅槃。或者,已度是证得一切知,已到彼岸是证得阿罗汉。那么,已到彼岸者堪能调伏什么?\textbf{堪能调伏暴流},即度过、越过欲暴流等的四种暴流,到达那彼岸。且现在于我,由无须再度,\textbf{筏已没有意义},所以,只有我说「那么,若您愿意,下雨吧!天」才是适当的。\end{enumerate}

\subsection\*{\textbf{22}}

\textbf{「我的妻子顺从、不动摇,」有财者牛场主说,「长期同居,适意可人,\\}
\textbf{「我未听到任何关于她的过恶,那么,若您愿意,下雨吧!天!」}

“Gopī mama assavā alolā, \textit{(iti Dhaniyo gopo)} dīgharattaṃ saṃvāsiyā manāpā;\\
tassā na suṇāmi kiñci pāpaṃ, atha ce patthayasī pavassa deva”. %\hfill\textcolor{gray}{\footnotesize 5}

\begin{enumerate}\item 有财者听到后,仍以先前的方法说了此颂。这里,\textbf{顺从},即服从,遵从任何事。\textbf{不动摇},因为女性由五种动摇事而动摇:食物、妆饰、别的男子、财产、游玩。女性由对饭、饼、酒等类食物的动摇,甚至吃隔夜的食物,嚼烫手的东西,付加倍的钱去喝酒。由对妆饰的动摇,当得不到别的化妆品时,甚至用水与油来梳头、洗脸。由对别的男子的动摇,她甚至被儿子召到那种地方时,首先想到的是不正法。由对财产的动摇,如说\begin{quoting}捉住了鹅王,他们便丧尽黄金。(本生第 1:136 颂)\end{quoting}由对游玩的动摇,便习于逛园林等,挥霍所有钱财。这里,有财者为显示「我的妻子连一个动摇都没有」而说不动摇。
\item \textbf{长期同居},即长时一起居住,从少年时期起便共同成长,以此显示她不认得别的男子。\textbf{适意可人},即显示如是不认得别的男子者唯执意于我。\textbf{我未听到任何关于她的过恶},即如「与名为某某者一起戏笑言谈」者,显示我未听到关于她有任何越行的过失。\end{enumerate}

\subsection\*{\textbf{23}}

\textbf{「我的心顺从、解脱,」世尊说,「长期遍修,已善调御,\\}
\textbf{「我已没有过恶,那么,若您愿意,下雨吧!天!」}

“Cittaṃ mama assavaṃ vimuttaṃ, \textit{(iti Bhagavā)} dīgharattaṃ paribhāvitaṃ sudantaṃ;\\
pāpaṃ pana me na vijjati, atha ce patthayasī pavassa deva”. %\hfill\textcolor{gray}{\footnotesize 6}

\begin{enumerate}\item 于是,世尊仍以先前的方法教诫因这些品德而满足于妻子的有财者,说了此颂,意义相似且字句相似。这里,句子的意义自明,而其旨趣为:有财者!你满足于「我的妻子顺从」,而她对你或许顺从,或许不是,别人的心难以知晓,尤其是女性的。因为他们用肚子带着尚且不能守护女性\footnote{用肚子带着尚且不能守护女性事,见\textbf{本生}第 9:87~95 颂的义注。},由如是难以守护其心,像你这样的人无法知晓女人是否不动摇、同居、适意可人或无过恶。
\item 而\textbf{我的心顺从},遵守教诫,受制于我,我却不受制于它。且在双神变中,当六色的水火之流转起时,它的顺从对所有人都明了。因为当化作火时应证火遍,当化作水时应证水遍,当化作青等时应证青等遍。即便对诸佛,二心也不能同时转起,而唯有一心因顺从而成主宰。
\item 且这(心)由离于一切烦恼之束缚而\textbf{解脱},由解脱故,唯它不动摇,而非你的妻子。且由从燃灯佛时起,由以布施、持戒等\textbf{长期遍修}故,得为同居,而非你的妻子。它由以无上调御而调御为\textbf{已善调御},由已善调御故,以自身之力舍弃了六门的躁动,由唯服从于我的意趣故,得为适意可人,而非你的妻子。
\item \textbf{我已没有过恶},即世尊以此显示其自心的无过恶,如有财者显示妻子般。且其无过恶不仅唯于等正觉时,当知于二十九年俱贪等时,对居于俗家间的(菩萨)亦然。因为即便在彼时,他也未生起与在家状态相应的、为智者所呵责的身语意恶行。随后,在未现等觉的六年与现等觉的一年计七年间,魔罗便跟随如来,「或许我能发现他哪怕毫尖所刺之量的恶行」,他未曾见而生厌,说了此颂:\begin{quoting}七年间,我步步跟随着世尊,\\没能得到具念的等正觉者的机会。(经集第 449 颂)\end{quoting}在成佛后,郁多罗学童跟随了他七个月,欲观察等正行,他未见任何过失而得出\begin{quoting}世尊是遍净正行者。(中部第 91 经)\end{quoting}因为有四事如来不须守护,如说:\begin{quoting}诸比丘!有四事如来不须守护。哪四事?诸比丘!如来是遍净身正行者,无有如来的身恶行,如来须以「莫让别人知晓我的这事」守护之。诸比丘!如来是遍净语正行者……遍净意正行者……遍净活命者,无有如来的邪命,如来须以「莫让别人知晓我的这事」守护之。(增支部第 7:58 经)\end{quoting}如是,因为如来的心不仅于等正觉时,于之前也无有过恶,所以说「我已没有过恶」。其旨趣为:听不到过恶的,唯有我的心,而非你的妻子,所以,如果以满足于这些品德而当说「那么,若您愿意,下雨吧!天」,则唯有我当说。\end{enumerate}

\subsection\*{\textbf{24}}

\textbf{「我以自己的酬劳养活,」有财者牛场主说,「儿女与我一起,无病,\\}
\textbf{「我未听到任何关于他们的过恶,那么,若您愿意,下雨吧!天!」}

“Attavetanabhato’ham asmi, \textit{(iti Dhaniyo gopo)} puttā ca me samāniyā arogā;\\
tesaṃ na suṇāmi kiñci pāpaṃ, atha ce patthayasī pavassa deva”. %\hfill\textcolor{gray}{\footnotesize 7}

\begin{enumerate}\item 听闻此后,有财者欲饮更多善说之味,为显示自己自由的状态,说了此颂。这里,\textbf{以自己的酬劳养活},即唯以属于我的衣食养活,唯作自己的工作而活命,即显示不是收取他人的酬劳,作他人的工作。\textbf{一起},即住在近旁、未分居。\textbf{无病},即无疾,显示全都手脚有力。\textbf{我未听到任何关于他们的过恶},即我未听到任何关于他们「作贼、通奸、习恶」等的过恶。\end{enumerate}

\subsection\*{\textbf{25}}

\textbf{「我不是任何人的雇工,」世尊说,「我凭所得在一切世间游行,\\}
\textbf{「酬劳已没有意义,那么,若您愿意,下雨吧!天!」}

“Nāhaṃ bhatako’smi kassaci, \textit{(iti Bhagavā)} nibbiṭṭhena carāmi sabbaloke;\\
attho bhatiyā na vijjati, atha ce patthayasī pavassa deva”. %\hfill\textcolor{gray}{\footnotesize 8}

\begin{enumerate}\item 如是说已,世尊仍以先前的方法教诫有财者而说了此颂。这里句子的意义也自明,其旨趣为:你认为「我是自由的」而满足,但从第一义上说,作了自己的业而活命者即是奴隶,由作为渴爱的奴隶及由沦为雇工而未解脱,如\begin{quoting}世间亏欠、无厌,为渴爱的奴隶。(中部第 82 经)\end{quoting}所说,而从第一义上说,\textbf{我不是任何人的雇工}。因为我不是任何别人或自己的雇工。
\item 什么原因?因为\textbf{我凭所得在一切世间游行}。因为我从燃灯足下起,直至觉悟,始终是一切知智的雇工,但证得一切知便得其所得,如得了俸禄的王的雇工一般,凭所得的一切知性及出世间定之乐活命。
\item 现在,由于我已没有更多应作或需熟习的已作,如那些未舍弃结生者而有任何当证,\textbf{酬劳已没有意义},所以,如果以满足于自由而当说「那么,若您愿意,下雨吧!天」,则唯有我当说。\end{enumerate}

\subsection\*{\textbf{26}}

\textbf{「有母牛,有奶牛,」有财者牛场主说,「也有孕牛和待配的牛,\\}
\textbf{「这里还有作为头牛的公牛,那么,若您愿意,下雨吧!天!」}

“Atthi vasā atthi dhenupā, \textit{(iti Dhaniyo gopo)} godharaṇiyo paveṇiyo pi atthi;\\
usabho pi gavampatîdha atthi, atha ce patthayasī pavassa deva”. %\hfill\textcolor{gray}{\footnotesize 9}

\begin{enumerate}\item 听闻此后,有财者仍未满于善说之甘露,为显示自己有圆满的五类牛群而说此颂。这里,\textbf{母牛},即未驯服、已长大的牛犊。\textbf{奶牛},即吃奶的幼犊,或是哺乳的牛。\textbf{孕牛},即怀胎者。\textbf{待配的牛},即适龄、有力、希望交媾的牛。\textbf{作为头牛的公牛},即在晨朝,由牧牛人澡浴、喂饲、给予五指(的印记)、扎好花鬘后,派遣道「去!老大!把牛群带到牧场,保护它们,再带回来」,如是受派遣后,它便阻止牛群去非牧场,带到牧场,庇护狮虎等的怖畏,再予带回,即显示我\textbf{这里}的牛群中有这样作为头牛的公牛。\end{enumerate}

\subsection\*{\textbf{27}}

\textbf{「没有母牛,没有奶牛,」世尊说,「也没有孕牛和待配的牛,\\}
\textbf{「这里也没有作为头牛的公牛,那么,若您愿意,下雨吧!天!」}

“Natthi vasā natthi dhenupā, \textit{(iti Bhagavā)} godharaṇiyo paveṇiyo pi natthi;\\
usabho pi gavampatîdha natthi, atha ce patthayasī pavassa deva”. %\hfill\textcolor{gray}{\footnotesize 10}

\begin{enumerate}\item 如是说已,世尊同样为教诫有财者而说了这相反的颂。其中的旨趣为:\textbf{这里},在我们的教法里,以未驯服及已长大之义,缠被称为\textbf{母牛},就幼犊而言,以作为母牛的根本之义,或就哺乳的牛而言,以漏出之义,随眠被称为\textbf{奶牛},以结生的持胎之义,福、非福、不动的行作思\footnote{行作 \textit{abhisaṅkhāra}:这里沿用叶均的译名,其字面意思即「准备、积累、决心」等。}被称为\textbf{孕牛},以希求交合之义,希求、渴爱被称为\textbf{待配的牛},以统领、先导、最胜之义,行作识\footnote{行作识,参见\textbf{牟尼经}第 211 颂注。}被称为\textbf{作为头牛的公牛},而(以上这些都)\textbf{没有}。我满足于这离于一切轭安稳的无有性,而你却满足于作为忧等依处的有性,所以,只有我说这「那么,若您愿意,下雨吧!天」才是适当的。\end{enumerate}

\subsection\*{\textbf{28}}

\textbf{「埋好的柱子无法撼动,」有财者牛场主说,「文阇草编的绳子崭新、整齐,\\}
\textbf{「任凭几头牛犊也扯不断,那么,若您愿意,下雨吧!天!」}

“Khilā nikhātā asampavedhī, \textit{(iti Dhaniyo gopo)} dāmā muñjamayā navā susaṇṭhānā;\\
na hi sakkhinti dhenupā pi chettuṃ, atha ce patthayasī pavassa deva”. %\hfill\textcolor{gray}{\footnotesize 11}

\begin{enumerate}\item 听闻此后,有财者欲证更多善说的甘露之味,为显示自己的牛群被牢牢地系在柱子上而说此颂。这里,\textbf{柱子},即系牛群的柱子。\textbf{埋好},小者经锤击入地,大者则先挖后植。\textbf{绳子},即为系牛犊而制的适于绑扎的特别的绳索。\textbf{崭新},即做好不久。\textbf{整齐},即形制完好,或形制适当。\end{enumerate}

\subsection\*{\textbf{29}}

\textbf{「如公牛扯断了束缚,」世尊说,「如大象撕裂了腐蔓,\\}
\textbf{「我绝不再入胎室,那么,若您愿意,下雨吧!天!」}

“Usabhor iva chetva bandhanāni, \textit{(iti Bhagavā)} nāgo pūtilataṃ va dālayitvā;\\
nāhaṃ pun’upessaṃ gabbhaseyyaṃ, atha ce patthayasī pavassa deva”. %\hfill\textcolor{gray}{\footnotesize 12}

\begin{enumerate}\item 如是说已,世尊已知晓有财者的根熟之时,仍以先前的方法教诫他,说了这显明四谛之颂。这里,\textbf{公牛} \textit{usabha},即作为牛父、牛统领、牛群主的有力者。然而有人说「百牛的胜者为 usabha,千牛的胜者为 vasabha,百千牛的胜者为 nisabha」,另有人说「一村之田的胜者为 usabha,二村的为 vasabha,于一切处不败者为 nisabha」,这一切都是戏论,且当知 usabha, vasabha, nisabha 等都是依无对等之义,如说\begin{quoting}先生!沙门乔达摩确是牛王 \textit{nisabha}。(相应部第 1:38 经)\end{quoting}\textbf{束缚},即绳索的束缚及烦恼的束缚。\textbf{腐蔓},即藤蔓。因为好比金色之身被称为腐身,百岁的狗被称为小狗,当天出生的狼被称为老狼,如是新鲜的藤蔓以无实之义被称为腐蔓。\textbf{撕裂},即斩断。胎与室为\textbf{胎室},这里以胎摄胎生,以室摄其余,或当知是以胎室为首来说所有。其余的词义于此自明。
\item 而其旨趣为:有财者!你满足于束缚,而我如困恼于束缚、具有强力和精进的大公牛般,以第四圣道的强力和精进扯断了五上分结的束缚,如大象般,以下三圣道的强力和精进撕裂了五下分结的腐蔓。或者说,如公牛扯断了随眠的束缚,如大象撕裂了缠的腐蔓。所以,\textbf{我绝不再入胎室},我解脱于以生苦为依处的一切苦,光荣地说「那么,若您愿意,下雨吧!天」。所以,如果你也希望像我一样说,扯断这些束缚!
\item 且此中,束缚是集谛,胎室是苦谛,此中「不入」的不接近以无余依为灭谛,「扯断、撕裂」的扯断、撕裂以有余依(为灭谛),以之扯断、撕裂者是道谛。\end{enumerate}

\subsection\*{\textbf{30}}

\textbf{顷刻之间,大云降雨,遍满低洼与高地,\\}
\textbf{听到天降大雨,有财者便说了此义:}

Ninnañ ca thalañ ca pūrayanto, mahāmegho pavassi tāvad eva;\\
sutvā devassa vassato, imam atthaṃ Dhaniyo abhāsatha: %\hfill\textcolor{gray}{\footnotesize 13}

\begin{enumerate}\item 如是,听了这显明四谛之颂,在偈颂的终了,有财者与夫人及其两个女儿等四人便住于须陀洹果。于是,有财者以与不坏净相应的、植根于如来的信,以慧眼得见世尊的法身,心为法性所激荡,便想:「下至无间,上至有顶,除世尊外,有谁能如是作狮子吼『我扯断了束缚,我不再入胎室』?难道是大师到了我处?」随后,世尊便在有财者的住处放出六色光网状的身光,如黄金散发金光般:「现在,随心所欲地看吧!」于是,有财者看到住处内如日月照临,周围如千灯炽燃般灿烂,便明白「世尊来临」,就在此时,云也降了雨。因此,结集者们说了此颂。
\item 这里,\textbf{低洼},即沼泽。\textbf{高地},即高坂。如是,\textbf{大云降雨,遍满}这高坂与斜坡,抹平了一切,即是说开始下雨。\textbf{顷刻之间},即在世尊放出身光且有财者以「大师到了我处」放出信所成的心光的刹那而降雨。然而,有人解释说「即在日出的刹那」。如是,在这有财者生起信、如来遍满光、日出的刹那,\textbf{有财者听到天降大雨}之声,生起喜悦,\textbf{便说了此义},即以下二颂。\end{enumerate}

\subsection\*{\textbf{31}}

\textbf{「我们的所得确实匪浅,我们得见世尊,\\}
\textbf{「我们皈依您,具眼者!请您作我们的大师,大牟尼!}

“Lābhā vata no anappakā, ye mayaṃ Bhagavantaṃ addasāma;\\
saraṇaṃ taṃ upema cakkhuma, satthā no hohi tuvaṃ mahāmuni. %\hfill\textcolor{gray}{\footnotesize 14}

\begin{enumerate}\item 这里,因为有财者及妻儿由通达圣道,以出世间眼见到世尊的法身,以世间眼见到色身,便获得了信,所以说「我们的所得确实匪浅,我们得见世尊」。这里,\textbf{确实},即惊异之义的不变词。\textbf{匪浅},即广大。余皆自明。
\item \textbf{我们皈依您},此中虽然由通达道便已成就皈依,但那里唯是决定行,现在则以言语自捐\footnote{据菩提比丘注 351,有四种世间的皈依,即自捐 \textit{attasanniyyātana}、以彼为归宿 \textit{tapparāyaṇatā}、成为弟子 \textit{sissabhāvūpagamana} 及跪拜 \textit{paṇipāta},见\textbf{长部}第 2 经义注。本颂的义注提到了其中三种,未提及「以彼为归宿」,代之以「不动皈依 \textit{acala°}」。}。或者,以道之力得至捐弃皈依、不动皈依,为向他人以言语表明而行跪拜皈依。\textbf{具眼者},即世尊以自然之眼、天眼、慧眼、普眼、佛眼等五眼为具眼者,称呼他而说「我们皈依您,具眼者」。\textbf{请您作我们的大师,大牟尼},他说此语以圆满成为弟子的皈依。\end{enumerate}

\subsection\*{\textbf{32}}

\textbf{「妻子与我都顺从,愿在善逝处修行梵行,\\}
\textbf{「我们愿得达生死的彼岸,得尽苦的边际。」}

Gopī ca ahañ ca assavā, brahmacariyaṃ Sugate carāmase;\\
jātimaraṇassa pāragū, dukkhass’antakarā bhavāmase”. %\hfill\textcolor{gray}{\footnotesize 15}

\begin{enumerate}\item 「妻子与我都顺从,愿在善逝处修行梵行」,即以受持(而说)。这里,\textbf{梵行},即离淫欲、道、沙门法、教法、满足于自己的妻子等的同义语。因为在\begin{quoting}行梵行。(中部第 8 经)\end{quoting}等处,离淫欲被称为梵行,在\begin{quoting}五学!这即是我的梵行,导向完全的厌离。(长部第 16 经)\end{quoting}等处,则是道,在\begin{quoting}我证知,舍利弗!已行具足四支的梵行。(中部第 12 经)\end{quoting}等处,则是沙门法,在\begin{quoting}此梵行既成功,且富有。(长部第 29 经)\end{quoting}等处,则是教法,在\begin{quoting}我们不违越妻子们,妻子们也不违越我们,\\除了她们,我们行梵行,所以我们的孩子不会死去。(本生第 10:97 颂)\end{quoting}等处,则是满足于自己的妻子,而在此处是指以沙门法梵行为前导的更上的道梵行。
\item \textbf{在善逝处},即在善逝跟前。因为世尊由避二边而善行,并由具足善净的圣道之行,由行于被称为涅槃的善妙之处,被称为「善逝」。且此中的依格为附近之义,所以意为「在善逝跟前」。
\item 如是,有财者以修行梵行为由,向世尊请求出家,为显明出家的目的,说了下半颂。\textbf{生死的彼岸},即名涅槃,我们将以阿罗汉道得达。\textbf{苦},即流转之苦。\textbf{得尽边际},即令无有。如是说已,据说,两人又再次顶礼了世尊,如是请求出家:「请世尊度我们出家!」\end{enumerate}

\subsection\*{\textbf{33}}

\textbf{「有孩子的因孩子而欢喜,」恶者魔罗说,「同样,有牛的因牛而欢喜,\\}
\textbf{「因为依持是人的欢喜,若离开依持,他就不会欢喜。」}

“Nandati puttehi puttimā, \textit{(iti Māro pāpimā)} gomā\footnote{PTS as \textit{gomiko}, B\textsuperscript{i} \textit{gopiyo}, Pj \textit{gomiyo}.} gohi tath’eva nandati;\\
upadhī hi narassa nandanā, na hi so nandati yo nirūpadhi”. %\hfill\textcolor{gray}{\footnotesize 16}

\begin{enumerate}\item 于是,恶者魔罗见到两人如是顶礼并请求出家,「他们想要越过我的境域,那就让我来为他们制造障碍」,前来显示居家的功德,说了此颂。这里,\textbf{欢喜},即满足、喜悦。\textbf{魔罗},即自在地\footnote{自在地:菩提比丘注 357,即指他化自在天。}中的某个暴力天子。因为他对想要越过其处的人,若能杀则杀,若不能则希望其死,以此被称为「魔罗」。\textbf{恶者},即恶劣之人,或恶行者。这(恶者魔罗说)是结集者们的话,并于一切偈颂中如此。且好比有孩子的因孩子,\textbf{同样,有牛的因牛而欢喜},即同样因牛而欢喜之义。
\item 如是说已,现在,他指出成就此义的缘由:\textbf{因为依持是人的欢喜}。这里,依持\footnote{依持 \textit{upadhi}:杂阿含经第 291 经作「亿波提」,第 1004 经作「有余」,别译杂阿含经第 142 经作「受身」。},即四种依持:爱欲依持、蕴依持、烦恼依持、行作依持。因为爱欲,如\begin{quoting}缘种种五欲,乐与喜生起,此即爱欲之味。(中部第 13 经)\end{quoting}所说,由作为乐的依处,以「此中的乐被持取」的语义而被称为依持,蕴由作为源于蕴之苦的依处,烦恼由作为苦处之苦的依处,行作由作为有之苦的依处(而被称为依持)。而此处是指爱欲依持,它依有情与行而有两种\footnote{有情与行:菩提比丘注 360,爱欲依持的两分是义注中对可执取的对象的区分,「行」可以指有情的部分或非生命体,如财产及其它物质所有。},这里,为显示主要是系缚于有情,便在说了「因孩子、因牛」后,说了原因:「因为依持是人的欢喜。」
\item 其义为:因为这些爱欲依持是人的欢喜,它们带来喜悦,使人欢喜,所以当知「有孩子的因孩子而欢喜,同样,有牛的因牛而欢喜,而你有孩子,也有牛,所以你应因这些而欢喜,切莫期望出家!因为出家人没有这些依持,在这样的情况下,你虽希求苦尽,将要受苦」。
\item 现在,他又指出成就此义的缘由:\textbf{若离开依持,他就不会欢喜}。其义为:因为若无这些依持者,他与可爱的亲属别离,无财产的资助,便不欢喜,所以你舍弃了这些依持而出家,将要受苦。\end{enumerate}

\subsection\*{\textbf{34}}

\textbf{「有孩子的因孩子而忧伤,」世尊说,「同样,有牛的因牛而忧伤,\\}
\textbf{「因为依持是人的忧伤,若离开依持,他就不会忧伤。」}

“Socati puttehi puttimā, \textit{(iti Bhagavā)} gomā gohi tath’eva socati;\\
upadhī hi narassa socanā, na hi so socati yo nirūpadhī” ti. %\hfill\textcolor{gray}{\footnotesize 17}

\begin{enumerate}\item 于是,世尊了知了「这是恶者魔罗前来障碍他们」,好比果子落到果子上,即以魔罗使用的譬喻来反驳魔罗的话,掉转前颂,为显示「依持是忧伤的依处」,说了此颂。
\item 这里,一切句子的意义自明,而其旨趣为:恶者!莫如是说「有孩子的因孩子而欢喜」!因为与一切可爱、可喜的分离、分别不可避免,这是规律,且有情因与这些可爱、可喜的孩子、妻子、牛、马、骡、货币、黄金等的分别,心愈加为忧箭所伤,心甚至疯狂、散乱,经历死亡,或受与死相当之苦,所以应如是把握:\textbf{有孩子的因孩子而忧伤},且好比有孩子的因孩子,\textbf{同样,有牛的因牛而忧伤}。什么原因?\textbf{因为依持是人的忧伤},且正因为依持是人的忧伤,所以\textbf{若离开依持,他就不会忧伤}。
\item 若以舍弃对依持的执著而离开依持,\begin{quoting}他即满足于防护身体之衣、防护胃腹之食,无论前往何处,唯持(衣食)而往,好比鸟……他了知「不受后有」。(中部第 27 经)\end{quoting}如是,以「若离开依持,他就不会忧伤」根除一切忧伤。如此,世尊以阿罗汉为顶点终结了开示。或者,若离开依持,没有烦恼,他就不会忧伤。因为只要有烦恼,一切依持即是忧伤之果,而由舍弃了烦恼,则无忧伤。如是,也以阿罗汉为顶点终结了开示。
\item 当开示终了,有财者与妻子二人便出了家。世尊仍从空中回到了祇园。他们出家后便证得了阿罗汉,并在居住之处教他们的牧牛人建了寺,时至今日,仍被称作「牧牛人寺」。\end{enumerate}

\begin{center}\vspace{1em}有财者经第二\\Dhaniyasuttaṃ dutiyaṃ.\end{center}

%\begin{flushright}癸卯七月廿日二稿\end{flushright}