\chapter{蕩之什詁訓傳第二十五}

\section{蕩}

%{\footnotesize 八章、章八句}

\textbf{蕩,召穆公傷周室大壞也。厲王無道,天下蕩蕩,無綱紀文章,故作是詩也。}

\textbf{蕩蕩上帝,下民之辟。}{\footnotesize 上帝以託君王也。辟,君也。箋云蕩蕩,法度廢壞之貌。厲王乃以此居人上,為天下之君,言其無可則象之甚。}\textbf{疾威上帝,其命多辟。}{\footnotesize 疾病人矣,威罪人矣。箋云疾病人者,重賦斂也,威罪人者,峻刑法也。其政教又多邪辟,不由舊章。}\textbf{天生烝民,其命匪諶。靡不有初,鮮克有終。}{\footnotesize 諶,誠也。箋云烝眾、鮮寡、克能也。天之生此眾民,其教道之,非當以誠信使之忠厚乎,今則不然,民始皆庶幾於善道,後更化於惡俗。}

\begin{quoting}諶 \texttt{chén}。\textbf{陳奐}言天生此天下之眾民,何其政教之不誠也。\end{quoting}

\textbf{文王曰咨,咨女殷商。曾是彊禦,曾是掊克,曾是在位,曾是在服。}{\footnotesize 咨,嗟也。彊禦,彊梁禦善也。掊克,自伐而好勝人也。服,服政事也。箋云厲王弭謗,穆公朝廷之臣,不敢斥今王之惡,故上陳文王咨嗟殷紂以切刺之,女曾任用是惡人,使之處位執職事也。}\textbf{天降滔德,女興是力。}{\footnotesize 天君、滔慢也。箋云厲王施倨慢之化,女群臣又相與而力為之,言競於惡。}

\begin{quoting}禦,魯詩齊詩作圉,彊、禦二字同義。\textbf{朱熹}彊禦,暴虐之臣也,掊克,聚斂之臣也。\textbf{孫鑛}批評詩經曰明是彊禦在位,掊克在服,乃分作四句,各喚以「曾是」字,以肆其態。\textbf{王安石}彊禦掊克,是謂滔德。\end{quoting}

\textbf{文王曰咨,咨女殷商。而秉義類,彊禦多懟。流言以對,寇攘式內。}{\footnotesize 對,遂也。箋云義之言宜也。類善、式用也。女執事之臣宜用善人,反任彊禦眾懟為惡者,皆流言謗毀賢者,王若問之,則又以對,寇盜攘竊為姦宄者而王信之,使用事於內。}\textbf{侯作侯祝,靡屆靡究。}{\footnotesize 作祝,詛也。屆極、究窮也。箋云侯,維也。王與群臣乖爭而相疑,日祝詛求其凶咎無極已。}

\begin{quoting}懟,怨也。內,入也,金文內、入同用。作祝,同詛咒。\end{quoting}

\textbf{文王曰咨,咨女殷商。女炰烋于中國,斂怨以為德。}{\footnotesize 炰烋,猶彭亨也。箋云炰烋,自矜氣健之貌。斂聚群不逞作怨之人,謂之有德而任用之。}\textbf{不明爾德,時無背無側。}{\footnotesize 後無臣,側無人也。箋云無臣無人,謂賢者不用。}\textbf{爾德不明,以無陪無卿。}{\footnotesize 無陪貳也,無卿士也。}

\begin{quoting}炰烋 \texttt{páo xiāo},亦作咆哮。時,韓詩作以。側,齊詩作仄。漢書顏注「不別善惡,有逆背傾仄者,有堪為卿大夫者,皆不知之也」。\end{quoting}

\textbf{文王曰咨,咨女殷商。天不湎爾以酒,不義從式。}{\footnotesize 義,宜也。箋云式,法也。天不同女顏色以酒,有沈湎於酒者,是乃過也,不宜從而法行之。}\textbf{既愆爾止,靡明靡晦。式號式呼,俾晝作夜。}{\footnotesize 使晝為夜也。箋云愆,過也。女既過沈湎矣,又不為明晦,無有止息,醉則號呼相俲,用晝日作夜,不視政事。}

\begin{quoting}釋文引韓詩「飲酒閉門不出客曰湎」。止,容止。\end{quoting}

\textbf{文王曰咨,咨女殷商。如蜩如螗,如沸如羹。}{\footnotesize 蜩,蟬也。螗,蝘也。箋云飲酒號呼之聲如蜩螗之鳴,其笑語沓沓又如湯之沸、羹之方熟。}\textbf{小大近喪,人尚乎由行。}{\footnotesize 言居人上,欲用行是道也。箋云殷紂之時,君臣失道如此且喪亡矣,時人化之甚,尚欲從而行之,不知其非。}\textbf{內奰于中國,覃及鬼方。}{\footnotesize 奰,怒也,不醉而怒曰奰。鬼方,遠方也。箋云此言時人忕於為惡,雖有不醉猶好怒。}

\begin{quoting}\textbf{馬瑞辰}詩意蓋謂時人悲嘆之聲如蜩螗之鳴,憂亂之心如沸羹之熟。奰 \texttt{bì}。覃,延及。\end{quoting}

\textbf{文王曰咨,咨女殷商。匪上帝不時,殷不用舊。}{\footnotesize 箋云此言紂之亂非其生不得其時,乃不用先王之故法之所致。}\textbf{雖無老成人,尚有典刑。}{\footnotesize 箋云老成人,謂若伊尹、伊陟、臣扈之屬。雖無此臣,猶有常事故法可案用也。}\textbf{曾是莫聽,大命以傾。}{\footnotesize 箋云莫,無也。朝廷君臣皆任喜怒,曾無用典刑治事者,以至誅滅。}

\textbf{文王曰咨,咨女殷商。人亦有言,顛沛之揭,枝葉未有害,本實先撥。}{\footnotesize 顛仆、沛拔也。揭,見根貌。箋云揭,蹶貌。撥,猶絕也。言大木揭然將蹶,枝葉未有折傷,其根本實先絕,乃相隨俱顛拔,喻紂之官職雖俱存,紂誅亦皆死。}\textbf{殷鑒不遠,在夏后之世。}{\footnotesize 箋云此言殷之明鏡不遠也,近在夏后之世,謂湯誅桀也,後武王誅紂,今之王者何以不用為戒。}

\begin{quoting}撥,同敗,列女傳引詩即作敗。\end{quoting}

\section{抑}

%{\footnotesize 十二章、三章章八句、九章章十句}

\textbf{抑,衛武公刺厲王,亦以自警也。}{\footnotesize 自警者,「如彼泉流,無淪胥以亡」。}

\begin{quoting}毛序蓋據國語楚語「昔衛武公年數九十有五矣⋯⋯於是作懿戒以自儆也」,韋昭注「昭謂懿詩,大雅抑之篇也,懿讀曰抑」。\end{quoting}

\textbf{抑抑威儀,維德之隅。人亦有言,靡哲不愚。}{\footnotesize 抑抑,密也。隅,廉也。靡哲不愚,國有道則知,國無道則愚。箋云人密審於威儀抑抑然,是其德必嚴正也。古之賢者,道行心平,可外占而知內,如宮室之制,內有繩直則外有廉隅,今王政暴虐,賢者皆佯愚不為,容貌如不肖然。}\textbf{庶人之愚,亦職維疾。哲人之愚,亦維斯戾。}{\footnotesize 職主、戾罪也。箋云庶,眾也。眾人性無知,以愚為主,言是其常也,賢者而為愚,畏懼於罪也。}

\begin{quoting}隅,同偶,漢劉熊碑引詩正作偶。\end{quoting}

\textbf{無競維人,四方其訓之。有覺德行,四國順之。}{\footnotesize 無競,競也。訓教、覺直也。箋云競,彊也。人君為政,無彊於得賢人,得賢人則天下教化,於其俗有大德行,則天下順從其政,言在上所以倡道。}\textbf{訏謨定命,遠猶辰告。}{\footnotesize 訏大、謨謀、猶道、辰時也。箋云猶,圖也。大謀定命,謂正月始和,布政于邦國都鄙也,為天下遠圖庶事,而以歲時告施之。}\textbf{敬慎威儀,維民之則。}{\footnotesize 箋云則,法也。}

\begin{quoting}呂覽求人篇高誘注「國之強惟在得人」。訓,左傳襄二十六年引詩作順。訏 \texttt{xū}。\end{quoting}

\textbf{其在于今,興迷亂于政。顛覆厥德,荒湛于酒。}{\footnotesize 箋云于今,謂今厲王也。興,猶尊尚也。王尊尚小人,迷亂於政事者,以傾敗其功德,荒廢其政事,又湛樂於酒,言愛小人之甚。}\textbf{女雖湛樂從,弗念厥紹。罔敷求先王,克共明刑。}{\footnotesize 紹繼、共執、刑法也。箋云罔,無也。女君臣雖好樂嗜酒而相從,不當念繼女之後人將俲女所為,無廣索先王之道與能執法度之人乎,切責之也。}

\begin{quoting}興,同虛,語詞。\textbf{陳奐}釋詞云「雖,維也」,古雖、維聲通,書無逸篇云「惟耽樂之從」,文義正與此同。共,同拱。\end{quoting}

\textbf{肆皇天弗尚,如彼泉流,無淪胥以亡。}{\footnotesize 淪,率也。箋云肆,故今也。胥,皆也。王為政如是,故今皇天不高尚之,所謂仍下災異也。王自絕於天,如泉水之流,稍就虛竭,無見率引為惡,皆與之以亡。戒群臣不中行者,將并誅之。}\textbf{夙興夜寐,洒埽廷內,維民之章。}{\footnotesize 洒灑、章表也。箋云章,文章法度也。厲王之時,不恤政事,故戒群臣掌事者以此也。}\textbf{修爾車馬,弓矢戎兵,用戒戎作,用逷蠻方。}{\footnotesize 逷,遠也。箋云逷,當作剔,剔,治也。蠻方,蠻畿之外也。此時中國微弱,故復戒將率之臣以治軍實,女當用此備兵事之起,用此治九州之外不服者。}

\begin{quoting}\textbf{馬瑞辰}爾雅「尚,右也」,右通作祐,祐者助也,弗尚即弗右耳。淪胥,相率、相隨。戎兵,魯詩作戈兵。用以、戒備、作起也。\end{quoting}

\textbf{質爾人民,謹爾侯度,用戒不虞。}{\footnotesize 質,成也。不虞,非度也。箋云侯,君也。此時萬民失職,亦不肯趨公事,故又戒鄉邑之大夫及邦國之君,平女萬民之事,慎女為君之法度,用備不億度而至之事。}\textbf{慎爾出話,敬爾威儀,無不柔嘉。}{\footnotesize 話,善言也。箋云言謂教令也。柔安、嘉善也。}\textbf{白圭之玷,尚可磨也,斯言之玷,不可為也。}{\footnotesize 玷,缺也。箋云斯,此也。玉之缺尚可磨鑢而平,人君政教一失,誰能反覆之。}

\begin{quoting}質,齊詩作誥,魯詩韓詩作吿。\end{quoting}

\textbf{無易由言,無曰苟矣,莫捫朕舌,言不可逝矣。}{\footnotesize 莫無、捫持也。箋云由於、逝往也。女無輕易於教令,無曰苟且如是,今人無持我舌者,而自聽恣也,教令一往行於下,其過誤可得而已之乎。}\textbf{無言不讎,無德不報。惠于朋友,庶民小子。}{\footnotesize 讎,用也。箋云惠,順也。教令之出如賣物,物善則其售賈貴,物惡則其售賈賤,德加於民,民則以義報之,王又當施順道於諸侯,下及庶民之子弟。}\textbf{子孫繩繩,萬民靡不承。}{\footnotesize 箋云繩繩,戒也。王之子孫敬戒行王之教令,天下之民不承順之乎,言承順也。}

\begin{quoting}讎,魯詩韓詩作酬,答也。\end{quoting}

\textbf{視爾友君子,輯柔爾顏,不遐有愆。}{\footnotesize 輯,和也。箋云柔安、遐遠也。今視女之諸侯及卿大夫皆脅肩諂笑,以和安女顏色,是於正道不遠有罪過乎,言其近也。}\textbf{相在爾室,尚不媿于屋漏。無曰不顯,莫予云覯。}{\footnotesize 西北隅謂之屋漏。覯,見也。箋云相助、顯明也。諸侯卿大夫助祭在女宗廟之室,尚無肅敬之心,不慙媿於屋漏有神見人之為也,女無謂是幽昧不明,無見我者,神見女矣。屋,小帳也。漏,隱也。禮,祭於奧,既畢,改設饌於西北隅而厞隱之處,此祭之末也。}\textbf{神之格思,不可度思,矧可射思。}{\footnotesize 格,至也。箋云矧況、射厭也。神之來至去止,不可度知,況可於祭末而有厭倦乎。}

\begin{quoting}\textbf{朱熹}言視爾友於君子之時,和柔爾之顏色,其戒懼之意,常若自省曰,豈不至於有過乎。尚,通上。屋漏,孔疏引孫炎曰「當室之白日光所漏入」。矧 \texttt{shěn}。射,同斁,厭也。\end{quoting}

\textbf{辟爾為德,俾臧俾嘉。淑慎爾止,不愆于儀。不僭不賊,鮮不為則。}{\footnotesize 女為善則民為善矣。止,至也。為人君止於仁,為人臣止於敬,為人子止於孝,為人父止於慈,與國人交止於信。僭,差也。箋云辟,法也。止,容止也。當審法度女之施德,使之為民臣所善所美,又當善慎女之容止,不可過差於威儀,女所行不不信、不殘賊者,少矣其不為人所法。}\textbf{投我以桃,報之以李。}{\footnotesize 箋云此言善往則善來,人無行而不得其報也。投,猶擲也。}\textbf{彼童而角,實虹小子。}{\footnotesize 童,羊之無角者也。而角,自用也。虹,潰也。箋云童羊,譬王后也,而角者,喻與政事有所害也,此人實潰亂小子之政。禮,天子未除喪,稱小子。}

\begin{quoting}辟,明也。賊,貳之訛也,慝也,僭慝,差爽。\end{quoting}

\textbf{荏染柔木,言緡之絲。溫溫恭人,維德之基。}{\footnotesize 緡,被也。溫溫,寬柔也。箋云柔忍之木荏染然,人則被之弦以為弓,寬柔之人溫溫然,則能為德之基止,言內有其性,乃可以有為德也。}\textbf{其維哲人,告之話言,順德之行。其維愚人,覆謂我僭,民各有心。}{\footnotesize 話言,古之善言也。箋云覆,猶反也。僭,不信也。語賢知之人以善言,則順行之,告愚人,反謂我不信,民各有心,二者意不同也。}

\begin{quoting}話言,釋文引說文作詁言,云「詁,故言也」。\end{quoting}

\textbf{於乎小子,未知臧否。匪手攜之,言示之事。匪面命之,言提其耳。}{\footnotesize 箋云臧,善也。於乎,傷王不知善否,我非但以手攜掣之,親示以其事之是非,我非但對面語之,親提撕其耳。此言以教道之孰,不可啟覺。}\textbf{借曰未知,亦既抱子。}{\footnotesize 借,假也。箋云假令人云王尚幼少,未有所知,亦已抱子長大矣,不幼少也。}\textbf{民之靡盈,誰夙知而莫成。}{\footnotesize 莫,晚也。箋云萬民之意皆持不滿於王,誰早有所知而反晚成與,言王之無成,本無知故也。}

\begin{quoting}借,齊詩作藉。\end{quoting}

\textbf{昊天孔昭,我生靡樂。視爾夢夢,我心慘慘。}{\footnotesize 夢夢,亂也。慘慘,憂不樂也。箋云孔甚、昭明也。昊天乎乃甚明察,我生無可樂也,視王之意夢夢然,我心之憂悶慘慘然,愬其自恣,不用忠臣。}\textbf{誨爾諄諄,聽我藐藐。匪用為敎,覆用為虐。}{\footnotesize 藐藐然,不入也。箋云我教告王,口語諄諄,然王聽聆之藐藐然忽略,不用我所言為政令,反謂之有妨害於事,不受忠言。}\textbf{借曰未知,亦聿既耄。}{\footnotesize 耄,老也。}

\begin{quoting}夢 \texttt{méng}。慘 \texttt{cǎo},魯詩作懆。虐,同謔,\textbf{馬瑞辰}詩蓋言不用其言為教令,反用其言為戲謔耳。\end{quoting}

\textbf{於乎小子,告爾舊止。聽用我謀,庶無大悔。}{\footnotesize 箋云舊,久也。止,辭也。庶幸、悔恨也。}\textbf{天方艱難,曰喪厥國。}{\footnotesize 箋云天以王為惡如是,故出艱難之事,謂下災異、生兵寇,將以滅亡。}\textbf{取譬不遠,昊天不忒。回遹其德,俾民大棘。}{\footnotesize 箋云今我為王取譬喻不及遠也,維近耳,王當如昊天之德有常不差忒也,王反為無常,維邪其行為貪暴,使民之財匱盡而大困急。}

\begin{quoting}回遹 \texttt{yù},邪僻。\end{quoting}

\section{桑柔}

%{\footnotesize 十六章、八章章八句、八章章六句}

\textbf{桑柔,芮伯刺厲王也。}{\footnotesize 芮伯,畿內諸侯,王卿士也,字良夫。}

\textbf{菀彼桑柔,其下侯旬。捋采其劉,瘼此下民。}{\footnotesize 興也。菀,茂貌。旬,言陰均也。劉,爆爍而希也。瘼,病也。箋云桑之柔濡,其葉菀然茂盛,謂蠶始生時也,人庇陰其下者均得其所,及已捋采之,則葉爆爍而疏,人息其下則病於爆爍。興者,喻民當被王之恩惠,群臣恣放,損王之德。}\textbf{不殄心憂,倉兄填兮。}{\footnotesize 倉喪、兄滋、填久也。箋云殄,絕也。民心之憂無絕已,喪亡之道滋久長。}\textbf{倬彼昊天,寧不我矜。}{\footnotesize 昊天,斥王者也。箋云倬,明大貌。昊天乃倬然明大,而不矜哀下民怨愬之言。}

\begin{quoting}倉兄,愴怳 \texttt{chuàng huǎng},淒涼紛亂貌。\end{quoting}

\textbf{四牡騤騤,旟旐有翩。亂生不夷,靡國不泯。}{\footnotesize 騤騤,不息也。鳥隼曰旟,龜蛇曰旐。翩翩,在路不息也。夷平、泯滅也。箋云軍旅久出征伐,而亂日生不平,無國而不見殘滅也。言王之用兵不得其所,適長寇虐。}\textbf{民靡有黎,具禍以燼。}{\footnotesize 黎,齊也。箋云黎,不齊也。具,猶俱也。災餘曰燼。言時民無有不齊被兵寇之害者,俱遇此禍以為燼者,言害所及廣。}\textbf{於乎有哀,國步斯頻。}{\footnotesize 步行、頻急也。箋云頻,猶比也。哀哉國家之政,行此禍害比比然。}

\begin{quoting}\textbf{王引之}泯,亂也,承上亂生不夷,故云靡國不亂也。\textbf{姚際恆}民靡有黎,猶「周餘黎民,靡有孑遺」之意。\textbf{王引之}黎,眾也,言民多死于禍亂,不復如前日之眾多,但留餘燼耳。\end{quoting}

\textbf{國步蔑資,天不我將。靡所止疑,云徂何往。}{\footnotesize 疑,定也。箋云蔑,猶輕也。將,猶養也。徂,行也。國家為政行此輕蔑民之資用,是天不養我也,我從兵役無有止息時,今復云行,當何之往也。}\textbf{君子實維,秉心無競。誰生厲階,至今為梗。}{\footnotesize 競彊、厲惡、梗病也。箋云君子謂諸侯及卿大夫也。其執心不彊於善而好以力爭,誰始生此禍者,乃至今日相梗不止。}

\begin{quoting}資,同濟。\textbf{馬瑞辰}猶言天不扶助我耳。疑,與止同義。維,同惟,思也。\end{quoting}

\textbf{憂心慇慇,念我土宇。我生不辰,逢天僤怒。自西徂東,靡所定處。}{\footnotesize 宇居、僤厚也。箋云辰,時也。此士卒從軍久,勞苦自傷之言。}\textbf{多我覯痻,孔棘我圉。}{\footnotesize 圉,垂也。箋云痻,病也。圉,當作禦。多矣我之遇困病,甚急矣我之禦寇之事。}

\begin{quoting}慇慇,魯詩作隱隱。僤,同憚。圉,邊疆。\end{quoting}

\textbf{為謀為毖,亂況斯削。}{\footnotesize 毖,慎也。箋云女為軍旅之謀,為重慎兵事也,而亂滋甚於此,日見侵削,言其所任非賢。}\textbf{告爾憂恤,誨爾序爵。誰能執熱,逝不以濯。}{\footnotesize 濯所以救熱也,禮亦所以救亂也。箋云恤,亦憂也。逝,猶去也。我語女以憂天下之憂,教女以次序賢能之爵,其為之當如手持熱物之用濯,謂治國之道當用賢者。}\textbf{其何能淑,載胥及溺。}{\footnotesize 箋云淑善、胥相、及與也。女若云此於政事何能善乎,則女君臣皆相與陷溺於禍難也。}

\begin{quoting}\textbf{馬瑞辰}亂況,猶亂狀也,詩蓋言在上者如善其謀、慎其事,亂狀斯能削減耳。又曰公羊隱七年傳「不與夷狄之執中國也」,何注「執者,治之也」,救亦治也,執熱即治熱,亦即救熱,言誰能救熱而不以濯也。逝,語詞。\textbf{蘇轍}賢者之能已亂,猶濯之能解熱耳,不然則其何能善哉,相與入於陷溺而已。\end{quoting}

\textbf{如彼遡風,亦孔之僾。民有肅心,荓云不逮。好是稼穡,力民代食。}{\footnotesize 遡鄉、僾唈、荓使也。力民代食,代無功者食天祿也。箋云肅進、逮及也。今王之為政,見之使人唈然,如鄉疾風不能息也,王為政,民有進於善道之心,當任用之,反却退之,使不及門,但好任用是居家吝嗇、於聚斂作力之人,令代賢者處位食祿。明王之法,能治人者食於人,不能治人者食人,禮記曰「與其有聚斂之臣,寧有盜臣,聚斂之臣害民,盜臣害財」。}\textbf{稼穡維寶,代食維好。}{\footnotesize 箋云此言王不尚賢,但貴吝嗇之人與愛代食者而已。}

\begin{quoting}僾,氣噎而呼吸不暢貌。荓 \texttt{pīng}。\end{quoting}

\textbf{天降喪亂,滅我立王。降此蟊賊,稼穡卒痒。}{\footnotesize 箋云滅,盡也。蟲食苗根曰蟊,食節曰賊。耕種曰稼,收斂曰穡。卒盡、痒病也。天下喪亂國家之災,以窮盡我王所恃而立者,謂蟲孽為害,五穀盡病。}\textbf{哀恫中國,具贅卒荒。靡有旅力,以念穹蒼。}{\footnotesize 贅屬、荒虛也。穹蒼,蒼天。箋云恫,痛也。哀痛乎中國之人,皆見繫屬於兵役,家家空虛,朝廷曾無有同力諫諍,念天所為下此災。}

\begin{quoting}\textbf{陳奐}具贅卒荒,承上文「降此蟊賊,稼穡卒痒」言之,猶云饑饉薦臻耳。\end{quoting}

\textbf{維此惠君,民人所瞻。秉心宣猶,考慎其相。}{\footnotesize 相,質也。箋云惠順、宣徧、猶謀、慎誠、相助也。維至德順民之君,為百姓所瞻仰者,乃執正心,舉事徧謀於眾,又考誠其輔相之行,然後用之,言擇賢之審。}\textbf{維彼不順,自獨俾臧。自有肺腸,俾民卒狂。}{\footnotesize 箋云臧,善也。彼不施順道之君,自多足獨謂賢,言其所任之臣皆善人也,不復考慎,自有肺腸行其心中之所欲,乃使民盡迷惑如狂,是又不宣猶。}

\begin{quoting}\textbf{馬瑞辰}秉心宣猶,言其執心明且順耳。\textbf{林義光}考慎其相,言不僅求自利,亦必思利人,與下文自獨俾臧相對,自獨俾臧,使己獨利也。\end{quoting}

\textbf{瞻彼中林,甡甡其鹿。朋友已譖,不胥以穀。}{\footnotesize 甡甡,眾多也。箋云譖,不信也。胥,相也。以,猶與也。穀,善也。視彼林中,其鹿相輩耦行,甡甡然眾多,今朝廷群臣皆相欺,皆不相與以善道,言其鹿之不如。}\textbf{人亦有言,進退維谷。}{\footnotesize 谷,窮也。箋云前無明君,却迫罪役,故窮也。}

\begin{quoting}\textbf{馬瑞辰}蓋鹿性旅行,見食相呼,有朋友群聚之象,故詩以興朋友之不相善。谷,同鞠。\end{quoting}

\textbf{維此聖人,瞻言百里。維彼愚人,覆狂以喜。}{\footnotesize 瞻言百里,遠慮也。箋云聖人所視而言者百里,言見事遠而王不用,有愚闇之人為王言其事,淺且近耳,王反迷惑信用之而喜。}\textbf{匪言不能,胡斯畏忌。}{\footnotesize 箋云胡之言何也。賢者見此事之是非,非不能分別皁白言之於王也,然不言之,何也,此畏懼犯顏得罪罰。}

\textbf{維此良人,弗求弗迪。維彼忍心,是顧是復。}{\footnotesize 迪,進也。箋云良,善也。國有善人,王不求索不進用之,有忍為惡之心者,王反顧念而重復之,言其忽賢者而愛小人。}\textbf{民之貪亂,寧為荼毒。}{\footnotesize 箋云貪,猶欲也。天下之民苦王之政,欲其亂亡,故安為苦毒之行相侵暴,慍恚使之然。}

\textbf{大風有隧,有空大谷。}{\footnotesize 隧,道也。箋云西風謂之大風。大風之行有所從而來,必從大空谷之中,喻賢愚之所行各由其性也。}\textbf{維此良人,作為式穀。維彼不順,征以中垢。}{\footnotesize 中垢,言闇冥也。箋云作起、式用、征行也。賢者在朝則用其善道,不順之人則行闇冥,受性於天,不可變也。}

\begin{quoting}白駒毛傳「空,大也」。\end{quoting}

\textbf{大風有隧,貪人敗類。聽言則對,誦言如醉。}{\footnotesize 類,善也。箋云類,等夷也。對,答也。貪惡之人見道聽之言則應答之,見誦詩書之言則冥臥如醉,居上位而行此,人或俲之。}\textbf{匪用其良,覆俾我悖。}{\footnotesize 覆,反也。箋云居上位而不用善,反使我為悖逆之行,是形其敗類之驗。}

\begin{quoting}廣雅「聽、聆,從也」。\end{quoting}

\textbf{嗟爾朋友,予豈不知而作。如彼飛蟲,時亦弋獲。}{\footnotesize 箋云嗟爾朋友者,親而切瑳之也。而,猶女也。我豈不知女所行者惡與,直知之,女所行如是,猶鳥飛行自恣東西南北時,亦為弋射者所得,言放縱久無所拘制,則將遇伺女之間者得誅女也。}\textbf{既之陰女,反予來赫。}{\footnotesize 赫,炙也。箋云之,往也。口距人謂之赫。我恐女見弋獲,既往覆陰女,謂啟告之以患難也,女反赫我,出言悖怒,不受忠告。}

\begin{quoting}\textbf{馬瑞辰}詩以飛鳥之難射,時亦以弋射獲之,喻貪人之難知,時亦以窺測得之耳。之,語詞。陰,同諳,熟知也。\end{quoting}

\textbf{民之罔極,職涼善背。}{\footnotesize 涼,薄也。箋云職主、涼信也。民之行失其中者,主由為政者信用小人,工相欺違。}\textbf{為民不利,如云不克。}{\footnotesize 箋云克,勝也。為政者害民,如恐不得其勝,言至酷也。}\textbf{民之回遹,職競用力。}{\footnotesize 箋云競,逐也。言民之行維邪者,主由為政者逐用彊力相尚故也,言民愁困,用生多端。}

\textbf{民之未戾,職盜為寇。}{\footnotesize 戾,定也。箋云為政者主作盜賊為寇害,令民心動搖不安定也。}\textbf{涼曰不可,覆背善詈。}{\footnotesize 箋云善,猶大也。我諫止之以信,言女所行者不可,反背我而大詈,言距己諫之甚。}\textbf{雖曰匪予,既作爾歌。}{\footnotesize 箋云予,我也。女雖觝距己言,此政非我所為,我已作女所行之歌,女當受之而改悔。}

\begin{quoting}廣雅「戾,善也」。涼,同諒,誠也,\textbf{林義光}涼曰不可者,正告之以不可也。匪,同誹。\end{quoting}

\section{雲漢}

%{\footnotesize 八章、章十句}

\textbf{雲漢,仍叔美宣王也。宣王承厲王之烈,內有撥亂之志,遇烖而懼,側身脩行欲銷去之,天下喜於王化復行,百姓見憂,故作是詩也。}{\footnotesize 仍叔,周大夫也,春秋魯桓公五年「夏,天王使仍叔之子來聘」。烈,餘也。}

\begin{quoting}\textbf{釋文}自此至常武六篇,宣王之變大雅。\textbf{姚際恆}棫樸篇以雲漢喻文章則曰為章,此以雲漢言旱則曰昭回。\end{quoting}

\textbf{倬彼雲漢,昭回于天。}{\footnotesize 回,轉也。箋云雲漢,謂天河也。昭,光也。倬然天河水氣也,精光轉運於天。時旱渴雨,故宣王夜仰視天河,望其候焉。}\textbf{王曰於乎,何辜今之人。天降喪亂,饑饉薦臻。}{\footnotesize 薦重、臻至也。箋云辜,罪也。王憂旱而嗟歎云,何罪與今時天下之人,天仍下旱災亡亂之道,饑饉之害復重至也。}\textbf{靡神不舉,靡愛斯牲。圭璧既卒,寧莫我聽。}{\footnotesize 箋云靡、莫皆無也。言王為旱之故求於群神,無不祭也,無所愛於三牲,禮神之圭璧又已盡矣,曾無聽聆我之精誠而興雲雨。}

\begin{quoting}薦,齊詩作荐。禮記王制鄭注「舉,猶祭也」。周人祭天則堆柴焚玉,祭山神地神則埋玉,祭水則沉玉於水,祭人鬼則藏玉。\end{quoting}

\textbf{旱既大甚,蘊隆蟲蟲。}{\footnotesize 蘊蘊而暑,隆隆而雷,蟲蟲而熱。箋云隆隆而雷,非雨雷也,雷聲尚殷殷然。}\textbf{不殄禋祀,自郊徂宮。上下奠瘞,靡神不宗。}{\footnotesize 上祭天,下祭地,奠其禮,瘞其物。宗,尊也。國有凶荒,則索鬼神而祭之。箋云宮,宗廟也。為旱故絜祀不絕,從郊而至宗廟,奠瘞天地之神,無不齊肅而尊敬之,言徧至也。}\textbf{后稷不克,上帝不臨。耗斁下土,寧丁我躬。}{\footnotesize 丁,當也。箋云克,當作刻,刻,識也。斁,敗也。奠瘞群神而不得雨,是我先祖后稷不識知我之所困與,天不視我之精誠與,猶以旱耗敗天下為害,曾使當我之身有此乎。先后稷,後上帝,亦從宮之郊。}

\begin{quoting}蟲,韓詩作炯。斁 \texttt{dù}。\end{quoting}

\textbf{旱既大甚,則不可推。兢兢業業,如霆如雷。周餘黎民,靡有孑遺。}{\footnotesize 推,去也。兢兢,恐也。業業,危也。孑然遺失也。箋云黎,眾也。旱既不可移去,天下困於饑饉,皆心動意懼,兢兢然,業業然,狀如有雷霆近發於上,周之眾民多有死亡者矣,今其餘無有孑遺者,言又餓病也。}\textbf{昊天上帝,則不我遺。胡不相畏,先祖于摧。}{\footnotesize 摧,至也。箋云摧,當作嗺,嗺,嗟也。天將遂旱餓殺我與,先祖何不助我恐懼,使天雨也,先祖之神于嗟乎,告困之辭。}

\textbf{旱既大甚,則不可沮。赫赫炎炎,云我無所。大命近止,靡瞻靡顧。}{\footnotesize 沮,止也。赫赫,旱氣也。炎炎,熱氣也。大命近止,民近死亡也。箋云旱既不可却止,熱氣大盛,人皆不堪,言我無所庇陰而處,眾民之命近將死亡,天曾無所視、無所顧,於此國中而哀閔之。}\textbf{群公先正,則不我助。父母先祖,胡寧忍予。}{\footnotesize 先正,百辟卿士也。先祖文武,為民父母也。箋云百辟卿士雩祀所及者,今曾無肯助我憂旱,先祖文武又何為施忍於我,不使天雨。}

\begin{quoting}云,雲古字,庇蔭也。\end{quoting}

\textbf{旱既大甚,滌滌山川。旱魃為虐,如惔如焚。我心憚暑,憂心如熏。}{\footnotesize 滌滌,旱氣也,山無木,川無水。魃,旱神也。惔,燎之也。憚勞、熏灼也。箋云憚,猶畏也。旱既害於山川矣,其氣生魃而害益甚,草木燋枯,如見焚燎然,王心又畏難此熱氣如灼爛於火,言熱氣至極。}\textbf{群公先正,則不我聞。昊天上帝,寧俾我遯。}{\footnotesize 箋云不我聞者,忽然不聽我之所言也,天曾將使我心遜遯慙愧於天下,以無德也。}

\begin{quoting}魃 \texttt{bá}。惔,三家詩作炎。\textbf{馬瑞辰}聞,當讀問,問猶恤問也。\end{quoting}

\textbf{旱既大甚,黽勉畏去。胡寧瘨我以旱,憯不知其故。}{\footnotesize 箋云瘨,病也。黽勉,急禱請也。欲使所尤畏者去,所尤畏者,魃也,天何曾病我以旱,曾不知為政所失而致此害。}\textbf{祈年孔夙,方社不莫。昊天上帝,則不我虞。敬恭明神,宜無悔怒。}{\footnotesize 悔,恨也。箋云虞,度也。我祈豐年甚早,祭四方與社又不晚,天曾不度知我心,肅事明神如是,明神宜不恨怒於我,我何由當遭此旱也。}

\begin{quoting}畏去,當讀作畏却,擔憂也。瘨 \texttt{diān}。憯 \texttt{cǎn},曾也。甫田毛傳「社,后土也,方,迎四方氣於郊也」。廣雅釋詁「虞,助也」。\end{quoting}

\textbf{旱既大甚,散無友紀。鞫哉庶正,疚哉冢宰。趣馬師氏,膳夫左右。}{\footnotesize 歲凶年穀不登則趣馬不秣,師氏弛其兵,馳道不除,祭事不縣,膳夫徹膳,左右布而不脩,大夫不食粱,士飲酒不樂。箋云人君以群臣為友,散無其紀者,凶年祿餼不足,又無賞賜也。鞫,窮也。庶正,眾官之長也。疚,病也。窮哉病哉者,念此諸臣勤於事而困於食,以此言勞倦也。}\textbf{靡人不周,無不能止。}{\footnotesize 周,救也。無不能止,言無止不能也。箋云周,當作賙。王以諸臣困於食,人人賙給之,權救其急,後日乏無,不能豫止。}\textbf{瞻卬昊天,云如何里。}{\footnotesize 箋云里,憂也。王愁悶於不雨,但仰天曰當如我之憂何。}

\begin{quoting}友,同有。\textbf{王肅}無不能而止者,其發倉廩,散積聚,有分無,多分寡,無敢有不能而止者,言上下同也。卬,同仰。里,同悝。\end{quoting}

\textbf{瞻卬昊天,有嘒其星。大夫君子,昭假無贏。大命近止,無棄爾成。}{\footnotesize 嘒,眾星貌。假,至也。箋云假,升也。王仰天見眾星順天而行嘒嘒然,意感,故謂其卿大夫曰,天之光耀升行不休,無自贏緩之時,今眾民之命近將死亡,勉之助我,無棄女之成功者,若其在職,復無幾何以勸之也。}\textbf{何求為我,以戾庶正。}{\footnotesize 戾,定也。箋云使女無棄成功者何,但求為我身乎,乃欲以安定眾官之長,憂其職事。}\textbf{瞻卬昊天,曷惠其寧。}{\footnotesize 箋云曷,何也。王仰天曰,當何時順我之求,令心安乎,渴雨之至也,得雨則心安。}

\begin{quoting}假,通格,昭假,禱告也。\textbf{王肅}大夫君子,公卿大夫也,昭其至誠於天下,無敢有私贏之而不敷散,大夫君子所以無私贏者,以民近於死亡,當賑救之,以全汝之成功。\textbf{吳闓生}曷惠,猶曷維也。甲骨常用虛字「叀」即此惠字。\end{quoting}

\section{崧高}

%{\footnotesize 八章、章八句}

\textbf{崧高,尹吉甫美宣王也。天下復平,能建國親諸侯,褒賞申伯焉。}{\footnotesize 尹吉甫、申伯皆周之卿士也,尹,官氏,申,國名。}

\textbf{崧高維嶽,駿極于天。維嶽降神,生甫及申。}{\footnotesize 崧,高貌,山大而高曰崧。嶽,四嶽也,東嶽岱,南嶽衡,西嶽華,北嶽恆。堯之時,姜氏為四伯,掌四嶽之祀,述諸侯之職,於周則有甫、有申、有齊、有許也。駿大、極至也。嶽降神靈,和氣以生,申甫之大功。箋云降,下也。四嶽,卿士之官,掌四時者也,因主方嶽巡守之事,在堯時姜姓為之,德當嶽神之意,而福興其子孫,歷虞夏商世有國土,周之甫也、申也、齊也、許也皆其苗胄。}\textbf{維申及甫,維周之翰。四國于蕃,四方于宣。}{\footnotesize 翰,榦也。箋云申,申伯也,甫,甫侯也,皆以賢知入為周之楨榦之臣,四國有難則往扞禦之,為之蕃屏,四方恩澤不至則往宣暢之。甫侯相穆王,訓夏贖刑,美此俱出四嶽,故連言之。}

\begin{quoting}崧,三家詩作嵩,爾雅釋山「嵩高為中嶽」。甫,即呂,蔡邕司空楊公碑「昔在申呂,匡佐周宣,崧高作誦,大雅揚言」。申、呂皆在南陽附近。宣,同垣。\end{quoting}

\textbf{亹亹申伯,王纘之事。于邑于謝,南國是式。}{\footnotesize 謝,周之南國也。箋云亹亹,勉也。纘繼、于往、于於、式法也。亹亹然勉於德不倦之臣有申伯,以賢人為王之卿士,佐王有功,王又欲使繼其故諸侯之事,往作邑於謝,南方之國皆統理施其法度。時改大其邑,使為侯伯,故云然。}\textbf{王命召伯,定申伯之宅。登是南邦,世執其功。}{\footnotesize 召伯,召公也。登,成也。功,事也。箋云之,往也。申伯忠臣,不欲離王室,故王使召公定其意,令往居謝,成法度於南邦,世世持其政事,傳子孫也。}

\textbf{王命申伯,式是南邦,因是謝人,以作爾庸。}{\footnotesize 庸,城也。箋云庸,功也。召公既定申伯之居,王乃親命之,使為法度於南邦,今因是故謝邑之人而為國,以起女之功勞,言尤章顯也。}\textbf{王命召伯,徹申伯土田。}{\footnotesize 徹,治也。箋云治者,正其井牧,定其賦稅。}\textbf{王命傅御,遷其私人。}{\footnotesize 御,治事之官也。私人,家臣也。箋云傅御者,貳王治事,謂冢宰也。}

\textbf{申伯之功,召伯是營。有俶其城,寢廟既成。}{\footnotesize 俶,作也。箋云申伯居謝之事,召公營其位而作城郭及寢廟,定其人神所處。}\textbf{既成藐藐,王錫申伯,四牡蹻蹻,鉤膺濯濯。}{\footnotesize 藐藐,美貌。蹻蹻,壯貌。鉤膺,樊纓也。濯濯,光明也。箋云召公營位,築之已成,以形貌告於王,王乃賜申伯,為將遣之。}

\begin{quoting}功,事也。說文「俶,善也」。\end{quoting}

\textbf{王遣申伯,路車乘馬。我圖爾居,莫如南土。}{\footnotesize 乘馬,四馬也。箋云王以正禮遣申伯之國,故復有車馬之賜,因告之曰,我謀女之所處,無如南土之最善。}\textbf{錫爾介圭,以作爾寶。}{\footnotesize 寶,瑞也。箋云圭長尺二寸謂之介,非諸侯之圭,故以為寶。諸侯之瑞圭自九寸而下。}\textbf{往䢋王舅,南土是保。}{\footnotesize 䢋,已也。申伯,宣王之舅也。箋云䢋,辭也,聲如彼記之子之記。保,守也、安也。}

\begin{quoting}介,魯詩作玠,大也。\textbf{陳奐}往䢋王舅,言王舅往耳。\end{quoting}

\textbf{申伯信邁,王餞于郿。}{\footnotesize 郿,地名。箋云邁,行也。申伯之意不欲離王室,王告語之復重,於是意解而信行。餞,送行飲酒也。時王蓋省岐周,故于郿云。}\textbf{申伯還南,謝于誠歸。}{\footnotesize 箋云還南者,北就王命于岐周而還反也。謝于誠歸,誠歸于謝。}\textbf{王命召伯,徹申伯土疆。以峙其粻,式遄其行。}{\footnotesize 箋云粻糧、式用、遄速也。王使召公治申伯土界之所至,峙其糧者,令廬巿有止宿之委積,用是速申伯之行。}

\begin{quoting}峙,儲備。\end{quoting}

\textbf{申伯番番,既入于謝,徒御嘽嘽。}{\footnotesize 番番,勇武貌。諸侯有大功則賜虎賁徒御。嘽嘽,徒行者、御車者嘽嘽喜樂也。箋云申伯之貌有威武番番然,其入謝國,車徒之行嘽嘽安舒,言得禮也。禮,入國不馳。}\textbf{周邦咸喜,戎有良翰。}{\footnotesize 箋云周,徧也。戎,猶女也。翰,榦也。申伯入謝,徧邦內皆喜曰「女乎有善君也」,相慶之言。}\textbf{不顯申伯,王之元舅,文武是憲。}{\footnotesize 不顯申伯,顯矣申伯也。文武是憲,言有文有武也。箋云憲,表也,言為文武之表式。}

\begin{quoting}番 \texttt{bō}。嘽 \texttt{tān}。\end{quoting}

\textbf{申伯之德,柔惠且直。揉此萬邦,聞于四國。}{\footnotesize 箋云揉,順也。四國,猶言四方也。}\textbf{吉甫作誦,其詩孔碩。其風肆好,以贈申伯。}{\footnotesize 吉甫,尹吉甫也。作是工師之誦也。肆,長也。贈,增也。箋云碩,大也。吉甫為此誦也,言其詩之意甚美大,風切申伯,又使之長行善道,以此贈申伯者,送之令以為樂。}

\begin{quoting}\textbf{朱熹}風,聲。肆,極也。\end{quoting}

\section{烝民}

%{\footnotesize 八章、章八句}

\textbf{烝民,尹吉甫美宣王也。任賢使能,周室中興焉。}

\begin{quoting}\textbf{朱熹}宣王命樊侯仲山甫築城於齊,而尹吉甫作詩以送之。\textbf{姚際恆}三百篇說理始此,蓋在宣王之世矣。\end{quoting}

\textbf{天生烝民,有物有則。民之秉彝,好是懿德。}{\footnotesize 烝眾、物事、則法、彞常、懿美也。箋云秉,執也。天之生眾民,其性有物象,謂五行仁義禮智信也,其情有所法,謂喜怒哀樂好惡也,然而民所執持有常道,莫不好有美德之人。}\textbf{天監有周,昭假于下。保茲天子,生仲山甫。}{\footnotesize 仲山甫,樊侯也。箋云監視、假至也。天視周王之政教,其光明乃至于下,謂及眾民也,天安愛此天子宣王,故生樊侯仲山甫使佐之,言天亦好是懿德也。書曰「天聦明,自我民聦明」。}

\begin{quoting}\textbf{胡承珙}有物指天,有則指人之法天。昭假,祈禱。\end{quoting}

\textbf{仲山甫之德,柔嘉維則。令儀令色,小心翼翼。}{\footnotesize 箋云嘉美、令善也。善威儀,善顏色容貌,翼翼然恭敬。}\textbf{古訓是式,威儀是力。天子是若,明命使賦。}{\footnotesize 古故、訓道、若順、賦布也。箋云故訓,先王之遺典也。式,法也。力,猶勤也,勤威儀者,恪居官次,不解于位也。是順從行其所為也,顯明王之政教,使群臣施布之。}

\begin{quoting}古,魯詩作故。賦,同敷,頒布。\end{quoting}

\textbf{王命仲山甫,式是百辟,纘戎祖考,王躬是保。}{\footnotesize 戎,大也。箋云戎,猶女也。躬,身也。王曰「女施行法度於是百君,繼女先祖先父始見命者之功德,王身是安」,使盡心力於王室。}\textbf{出納王命,王之喉舌。賦政于外,四方爰發。}{\footnotesize 喉舌,冢宰也。箋云出王命者,王口所自言,承而施之也,納王命者,時之所宜,復於王也,其行之也,皆奉順其意,如王口喉舌親所言也,以布政於畿外,天下諸侯於是莫不發應。}

\textbf{肅肅王命,仲山甫將之。邦國若否,仲山甫明之。}{\footnotesize 將,行也。箋云肅肅,敬也。言王之政教甚嚴敬也,仲山甫則能奉行之。若,順也,順否,猶臧否,謂善惡也。}\textbf{既明且哲,以保其身。夙夜匪解,以事一人。}{\footnotesize 箋云夙早、夜莫、匪非也。一人,斥天子。}

\begin{quoting}肅肅,齊詩作赫赫。若,猶惟,語詞。\textbf{于省吾}言邦國當沉晦之時,仲山甫有以通其閉塞,若否乃古人語例,毛公鼎「虩許上下若否」,言上下隔閡不相融洽也。解,魯詩、韓詩作懈。\end{quoting}

\textbf{人亦有言,柔則茹之,剛則吐之。}{\footnotesize 箋云柔,猶濡毳也。剛,堅彊也。剛柔之在口,或茹之,或吐之,喻人之於敵強弱。}\textbf{維仲山甫,柔亦不茹,剛亦不吐。不侮矜寡,不畏彊禦。}

\begin{quoting}矜,左傳昭元年引詩作鰥。彊禦,漢書王莽傳引詩作彊圉,強暴凌弱。孔疏「不侮不畏即是不茹不吐,既言其喻,又言其實」。\end{quoting}

\textbf{人亦有言,德輶如毛,民鮮克舉之,我儀圖之。}{\footnotesize 儀,宜也。箋云輶輕、儀匹也。人之言云「德甚輕,然而眾人寡能獨舉之以行者」,言政事易耳,而人不能行者,無其志也,我與倫匹圖之而未能為也。我,吉甫自我也。}\textbf{維仲山甫舉之,愛莫助之。}{\footnotesize 愛,隱也。箋云愛,惜也。仲山甫能獨舉此德而行之,惜乎莫能助之者,多仲山甫之德,歸功言耳。}\textbf{袞職有闕,維仲山甫補之。}{\footnotesize 有袞冕者,君之上服也。仲山甫補之,善補過也。箋云袞職者,不敢斥王之言也。王之職有闕,輒能補之者,仲山甫也。}

\begin{quoting}輶 \texttt{yóu}。儀圖,揣度。\textbf{馬瑞辰}隱者見之不真,凡舉物者皆有形而德之舉也無形,凡有形者可助而無形者不可助,故曰「愛莫助之」。\textbf{俞樾}職讀為識,識猶適也。\end{quoting}

\textbf{仲山甫出祖,四牡業業,征夫捷捷,每懷靡及。}{\footnotesize 言述職也。業業,言高大也。捷捷,言樂事也。箋云祖者,將行犯軷之祭也。懷私為每懷。仲山甫犯軷而將行,車馬業業然動,眾行夫捷捷然至,仲山甫則戒之曰「既受君命,當速行」,每人懷其私而相稽留,將無所及於事。}\textbf{四牡彭彭,八鸞鏘鏘。王命仲山甫,城彼東方。}{\footnotesize 東方,齊也。古者諸侯之居逼隘,則王者遷其邑而定其居,蓋去薄姑而遷於臨菑也。箋云彭彭,行貌。鏘鏘,鳴聲。以此車馬命仲山甫使行,言其盛也。}

\begin{quoting}孔疏引王肅曰「仲山甫雖有柔和明知之德,猶自謂無及」。\end{quoting}

\textbf{四牡騤騤,八鸞喈喈。仲山甫徂齊,式遄其歸。}{\footnotesize 騤騤,猶彭彭也。喈喈,猶鏘鏘也。遄,疾也。言周之望仲山甫也。箋云望之,故欲其用是疾歸。}\textbf{吉甫作誦,穆如清風。仲山甫永懷,以慰其心。}{\footnotesize 清微之風,化養萬物者也。箋云穆,和也。吉甫作此工歌之誦,其調和人之性,如清風之養萬物然,仲山甫述職,多所思而勞,故述其美以慰安其心。}

\section{韓奕}

%{\footnotesize 六章、章十二句}

\textbf{韓奕,尹吉甫美宣王也。能錫命諸侯。}{\footnotesize 梁山於韓國之山最高大,為國之鎮,祈望祀焉,故美大其貌奕奕然,謂之韓奕也。梁山,今左馮翊夏陽西北。韓,姬姓之國也,後為晉所滅,故大夫韓氏以為邑名焉。幽王九年,王室始騷,鄭桓公問於史伯曰「周衰,其孰興乎」,對曰「武實昭文之功,文之祚盡,武其嗣乎,武王之子,應、韓不在,其晉乎」。}

\begin{quoting}\textbf{陳廷傑}此詩專美韓侯。\end{quoting}

\textbf{奕奕梁山,維禹甸之。有倬其道,韓侯受命。}{\footnotesize 奕奕,大也。甸,治也。禹治梁山,除水災,宣王平大亂,命諸侯。有倬其道,有倬然之道者也。受命,受命為侯伯也。箋云梁山之野堯時俱遭洪水,禹甸之者,決除其災,使成平田、定貢賦於天子,周有厲王之亂,天下失職,今有倬然著明復禹之功者,韓侯受王命為侯伯。}\textbf{王親命之,纘戎祖考,無廢朕命,夙夜匪解,虔共爾位。}{\footnotesize 戎大、虔固、共執也。箋云戎,猶女也。朕,我也。古之恭字或作共。}\textbf{朕命不易,榦不庭方,以佐戎辟。}{\footnotesize 庭,直也。箋云我之所命者,勿改易不行,當為不直違失法度之方作楨榦而正之,以佐助女君,女君,王自謂也。}

\begin{quoting}梁山,在今河北固安附近。\textbf{馬瑞辰}易,當讀為難易之易,周頌「命不易哉」、書大誥「爾亦不知天命不易」讀與此同。\textbf{陳奐}榦不庭方,言四方有不直者則正之,侯伯得專征伐也。\end{quoting}

\textbf{四牡奕奕,孔脩且張,韓侯入覲。以其介圭,入覲于王。}{\footnotesize 脩長、張大、覲見也。箋云諸侯秋見天子曰覲。韓侯乘長大之四牡,奕奕然以時覲於宣王,覲於宣王而奉享禮,貢國所出之寶,善其尊宣王以常職來也。書曰「黑水西河,其貢璆琳琅玕」。此覲乃受命,先言受命者,顯其美也。}\textbf{王錫韓侯,淑旂綏章,簟茀錯衡,玄袞赤舄,鉤膺鏤鍚,鞹鞃淺幭,鞗革金厄。}{\footnotesize 淑,善也。交龍為旂。綏,大綏也。錯衡,文衡也。鏤鍚,有金鏤其鍚也。鞹,革也。鞃,軾中也。淺,虎皮淺毛也。幭,覆式也。厄,烏蠋也。箋云王為韓侯以常職來朝享之故,故多錫以厚之。善旂,旂之善色者也。綏,所引以登車,有采章也。簟茀,漆簟以為車蔽,今之藩也。鉤膺,樊纓也。眉上曰錫,刻金飾之,今當盧也。鞗革,謂轡也,以金為小環,往往纏搤之。}

\begin{quoting}綏,通嘉,好也。鞹 \texttt{kuò}。淺幭 \texttt{mèi},覆於軾上之虎皮。厄,同軛。\end{quoting}

\textbf{韓侯出祖,出宿于屠。顯父餞之,清酒百壺。}{\footnotesize 屠,地名也。顯父,有顯德者也。箋云祖,將去而犯軷也,既覲而反國必祖者,尊其所往,去則如始行焉,祖於國外,畢乃出宿,示行不留於是也。顯父,周之卿士也。餞送之,故有酒。}\textbf{其殽維何,炰鼈鮮魚。其蔌維何,維筍及蒲。其贈維何,乘馬路車。}{\footnotesize 蔌,菜殽也。筍,竹也。蒲,蒲蒻也。箋云炰鼈,以火熟之也。鮮魚,中膾者也。筍,竹萌也。蒲,深蒲也。贈,送也。王既使顯父餞之,又使送以車馬,所以贈厚意也。人君之車曰路車,所駕之馬曰乘馬。}\textbf{籩豆有且,侯氏燕胥。}{\footnotesize 箋云且,多貌。胥,皆也。諸侯在京師未去者,於顯父餞之時皆來相與燕,其籩豆且然,榮其多也。}

\begin{quoting}屠,通杜,即戶縣之杜陵。炰 \texttt{páo}。鮮,通斯,析也,析魚,即膾魚。蔌 \texttt{sù}。\end{quoting}

\textbf{韓侯取妻,汾王之甥,蹶父之子。}{\footnotesize 汾,大也。蹶父,卿士也。箋云汾王,厲王也,厲王流于彘,彘在汾水之上,故時人因以號之,猶言莒郊公、黎比公也。姊妹之子為甥。王之甥,卿士之子,言尊貴也。}\textbf{韓侯迎止,于蹶之里。百兩彭彭,八鸞鏘鏘,不顯其光。}{\footnotesize 里,邑也。箋云于蹶之里,蹶父之里。百兩,百乘。不顯,顯也。光,猶榮也,氣有榮光也。}\textbf{諸娣從之,祁祁如雲。韓侯顧之,爛其盈門。}{\footnotesize 祁祁,徐靚也。如雲,言眾多也。諸侯一取九女,二國媵之,諸娣,眾妾也。顧之,曲顧道義也。箋云媵者必娣姪從之,獨言娣者,舉其貴者。爛爛,粲然鮮明且眾多之貌。}

\begin{quoting}蹶父 \texttt{guì fǔ}。諸,魯詩作姪。\end{quoting}

\textbf{蹶父孔武,靡國不到。為韓姞相攸,莫如韓樂。}{\footnotesize 姞,蹶父姓也。箋云相視、攸所也。蹶父甚武健,為王使於天下,國國皆至,為其女韓侯夫人姞氏視其所居,韓國最樂。}\textbf{孔樂韓土,川澤訏訏。魴鱮甫甫,麀鹿噳噳。有熊有羆,有貓有虎。}{\footnotesize 訏訏,大也。甫甫然,大也。噳噳然,眾也。貓,似虎淺毛者也。箋云甚樂矣韓之國土也,川澤寬大,眾魚禽獸備有,言饒富也。}\textbf{慶既令居,韓姞燕譽。}{\footnotesize 箋云慶,善也。蹶父既善韓之國土,使韓姞嫁焉而居之,韓姞則安之,盡其婦道,有顯譽。}

\begin{quoting}訏 \texttt{xū}。噳 \texttt{yǔ}。\end{quoting}

\textbf{溥彼韓城,燕師所完。}{\footnotesize 師,眾也。箋云溥大、燕安也。大矣彼韓國之城,乃古平安時眾民之所築完。}\textbf{以先祖受命,因時百蠻。王錫韓侯,其追其貊。奄受北國,因以其伯。}{\footnotesize 韓侯之先祖,武王之子也。因時百蠻,長是蠻服之百國也。追、貊,戎狄國也。奄,撫也。箋云韓侯先祖有功德者,受先王之命,封為韓侯,居韓城為侯伯,其州界外接蠻服,因見使時節百蠻貢獻之往來,後君微弱,用失其業,今王以韓侯先祖之事如是而韓侯賢,故於入覲使復其先祖之舊職,賜之蠻服追貊之戎狄令撫柔,其所受王畿北面之國,因以其先祖侯伯之事盡予之,皆美其為人子孫能興復先祖之功。其後追也貊也為玁狁所逼,稍稍東遷也。}\textbf{實墉實壑,實畝實藉。}{\footnotesize 實墉實壑,言高其城、深其壑也。箋云實,當作寔,趙魏之東,實寔同聲,寔,是也。藉,稅也。韓侯之先祖微弱,所伯之國多滅絕,今復舊職,興滅國、繼絕世,故築治是城,濬修是壑,井牧是田畝,收斂是賦稅,使如故常。}\textbf{獻其貔皮,赤豹黃羆。}{\footnotesize 貔,猛獸也。追貊之國來貢,而侯伯揔領之。}

\begin{quoting}燕,釋文「北燕國」。時,是也。貊 \texttt{mò}。\end{quoting}

\section{江漢}

%{\footnotesize 六章、章八句}

\textbf{江漢,尹吉甫美宣王也。能興衰撥亂,命召公平淮夷。}{\footnotesize 召公,召穆公也,名虎。}

\begin{quoting}\textbf{郭沫若}大雅江漢之篇與世存召伯虎簋銘之一所記乃同時事,簋銘云「對揚朕宗君其休,用作列祖召公嘗簋」,詩云「作召公考,天子萬壽」,文例正同。\textbf{崔述}豐鎬考信錄曰此詩前三章敘召公經略江漢之事,乃國家大政,後三章耑言召公受賜事。\end{quoting}

\textbf{江漢浮浮,武夫滔滔。匪安匪遊,淮夷來求。}{\footnotesize 浮浮,眾彊貌。滔滔,廣大貌。淮夷,東國,在淮浦而夷行也。箋云匪,非也。江漢之水合而東流浮浮然,宣王於是水上命將帥、遣士眾,使循流而下滔滔然,其順王命而行,非敢斯須自安也,非敢斯須遊止也,主為來求淮夷所處。據至其竟,故言來。}\textbf{既出我車,既設我旟。匪安匪舒,淮夷來鋪。}{\footnotesize 鋪,病也。箋云車,戎車也。鳥隼曰旟。兵至竟而期戰地,其日出戎車建旟,又不自安不舒行者,主為來伐討淮夷也。據至戰地,故又言來。}

\begin{quoting}浮浮,魯詩作陶陶,陶、滔古通用。王引之、陳奐以為當作「江漢滔滔,武夫浮浮」。來,語詞。求,通糾,誅伐也。方言「鋪,止也」。\end{quoting}

\textbf{江漢湯湯,武夫洸洸。經營四方,告成于王。}{\footnotesize 洸洸,武貌。箋云召公既受命伐淮夷,服之,復經營四方之叛國,從而伐之,克勝則使傳遽告功於王。}\textbf{四方既平,王國庶定。時靡有爭,王心載寧。}{\footnotesize 箋云庶幸、時是也。載之言則也。召公忠臣,順於王命,此述其志也。}

\begin{quoting}湯 \texttt{shāng}。洸 \texttt{guāng}。\end{quoting}

\textbf{江漢之滸,王命召虎。式辟四方,徹我疆土。匪疚匪棘,王國來極。}{\footnotesize 召虎,召穆公也。箋云滸,水厓也。式法、疚病、棘急、極中也。王於江漢之水上命召公,使以王法征伐開辟四方,治我疆界於天下,非可以兵病害之也,非可以兵急操切之也,使來於王國受政教之中正而已。齊桓公經陳鄭之間及伐北戎,則違此言者。}\textbf{于疆于理,至于南海。}{\footnotesize 箋云于,往也。于,於也。召公於有叛戾之國,則往正其竟界,修其分理,周行四方,至於南海而功大成事終也。}

\begin{quoting}國語韋昭注「南海,群蠻也」,即今江蘇東部近海之地也。\end{quoting}

\textbf{王命召虎,來旬來宣。文武受命,召公維翰。}{\footnotesize 旬,徧也。召公,召康公也。箋云來,勤也。旬,當作營。宣,徧也。召康公名奭,召虎之始祖也。王命召虎,女勤勞於經營四方,勤勞於徧疆理眾國,昔文王武王受命,召康公為之楨榦之臣,以正天下。為虎之勤勞,故述其祖之功以勸之。}\textbf{無曰予小子,召公是似。肇敏戎公,用錫爾祉。}{\footnotesize 似嗣、肇謀、敏疾、戎大、公事也。箋云戎,猶女也。女無自減損曰我小子耳,女之所為乃嗣女先祖召康公之功,今謀女之事,乃有敏德,我用是故,將賜女福慶也。王為虎之志大謙,故進之云爾。}

\begin{quoting}旬,同巡。肇,始也。敏,同謀。\textbf{于省吾}始謀大事,用錫爾福祉也。\end{quoting}

\textbf{釐爾圭瓚,秬鬯一卣,告于文人。}{\footnotesize 釐,賜也。秬,黑黍也。鬯,香草也,築煮合而鬱之曰鬯。卣,器也。九命錫圭瓚秬鬯。文人,文德之人也。箋云秬鬯,黑黍酒也,謂之鬯者,芬香條鬯也。王賜召虎以鬯酒一尊,使以祭其宗廟,告其先祖諸有德美見記者。}\textbf{錫山土田,于周受命,自召祖命。}{\footnotesize 諸侯有大功德,賜之名山土田附庸。箋云周,岐周也。自,用也。宣王欲尊顯召虎,故如岐周,使虎受山川土田之賜,命用其祖召康公受封之禮。岐周,周之所起,為其先祖之靈,故就之。}\textbf{虎拜稽首,天子萬年。}{\footnotesize 箋云拜稽首者,受王命策書也,臣受恩,無可以報謝者,稱言使君壽考而已。}

\begin{quoting}釐,同賚。\end{quoting}

\textbf{虎拜稽首,對揚王休。作召公考,天子萬壽。明明天子,令聞不已。矢其文德,洽此四國。}{\footnotesize 對遂、考成、矢施也。箋云對答、休美、作為也。虎既拜而答王策命之時,稱揚王之德美,君臣之言宜相成也,王命召虎用召祖命,故虎對王,亦為召康公受王命之時對成王命之辭,謂如其所言也,如其所言者,「天子萬壽」以下是也。}

\begin{quoting}對,報答。\textbf{郭沫若}考乃簋之假借字。明明,猶勉勉。矢,魯詩作弛,寬緩也。洽,禮記孔子閒居引作協。\end{quoting}

\section{常武}

%{\footnotesize 六章、章八句}

\textbf{常武,召穆公美宣王也。有常德以立武事,因以為戒然。}{\footnotesize 戒者,「王舒保作,匪紹匪遊,徐方繹騷」。}

\begin{quoting}\textbf{王質}詩總聞曰自南仲以來,累世著武,故曰常武。\end{quoting}

\textbf{赫赫明明,王命卿士,南仲大祖,大師皇父,整我六師,以修我戎。}{\footnotesize 赫赫然盛也,明明然察也。王命南仲於大祖,皇甫為大師。箋云南仲,文王時武臣也。顯著乎,昭察乎,宣王之命卿士為大將也,乃用其以南仲為大祖者,今大師皇父是也,使之整齊六軍之眾,治其兵甲之事。命將必本其祖者,因有世功,於是尤顯。大師者,公兼官也。}\textbf{既敬既戒,惠此南國。}{\footnotesize 箋云敬之言警也。警戒六軍之眾,以惠淮浦之旁國,謂敕以無暴掠為之害也。每軍各有將,中軍之將尊也。}

\begin{quoting}大祖,謂后稷廟也。\textbf{馬瑞辰}據竹書紀年「幽王元年,王錫大師尹氏皇父命」,則皇父實為尹氏,即二章所云「王謂尹氏」也。敬,同儆。\end{quoting}

\textbf{王謂尹氏,命程伯休父,左右陳行,戒我師旅,率彼淮浦,省此徐土。}{\footnotesize 尹氏掌命卿士,程伯休父始命為大司馬。浦,厓也。箋云尹氏,天子世大夫也。率,循也。王使大夫尹氏策命程伯休父於軍將行治兵之時,使其士眾左右陳列而敕戒之,使循彼淮浦之旁,省視徐國之土地叛逆者。軍禮,司馬掌其誓戒。}\textbf{不畱不處,三事就緒。}{\footnotesize 誅其君,弔其民,為之立三有事之臣。箋云緒,業也。王又使軍將豫告淮浦徐土之民云,不久處於是也,女三農之事皆就其業。為其驚怖,先以言安之。}

\begin{quoting}國語「重黎氏世敘天地,其在周,程伯休父其後也,當宣王時,失其官守,而為司馬氏」。玉海「徐,贏姓,伯益佐禹有功,封其子若木於徐」,在今安徽泗縣北。三事,三卿,即十月之交之「擇三有事」、雨無正之「三事大夫」。\end{quoting}

\textbf{赫赫業業,有嚴天子。王舒保作,匪紹匪遊,徐方繹騷。}{\footnotesize 赫赫然盛也,業業然動也。嚴然而威。舒,徐也。保,安也。匪紹匪遊,不敢繼以敖遊也。繹陳、騷動也。箋云作,行也。紹,緩也。繹,當作驛。王之軍行,其貌赫赫業業然,有尊嚴於天子之威,謂聞見者莫不憚之。王舒安,謂軍行三十里,亦非解緩也,亦非敖游也,徐國傳遽之驛見之,知王兵必克,馳走以相恐動。}\textbf{震驚徐方,如雷如霆,徐方震驚。}{\footnotesize 箋云震,動也。驛馳走相恐懼,以驚動徐國,如雷霆之恐怖人然,徐國則驚動而將服罪。}

\begin{quoting}王舒保作,\textbf{朱熹}言王舒徐而安行也。\end{quoting}

\textbf{王奮厥武,如震如怒。進厥虎臣,闞如虓虎。鋪敦淮濆,仍執醜虜。}{\footnotesize 虎之自怒虓然。濆厓、仍就、虜服也。箋云進,前也。敦,當作屯。醜,眾也。王奮揚其威武,而震雷其聲,而勃怒其色,前其虎臣之將闞然如虎之怒,陳屯其兵於淮水大防之上以臨敵,就執其眾之降服者。}\textbf{截彼淮浦,王師之所。}{\footnotesize 截,治也。箋云治淮之旁國有罪者,就王師而斷之。}

\begin{quoting}\textbf{陳奐}虎臣,即虎賁氏,啟行之元戎也。闞 \texttt{hǎn},怒貌。虓 \texttt{xiāo},說文「虎鳴也」。鋪,韓詩作敷,布陣。敦,同頓。濆 \texttt{fén}。仍,頻也。\end{quoting}

\textbf{王旅嘽嘽,如飛如翰,如江如漢,如山之苞,如川之流。}{\footnotesize 嘽嘽然盛也。疾如飛,摯如翰。苞,本也。箋云嘽嘽,閒暇有餘力之貌。其行疾,自發舉如鳥之飛也,翰,其中豪俊也。江漢以喻盛大也,山本以喻不可驚動也,川流以喻不可禦也。}\textbf{緜緜翼翼,不測不克,濯征徐國。}{\footnotesize 緜緜,靚也。翼翼,敬也。濯,大也。箋云王兵安靚且皆敬,其勢不可測度,不可攻勝,既服淮浦矣,今又以大征徐國,言必勝也。}

\textbf{王猶允塞,徐方既來。}{\footnotesize 猶,謀也。箋云猶尚、允信也。王重兵,兵雖臨之,尚守信自實滿,兵未陳而徐國已來告服,所謂善戰者不陳。}\textbf{徐方既同,天子之功。四方既平,徐方來庭。}{\footnotesize 來王庭也。}\textbf{徐方不回,王曰還歸。}{\footnotesize 箋云回,猶違也。還歸,振旅也。}

\begin{quoting}來,齊詩作倈,歸服。\textbf{王先謙}言王道誠信充實,遠人自服。\end{quoting}

\section{瞻卬}

%{\footnotesize 七章、三章章十句、四章章八句}

\textbf{瞻卬,凡伯刺幽王大壞也。}{\footnotesize 凡伯,天子大夫也,春秋魯隱公七年「冬,天王使凡伯來聘」。}

\begin{quoting}\textbf{釋文}此及召旻二篇,幽王之變大雅也。\end{quoting}

\textbf{瞻卬昊天,則不我惠。孔填不寧,降此大厲。}{\footnotesize 昊天,斥王也。填久、厲惡也。箋云惠,愛也。仰視幽王為政,則不愛我下民,甚久矣天下不安,王乃下此大惡以敗亂之。}\textbf{邦靡有定,士民其瘵。蟊賊蟊疾,靡有夷屆。罪罟不收,靡有夷瘳。}{\footnotesize 瘵病、夷常也。罪罟,設罪以為罟。瘳,愈也。箋云屆,極也。天下騷擾,邦國無有安定者,士卒與民皆勞病,其為殘酷痛病於民,如蟊賊之害禾稼然,為之無常,亦無止息時,施刑罪以羅罔天下而不收斂,為之亦無常,無止息時。此目王所下大惡。}

\begin{quoting}填,塵古字,久也。瘵 \texttt{zhài}。夷,語詞。\end{quoting}

\textbf{人有土田,女反有之。人有民人,女覆奪之。}{\footnotesize 箋云此言王削黜諸侯及卿大夫無罪者。覆,猶反也。}\textbf{此宜無罪,女反收之。彼宜有罪,女覆說之。}{\footnotesize 收,拘收也。說,赦也。}

\begin{quoting}廣雅「有,取也」。說,同脫。\end{quoting}

\textbf{哲夫成城,哲婦傾城。}{\footnotesize 哲,知也。箋云哲謂多謀慮也。城,猶國也。丈夫,陽也,陽動故多謀慮則成國,婦人,陰也,陰靜故多謀慮乃亂國。}\textbf{懿厥哲婦,為梟為鴟。}{\footnotesize 箋云懿,有所痛傷之聲也。厥,其也,其,幽王也。梟鴟,惡聲之鳥,喻褒姒之言無善。}\textbf{婦有長舌,維厲之階。亂匪降自天,生自婦人。匪敎匪誨,時維婦寺。}{\footnotesize 寺,近也。箋云長舌喻多言語。是王降大厲之階,階,所由上下也,今王之有此亂政,非從天而下,但從婦人出耳,又非有人教王為亂、語王為惡者,是惟近愛婦人,用其言故也。}

\begin{quoting}懿,同噫。寺,同侍。\end{quoting}

\textbf{鞫人忮忒,譖始竟背。豈曰不極,伊胡為慝。}{\footnotesize 忮害、忒變也。箋云鞫,窮也。譖,不信也。竟,猶終也。胡何、慝惡也。婦人之長舌者多謀慮,好窮屈人之語,忮害轉化,其言無常,始於不信,終於背違,人豈謂其是不得中乎,反云維我言何用為惡不信也。}\textbf{如賈三倍,君子是識。婦無公事,休其蠶織。}{\footnotesize 休,息也。婦人無與外政,雖王后猶以蠶織為事。古者天子為藉千畝,冕而朱紘,躬秉耒,諸侯為藉百畝,冕而青紘,躬秉耒,以事天地山川社稷先古,敬之至也。天子諸侯必有公桑蠶室,近川而為之,築宮仞有三尺,棘牆而外閉之,及大昕之朝,君皮弁素積,卜三宮之夫人、世婦之吉者,使入蠶于蠶室,奉種浴于川,桑于公桑,風戾以食之,歲既單矣,世婦卒蠶,奉繭以示于君,遂獻繭于夫人,夫人曰此所以為君服與,遂副褘而受之,少牢以禮之,及良日,后夫人繅,三盆手,遂布于三宮夫人世婦之吉者,使繅,遂朱綠之、玄黃之,以為黼黻文章,服既成矣,君服之以祀先王先公,敬之至也。箋云識,知也。賈物而有三倍之利者,小人所宜知也,君子反知之,非其宜也,今婦人休其蠶桑織絍之職而與朝廷之事,其為非宜亦猶是也。孔子曰「君子喻於義,小人喻於利」。}

\begin{quoting}忮,同歧。譖,同僭,虛妄。\textbf{林義光}鞫讀為吿,吿、鞫古同音,吿人歧忒者,吿人之言兩歧而差忒也,僭始竟背者,虛妄於始而背之於終也,蓋凡事為婦人所主持,則王之所以吿人者其後或因哲婦之阻撓而終背其初約,由是與所吿之言兩歧差忒,而始言成為虛妄矣。慝,韓詩作嬺,悦愛、歡喜。\textbf{林義光}識讀為職,識與職古通用。\end{quoting}

\textbf{天何以刺,何神不富。舍爾介狄,維予胥忌。}{\footnotesize 刺責、富福、狄遠、忌怨也。箋云介,甲也。王之為政既無過惡,天何以責王見變異乎,神何以不福王而有災害也,王不念此而改修德,乃舍女被甲夷狄來侵犯中國者,反與我相怨,謂其疾怨群臣叛違也。}\textbf{不弔不祥,威儀不類。人之云亡,邦國殄瘁。}{\footnotesize 類善、殄盡、瘁病也。箋云弔,至也。王之為政,德不至於天矣,不能致徵祥於神矣,威儀又不善於朝廷矣,賢人皆言奔亡,則天下邦國將盡困病。}

\textbf{天之降罔,維其優矣。人之云亡,心之憂矣。}{\footnotesize 優,渥也。箋云優,寬也。天下羅罔以取有罪亦甚寬,謂但以災異譴告之,不指加罰於其身,疾王為惡之甚,賢者奔亡,則人心無不憂。}\textbf{天之降罔,維其幾矣。人之云亡,心之悲矣。}{\footnotesize 幾,危也。箋云幾,近也。言災異譴告離人身近,愚者不能覺。}

\textbf{觱沸檻泉,維其深矣。心之憂矣,寧自今矣。不自我先,不自我後。}{\footnotesize 箋云檻泉正出,涌出也,觱沸,其貌。涌泉之源,所由者深,喻己憂所從來久也,惡政不先己、不後己,怪何故正當之。}\textbf{藐藐昊天,無不克鞏。}{\footnotesize 藐藐,大貌。鞏,固也。箋云藐藐,美也。王者有美德藐藐然,無不能自堅固於其位者,微箴之也。}\textbf{無忝皇祖,式救爾後。}{\footnotesize 箋云式,用也。後,謂子孫也。}

\begin{quoting}檻,同濫。鞏,同恐,懼也。\end{quoting}

\section{召旻}

%{\footnotesize 七章、四章章五句、三章章七句}

\textbf{召旻,凡伯刺幽王大壞也。旻,閔也,閔天下無如召公之臣也。}{\footnotesize 閔,病也。}

\begin{quoting}\textbf{蘇轍}首章稱旻天,卒章稱召公,故謂之召旻,以別小旻而已。\end{quoting}

\textbf{旻天疾威,天篤降喪。瘨我饑饉,民卒流亡。}{\footnotesize 箋云天,斥王也。疾,猶急也。瘨,病也。病乎幽王之為政也,急行暴虐之法,厚下喪亂之教,謂重賦稅也,病國中以饑饉,令民盡流移。}\textbf{我居圉卒荒。}{\footnotesize 圉,垂也。箋云荒,虛也。國中至邊竟以此故盡空虛。}

\begin{quoting}瘨 \texttt{diān}。\end{quoting}

\textbf{天降罪罟,蟊賊內訌。}{\footnotesize 訌,潰也。箋云訌,爭訟相陷入之言也。王施刑罪以羅罔天下眾為殘酷之人,雖外以害人,又自內爭相讒惡。}\textbf{昬椓靡共,潰潰回遹。實靖夷我邦。}{\footnotesize 椓,夭椓也。潰潰,亂也。靖謀、夷平也。箋云昬、椓皆奄人也,昬,其官名也,椓,椓毀陰者也。王遠賢者而近任刑奄之人,無肯共其職事者,皆潰潰然維邪是行,皆謀夷滅王之國。}

\begin{quoting}共,通供,供職。回遹 \texttt{yù},邪僻。\end{quoting}

\textbf{臯臯訿訿,曾不知其玷。}{\footnotesize 臯臯,頑不知道也。訿訿,窳不供事也。箋云玷,缺也。王政已大壞,小人在位,曾不知大道之缺。}\textbf{兢兢業業,孔填不寧,我位孔貶。}{\footnotesize 貶,隊也。箋云兢兢,戒也,業業,危也。天下之人戒懼危怖,甚久矣其不安也,我王之位又甚隊矣。言見侵侮,政教不行,後犬戎伐之,而周與諸侯無異也。}

\begin{quoting}\textbf{馬瑞辰}臯臯訿訿 \texttt{zǐ},皆極言小人讒毀人之狀。\end{quoting}

\textbf{如彼歲旱,草不潰茂,如彼棲苴。}{\footnotesize 潰,遂也。苴,水中浮草也。箋云潰茂之潰當作彙,彙,茂貌。王無恩惠於天下,天下之人如旱歲之草,皆枯槁無潤澤,如樹上之棲苴。}\textbf{我相此邦,無不潰止。}{\footnotesize 箋云潰,亂也。無不亂者,言皆亂也,春秋傳曰「國亂曰潰,邑亂曰叛」。}

\begin{quoting}\textbf{馬瑞辰}釋文謂棲息,蓋謂枯草偃臥有似棲息也。楚辭九章王逸注「生曰草,枯曰苴 \texttt{chá}」。止,語詞。\end{quoting}

\textbf{維昔之富,不如時。}{\footnotesize 往者富仁賢,今也富讒佞。箋云富,福也。時,今時也。}\textbf{維今之疚,不如茲。}{\footnotesize 今則病賢也。箋云茲,此也,此者,此古昔明王。}\textbf{彼疏斯粺,胡不自替,職兄斯引。}{\footnotesize 彼宜食疏,今反食精稗。替廢、兄茲也。引,長也。箋云疏,麤也,謂糲米也。職,主也。彼賢者祿薄食麤,而此昬椓之黨反食精稗,女小人耳,何不自廢退,使賢者得進,乃茲復主長此為亂之事乎,責之也。米之率,糲十,稗九,鑿八,侍御七。}

\begin{quoting}兄,同況。\end{quoting}

\textbf{池之竭矣,不云自頻。}{\footnotesize 頻,厓也。箋云頻,當作濱,厓,猶外也。自,由也。池水之益由外灌焉,今池竭,人不言由外無益者與,言由之也,喻王猶池也,政之亂由外無賢臣益之。}\textbf{泉之竭矣,不云自中。}{\footnotesize 泉水從中以益者也。箋云泉者,中水生則益深,水不生則竭,喻王猶泉也,政之亂又由內無賢妃益之。}\textbf{溥斯害矣,職兄斯弘,不烖我躬。}{\footnotesize 箋云溥,猶徧也。今時徧有此內外之害矣,乃茲復主大此為亂之事,是不烖王之身乎,責王也。烖,謂見誅伐。}

\textbf{昔先王受命,有如召公,日辟國百里,今也日蹙國百里。}{\footnotesize 辟開、蹙促也。箋云先王受命,謂文王武王時也。召公,召康公也,言有如者,時賢臣多,非獨召公也。今,今幽王臣。}\textbf{於乎哀哉,維今之人,不尚有舊。}{\footnotesize 箋云哀哉,哀其不高尚賢者,尊任有舊德之臣,將以喪亡其國。}

%\begin{flushright}蕩之什十一篇、九十二章、七百六十九句\end{flushright}