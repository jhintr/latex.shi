\chapter{鴻鴈之什詁訓傳第十八}

\section{鴻鴈}

%{\footnotesize 三章、章六句}

\textbf{鴻鴈,美宣王也。萬民離散,不安其居,而能勞來還定安集之,至于矜寡,無不得其所焉。}{\footnotesize 宣王承厲王衰亂之敝而起興復先王之道,以安集眾民為始也,書曰「天將有立父母,民之有政有居」,宣王之為是務。}

\begin{quoting}\textbf{朱熹}流民以鴻鴈哀鳴自比而作此歌也。\end{quoting}

\textbf{鴻鴈于飛,肅肅其羽。}{\footnotesize 興也。大曰鴻,小曰鴈。肅肅,羽聲也。箋云鴻鴈知辟陰陽寒暑,興者,喻民知去無道、就有道。}\textbf{之子于征,劬勞于野。}{\footnotesize 之子,侯伯卿士也。劬勞,病苦也。箋云侯伯卿士,謂諸侯之伯與天子卿士也,是時民既離散,邦國有壞滅者,侯伯久不述職,王使廢於存省,諸侯於是始復之,故美焉。}\textbf{爰及矜人,哀此鰥寡。}{\footnotesize 矜,憐也。老無妻曰鰥,偏喪曰寡。箋云爰,曰也。王之意,不徒使此為諸侯之事與安集萬民而已,王曰「當及此可憐之人」,謂貧窮者欲令賙餼之,鰥寡則哀之,其孤獨者收斂之,使有所依附。}

\begin{quoting}說文「雁,雁鳥也,鴈,鵝也」。爾雅釋言「矜,苦也」。之子,\textbf{朱熹}流民自相謂也。\end{quoting}

\textbf{鴻鴈于飛,集于中澤。}{\footnotesize 中澤,澤中也。箋云鴻鴈之性安居澤中,今飛又集于澤中,猶民去其居而離散,今見還定安集。}\textbf{之子于垣,百堵皆作。}{\footnotesize 一丈為板,五板為堵。箋云侯伯卿士又於壞滅之國徵民起屋舍,築牆壁,百堵同時而起,言趨事也。春秋傳曰「五板為堵,五堵為雉」,雉長三丈則板六尺。}\textbf{雖則劬勞,其究安宅。}{\footnotesize 究,窮也。箋云此勸萬民之辭,女今雖病勞,終有安居。}

\textbf{鴻鴈于飛,哀鳴嗸嗸。}{\footnotesize 未得所安集則嗸嗸然。箋云此之子所未至者。}\textbf{維此哲人,謂我劬勞。}{\footnotesize 箋云此哲人,謂知王之意及之子之事者。我,之子自我也。}\textbf{維彼愚人,謂我宣驕。}{\footnotesize 宣,示也。箋云謂我役作眾民為驕奢。}

\begin{quoting}\textbf{經義述聞}宣驕與劬勞相對成文,劬亦勞也,宣亦驕也。\end{quoting}

\section{庭燎}

%{\footnotesize 三章、章五句}

\textbf{庭燎,美宣王也。因以箴之。}{\footnotesize 諸侯將朝,宣王以夜未央之時問夜早晚,美者,美其能自勤以政事。因以箴者,王有雞人之官,凡國事為期則告之以時,王不正其官而問夜早晚。}

\textbf{夜如何其,}{\footnotesize 箋云此宣王以諸侯將朝,夜起曰「夜如何其」,問早晚之辭。}\textbf{夜未央,庭燎之光。君子至止,鸞聲將將。}{\footnotesize 央,旦也。庭燎,大燭。君子,謂諸侯也。將將,鸞鑣聲也。箋云夜未央,猶言夜未渠央也,而於庭設大燭,使諸侯早來朝,聞鸞聲將將然。}

\begin{quoting}其 \texttt{jī},語尾助詞,亦作己。鸞,亦作鑾,鈴也,爾雅釋天「有鈴曰旂」,\textbf{林義光}金文皆言錫鸞旂,此詩三章云「言觀其旂」,而采菽、泮水亦皆以鸞聲與旂並言,則鸞為旂上之鸞,非車上之鸞也。\end{quoting}

\textbf{夜如何其,夜未艾,庭燎晣晣。君子至止,鸞聲噦噦。}{\footnotesize 艾,久也。晣晣,明也。噦噦,徐行有節也。箋云芟末曰艾,以言夜先雞鳴時。}

\begin{quoting}\textbf{廣雅疏證}襄九年左傳「大勞未艾」,杜注云「艾,息也」,哀二年傳「憂未艾也」,宣十二年傳「憂未歇也」,歇、息、艾者皆已也。\end{quoting}

\textbf{夜如何其,夜鄉晨,庭燎有煇。君子至止,言觀其旂。}{\footnotesize 煇,光也。箋云晨,明也。上二章聞鸞聲爾,今夜鄉明,我見其旂,是朝之時也。朝禮,別色始入。}

\begin{quoting}煇,說文「光也」,段注「朝旦為煇,日中為光」,禮記玉藻「揖私朝,煇如也,登車則有光」。\end{quoting}

\section{沔水}

%{\footnotesize 三章、二章章八句、一章六句}

\textbf{沔水,規宣王也。}{\footnotesize 規者,正圓之器也,規主仁恩也,以恩親正君曰規,春秋傳曰「近臣盡規」。}

\textbf{沔彼流水,朝宗于海。}{\footnotesize 興也。沔,水流滿也,水猶有所朝宗。箋云興者,水流而入海,小就大也,喻諸侯朝天子亦猶是也。諸侯春見天子曰朝,夏見曰宗。}\textbf{鴥彼飛隼,載飛載止。}{\footnotesize 箋云載之言則也,言隼欲飛則飛,欲止則止,喻諸侯之自驕恣,欲朝不朝,自由無所在心也。}\textbf{嗟我兄弟,邦人諸友。莫肯念亂,誰無父母。}{\footnotesize 邦人諸友,謂諸侯也。兄弟,同姓臣也。京師者,諸侯之父母也。箋云我,我王也。莫,無也。我同姓異姓之諸侯,女自恣聽不朝,無肯念此於禮法為亂者,女誰無父母乎,言皆生於父母也。臣之道,資於事父以事君。}

\begin{quoting}\textbf{馬瑞辰}念與尼雙聲,尼,止也,故念亦有止義。潛夫論釋難篇曰且一國盡亂,無有安身,詩云「莫肯念亂,誰無父母」,言將皆為害,然有親者,憂將深也。\end{quoting}

\textbf{沔彼流水,其流湯湯。}{\footnotesize 言放縱無所入也。箋云湯湯,波流盛貌,喻諸侯奢僭,既不朝天子,復不事侯伯。}\textbf{鴥彼飛隼,載飛載揚。}{\footnotesize 言無所定止也。箋云則飛則揚,喻諸侯出兵妄相侵伐。}\textbf{念彼不蹟,載起載行。心之憂矣,不可弭忘。}{\footnotesize 不蹟,不循道也。弭,止也。箋云彼,諸侯也,諸侯不循法度,妄興師出兵,我念之憂,不能忘也。}

\begin{quoting}忘,同亡,忘亦弭也。\end{quoting}

\textbf{鴥彼飛隼,率彼中陵。}{\footnotesize 箋云率,循也。隼之性待鳥雀而食,飛循陵阜者是其常也,喻諸侯之守職順法度者亦是其常也。}\textbf{民之訛言,寧莫之懲。}{\footnotesize 懲,止也。箋云訛,偽也。言時不令小人好詐偽,為交易之言,使見怨咎,安然無禁止。}\textbf{我友敬矣,讒言其興。}{\footnotesize 疾王不能察讒也。箋云我,我天子也。友,謂諸侯也。言諸侯有敬其職、順法度者,讒人猶興其言以毀惡之,王與侯伯不當察之。}

\begin{quoting}\textbf{朱熹}卒章脫前兩句耳。敬,同警。\end{quoting}

\section{鶴鳴}

%{\footnotesize 二章、章九句}

\textbf{鶴鳴,誨宣王也。}{\footnotesize 誨,教也,教宣王求賢人之未仕者。}

\begin{quoting}荀子儒效篇曰君子隱而顯,微而明,辭讓而勝,詩云「鶴鳴于九臯,聲聞于天」,此之謂也。\textbf{陳奐}詩全篇皆興也,鶴魚檀石,皆以喻賢人。\end{quoting}

\textbf{鶴鳴于九臯,聲聞于野。}{\footnotesize 興也。臯,澤也。言身隱而名著也。箋云臯澤中水溢出所為坎,自外數至九,喻深遠也,鶴在中鳴焉而野聞其鳴聲,興者,喻賢者雖隱居,人咸知之。}\textbf{魚潛在淵,或在于渚。}{\footnotesize 良魚在淵,小魚在渚。箋云此言魚之性寒則逃於淵,溫則見於渚,喻賢者世亂則隱,治平則出,在時君也。}\textbf{樂彼之園,爰有樹檀,其下維蘀。}{\footnotesize 何樂於彼園之觀乎?蘀,落也。尚有樹檀而下其蘀。箋云之往、爰曰也。言所以之彼園而觀者,人曰有樹檀,檀下有蘀。此猶朝廷之尚賢者而下小人,是以往也。}\textbf{它山之石,可以為錯。}{\footnotesize 錯,石也,可以琢玉。舉賢用滯,則可以治國。箋云它山,喻異國。}

\begin{quoting}九臯,釋文「九折之澤」。古書引詩多無「于」字。\textbf{馬瑞辰}下章榖為木名,則此章蘀亦木名,不得泛指落木,王尚書經義述聞「蘀,疑當讀為檡」。\end{quoting}

\textbf{鶴鳴于九臯,聲聞于天。}{\footnotesize 箋云天,高遠也。}\textbf{魚在于渚,或潛在淵。}{\footnotesize 箋云時寒則魚去渚,逃於淵。}\textbf{樂彼之園,爰有樹檀,其下維榖。}{\footnotesize 榖,惡木也。}\textbf{它山之石,可以攻玉。}{\footnotesize 攻,錯也。}

\section{祈父}

%{\footnotesize 三章、章四句}

\textbf{祈父,刺宣王也。}{\footnotesize 刺其用祈父不得其人也,官非其人則職廢。祈父之職掌六軍之事,有九伐之法。祈、圻、畿同。}

\textbf{祈父!}{\footnotesize 祈父,司馬也,職掌封祈之兵甲。箋云此司馬也,時人以其職號之,故曰祈父,書曰「若疇祈父」,謂司馬。司馬掌祿士,故司士屬焉,又有司右,主勇力之士。}\textbf{予王之爪牙,胡轉予于恤,靡所止居。}{\footnotesize 恤,憂也。宣王之末,司馬職廢,姜戎為敗。箋云予我、轉移也。此勇力之士責司馬之辭也,我乃王之爪牙,爪牙之士當為王閑守之衛,女何移我於憂,使我無所止居乎,謂見使從軍,與姜戎戰於千畝而敗之時也。六軍之士出自六鄉,法不取於王之爪牙之士。}

\begin{quoting}圻,邊境。恤,憂處也。\end{quoting}

\textbf{祈父!予王之爪士,}{\footnotesize 士,事也。}\textbf{胡轉予于恤,靡所厎止。}{\footnotesize 厎,至也。}

\begin{quoting}\textbf{馬瑞辰}按爪士猶言虎士,周官「虎賁氏屬有虎士八百人」即此,說苑雜事篇「虎豹愛爪」,故虎士亦云爪士,虎賁為宿衛之臣,故以移於戰爭為怨耳。\end{quoting}

\textbf{祈父!亶不聦,}{\footnotesize 亶,誠也。}\textbf{胡轉予于恤,有母之尸饔。}{\footnotesize 尸,陳也。孰食曰饔。箋云己從軍而母為父陳饌飲食之具,自傷不得供養也。}

\begin{quoting}\textbf{林義光}不聦,謂不聞人民疾苦。\textbf{陳奐}有母之尸饔,有母二字當逗讀,之猶則也,言我從軍以出,有母不得終養,歸則惟陳饔以祭,是可憂也。孔疏引許慎五經異義「有母之尸饔,謂陳饔以祭,志養不及親」。韓詩外傳「往而不可還者親也,至而不可加者年也,是故孝子欲養而親不待也,木欲直而時不待也,是故椎牛而祭墓,不如雞豚逮親存也」。\end{quoting}

\section{白駒}

%{\footnotesize 四章、章六句}

\textbf{白駒,大夫刺宣王也。}{\footnotesize 刺其不能留賢也。}

\begin{quoting}蔡邕琴操「白駒者,失朋友之所作也」。曹植釋思賦「彼朋友之離別,猶求思乎白駒」。\end{quoting}

\textbf{皎皎白駒,食我場苗。縶之維之,以永今朝。}{\footnotesize 宣王之末,不能用賢,賢者有乘白駒而去者。縶絆、維繫也。箋云永,久也。願此去者,乘其白駒而來,使食我場中之苗,我則絆之繫之,以久今朝,愛之,欲留之。}\textbf{所謂伊人,於焉逍遙。}{\footnotesize 箋云伊,當作繄,繄,猶是也。所謂是乘白駒而去之賢人,今於何遊息乎,思之甚也。}

\begin{quoting}縶、馽古同字,說文「馽,絆馬足也」。以永今朝,以盡今朝,留客之辭。\end{quoting}

\textbf{皎皎白駒,食我場藿。縶之維之,以永今夕。}{\footnotesize 藿,猶苗也。夕,猶朝也。}\textbf{所謂伊人,於焉嘉客。}

\textbf{皎皎白駒,賁然來思。}{\footnotesize 賁,飾也。箋云願其來而得見之。易卦曰「山下有火,賁」,賁,黃白色也。}\textbf{爾公爾侯,逸豫無期。}{\footnotesize 爾公爾侯邪,何為逸樂無期以反也。}\textbf{慎爾優游,勉爾遁思。}{\footnotesize 慎,誠也。箋云誠女優遊,使待時也,勉女遁思,度己終不得見,自訣之辭。}

\begin{quoting}\textbf{馬瑞辰}釋文「賁,徐音奔」,奔、賁古通用,詩「鶉之奔奔」,表記、呂氏春秋引詩俱作「賁賁」是也,考工記弓人鄭注「奔,猶疾也」,賁然,蓋狀馬來疾行之貌。思,語詞。\textbf{胡承珙}謂爾宜為公也,爾宜為侯也,何為逸樂無期以反也,如此,於愛賢留賢之意乃合。豫,說文「象之大者」,段注「引申之,凡大皆稱豫,故淮南子云『市不豫價』,大必寬豫,故事先而備謂之豫,寬大則樂,故釋詁曰『豫,樂也』」。勉,同免。說文「遁,遷也,一曰逃也」。\end{quoting}

\textbf{皎皎白駒,在彼空谷。}{\footnotesize 空,大也。}\textbf{生芻一束,其人如玉。}{\footnotesize 箋云此戒之也。女行所舍,主人之餼雖薄,要就賢人,其德如玉然。}\textbf{毋金玉爾音,而有遐心。}{\footnotesize 箋云毋愛女聲音而有遠我之心,以恩責之也。}

\begin{quoting}在彼空谷,文選班固西都賦、陸璣苦寒行注引韓詩作「在彼穹谷」。音,音訊。\end{quoting}

\section{黃鳥}

%{\footnotesize 三章、章七句}

\textbf{黃鳥,刺宣王也。}{\footnotesize 刺其以陰禮教親而不至,聯兄弟之不固。}

\begin{quoting}\textbf{朱熹}民適異國,不得其所,故作此詩。\end{quoting}

\textbf{黃鳥黃鳥,無集于榖,無啄我粟。}{\footnotesize 興也。黃鳥,宜集榖啄粟者。箋云興者,喻天下室家不以其道而相去,是失其性。}\textbf{此邦之人,不我肯穀。}{\footnotesize 穀,善也。箋云不肯以善道與我。}\textbf{言旋言歸,復我邦族。}{\footnotesize 宣王之末,天下室家離散,妃匹相去,有不以禮者。箋云言我、復反也。}

\begin{quoting}周禮秋官大司寇「四閭為族」,注「百家也」。\end{quoting}

\textbf{黃鳥黃鳥,無集于桑,無啄我粱。此邦之人,不可與明。}{\footnotesize 不可與明夫婦之道。箋云明,當為盟,盟,信也。}\textbf{言旋言歸,復我諸兄。}{\footnotesize 婦人有歸宗之義。箋云宗,謂宗子也。}

\textbf{黃鳥黃鳥,無集于栩,無啄我黍。此邦之人,不可與處。}{\footnotesize 處,居也。}\textbf{言旋言歸,復我諸父。}{\footnotesize 諸父,猶諸兄也。}

\section{我行其野}

%{\footnotesize 三章、章六句}

\textbf{我行其野,刺宣王也。}{\footnotesize 刺其不正嫁取之數而有荒政,多淫昏之俗。}

\begin{quoting}\textbf{朱熹}民適異國,依其婚姻而不見收卹,故作此詩。孔疏引王肅云「行遇惡木,言己適人遇惡人也」。\end{quoting}

\textbf{我行其野,蔽芾其樗。昬姻之故,言就爾居。}{\footnotesize 樗,惡木也。箋云樗之蔽芾始生,謂仲春之時,嫁取之月。婦之父、壻之父相謂昏姻。言,我也。我乃以此二父之命,故我就女居,我豈其無禮來乎,責之也。}\textbf{爾不我畜,復我邦家。}{\footnotesize 畜,養也。箋云宣王之末,男女失道,以求外昏,棄其舊姻而相怨。}

\textbf{我行其野,言采其蓫。昬姻之故,言就爾宿。}{\footnotesize 蓫,惡菜也。箋云蓫,牛蘈也,亦仲春時生,可采也。}\textbf{爾不我畜,言歸斯復。}{\footnotesize 復,反也。}

\begin{quoting}蓫 \texttt{zhú}。言、斯皆語詞。\end{quoting}

\textbf{我行其野,言采其葍。不思舊姻,求爾新特。}{\footnotesize 葍,惡菜也。新特,外昏也。箋云葍,䔰也,亦仲春時生,可采也。壻之父曰姻。我采䔰之時,以禮來嫁女,女不思女老父之命而棄我,而求女新外昏時來之女,責之也,不以禮嫁,必無肯媵之。}\textbf{成不以富,亦祇以異。}{\footnotesize 祇,適也。箋云女不以禮為室家,成事不足以得富也,女亦適以此自異於人道,言可惡也。}

\begin{quoting}葍 \texttt{fú}。\textbf{馬瑞辰}新特,謂新婦,特,當讀為「實維我特」之特,毛傳訓匹是也,新特,猶新昏也。成,同誠。亦祇以異,即喜新厭舊也。\end{quoting}

\section{斯干}

%{\footnotesize 九章、四章章七句、五章章五句}

\textbf{斯干,宣王考室也。}{\footnotesize 考,成也。德行國富,人民殷眾而皆佼好,骨肉和親,宣王於是築宮廟群寢,既成而釁之,歌斯干之詩以落之,此之謂成室。宗廟成,則又祭先祖。}

\begin{quoting}王筠毛詩重言曰凡其、彼、有、斯,皆重言也。\end{quoting}

\textbf{秩秩斯干,幽幽南山。}{\footnotesize 興也。秩秩,流行也。干,澗也。幽幽,深遠也。箋云興者,喻宣王之德如澗水之源秩秩流出,無極已也,國以饒富,民取足焉,如於深山。}\textbf{如竹苞矣,如松茂矣。}{\footnotesize 苞,本也。箋云言時民殷眾,如竹之本生矣,其佼好又如松柏之暢茂矣。}\textbf{兄及弟矣,式相好矣,無相猶矣。}{\footnotesize 猶,道也。箋云猶,當作瘉,瘉,病也,言時人骨肉用是相愛好,無相詬病也。}

\begin{quoting}秩,本義為積,段注「積之必有次序成文理,是曰秩」。前二句寫建築地勢,賦也。王雪山「如,非喻,乃枚舉焉爾」。苞,茂也。猶,同猷,方言「猷,詐也」,廣雅「猶,欺也」。\end{quoting}

\textbf{似續妣祖,}{\footnotesize 似,嗣也。箋云似,讀作巳午之巳,巳續妣祖者,謂巳成其宮廟也,妣,先妣姜嫄也,祖,先祖也。}\textbf{築室百堵,西南其戶。}{\footnotesize 西鄉戶、南鄉戶也。箋云此築室者,謂築燕寢也。百堵,百堵一時起也。天子之寢有左右房,西其戶者,異於一房者之室戶也,又云南其戶者,宗廟及路寢制如明堂,每室四戶,是室一南戶爾。}\textbf{爰居爰處,爰笑爰語。}{\footnotesize 箋云爰,於也,於是居,於是處,於是笑,於是語,言諸寢之中皆可安樂。}

\textbf{約之閣閣,椓之橐橐。}{\footnotesize 約,束也。閣閣,猶歷歷。橐橐,用力也。箋云約,謂縮板也。椓,謂㨨土也。}\textbf{風雨攸除,鳥鼠攸去,君子攸芋。}{\footnotesize 芋,大也。箋云芋,當作幠,幠,覆也。寢廟既成,其牆屋弘殺則風雨之所除也,其堅致則鳥鼠之所去也,其堂室相稱則君子之所覆蓋。}

\begin{quoting}閣閣、橐橐皆聲也。芋,魯詩作宇,說文「宇,屋邊也」,引申為庇覆。\end{quoting}

\textbf{如跂斯翼,}{\footnotesize 如人之跂竦翼爾。}\textbf{如矢斯棘,如鳥斯革,}{\footnotesize 棘,稜廉也。革,翼也。箋云棘,戟也,如人挾弓矢戟其肘,如鳥夏暑希革張其翼時。}\textbf{如翬斯飛,君子攸躋。}{\footnotesize 躋,升也。箋云伊洛而南,素質五色皆備成章曰翬。此章四如者,皆謂廉隅之正,形貌之顯也。翬者,鳥之奇異者也,故以成之焉。此章主於宗廟,君子所升,祭祀之時。}

\begin{quoting}跂,同企。翼,端正貌,論語「趨進,翼如也」,孔注「言端好」。\end{quoting}

\textbf{殖殖其庭,有覺其楹,}{\footnotesize 殖殖,言平正也。有覺,言高大也。箋云覺,直也。}\textbf{噲噲其正,噦噦其冥,}{\footnotesize 正,長也。冥,幼也。箋云噲噲,猶快快也。正,晝也。噦噦,猶煟煟也。冥,夜也。言居之晝日則快快然,夜則煟煟然,皆寬明之貌。}\textbf{君子攸寧。}{\footnotesize 箋云此章主於寢,君子所安,燕息之時。}

\textbf{下莞上簟,乃安斯寢。}{\footnotesize 箋云莞,小蒲之席也。竹葦曰簟。寢既成,乃鋪席與群臣安燕為歡以落之。}\textbf{乃寢乃興,乃占我夢。}{\footnotesize 言善之應人也。箋云興,夙興也,有善夢則占之。}\textbf{吉夢維何,維熊維羆,維虺維蛇。}{\footnotesize 箋云熊羆之獸,虺蛇之蟲,此四者,夢之吉祥也。}

\begin{quoting}說文「莞 \texttt{ɡuān},草也,可以作席」。\end{quoting}

\textbf{大人占之,維熊維羆,男子之祥,維虺維蛇,女子之祥。}{\footnotesize 箋云大人占之,謂以聖人占夢之法占之也,熊羆在山,陽之祥也,故為生男,虺蛇穴處,陰之祥也,故為生女也。}

\textbf{乃生男子,載寢之牀,載衣之裳,載弄之璋。}{\footnotesize 半珪曰璋。裳,下之飾也。璋,臣之職也。箋云男子生而臥於床,尊之也。裳,晝日衣也,衣以裳者,明當主於外事也,玩以璋者,欲其比德焉,正以璋者,明成之有漸。}\textbf{其泣喤喤,朱芾斯皇,室家君王。}{\footnotesize 箋云皇,猶煌煌也。芾者,天子純朱,諸侯黃朱。室家,一家之內。宣王所生之子,或且為天子,或且為諸侯,皆將佩朱芾煌煌然。}

\textbf{乃生女子,載寢之地,載衣之裼,載弄之瓦。}{\footnotesize 裼,褓也。瓦,紡塼也。箋云臥於地,卑之也。褓,夜衣也,明當主於內事。紡塼,習其一所有事也。}\textbf{無非無儀,唯酒食是議,無父母詒罹。}{\footnotesize 婦人質無威儀也。罹,憂也。箋云儀,善也。婦無所專於家事,有非非婦人也,有善亦非婦人也,婦人之事,惟議酒食爾,無遺父母之憂。}

\begin{quoting}說文「非,韋也」,無非,不違命。\textbf{馬瑞辰}儀又通作議,昭六年左傳「昔先王議事以制」,王引之曰「議讀為儀,儀,度也,制,斷也,謂度事之輕重以為制斷也」,今按婦人從人者也,不自度事以自專制,故曰無儀。\end{quoting}

\section{無羊}

%{\footnotesize 四章、章八句}

\textbf{無羊,宣王考牧也。}{\footnotesize 厲王之時,牧人之職廢,宣王始興而復之,至此而成,謂復先王牛羊之數。}

\textbf{誰謂爾無羊,三百維群。誰謂爾無牛,九十其犉。}{\footnotesize 黃牛黑唇曰犉。箋云爾,女也,女,宣王也。宣王復古之牧法,汲汲於其數,故歌此詩以解之也。誰謂女無羊,今乃三百頭為一群,誰謂爾無牛,今乃犉者九十頭,言其多矣,足如古也。}\textbf{爾羊來思,其角濈濈。}{\footnotesize 聚其角而息濈濈然。箋云言此者,美畜產得其所。}\textbf{爾牛來思,其耳濕濕。}{\footnotesize 呞而動,其耳濕濕然。}

\begin{quoting}呞 \texttt{shī},牛反芻也。\end{quoting}

\textbf{或降于阿,或飲于池,或寢或訛。}{\footnotesize 訛,動也。箋云言此者,美其無所驚畏也。}\textbf{爾牧來思,何蓑何笠,或負其餱。}{\footnotesize 何,揭也。蓑所以備雨,笠所以御暑。箋云言此者,美牧人寒暑飲食有備。}\textbf{三十維物,爾牲則具。}{\footnotesize 異毛色者三十也。箋云牛羊之色異者三十,則女之祭祀,索則有之。}

\begin{quoting}釋文「訛,韓詩作譌,譌,覺也」。物,謂毛色也。\end{quoting}

\textbf{爾牧來思,以薪以蒸,以雌以雄。}{\footnotesize 箋云此言牧人有餘力則取薪蒸、搏禽獸以來歸也。麄曰薪,細曰蒸。}\textbf{爾羊來思,矜矜兢兢,不騫不崩。}{\footnotesize 矜矜兢兢,以言堅彊也。騫,虧也。崩,群疾也。}\textbf{麾之以肱,畢來既升。}{\footnotesize 肱,臂也。升,升入牢也。箋云此言擾馴從人意也。}

\begin{quoting}\textbf{林義光}不騫不崩,言群羊馴謹相隨,無走失之患也。麾,同揮。畢、既同義,盡、完全。\end{quoting}

\textbf{牧人乃夢,眾維魚矣,旐維旟矣。}{\footnotesize 箋云牧人乃夢見人眾相與捕魚,又夢見旐與旟,占夢之官得而獻之于宣王,將以占國事也。}\textbf{大人占之,眾維魚矣,實維豐年,}{\footnotesize 陰陽和則魚眾多矣。箋云魚者,庶人之所以養也,今人眾相與捕魚,則是歲孰相供養之祥也,易中孚卦曰「豚魚吉」。}\textbf{旐維旟矣,室家溱溱。}{\footnotesize 溱溱,眾也。旐旟所以聚眾也。箋云溱溱,子孫眾多也。}

\begin{quoting}旐,\textbf{于省吾}以為應讀作兆,以金文中白、會皆从㫃也,眾、兆雙聲疊義,古之疊義連語往往分用,此詩「維」為句中助詞,本謂所夢的是魚之眾與旟 \texttt{yú} 之多,眾魚為豐年之徵,兆旟為家室繁盛之驗。\textbf{陳奐}實,當作寔,寔,是也。\end{quoting}

%\begin{flushright}鴻鴈之什十篇、三十二章、二百三十三句\end{flushright}