\chapter{Lesson 12}

\section*{Reading 1}

Evaṃ me sutaṃ ekaṃ samayaṃ Bhagavā Sāvatthiyaṃ viharati Jetavane Anāthapiṇḍikassa ārāme. Atha kho Bhagavā pubbaṇhasamayaṃ nivāsetvā pattacīvaram ādāya Sāvatthiṃ piṇḍāya pāvisi. Tena kho pana samayena Aggikabhāradvājassa brāhmaṇassa nivesane aggi pajjalito hoti āhuti paggahitā. Atha kho Bhagavā Sāvatthiyaṃ sapadānaṃ piṇḍāya caramāno yena Aggikabhāradvājassa brāhmaṇassa nivesanaṃ ten’upasaṅkami. Addasā kho Aggikabhāradvājo brāhmaṇo Bhagavantaṃ dūrato va āgacchantaṃ. Disvāna Bhagavantaṃ etadavoca “Tatr’eva, muṇḍaka, tatr’eva, samaṇaka, tatr’eva, vasalaka tiṭṭhāhī ” ti. Evaṃ vutte, Bhagavā Aggikabhāradvājaṃ brāhmaṇaṃ etadavoca “Jānāsi pana tvaṃ, brāhmaṇa, vasalaṃ vā vasalakaraṇe vā dhamme” ti? “Na khvāhaṃ, bho Gotama, jānāmi vasalaṃ vā vasalakaraṇe vā dhamme, sādhu me bhavaṃ Gotamo tathā dhammaṃ desetu, yathāhaṃ jāneyyaṃ vasalaṃ vā vasalakaraṇe vā dhamme” ti. “Tena hi, brāhmaṇa, suṇāhi, sādhukaṃ manasi karohi, bhāsissāmī” ti. “Evaṃ, bho” ti kho Aggikabhāradvājo brāhmaṇo Bhagavato paccassosi. Bhagavā etadavoca

“Kodhano upanāhī ca,\\
pāpamakkhī ca yo naro,\\
vipannadiṭṭhi māyāvī,\\
taṃ jaññā ‘vasalo’ iti.

Ekajaṃ vā dvijaṃ vāpi,\\
yo’dha pāṇaṃ vihiṃsati,\\
yassa pāṇe dayā natthi,\\
taṃ jaññā ‘vasalo’ iti.

Yo hanti parirundhati,\\
gāmāni nigamāni ca,\\
niggāhako samaññāto,\\
taṃ jaññā ‘vasalo’ iti.

Yo mātaraṃ pitaraṃ vā,\\
jiṇṇakaṃ gatayobbanaṃ,\\
pahu santo na bharati,\\
taṃ jaññā ‘vasalo’ iti.

Yo mātaraṃ pitaraṃ vā,\\
bhātaraṃ bhaginiṃ sasuṃ,\\
hanti roseti vācāya,\\
taṃ jaññā ‘vasalo’ iti.

Rosako kadariyo ca,\\
pāpiccho maccharī saṭho,\\
ahiriko anottappī,\\
taṃ jaññā ‘vasalo’ iti.

Na jaccā vasalo hoti,\\
na jaccā hoti brāhmaṇo,\\
kammanā vasalo hoti,\\
kammanā hoti brāhmaṇo” ti.

Evaṃ vutte, Aggikabhāradvājo brāhmaṇo Bhagavantaṃ etadavoca “abhikkantaṃ, bho Gotama, abhikkantaṃ, bho Gotama. Seyyathāpi, bho Gotama, nikkujjitaṃ vā ukkujjeyya, paṭicchannaṃ vā vivareyya, mūḷhassa vā maggaṃ ācikkheyya, andhakāre vā telapajjotaṃ dhāreyya cakkhumanto rūpāni dakkhintī ti, evameva bhotā Gotamena anekapariyāyena dhammo pakāsito. Esāhaṃ bhagavantaṃ Gotamaṃ saraṇaṃ gacchāmi dhammañ ca bhikkhusaghañ ca, upāsakaṃ maṃ bhavaṃ Gotamo dhāretu ajjatagge pāṇ’upetaṃ saraṇaṃ gatan” ti. \hfill(Sn 1.7)

\section*{Reading 2}

Kacci abhiṇhasaṃvāsā,\\
nāvajānāsi paṇḍitaṃ?\\
ukkādhāro manussānaṃ,\\
kacci apacito tayā?

Nāhaṃ abhiṇhasaṃvāsā,\\
avajānāmi paṇḍitaṃ,\\
ukkādhāro manussānaṃ,\\
niccaṃ apacito mayā.

Pañca kāmaguṇe hitvā,\\
piyarūpe manorame,\\
saddhāya gharā nikkhamma,\\
dukkhass’antakaro bhava.

Mitte bhajassu kalyāṇe,\\
pantañ ca sayanāsanaṃ,\\
vivittaṃ appanigghosaṃ,\\
mattaññū hohi bhojane.

Cīvare piṇḍapāte ca,\\
paccaye sayanāsane,\\
etesu taṇhaṃ mākāsi,\\
mā lokaṃ punarāgami.

Saṃvuto pātimokkhasmiṃ,\\
indriyesu ca pañcasu,\\
sati kāyagatā ty’atthu,\\
nibbidābahulo bhava.

Nimittaṃ parivajjehi,\\
subhaṃ rāgūpasaṃhitaṃ,\\
asubhāya cittaṃ bhāvehi,\\
ekaggaṃ susamāhitaṃ.

Animittañ ca bhāvehi,\\
mānānusayam ujjaha,\\
tato mānābhisamayā,\\
upasanto carissasī ti.

Itthaṃ sudaṃ Bhagavā āyasmantaṃ rāhulaṃ imāhi gāthāhi abhiṇhaṃ ovadati. \hfill(Sn 2.11)

\section*{Reading 3}

Atha kho āyasmā Ānando yena Bhagavā ten’upasaṅkami, upasaṅkamitvā Bhagavantaṃ abhivādetvā ekamantaṃ nisīdi. Ekamantaṃ nisinno kho āyasmā Ānando Bhagavantaṃ etadavoca “Tīṇ’imāni, bhante, gandhajātāni, yesaṃ anuvātaññeva gandho gacchati, no paṭivātaṃ. Katamāni tīṇi? Mūlagandho, sāragandho, pupphagandho, imāni kho, bhante, tīṇī gandhajātāni, yesaṃ anuvātaññeva gandho gacchati, no paṭivātaṃ. Atthi nu kho, bhante, kiñci gandhajātaṃ yassa anuvātam pi gandho gacchati, paṭivātam pi gandho gacchati, anuvātapaṭivātam pi gandho gacchatī” ti?

“Atth’Ānanda, kiñci gandhajātaṃ yassa anuvātam pi gandho gacchati, paṭivātam pi gandho gacchati, anuvātapaṭivātam pi gandho gacchatī” ti.

“Katamañ ca pana, bhante, gandhajātaṃ yassa anuvātam pi gandho gacchati, paṭivātam pi gandho gacchati, anuvātapaṭivātam pi gandho gacchatī” ti?

“Idh’Ānanda, yasmiṃ gāme vā nigame vā itthī vā puriso vā buddhaṃ saraṇaṃ gato hoti, dhammaṃ saraṇaṃ gato hoti, saṅghaṃ saraṇaṃ gato hoti, pāṇātipātā paṭivirato hoti, adinnādānā paṭivirato hoti, kāmesu micchācārā paṭivirato hoti, musāvādā paṭivirato hoti, surāmerayamajjapamādaṭṭhānā paṭivirato hoti, sīlavā hoti kalyāṇadhammo, vigatamalamaccherena cetasā agāraṃ ajjhāvasati….

Tassa disāsu samaṇabrāhmaṇā vaṇṇaṃ bhāsanti ‘asukasmiṃ nāma gāme vā nigame vā itthī vā puriso vā buddhaṃ saraṇaṃ gato hoti, dhammaṃ saraṇaṃ gato hoti, saṅghaṃ saraṇaṃ gato hoti, pāṇātipātā paṭivirato hoti, adinnādānā paṭivirato hoti, kāmesumicchācārā paṭivirato hoti, musāvādā paṭivirato hoti, surāmerayamajjapamādaṭṭhānā paṭivirato hoti, sīlavā hoti kalyāṇadhammo, vigatamalamaccharena cetasā agāraṃ ajjhāvasati…’ ti.

Devatāpissa vaṇṇaṃ bhāsanti ‘asukasmiṃ nāma gāme vā nigame vā itthī vā puriso vā buddhaṃ saraṇaṃ gato hoti, dhammaṃ saraṇaṃ gato hoti… pe… sīlavā hoti kalyāṇadhammo, vigatamalamaccherena cetasā agāraṃ ajjhāvasati…’ ti.

Idaṃ kho taṃ, Ānanda, gandhajātaṃ yassa anuvātam pi gandho gacchati, paṭivātam pi gandho gacchati, anuvātapaṭivātam pi gandho gacchatī” ti.

“Na pupphagandho paṭivātam eti,\\
na candanaṃ tagaramallikā vā.\\
satañ ca gandho paṭivātam eti,\\
sabbā disā sappuriso pavāti.” \hfill(A 3.80)

\section*{Reading 4}

Sāvatthiyaṃ Adinnapubbako nāma brāhmaṇo ahosi. Tena kassaci kiñci na dinnapubbaṃ. Tassa eko va putto ahosi, piyo manāpo. Brāhmaṇo puttassa pilandhanaṃ dātukāmo “sace suvaṇṇakārassa ācikkhissāmi, vetanaṃ dātabbaṃ bhavissatī” ti sayam eva suvaṇṇaṃ koṭṭetvā maṭṭāni kuṇḍalāni katvā adāsi, ten’assa putto ‘Maṭṭakuṇḍalī’ ti paññāyi.

Tassa soḷasavassakāle paṇḍurogo udapādi. Brāhmaṇo vejjānaṃ santikaṃ gantvā “tumhe asukarogassa kiṃ bhesajjaṃ karothā” ti pucchi. Te assa yaṃ vā taṃ vā rukkhatacādiṃ ācikkhiṃsu. So taṃ āharitvā bhesajjaṃ kari. Tathā karontass’eva tassa rogo balavā ahosi. Brāhmaṇo tassa dubbalabhāvaṃ ñatvā ekaṃ vejjaṃ pakkosi. So taṃ oloketvā “amhākaṃ ekaṃ kiccaṃ atthi, aññaṃ vejjaṃ pakkositvā tikicchāpehī” ti vatvā nikkhami.

Brāhmaṇo tassa maraṇasamayaṃ ñatvā “imassa dassan’atthāya āgatāgatā antogehe sāpateyyaṃ passissanti, tasmā naṃ bahi karissāmī” ti puttaṃ nīharitvā bahi āḷinde nipajjāpesi. Tasmiṃ kālakate brāhmaṇo tassa sarīraṃ jhāpetvā, devasikaṃ āḷāhanaṃ gantvā “kahaṃ ekaputtaka, kahaṃ ekaputtakā” ti rodi. \hfill(Rasv)

\section*{Dhammacakkappavattanasuttaṃ}

Evaṃ me sutaṃ. Ekaṃ samayaṃ Bhagavā Bārāṇasiyaṃ viharati Isipatane Migadāye. Tatra kho Bhagavā pañcavaggiye bhikkhū āmantesi “Dve’me, bhikkhave, antā pabbajitena na sevitabbā. Katame dve? Yo cāyaṃ kāmesu kāmasukhallikānuyogo hīno, gammo, pothujjaniko, anariyo, anatthasaṃhito, yo cāyaṃ attakilamathānuyogo dukkho, anariyo, anatthasaṃhito.

Ete kho, bhikkhave, ubho ante anupagamma majjhimā paṭipadā Tathāgatena abhisambuddhā cakkhukaraṇī, ñāṇakaraṇī, upasamāya, abhiññāya, sambodhāya, nibbānāya saṃvattati. Katamā ca sā, bhikkhave, majjhimā paṭipadā Tathāgatena abhisambuddhā cakkhukaraṇī, ñāṇakaraṇī, upasamāya, abhiññāya, sambodhāya, nibbānāya saṃvattati?

Ayam eva ariyo aṭṭhaṅgiko maggo, seyyathidaṃ sammādiṭṭhi, sammāsaṅkappo, sammāvācā, sammākammanto, sammāājīvo, sammāvāyāmo, sammāsati, sammāsamādhi. Ayaṃ kho sā, bhikkhave, majjhimā paṭipadā Tathāgatena abhisambuddhā cakkhukaraṇī, ñāṇakaraṇī, upasamāya, abhiññāya, sambodhāya, nibbānāya saṃvattati.

Idaṃ kho pana, bhikkhave, dukkhaṃ ariyasaccaṃ jāti pi dukkhā, jarā pi dukkhā, byādhi pi dukkho, maraṇam pi dukkhaṃ, appiyehi sampayogo dukkho, piyehi vippayogo dukkho, yam p’icchaṃ na labhati tam pi dukkhaṃ, saṅkhittena pañcupādānakkhandhā dukkhā.

Idaṃ kho pana, bhikkhave, dukkhasamudayaṃ ariyasaccaṃ yāyaṃ taṇhā ponobbhavikā‚ nandirāgasahagatā tatratatrābhinandinī, seyyathidaṃ kāmataṇhā, bhavataṇhā, vibhavataṇhā.

Idaṃ kho pana, bhikkhave, dukkhanirodhaṃ ariyasaccaṃ yo tassā yeva taṇhāya asesavirāganirodho, cāgo, paṭinissaggo, mutti, anālayo.

Idaṃ kho pana, bhikkhave, dukkhanirodhagāminī paṭipadā ariyasaccaṃ ayam eva ariyo aṭṭhaṅgiko maggo, seyyathidaṃ sammādiṭṭhi, sammāsaṅkappo, sammāvācā, sammākammanto, sammāājīvo, sammāvāyāmo, sammāsati, sammāsamādhi.

\begin{center}
    * * * * * * *
\end{center}

‘Idaṃ dukkhaṃ ariyasaccan’ ti me, bhikkhave, pubbe ananussutesu dhammesu cakkhuṃ udapādi, ñāṇaṃ udapādi, paññā udapādi, vijjā udapādi, āloko udapādi.

‘Taṃ kho pan’idaṃ dukkhaṃ ariyasaccaṃ pariññeyyan’ ti me, bhikkhave, pubbe ananussutesu dhammesu cakkhuṃ udapādi, ñāṇaṃ udapādi, paññā udapādi, vijjā udapādi, āloko udapādi.

‘Taṃ kho pan’idaṃ dukkhaṃ ariyasaccaṃ pariññātan’ ti me, bhikkhave, pubbe ananussutesu dhammesu cakkhuṃ udapādi, ñāṇaṃ udapādi, paññā udapādi, vijjā udapādi, āloko udapādi.

‘Idaṃ dukkhasamudayaṃ ariyasaccan’ ti me, bhikkhave, pubbe ananussutesu dhammesu cakkhuṃ udapādi, ñāṇaṃ udapādi, paññā udapādi, vijjā udapādi, āloko udapādi.

‘Taṃ kho pan’idaṃ dukkhasamudayaṃ ariyasaccaṃ pahātabban’ ti me, bhikkhave, pubbe ananussutesu dhammesu cakkhuṃ udapādi, ñāṇaṃ udapādi, paññā udapādi, vijjā udapādi, āloko udapādi.

‘Taṃ kho pan’idaṃ dukkhasamudayaṃ ariyasaccaṃ pahīnan’ ti me, bhikkhave, pubbe ananussutesu dhammesu cakkhuṃ udapādi, ñāṇaṃ udapādi, paññā udapādi, vijjā udapādi, āloko udapādi.

‘Idaṃ dukkhanirodhaṃ ariyasaccan’ ti me, bhikkhave, pubbe ananussutesu dhammesu cakkhuṃ udapādi, ñāṇaṃ udapādi, paññā udapādi, vijjā udapādi, āloko udapādi.

‘Taṃ kho pan’idaṃ dukkhanirodhaṃ ariyasaccaṃ sacchikātabban’ ti me, bhikkhave, pubbe ananussutesu dhammesu cakkhuṃ udapādi, ñāṇaṃ udapādi, paññā udapādi, vijjā udapādi, āloko udapādi.

‘Taṃ kho pan’idaṃ dukkhanirodhaṃ ariyasaccaṃ sacchikatan’ ti me, bhikkhave, pubbe ananussutesu dhammesu cakkhuṃ udapādi, ñāṇaṃ udapādi, paññā udapādi, vijjā udapādi, āloko udapādi.

‘Idaṃ dukkhanirodhagāminī paṭipadā ariyasaccan’ ti me, bhikkhave, pubbe ananussutesu dhammesu cakkhuṃ udapādi, ñāṇaṃ udapādi, paññā udapādi, vijjā udapādi, āloko udapādi.

‘Taṃ kho pan’idaṃ dukkhanirodhagāminī paṭipadā ariyasaccaṃ bhāvetabban’ ti me, bhikkhave, pubbe ananussutesu dhammesu cakkhuṃ udapādi, ñāṇaṃ udapādi, paññā udapādi, vijjā udapādi, āloko udapādi.

‘Taṃ kho pan’idaṃ dukkhanirodhagāminī paṭipadā ariyasaccaṃ bhāvitan’ ti me, bhikkhave, pubbe ananussutesu dhammesu cakkhuṃ udapādi, ñāṇaṃ udapādi, paññā udapādi, vijjā udapādi, āloko udapādi.

\begin{center}
    * * * * * * *
\end{center}

Yāvakīvañ ca me, bhikkhave, imesu catūsu ariyasaccesu evaṃ tiparivaṭṭaṃ dvādasākāraṃ yathābhūtaṃ ñāṇadassanaṃ na suvisuddhaṃ ahosi, n’eva tāvāhaṃ, bhikkhave, sadevake loke samārake sabrahmake sassamaṇabrāhmaṇiyā pajāya sadevamanussāya ‘anuttaraṃ sammāsambodhiṃ abhisambuddho’ ti paccaññāsiṃ.

Yato ca kho me, bhikkhave, imesu catūsu ariyasaccesu evaṃ tiparivaṭṭaṃ dvādasākāraṃ yathābhūtaṃ ñāṇadassanaṃ suvisuddhaṃ ahosi, athāhaṃ, bhikkhave, sadevake loke samārake sabrahmake sassamaṇabrāhmaṇiyā pajāya sadevamanussāya ‘anuttaraṃ sammāsambodhiṃ abhisambuddho’ ti paccaññāsiṃ.

Ñāṇañ ca pana me dassanaṃ udapādi ‘akuppā me vimutti‚ ayaṃ antimā jāti, natth’idāni punabbhavo’” ti.

Idam avoca Bhagavā. Attamanā pañcavaggiyā bhikkhū Bhagavato bhāsitaṃ abhinandun ti.

\begin{center}
    * * * * * * *
\end{center}

Imasmiñ ca pana veyyākaraṇasmiṃ bhaññamāne āyasmato Koṇḍaññassa virajaṃ vītamalaṃ dhammacakkhuṃ udapādi “yaṃ kiñci samudayadhammaṃ, sabbaṃ taṃ nirodhadhamman” ti.

Pavattite ca pana Bhagavatā dhammacakke Bhummā devā saddamanussāvesuṃ “etaṃ Bhagavatā Bārāṇasiyaṃ Isipatane Migadāye anuttaraṃ dhammacakkaṃ pavattitaṃ appaṭivattiyaṃ samaṇena vā brāhmaṇena vā devena vā mārena vā brahmunā vā kenaci vā lokasmin” ti.

Bhummānaṃ devānaṃ saddaṃ sutvā Cātumahārājikā devā saddamanussāvesuṃ “etaṃ Bhagavatā Bārāṇasiyaṃ Isipatane Migadāye anuttaraṃ dhammacakkaṃ pavattitaṃ, appaṭivattiyaṃ samaṇena vā brāhmaṇena vā devena vā mārena vā brahmunā vā kenaci vā lokasmin” ti.

Cātumahārājikānaṃ devānaṃ saddaṃ sutvā Tāvatiṃsā devā… pe… Yāmā devā… pe… Tusitā devā… pe… Nimmānaratī devā… pe… Paranimmitavasavattī devā… pe… Brahmakāyikā devā saddamanussāvesuṃ “etaṃ Bhagavatā Bārāṇasiyaṃ Isipatane Migadāye anuttaraṃ dhammacakkaṃ pavattitaṃ appaṭivattiyaṃ samaṇena vā brāhmaṇena vā devena vā mārena vā brahmunā vā kenaci vā lokasmin” ti.

Itīha tena khaṇena, tena layena‚ tena muhuttena yāva Brahmalokā saddo abbhuggacchi. Ayañ ca dasasahassi lokadhātu saṅkampi sampakampi sampavedhi. Appamāṇo ca uḷāro obhāso loke pāturahosi atikkamma devānaṃ devānubhāvaṃ.

Atha kho Bhagavā udānaṃ udānesi “Aññāsi vata, bho Koṇḍañño, aññāsi vata, bho Koṇḍañño” ti.

Iti h’idaṃ āyasmato Koṇḍaññassa ‘Aññāsi Koṇḍañño’ tv’eva nāmaṃ ahosī ti. \hfill(S 56.11 Dhammacakkappavattanasuttaṃ)