\chapter{儉嗇第二十九}

\subsection*{1}

\textbf{和嶠性至儉,家有好李,王武子求之,與不過數十,王武子因其上直,率將少年能食之者,持斧詣園,飽共噉畢,伐之,送一車枝與和公,問曰:「何如君李?」和既得,唯笑而已。}{\footnotesize \textbf{晉諸公贊}曰嶠性不通,治家富擬王公,而至儉,將有犯義之名。\textbf{語林}曰嶠諸弟往園中食李,而皆計核責錢,故嶠婦弟王濟伐之也。}

\subsection*{2}

\textbf{王戎儉吝,其從子婚,與一單衣,後更責之。}{\footnotesize \textbf{王隱晉書}曰戎性至儉,不能自奉養,財不出外,天下人謂為膏肓之疾。}

\subsection*{3}

\textbf{司徒王戎,既貴且富,區宅僮牧、膏田水碓之屬,洛下無比,契疏鞅掌,每與夫人燭下散籌算計。}{\footnotesize \textbf{晉諸公贊}曰戎性簡要,不治儀望,自遇甚薄,而產業過豐,論者以為台輔之望不重。\textbf{王隱晉書}曰戎好治生,園田周徧天下,翁嫗二人,常以象牙籌晝夜筭計家資。\textbf{晉陽秋}曰戎多殖財賄,常若不足,或謂戎故以此自晦也。\textbf{戴逵}論之曰王戎晦默於危亂之際,獲免憂禍,既明且哲,於是在矣,或曰「大臣用心,豈其然乎」,逵曰「運有險易,時有昏明,如子之言,則蘧瑗、季札之徒皆負責矣,自古而觀,豈一王戎也哉」。}

\subsection*{4}

\textbf{王戎有好李,賣之,恐人得其種,恆鑽其核。}

\subsection*{5}

\textbf{王戎女適裴頠,貸錢數萬,女歸,戎色不說,女遽還錢,乃釋然。}

\subsection*{6}

\textbf{衛江州在尋陽,}{\footnotesize \textbf{永嘉流人名}曰衛展,字道舒,河東安邑人,祖列,彭城護軍,父韶,廣平令,展光熙初除鷹揚將軍、江州刺史。}\textbf{有知舊人投之,都不料理,唯餉王不留行一斤,此人得餉,便命駕,}{\footnotesize \textbf{本草}曰王不留行,生太山,治金瘡,除風,久服之,輕身。}\textbf{李弘範聞之曰:「家舅刻薄,乃復驅使草木。」}{\footnotesize \textbf{中興書}曰李軌,字弘範,江夏人,仕至尚書郎。\textbf{按}軌,劉氏之甥,此應弘度,非弘範也。}

\subsection*{7}

\textbf{王丞相儉節,帳下甘果盈溢不散,涉春爛敗,都督白之,公令舍去,曰:「慎不可令大郎知。」}{\footnotesize 王悅也。}

\subsection*{8}

\textbf{蘇峻之亂,庾太尉南奔見陶公,陶公雅相賞重,陶性儉吝,及食,噉薤,庾因留白,陶問:「用此何為?」庾云:「故可種。」於是大歎庾非唯風流,兼有治實。}

\subsection*{9}

\textbf{郗公大聚歛,有錢數千萬,嘉賓意甚不同,常朝旦問訊,郗家法,子弟不坐,因倚語移時,遂及財貨事,郗公曰:「汝正當欲得吾錢耳。」迺開庫一日,令任意用,郗公始正謂損數百萬許,嘉賓遂一日乞與親友,周旋略盡,郗公聞之,驚怪不能已已。}{\footnotesize \textbf{中興書}曰超少卓犖而不羈,有曠世之度。}