\chapter{鹿鳴之什詁訓傳第十六}

\begin{quoting}雅本樂器名,孳乳而為樂調名,左傳昭二十年「天子之樂曰雅」。雅有小大之別,當以音樂別之,不以政之大小論也。\textbf{釋文}從鹿鳴至菁菁者莪凡二十二篇,皆正小雅,六篇亡,今唯十六篇,從此至魚麗十篇是文武之小雅,先其文王以治內,後其武王以治外,宴勞嘉賓、親睦九族事非隆重,故為小雅,皆聖人之迹,故謂之正。\end{quoting}

\section{鹿鳴}

%{\footnotesize 三章、章八句}

\textbf{鹿鳴,燕群臣嘉賓也。既飲食之,又實幣帛筐篚以將其厚意,然後忠臣嘉賓得盡其心矣。}{\footnotesize 飲之而有幣,酬幣也,食之而有幣,侑幣也。}

\begin{quoting}據臧琳經義雜記考證,魏武帝時有杜夔者,尚能歌此調。\end{quoting}

\textbf{呦呦鹿鳴,食野之苹。}{\footnotesize 興也。苹,蓱也。鹿得蓱,呦呦然鳴而相呼,懇誠發乎中,以興嘉樂賓客,當有懇誠相招呼以成禮也。箋云苹,藾蕭也。}\textbf{我有嘉賓,鼓瑟吹笙。吹笙鼓簧,承筐是將。}{\footnotesize 簧,笙也,吹笙而鼓簧矣。筐,篚屬,所以行幣帛也。箋云承,猶奉也,書曰「篚厥玄黃」。}\textbf{人之好我,示我周行。}{\footnotesize 周至、行道也。箋云示,當作寘,寘,置也。周行,周之列位也。好,猶善也。人有以德善我者,我則置之於周之列位,言己維賢是用。}

\begin{quoting}笙,竹匏所製。楚辭王逸注「笙中有舌曰簧」。\end{quoting}

\textbf{呦呦鹿鳴,食野之蒿。}{\footnotesize 蒿,菣也。}\textbf{我有嘉賓,德音孔昭。視民不恌,君子是則是傚。}{\footnotesize 恌,愉也。是則是傚,言可法傚也。箋云德音,先王道德之教也。孔甚、昭明也。視,古示字也。飲酒之禮,於旅也語,嘉賓之語先王德教甚明,可以示天下之民,使之不愉於禮義,是乃君子所法傚,言其賢也。}\textbf{我有旨酒,嘉賓式燕以敖。}{\footnotesize 敖,遊也。}

\begin{quoting}德音,\textbf{于省吾}謂應作「德言」,即人內在之德性與外在之言語。恌,左傳昭十年及說文引詩皆作佻,偷薄、不厚道之意。\textbf{馬瑞辰}爾雅舍人注云「敖,意舒也」,凡人樂則意舒,是知敖有樂意。嘉賓式燕以敖,猶南有嘉魚詩「嘉賓式燕以樂」,車舝詩「式燕且喜、式燕且譽」也。\end{quoting}

\textbf{呦呦鹿鳴,食野之芩。}{\footnotesize 芩,草也。}\textbf{我有嘉賓,鼓瑟鼓琴。鼓瑟鼓琴,和樂且湛。}{\footnotesize 湛,樂之久。}\textbf{我有旨酒,以燕樂嘉賓之心。}{\footnotesize 燕,安也。夫不能致其樂則不能得其志,不能得其志則嘉賓不能竭其力。}

\begin{quoting}\textbf{馬瑞辰}苹、芩皆蒿屬,足證古人因物起興每多以類相從。又曰此詩三章,文法參差而義實相承,首章前六句言我之敬賓,後二句言賓之善我,二章前六句即承首章人之好我言,後二句乃言我之樂賓,三章即接言賓之樂,後二句又申我之樂賓,以明賓之樂實我有以致之也。\end{quoting}

\section{四牡}

%{\footnotesize 五章、章五句}

\textbf{四牡,勞使臣之來也。有功而見知,則說矣。}{\footnotesize 文王為西伯之時,三分天下有其二以服事殷,使臣以王事往來於其職,於其來也,陳其功苦以歌樂之。}

\begin{quoting}左傳襄四年「四牡,君所以勞使臣也」。儀禮燕禮、鄉飲酒禮亦歌此詩。\end{quoting}

\textbf{四牡騑騑,周道倭遲,}{\footnotesize 騑騑,行不止之貌。周道,岐周之道也。倭遲,歷遠之貌。文王率諸侯撫叛國而朝聘乎紂,故周公作樂,以歌文王之道,為後世法。}\textbf{豈不懷歸?王事靡盬,我心傷悲。}{\footnotesize 盬,不堅固也。思歸者,私恩也,靡盬者,公義也,傷悲者,情思也。箋云無私恩,非孝子也,無公義,非忠臣也,君子不以私害公,不以家事辭王事。}

\textbf{四牡騑騑,嘽嘽駱馬,}{\footnotesize 嘽嘽,喘息之貌,馬勞則喘息。白馬黑鬣曰駱。}\textbf{豈不懷歸?王事靡盬,不遑啟處。}{\footnotesize 遑暇、啟跪、處居也。臣受命,舍幣于禰乃行。}

\begin{quoting}嘽,說文引詩亦作「痑」。啟,同跽,小跪。處,居也,居,同凥,說文「凥,處也,从尸得几而止」,安坐。古者席地而坐,臀不著足跟為跪,跪而聳身直腰為跽。\end{quoting}

\textbf{翩翩者鵻,載飛載下,集于苞栩。}{\footnotesize 鵻,夫不也。箋云夫不,鳥之愨謹者,人皆愛之,可以不勞,猶則飛則下,止於栩木,喻人雖無事,其可獲安乎,感厲之。}\textbf{王事靡盬,不遑將父。}{\footnotesize 將,養也。}

\textbf{翩翩者鵻,載飛載止,集于苞杞。}{\footnotesize 杞,枸檵也。}\textbf{王事靡盬,不遑將母。}

\begin{quoting}\textbf{馬瑞辰}左氏昭十七年傳「祝鳩氏,司徒也」,孔疏引樊光曰「祝鳩,夫不,孝,故為司徒」,是知詩以鵻取興者,正取其為孝鳥,故以興使臣之不遑將父、不遑將母,為鵻之不若耳。\end{quoting}

\textbf{駕彼四駱,載驟駸駸,}{\footnotesize 駸駸,驟貌。}\textbf{豈不懷歸?是用作歌,將母來諗。}{\footnotesize 諗,念也。父兼尊親之道,母至親而尊不至。箋云諗,告也。君勞使臣,述序其情,女曰我豈不思歸乎,誠思歸也,故作此詩之歌,以養父母之志來告於君也。人之思恆思親者,再言將母,亦其情也。}

\begin{quoting}驟,說文「馬疾步也」。是用,即因此。\textbf{經義述聞}來,猶是也。\end{quoting}

\section{皇皇者華}

%{\footnotesize 五章、章四句}

\textbf{皇皇者華,君遣使臣也。送之以禮樂,言遠而有光華也。}{\footnotesize 言臣出使能揚君之美,延其譽於四方,則為不辱命也。}

\begin{quoting}左傳襄四年「皇皇者華,君教使臣」。儀禮燕禮、鄉飲酒禮亦歌此詩。春秋宴會時常歌鹿鳴、四牡及此詩。\end{quoting}

\textbf{皇皇者華,于彼原隰。}{\footnotesize 皇皇,猶煌煌也。高平曰原,下濕曰隰。忠臣奉使,能光君命,無遠無近,如華不以高下易其色。箋云無遠無近,維所之則然。}\textbf{駪駪征夫,每懷靡及。}{\footnotesize 駪駪,眾多之貌。征夫,行人也。每雖、懷和也。箋云春秋外傳曰「懷私為每懷也」,和當為私。眾行夫既受君命,當速行,每人懷其私相稽留,則於事將無所及。}

\begin{quoting}皇、煌古今字。駪,楚辭王逸注引詩作「侁」,國語引作「莘」。\end{quoting}

\textbf{我馬維駒,六轡如濡。}{\footnotesize 箋云如濡,言鮮澤也。}\textbf{載馳載驅,周爰咨諏。}{\footnotesize 忠信為周,訪問於善為咨,咨事為諏。箋云爰,於也。大夫出使,馳驅而行,見忠信之賢人,則於是訪問,求善道也。}

\begin{quoting}駒,釋文「本亦作驕」,說文「馬高六尺為驕,詩曰我馬維驕」。\end{quoting}

\textbf{我馬維騏,六轡如絲。}{\footnotesize 言調忍也。}\textbf{載馳載驅,周爰咨謀。}{\footnotesize 咨事之難易為謀。}

\textbf{我馬維駱,六轡沃若。載馳載驅,周爰咨度。}{\footnotesize 咨禮義所宜為度。}

\textbf{我馬維駰,六轡既均。}{\footnotesize 陰白雜毛曰駰。均,調也。}\textbf{載馳載驅,周爰咨詢。}{\footnotesize 親戚之謀為詢。兼此五者,雖有中和,當自謂無所及成於六德也。箋云中和,謂忠信也。五者,咨也、諏也、謀也、度也、詢也,雖得此於忠信之賢人,猶當云己將無所及於事,則成六德,言慎其事。}

\section{常棣}

%{\footnotesize 八章、章四句}

\textbf{常棣,燕兄弟也。閔管蔡之失道,故作常棣焉。}{\footnotesize 周公弔二叔之不咸而使兄弟之恩疏,召公為作此詩而歌之以親之。}

\textbf{常棣之華,鄂不韡韡。}{\footnotesize 興也。常棣,棣也。鄂,猶鄂鄂然,言外發也。韡韡,光明也。箋云承華者曰鄂,不,當作拊,拊,鄂足也,鄂足得華之光明則韡韡然盛。興者,喻弟以敬事兄,兄以榮覆弟,恩義之顯亦韡韡然。古聲不、拊同。}\textbf{凡今之人,莫如兄弟。}{\footnotesize 聞常棣之言為今也。箋云聞常棣之言,始聞常棣華鄂之說也,如此則人之恩親無如兄弟之厚。}

\begin{quoting}\textbf{經傳釋詞}鄂不韡韡,猶言夭之沃沃。韡,同煒。\textbf{于省吾}鄂不,猶言胡不、遐不,鄂、胡、遐三字,就聲言之並屬淺喉,就韻言之並屬魚部。\end{quoting}

\textbf{死喪之威,兄弟孔懷。}{\footnotesize 威畏、懷思也。箋云死喪可畏怖之事,維兄弟之親甚相思念。}\textbf{原隰裒矣,兄弟求矣。}{\footnotesize 裒,聚也。求矣,言求兄弟也。箋云原也隰也以相與聚居之故,故能定高下之名,猶兄弟相求,故能立榮顯之名。}

\textbf{脊令在原,兄弟急難。}{\footnotesize 脊令,雝渠也,飛則鳴,行則搖,不能自舍耳。急難,言兄弟之相救於急難。箋云雝渠水鳥,而今在原,失其常處,則飛則鳴,求其類,天性也,猶兄弟之於急難。}\textbf{每有良朋,況也永歎。}{\footnotesize 況茲、永長也。箋云每有,雖也。良,善也。當急難之時,雖有善同門來,茲對之長嘆而已也。}

\textbf{兄弟鬩于牆,外禦其務。}{\footnotesize 鬩,狠也。箋云禦禁、務侮也。兄弟雖內鬩而外禦侮也。}\textbf{每有良朋,烝也無戎。}{\footnotesize 烝填、戎相也。箋云當急難之時,雖有善同門來,久也猶無相助己者。古聲填、寘、塵同。}

\begin{quoting}釋文「填,依字音田,與寘同,又依古聲音塵,塵,久也,故箋申之云古聲填、寘、塵同」。\end{quoting}

\textbf{喪亂既平,既安且寧。雖有兄弟,不如友生。}{\footnotesize 兄弟尚恩怡怡然,朋友以義切切然。箋云平,猶正也。安寧之時,以禮義相琢磨,則友生急。}

\begin{quoting}\textbf{馬瑞辰}生,語詞也,唐人詩「太瘦生」之類,皆以生為語助詞,實此詩及伐木詩「友生」倡之也。\end{quoting}

\textbf{儐爾籩豆,飲酒之飫。}{\footnotesize 儐陳、飫私也。不脫屨升堂謂之飫。箋云私者,圖非常之事,若議大疑於堂,則有飫禮焉。聽朝為公。}\textbf{兄弟既具,和樂且孺。}{\footnotesize 九族會曰和。孺,屬也。王與親戚燕則尚毛。箋云九族,從己上至高祖、下及玄孫之親也。屬者,以昭穆相次序。}

\begin{quoting}儐,韓詩作賓,經典中兩字常互用。飫 \texttt{yù},韓詩作醧,\textbf{馬瑞辰}以古音讀之,醧與豆、具、孺韻正協,作飫則聲入蕭宵部。皇侃「和,即樂也」。\end{quoting}

\textbf{妻子好合,如鼓瑟琴。}{\footnotesize 箋云好合,志意合也,合者,如鼓瑟琴之聲相應和也。王與族人燕,則宗婦內宗之屬亦從后於房中。}\textbf{兄弟既翕,和樂且湛。}{\footnotesize 翕,合也。}

\textbf{宜爾室家,樂爾妻帑。}{\footnotesize 帑,子也。箋云族人和,則得保樂其家中之大小。}\textbf{是究是圖,亶其然乎。}{\footnotesize 究深、圖謀、亶信也。箋云女深謀之,信其如是。}

\begin{quoting}樂 \texttt{lē}。帑,魯詩作孥。\end{quoting}

\section{伐木}

%{\footnotesize 六章、章六句}

\textbf{伐木,燕朋友故舊也。自天子至于庶人,未有不須友以成者,親親以睦,友賢不棄,不遺故舊,則民德歸厚矣。}

\textbf{伐木丁丁,鳥鳴嚶嚶。}{\footnotesize 興也。丁丁,伐木聲也。嚶嚶,驚懼也。箋云丁丁嚶嚶,相切直也。言昔日未居位在農之時,與友生於山巖伐木,為勤苦之事,猶以道德相切正也。嚶嚶,兩鳥聲也,其鳴之志,似於有友道然,故連言之。}\textbf{出自幽谷,遷于喬木。}{\footnotesize 幽深、喬高也。箋云遷,徙也,謂鄉時之鳥出從深谷,今移處高木。}\textbf{嚶其鳴矣,求其友聲。}{\footnotesize 君子雖遷於高位,不可以忘其朋友。箋云嚶其鳴矣,遷處高木者,求其友聲,求其尚在深谷者,其相得,則復鳴嚶嚶然。}

\begin{quoting}說文「遷,登也」。\end{quoting}

\textbf{相彼鳥矣,猶求友聲。矧伊人矣,不求友生。}{\footnotesize 矧,況也。箋云相,視也。鳥尚知居高木呼其友,況是人乎,可不求之。}\textbf{神之聽之,終和且平。}{\footnotesize 箋云以可否相增減,曰和平齊等也。此言心誠求之,神若聽之,使得如志,則友終相與和而齊功也。}

\textbf{伐木許許,釃酒有藇。}{\footnotesize 許許,杮貌。以筐曰釃,以藪曰湑。藇,美貌。箋云此言前者伐木許許之人,今則有酒而釃之,本其故也。}\textbf{既有肥羜,以速諸父。}{\footnotesize 羜,未成羊也。天子謂同姓諸侯、諸侯謂同姓大夫皆曰父,異姓則稱舅。國君友其賢臣,大夫士友其宗族之仁者。箋云速,召也。有酒有羜,今以召族人飲酒。}\textbf{寧適不來,微我弗顧。}{\footnotesize 微,無也。箋云寧召之,適自不來,無使言我不顧念也。}

\begin{quoting}許許,或作滸滸,同所,說文「所,伐木聲也,从斤戶聲,詩曰伐木所所」,段注「丁丁,刀斧聲,所所,鋸聲」。\end{quoting}

\textbf{於粲洒埽,陳饋八簋。}{\footnotesize 粲,鮮明貌。圓曰簋,天子八簋。箋云粲然已灑拚矣,陳其黍稷矣,謂為食禮。}\textbf{既有肥牡,以速諸舅。寧適不來,微我有咎。}{\footnotesize 咎,過也。}

\textbf{伐木于阪,釃酒有衍。}{\footnotesize 衍,美貌。箋云此言伐木于阪,亦本之也。}\textbf{籩豆有踐,兄弟無遠。}{\footnotesize 箋云踐,陳列貌。兄弟,父之黨,母之黨。}\textbf{民之失德,乾餱以愆。}{\footnotesize 餱,食也。箋云失德,謂見謗訕也。民尚以乾餱之食獲愆過於人,況天子之饌,反可以恨兄弟乎,故不當遠之。}

\begin{quoting}說文「餱,乾食也」。漢書宣帝紀引詩,顏注「人無恩德不相飲食」。\end{quoting}

\textbf{有酒湑我,無酒酤我。}{\footnotesize 湑,莤之也。酤,一宿酒也。箋云酤,買也。此族人陳王之恩也,王有酒則泲莤之,王無酒酤買之,要欲厚於族人。}\textbf{坎坎鼓我,蹲蹲舞我。}{\footnotesize 蹲蹲,舞貌。箋云為我擊鼓坎坎然,為我興舞蹲蹲然,謂以樂樂己。}\textbf{迨我暇矣,飲此湑矣。}{\footnotesize 箋云迨,及也。此又述王意也,王曰「及我今之閒暇,共飲此湑酒」,欲其無不醉之意。}

\begin{quoting}湑我、酤我,即我湑、我酤,下鼓我、舞我同。\end{quoting}

\section{天保}

%{\footnotesize 六章、章六句}

\textbf{天保,下報上也。君能下下以成其政,臣能歸美以報其上焉。}{\footnotesize 下下,謂鹿鳴至伐木皆君所以下臣也,臣亦宜歸美於王,以崇君之尊而福祿之,以答其歌。}

\begin{quoting}\textbf{朱熹}文王時周未有先王者,此必武王以後所作也。\end{quoting}

\textbf{天保定爾,亦孔之固。}{\footnotesize 固,堅也。箋云保安、爾女也,女,王也。天之安定女,亦甚堅固。}\textbf{俾爾單厚,何福不除。}{\footnotesize 俾使、單信也。或曰單,厚也,除,開也。箋云單,盡也。天使女盡厚天下之民,何福而不開,皆開出以予之。}\textbf{俾爾多益,以莫不庶。}{\footnotesize 庶,眾也。箋云莫,無也。使女每物益多,以是故無不眾也。}

\begin{quoting}\textbf{陳奐}通篇十爾字,皆指君上也。單,魯詩作亶,亶本字。\textbf{馬瑞辰}除 \texttt{zhù}、余古通用,余、予古今字,何福不除,猶云何福不予。以,語詞,下同。\end{quoting}

\textbf{天保定爾,俾爾戩穀。罄無不宜,受天百祿。}{\footnotesize 戩福、穀祿、罄盡也。箋云天使女所福祿之人,謂群臣也,其舉事盡得其宜,受天之多祿。}\textbf{降爾遐福,維日不足。}{\footnotesize 箋云遐,遠也。天又下予女以廣遠之福,使天下溥蒙之,汲汲然如日且不足也。}

\textbf{天保定爾,以莫不興。}{\footnotesize 箋云興,盛也,無不盛者,使萬物皆盛,草木暢茂,禽獸碩大。}\textbf{如山如阜,如岡如陵。}{\footnotesize 言廣厚也。高平曰陸,大陵曰阜,大阜曰陵。箋云此言其福祿委積高大也。}\textbf{如川之方至,以莫不增。}{\footnotesize 箋云川之方至,謂其水縱長之時也,萬物之收皆增多也。}

\textbf{吉蠲為饎,是用孝享。}{\footnotesize 吉善、蠲絜也。饎,酒食也。享,獻也。箋云謂將祭祀也。}\textbf{禴祠烝嘗,于公先王。}{\footnotesize 春曰祠,夏曰禴,秋曰嘗,冬曰烝。公,事也。箋云公,先公,謂后稷至諸盩。}\textbf{君曰卜爾,萬壽無疆。}{\footnotesize 君,先君也,尸所以象神。卜,予也。箋云君曰卜爾者,尸嘏主人,傳神辭也。}

\begin{quoting}蠲,魯詩作圭。孝、享雙聲,二字同義。\textbf{馬瑞辰}釋詁「畀,予也」,畀與卜雙聲,卜訓予者,或即畀之假借。\end{quoting}

\textbf{神之弔矣,詒爾多福。}{\footnotesize 弔至、詒遺也。箋云神至者,宗廟致敬,鬼神著矣,此之謂也。}\textbf{民之質矣,日用飲食。}{\footnotesize 質,成也。箋云成,平也,民事平,以禮飲食,相燕樂而已。}\textbf{群黎百姓,徧為爾德。}{\footnotesize 百姓,百官族姓也。箋云黎,眾也,群眾百姓徧為女之德,言則而象之。}

\begin{quoting}韋昭「百姓,百官,受氏姓也」,此處與群黎相對。\textbf{馬瑞辰}為當讀如「式訛爾心」之訛,訛,化也,徧為爾德,猶云徧化爾德也,為與化古皆讀如譌,故為、訛、化古並通用。\end{quoting}

\textbf{如月之恆,如日之升。}{\footnotesize 恆弦、升出也,言俱進也。箋云月上弦而就盈,日始出而就明。}\textbf{如南山之壽,不騫不崩。}{\footnotesize 騫,虧也。}\textbf{如松柏之茂,無不爾或承。}{\footnotesize 箋云或之言有也。如松柏之枝葉常茂盛青青,相承無衰落也。}

\begin{quoting}恆 \texttt{gēng},釋文「亦作緪」,說文「緪,大索也」。或,語詞。\end{quoting}

\section{采薇}

%{\footnotesize 六章、章八句}

\textbf{采薇,遣戍役也。文王之時,西有昆夷之患,北有玁狁之難,以天子之命命將率、遣戍役以守衛中國,故歌采薇以遣之,出車以勞還,杕杜以勤歸也。}{\footnotesize 文王為西伯服事殷之時也。昆夷,西戎也。天子,殷王也。戍,守也。西伯以殷王之命,命其屬為將率,將戍役禦西戎及北狄之難,歌采薇以遣之。杕杜勤歸者,以其勤勞之故,於其歸,歌杕杜以休息之。}

\textbf{采薇采薇,薇亦作止。}{\footnotesize 薇菜、作生也。箋云西伯將遣戍役,先與之期以采薇之時,今薇生矣,先輩可以行也。重言采薇者,丁寧行期也。}\textbf{曰歸曰歸,歲亦莫止。}{\footnotesize 箋云莫,晚也。曰女何時歸乎,何時歸乎,亦歲晚之時乃得歸也,又丁寧歸期,定其心也。}\textbf{靡室靡家,玁狁之故。不遑啟居,玁狁之故。}{\footnotesize 玁狁,北狄也。箋云北狄,今匈奴也。靡無、遑暇、啟跪也。古者師出不踰時,今薇生而行,歲晚乃得歸,使女無室家夫婦之道、不暇跪居者,有玁狁之難,故曉之也。}

\textbf{采薇采薇,薇亦柔止。}{\footnotesize 柔,始生也。箋云柔謂脆脕之時。}\textbf{曰歸曰歸,心亦憂止。}{\footnotesize 箋云憂止者,憂其歸期將晚。}\textbf{憂心烈烈,載飢載渴。}{\footnotesize 箋云烈烈,憂貌。則飢則渴,言其苦也。}\textbf{我戍未定,靡使歸聘。}{\footnotesize 聘,問也。箋云定,止也。我方守於北狄,未得止息,無所使歸問,言所以憂。}

\textbf{采薇采薇,薇亦剛止。}{\footnotesize 少而剛也。箋云剛謂少堅忍時。}\textbf{曰歸曰歸,歲亦陽止。}{\footnotesize 陽歷陽月也。箋云十月為陽,時坤用事,嫌於無陽,故以名此月為陽。}\textbf{王事靡盬,不遑啟處。}{\footnotesize 箋云盬,不堅固也。處,猶居也。}\textbf{憂心孔疚,我行不來。}{\footnotesize 疚病、來至也。箋云我,戍役自我也。來,猶反也,據家曰來。}

\begin{quoting}漢書五行志引左氏說,謂周六月、夏四月為正陽純乾之月。\end{quoting}

\textbf{彼爾維何,維常之華。}{\footnotesize 爾,華盛貌。常,常棣也。箋云此言彼爾者乃常棣之華,以興將率車馬服飾之盛。}\textbf{彼路斯何,君子之車。}{\footnotesize 箋云斯,此也。君子謂將率。}\textbf{戎車既駕,四牡業業。}{\footnotesize 業業然壯也。}\textbf{豈敢定居,一月三捷。}{\footnotesize 捷,勝也。箋云定,止也。將率之志,往至所征之地,不敢止而居處自安也,往則庶乎一月之中三有勝功,謂侵也、伐也、戰也。}

\begin{quoting}爾,同薾,說文「薾,華盛貌」。路,同輅。\end{quoting}

\textbf{駕彼四牡,四牡騤騤。君子所依,小人所腓。}{\footnotesize 騤騤,彊也。腓,辟也。箋云腓,當作芘。此言戎車者,將率之所依乘,戍役之所芘倚。}\textbf{四牡翼翼,象弭魚服。}{\footnotesize 翼翼,閑也。象弭,弓反末也,所以解紒也。魚服,魚皮也。箋云弭,弓反末彆者,以象骨為之,以助御者解轡紒,宜滑也。服,矢服也。}\textbf{豈不日戒,玁狁孔棘。}{\footnotesize 箋云戒,警敕軍事也。孔甚、棘急也。言君子小人豈不日相警戒乎,誠日相警戒也,玁狁之難甚急,豫述其苦以勸之。}

\begin{quoting}\textbf{陳奐}君子所依,謂依於車中者也,依猶倚也。翼翼,閑也,爾雅「閑,習也」,謂訓練有素而戰陣整齊。爾雅釋器「弓有緣者謂之弓,無緣者謂之弭 \texttt{mǐ}」,既夕禮疏引孫炎注「緣謂繳束而漆之,弭謂不以繳束,骨飾兩頭者也」。服,同箙,盛矢器也。\end{quoting}

\textbf{昔我往矣,楊柳依依。今我來思,雨雪霏霏。}{\footnotesize 楊柳,蒲柳也。霏霏,甚也。箋云我來戍止,而謂始反時也。上三章言戍役,次二章言將率之行,故此章重序其往反之時,極言其苦以說之。}\textbf{行道遲遲,載渴載飢。}{\footnotesize 遲遲,長遠也。箋云行反在於道路猶飢渴,言至苦也。}\textbf{我心傷悲,莫知我哀。}{\footnotesize 君子能盡人之情,故人忘其死。}

\begin{quoting}\textbf{胡承珙}爾雅祇以柳為大名,曰檉、曰旄、曰楊,其種各異,古人言楊柳者,謂名楊之柳。\textbf{馬瑞辰}韓詩薛君章句曰「依依,盛貌」,毛詩無傳,據車舝詩「依彼平林」傳「依,茂木貌」,則依依亦當訓盛,與韓詩同,依、殷古同聲,依依猶殷殷,殷亦盛也。\end{quoting}

\section{出車}

%{\footnotesize 六章、章八句}

\textbf{出車,勞還率也。}{\footnotesize 遣將率及戍役,同歌同時,欲其同心也,反而勞之,異歌異日,殊尊卑也,禮記曰「賜君子小人不同日」,此其義也。}

\begin{quoting}\textbf{王國維}鬼方昆夷玁狁考曰出車咏南仲伐玁狁之事,南仲亦見大雅常武篇,則南仲自是宣王時人,出車亦宣王時詩也。徵之古器,則凡紀玁狁事者,亦皆宣王時器。\end{quoting}

\textbf{我出我車,于彼牧矣。}{\footnotesize 出車就馬於牧地。箋云上我,我殷王也,下我,將率自謂也。西伯以天子之命,出我戎車於所牧之地,將使我出征伐。}\textbf{自天子所,謂我來矣。}{\footnotesize 箋云自,從也,有人從王所來,謂我來矣,謂以王命召己,將使為將率也。先出戎車,乃召將率,將率尊也。}\textbf{召彼僕夫,謂之載矣。王事多難,維其棘矣。}{\footnotesize 僕夫,御夫也。箋云棘,急也。王命召己,己即召御夫,使裝載物而往,王之事多難,其召我必急,欲疾趨之,此序其忠敬也。}

\begin{quoting}于,往。\textbf{馬瑞辰}廣雅「謂,使也」,謂我來,即使我來,下文謂之載,即使之載也。\end{quoting}

\textbf{我出我車,于彼郊矣。設此旐矣,建彼旄矣。}{\footnotesize 龜蛇曰旐。旄,干旄。箋云設旐者,屬之於干旄而建之戎車,將率既受命,行乃乘焉。牧地在遠郊。}\textbf{彼旟旐斯,胡不旆旆。}{\footnotesize 鳥隼曰旟。旆旆,旒垂貌。}\textbf{憂心悄悄,僕夫況瘁。}{\footnotesize 箋云況,茲也。將率既受命,行而憂,臨事而懼也。御夫則茲益憔悴,憂其馬之不正。}

\begin{quoting}\textbf{馬瑞辰}說文「況,寒水也」,因通為寒苦之稱,苦亦病也,況、瘁皆為病。\end{quoting}

\textbf{王命南仲,往城于方。出車彭彭,旂旐央央。}{\footnotesize 王,殷王也。南仲,文王之屬。方,朔方,近玁狁之國也。彭彭,四馬貌。交龍為旂。央央,鮮明也。箋云王使南仲為將率,往築城於朔方,為軍壘以禦北狄之難。}\textbf{天子命我,城彼朔方。赫赫南仲,玁狁于襄。}{\footnotesize 朔方,北方也。赫赫,盛貌。襄,除也。箋云此我,我戍役也,戍役築壘而美其將率自此出征也。}

\begin{quoting}南仲,亦作南中、張仲,六月鄭箋「張仲,吉甫之友,其性孝友」。方,即六月「侵鎬及方」之方。襄,釋文「本或作攘」。\end{quoting}

\textbf{昔我往矣,黍稷方華。今我來思,雨雪載塗。王事多難,不遑啟居。}{\footnotesize 塗,凍釋也。箋云黍稷方華,朔方之地六月時也,以此時始出壘征伐玁狁,因伐西戎,至春凍始釋而來反,其間非有休息。}\textbf{豈不懷歸,畏此簡書。}{\footnotesize 簡書,戒命也,鄰國有急,以簡書相告,則奔命救之。}

\textbf{喓喓草蟲,趯趯阜螽。}{\footnotesize 箋云草蟲鳴,阜螽躍而從之,天性也,喻近西戎之諸侯聞南仲既征玁狁、將伐西戎之命,則跳躍而鄉望之,如阜螽之聞草蟲鳴焉。草蟲鳴,晚秋之時也,此以其時所見而興之。}\textbf{未見君子,憂心忡忡。既見君子,我心則降。}{\footnotesize 箋云君子,斥南仲也。降,下也。}\textbf{赫赫南仲,薄伐西戎。}

\begin{quoting}此為晚秋時節。\end{quoting}

\textbf{春日遲遲,卉木萋萋。倉庚喈喈,采蘩祁祁。執訊獲醜,薄言還歸。}{\footnotesize 卉,草也。訊,辭也。箋云訊言、醜眾也。伐西戎以凍釋時,反朔方之壘息戍役,至此時而歸京師,稱美時物以及其事,喜而詳之也。執其可言問所獲之眾以歸者,當獻之也。}\textbf{赫赫南仲,玁狁于夷。}{\footnotesize 夷,平也。箋云平者,平之於王也。此時亦伐西戎,獨言平玁狁者,玁狁大,故以為始以為終。}

\begin{quoting}\textbf{陳奐}倉庚、采蘩,二月時也。訊,金文象繫縛之人,即俘虜,執訊言俘生,獲醜言殺死。\end{quoting}

\section{杕杜}

%{\footnotesize 四章、章七句}

\textbf{杕杜,勞還役也。}{\footnotesize 役,戍役也。}

\textbf{有杕之杜,有睆其實。}{\footnotesize 興也。睆,實貌。杕杜猶得其時蕃滋,役夫勞苦,不得盡其天性。}\textbf{王事靡盬,繼嗣我日。}{\footnotesize 箋云嗣,續也。王事無不堅固,我行役續嗣其日,言常勞苦無休息。}\textbf{日月陽止,女心傷止,征夫遑止。}{\footnotesize 箋云十月為陽。遑,暇也。婦人思望其君子,陽月之時已憂傷矣,征夫如今已閒暇且歸也,而尚不得歸,故序其男女之情以說之。陽月而思望之者,以初時云「歲亦莫止」。}

\begin{quoting}\textbf{馬瑞辰}蓋以春行,至杕杜成實,已近秋矣,過期不返,故曰繼嗣我日。\end{quoting}

\textbf{有杕之杜,其葉萋萋。王事靡盬,我心傷悲。}{\footnotesize 箋云傷悲者,念其君子於今勞苦。}\textbf{卉木萋止,女心悲止,征夫歸止。}{\footnotesize 室家踰時則思。}

\begin{quoting}此蓋次年春時。\end{quoting}

\textbf{陟彼北山,言采其杞。王事靡盬,憂我父母。}{\footnotesize 箋云杞非常菜也,而升北山采之,託有事以望君子。}\textbf{檀車幝幝,四牡痯痯,征夫不遠。}{\footnotesize 檀車,役車也。幝幝,敝貌。痯痯,罷貌。箋云不遠者,言其來,喻路近。}

\begin{quoting}\textbf{朱熹}登山采杞,則春已暮而杞可食矣。檀木堅,古人用以製輪。幝幝 \texttt{chǎn},車敝而緩。痯痯 \texttt{guǎn}。\end{quoting}

\textbf{匪載匪來,憂心孔疚。}{\footnotesize 箋云匪非、疚病也。君子至期不裝載,意不為來,我念之,憂心甚病。}\textbf{期逝不至,而多為恤。}{\footnotesize 逝往、恤憂也。遠行不必如期,室家之情以期望之。}\textbf{卜筮偕止,會言近止,征夫邇止。}{\footnotesize 卜之筮之,會人占之。邇,近也。箋云偕俱、會合也。或卜之,或筮之,俱占之,合言於繇為近,征夫如今近耳。}

\begin{quoting}而多為恤,即而恤為多。\end{quoting}

\section{魚麗}

%{\footnotesize 六章、三章章四句、三章章二句}

\textbf{魚麗,美萬物盛多,能備禮也。文武以天保以上治內,采薇以下治外,始於憂勤,終於逸樂,故美萬物盛多,可以告於神明矣。}{\footnotesize 內謂諸夏也,外謂夷狄也。告於神明者,於祭祀而歌之。}

\textbf{魚麗于罶,鱨鯊。}{\footnotesize 麗,歷也。罶,曲梁也,寡婦之笱也。鱨,揚也。鯊,鮀也。太平而後,微物眾多,取之有時,用之有道,則物莫不多矣。古者不風不暴,不行火,草木不折,不操斧斤,不入山林,豺祭獸然後殺,獺祭魚然後漁,鷹隼擊然後罻羅設,是以天子不合圍,諸侯不掩群,大夫不麛不卵,士不隱塞,庶人不數罟,罟必四寸然後入澤梁,故山不童,澤不竭,鳥獸魚鱉皆得其所然。}\textbf{君子有酒,旨且多。}{\footnotesize 箋云酒美而此魚又多也。}

\begin{quoting}\textbf{陳奐}言魚在罶錄錄歷歷然也。\textbf{馬瑞辰}旨且多、多且旨、旨且有,自專指酒言之。\end{quoting}

\textbf{魚麗于罶,魴鱧。}{\footnotesize 鱧,鮦也。}\textbf{君子有酒,多且旨。}{\footnotesize 箋云酒多而此魚又美也。}

\textbf{魚麗于罶,鰋鯉。}{\footnotesize 鰋,鮎也。}\textbf{君子有酒,旨且有。}{\footnotesize 箋云酒美而此魚又有。}

\textbf{物其多矣,維其嘉矣。}{\footnotesize 箋云魚既多又善。}

\textbf{物其旨矣,維其偕矣。}{\footnotesize 箋云魚既美又齊等。}

\begin{quoting}\textbf{經義述聞}廣雅曰「皆,嘉也」,皆與偕古字通。\end{quoting}

\textbf{物其有矣,維其時矣。}{\footnotesize 箋云魚既有又得其時。}

\begin{quoting}頍弁「爾肴既時」,傳「時,善也」,則嘉、偕、時均訓善。\end{quoting}

\section*{南陔·白華·華黍}

\textbf{南陔,孝子相戒以養也。白華,孝子之絜白也。華黍,時和歲豐,宜黍稷也。有其義而亡其辭。}{\footnotesize 此三篇者,鄉飲酒、燕禮用焉,曰「笙入,立于縣,時奏南陔、白華、華黍」是也。孔子論詩,雅頌各得其所,時俱在耳,篇第當在於此,遭戰國及秦之世而亡之,其義則與眾篇之義合編,故存。至毛公為詁訓傳,乃分眾篇之義,各置於其篇端,云又推其亡者,以見在為數,故推改什首,遂通耳,而下非孔子之舊。}

%\begin{flushright}鹿鳴之什十篇、五十五章、三百一十五句\end{flushright}