\chapter{節南山之什詁訓傳第十九}

\begin{quoting}\textbf{釋文}從此至何草不黃凡四十四篇,前儒申毛皆以為幽王之變小雅,鄭以十月之交以下四篇是厲王之變小雅,漢興之初,師移其篇次,毛為詁訓,因改其第焉。\end{quoting}

\section{節南山}

%{\footnotesize 十章、六章章八句、四章章四句}

\textbf{節南山,家父刺幽王也。}{\footnotesize 家父,字,周大夫也。}

\begin{quoting}此詩古唯以「節」名篇,如左傳昭二年「季武子賦節之卒章」,十月之交鄭箋「節刺師尹不平」,大戴記盧辯注「小雅節之四章」。\textbf{姚際恆}謂南山即終南山,故詩為東遷前作品。\textbf{胡承珙}其所以刺尹氏者,大要有二事,為政不平而委任小人也,又詩詞專責尹氏,而刺王之旨自在言外。\end{quoting}

\textbf{節彼南山,維石巖巖。}{\footnotesize 興也。節,高峻貌。巖巖,積石貌。箋云興者,喻三公之位人所尊嚴。}\textbf{赫赫師尹,民具爾瞻。憂心如惔,不敢戲談。}{\footnotesize 赫赫,顯盛貌。師,大師,周之三公也。尹,尹氏,為大師。具俱、瞻視、惔燔也。箋云此言尹氏,女居三公之位,天下之民俱視女之所為,皆憂心如火灼爛之矣,又畏女之威,不敢相戲而言語,疾其貪暴,脅下以形辟也。}\textbf{國既卒斬,何用不監。}{\footnotesize 卒盡、斬斷、監視也。箋云天下之諸侯日相侵伐,其國已盡絕滅,女何用為職不監察之。}

\begin{quoting}節,同嶻,段注「嶻嶭,嵯峨語音之轉」。惔 \texttt{tán},釋文「韓詩作炎」。何用,何以。\end{quoting}

\textbf{節彼南山,有實其猗。}{\footnotesize 實滿、猗長也。箋云猗,倚也。言南山既能高峻,又以草木平滿其旁倚之畎谷,使之齊均也。}\textbf{赫赫師尹,不平謂何。}{\footnotesize 箋云責三公之不均平,不如山之為也。謂何,猶云何也。}\textbf{天方薦瘥,喪亂弘多。}{\footnotesize 薦重、瘥病、弘大也。箋云天氣方今又重以疫病,長幼相亂而死喪甚大多也。}\textbf{民言無嘉,憯莫懲嗟。}{\footnotesize 憯,曾也。箋云懲,止也。天下之民皆以災害相弔唁,無一嘉慶之言,曾無以恩德止之者,嗟乎奈何。}

\begin{quoting}\textbf{經義述聞}詩常例,凡言有蕡其實、有鶯其語、有略其耜、有捄其角,末一字皆實指其物,有實其猗文義亦然也,猗疑當讀為阿,猗、阿二字通用,實,廣大貌,閟宮篇「實實枚枚」,傳曰「實實,廣大也」,有實其猗者,言南山之阿實然廣大也。\textbf{馬瑞辰}爾雅「偏高曰阿丘」,阿為偏高不平之地,故詩以興師尹之不平耳。瘥 \texttt{cuó}。憯 \texttt{cǎn}。嗟,語詞。\end{quoting}

\textbf{尹氏大師,維周之氐。秉國之均,四方是維。天子是毗,俾民不迷。}{\footnotesize 氐本、均平、毗厚也。箋云氐,當作桎轄之桎。毗,輔也。言尹氏作大師之官,為周之桎轄,持國政之平,維制四方,上輔天子,下教化天下,使民無迷惑之憂,言任至重。}\textbf{不弔昊天,不宜空我師。}{\footnotesize 弔至、空窮也。箋云至,猶善也,不善乎昊天,愬之也,不宜使此人居尊官,困窮我之眾民也。}

\begin{quoting}氐,通柢,爾雅「柢,本也」。毗,釋文「王作埤」,說文「埤,增也,增,益也」。\textbf{林義光}不弔,不淑也,金文叔字皆借弔字為之。\end{quoting}

\textbf{弗躬弗親,庶民弗信。弗問弗仕,勿罔君子。}{\footnotesize 庶民之言不可信,勿罔上而行也。箋云仕,察也。勿,當作未。此言王之政不躬而親之,則恩澤不信於眾民矣,不問而察之,則下民未罔其上矣。}\textbf{式夷式已,無小人殆。}{\footnotesize 式用、夷平也。用平則已,無以小人之言至於危殆也。箋云殆,近也。為政當用平正之人,用能紀理其事者,無小人近。}\textbf{瑣瑣姻亞,則無膴仕。}{\footnotesize 瑣瑣,小貌。兩壻相謂曰亞。膴,厚也。箋云壻之父曰姻。瑣瑣昏姻妻黨之小人,無厚任用之,置之大位,重其祿也。}

\begin{quoting}躬、親二字同義。庶民與君子相對。\textbf{姚際恆}以君子而弗咨詢之,弗仕使之,是誣罔君子也,故戒其勿。\textbf{馬瑞辰}夷謂平其心,即下章「君子如夷」也,已謂知所止,即下章「君子如屆」也。亞,即連襟。膴 \texttt{wǔ}。\end{quoting}

\textbf{昊天不傭,降此鞠訩。昊天不惠,降此大戾。}{\footnotesize 傭均、鞠盈、訩訟也。箋云盈,猶多也。戾,乖也。昊天乎,師氏為政不均,乃下此多訟之俗,又為不和順之行,乃下此乖爭之化,疾時民俲為之,愬之於天。}\textbf{君子如屆,俾民心闋。君子如夷,惡怒是違。}{\footnotesize 屆極、闋息、夷易、違去也。箋云屆,至也。君子,斥在位者。如行至誠之道,則民鞠訩之心息,如行平易之政,則民乖爭之情去,言民之失由於上可反復也。}

\begin{quoting}鞠,窮也、極也。\textbf{馬瑞辰}訩當讀如「日月吿凶」之凶,謂凶咎也,鞠凶猶言極凶,與大戾同義,故皆為天所降。\end{quoting}

\textbf{不弔昊天,亂靡有定。式月斯生,俾民不寧。憂心如酲,誰秉國成。}{\footnotesize 病酒曰酲。成,平也。箋云弔,至也,至,猶善也。定止、式用也。不善乎昊天,天下之亂無肯止之者,用月此生,言月月益甚也,使民不得安,我今憂之,如病酒之酲矣,觀此君臣,誰能持國之平乎,言無有也。}\textbf{不自為政,卒勞百姓。}{\footnotesize 箋云卒,終也。昊天不自出政教,則終窮苦百姓,欲使昊天出圖書有所授命,民乃得安。}

\begin{quoting}\textbf{馬瑞辰}秉國成即執國政也。卒,同瘁。\end{quoting}

\textbf{駕彼四牡,四牡項領。}{\footnotesize 項,大也。箋云四牡者,人君所乘駕,今但養大其領,不肯為用,喻大臣自恣,王不能使。}\textbf{我瞻四方,蹙蹙靡所騁。}{\footnotesize 騁,極也。箋云蹙蹙,縮小之貌。我視四方土地日見侵削於夷狄蹙蹙然,雖欲馳騁,無所之也。}

\begin{quoting}項,肥大,項領,指馬久不行,頸則肥大也。\end{quoting}

\textbf{方茂爾惡,相爾矛矣。}{\footnotesize 茂,勉也。箋云相,視也。方爭訟自勉於惡之時,則視女矛矣,言欲戰鬪相殺傷矣。}\textbf{既夷既懌,如相醻矣。}{\footnotesize 懌,服也。箋云夷,說也。言大臣之乖爭本無大讎,其已相和順而說懌,則如賓主飲酒相醻酢也。}

\begin{quoting}茂,盛也。此章言小人之交,謂尹氏等。\end{quoting}

\textbf{昊天不平,我王不寧。不懲其心,覆怨其正。}{\footnotesize 正,長也。箋云昊天乎,師尹為政不平,使我王不得安寧,女不懲止女之邪心而反怨憎其正也。}

\textbf{家父作誦,以究王訩。}{\footnotesize 家父,大夫也。箋云究,窮也。大夫家父作此詩而為王誦之,以窮極王之政所以致多訟之本意。}\textbf{式訛爾心,以畜萬邦。}{\footnotesize 箋云訛化、畜養也。}

\begin{quoting}家父,三家詩作嘉父,\textbf{陳奐}食采于家,以邑為氏者也,何注公羊傳云「家,采地,父,字」是也。\end{quoting}

\section{正月}

%{\footnotesize 十三章、八章章八句、五章章六句}

\textbf{正月,大夫刺幽王也。}

\begin{quoting}\textbf{孔穎達}詩人明得失之迹,見微知著,以褒姒淫妬,知其必滅周也。\end{quoting}

\textbf{正月繁霜,我心憂傷。}{\footnotesize 正月,夏之四月。繁,多也。箋云夏之四月,建巳之月,純陽用事而霜多,急恆寒若之異,傷害萬物,故心為之憂傷。}\textbf{民之訛言,亦孔之將。}{\footnotesize 將,大也。箋云訛,偽也。人以偽言相陷入,使王行酷暴之刑,致此災異,故言亦甚大也。}\textbf{念我獨兮,憂心京京。哀我小心,癙憂以痒。}{\footnotesize 京京,憂不去也。癙、痒皆病也。箋云念我獨兮者,言我獨憂此政也。}

\begin{quoting}周之正月為夏曆十一月,\textbf{高亨}經文與傳文「正」均當作四,形似而誤。癙,亦作鼠,雨無正曰「鼠思泣血」,癙憂,鬱悶。以,而。說文「痒,瘍也,瘍,創也」。\end{quoting}

\textbf{父母生我,胡俾我瘉。不自我先,不自我後。}{\footnotesize 父母,謂文武也。我,我天下。瘉,病也。箋云自,從也。天使父母生我,何故不長遂我,而使我遭此暴虐之政而病,此何不出我之前、居我之後,窮苦之情,苟欲免身。}\textbf{好言自口,莠言自口。}{\footnotesize 莠,醜也。箋云自,從也。此疾訛言之人,善言從女口出,惡言亦從女口出,女口一爾,善也惡也同出其中,謂其可賤。}\textbf{憂心愈愈,是以有侮。}{\footnotesize 愈愈,憂懼也。箋云我心憂政如是,是與訛言者殊塗,故用是見侵侮也。}

\textbf{憂心惸惸,念我無祿。}{\footnotesize 惸惸,憂意也。箋云無祿者,言不得天祿,自傷值今生也。}\textbf{民之無辜,并其臣僕。}{\footnotesize 古者有罪不入於刑則役之圜土,以為臣僕。箋云辜,罪也。人之尊卑有十等,僕第九,臺第十。言王既刑殺無罪,并及其家之賤者,不止於所罪而已,書曰「越茲麗刑并制」。}\textbf{哀我人斯,于何從祿。}{\footnotesize 箋云斯此、于於也。哀乎,今我民人見遇如此,當於何從得天祿,免於是難。}\textbf{瞻烏爰止,于誰之屋。}{\footnotesize 富人之屋,烏所集也。箋云視烏集於富人之屋,以言今民亦當求明君而歸之。}

\begin{quoting}\textbf{馬瑞辰}古以罪人為臣僕,詩云「并其臣僕」,謂使無罪者并為臣僕,在罪人之列。管錐編引張穆㐆齋文集云「烏者,周家受命之祥,周之臣民相傳以熟」。\end{quoting}

\textbf{瞻彼中林,侯薪侯蒸。}{\footnotesize 中林,林中也。薪、蒸,言似而非。箋云侯,維也。林中大木之處而維有薪蒸爾,喻朝廷宜有賢者而但聚小人也。}\textbf{民今方殆,視天夢夢。}{\footnotesize 王者為亂夢夢然。箋云方,且也。民今且危亡,視王者所為反夢夢然而亂無統理安人之意。}\textbf{既克有定,靡人弗勝。}{\footnotesize 勝,乘也。箋云王既能有所定,尚復事之小者爾,無人而不勝,言凡人所定皆勝王也。}\textbf{有皇上帝,伊誰云憎。}{\footnotesize 皇,君也。箋云伊,讀當為繄,繄,猶是也。有君上帝者,以情告天也,使王暴虐如是,是憎惡誰乎,欲天指害其所憎而已。}

\begin{quoting}\textbf{馬瑞辰}言天如有止亂之心,則此訛言之小人無不能勝之者,乃天能勝人而不肯止亂,不知天意果誰憎乎。\end{quoting}

\textbf{謂山蓋卑,為岡為陵。}{\footnotesize 在位非君子,乃小人也。箋云此喻為君子賢者道,人尚謂之卑,況為凡庸小人之行。}\textbf{民之訛言,寧莫之懲。}{\footnotesize 箋云小人在位,曾無欲止眾民之為偽言相陷害也。}\textbf{召彼故老,訊之占夢。}{\footnotesize 故老,元老。訊,問也。箋云君臣在朝,侮慢元老,召之不問政事,但問占夢,不尚道德而信徵祥之甚。}\textbf{具曰予聖,誰知烏之雌雄。}{\footnotesize 君臣俱自謂聖也。箋云時君臣賢愚適同,如烏雌雄相似,誰能別異之乎。}

\begin{quoting}蓋,同盍,何也,下章同。\textbf{馬瑞辰}詩意蓋謂訛言以山為卑,而其實乃為高陵,以證其言之不實,故繼以「民之訛言,寧莫之懲」。\textbf{朱熹}占夢,官名,掌占夢者也。召、訊互文。說文「聖,通也」,尚書洪範傳「於事無不通謂之聖」。\end{quoting}

\textbf{謂天蓋高,不敢不局。謂地蓋厚,不敢不蹐。維號斯言,有倫有脊。}{\footnotesize 局,曲也。蹐,累足也。倫道、脊理也。箋云局蹐者,天高而有雷霆,地厚而有陷淪也,此民疾苦王政上下皆可畏怖之言也,維民號呼而發此言,皆有道理所以至然者,非徒苟妄為誣辭。}\textbf{哀今之人,胡為虺蜴。}{\footnotesize 蜴,螈也。箋云虺蜴之性,見人則走,哀哉,今之人何為如是,傷時政也。}

\begin{quoting}局,釋文「本又作跼」,傴 \texttt{yǔ} 僂。說文「蹐,小步也」。脊,同迹。\textbf{朱熹}哀今之人,胡為肆毒以害人,而使之至此乎。\end{quoting}

\textbf{瞻彼阪田,有菀其特。}{\footnotesize 言朝廷曾無桀臣。箋云阪田,崎嶇墝埆之處,而有菀然茂特之苗,喻賢者在間辟隱居之時。}\textbf{天之扤我,如不我克。}{\footnotesize 扤,動也。箋云我,我特苗也。天以風雨動搖我,如將不勝我,謂其迅疾也。}\textbf{彼求我則,如不我得。}{\footnotesize 箋云彼,王也,王之始徵求我,如恐不得我,言其禮命之繁多。}\textbf{執我仇仇,亦不我力。}{\footnotesize 仇仇,猶謷謷也。箋云王既得我,執留我,其禮待我謷謷然,亦不問我在位之功力,言其有貪賢之名,無用賢之實。}

\begin{quoting}菀,同鬱,茂盛貌。扤 \texttt{yuè},同抈,說文「抈,折也」。則,語末助詞。仇,同扏,緩持、不固。禮記緇衣引此章後四句,鄭注「言君始求我,如恐不得,既得我,執我仇仇然不堅固,亦不力用我,是不親信我也」。\end{quoting}

\textbf{心之憂矣,如或結之。今茲之正,胡然厲矣。}{\footnotesize 厲,惡也。箋云茲此、正長也。心憂如有結之者,憂今此之君臣何一然為惡如是。}\textbf{燎之方揚,寧或滅之。}{\footnotesize 滅之以水也。箋云火田為燎。燎之方盛之時,炎熾熛怒,寧有能滅息之者,言無有也,以無有,喻有之者為甚也。}\textbf{赫赫宗周,襃姒烕之。}{\footnotesize 宗周,鎬京也。褒,國也。姒,姓也。烕,滅也。有褒國之女,幽王惑焉而以為后,詩人知其必滅周也。}

\begin{quoting}\textbf{朱熹}正,政也,厲,暴惡也,言我心之憂如結者,為國政之暴惡也。\end{quoting}

\textbf{終其永懷,又窘陰雨。}{\footnotesize 窘,困也。箋云窘,仍也。終王之所行,其長可憂傷矣,又將仍憂於陰雨,陰雨喻君有泥陷之難。}\textbf{其車既載,乃棄爾輔。}{\footnotesize 大車重載,又棄其輔。箋云以車之載物喻王之任國事也,棄輔謂遠賢也。}\textbf{載輸爾載,將伯助予。}{\footnotesize 將請、伯長也。箋云輸,墮也。棄女車輔則墮女之載,乃請長者見助,以言國危而求賢者已晚矣。}

\begin{quoting}\textbf{陳奐}輔者掩輿之版,大東傳「箱,大車之箱也」,方言「箱謂之輫」,爾雅「棐,輔也」,棐與輫通,大車掩版置諸兩旁可以任載,今大車既重載矣,而又棄其兩旁之版,則所載必墮,此其顯喻也,車之有輔,興國之有輔臣。載,語首助詞,則也。將 \texttt{qiānɡ}。\end{quoting}

\textbf{無棄爾輔,員于爾輻。}{\footnotesize 員,益也。}\textbf{屢顧爾僕,不輸爾載。}{\footnotesize 箋云屢,數也。僕,將車者也。顧,猶視也、念也。}\textbf{終踰絕險,曾是不意。}{\footnotesize 箋云女不棄車之輔,數顧女僕,終用是踰度陷絕之險,女不曾以是為意乎,以商事喻治國。}

\begin{quoting}員 \texttt{yún}。\end{quoting}

\textbf{魚在于沼,亦匪克樂。潛雖伏矣,亦孔之炤。}{\footnotesize 沼,池也。箋云池魚之所樂而非能樂,其潛伏於淵又不足以逃,甚炤炤易見,以喻時賢者在朝廷,道不行無所樂,退而窮處又無所止也。}\textbf{憂心慘慘,念國之為虐。}{\footnotesize 慘慘,猶戚戚也。}

\textbf{彼有旨酒,又有嘉殽。}{\footnotesize 言禮物備也。箋云彼,彼尹氏大師也。}\textbf{洽比其鄰,昬姻孔云。}{\footnotesize 洽合、鄰近、云旋也。是言王者不能親親以及遠。箋云云,猶友也。言尹氏富,獨與兄弟相親友為朋黨也。}\textbf{念我獨兮,憂心慇慇。}{\footnotesize 慇慇然痛也。箋云此賢者孤特自傷也。}

\begin{quoting}洽,同協,洽協雙聲,說文「協,同眾之和也」。\end{quoting}

\textbf{佌佌彼有屋,蔌蔌方有穀。}{\footnotesize 佌佌,小也。蔌蔌,陋也。箋云穀,祿也。言小人富而窶陋將貴也。}\textbf{民今之無祿,天夭是椓。}{\footnotesize 君夭之,在位椓之。箋云民於今而無祿者,天以薦瘥夭殺之,是王者之政又復椓破之,言遇害甚也。}\textbf{哿矣富人,哀此惸獨。}{\footnotesize 哿可、獨單也。箋云此言王政如是,富人已可,惸獨將困也。}

\begin{quoting}釋文「方穀,本或作方有穀,非也」,\textbf{馬瑞辰}詩蓋以「佌佌彼有屋」與「民今之無祿」相對,以「蔌蔌方穀」與「天夭是椓」相對。天夭,或作夭夭,似可。\textbf{經義述聞}哿與哀相對為文,哀者憂悲,哿者歡樂也,言樂矣彼有屋之富人,悲哉此無祿之惸獨也。\end{quoting}

\section{十月之交}

%{\footnotesize 八章、章八句}

\textbf{十月之交,大夫刺幽王也。}{\footnotesize 當為刺厲王,作詁訓傳時移其篇第,因改之耳。節刺師尹不平,亂靡有定,此篇譏皇父擅恣,日月告凶,正月惡褒姒滅周,此篇疾艷妻煽方處,又幽王時,司徒乃鄭桓公友,非此篇之所云番也,是以知然。}

\textbf{十月之交,朔月辛卯。日有食之,亦孔之醜。}{\footnotesize 之交,日月之交會。醜,惡也。箋云周之十月,夏之八月也,八月朔日,日月交會而日食,陰侵陽、臣侵君之象。日辰之義,日為君,辰為臣,辛,金也,卯,木也,又以卯侵辛,甚惡也。}\textbf{彼月而微,此日而微。}{\footnotesize 月,臣道,日,君道。箋云微,謂不明也,彼月則有微,今此日反微,非其常,為異尤大也。}\textbf{今此下民,亦孔之哀。}{\footnotesize 箋云君臣失道,災害將起,故下民亦甚可哀。}

\begin{quoting}朔月,即月朔,初一日。阮元謂前人「並推定此日食在周幽王六年十月建酉,辛卯朔日入食限,載在史志」。\end{quoting}

\textbf{日月告凶,不用其行。四國無政,不用其良。}{\footnotesize 箋云告凶,告天下以凶亡之徵也。行,道度也,不用之者,謂相干犯也。四方之國無政治者,由天子不用善人也。}\textbf{彼月而食,則維其常。此日而食,于何不臧。}{\footnotesize 箋云臧,善也。}

\begin{quoting}用,由。而,猶之也。維,齊詩作惟,是。\textbf{馬瑞辰}考春秋經書日食三十有六,而月食則不書,此古人重日食而輕月食之證。于何不臧,孔疏「猶言一何不善,為不善之大」。\end{quoting}

\textbf{爗爗震電,不寧不令。}{\footnotesize 爗爗,震電貌。震,雷也。箋云雷電過常,天下不安,政教不善之徵。}\textbf{百川沸騰,山冢崒崩。}{\footnotesize 沸出、騰乘也。山頂曰冢。箋云崒者,崔嵬。百川沸出相乘陵者,由貴小人也,山頂崔嵬者崩,君道壞也。}\textbf{高岸為谷,深谷為陵。}{\footnotesize 言易位也。箋云易位者,君子居下、小人處上之謂也。}\textbf{哀今之人,胡憯莫懲。}{\footnotesize 箋云憯曾、懲止也。變異如此,禍亂方至,哀哉,今在位之人,何曾無以道德止之。}

\begin{quoting}爗 \texttt{yè}。崒,同碎,\textbf{馬瑞辰}碎崩與沸騰相對成文。國語周語「幽王二年,西周三川皆震,是歲也,三川竭,岐山崩」,三川,涇渭洛也。憯 \texttt{cǎn},同朁。\end{quoting}

\textbf{皇父卿士,番維司徒。家伯維宰,仲允膳夫。棸子內史,蹶維趣馬。楀維師氏,豔妻煽方處。}{\footnotesize 艷妻,褒姒,美色曰艷。煽,熾也。箋云皇父、家伯、仲允皆字,番、棸、蹶、楀皆氏,厲王淫於色,七子皆用,后嬖寵方熾之時,並處位,言妻黨盛,女謁行之甚也,敵夫曰妻。司徒之職掌天下土地之圖、人民之數,冢宰掌建邦之六典,皆卿也,膳夫,上士也,掌王之飲食膳羞,內史,中大夫也,掌爵祿廢置、殺生予奪之法,趣馬,中士也,掌王馬之政,師氏,亦中大夫也,掌司朝得失之事,六人之中,雖官有尊卑,權寵相連,朋黨於朝,是以疾焉。皇父則為之端首,兼擅群職,故但目以卿士云。}

\begin{quoting}棸 \texttt{zōu},齊詩作掫。楀 \texttt{jǔ}。\end{quoting}

\textbf{抑此皇父,豈曰不時。胡為我作,不即我謀。徹我牆屋,田卒汙萊。}{\footnotesize 時,是也。下則汙,高則萊。箋云抑之言噫,噫是皇父,疾而呼之,女豈曰我所為不是乎,言其不自知惡也,女何為役作我,不先就與我謀,使我得遷徙,乃反徹毀我牆屋,令我不得趨農田,卒為汙萊乎,此皇父所築邑人之怨辭。}\textbf{曰予不戕,禮則然矣。}{\footnotesize 箋云戕,殘也,言皇父既不自知不是,反云我不殘敗女田業,禮,下供上役,其道當然,言文過也。}

\begin{quoting}\textbf{馬瑞辰}時當讀如「使民以時」之時。\textbf{陳奐}此謂田盡不治則下者積水,高者薉草矣。\end{quoting}

\textbf{皇父孔聖,作都于向。擇三有事,亶侯多藏。}{\footnotesize 皇父甚自謂聖。向,邑也。擇三有事,有司,國之三卿,信維貪淫多藏之人也。箋云專權足己,自比聖人,作都立三卿,皆取聚斂之臣,言不知厭也。禮,畿內諸侯二卿。}\textbf{不憖遺一老,俾守我王。}{\footnotesize 箋云憖者,心不欲自強之辭也,言盡將舊在位之人與之皆去,無留衛王。}\textbf{擇有車馬,以居徂向。}{\footnotesize 箋云又擇民之富有車馬者,以往居于向也。}

\begin{quoting}周禮載師鄭注「家邑,大夫之采地,小都,卿之采地,大都,公之采地」。孔疏「三卿者,依周制而言,謂立司徒兼冢宰之事,立司馬兼宗伯之事,立司空兼司寇之事」。亶 \texttt{dǎn},信、確實。侯,維、是。憖 \texttt{yìn},願、肯。\textbf{于省吾}以居徂向即徂向以居,特倒文以與藏、王為韻耳。\end{quoting}

\textbf{黽勉從事,不敢告勞。}{\footnotesize 箋云詩人賢者見時如是,自勉以從王事,雖勞不敢自謂勞,畏刑罰也。}\textbf{無罪無辜,讒口嚻嚻。}{\footnotesize 箋云嚻嚻,眾多貌。時人非有辜罪,其被讒口見椓譖嚻嚻然。}\textbf{下民之孽,匪降自天。噂沓背憎,職競由人。}{\footnotesize 噂,猶噂噂,沓,猶沓沓。職,主也。箋云孽,妖孽,謂相為災害也。下民有此害,非從天隋也。噂噂沓沓,相對談語,背則相憎,逐為此者,主由人也。}

\begin{quoting}嚻 \texttt{áo}。噂沓 \texttt{zǔn tà},聚則相合,與背憎相對。\textbf{陳奐}由人與自天對文,職競由人,言不從天降,而主從人之競為惡也。\end{quoting}

\textbf{悠悠我里,亦孔之痗。}{\footnotesize 悠悠,憂也。里,病也。痗,病也。箋云里,居也。悠悠乎,我居今之世亦甚困病。}\textbf{四方有羨,我獨居憂。}{\footnotesize 羨,餘也。箋云四方之人盡有饒餘,我獨居此而憂。}\textbf{民莫不逸,我獨不敢休。}{\footnotesize 箋云逸,逸豫也。}\textbf{天命不徹,我不敢傚我友自逸。}{\footnotesize 徹,道也。親屬之臣,心不能已。箋云不道者,言王不循天之政教。}

\begin{quoting}里,同悝 \texttt{kuī},爾雅釋詁「悝,憂也」。居,語詞。\textbf{陳奐}爾雅釋訓「不遹 \texttt{yù}、不蹟、不徹,不道也」,傳釋徹為道正本爾雅,天命不道,謂天之令不循道而行,遂有日食震電之變。\textbf{姚際恆}我友自逸,皆指七子輩也。\end{quoting}

\section{雨無正}

%{\footnotesize 七章、二章章十句、二章章八句、三章章六句}

\textbf{雨無正,大夫刺幽王也。雨自上下者也,眾多如雨,而非所以為政也。}{\footnotesize 亦當為刺厲王。王之所下教令甚多而無正也。}

\begin{quoting}呂祖謙東塾讀詩記引董氏曰韓詩作雨無極,正大夫刺幽王也,章句曰「無,眾也」,書曰「庶草繁蕪」,說文曰「蕪,豐也」,則雨眾多者,其為政令不得一也,故為正大夫之刺。\end{quoting}

\textbf{浩浩昊天,不駿其德。降喪饑饉,斬伐四國。}{\footnotesize 駿,長也。穀不熟曰饑,蔬不熟曰饉。箋云此言王不能繼長昊天之德,至使昊天下此死喪饑饉之災,而天下諸侯於是更相侵伐。}\textbf{旻天疾威,弗慮弗圖。}{\footnotesize 箋云慮、圖皆謀也。王既不駿昊天之德,今昊天又疾其政,以刑罰威恐天下而不慮不圖。}\textbf{舍彼有罪,既伏其辜。若此無罪,淪胥以鋪。}{\footnotesize 舍除、淪率也。箋云胥相、鋪徧也。言王使此無罪者見牽率相引而徧得罪也。}

\begin{quoting}\textbf{馬瑞辰}廣雅「暴,疾也」,疾、威二字平列,朱子集傳云「疾威猶言暴虐」是也。\textbf{經義述聞}伏者,藏也、隱也,凡戮有罪者,當聲其罪而誅之,今王之舍彼有罪也,則既隱藏其罪而不之發矣,蓋惟其欲舍有罪之人,是以匿其罪狀耳。鋪,韓詩作痡,病也、痛也。\end{quoting}

\textbf{周宗既滅,靡所止戾。}{\footnotesize 戾,定也。箋云周宗,鎬京也。是時諸侯不朝王,民不堪命,王流于彘,無所安定也。}\textbf{正大夫離居,莫知我勩。}{\footnotesize 勩,勞也。箋云正,長也。長官之大夫於王流于彘而皆散處,無復知我民之見罷勞也。}\textbf{三事大夫,莫肯夙夜。邦君諸侯,莫肯朝夕。}{\footnotesize 箋云王流在外,三公及諸侯隨王而行者皆無君臣之禮,不肯晨夜朝莫省王也。}\textbf{庶曰式臧,覆出為惡。}{\footnotesize 覆,反也。箋云人見王之失所,庶幾其自改悔而用善人,反出教令,復為惡也。}

\begin{quoting}周宗,馬瑞辰以為當是「宗周」誤倒,左傳昭十六年引詩正作宗周。勩 \texttt{yì},昭十六年引詩作肄,杜注「肄,勞也」。\textbf{陳奐}三公大夫,言內也,邦君諸侯,言外也。爾雅「庶,幸也」,表希望。\end{quoting}

\textbf{如何昊天,辟言不信。如彼行邁,則靡所臻。}{\footnotesize 辟,法也。箋云如何乎昊天,痛而愬之也,為陳法度之言不信之也,我之言不見信,如行而無所至也。}\textbf{凡百君子,各敬爾身。胡不相畏,不畏于天。}{\footnotesize 箋云凡百君子,謂眾在位者,各敬慎女之身,正君臣之禮,何為上下不相畏乎,上下不相畏者,是不畏于天者也。}

\textbf{戎成不退,飢成不遂。曾我暬御,憯憯日瘁。}{\footnotesize 戎兵、遂安也。暬御,侍御也。瘁,病也。箋云兵成而不退,謂王見流于彘,無御止之者,飢成而不安,謂王在彘乏於飲食之蓄,無輸粟歸餼者,此二者曾但侍御左右小臣憯憯憂之,大臣無念之者。}\textbf{凡百君子,莫肯用訊。聽言則答,譖言則退。}{\footnotesize 以言進退人也。箋云訊,告也,眾在位者無肯用此相告語者,言不憂王之事也。答,猶距也。有可聽用之言,則共以辭距而違之,有譖毀之言,則共為排退之,群臣並為不忠,惡直醜正。}

\begin{quoting}\textbf{陳奐}楚語「居寢有暬御之箴」,韋注云「暬 \texttt{xiè},近也」。憯憯 \texttt{cǎn},唐石經作慘慘,憂傷貌。戴震「訊乃誶 \texttt{suì} 字轉寫之譌,誶吿,訊問,聲義不相通借」,魯詩正作誶。\textbf{馬瑞辰}廣韻「憯,毀也」,毀猶謗也,古以諫言為誹謗,故堯有誹謗之木,憯言即諫言也。\end{quoting}

\textbf{哀哉不能言,匪舌是出,維躬是瘁。}{\footnotesize 哀賢人不得言,不得出是舌也。箋云瘁,病也。不能言,言之拙也。言非可出於舌,其身旋見困病。}\textbf{哿矣能言,巧言如流,俾躬處休。}{\footnotesize 哿,可也,可矣,世所謂能言也,巧言從俗,如水轉流。箋云巧,猶善也,謂以事類風切剴微之言,如水之流,忽然而過,故不悖逆,使身居安休休然。亂世之言,順說為上。}

\textbf{維曰于仕,孔棘且殆。云不可使,得罪于天子。亦云可使,怨及朋友。}{\footnotesize 于,往也。箋云棘,急也。不可使者,不正不從也,可使者,雖不正從也。居今衰亂之世,云往仕乎,甚急迮且危,急迮且危,以此二者也。}

\textbf{謂爾遷于王都,曰予未有室家。}{\footnotesize 賢者不肯遷于王都也。箋云王流于彘,正大夫離居,同姓之臣從王,思其友而呼之,謂曰「女今可遷居王都」,謂彘也,其友辭之云「我未有室家於王都可居也」。}\textbf{鼠思泣血,無言不疾。}{\footnotesize 無聲曰泣血。無所言而不見疾也。箋云鼠,憂也。既辭之以無室家,為其意恨,又患不能距止之,故云我憂思泣血,欲遷王都見女,今我無一言而不道疾者,言己方困於病,故未能也。}\textbf{昔爾出居,誰從作爾室。}{\footnotesize 遭亂世,義不得去,思其友而不肯反者也。箋云往始離居之時,誰隨為女作室,女猶自作之爾,今反以無室家距我,恨之辭。}

\begin{quoting}思,語詞。疾,通嫉。\end{quoting}

\section{小旻}

%{\footnotesize 六章、三章章八句、三章章七句}

\textbf{小旻,大夫刺幽王也。}{\footnotesize 所刺列於十月之交、雨無正為小,故曰小旻。亦當為刺厲王。}

\begin{quoting}\textbf{蘇轍}小旻、小宛、小弁、小明四詩皆以小名篇,所以別其為小雅也,其在小雅者謂之小,故其在大雅者謂之召旻、大明,獨宛、弁闕焉。\end{quoting}

\textbf{旻天疾威,敷于下土。}{\footnotesize 敷,布也。箋云旻天之德,疾王者以刑罰威恐萬民,其政教乃布於下土,言天下徧知。}\textbf{謀猶回遹,何日斯沮。}{\footnotesize 回邪、遹辟、沮壞也。箋云猶道、沮止也。今王謀為政之道,回辟不循旻天之德已甚矣,心猶不悛,何日此惡將止。}\textbf{謀臧不從,不臧覆用。我視謀猶,亦孔之邛。}{\footnotesize 邛,病也。箋云臧,善也,謀之善者不從,其不善者反用之,我視王謀為政之道,亦甚病天下。}

\begin{quoting}猶、猷古通,爾雅釋詁「猷,謀也」。\end{quoting}

\textbf{潝潝訿訿,亦孔之哀。}{\footnotesize 潝潝然患其上,訿訿然思不稱乎上。箋云臣不事君,亂之階也,甚可哀也。}\textbf{謀之其臧,則具是違。謀之不臧,則具是依。我視謀猶,伊于胡厎。}{\footnotesize 箋云于往、厎至也。謀之善者俱背違之,其不善者依就之,我視今君臣之謀道,往行之將何所至乎,言必至於亂。}

\begin{quoting}\textbf{方玉潤}引曹氏粹中曰潝潝 \texttt{xī} 然相和者,黨同而無公是,訿訿 \texttt{zǐ} 然相毀者,伐異而無公非。韓詩潝作翕,魯詩訿作呰。\end{quoting}

\textbf{我龜既厭,不我告猶。}{\footnotesize 猶,道也。箋云猶,圖也。卜筮數而瀆龜,龜靈厭之,不復告其所圖之吉凶,言雖得兆,占繇不中。}\textbf{謀夫孔多,是用不集。}{\footnotesize 集,就也。箋云謀事者眾而非賢者,是非相奪,莫適可從,故所為不成。}\textbf{發言盈庭,誰敢執其咎。}{\footnotesize 謀人之國,國危則死之,古之道也。箋云謀事者眾,訩訩滿庭而無敢決當是非,事若不成,誰云己當其咎責者,言小人爭知而讓過。}\textbf{如匪行邁謀,是用不得于道。}{\footnotesize 箋云匪,非也。君臣之謀事如此,與不行而坐圖遠近,是於道路無進於跬步何以異乎。}

\begin{quoting}左傳襄八年引末二句,杜注「匪,彼也,行邁謀,謀于路人也,不得于道,眾無適從也」。\end{quoting}

\textbf{哀哉為猶,匪先民是程,匪大猶是經。維邇言是聽,維邇言是爭。}{\footnotesize 古曰在昔,昔曰先民。程法、經常、猶道、邇近也。爭為近言。箋云哀哉,今之君臣謀事,不用古人之法,不循大道之常,而徒聽順近言之同者,爭近言之異者,言見動軔則泥陷,不至於遠也。}\textbf{如彼築室于道謀,是用不潰于成。}{\footnotesize 潰,遂也。箋云如當路築室,得人而與之謀所為,路人之意不同,故不得遂成也。}

\begin{quoting}\textbf{馬瑞辰}經,朱彬謂當訓行,是也,孟子「經德不回」,趙注「經,行也」,匪大猶是經,猶云匪大道是遵循耳。\end{quoting}

\textbf{國雖靡止,或聖或否。民雖靡膴,或哲或謀,或肅或艾。}{\footnotesize 靡止,言小也。人有通聖者,有不能者,亦有明哲者,有聦謀者。艾,治也。有恭肅者,有治理者。箋云靡無、止禮、膴法也。言天下諸侯今雖無禮,其心性猶有通聖者、有賢者,民雖無法,其心性猶有知者、有謀者、有肅者、有艾者,王何不擇焉,置之於位而任之為治乎,書曰「睿作聖,明作哲,聦作謀,恭作肅,從作乂」,詩人之意,欲王敬用五事,以明天道,故云然。}\textbf{如彼泉流,無淪胥以敗。}{\footnotesize 箋云淪,率也。王之為政當如源泉之流,行則清,無相牽率為惡,以自濁敗。}

\begin{quoting}\textbf{馬瑞辰}傳以靡止為小,則止宜訓大矣,抑詩「淑慎爾止」,傳「止,至也」,爾雅「晊,大也」,釋文「晊,本又作至」,易「至哉坤元」猶言「大哉乾元」也,止與至同義,至為大,則止亦為大矣。靡膴,釋文引韓詩作「靡腜,猶無幾何」。\end{quoting}

\textbf{不敢暴虎,不敢馮河。人知其一,莫知其他。}{\footnotesize 馮,陵也。徒涉曰馮河,徒搏曰暴虎。一,非也。他,不敬小人之危殆也。箋云人皆知暴虎、馮河立至之害,而無知當畏慎小人能危亡也。}\textbf{戰戰兢兢,}{\footnotesize 戰戰,恐也。兢兢,戒也。}\textbf{如臨深淵,}{\footnotesize 恐隊也。}\textbf{如履薄冰。}{\footnotesize 恐陷也。}

\section{小宛}

%{\footnotesize 六章、章六句}

\textbf{小宛,大夫刺幽王也。}{\footnotesize 亦當為刺厲王。}

\begin{quoting}\textbf{朱熹}此大夫遭時之亂,而兄弟相戒以免禍之詩。國語晉語「秦伯賦鳩飛」,韋注以鳩飛即小宛。\end{quoting}

\textbf{宛彼鳴鳩,翰飛戾天。}{\footnotesize 興也。宛,小貌。鳴鳩,鶻鵰。翰高、戾至也。行小人之道,責高明之功,終不可得。}\textbf{我心憂傷,念昔先人。}{\footnotesize 先人,文武也。}\textbf{明發不寐,有懷二人。}{\footnotesize 明發,發夕至明。}

\begin{quoting}\textbf{馬瑞辰}戾者,厲之假借,文選卷一李善注引韓詩作翰飛厲天,云「厲,附也」,厲天,猶俗云摩天耳。明、發二字同義,醒也,賈誼新書先醒篇「辟猶俱醉而獨先發也」,漢書鄒陽傳「發悟于心」,晏子「景公飲酒三日而後發」,廣雅釋詁「明、覺,發也」。\end{quoting}

\textbf{人之齊聖,飲酒溫克。}{\footnotesize 齊正、克勝也。箋云中正通知之人,飲酒雖醉,猶能溫藉自持以勝。}\textbf{彼昬不知,壹醉日富。}{\footnotesize 醉日而富矣。箋云童昏無知之人,飲酒一醉,自謂日益富,夸淫自恣,以財驕人。}\textbf{各敬爾儀,天命不又。}{\footnotesize 又,復也。箋云今女君臣各敬慎威儀,天命所去,不復來也。}

\begin{quoting}爾雅釋言「疾、齊,壯也」,郭注「壯,壯事,謂速也」,史記五帝紀「幼而徇齊」。溫,同蘊,蘊藉,含蓄。壹,語首助詞。敬,同警。\end{quoting}

\textbf{中原有菽,庶民采之。}{\footnotesize 中原,原中也。菽,藿也,力采者則得之。箋云藿生原中,非有主也,以喻王位無常家也,勤於德者則得之。}\textbf{螟蛉有子,蜾蠃負之。}{\footnotesize 螟蛉,桑蟲也。蜾蠃,蒲盧也。負,持也。箋云蒲盧取桑蟲之子負持而去,煦嫗養之,以成其子,喻有萬民不能治,則能治者將得之。}\textbf{敎誨爾子,式穀似之。}{\footnotesize 箋云式用、穀善也。今有教誨女之萬民用善道者,亦似蒲盧,言將得而子也。}

\begin{quoting}\textbf{馬瑞辰}戰國策言韓地民之所食,大抵豆飯藿羹,藿對豆言,是為豆葉,文選李善注引說文「藿,豆之葉也」,詩但言菽,傳知其不為豆而為藿者,蓋因豆皆有主,惟葉任人采,其主不禁。蜾蠃 \texttt{guǒ luǒ},寄生蜂,其取幼蟲實為捕食耳。似,同嗣。\end{quoting}

\textbf{題彼脊令,載飛載鳴。}{\footnotesize 題,視也。脊令不能自舍,君子有取節爾。箋云題之為言視睇也,載之言則也,則飛則鳴,翼也口也,不有止息。}\textbf{我日斯邁,而月斯征。}{\footnotesize 箋云我,我王也。邁、征皆行也。王日此行,謂日視朝也,而月此行,謂月視朔也,先王制此禮,使君與群臣議政事,日有所決,月有所行,亦無時止息。}\textbf{夙興夜寐,毋忝爾所生。}{\footnotesize 忝,辱也。}

\begin{quoting}題,同𧡰,魯詩作相,均為視義。古以脊令比於兄弟,脊令飛鳴,喻兄弟遠行。爾所生,謂父母也。\end{quoting}

\textbf{交交桑扈,率場啄粟。}{\footnotesize 交交,小貌。桑扈,竊脂也。言上為亂政而求下之治,終不可得也。箋云竊脂肉食,今無肉而循場啄粟,失其天性,不能以自活。}\textbf{哀我填寡,宜岸宜獄。握粟出卜,自何能穀。}{\footnotesize 填盡、岸訟也。箋云仍得曰宜。自從、穀生也。可哀哉,我窮盡寡財之人仍有獄訟之事,無可以自救,但持粟行卜,求其勝負,從何能得生。}

\begin{quoting}填,釋文「韓詩作疹,疹,苦也」。宜,訓殆,左傳成二年「異哉,夫子有三軍之懼,又有桑中之喜,宜將竊妻以逃者也」,成六年「士貞伯曰,鄭伯其死乎,視流而行速,不安其位,宜不能久」。岸,釋文「韓詩作犴,云鄉亭之繫曰犴,朝廷曰獄」。握粟出卜,\textbf{馬瑞辰}此有二義,一謂以粟祀神,一謂以粟酬卜,說文「貞,卜問也,从卜,貝以為贄」,繫傳引詩握粟出卜云「古者求卜必用貝,握粟其至微者也」,今按二義本自相通,蓋始用糈米以享神,繼即以之酬卜」。\end{quoting}

\textbf{溫溫恭人,}{\footnotesize 溫溫,和柔貌。}\textbf{如集于木。}{\footnotesize 恐隊也。}\textbf{惴惴小心,如臨于谷。}{\footnotesize 恐隕也。}\textbf{戰戰兢兢,如履薄冰。}{\footnotesize 箋云衰亂之世,賢人君子雖無罪猶恐懼。}

\section{小弁}

%{\footnotesize 八章、章八句}

\textbf{小弁,刺幽王也。大子之傅作焉。}

\begin{quoting}漢今文學家以為宣王時尹吉甫之子伯奇所作,班固馮奉世傳讚「讒邪交亂,貞良被害,自古而然,故伯奇放流,屈原赴湘,小弁之詩作,離騷之辭興」,趙岐孟子章句「小弁,小雅之篇,伯奇之詩也,伯奇仁人而父虐之,故作小弁之詩」。\end{quoting}

\textbf{弁彼鸒斯,歸飛提提。}{\footnotesize 興也。弁,樂也。鸒,卑居,卑居,雅烏也。提提,群貌。箋云樂乎彼雅烏出食在野甚飽,群飛而歸提提然,興者,喻凡人之父子兄弟出入宮庭,相與飲食,亦提提然樂,傷今大子獨不。}\textbf{民莫不穀,我獨于罹。}{\footnotesize 幽王取申女,生大子宜咎,又說褒姒,生子伯服,立以為后,而放宜咎,將殺之。箋云穀養、于曰、罹憂也。天下之人無不父子相養者,我大子獨不然,曰以憂也。}\textbf{何辜于天,我罪伊何。}{\footnotesize 舜之怨慕,日號泣于旻天、于父母。}\textbf{心之憂矣,云如之何。}

\begin{quoting}弁 \texttt{pán}。斯,語詞。提提 \texttt{shí}。\end{quoting}

\textbf{踧踧周道,鞫為茂草。}{\footnotesize 踧踧,平易也。周道,周室之通道。鞫,窮也。箋云此喻幽王信褒姒之讒,亂其德政,使不通於四方。}\textbf{我心憂傷,惄焉如擣。假寐永歎,維憂用老。心之憂矣,疢如疾首。}{\footnotesize 惄,思也。擣,心疾也。箋云不脫冠衣而寐曰假寐。疢,猶病者也。}

\begin{quoting}踧踧 \texttt{dì}。鞫 \texttt{jū},\textbf{陳奐}通達之大道,其平易踧踧然,今為茂草所塞。疢 \texttt{chèn},本義為熱病。\end{quoting}

\textbf{維桑與梓,必恭敬止。}{\footnotesize 父之所樹,己尚不敢不恭敬。}\textbf{靡瞻匪父,靡依匪母。不屬于毛,不罹于裏。}{\footnotesize 毛在外陽,以言父,裏在內陰,以言母。箋云此言人無不瞻仰其父取法則者,無不依恃其母以長大者,今我獨不得父皮膚之氣乎,獨不處母之胞胎乎,何曾無恩於我。}\textbf{天之生我,我辰安在。}{\footnotesize 辰,時也。箋云此言我生所值之辰安所在乎,謂六物之吉凶。}

\begin{quoting}\textbf{馬瑞辰}懷父母,覩其樹因思其人也,至後世以桑梓為故里之稱。罹,同麗,附也。\end{quoting}

\textbf{菀彼柳斯,鳴蜩嘒嘒。有漼者淵,萑葦淠淠。}{\footnotesize 蜩,蟬也。嘒嘒,聲也。漼,深貌。淠淠,眾也。箋云柳木茂盛則多蟬,淵深而旁生萑葦,言大者之旁無所不容。}\textbf{譬彼舟流,不知所屆。}{\footnotesize 箋云屆,至也。言今大子不為王及后所容而見放逐,狀如舟之流行,無制之者,不知終所至者也。}\textbf{心之憂矣,不遑假寐。}{\footnotesize 箋云遑,暇也。}

\begin{quoting}菀 \texttt{wǎn},菀柳傳「茂木也」,說文段注「假借為鬱字」。漼 \texttt{cuǐ}。淠淠 \texttt{pèi}。\end{quoting}

\textbf{鹿斯之奔,維足伎伎。雉之朝雊,尚求其雌。}{\footnotesize 伎伎,舒貌,謂鹿之奔走,其足伎伎然舒也。箋云雊,雉鳴也。尚,猶也。鹿之奔走,其勢宜疾,而足伎伎然舒,留其群也,雉之鳴猶知求其雌,今大子之放棄,其妃匹不得與之去,又鳥獸之不如也。}\textbf{譬彼壞木,疾用無枝。}{\footnotesize 壞,瘣也,謂傷病也。箋云大子放逐而不得生子,猶內傷病之,木內有疾,故無枝也。}\textbf{心之憂矣,寧莫之知。}{\footnotesize 箋云寧,猶曾也。}

\begin{quoting}斯,語詞。\textbf{馬瑞辰}伎伎實速行之貌,詩言維足伎伎,蓋言鹿善從其群,見前有鹿則飛行以奔之,與雉求其雌取興正同。雊 \texttt{gòu}。瘣 \texttt{huì}。\end{quoting}

\textbf{相彼投兔,尚或先之。行有死人,尚或墐之。}{\footnotesize 墐,路冢也。箋云相視、投掩、行道也。視彼人將掩兔,尚有先驅走之者,道中有死人,尚有覆掩之成其墐者,言此所不知,其心不忍。}\textbf{君子秉心,維其忍之。}{\footnotesize 箋云君子,斥幽王也。秉,執也。言王之執心,不如彼二人。}\textbf{心之憂矣,涕既隕之。}{\footnotesize 隕,隊也。}

\textbf{君子信讒,如或醻之。}{\footnotesize 箋云醻,旅醻也,如醻之者,謂受而行之。}\textbf{君子不惠,不舒究之。}{\footnotesize 箋云惠愛、究謀也。王不愛大子,故聞讒言則放之,不舒謀也。}\textbf{伐木掎矣,析薪杝矣。}{\footnotesize 伐木者掎其巔,析薪者隨其理。箋云掎其巔者,不欲妄踣之,杝謂觀其理也,必隨其理者,不欲妄挫折之,以言今王之遇大子,不如伐木析薪。}\textbf{舍彼有罪,予之佗矣。}{\footnotesize 佗,加也。箋云予,我也。舍褒姒讒言之罪,而妄加我大子。}

\begin{quoting}醻,同酬。杝,說文「讀若他」,掎、杝、佗三字為韻。\end{quoting}

\textbf{莫高匪山,莫浚匪泉。}{\footnotesize 浚,深也。箋云山高矣,人登其巔,泉深矣,人入其淵,以言人無所不至,雖辟逃之,猶有默存者焉。}\textbf{君子無易由言,耳屬于垣。}{\footnotesize 箋云由,用也。王無輕用讒人之言,人將有屬耳於壁而聽之者,知王有所受之,知王心之不正也。}\textbf{無逝我梁,無發我笱。}{\footnotesize 箋云逝,之也,之人梁,發人笱,此必有盜魚之罪,以言褒姒淫色來嬖於王,盜我大子母子之寵。}\textbf{我躬不閱,遑恤我後。}{\footnotesize 念父,孝也。高子曰「小弁,小人之詩也」,孟子曰「何以言之」,曰「怨乎」,孟子曰「固哉夫高叟之為詩也,有越人於此,關弓而射我,我則談笑而道之,無他,踈之也,兄弟關弓而射我,我則垂涕泣而道之,無他,戚之也,然則小弁之怨,親親也,親親,仁也,固哉夫高叟之為詩」,曰「凱風何以不怨」,曰「凱風,親之過小者也,小弁,親之過大者也,親之過大而不怨,是愈踈也,親之過小而怨,是不可譏也,愈踈,不孝也,不可譏,亦不孝也,孔子曰『舜其至孝矣,五十而慕』」。箋云念父,孝也,大子念王將受讒言不止,我死之後,懼復有被讒者,無如之何,故自決云「我身尚不能自容,何暇乃憂我死之後也」。}

\begin{quoting}爾雅釋詁「繇,於也」,繇、由古通用。\textbf{胡承珙}山高泉深莫能窮測也,以喻人心之險猶山川,君子苟輕易其言,耳屬者必將迎合風旨而交構其間矣。\end{quoting}

\section{巧言}

%{\footnotesize 六章、章八句}

\textbf{巧言,刺幽王也。大夫傷於讒,故作是詩也。}

\begin{quoting}\textbf{胡承珙}詩以悠悠昊天發端,而取五章之巧言名篇,蓋讒人之言非巧不入,詩人所深惡也。詩言亂者凡十,言君子者七。\end{quoting}

\textbf{悠悠昊天,曰父母且。無罪無辜,亂如此幠。}{\footnotesize 幠,大也。箋云悠悠,思也。幠,敖也。我憂思乎昊天,愬王也,始者言其且為民之父母,今乃刑殺無罪無辜之人,為亂如此,甚敖慢無法度也。}\textbf{昊天已威,予慎無罪。昊天大幠,予慎無辜。}{\footnotesize 威畏、慎誠也。箋云已、泰皆言甚也。昊天乎,王甚可畏,王甚敖慢,我誠無罪而罪我。}

\begin{quoting}曰,同聿,發語詞。且 \texttt{jū},語詞。幠 \texttt{hū},本義為覆,引申為大,又為傲慢。\end{quoting}

\textbf{亂之初生,僭始既涵。}{\footnotesize 僭數、涵容也。箋云僭,不信也。既盡、涵同也。王之初生亂萌,群臣之言,不信與信,盡同之不別也。}\textbf{亂之又生,君子信讒。}{\footnotesize 箋云君子,斥在位者也。在位者信讒人之言,是復亂之所生。}\textbf{君子如怒,亂庶遄沮。}{\footnotesize 遄疾、沮止也。箋云君子見讒人如怒責之,則此亂庶幾可疾止也。}\textbf{君子如祉,亂庶遄已。}{\footnotesize 祉,福也。箋云福者,福賢者,謂爵祿之也,如此則亂亦庶幾可疾止也。}

\begin{quoting}僭,同譖,說文「譖,愬也」。\textbf{陳奐}福亦喜也,遄已猶遄沮也。\end{quoting}

\textbf{君子屢盟,亂是用長。}{\footnotesize 凡國有疑,會同則用盟而相要也。箋云屢,數也。盟之所以數者,由世衰亂多相背違。時見曰會,殷見曰同,非此時而盟謂之數。}\textbf{君子信盜,亂是用暴。}{\footnotesize 盜,逃也。箋云盜,謂小人也,春秋傳曰「賤者窮諸盜」。}\textbf{盜言孔甘,亂是用餤。}{\footnotesize 餤,進也。}\textbf{匪其止共,維王之邛。}{\footnotesize 邛,病也。箋云小人好為讒佞,既不共其職事,又為王作病。}

\begin{quoting}餤 \texttt{tán},本義為進食,引申為加劇。\end{quoting}

\textbf{奕奕寢廟,君子作之。秩秩大猷,聖人莫之。他人有心,予忖度之。躍躍毚兔,遇犬獲之。}{\footnotesize 奕奕,大貌。秩秩,進知也。莫,謀也。毚兔,狡兔也。箋云此四事者,言各有所能也,因己能忖度讒人之心,故列道之爾。猷,道也,大道,治國之禮法。遇犬,犬之馴者,謂田犬也。}

\begin{quoting}莫,齊詩作謨。躍躍 \texttt{tì},三家詩作趯趯。毚 \texttt{chán}。\end{quoting}

\textbf{荏染柔木,君子樹之。往來行言,心焉數之。}{\footnotesize 荏染,柔意也。柔木,椅桐梓漆也。箋云此言君子樹善木,如人心思數善言而出之,善言者,往亦可行,來亦可行,於彼亦可,於己亦可,是之謂行也。}\textbf{蛇蛇碩言,出自口矣。}{\footnotesize 蛇蛇,淺意也。箋云碩,大也,大言者,言不顧其行,徒從口出,非由心也。}\textbf{巧言如簧,顏之厚矣。}{\footnotesize 箋云顏之厚者,出言虛偽而不知慚於人也。}

\begin{quoting}行言,道上流言也。\textbf{王先謙}樹木必由我心擇而取之,行言亦必由我心審而出之,非可苟也。\textbf{馬瑞辰}蛇蛇 \texttt{yí},即訑訑之假借,廣雅「訑,欺也」,玉篇「訑,詭言也」,蛇蛇蓋大言欺世之貌。\end{quoting}

\textbf{彼何人斯,居河之麋。}{\footnotesize 水草交謂之麋。箋云何人者,斥讒人也,賤而惡之,故曰何人。}\textbf{無拳無勇,職為亂階。}{\footnotesize 拳,力也。箋云言無力勇者,謂易誅除也。職,主也。此人主為亂作階,言亂由之來也。}\textbf{既微且尰,爾勇伊何。}{\footnotesize 骭瘍為微,腫足為尰。箋云此人居下濕之地,故生微腫之疾,人憎惡之,故言女勇伊何,何所能也。}\textbf{為猶將多,爾居徒幾何。}{\footnotesize 箋云猶謀、將大也。女作讒佞之謀大多,女所與居之眾幾何人,素能然乎。}

\begin{quoting}麋,齊詩作湄,爾雅「水草交為湄」,為傳所本。拳,同捲。微,同癓。骭 \texttt{gàn},小腿。\textbf{馬瑞辰}為猶將多,言其為欺詐且多也。居,語詞。釋文「傃,音素」,平素。\end{quoting}

\section{何人斯}

%{\footnotesize 八章、章六句}

\textbf{何人斯,蘇公刺暴公也。暴公為卿士而譖蘇公焉,故蘇公作是詩以絕之。}{\footnotesize 暴也、蘇也皆畿內國名也。}

\textbf{彼何人斯,其心孔艱。胡逝我梁,不入我門。}{\footnotesize 箋云孔甚、艱難、逝之也。梁,魚梁也,在蘇國之門外。彼何人乎,謂與暴公俱見於王者也,其持心甚難知,言其性堅固,似不妄也,暴公譖己之時,女與之乎,今過我國,何故近之我梁而不入見我乎,疑其與之而未察,斥其姓名為大切,故言何人。}\textbf{伊誰云從,維暴之云。}{\footnotesize 云,言也。箋云譖我者是言從誰生乎,乃暴公之所言也。由己情而本之,以解何人意。}

\textbf{二人從行,誰為此禍。胡逝我梁,不入唁我。}{\footnotesize 箋云二人者,謂暴公與其侶也,女相隨而行見王,誰作我是禍乎,時蘇公以得譴讓也,女即不為,何故近之我梁而不入弔唁我乎。}\textbf{始者不如今,云不我可。}{\footnotesize 箋云女始者於我甚厚,不如今日也,今日云我所行有何不可者乎,何更於己薄也。}

\begin{quoting}可,同哿,嘉好之義。\end{quoting}

\textbf{彼何人斯,胡逝我陳。我聞其聲,不見其身。}{\footnotesize 陳,堂塗也。箋云堂塗者,公館之堂塗也,女即不為,何故近之我館庭,使我得聞女之音聲,不得睹女之身乎。}\textbf{不愧于人,不畏于天。}{\footnotesize 箋云女今不入唁我,何所愧畏乎,皆疑之未察之辭。}

\begin{quoting}爾雅釋宮「堂塗謂之陳」,孫炎注「堂下至門之徑」。\textbf{胡承珙}凡通問皆可謂之聲,聞其聲不見其身者,蓋通問而不請見也。\end{quoting}

\textbf{彼何人斯,其為飄風。胡不自北,胡不自南。胡逝我梁,祇攪我心。}{\footnotesize 飄風,暴起之風。攪,亂也。箋云祇,適也。何人乎,女行來而去疾如飄風,不欲入見我,何不乃從我國之南,不則乃從我國之北,何近之我梁,適亂我之心,使我疑女。}

\textbf{爾之安行,亦不遑舍。爾之亟行,遑脂爾車。壹者之來,云何其盱。}{\footnotesize 箋云遑暇、亟疾、盱病也。女可安行乎,則何不暇舍息乎,女當疾行乎,則又何暇脂女車乎,極其情,求其意,終不得一者之來見我,於女亦何病乎。}

\begin{quoting}\textbf{陳奐}壹者猶言乃者,乃者謂曩日也。\end{quoting}

\textbf{爾還而入,我心易也。還而不入,否難知也。壹者之來,俾我祇也。}{\footnotesize 易說、祇病也。箋云還,行反也。否,不通也。祇,安也。女行反入見我,我則解說也,反又不入見我,則我與女情不通,女與於譖我與不,復難知也,一者之來見我,我則知之,是使我心安也。}

\begin{quoting}易,通懌,喜悅,\textbf{陳奐}釋文引韓詩作施,施,善也。否,語詞。祇,通疧,痛苦。\end{quoting}

\textbf{伯氏吹壎,仲氏吹篪。}{\footnotesize 土曰壎,竹曰篪。箋云伯仲,喻兄弟也。我與女恩如兄弟,其相應和如壎篪,以言俱為王臣,宜相親愛。}\textbf{及爾如貫,諒不我知。出此三物,以詛爾斯。}{\footnotesize 三物,豕犬雞也,民不相信則盟詛之,君以豕,臣以犬,民以雞。箋云及與、諒信也。我與女俱為王臣,其相比次如物之在繩索之貫也,今女心誠信而我不知,且共出此三物,以詛女之此事,為其情之難知,己又不欲長怨,故設之以此言。}

\begin{quoting}篪 \texttt{chí}。\end{quoting}

\textbf{為鬼為蜮,則不可得。有靦面目,視人罔極。}{\footnotesize 蜮,短狐也。靦,姡也。箋云使女為鬼為蜮也,則女誠不可得見也,姡然有面目,女乃人也,人相視無有極時,終必與女相見。}\textbf{作此好歌,以極反側。}{\footnotesize 反側,不正直也。箋云好,猶善也。反側,輾轉也。作八章之歌,求女之情,女之情反側極於是也。}

\begin{quoting}靦 \texttt{tiǎn}。姡 \texttt{huó}。視,同示。\textbf{陳奐}書洪範云「無反無側,王道正直」,此傳義之所本。\end{quoting}

\section{巷伯}

%{\footnotesize 七章、四章章四句、一章五句、一章八句、一章六句}

\textbf{巷伯,刺幽王也。寺人傷於讒,故作是詩也。}{\footnotesize 巷伯,奄官。寺人,內小臣也。奄官上士四人,掌王后之命,於宮中為近,故謂之巷伯,與寺人之官相近。讒人譖寺人,寺人又傷其將及巷伯,故以名篇。}

\begin{quoting}\textbf{陳奐}周禮無巷伯之官,唯襄九年左傳「令司宮、巷伯儆宮」與此詩巷伯同,左傳以巷伯次司宮,猶周禮之寺人次內小臣,杜預云「巷伯即寺人」,當是賈服舊注,巷伯即經所謂寺人孟子也。\end{quoting}

\textbf{萋兮斐兮,成是貝錦。}{\footnotesize 興也。萋斐,文章相錯也。貝錦,錦文也。箋云錦文者,文如餘泉、餘蚳之貝文也。興者,喻讒人集作己過,以成於罪,猶女工之集采色以成錦文。}\textbf{彼譖人者,亦已大甚。}{\footnotesize 箋云大甚者,謂使己得重罪也。}

\textbf{哆兮侈兮,成是南箕。}{\footnotesize 哆,大貌。南箕,星也。侈之言是必有因也,斯人自謂辟嫌之不審也。昔者顏叔子獨處于室,鄰之嫠婦又獨處于室,夜,暴風雨至而室壞,婦人趨而至,顏叔子納之而使執燭,放乎旦而蒸盡,縮屋而繼之,自以為辟嫌之不審矣,若其審者,宜若魯人然,魯人有男子獨處于室,鄰之嫠婦亦獨處于室,夜,暴風雨至而室壞,婦人趨而託之,男子閉戶而不納,婦人自牖與之言曰「子何為不納我乎」,男子曰「吾聞之也,男子不六十不閒居,今子幼,吾亦幼,不可以納子」,婦人曰「子何不若柳下惠然,嫗不逮門之女,國人不稱其亂」,男子曰「柳下惠固可,吾固不可,吾將以吾不可學柳下惠之可」,孔子曰「欲學柳下惠者,未有似於是也」。箋云箕星哆然,踵狹而舌廣,今讒人之因寺人之近嫌而成言其罪,猶因箕星之哆而侈大之。}\textbf{彼譖人者,誰適與謀。}{\footnotesize 箋云適,往也,誰往就女謀乎,怪其言多且巧。}

\begin{quoting}哆 \texttt{chǐ},魯詩作誃,張口貌。\textbf{馬瑞辰}適,悦也。\end{quoting}

\textbf{緝緝翩翩,謀欲譖人。}{\footnotesize 緝緝,口舌聲。翩翩,往來貌。}\textbf{慎爾言也,謂爾不信。}{\footnotesize 箋云慎,誠也,女誠心而後言,王將謂女不信而不受,欲其誠者,惡其不誠也。}

\begin{quoting}緝緝 \texttt{qī},三家詩作咠咠,\textbf{馬瑞辰}說文「咠,聶語也,聶,附耳私小語也」,緝緝即咠咠之假借,說文「諞,便,巧言也」,翩翩即諞諞之假借,詩言緝緝者,言之密也,翩翩者,言之巧也。\end{quoting}

\textbf{捷捷幡幡,謀欲譖言。}{\footnotesize 捷捷,猶緝緝也。幡幡,猶翩翩也。}\textbf{豈不爾受,既其女遷。}{\footnotesize 遷,去也。箋云遷之言訕也,王倉卒豈將不受女言乎,己則亦將復訕誹女。}

\textbf{驕人好好,勞人草草。}{\footnotesize 好好,喜也。草草,勞心也。箋云好好者,喜讒言之人也,草草者,憂將妄得罪也。}\textbf{蒼天蒼天,視彼驕人,矜此勞人。}

\begin{quoting}\textbf{陳奐}草讀為慅,假借字也,月出「勞心慅兮」,重言曰慅慅。\end{quoting}

\textbf{彼譖人者,誰適與謀。取彼譖人,投畀豺虎。豺虎不食,投畀有北。}{\footnotesize 投,棄也。北方寒涼而不毛。}\textbf{有北不受,投畀有昊。}{\footnotesize 昊,昊天也。箋云付與昊天,制其罪也。}

\textbf{楊園之道,猗于畝丘。}{\footnotesize 楊園,園名。猗,加也。畝丘,丘名。箋云欲之楊園之道,當先歷畝丘,以言此讒人欲譖大臣,故從近小者始。}\textbf{寺人孟子,作為此詩。凡百君子,敬而聽之。}{\footnotesize 寺人而曰孟子者,罪已定矣,而將踐刑作此詩也。箋云寺人,王之正內五人。作,起也,孟子起而為此詩,欲使眾在位者慎而知之。既言寺人,復自著孟子者,自傷將去此官也。}

\begin{quoting}敬,同儆,儆惕。\end{quoting}

%\begin{flushright}節南山之什十篇、七十九章、五百五十二句\end{flushright}