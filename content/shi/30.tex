\chapter{那詁訓傳第三十}

\begin{quoting}\textbf{釋文}商者,契所封之地名。成湯伐桀,王天下,遂以為國號。後世有中宗高宗中興,時有作詩頌之者。當周宣王之時,宋大夫正考父校商之名頌十二篇於周之大師,以那為首,歸而祭於先王。孔子錄詩之時,止五篇而已,乃列之以備三頌。案商頌,宋詩也。\end{quoting}

\section{那}

%{\footnotesize 一章、二十二句}

\textbf{那,祀成湯也。微子至于戴公,其間禮樂廢壞,有正考甫者,得商頌十二篇於周之大師,以那為首。}{\footnotesize 禮樂廢壞者,君怠慢於為政,不修祭祀、朝聘、養賢、待賓之事,有司忘其禮之儀制,樂師失其聲之曲折,由是散亡也,自正考甫至孔子之時又無七篇矣。正考甫,孔子之先也,其祖弗甫何,以有宋而授厲公。}

\textbf{猗與那與,置我鞉鼓。}{\footnotesize 猗,歎辭。那,多也。鞉鼓,樂之所成也。夏后氏足鼓,殷人置鼓,周人縣鼓。箋云置,讀曰植,植鞉鼓者,為楹貫而樹之。美湯受命伐桀,定天下而作濩樂,故歎之。多其改夏之制,乃始植我殷家之樂鞉與鼓也。鞉雖不植,貫而搖之,亦植之類。}\textbf{奏鼓簡簡,衎我烈祖。湯孫奏假,綏我思成。}{\footnotesize 衎,樂也。烈祖,湯有功烈之祖也。假,大也。箋云奏鼓,奏堂下之樂也。烈祖,湯也。湯孫,大甲也。假升、綏安也。以金奏堂下諸縣,其聲和大簡簡然,以樂我功烈之祖成湯,湯孫大甲又奏升堂之樂,弦歌之,乃安我心所思而成之,謂神明來格也。禮記曰「齊之日,思其居處,思其笑語,思其志意,思其所樂,思其所耆,齊三日,乃見其所為齊者,祭之日,入室,僾然必有見乎其位,周旋出戶,肅然必有聞乎其容聲,出戶而聽,愾然必有聞乎其歎息之聲」,此之謂思成。}\textbf{鞉鼓淵淵,嘒嘒管聲。既和且平,依我磬聲。}{\footnotesize 嘒嘒然和也。平,正平也。依,倚也。磬,聲之清者也,以象萬物之成。周尚臭,殷尚聲。箋云磬,玉磬也。堂下諸縣與諸管聲皆和平不相奪倫,又與玉磬之聲相依,亦謂和平也。玉磬尊,故異言之。}\textbf{於赫湯孫,穆穆厥聲。庸鼓有斁,萬舞有奕。}{\footnotesize 於赫湯孫,盛矣湯為人子孫也。大鍾曰庸。斁斁然盛也,奕奕然閑也。箋云穆穆,美也。於盛矣湯孫,呼大甲也,此樂之美其聲,鍾鼓則斁斁然有次序,其干舞又閑習。}\textbf{我有嘉客,亦不夷懌。自古在昔,先民有作。溫恭朝夕,執事有恪。}{\footnotesize 夷,說也。先王稱之曰自古,古曰在昔,昔曰先民。有作,有所作也。恪,敬也。箋云嘉客,謂二王後及諸侯來助祭者。我客之來助祭者亦不說懌乎,言說懌也,乃大古而有此助祭之禮,非專於今也,其禮儀溫溫然恭敬,執事薦饌則又敬也。}\textbf{顧予烝嘗,湯孫之將。}{\footnotesize 箋云顧,猶念也。將,猶扶助也。嘉客念我殷家有時祭之事而來者,乃大甲之扶助也,序助者來之意也。}

\begin{quoting}猗那,即猗儺、婀娜、旖旎。\textbf{皮錫瑞}湯孫,乃主祭者之號,自當屬宋襄公,且萬舞之名至周始有也。\textbf{馬瑞辰}湯孫奏假,皆祭者致神之謂也,小爾雅、說文並曰「奏,進也」,上致乎神曰奏假。綏,通遺。思,語詞。成,備也福也。\end{quoting}

\section{烈祖}

%{\footnotesize 一章、二十二句}

\textbf{烈祖,祀中宗也。}{\footnotesize 中宗,殷王大戊,湯之玄孫也,有桑穀之異,懼而修德,殷道復興,故表顯之,號為中宗。}

\begin{quoting}\textbf{王質}詩總聞曰前詩聲也,所言皆音樂,此詩臭也,所言皆飲食也,商尚聲,亦尚臭,二詩當是各一節,那奏聲之詩,此薦臭之詩也。\end{quoting}

\textbf{嗟嗟烈祖,有秩斯祜。申錫無疆,及爾斯所。既載清酤,賚我思成。}{\footnotesize 秩常、申重、酤酒、賚賜也。箋云祜,福也。賚,讀如往來之來。嗟嗟乎我功烈之祖成湯,既有此王天下之常福,天又重賜之以無竟界之期,其福乃及女之此所,女,女中宗也,言承湯之業,能興之也,既載清酒於尊,酌以祼獻而神靈來至,我致齊之所思則用成。重言嗟嗟,美歎之深。}\textbf{亦有和羹,既戒既平。鬷假無言,時靡有爭。綏我眉壽,黃耇無疆。}{\footnotesize 戒至、鬷緫、假大也。緫大,無言無爭也。箋云和羹者,五味調,腥孰得節,食之於人性安和,喻諸侯有和順之德也。我既祼獻,神靈來至,亦復由有和順之諸侯來助祭也,其在廟中既恭肅敬戒矣,既齊立平列矣,至于設薦進俎,又緫升堂而齊一,皆服其職、勸其事,寂然無言語者、無爭訟者,此由其心平性和,神靈用之故,安我以壽考之福,歸美焉。}\textbf{約軧錯衡,八鸞鶬鶬。以假以享,我受命溥將。自天降康,豐年穰穰。}{\footnotesize 八鸞鶬鶬,言文德之有聲也。假,大也。箋云約軧,轂飾也。鸞在鑣,四馬則八鸞。假,升也。享,獻也。將,猶助也。諸侯來助祭者,乘篆轂金飾錯衡之車,駕四馬,其鸞鶬鶬然聲和,言車服之得其正也,以此來朝,升堂獻其國之所有,於我受政教,至祭祀又溥助我,言得萬國之歡心也,天於是下平安之福,使年豐。}\textbf{來假來饗,降福無疆。}{\footnotesize 箋云饗,謂獻酒使神饗之也。諸侯助祭者來升堂,來獻酒,神靈又下與我久長之福也。}\textbf{顧予烝嘗,湯孫之將。}{\footnotesize 箋云此祭中宗,諸侯來助之,所言湯孫之將者,中宗之饗此祭,由湯之功,故本言之。}

\begin{quoting}秩,大貌。戒,通屆。鬷假,齊詩作奏假。約軧錯衡,見采芑注。溥,廣也,將,長也。\end{quoting}

\section{玄鳥}

%{\footnotesize 一章、二十二句}

\textbf{玄鳥,祀高宗也。}{\footnotesize 祀,當作祫,祫,合也。高宗,殷王武丁,中宗玄孫之孫也,有雊雉之異,又懼而修德,殷道復興,故亦表顯之,號為高宗云,崩而始合祭於契之廟,歌是詩焉。古者君喪三年既畢,禘於其廟,而後祫祭於大祖,明年春,禘於群廟,自此之後,五年而再殷祭,一禘一祫,春秋謂之大事。}

\textbf{天命玄鳥,降而生商,宅殷土芒芒。}{\footnotesize 玄鳥,鳦也。春分玄鳥降,湯之先祖有娀氏女簡狄配高辛氏帝,帝率與之祈于郊禖而生契,故本其為天所命,以玄鳥至而生焉。芒芒,大貌。箋云降,下也。天使鳦下而生商者,謂鳦遺卵,娀氏之女簡狄吞之而生契,為堯司徒,有功封商,堯知其後將興,又錫其姓焉。自契至湯八遷,始居亳之殷地而受命,國日以廣大芒芒然。湯之受命,由契之功,故本其天意。}\textbf{古帝命武湯,正域彼四方。方命厥后,奄有九有。}{\footnotesize 正長、域有也。九有,九州也。箋云古帝,天也。天帝命有威武之德者成湯,使之長有邦域,為政於天下,方命其君,謂徧告諸侯也,湯有是德,故覆有九州,為之王也。}\textbf{商之先后,受命不殆,在武丁孫子。}{\footnotesize 武丁,高宗也。箋云后,君也。商之先君受天命而行之不解殆者,在高宗之孫子,言高宗興湯之功,法度明也。}\textbf{武丁孫子,武王靡不勝。龍旂十乘,大糦是承。}{\footnotesize 勝,任也。箋云交龍為旂。糦,黍稷也。高宗之孫子有武功、有王德於天下者,無所不勝服,乃有諸侯建龍旂者十乘,奉承黍稷而進之者,亦言得諸侯之歡心。十乘者,二王後、八州之大國。}\textbf{邦畿千里,維民所止,肇域彼四海。}{\footnotesize 畿,疆也。箋云止,猶居也。肇,當作兆。王畿千里之內,其民居安,乃後兆域正天下之經界,言其為政自內及外。}\textbf{四海來假,來假祁祁,景員維河。殷受命咸宜,百祿是何。}{\footnotesize 景大、員均、何任也。箋云假,至也。祁祁,眾多也。員,古文作云。河之言何也。天下既蒙王之政令,皆得其所而來朝覲貢獻,其至也祁祁然眾多,其所貢於殷大至所云維言何乎,言殷王之受命皆其宜也,百祿是何,謂當檐負天之多福。}

\begin{quoting}鳦,即燕子也。宅殷土芒芒,魯詩作殷社芒芒,古社、土通用。史記殷本紀,湯曰「吾甚武,號曰武王」。正,通征。九有,韓詩作九域,九域,九州也。\textbf{王引之}疑經文兩言武丁,皆武王之譌,而「武王靡不勝」則武丁之譌。糦,韓詩作饎 \texttt{chì},酒食也。邦畿千里,文選西京賦注引詩作「封畿千里」。\textbf{朱熹}景,山名,商所都也,春秋傳亦曰「商湯有景亳之命」是也,員,與下章「幅隕」義同,蓋言周也,河,大河也,言景山四周皆大河也。\end{quoting}

\section{長發}

%{\footnotesize 七章、一章八句、四章章七句、一章九句、一章六句}

\textbf{長發,大禘也。}{\footnotesize 大禘,郊祭天也,禮記曰「王者禘其祖之所自出,以其祖配之」是謂也。}

\textbf{濬哲維商,長發其祥。洪水芒芒,禹敷下土方。外大國是疆,幅隕既長。}{\footnotesize 濬深、洪大也。諸夏為外。幅,廣也。隕,均也。箋云長,猶久也。隕,當作圓,圓,謂周也。深知乎維商家之德也,久發見其禎祥矣,乃用洪水,禹敷下土正四方、定諸夏、廣大其竟界之時,始有王天下之萌兆,歷虞夏之世,故為久也。}\textbf{有娀方將,帝立子生商。}{\footnotesize 有娀,契母也。將,大也。契生商也。箋云帝,黑帝也。禹敷下土之時,有娀氏之國亦始廣大,有女簡狄吞鳦卵而生契,堯封之於商,後湯王因以為天下號,故云「帝立子生商」。}

\begin{quoting}\textbf{陳奐}禹有天下曰夏,故畿內為夏,畿外為諸夏也。\end{quoting}

\textbf{玄王桓撥,受小國是達,受大國是達。率履不越,遂視既發。}{\footnotesize 玄王,契也。桓大、撥治、履禮也。箋云承黑帝而立子,故謂契為玄王。遂,猶徧也。發,行也。玄王廣大其政治,始堯封之商,為小國,舜之末年乃益其土地為大國,皆能達其教令,使其民循禮,不得踰越,乃徧省視之,教令則盡行也。}\textbf{相土烈烈,海外有截。}{\footnotesize 相土,契孫也。烈烈,威也。箋云截,整齊也。相土居夏后之世,承契之業,入為王官之伯,出長諸侯,其威武之盛烈烈然,四海之外率服截爾整齊。}

\begin{quoting}\textbf{王先謙}發,明也,釋文引韓詩文,蓋以桓撥二字平列,訓桓為武,訓發為明,言玄王有英明之姿。履,三家詩作禮。\end{quoting}

\textbf{帝命不違,至于湯齊。}{\footnotesize 至湯與天心齊。箋云帝命不違者,天之所以命契之事世世行之,其德浸大,至於湯而當天心。}\textbf{湯降不遲,聖敬日躋。昭假遲遲,上帝是祗。帝命式于九圍。}{\footnotesize 不遲,言疾也。躋,升也。九圍,九州也。箋云降下、假暇、祗敬、式用也。湯之下士尊賢甚疾,其聖敬之德日進,然而以其德聦明寬暇天下之人遲遲然,言急於己而緩於人,天用是故愛敬之也,天於是又命之,使用事於天下,言王之也。}

\begin{quoting}\textbf{馬瑞辰}詩總括相土以下諸君,謂商先君不違天命,至湯皆齊一。\end{quoting}

\textbf{受小球大球,為下國綴旒,何天之休。}{\footnotesize 球玉、綴表、旒章也。箋云綴,猶結也。旒,旌旗之垂者也。休,美也。湯既為天所命,則受小玉,謂尺二寸圭也,受大玉,謂珽也,長三尺,執圭搢珽,以與諸侯會同,結定其心,如旌旗之旒縿著焉,檐負天之美譽,為眾所歸鄉。}\textbf{不競不絿,不剛不柔,敷政優優,百祿是遒。}{\footnotesize 絿,急也。優優,和也。遒,聚也。箋云競,逐也,不逐,不與人爭前後。}

\textbf{受小共大共,為下國駿厖,何天之龍。}{\footnotesize 共法、駿大、厖厚、龍和也。箋云共,執也,小共大共,猶所執搢小球大球也。駿之言俊也。龍,當作寵,寵,榮名之謂。}\textbf{敷奏其勇,不震不動,不戁不竦,百祿是緫。}{\footnotesize 戁恐、竦懼也。箋云不震不動,不可驚憚也。}

\begin{quoting}\textbf{馬瑞辰}駿與恂、厖與蒙古並聲近通用,為下國恂蒙,猶云為下國庇覆耳。戁 \texttt{rǎn}。\end{quoting}

\textbf{武王載旆,有虔秉鉞。如火烈烈,則莫我敢曷。}{\footnotesize 武王,湯也。旆,旗也。虔固、曷害也。箋云有之言又也。上既美其剛柔得中,勇毅不懼,於是有武功、有王德,及建旆興師出伐,又固持其鉞,志在誅有罪也,其威勢如猛火之炎熾,誰敢禦害我。}\textbf{苞有三蘖,莫遂莫達,九有有截。}{\footnotesize 苞本、蘖餘也。箋云苞,豐也。天豐大先三正之後世,謂居以大國,行天子之禮樂,然而無有能以德自遂達於天者,故天下歸鄉湯,九州齊壹截然。}\textbf{韋顧既伐,昆吾夏桀。}{\footnotesize 有韋國者,有顧國者,有昆吾國者。箋云韋,豕韋,彭姓也,顧、昆吾皆己姓也,三國黨於桀惡。湯先伐韋、顧,克之,昆吾、夏桀則同時誅也。}

\begin{quoting}\textbf{王引之}韓詩外傳引詩並作「武王載發」,說文則作「武王載坺」,發,正字也,旆、坺皆借字也,發,謂起師伐桀也。曷,魯詩、韓詩作遏,爾雅釋詁「曷、遏,止」。\textbf{朱熹}本則夏桀,蘖則韋也、顧也、昆吾也。\textbf{王先謙}遂與達皆草木生長之稱,莫遂莫達,喻三國不能復興。\end{quoting}

\textbf{昔在中葉,有震且業。允也天子,降予卿士。}{\footnotesize 葉,世也。業,危也。箋云中世,謂相土也。震,猶威也。相土始有征伐之威,以為子孫討惡之業,湯遵而興之,信也天命而子之,下予之卿士,謂生賢佐也,春秋傳曰「畏君之震,師徒橈敗」。}\textbf{實維阿衡,實左右商王。}{\footnotesize 阿衡,伊尹也。左右,助也。箋云阿倚、衡平也。伊尹,湯所依倚而取平,故以為官名。商王,湯也。}

\begin{quoting}\textbf{馬瑞辰}下文「允也天子」指湯,承上言之,則中葉宜指湯時,蓋自殷有天下言則湯為開創之君,自玄王立國言則湯為中葉矣。爾雅釋詁「業,大也」。\textbf{馬瑞辰}段玉裁曰「伊與阿、尹與衡皆雙聲,即一聲之轉」,今按段說是也,伊尹即阿衡之轉,故毛傳以阿衡為伊尹,箋亦以阿衡為官名。\end{quoting}

\section{殷武}

%{\footnotesize 六章、三章章六句、二章章七句、一章五句}

\textbf{殷武,祀高宗也。}

\begin{quoting}\textbf{王先謙}韓說曰,宋襄公去奢即儉。\end{quoting}

\textbf{撻彼殷武,奮伐荊楚。冞入其阻,裒荊之旅。}{\footnotesize 撻,疾意也。殷武,殷王武丁也。荊楚,荊州之楚國也。冞深、裒聚也。箋云有鍾鼓曰伐。冞,冒也。殷道衰而楚人叛,高宗撻然奮揚威武,出兵伐之,冒入其險阻,謂踰方城之隘,克其軍率而俘虜其士眾。}\textbf{有截其所,湯孫之緒。}{\footnotesize 箋云緒,業也。所,猶處也。高宗所伐之處,國邑皆服其罪,更自敕整截然齊壹,是乃湯孫大甲之等功業。}

\begin{quoting}\textbf{朱熹}易曰「高宗伐鬼方,三年克之」,蓋謂此歟。\end{quoting}

\textbf{維女荊楚,居國南鄉。昔有成湯,自彼氐羌,莫敢不來享,莫敢不來王,曰商是常。}{\footnotesize 鄉,所也。箋云氐羌,夷狄國在西方者也。享,獻也。世見曰王。維女楚國,近在荊州之域,居中國之南方而背叛乎,成湯之時,乃氐羌遠夷之國來獻來見,曰商王是吾常君也,此所用責楚之義,女乃遠夷之不如。}

\begin{quoting}\textbf{馬瑞辰}成湯仍當為生時之號。自,通雖。不來王,不來朝也,左傳隱九年「宋公不王」。\end{quoting}

\textbf{天命多辟,設都于禹之績。歲事來辟,勿予禍適,稼穡匪解。}{\footnotesize 辟君、適過也。箋云多,眾也。來辟,猶來王也。天命乃令天下眾君諸侯立都於禹所治之功,以歲時來朝覲於我殷王者,勿罪過與之禍適,徒敕以勸民稼穡,非可解倦。時楚不修諸侯之職,此所用告曉楚之義也。禹平水土,弼成五服而諸侯之國定,是以云然。}

\begin{quoting}績,通蹟、迹,禹迹,九州也。\textbf{王引之}予,猶施也,禍,讀為過,廣雅「謫,過責也」,勿予過責,言不施過責也。\end{quoting}

\textbf{天命降監,下民有嚴。不僭不濫,不敢怠遑。命于下國,封建厥福。}{\footnotesize 嚴,敬也。不僭不濫,賞不僭、刑不濫也。封,大也。箋云降下、遑暇也。天命乃下視下民,有嚴明之君,能明德慎罰,不敢怠惰自暇於政事者,則命之於小國以為天子,大立其福,謂命湯使由七十里王天下也。時楚僭號王位,此又所用告曉楚之義。}

\begin{quoting}嚴,通儼,敬也。\end{quoting}

\textbf{商邑翼翼,四方之極。赫赫厥聲,濯濯厥靈。壽考且寧,以保我後生。}{\footnotesize 商邑,京師也。箋云極,中也。商邑之禮俗翼翼然可則俲,乃四方之中正也,赫赫乎其出政教也,濯濯乎其見尊敬也,王乃壽考且安,以此全守我子孫。此又用商德重告曉楚之義。}

\begin{quoting}商邑,三家詩作京邑。\end{quoting}

\textbf{陟彼景山,松柏丸丸。是斷是遷,方斵是虔。松桷有梴,旅楹有閑,寢成孔安。}{\footnotesize 丸丸,易直也。遷徙、虔敬也。梴,長貌。旅,陳也。寢,路寢也。箋云椹謂之虔。升景山,掄材木,取松柏易直者,斷而遷之,正斲於椹上,以為桷與眾楹,路寢既成,王居之甚安,謂施政教得其所也。高宗之前王有廢政教不修寢廟者,高宗復成湯之道,故新路寢焉。}

\begin{quoting}景山,見玄鳥注。\textbf{馬瑞辰}方,猶是也,或言方,或言是,互文以見參差。又曰虔,當讀如虔劉之虔,方言「虔,殺也」,淮南說林高注「殺,猶削也」,是斷是遷,是斬伐木於在山之時,方斵是虔,是削伐於作室之際。桷 \texttt{jué},方椽。梴 \texttt{chān}。旅,通鑢,磨也。\textbf{王先謙}韓說曰,閑,大也,謂閑然大也。\end{quoting}

%\begin{flushright}那五篇、十六章、百五十四句\end{flushright}