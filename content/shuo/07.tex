\chapter{識鑒第七}

\subsection*{1}

\textbf{曹公少時見喬玄,玄謂曰:「天下方亂,群雄虎爭,撥而理之,非君乎?然君實亂世之英雄,治世之姦賊,恨吾老矣,不見君富貴,當以子孫相累。」}{\footnotesize \textbf{續漢書}曰玄,字公祖,梁國睢陽人,少治禮及嚴氏春秋,累遷尚書令,玄嚴明有才略,長於知人,初,魏武帝為諸生,未知名也,玄甚異之。\textbf{魏書}曰玄見太祖曰「吾見士多矣,未有若君者,天下將亂,非命世之才不能濟也,能安之者,其在君乎」。\textbf{按}世語曰「玄謂太祖『君未有名,可交許子將』,太祖乃造子將,子將納焉」,孫盛雜語曰「太祖嘗問許子將『我何如人』,固問,然後子將答曰『治世之能臣,亂世之姦雄』,太祖大笑」,世說所言謬矣。}

\subsection*{2}

\textbf{曹公問裴潛曰:「卿昔與劉備共在荊州,卿以備才如何?」潛曰:「使居中國,能亂人,不能為治,若乘邊守險,足為一方之主。」}{\footnotesize \textbf{魏志}曰潛,字文行,河東人,避亂荊州,劉表待以賓客禮,潛私謂王粲、司馬芝曰「劉牧非霸王之才,而欲以西伯自處,其敗無日矣」,遂南渡,適長沙。}

\subsection*{3}

\textbf{何晏、鄧颺、夏侯玄並求傅嘏交,而嘏終不許,}{\footnotesize \textbf{魏略}曰鄧颺,字玄茂,南陽宛人,鄧禹之後也,少得士名,明帝時為中書郎,以與李勝等為浮華被斥,正始中,遷侍中、尚書,為人好貨,臧艾以父妾與颺,得顯官,京師為之語曰「以官易富鄧玄茂」,何晏選不得人,頗由颺,以黨曹爽誅。}\textbf{諸人乃因荀粲說合之,謂嘏曰:「夏侯太初一時之傑士,虛心於子,而卿意懷不可交,合則好成,不合則致隟,二賢若穆,則國之休,此藺相如所以下廉頗也。」}{\footnotesize \textbf{史記}曰相如以功大拜上卿,位在廉頗右,頗怒,欲辱之,相如毎稱疾,望見,引車避匿,其舍人欲去之,相如曰「夫以秦王之威而吾廷叱之,何畏廉將軍哉?顧秦彊趙弱,秦以吾二人故,不敢加兵於趙,今兩虎鬬,勢不俱生,吾以公家急而後私讐也」,頗聞,謝罪。}\textbf{傅曰:「夏侯太初志大心勞,能合虛譽,誠所謂利口覆國之人,何晏、鄧颺有為而躁,博而寡要,外好利而內無關籥,貴同惡異,多言而妬前,多言多釁,妬前無親,以吾觀之,此三賢者,皆敗德之人耳,遠之猶恐罹禍,況可親之邪?」後皆如其言。}{\footnotesize \textbf{傅子}曰是時何晏以才辯顯於貴戚之間,鄧颺好交通,合徒黨、鬻聲名於閭閻,夏侯玄以貴臣子,少有重名,皆求交於嘏,嘏不納也,嘏友人荀粲有清識遠志,然猶勸嘏結交云。}

\subsection*{4}

\textbf{晉武帝講武於宣武場,帝欲偃武修文,親自臨幸,悉召群臣,山公謂不宜爾,因與諸尚書言孫、吳用兵本意,遂究論,舉坐無不咨嗟,皆曰:「山少傅乃天下名言。」}{\footnotesize \textbf{史記}曰孫武,齊人,吳起,衛人,並善兵法。\textbf{竹林七賢論}曰咸寧中,吳既平,上將為桃林、華山之事,息役弭兵,示天下以大安,於是州郡悉去兵,大郡置武吏百人,小郡五十人,時京師猶講武,山濤因論孫、吳用兵本意,濤為人常簡默,蓋以為國者不可以忘戰,故及之。\textbf{名士傳}曰濤居魏晉之閒,無所標名,嘗與尚書盧欽言及用兵本意,武帝聞之,曰「山少傅名言也」。}\textbf{後諸王驕汰,輕遘禍難,於是寇盜處處蟻合,郡國多以無備,不能制服,遂漸熾盛,皆如公言,時人以謂山濤不學孫、吳而闇與之理會,王夷甫亦歎云:「公闇與道合。」}{\footnotesize \textbf{竹林七賢論}曰永寧之後,諸王構禍,狡虜欻起,皆如濤言。\textbf{名士傳}曰王夷甫推歎濤「晻晻為與道合,其深不可測」,皆此類也。}

\subsection*{5}

\textbf{王夷甫父乂為平北將軍,有公事,使行人論不得,時夷甫在京師,命駕見僕射羊祜、尚書山濤,夷甫時總角,姿才秀異,敘致既快,事加有理,濤甚奇之,既退,看之不輟,乃歎曰:「生兒不當如王夷甫邪?」羊祜曰:「亂天下者,必此子也。」}{\footnotesize \textbf{晉陽秋}曰夷甫父乂,有簡書,將免官,夷甫年十七,見所繼從舅羊祜,申陳事狀,辭甚俊偉,祜不然之,夷甫拂衣而起,祜顧謂賓客曰「此人必將以盛名處當世大位,然敗俗傷化者,必此人也」。\textbf{漢晉春秋}曰初,羊祜以軍法欲斬王戎,夷甫又忿祜言其必敗,不相貴重,天下為之語曰「二王當朝,世人莫敢稱羊公之有德」。}

\subsection*{6}

\textbf{潘陽仲見王敦少時,謂曰:「君蜂目已露,但豺聲未振耳,必能食人,亦當為人所食。」}{\footnotesize \textbf{晉陽秋}曰潘滔,字陽仲,滎陽人,太常尼從子也,有文學才識,永嘉末為河南尹,遇害。\textbf{漢晉春秋}曰初,王夷甫言東海王越,轉王敦為揚州,潘滔初為太傅長史,言於太傅曰「王處仲蜂目已露,豺聲未發,今樹之江外,肆其豪彊之心,是賊之也」。\textbf{晉陽秋}曰敦為太子舍人,與滔同僚,故有此言。習、孫二說,便小遷異。\textbf{春秋傳}曰楚令尹子上謂世子商臣蜂目而豺聲,忍人也。}

\subsection*{7}

\textbf{石勒不知書,}{\footnotesize \textbf{石勒傳}曰勒,字世龍,上黨武鄉人,匈奴之苗裔也,雄勇好騎射,晉元康中,流宕山東,與平原茌平人師歡家庸,耳恆聞鼓角鞞鐸之音,勒私異之,初,勒鄉里原上地中生石日長,類鐵騎之象,國中生人參,葩葉甚盛,于時父老相者皆云「此胡體貌奇異,有不可知」,勸邑人厚遇之,人多哂而不信,永嘉初,豪傑並起,與胡王陽等十八騎詣汲桑,為左前督,桑敗,共推勒為主,攻下州縣,都於襄國,後僭正號,死,諡明皇帝。}\textbf{使人讀漢書,聞酈食其勸立六國後,刻印將授之,大驚曰:「此法當失,云何得遂有天下?」至留侯諫,乃曰:「賴有此耳。」}{\footnotesize \textbf{鄧粲晉紀}曰勒不知書,目不識字,毎於軍中令人誦讀,聽之,皆解其意。\textbf{漢書}曰項羽急圍漢王於滎陽,漢王與酈食其謀撓楚權,食其勸立六國後,王令趣刻印,張良入諫,以為不可,輟食吐哺,駡酈生曰「豎儒,幾敗乃公事」,趣令銷印。}

\subsection*{8}

\textbf{衛玠年五歲,神衿可愛,祖太保曰:「此兒有異,顧吾老,不見其大耳。」}{\footnotesize \textbf{晉諸公贊}曰瓘,字伯玉,河東安邑人,少以明識清允稱,傅嘏極貴重之,謂之甯武子,仕至太保,為楚王瑋所害。\textbf{玠別傳}曰玠有虛令之秀,清勝之氣,在群伍之中,有異人之望,祖太保見玠五歲曰「此兒神爽聰令,與眾大異,恐吾年老,不及見爾」。}

\subsection*{9}

\textbf{劉越石云:「華彥夏識能不足,彊果有餘。」}{\footnotesize \textbf{虞預晉書}曰華軼,字彥夏,平原人,魏太尉歆曾孫也,累遷江州刺史,傾心下士,甚得士歡心,以不從元皇命見誅。\textbf{漢晉春秋}曰劉琨知軼必敗,謂其自取之也。}

\subsection*{10}

\textbf{張季鷹辟齊王東曹掾,在洛見秋風起,因思吳中菰菜羹、鱸魚膾,曰:「人生貴得適意爾,何能羈宦數千里以要名爵?」遂命駕便歸,俄而齊王敗,時人皆謂為見機。}{\footnotesize \textbf{文士傳}曰張翰,字季鷹,父儼,吳大鴻臚,翰有清才美望,博學善屬文,造次立成,辭義清新,大司馬齊王冏辟為東曹掾,翰謂同郡顧榮曰「天下紛紛未已,夫有四海之名者,求退良難,吾本山林間人,無望於時久矣,子善以明防前,以智慮後」,榮捉其手,愴然曰「吾亦與子採南山蕨,飲三江水爾」,翰以疾歸,榮以輒去除吏名,性至孝,遭母艱,哀毀過禮,自以年宿,不營當世,以疾終於家。}

\subsection*{11}

\textbf{諸葛道明初過江左,自名道明,名亞王、庾之下,}{\footnotesize \textbf{中興書}曰恢避難過江,與潁川荀道明、陳留蔡道明俱有名譽,號曰中興三明,時人為之語曰「京都三明各有名,蔡氏儒雅荀葛清」。}\textbf{先為臨沂令,丞相謂曰:「明府當為黑頭公。」}{\footnotesize \textbf{語林}曰丞相拜司空,諸葛道明在公坐,指冠冕曰「君當復著此」。}

\subsection*{12}

\textbf{王平子素不知眉子,曰:「志大無量,終當死塢壁間。」}{\footnotesize \textbf{晉諸公贊}曰王玄,字眉子,夷甫子也,東海王越辟為掾,後行陳留太守,大行威罰,為塢人所害。}

\subsection*{13}

\textbf{王大將軍始下,楊朗苦諫不從,遂為王致力,乘「中鳴雲露車」逕前曰:「聽下官鼓音,一進而捷。」王先把其手曰:「事克,當相用為荊州。」既而忘之,以為南郡,}{\footnotesize \textbf{晉百官名}曰朗,字世彥,弘農人。\textbf{楊氏譜}曰朗祖囂,典軍校尉,父淮,冀州刺史。\textbf{王隱晉書}曰朗有器識才量,善能當世,仕至雍州刺史。}\textbf{王敗後,明帝收朗,欲殺之,帝尋崩,得免,後兼三公,署數十人為官屬,此諸人當時並無名,後皆被知遇,于時稱其知人。}

\subsection*{14}

\textbf{周伯仁母冬至舉酒賜三子曰:「吾本謂度江託足無所,爾家有相,爾等並羅列吾前,復何憂?」周嵩起,長跪而泣曰:「不如阿母言,伯仁為人志大而才短,名重而識闇,好乘人之弊,此非自全之道,嵩性狼抗,亦不容於世,惟阿奴碌碌,當在阿母目下耳。」}{\footnotesize \textbf{鄧粲晉紀}曰阿奴,嵩之弟周謨也。三周並已見。}

\subsection*{15}

\textbf{王大將軍既亡,王應欲投世儒,世儒為江州,王含欲投王舒,舒為荊州,含語應曰:「大將軍平素與江州云何,而汝欲歸之?」應曰:「此迺所以宜往也,}{\footnotesize \textbf{晉陽秋}曰應,字安期,含子也,敦無子,養為嗣,以為武衛將軍,用為副貳,伏誅。}\textbf{江州當人彊盛時,能抗同異,此非常人所行,及覩衰厄,必興愍惻,}{\footnotesize \textbf{王彬別傳}曰彬,字世儒,琅邪人,祖覽、父正並有名德,彬爽氣出儕類,有雅正之韻,與元帝姨兄弟,佐佑皇業,累遷侍中,從兄敦下石頭,害周伯仁,彬與顗素善,往哭其屍,甚慟,既而見敦,敦怪其有慘容而問之,答曰「向哭周伯仁,情不能已」,敦曰「伯仁自致刑戮,汝復何為者哉」,彬曰「伯仁清譽之士,有何罪」,因數敦曰「抗旌犯上,殺戮忠良」,音辭忼慨,與涙俱下,敦怒甚,丞相在坐,代為之解,命彬曰「拜謝」,彬曰「有足疾,比來見天子尚不能拜,何跪之有」,敦曰「腳疾何如頸疾」,以親故不害之,累遷江州刺史、左僕射,贈衛將軍。}\textbf{荊州守文,豈能作意表行事?」含不從,遂共投舒,舒果沈含父子於江,}{\footnotesize \textbf{王舒傳}曰舒,字處明,琅邪人,祖覽,知名,父會,御史,舒器業簡素,有文武幹,中宗用為北中郎將、荊州刺史、尚書僕射,出為會稽太守,以父名會,累表自陳,討蘇峻有功,封彭澤侯,贈車騎大將軍。}\textbf{彬聞應當來,密具船以待之,竟不得來,深以為恨。}{\footnotesize 含之投舒,舒遣軍逆之,含父子赴水死,昔酈寄賣友見譏,況販兄弟以求安,舒非人矣。}

\subsection*{16}

\textbf{武昌孟嘉作庾太尉州從事,已知名,褚太傅有知人鑒,罷豫章還,過武昌,問庾曰:「聞孟從事佳,今在此不?」庾云:「卿自求之。」褚眄睞良久,指嘉曰:「此君小異,得無是乎?」庾大笑曰:「然。」于時既歎褚之默識,又欣嘉之見賞。}{\footnotesize \textbf{嘉別傳}曰嘉,字萬年,江夏鄳人,曾祖父宗,吳司空,祖父揖,晉廬陵太守,宗葬武昌陽新縣,子孫家焉,嘉少以清操知名,太尉庾亮領江州,辟嘉部廬陵從事,下都還,亮引問風俗得失,對曰「待還,當問從事吏」,亮舉麈尾掩口而笑,語弟翼曰「孟嘉故是盛德人」,轉勸學從事,太傅褚裒有器識,亮正旦大會,裒問亮「聞江州有孟嘉,何在」,亮曰「在坐,卿但自覓」,裒歷觀久之,指嘉曰「將無是乎」,亮欣然而笑,喜裒得嘉,奇嘉為裒所得,乃益器之,後為征西桓溫參軍,九月九日溫遊龍山,參寮畢集,時佐史並著戎服,風吹嘉帽墮落,溫戒左右勿言,以觀其舉止,嘉初不覺,良久如廁,命取還之,令孫盛作文嘲之,成,著嘉坐,嘉還即答,四坐嗟歎,嘉喜酣暢,愈多不亂,溫問「酒有何好?而卿嗜之」,嘉曰「明公未得酒中趣爾」,又問「聽伎,絲不如竹,竹不如肉,何也」,答曰「漸近自然」,轉從事中郎,遷長史,年五十三而卒。}

\subsection*{17}

\textbf{戴安道年十餘歲,在瓦官寺畫,王長史見之曰:「此童非徒能畫,}{\footnotesize \textbf{續晉陽秋}曰逵善圖畫,窮巧丹青也。}\textbf{亦終當致名,恨吾老,不見其盛時耳。」}

\subsection*{18}

\textbf{王仲祖、謝仁祖、劉真長俱至丹陽墓所省殷揚州,殊有確然之志,}{\footnotesize \textbf{中興書}曰浩棲遲積年,累聘不至。}\textbf{既反,王、謝相謂曰:「淵源不起,當如蒼生何?」深為憂歎,劉曰:「卿諸人真憂淵源不起邪?」}

\subsection*{19}

\textbf{小庾臨終,自表以子園客為代,}{\footnotesize 園客,爰之小字也。\textbf{庾氏譜}曰爰之,字仲真,翼第二子。\textbf{中興書}曰爰之有父翼風,桓溫徙於豫章,年三十六而卒。}\textbf{朝廷慮其不從命,未知所遣,乃共議用桓溫,劉尹曰:「使伊去,必能克定西楚,然恐不可復制。」}{\footnotesize \textbf{陶侃別傳}曰庾翼薨,表其子爰之代為荊州,何充曰「陶公重勳也,臨終高讓,丞相未薨,敬豫為四品將軍,于今不改,親則道恩,優游散騎,未有超卓若此之授」,乃以徐州刺史桓溫為安西將軍、荊州刺史。\textbf{宋明帝文章志}曰翼表其子代任,朝廷畏憚之,議者欲以授桓溫,時簡文輔政,然之,劉惔曰「溫去必能定西楚,然恐不能復制,願大王自鎮上流,惔請為從軍司馬」,簡文不許,溫後果如惔所算也。}

\subsection*{20}

\textbf{桓公將伐蜀,在事諸賢咸以李勢在蜀既久,承藉累葉,且形據上流,三峽未易可克,惟劉尹云:「伊必能克蜀,觀其蒲博,不必得,則不為。」}{\footnotesize \textbf{華陽國志}曰李勢,字子仁,洛陽臨渭人,本巴西宕渠賨人也,其先李特,因晉亂據蜀,特子雄,稱號成都,勢祖驤,特弟也,驤生壽,壽篡位自立,勢即壽子也,晉安西將軍伐蜀,勢歸降,遷之揚州,自起至亡,六世三十七年。\textbf{溫別傳}曰初,朝廷以蜀處險遠,而溫眾寡少,縣軍深入,甚以憂懼,而溫直指成都,李勢面縛。\textbf{語林}曰劉尹見桓公毎嬉戲必取勝,謂曰「卿乃爾好利,何不焦頭」,及伐蜀,故有此言。}

\subsection*{21}

\textbf{謝公在東山畜妓,簡文曰:「安石必出,既與人同樂,亦不得不與人同憂。」}{\footnotesize \textbf{宋明帝文章志}曰安縱心事外,疎略常節,毎畜女妓,攜持遊肆也。}

\subsection*{22}

\textbf{郗超與謝玄不善,苻堅將問晉鼎,既已狼噬梁岐,又虎視淮陰矣,}{\footnotesize \textbf{車頻秦書}曰苻堅,字永固,武都氐人也,本姓蒲,祖父洪,詐稱讖文,改曰『符』,言己當王,應符命也,堅初生,有赤光流其室,及誕,背赤色隱起,若篆文,幼有美度,石虎司隸徐正名知人,堅六歲時,嘗戲於路,正見而異焉,問曰「符郎,此官街,小兒行戲,不畏縛邪」,堅曰「吏縛有罪,不縛小兒」,正謂左右曰「此兒有王霸相」,石氏亂,伯父健及父雄西入關,健夢天神使者朱衣冠,拜肩頭為龍驤將軍,肩頭,堅小字也,健即拜為龍驤,以應神命,後健僭帝號,死,子生立,凶暴,群臣殺之而立堅,堅立十五年,遣長樂公丕攻沒襄陽,十九年,大興師伐晉,眾號百萬,水陸俱進,次於項城,自項城至長安連旗千里,首尾不絕,乃遣告晉曰「已為晉君於長安城中建廣夏之室,今故大舉渡江相迎,克日入宅也」。}\textbf{于時朝議遣玄北討,人間頗有異同之論,惟超曰:「是必濟事,吾昔嘗與共在桓宣武府,見使才皆盡,雖履屐之間,亦得其任,以此推之,容必能立勳。」元功既舉,時人咸歎超之先覺,又重其不以愛憎匿善。}{\footnotesize \textbf{中興書}曰于時氐賊彊盛,朝議求文武良將可鎮靖北方者,衛大將軍安曰「惟兄子玄可任此事」,中書郎郗超聞而歎曰「安違眾舉親,明也,玄必不負其舉」。}

\subsection*{23}

\textbf{韓康伯與謝玄亦無深好,玄北征後,巷議疑其不振,康伯曰:「此人好名,必能戰。」}{\footnotesize \textbf{續晉陽秋}曰玄識局貞正,有經國之才略。}\textbf{玄聞之甚忿,常於眾中厲色曰:「丈夫提千兵、入死地,以事君親故發,不得復云為名。」}

\subsection*{24}

\textbf{褚期生少時,謝公甚知之,恆云:「褚期生若不佳者,僕不復相士。」}{\footnotesize 期生,褚爽小字也。\textbf{續晉陽秋}曰爽,字茂弘,河南人,太傅裒之孫,祕書監韶之子,太傅謝安見其少時,歎曰「若期生不佳,我不復論士」,及長,果俊邁有風氣,好老莊之言,當世榮譽,弗之屑也,惟與殷仲堪善,累遷中書郎、義興太守,女為恭帝皇后。}

\subsection*{25}

\textbf{郗超與傅瑗周旋,瑗見其二子,並總髮,超觀之良久,謂瑗曰:「小者才名皆勝,然保卿家,終當在兄。」即傅亮兄弟也。}{\footnotesize \textbf{傅氏譜}曰瑗,字叔玉,北地靈州人,歷護軍長史、安城太守。\textbf{宋書}曰迪,字長猷,瑗長子也,位至五兵尚書,贈太常。\textbf{丘淵之文章錄}曰亮,字季友,迪弟也,歷尚書令、左光祿大夫,元嘉三年,以罪伏誅。}

\subsection*{26}

\textbf{王恭隨父在會稽,王大自都來拜墓,}{\footnotesize 恭父蘊、王忱並已見。}\textbf{恭暫往墓下看之,二人素善,遂十餘日方還,父問恭:「何故多日?」對曰:「與阿大語,蟬連不得歸。」因語之曰:「恐阿大非爾之友,終乖愛好。」果如其言。}{\footnotesize 忱與恭為王緒所間,終成怨隙,別見。}

\subsection*{27}

\textbf{車胤父作南平郡功曹,太守王胡之避司馬無忌之難,置郡於酆陰,是時胤十餘歲,胡之毎出,嘗於籬中見而異焉,謂胤父曰:「此兒當致高名。」後遊集,恆命之,胤長,又為桓宣武所知,清通於多士之世,官至選曹尚書。}{\footnotesize \textbf{續晉陽秋}曰胤,字武子,南平人,父育,為郡主簿,太守王胡之有知人識,裁見,謂其父曰「此兒當成卿門戶,宜資令學問」,胤就業恭勤,博覽不倦,家貧不常得油,夏月則練囊盛數十螢火以繼日焉,及長,風姿美劭,機悟敏率,桓溫在荊州取為從事,一歲至治中,胤既博學多聞,又善於激賞,當時毎有盛坐,胤必同之,皆云「無車公不樂」,太傅謝公遊集之日,開筵以待之,累遷丹陽尹、護軍將軍、吏部尚書。}

\subsection*{28}

\textbf{王忱死,西鎮未定,朝貴人人有望,時殷仲堪在門下,雖居機要,資名輕小,人情未以方嶽相許,晉孝武欲拔親近腹心,遂以殷為荊州,事定,詔未出,王珣問殷曰:「陝西何故未有處分?」殷曰:「已有人。」王歷問公卿,咸云「非」,王自計才地必應在己,復問:「非我邪?」殷曰:「亦似非。」其夜詔出用殷,王語所親曰:「豈有黃門郎而受如此任?仲堪此舉迺是國之亡徵。」}{\footnotesize \textbf{晉安帝紀}曰孝武深為晏駕後計,擢仲堪代王忱為荊州,仲堪雖有美譽,議者未以方嶽相許也,既受腹心之任,居上流之重,議者謂其殆矣,終為桓玄所敗。}