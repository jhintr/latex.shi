\section{犀牛角经}

\begin{center}Khaggavisāṇa Sutta\end{center}\vspace{1em}

\begin{enumerate}\item 缘起为何?对一切经,有四种缘起:由自身的意乐、由他人的意乐、由事件的发生、由问题的主导。二重随观经等的缘起是由自身的意乐,慈经等由他人的意乐,蛇经等由事件的发生,如法经等由问题的主导。这里,犀牛角经的缘起,一般来说是由问题的主导,但区别来说,因为其中的某颂是由某某辟支佛在被问及时所说,某颂则未被问及,唯发出与自己所证的道法相应的慨叹\footnote{慨叹 \textit{udāna}:旧译作「自说」。},所以,某些颂的缘起是由问题的主导,某些则由自身的意乐。
\item 这里,这一般来说由问题的主导的缘起,从最初开始,当知如是:一时,世尊住舍卫国。尔时,尊者阿难独居宴坐,心中生起这样的想法:「诸佛的愿求与志向已显,声闻众的也同样,而诸辟支佛的则未显,我何不前往世尊处询问?」他便从宴坐起,前往世尊处,逐步问了此义。于是,世尊对他说了「宿行所行经」\footnote{宿行所行经:菩提比丘注 368 云,此经未见于三藏。}:\begin{quoting}阿难!于宿行所行,有此五功德:即于现法中提前成就已知,若未于现法中提前成就已知,则于死时成就已知,若未于死时成就已知,则成天子成就已知,或在诸佛现前时能迅速证知,或在最后时成辟支佛。\end{quoting}如是说已,复说:\begin{quoting}阿难!诸辟支佛具足志向、宿行所行,所以,当知一切佛、辟支佛、声闻等的愿求与志向。\end{quoting}他说:「尊者!诸佛的愿求多久才能圆满?」对于诸佛,阿难!最低限度是四阿僧祇又十万劫,中等限度是八阿僧祇又十万劫,最上限度是十六阿僧祇又十万劫,当知这些区分是依慧主导、信主导、精进主导。因为对慧主导者,信钝而慧利,对信主导者,慧中而信有力,对精进主导者,信、慧钝而精进有力。若未圆满四阿僧祇又十万劫,即便日日布施,与毗输安多罗的布施相等,即便积累与之相应的戒等一切波罗蜜法,而于中间成佛,无有是处。为什么?智未得胎、未至广大、未到成熟。好比经三月、四月、五月方成熟的谷物,若未至彼彼时,即便日日照料、灌水千遍,而于中间经半月、一月成熟,无有是处。为什么?谷未得胚、未至广大、未到成熟。如是,若未圆满四阿僧祇……无有是处。所以,为了智的成熟,应以所说的时间圆满波罗蜜。
\item 且以如是时间愿求佛性者,于志向之施行,应需八事成就,即:\begin{quoting}人身,性别之成就,因,见大师,\\出家,功德之成就,服务以及欲,\\由八法的汇合,志向得成。(佛种姓第 2:59 颂)\end{quoting}志向,即根本誓愿的同义语。这里,\textbf{人身},即以人出生。因为除了以人出生外,对于以其余出生者,即便以天出生,誓愿也不得成功,住立于此而愿求佛性者,在作了布施等福业后,唯应愿求人身,住立于此后当行誓愿,如是则能成功。\textbf{性别之成就},即男性。因为对女性、非男性、具两性特征者,即便以人出生,誓愿也不得成功,住立于此而愿求佛性者,在作了布施等福业后,唯应愿求男性,住立于此后当发誓愿,如是则能成功。\textbf{因},即阿罗汉的近依成就。因为对于在此自体中发起精进便能圆满阿罗汉者,则能成功,而非其他,好比善慧智者。因为他在燃灯足下出家已,以此自体便能圆满阿罗汉。\textbf{见大师},即面见诸佛,如是则能成功,而非其他,好比善慧智者。因为他当面见了燃灯而发誓愿。\textbf{出家},即非在家的身份,他已在教法内,或已在业论、作论的苦行者、游行者之列,好比善慧智者。因为他在成了名为善慧的苦行者而发誓愿。\textbf{功德之成就},即获得禅那等功德。因为唯有具足功德的出家人能成功,而非其他,好比善慧智者。他成五神通、得八等至已而发誓愿。\textbf{服务}\footnote{服务 \textit{adhikāra}:旧译作「供养」。},即主导行、遍舍之义。因为唯有遍舍性命等已而发誓愿者能成功,而非其他,好比善慧智者。因为他\begin{quoting}请佛陀与学生们脚踏着我而前行!\\莫踩在泥泞上!这将是我的利益。(佛种姓第 2:53 颂)\end{quoting}如是遍舍性命已而发誓愿。\textbf{欲},即愿作,其有力者能成功。且如果有人说「谁在地狱中受苦四阿僧祇又十万劫后,还想成佛」,若听到后敢说「我」者,当知则为有力,同样,如果有人说「谁在踏过填满无烟炭火的整个轮围后,还想成佛?谁在踏过铺满剑矛的整个轮围后,还想成佛?谁在度过盛满水的整个轮围后,还想成佛?谁在踏过覆以无间竹林的整个轮围后,还想成佛」,若听到后敢说「我」者,当知则为有力,而具足这样愿作之欲的善慧智者便发了誓愿。
\item 如是志向得遂的菩萨,不得此十八失能之处。即他从此以后,不会生盲,不会生聋,不疯,不傻,不跛,不投生蛮夷间,不入婢女之胎,不为定邪见者,不改变其性别,不造五无间业,不患麻风,于畜生中则身量不比鹌鹑小、不比大象大\footnote{不比大象大:缅甸本无,据 PTS 本补。},不投生于饥渴、焦灼的饿鬼,不于 Kālakañcika 阿修罗,不于无间地狱,不于世间之间隔\footnote{世间之间隔:即世间与世间之间。},于诸欲界不为魔罗,于诸色界不投生无想有、诸净居有,不于无色界,不跨至其它轮围。
\item 且他具足此四佛地,即勇猛、洞察、稳定、利益行。这里,当知\begin{quoting}勇猛即精进,洞察即是慧,\\稳定即决意,利益行即慈修习。\end{quoting}且他具足六种导向菩提成熟的意乐,即由具足离欲意乐,菩萨见爱欲的过失,由具足远离意乐,菩萨见聚会的过失,由具足无贪意乐,菩萨见贪的过失,由具足无嗔意乐,菩萨见嗔的过失,由具足无痴意乐,菩萨见痴的过失,由具足出离意乐,菩萨见一切有的过失。
\item 那么,诸辟支佛的愿求多久才能圆满?对于诸辟支佛,为二阿僧祇又十万劫,少于此则不可能。此处的原因当知即如前所述。且以如是时间愿求辟支佛性者,于志向之施行,应需五事成就,即:\begin{quoting}人身,性别之成就,得见离漏,\\服务以及欲,这些即志向之因。\end{quoting}这里,\textbf{得见离漏},即得见佛、辟支佛、声闻中的任何一个之义。其余唯如前述。
\item 那么,诸声闻的愿求多久才能圆满?二上首弟子为一阿僧祇又十万劫,八十大弟子为十万劫,佛父母、侍者、儿子也同样,少于此则不可能。此处的原因当知即如前所述。且对所有这些人,唯服务、欲二支具足者,志向得遂。
\item 如是,以此愿求、以此志向,在如所说品类的时间内圆满了波罗蜜后,诸佛投生于世间,或投生于刹帝利家,或于婆罗门家,诸辟支佛则于刹帝利、婆罗门或长者家中的某处,而上首弟子则唯于刹帝利或婆罗门家,如同诸佛。一切诸佛不于劫坏时投生,而于劫成时投生,诸辟支佛未值诸佛,唯于诸佛投生之时(的无间)投生\footnote{唯于诸佛投生之时(的无间)投生:菩提比丘注 379,各本均作「唯于诸佛投生之时投生」,唯在\textbf{譬喻} \textit{Apadāna} 的义注中作「唯于诸佛投生之时不投生」。译文据 PTS 所列的 B\textsuperscript{a} 本补充「无间 \textit{anantarā}」。}。
\item 诸佛由自己觉悟,且令他人觉悟,诸辟支佛唯由自己觉悟,而不能令他人觉悟,唯通达义味,而非法味。因为他们不能将出世间法提升为概念去开示,他们的法现观像是哑巴作了梦,又像是护林人进城尝了调味一般。他们圆满一切神变、等至、无碍解等,在功德的殊胜上,较诸佛低,较诸声闻高,度他人出家后,令修学等正行,以「应损减心,不应中止」等的诵说行布萨,或者仅说「今天是布萨」,且行布萨时,在香醉山的曼珠沙树下采集了宝鬘后而行。
\item 如是,世尊以一切行相对尊者阿难讲述了诸辟支佛的愿求与志向,现在,为了讲述这些以此愿求、以此志向而成的诸辟支佛,而以「对一切生物放下了棍杖」等方法说了这犀牛角经。这便是犀牛角经中一般来说由问题的主导的缘起。\end{enumerate}

\subsection\*{\textbf{35}}

\textbf{对一切生物放下了棍杖,也不恼害其中的某个,\\}
\textbf{不会希求孩子,遑论朋友?他应当独自游行,像犀牛角一样。}

Sabbesu bhūtesu nidhāya daṇḍaṃ, aviheṭhayaṃ aññataram pi tesaṃ;\\
na puttam iccheyya kuto sahāyaṃ\footnote{此句在\textbf{譬喻}第 1:90 颂中作「以慈心哀愍者 \textit{mettena cittena hitānukampī}」,语意与上两句更贴合。}, eko care khaggavisāṇakappo. %\hfill\textcolor{gray}{\footnotesize 1}

\begin{enumerate}\item 现在,当依区别来说。这里,此颂的缘起当知如是。据说,当这辟支佛沉潜于辟支菩萨地时,以二阿僧祇又十万劫圆满了波罗蜜后,在迦叶世尊的教法内出家,成了林居者,履行往还的义务而行沙门法。据说,未完全履行此义务便能圆满辟支菩提者是不存在的。那这\textbf{往还的义务}是什么?即带去与带回。我们将把它讲述清楚。于此,有比丘带去而不带回,或带回而不带去,或不带去也不带回,或带去又带回。
\item 这里,若比丘早起后,作了支提的庭院与菩提树的庭院的义务,给菩提树浇了水,装满饮水缸,置于饮水房,作了阿阇黎的义务与亲教师的义务,受持了八十二小义务与十四大义务而行,他在洗漱后,进入坐卧处,于独处的坐处等待行乞的时间,知时已到,便著下衣、扎腰带、穿上衣,将僧伽梨置于肩、将钵挂在一边后,作意于业处,先到支提的庭院,顶礼了支提与菩提树,在村附近披覆好衣,持钵入村乞食,如是进入的有利养、具福德的比丘为众优婆塞恭敬、尊重,回到护持家中或返程堂,当被众优婆塞问到各种问题时,因回答他们的问题,以及因开示法而起的散乱,丢掉了这作意而返,甚至回到寺庙还要为诸比丘讲述所问的问题、说法、遇到种种事务,在下午、初夜、中夜仍如是为诸比丘所耽搁,在后夜便不胜身体的粗重而入睡,而未作意于业处,这被称为\textbf{带去而不带回}。
\item 若他常患疾病,所吃的食物在拂晓时分仍未完全消化,早起后,不能去做所说的义务或作意于业处,而是希求着粥或药,到了时间便持衣钵入村。在那里得了粥、药或食物后,完结食事,坐在备好的坐处,作意于业处,不论得或未得殊胜,回到寺庙后,仍以此作意而住,这被称为\textbf{带回而不带去}。像这样喝了粥开始作观、在佛教内证阿罗汉的比丘,不可胜数。就在僧伽罗岛上,在各处村落中的坐堂内,无处不是比丘喝了粥而证阿罗汉的。
\item 若住于放逸,搁置重担,断绝一切义务,心为五种心的荒秽之结缚所缚而住,不从事业处之作意,入村乞食,为俗家的戏论耽搁,空空而返,这被称为\textbf{不带去也不带回}。
\item 若早起后,也以先前的方法履行了一切义务,直至行乞之时,结跏趺坐,作意于业处。业处有两种:一切处者与照顾者\footnote{两种业处,见\textbf{清净道论}·说取业处品第 57 段及以下。照顾业处,叶均译作「应用业处」。}。一切处者,即慈与死念,由其为一切处所需,故被称为\textbf{一切处者}。慈于居处等一切处需要。因为比丘对居处住于慈,为同梵行者喜爱,因此能无妨碍而安住,对诸天住于慈,为诸天守卫保护而乐住,对王、王大臣等住于慈,为其爱护而乐住,对村镇等住于慈,为一切行乞等处的人们恭敬、尊重而乐住。由修习死念,便舍弃了活命的欲求,不放逸而住。
\item 而由为随顺性行者所执取,应当始终照顾的十不净、遍、随念中的某个,或四界差别,由应当始终照顾、保护、修习,故被称为\textbf{照顾者},此即根本业处。这里,首先作意于一切处业处,而后作意于照顾业处,我们将以四界差别\footnote{四界差别,见\textbf{清净道论}·说定品第 27 段及以下。}为例说明之。
\item 他以界观察如是住立、如是划分的身体:在此身中,二十分中粗重的坚硬者为地界,十二分中起凝结作用的湿润者为水界,四分中致遍熟的热者为火界,六分中起支持的风者为风界,而于此不为四大种所触的间隙、空开为空界,了别此的心为识界,在此之外,没有其他的有情或人,仅仅是单纯的行的积聚。
\item 如是从初、中、后作意于业处,知时已到,从坐起,著下衣,以先前所说的方法到村中乞食。且在行进时,好比愚痴的凡夫迷惑于前进等:「自我在前进,前进由自我发生」或「我在前进,前进由我发生」,如是未起迷惑者:\begin{quoting}所谓「我前进」,即在心生起时,与此心俱的心等起的支持之风界生起,它遍满这由地界等排布而成的、共许为身体的骨聚,随后,由心造作的风界的遍满,这共许为身体的骨聚前进。对如是前进者,当一步步提起时,四界之中与风界随行的火界增盛,其余则弱,而移步、换步、离步时,与火界随行的风界增盛,其余则弱,而落步时,与地界随行的水界增盛,其余则弱,在放置、碾压时,与水界随行的地界增盛,其余则弱。如是,这些界,与各各自身生起的心一起,毁于处处。于此,哪个在前进?或者,前进是哪个的?如是,于一步步提起等的行相中,于一一行相中生起的界及其它不与之分离者为色法,使之等起的心及与之相应者为非色法,这些色、非色法此后不得至于移步、换步等的别的行相,唯毁于处处,故而为无常,而无常者即苦,苦者即无我。\end{quoting}如是作意于一切行相遍满的业处而行。
\item 欲求义利的族姓子在教法内出家已,在与十、二十、三十、四十、五十、六十、七十、百人一起居住时,订立规约而住:「朋友!你们不是由负债、怖畏、迫于生计而出家,而是于此欲解脱于苦而出家,所以,在行走中生起的烦恼唯应在行走时抑止,在站立、安坐、横卧中生起的烦恼唯应在横卧时抑止。」他们如是订立规约后,前去行乞,在半牛、一牛、半牛呼、一牛呼\footnote{牛 \textit{usabha}、牛呼 \textit{gāvuta} 都是长度单位,一牛为二十杖 \textit{yaṭṭhi} 或一百四十肘 \textit{hattha},一牛呼为八十牛或四分之一由旬。}间立有岩石,他们以此标记作意于业处而行。如果有人在行走时生起烦恼,即于此抑止之,不能如是者则站立,且从他后面而来者也站立。他自责「这比丘知道你已生起的寻,这对你是不适当的」,增长毗婆舍那已,即于此处跃入圣地,不能如是者则坐下,且从他后面而来者也坐下。他仍以此法。不能跃入圣地者,则镇伏此烦恼后,作意于业处而行。\footnote{此一大段 PTS 本阙如。}
\item 他不以离于业处的心而举足,若举足则返回,行至原处,如僧伽罗岛上\textbf{住在门廊的大弗沙天长老}一般。据说,他十九年履行往还的义务而住。人们沿路边耕作、播种、脱粒、作业时,看到长老这般行走,便会闲谈:「这长老一再返回来走,难道是路痴,或是忘了什么?」他不理会这些,唯以与业处相应的心行沙门法,在二十年间圆满了阿罗汉。在他证得阿罗汉的那天,居于经行道终点的天人以手指燃灯而立,四大王与诸天之因陀帝释、娑婆主梵天也前来护持。住在林中的大低舍长老看到这光后,第二天问他:「夜间在尊者的跟前有光,那是什么光?」长老混淆着说「光有灯光,也有摩尼光」等等。他催问「你在覆藏吗」,便承认了「是」而告知。又如\textbf{住在黑葛亭的大龙长老}一般。据说,当他履行往还的义务时,以「我当先供养世尊的大精勤」决意于七年间唯站立与经行,又在履行了十六年往还的义务后而证了阿罗汉。
\item 如是以唯与业处相应的心举足时,当以离相应的心而举即返回,去到村的附近,站在「是母牛还是出家人」的可疑之处,披覆好僧伽梨,持好钵,到达村口,从腋下的水瓶中取出水,含上一口再入村,「对前来布施食物或礼拜的人们,莫让我的业处因说『愿你们长寿』而散乱」。但如果他们问日子:「尊者!今天是七号还是八号?」则吞下水告知,如果没人问日子,则在离开时在村口吐出再走。
\item 又如僧伽罗岛上\textbf{在团花津寺中入雨安居的五十比丘}一般。据说,他们在入雨安居的布萨日订了规约:「未证得阿罗汉,我们将不彼此交谈。」入村乞食时,在村口含上口水而入,当被问日子时,则吞下水告知,若未被问,则在村口吐出再回到寺庙。于此,人们看到吐水处就知道「今天来了一个,今天两个」,便如是想「他们只是不和我们交谈,还是彼此间也如此?如果彼此间也不交谈,那肯定会起争论,我们快去让他们彼此原谅」,全都去了寺庙。于此,在进入雨安居的五十比丘中,他们未见到有两个比丘在同一场所。随后,他们中的具眼者便如是说:「先生!发生争辩的居处不是这样的,支提的庭院、菩提树的庭院洒扫干净,扫帚安放妥当,饮用水、洗净水善加准备。」于是他们便返回。这些比丘在三个月里开始修观,证了阿罗汉,在大自恣日举行了清净的自恣。
\item 如是,如住在黑葛亭的大龙长老及在团花津寺中入雨安居的诸比丘一般,唯以与业处相应的心举足,到达村子附近后,含上口水,观察好路线,行于无醉酒的恶人等、无争吵及无恶象恶马等处。且于此不要匆匆速行,没有名为速乞食者的头陀支,而应稳如满荷的水车至于不平之地般前行。他逐户到访,适时地等待,以观察愿不愿意布施,得了食物后,在合适的场所坐下,作意于业处,对食物起厌恶想,以给车轴加油、给伤口敷药、子息之肉等譬喻省察,食用八支具足的食物:不为嬉戏……,已受用者洗毕,稍稍解去食事之乏,如在饭前,如是在饭后、初夜及后夜作意于业处。这被称为\textbf{带去又带回}。如是,这带去又带回被称为往还的义务。
\item 履行此者,若具足近依,则在早年即证阿罗汉,若未在早年证得,则在中年证得,若未在中年证得,则在死时证得,若未在死时证得,则成为天子证得,若成为天子未证得,则成为辟支等觉而般涅槃,若未成为辟支等觉而般涅槃,则在诸佛跟前成速通达,如跋西耶长老,或成大慧者,如舍利弗长老。
\item 然而,这位辟支菩萨在迦叶世尊的教法内出家成林居者后,履行了二万年往还的义务,死后投生于欲界的天界,从此殁后,在波罗奈王正妃的胎内获取结生。善良的女人们在当天就会知道怀胎,而她是其中一个,所以就把住胎告知了国王。这是法性,即当具福的有情投胎时,女人会得到胎的照顾。所以,国王便给予她胎的照顾。从此以后,她不能吃过热的东西,或过冷、过酸、过咸、过辣、过苦的。因为当母亲吃过热的,胎儿如住在铜釜(地狱)一般,当过冷,如住在世间之间隔一般,当食用过酸、过咸、过辣、过苦的,胎儿的肢体会有猛烈的感受,如被剑劈开,灌入酸等一般。他们也防止她过度的行、住、坐、卧:「莫让得胎者受扰动之苦!」仅可以在铺以柔毯的地上经行等,获得色香等具足、味道适宜的饮食,围绕住她,才让她行、坐、起身。
\item 她经如是照顾,当临盆时,进入产房,在黎明时分便生下孩子,如摩以素馨油的雄黄丸一般,具足财福之相。随后,在第五天,经装饰、装扮后呈现给国王,国王心满意足,让六十六个乳母去服侍他。他以一切成就成长,不久即达青春。刚满十六岁,国王便灌顶以王位,并让种种舞女侍奉他。灌顶后的王子以「梵赐」之名统治了整个阎浮提二万座城市。因为阎浮提先前有八万四千座城市,减损至六万,再减损至四万,当一切减损之时仅有二万。而梵赐便在一切减损之时投生,因此他有二万座城市、二万座宫殿、二万匹象、二万匹马、二万乘车、二万步兵、二万侍女舞女以及二万大臣。
\item 当他尚在统治大王国时,即作了遍的遍作,生起五神通与八等至。而因为所谓的灌顶王难免要坐而听讼,所以有天早上用了早餐,他便坐在裁断处。于此即起高声、大声。他以「这声音是等至的随烦恼」,上到楼顶,以「我要证等至」而坐,竟不得证,等至因王权的散乱而退失。随后,他想「是王权更胜,还是沙门法」,了知到「王权之乐有限,过患非一,而沙门法之乐广大,功德非一,且为上人所从事」,便命令某大臣:「如法、公正地治理这王国!莫行非法!」授予一切后,上到楼阁,以等至之乐而住,除了送漱口的齿木者及送饭者,任何人不得接近。
\item 半月过后,王后问到:「在庭院游玩处、视察军队处、舞女处,哪里都见不到国王,他去哪里了?」他们便告知她此事。她便让人送信给大臣:「当你接受王国之时,也接受了我,来!与我共住!」大臣盖住双耳,拒绝道「这不可听」。她又再三派人,仍不愿意,便恐吓他:「如果你不做,我就让你下台,还会夺你性命!」他害怕道「女人就是决心坚定的代名词,有时候真会这么做」,一天就前去幽会,与她在寝宫共住。她雍容华贵,肌理细腻,他为她的抚触所染,心怀疑惧,频繁地前往,渐渐地,便开始如家主一般,无所畏惧地出入。
\item 随后,臣僚们便将此事告知国王。国王不相信。他们又再三地告知。随后,在藏匿亲见后,他召集了所有大臣并告知。他们说「这欺君者应当断手断脚」,提出从插在矛上等的一切刑罚。国王说:「对他捆缚砍斫,我会生起恼害,若取其性命,即是杀生,夺其财产,即是不与取,这样的事还是算了,将他驱逐出我的国家吧!」大臣们便驱逐了他。
\item 他带了自己的财宝妻儿,到了别的领土。那里的国王听闻后便问:「你来做什么?」「大王!我希望侍奉您。」他便领受了。过了几天,大臣受到亲信,便对那国王说:「大王!我看见无蜂之蜜,还没有人吃过。」国王想「难道他是心存戏弄而说的」,便没有听。他得了空,又再次告知,善加解释,国王问:「那是什么?」「大王!波罗奈的国土。」国王说:「把我领去后,你想谋害我吧?」他说:「大王!别这样说!如果你不相信,可以派遣人去。」他便派遣了人。他们去后,挖通了门阙,得从国王的寝室出来。
\item 国王看见后,问:「你们为何而来?」「大王!我们是贼。」国王让人给了他们钱财,教诫道「别再这样做」,即予遣散。他们回来后告知了那国王。他又再三如是打探,想「国王是具戒者」,武装好四支军队,前往国界间的某城,在此让人送信给大臣:「交城给我,或者交战!」他命人将此事告知梵赐:「请大王发令!我是交战,还是交城?」国王遣使说:「不应交战,交城后回到这里!」他便照做。敌王夺了此城后,在其余诸城也同样派使者送信,那些大臣们也同样告知梵赐,也以「不应交战,应回到这里」而返回了波罗奈。
\item 随后,大臣们对梵赐说:「大王!我们要与之一战!」国王便以「那我就要杀生了」遮止。大臣们以「大王!我们活捉了之后,带到这里来」等种种方法劝说国王,催促前往:「来!大王!」国王说:「如果你们不造杀伐劫掠有情的业,我就来。」大臣们说「我们不造,大王!展示了军威,把他们吓退」,武装好四支军队后,把灯装进水罐里,在夜间到达。
\item 敌王在这天夺取了波罗奈附近的城市后,想「现在还有什么」,在夜间解了甲胄,与部伍一起恣意入睡。随后,大臣们带了波罗奈王,来到敌王的营地,从所有罐子里取出灯,以一片光明的军队发出声响。敌王的大臣看到了大军而惧怕,跑向自己的国王大声说:「起来!吃无蜂之蜜!」第二、第三个(大臣)也同样。敌王因这声响醒来而怖畏、战栗。数百呼喊生起。
\item 他整晚都在悲泣「信了别人的话,我落到敌人手里了」,第二天想「国王是如法者,不行毁伤,我前去请求原谅」,前往国王处,双膝跪地,说:「大王!请原谅我的罪过!」国王教诫他后,说:「起来!我原谅你。」当国王如是说完,他即得了极大的安慰,便在波罗奈国附近的国家得了统治。他们彼此成了朋友。
\item 于是,梵赐看到两军站在一起相庆,想「当我有一念心的守护,即于众人未出小蝇所歃之量的血滴,哦!善哉!哦!好极!愿一切有情快乐!愿他们无敌对!愿他们无嗔害」,增长了慈禅,便以此为基础,触知诸行,证得辟支菩提之智,圆满了自成性\footnote{自成性 \textit{sayambhuta}:菩提比丘注 401,该词仅用于等正觉与辟支佛。}。
\item 他乐于道果之乐,坐于象背,大臣们跪拜后说:「该走了,大王!胜军当受恭敬,败军当给予伙食的费用。」他说:「我不是国王,我说,我叫辟支佛。」「大王说什么呢?没有这样的辟支佛。」「那么,我说,辟支佛是什么样的?」「辟支佛者,须发二指,俱八资具\footnote{八资具:菩提比丘注 402,也许和比丘的相同,即三衣、腰带、钵、剃刀、水滤、针线。}。」他以右手摩头,俗家相便即刻退去,现起出家装,须发二指,俱八资具,如百岁的长老一般。
\item 他既入四禅,从象背升于空中,坐在莲花上。大臣们礼拜后,问:「尊者!什么业处?您如何证得?」他因其为慈禅业处,且于此修观而证得,所以,为显示此义而说此慨叹之颂与解答之颂。
\item 这里,\textbf{一切},即无余。\textbf{生物},即有情,这于此是略说,我们将在宝经的注释中详说。\textbf{放下},即丢弃。\textbf{棍杖},即身语意之杖,为身恶行等的同义语。因为身恶行以刑罚之义为棍杖,即是说迫害、令遭不幸与灾祸,语恶行、意恶行也如是,或者仅以击打之棍杖为棍杖,也即是说放下它。\textbf{某个},即任一个。\textbf{其中},即这一切生物中。\textbf{不会希求孩子},即不会希求亲生、野生、领养、弟子等这四种孩子中的任一种孩子。
\item \textbf{独自},即以出家被称为独自,以无侣之义为独自,以舍弃渴爱为独自,以究竟离烦恼为独自,以现等觉辟支菩提为独自。因为即便存在于一千沙门之中,由切断了俗家的结缚而为独自,如是即\textbf{以出家被称为独自}。独立、独行、独坐、独卧,独自行止活动,如是即\textbf{以无侣之义为独自}。\begin{quoting}以爱为侣的人,轮回于漫长的旅途,\\到此处与他处,不得越过轮回。\\了知了这过患,『渴爱是苦的生起』,\\离爱、无取,比丘应具念而游行。(经集第 746~747 颂)\end{quoting}如是\textbf{以舍弃渴爱为独自}。舍弃一切烦恼,断其根本,如截多罗树头,使其消亡,于未来成不生法,如是即\textbf{以究竟离烦恼为独自}。无师、自成,唯由自己现等觉辟支菩提,如是即\textbf{以现等觉辟支菩提为独自}。
\item \textbf{游行}\footnote{游行:菩提比丘注 406,该词含义丰富,DOP 的解释有:行走、漫游、放牧,旅行,生活、举止,行为,经历、从事等等。这里将偈颂中的动词译作「游行」,义注中的名词 cariyā 则作「行」。},有这八种行,即具足誓愿者于四威仪中的威仪行,守护根门者于内处的处行,住于不放逸者于四念处的念行,从事增上心者于四禅那的定行,具足菩提者于四圣谛的智行,正行道者于四圣道的道行,证果者于四沙门果的证行,三种佛于一切有情的世间义利之行,于此,对辟支佛、声闻等唯有部分,如\begin{quoting}行,即威仪行等八种行。(无碍解道 1.197)\end{quoting}中详说,即具足这些行之义。或者,即如\begin{quoting}胜解者以信行,策励者以精进行,护持者以念行,不散乱者以定行,正知者以慧行,了知者以识行,如是行道者以「善法即处」的处行而行,以「如是行道即证殊胜」的殊胜行而行。(无碍解道 1.197)\end{quoting}所说的其它八种行,也是具足这些行之义。
\item \textbf{像犀牛角一样},此中,犀牛角即犀牛之兽的角。「一样 \textit{kappa}」一词的意义,我们将在吉祥经的注释中详说,而这里当知即为\begin{quoting}先生!我们正与和大师一样的弟子相谈。(中部第 24 经)\end{quoting}等处的「相似」。像犀牛角一样,即是说与犀牛角相同。至此先是逐词的释义。
\item 而旨趣与后续当知如是。这如前所说品类的棍杖被加于生物时,即非利益,以于彼等不施加之,并以作为其对治的、为他人带来利益的慈,对一切生物放下了棍杖。且唯由放下棍杖,好比未放下棍杖的有情以棍杖、刀剑、拳掌或土块恼害生物,如是,不恼害其中的某个。对由此慈业处而起的受、想、行、识及与之相伴的其余有为修观已,我证得此辟支菩提。以上是旨趣。
\item 而其后续为:如是说已,大臣们说:「现在,尊者!您要去哪里?」随后,他经转向「先前的辟支正觉者们都住在何处」后便即知晓,当答以「香醉山」时,他们又说:「现在,尊者!您要舍弃我们,不要我们了!」于是,辟支佛便说了「不会希求孩子」等等。这里的旨趣是:我现在不会希求亲生等的任一种孩子,遑论你们这样的朋友?所以,你们中若想和我一起去,或想成为我这样的,他应当独自游行,像犀牛角一样。或者,当他们说「现在,尊者!您要舍弃我们,不要我们了」时,这辟支佛说「不会希求孩子,遑论朋友」,看到自己以所说之义而独行的功德,生起欢喜、喜悦,便发出慨叹「他应当独自游行,像犀牛角一样」。如是说已,面对瞩目的大众,跃入空中,便去了香醉山。
\item \textbf{香醉山},是在喜马拉雅中超过小黑山、大黑山、龙围、月胎、日胎、金胁、雪山等七山而存者。在那里名为欢喜之源的山坡是诸辟支佛的居所。且有三个洞窟:金窟、摩尼窟、银窟。在摩尼窟的洞口,有名为曼珠沙的树,高一由旬,广一由旬,在辟支佛到来的当天,无论水中或陆上的花都会特别地绽放。在它周围有\textbf{一切宝亭}。于此,扫帚风去除了尘埃,作平风铺平了由一切宝制成的砂子,洒水风从阿耨达池带来水后浇洒,作香风从雪山带来一切香树的香,择花风采择了花并散下,敷盖风敷盖了一切处。且总是在那里设好坐处,在辟支佛出世的当天及布萨日,一切辟支佛聚集后,便坐在那里。这于此是天性:现等觉的辟支佛到此后,坐于设好的坐处,随后若在那时有其他的辟支佛在场,他们也在那刹那聚集,坐于设好的坐处,且落坐后,他们即入于某个等至后出起,随后,僧伽的上座为随喜一切之义,问新来的辟支佛业处「如何证得」,随后,他便对其说了自己的慨叹与解答之颂。
\item 世尊为尊者阿难所问,又再次对他说了此颂,阿难则在结集时(也说了此颂)。如是,一一颂在辟支佛现等觉处、曼珠沙亭、阿难问时及结集时被说了四次。\end{enumerate}

\subsection\*{\textbf{36}}

\textbf{从事交际便有诸多爱执,这苦追随爱执而生,\\}
\textbf{觉察着爱执所生的过患,他应当独自游行,像犀牛角一样。}

Saṃsaggajātassa bhavanti snehā, snehanvayaṃ dukkham idaṃ pahoti;\\
ādīnavaṃ snehajaṃ pekkhamāno, eko care khaggavisāṇakappo. %\hfill\textcolor{gray}{\footnotesize 2}

\begin{enumerate}\item 缘起为何?这辟支菩萨也在迦叶世尊的教法内以之前的方式行沙门法二万年,作了遍的遍作后,生起了初禅,确定了名色,作了相的思惟,未证得圣道,便投生到了梵界。他从此下堕,投生在波罗奈王正妃的胎内,仍以之前的方式成长,自从他了知了「这是女人、这是男子」的差别,便不喜欢在女人的手中,甚至不堪忍受涂身、洗浴、装饰等。唯有男子养育他,当哺乳时,乳母们裹以毯子,以男子的衣装哺乳。他嗅到女人的气味或听到(女人的)声音就大哭,等到了青春期,甚至都不愿瞧女人,因此他被称为\textbf{无女人香者}。
\item 当他十六岁生日时,国王想「我得让家族的世系延续」,从各个家族请来与之相当的女孩,命令某个大臣「让王子高兴」。大臣为了让他高兴,用方法在他附近让人围上屏风,准备好舞女。王子听到歌奏之声,便说:「这是谁的声音?」大臣说:「殿下!这是你的舞女的声音,诸有福者才有这样的舞女,享乐吧!殿下!你是大福者。」王子让人杖击并赶走了大臣。
\item 他告知了国王。国王与王子的母亲一起去到后,原谅了王子,又再次交代了大臣。王子受到他们极度地逼迫,给了上等的金子,命令金匠们「做个美人的塑像」。他们塑了与工艺天所造相仿的以一切庄严装饰的女人的形像。王子看后,惊异地摇头,交给父母:「如果我能得到这样的女人,我就接受。」父母想「我们的儿子有大福,肯定会有某个曾和他一起造福的女孩出现在世间」,让人把金像装上车,交代大臣们:「去!寻找这样的女孩!」
\item 他们拿了后,周行于十六大国,去到各各村落,在水岸等看到人群之处,就置好天人一样的金像,以种种花、布的装饰作了供养后,撑起华盖,站在一边:「如果有人先前曾见过这样的,让他来谈话。」以此方法,周游于除了末陀国的一切国土,轻视其为小国,初次未去即折返。
\item 随后,他们想「我们还是先去末陀国吧,别等进了波罗奈,国王再派遣我们」,便到了末陀国的沙竭罗城。在沙竭罗城,有名为末陀婆的国王,他的女儿年满十六,容貌端正。她的使女们为取洗澡水而到岸边,在那里远远地看到大臣们放置的金像,说着「派我们来取水,公主自己倒来了」,走近后说:「这不是主人,我们的主人比这还端正。」大臣们听后,去到国王处,以适当的方式请求女孩,他便给了。随后,他们送信给波罗奈王:「已经得了女孩,您是亲自迎接,还是我们带来?」他便遣使说:「若我前往,国家将会骚动,你们带来!」
\item 大臣们带了女孩离开城市,送信给王子:「已经得了与金像相仿的女孩。」王子听后,为贪染所胜,即从初禅退失,他一遍遍地派遣使者:「速速带来!速速带来!」他们在每个地方只住一夜,到了波罗奈后,驻扎城外,送信给国王:「今天能否入城?」国王命令道:「从上等家族带来的女孩,我们要在举行吉祥的仪式后,极恭敬地迎接,先带她到园林!」他们便遵行。她极其纤弱,为车驰所累,因旅途劳顿而伤风,竟如凋萎的花鬘般,当晚就死了。
\item 大臣们悲泣道:「我们失去了荣光。」国王和城民们悲泣道:「家族的世系断绝。」城中便起了大骚乱。王子甫一听闻,便起了大忧伤。随后,王子开始探寻忧伤的根源。他想到「这忧伤不是未生者的,而是生者的,所以,缘生而有忧伤,那么缘何而有生」,随后想「缘有而有生」,如是以先前修习的势力而如理作意,得见顺逆的缘起后,思惟诸行,即于此坐证得了辟支菩提。大臣们见到他乐于道果之乐、诸根寂静、心意寂静而坐,跪拜而说:「莫忧伤!殿下!阎浮提甚大,我们随后会带来比她更端庄的。」他说:「我非忧者,我是离忧的辟支佛。」此后的一切与前颂相同,唯除颂的注释。
\item 而在颂的注释中,\textbf{从事交际},即发生交际,于此,以见、闻、身体、交谈、受用的交际等而有五种交际。这里,见到彼此后,以眼识路而生起的贪染名为\textbf{见的交际}。这里,在僧伽罗岛的黑长湖村,地主家的女儿见到正在行乞的住于妙善寺的诵长部的青年比丘后,牵绊于心,无法得到他而死去。他见到她下裙的碎片后,想「我无法与穿着这样衣服的女子同住」,心碎而死。这青年即是例证。
\item 听到由他人谈论的美色等,或自己听到笑语歌声后,以耳识路而生起的贪染名为\textbf{闻的交际}。此处,住在山村的锻工之女与五个女孩一起去到荷塘,澡浴后,带上花鬘,当高声歌唱时,住在五闩窟的低舍青年正行于空中,听到声音后,因爱欲的贪染而退失殊胜,以至厄难,即是例证。
\item 以彼此肢体的接触而生起的贪染名为\textbf{身体的交际}。诵法的青年比丘于此便是例证。据说,青年比丘在大寺中说法。在到此的大众中,国王与后宫一起也来了。随后,国王的女儿对其容貌、声音生起强烈的贪染,而这青年也是。国王见此并观察后,让人围以屏风。他俩彼此抚摸后,便作拥抱。当人们撤去屏风后再看时,便见二人竟死了。
\item 以彼此谈话、交谈而生起的贪染名为\textbf{交谈的交际}。当比丘、比丘尼一起受用时生起的贪染名为\textbf{受用的交际}。于此二者,犯下波罗夷的比丘与比丘尼即是例证。据说,在名为椒芯的大寺祭典中,无畏大王趣恶准备了大布施,给两部僧伽施食。于此,在施热粥时,僧伽新沙弥尼给了未得(热粥的)僧伽新沙弥一个象牙手镯,并作了交谈。二人受具足后过了六十年,到了对岸,因彼此交谈而忆起从前,立刻生起爱执而违犯学处,成了波罗夷。
\item 如是,以五种交际中的任一交际而发生交际,\textbf{便有爱执},即由先前的贪染之缘而生起强烈的贪染。随后,\textbf{这苦追随爱执而生},追随着此爱执,这现世、后世的忧悲等种种行相的苦即生起。但另有人说「心投入所缘中为交际」,此后有爱执,从爱执有此苦。
\item 如是,说了这含义丰富的半颂后,那辟支佛说:「这追随爱执而生的忧等之苦,我探寻此苦的根源,得证辟支菩提。」如是说已,大臣们便说:「尊者!我们现在该做什么?」随后他说:「你们,或其他人,若想从此苦解脱,也都要\textbf{觉察着爱执所生的过患,他应当独自游行,像犀牛角一样}。」于此,当知「觉察着爱执所生的过患」是就「这苦追随爱执而生」而说的。或者,以如所说的交际而从事交际便有爱执,这苦追随爱执而生,如实地觉察着爱执所生的过患,我即得证。如是连接已,当知第四句以先前所说的慨叹而说。此后一切都与前颂中所说的相同。\end{enumerate}

\subsection\*{\textbf{37}}

\textbf{同情着朋友和知心,被牵绊的心便忽视义利,\\}
\textbf{觉察着这亲密中的怖畏,他应当独自游行,像犀牛角一样。}

Mitte suhajje anukampamāno, hāpeti atthaṃ paṭibaddhacitto;\\
etaṃ bhayaṃ santhave pekkhamāno, eko care khaggavisāṇakappo. %\hfill\textcolor{gray}{\footnotesize 3}

\begin{enumerate}\item 缘起为何?这辟支佛仍以前颂所说的方式投生后,在波罗奈执掌王权,生起了初禅,在省察「是沙门法殊胜还是王权殊胜」后,将王权交给四位大臣的手中而行沙门法。大臣们虽被要求「如法正当而行」,却仍取贿赂而行非法。他们取了贿赂,便欺负物主,一次就欺负了某个国王的心腹。他与国王的送饭者一起进入后,告知了一切。国王在第二天便亲自到了裁断处。随后,大众便起喧哗「大臣们把物主弄成了非物主」,好像要打大仗一样。于是,国王从裁断处出来,上到宫殿,坐而欲入等至,心却被这嘈杂所扰而不得入。他想「王权于我何有?还是沙门法殊胜」,舍弃了王权之乐,又再生起等至,仍以先前的方法修观,证了辟支菩提。当被问及业处时,说了此颂。
\item 这里,以慈行为\textbf{朋友},以好心为\textbf{知心}。因为有些单纯为了利益,只是朋友却非知心,有些则以在往来、坐立、交谈等中带来心的快乐而为知心,却非朋友,有些则以此两者,既是知心也是朋友。他们有两种:在家与出家。这里,在家有三种:资助者、同苦乐者、同情者,出家则特指讲述义利者。他们具足四支。如说:\begin{quoting}长者子!当知以四种原由,资助者为朋友、知心:他守护放逸者,守护放逸者的财产,为怖畏者的皈依,当出现应作的义务时,他给予两份的所需。(长部第 31 经第 261 段)\end{quoting}同样,\begin{quoting}长者子!当知以四种原由,同苦乐者为朋友、知心:他告知你秘密,隐匿你的秘密,在危难时不离弃,乃至为了你的义利舍弃生命。(长部第 31 经第 262 段)\end{quoting}同样,\begin{quoting}长者子!当知以四种原由,同情者为朋友、知心:他不喜你的失败,喜于你的成功,遮止诽谤者,赞叹称赏者。(长部第 31 经第 264 段)\end{quoting}同样,\begin{quoting}长者子!当知以四种原由,讲述义利者为朋友、知心:他遮止恶,令止于善,令闻未闻,宣说天道。(长部第 31 经第 263 段)\end{quoting}其中,这里是指在家的意思,但从语义上一切都适合。\textbf{同情},即怜悯,即欲为他们带来乐并除去苦。
\item \textbf{忽视义利},即忽视、亡失现法、后世、第一义的三种,以及自利、他利、两者之利的三种义利与亡失已得、不得未得的两种。\textbf{被牵绊的心},即以「我没有他就不能活,他是我的归趣,他是我的归宿」等,将自己置于低处而牵绊其心,也以「他们没有我就不能活,我是他们的归趣、他们的归宿」等,将自己置于高处而牵绊其心,而这里指的是后者。
\item \textbf{这怖畏},即这忽视义利的怖畏,是就退失自己的等至而说的。\textbf{亲密},即三种亲密:爱、见、朋友的亲密。这里,百八种渴爱为爱的亲密,六十二种见为见的亲密,由牵绊其心的对朋友的同情为朋友的亲密,这里指的是后者,因为他的等至以此退失。因此他说:「觉察着这亲密中的怖畏,我即得证。」其余当知仍与所说的相同。\end{enumerate}

\subsection\*{\textbf{38}}

\textbf{关切于妻子与儿女,好比修竹交织牵扯,\\}
\textbf{如竹笋般不受羁绊,他应当独自游行,像犀牛角一样。}

Vaṃso visālo va yathā visatto, puttesu dāresu ca yā apekkhā;\\
vaṃsakkaḷīro va asajjamāno, eko care khaggavisāṇakappo. %\hfill\textcolor{gray}{\footnotesize 4}

\begin{enumerate}\item 缘起为何?据说,先前在迦叶世尊的教法内,三位辟支菩萨出家后,履行了二万年往还的义务,投生到天界,从此堕后,其中的长者出生于波罗奈的王族中,其他二人则于边地的王族中。他们二人取了业处,舍弃王位而出家,渐次成了辟支佛,住在欢喜之源山坡,一天,从等至出起后,转向于「我们造了什么业而证此出世间之乐」,当省察时,得见迦叶佛时自己的所行。随后,他们转向于「第三个人在哪里」,见到他正统治波罗奈国后,忆念起他的功德「他曾天性具足少欲等的功德,于我们是教诫者、说示者、堪忍言语者、过失指摘者,噫!我们要给他显示所缘,令其解脱」,便寻找机会,一天,见到他去到一切庄严装饰的园林后,便从空中前往,站在园林入口的竹丛下。大众无厌于国王的形象,瞻仰着国王。随后,国王在观察「有没有人没注意我的形象」时,便看到了二辟支佛,且一见之下,便对他们生起爱执。
\item 他从象背下来,平静地靠近他们后,便问:「尊者!你们名为什么?」他们说:「大王!我们名为不受羁绊。」「尊者!不受羁绊是什么意思?」「无执著之义,大王!」随后,他们向其示以竹丛,便说:「大王!好比这竹丛为一切根茎枝叉所缠绕而立,手持利刃的男子无法斩断其根,转而拔出,如是,你为内外之结缚结,交织牵扯,于此执著,或者好比在其中间的这竹笋,由未生枝条之故,无所执著而立,则能从顶或从根斩断拔出,如是,我们不受羁绊,行于一切处。」立刻证入四禅,在国王的注视下,从空中去了欢喜之源山坡。随后,国王想:「我什么时候也能这样不受羁绊?」即于此坐而修观,证了辟支菩提。如先前一般,当被问及业处时,说了此颂。
\item 这里,\textbf{修竹},即繁茂之竹。\textbf{va},即强调之义,或作 eva,此处因连声而失落 e——它与别词相连,我们只结合其后分。\textbf{好比},即相似。\textbf{交织牵扯},即固著、缚结、纠缠。\textbf{关切},即渴爱、爱执。\textbf{如竹笋般不受羁绊},即好比竹笋般无执著。
\item 这是说的什么?好比修竹交织牵扯,关切于妻子与儿女也如是由纠缠于这些依处而交织牵扯,我如是见到关切的过患「关切者以此关切,将如修竹般交织牵扯」,以道智斩断此关切,如竹笋般,于色等、贪等、欲有等、所见等不以爱、慢、见等羁绊,得证辟支菩提。其余当知仍如前述。\end{enumerate}

\subsection\*{\textbf{39}}

\textbf{就像林野中不羁的鹿,随意地漫步于行处,\\}
\textbf{有智之士觉察着自由,他应当独自游行,像犀牛角一样。}

Migo araññamhi yathā abaddho, yen’icchakaṃ gacchati gocarāya;\\
viññū naro seritaṃ pekkhamāno, eko care khaggavisāṇakappo. %\hfill\textcolor{gray}{\footnotesize 5}

\begin{enumerate}\item 缘起为何?据说,一比丘在迦叶世尊的教法内行瑜伽,死后投生到波罗奈中富裕、多财、多产的商人家族而享福。随后犯了通奸,于此死后,投生到了地狱,于此受尽煎熬,以剩余的异熟在商人妻子的子宫内获得女人的结生。从地狱而来者肢体皆热,因此,商人妻子腹内如烧,经过困苦艰难地怀胎后,按时分娩。她从出生的当天开始,便受到父母及其余亲属、仆从等的嫌弃。且成年后,于所归之家,也受主人及舅姑的嫌弃、不喜、不中意。
\item 于是,当节日来临时,商人之子不愿与她庆祝,而招来妓女相娱。她从使女们跟前听到后,去到商人之子处,以种种方式抚慰,说:「主人!女人,哪怕是十王最小的女儿或转轮王的女儿,也一样是主人的使婢,当主人不相言谈时,便如万箭穿刺般受苦,若我应受宠爱,便请宠爱,否则,便请休遣,我将回自己的本家。」商人之子说:「好吧!夫人!别忧伤!准备娱乐吧,我们将庆祝节日!」商人之女就因这话而兴奋,「明天我要庆祝节日」,准备了很多食物。商人之子在第二天未告知便去了庆祝的地方。她坐着瞻望道路「现在要派人来了,现在要派人来了」,看到日头已高,便派人去。他们回来后,告知「商人之子已经去了」。她取了所有准备好的东西,上了车,出发去往园林。
\item 于是,在欢喜之源山坡上的辟支佛在第七天从灭中出起,在阿耨达洗了脸,嚼了槟榔齿木,转向于「今天要去哪里行乞」,看到这商人之女,了知到「在她对我行恭敬后,这业便会消除」,站在距山坡六十由旬外的雄黄之原,著了下衣,持了衣钵,入了神通足处的禅那,从空中前往,降落在她的路对面,朝着波罗奈前去。使女们看到他后,告知了商人之女。她下了车,恭敬地礼拜,取了钵,装满了具足众味的食物后,再覆以莲花、承以莲花,手持花束,去到辟支佛处,将钵授于其手,礼拜后,手持花束发愿:「尊者!好比这花,愿我于所生之处,能受大众的喜爱、中意!」如是发愿已,第二次发愿:「尊者!住胎甚苦,愿能免此而结生于莲花中!」第三次发愿:「尊者!女人甚可厌恶,即便转轮王之女也受制于人,所以,愿我能免于女性而为男子!」第四次发愿:「尊者!越过这轮回之苦后,在其终了,愿我能圆满您所证得的不死!」
\item 如是发了四个誓愿,供养了这莲花花束,五体投地礼拜了辟支佛,发了第五个誓愿:「愿我的容貌与体香如这花一般!」随后,辟支佛取了钵与花束,站在空中,以此颂随喜了商人之女:\begin{quoting}你所希望、发愿者,愿能速速成就!\\愿一切意图皆能圆满,如十五的月亮般!\end{quoting}决意「让商人之女看着我离开」,去了欢喜之源山坡。
\item 商人之女见到后,生起大喜,于有间所作的不善业因无机缘而消尽,她则变得如罗望子汁洗净的铜器般洁净。她夫家与本家的所有人都立刻满意,「我们都做了什么」,致以喜爱之词与礼物。商人之子派人「速速带商人之女来!我忘了她就来园林了」。从此以后,喜爱、珍惜她如胸前涂油的旃檀木、随身的珍珠串与花鬘一般。
\item 她于此直至寿尽,享受了自在与财富之乐,死后以男性投生于天界的莲花中。这天子行时,唯行于莲花的胎内,站、坐、卧时也唯于莲花的胎内。他们称其名为大莲花天子。如是,他以此神变的威力,唯顺逆轮回于六天界。
\item 尔时,波罗奈王有二万个女人,而国王竟不能从中得子。大臣们对国王说:「陛下!守护家族的世系需要儿子,若没有亲生的,野生的也可以保持家族的世系。」国王说「除了王后,让其余的嫔妃如法地野合七天」,便让她们恣意在外,即便如此,仍未得子。大臣们又说:「大王!王后以其福慧为一切女人之首,恐怕陛下还是应该在王后的胎中得子。」国王便将此事告知了王后。她说:「大王!言语真实、持戒的女人方能得子,无惭愧者如何有子?」上到宫殿后,受持五戒,反复思量。当持戒、反复思量五戒的王后生起希求子嗣之心时,帝释的坐处便发烫了。
\item 于是,帝释转向于坐处发烫的原因,了知此事后,想「我要给持戒的王后最好的儿子」,从空中去到王后的面前,站而发问:「殿下!你希求什么?」「儿子,大王!」「我会给你儿子,殿下!别再思量了!」说后回到天界,转向于「这里有没有寿尽者」,得知「这大莲花从此死后,将投生至更高的天界」,去到他的宫殿请求:「亲爱的大莲花!请去人世间吧!」他说:「大王!别这样说!人世间可厌。」「亲爱的!你在人间造福才投生于此,去到那里后可圆满众波罗蜜,去吧!亲爱的!」「大王!住胎甚苦,我不堪住于彼处。」「亲爱的!你为何要住胎?因为你这般造业,唯将出生于莲花之中,去吧!亲爱的!」他经再三劝说,便同意了。
\item 随后,大莲花从天界下堕,投生在波罗奈王园林石板池中的莲花胎内。当晚,王后在黎明时分,于睡梦中,为二万个女人围绕,去到园林,在石板池中的莲花塘里似得了孩子一般。破晓后,她守持着戒,也同样去到那里,便见到一朵莲花,不在岸边,也不在深处,一见之下,便对其生起爱子之心。她亲自进入,摘了这花,花刚被摘下,花瓣即敷开,她便于此看见一个男孩,如托盘中洒金的塑像般,见后即高声惊叹「我得了孩子」。大众纷纷恭喜,并送信给国王。
\item 国王听后,问「在哪里得的」,在听到所得的场所后,说「园林和池中的莲花都是我们的野外,所以这孩子因出生在我们的野外而为野生」,教人带入城中,让二万个女人去做乳母。她们若得知王子的喜好并让他吃所希求的食物,便获得一千(奖赏)。整个波罗奈为之震动,所有人给王子送去了数千礼物。带来礼物后,王子被劝着「吃这、嚼这」而烦扰、苦恼,去到城门,玩耍彩球。
\item 那时,某个辟支佛依于波罗奈,住在仙人堕处。他按时起来,作了坐卧处的义务、洗漱、作意等的一切义务后,从宴坐出起,转向于「今天我要去哪里乞食」,见到王子的成就,当省察「他先前造了什么业」时,便知「在布施给如我等者以食物后,发了四个愿,于此已成就了三个,尚有一未成,我当以方便给予其所缘」,因行乞去到王子的跟前。
\item 王子见到他后,说:「沙门!别来这里!因为他们也会对你说『吃这、嚼这』。」他因这一语便转身,进入自己的坐卧处。王子对仆从说:「这沙门因我所说即转身,是不是忿恨于我?」随后,虽然他们说「殿下!出家不是旨在忿恨,而是靠他人净喜之意的所施而存活」,他仍告知父母:「这沙门确实忿恨于我,我要去请他原谅。」骑了象,以盛大的王家威仪去往仙人堕处,见到鹿群后,便问:「它们是什么?」「主人!它们是鹿。」「对它们也会有说『吃这、嚼这、尝这』般的照料吗?」「没有,主人!哪里水草易得,它们便住在哪里。」王子便取了这所缘:「如它们般不受守护,想去哪里,就住哪里,什么时候我也能这样生活?」
\item 辟支佛也知晓了他的到来,洒扫了坐卧处的道路与经行道,弄平整后,经行了一二回,示以足迹,又洒扫了昼住处与茅蓬,弄平整后,示以进入的足迹,却不示以出去的足迹,便去了别处。王子到了那里,见到此地经洒扫并弄平整,听到仆从们说「看来这辟支佛住在这里」后,便说:「这沙门早上还很忿怒,现在要是见到自己的场所被象马等踩踏,会更加忿恨,你们就站在此处!」从象背下来,独自进入坐卧处,见到按义务善加洒扫的场所中的足迹,想「在此经行的这沙门不曾想过营生之务,看来确实只想着自己的利益」,心怀净喜,便上了经行道,抛开杂念,坐于石板,生起一境,进入茅蓬修观而证得辟支菩提之智,如先前一般,当被祭司问及业处时,坐在空中,说了此颂。\footnote{「心怀净喜,便上了经行道……说了此颂」一段,PTS 本如下:省察着便进了昼住处。在此见到足迹,仍如是思量,再随着足迹,打开门,进入茅蓬内,不见辟支佛,四处观察而见其坐处的石板,见后,想「坐在这里的这沙门不曾想过营生之务,确实只想着自己利益的沙门法」,也在此坐下,如理作意,渐次圆满了止观,证得了辟支菩提。他受用着出世之乐而不出。大臣们想「国王的命令甚严,『你们带着我的孩子在林野长时逗留』,也许会对我们施以棍杖,我们带着王子走吧」,进入茅蓬,不见辟支佛,却见王子如是坐着,想「他没见到辟支佛故坐而沉思」,便说:「殿下!辟支佛就住在这里,不会去到别处,我们明天再来请他原谅,别沉思『没看见辟支佛』!来!我们走!」王子说:「我不在沉思,我已成不思者。」「主人!发生了什么?」「我已成辟支佛。」如先前一般,当被问及业处时,说了此颂。}
\item 这里,\textbf{鹿}有两种,麋鹿与梅花鹿,且为所有林野中四足者的同义语,然而这里指的是梅花鹿。\textbf{林野},即除了村及村郊的其余的野外,这里指的是园林。\textbf{不羁},即不羁于绳索等的束缚,以此显示安心地游行。\textbf{随意地漫步于行处},即想往哪方去,就往那方的行处去。此即世尊所说:\begin{quoting}诸比丘!好比林野之鹿游行于深林之中,安心而行,安心而住,安心而坐,安心而卧,这是什么原因?诸比丘!不在猎人的视野所及。如是,诸比丘!比丘离欲……具足初禅而住,这即是,诸比丘!令魔盲暗,伤魔眼至无境,得至恶者不见之处。(中部第 26 经第 287 段)\end{quoting}\textbf{有智之士},即智者。\textbf{自由},即按己欲生活而不随他人。\textbf{觉察},即以慧眼观察。或者,即觉察法之自由与人之自由,因为不入于烦恼的势力,出世间法及具足此的人为自由者,对彼等状态的描述为自由。
\item 这说的是什么?「就像林野中不羁的鹿,随意地漫步于行处,我何时也能如是而行」,我被你们的存在四处围绕而受缚,不得随意而行,因不能随意而行,而见此随意而行之中的功德,止观渐至圆满,随后得证辟支菩提。所以,别的有智之士觉察着自由,也应独自游行,像犀牛角一样。其余当知仍如前述。\end{enumerate}

\subsection\*{\textbf{40}}

\textbf{在朋友中,于居止行游,总有应对商谈,\\}
\textbf{觉察着不被渴求的自由,他应当独自游行,像犀牛角一样。}

Āmantanā hoti sahāyamajjhe, vāse ṭhāne gamane cārikāya;\\
anabhijjhitaṃ seritaṃ pekkhamāno, eko care khaggavisāṇakappo. %\hfill\textcolor{gray}{\footnotesize 6}

\begin{enumerate}\item 缘起为何?据说,在过去,有国王名独语梵赐,生性柔和,当大臣们想与他商谈相关或不相关的事时,总把他单独拉到一边。某天,他已午睡,某大臣说「陛下!我有些事你必须得听听」,请求他去到一边,他便起身去了。他坐在大讲堂,又有人请求恩惠,在象背……在马背……在金车……坐轿子去园林,有人请求,国王便下来去到一边,更有人在巡行国土时请求,国王听到他的话后,也从象上下来,去到一边。
\item 如是,他于彼等生厌,便出了家。大臣们权欲增盛。其中一人去到国王处说:「大王!请赐我那片土地!」国王说:「某某食采于彼。」他不接受国王的话,「我去夺下那土地来受用」,去到那里挑起争端后,二人又再来到国王跟前,告知彼此的过失。国王想「这些无法使他们满足」,看到他们贪中的过患,修观而得证辟支菩提。仍如先前一般,说了这慨叹之颂。
\item 其义为:对住\textbf{在朋友中}者,\textbf{于}午睡处的\textbf{居}及大讲堂处的\textbf{止}、前往园林的\textbf{行}、巡行国土的\textbf{游},\textbf{总有}如「请听我这、请赐我这」般的\textbf{应对商谈},所以,我于此生厌,而这出家是圣人所事、具诸功德、一向是乐,即便如是,也不为被贪征服的一切可鄙之人所渴求、所希求,\textbf{觉察着}这\textbf{不被渴求的}、因不受他人控制及因法与人\footnote{法与人的自由,见第 39 颂注。}的\textbf{自由},我开始作观,渐次证得了辟支菩提。其余仍如前述。\end{enumerate}

\subsection\*{\textbf{41}}

\textbf{在朋友中有嬉戏、喜乐,在孩子中有深厚的爱怜,\\}
\textbf{厌烦着爱别离,他应当独自游行,像犀牛角一样。}

Khiḍḍā ratī hoti sahāyamajjhe, puttesu ca vipulaṃ hoti pemaṃ;\\
piyavippayogaṃ vijigucchamāno, eko care khaggavisāṇakappo. %\hfill\textcolor{gray}{\footnotesize 7}

\begin{enumerate}\item 缘起为何?在波罗奈,有国王名独子梵赐。而其独子可亲、可爱、等同性命,他于一切威仪中都会带着孩子而行。一天,他没有带他便去往园林,而王子竟于当天因病而死。大臣们想「国王出于爱子,怕是会心碎」,未予告知即火化了他。国王在园林中因醉于酒竟未念及孩子,第二天在澡浴饮食时也如是,当饱食而坐时,想起来说:「把我的孩子带来!」大臣们便以合适的方式告知了经过。于是,他为忧伤所胜,坐而如是如理作意:「此有故彼有,此生故彼生。」他如是渐次思惟顺逆的缘起,得证辟支菩提。除对颂的释义,其余都与从事交际一颂所说的相同。
\item 而在释义中,\textbf{嬉戏},即游戏,它有两种,身与语。此处,身即以象、马、车、弓、剑等游戏,语即歌咏、诵诗、口技等。\textbf{喜乐},即种种五欲之乐。\textbf{深厚},即深入骨髓而遍及整个自体。其余则自明。而后续与章句当知也如从事交际一颂所说,且此后的一切都如是。\end{enumerate}

\subsection\*{\textbf{42}}

\textbf{游行四方者无有障碍,随所遇而知足,\\}
\textbf{忍受危难,而不惊惧,他应当独自游行,像犀牛角一样。}

Cātuddiso appaṭigho ca hoti, santussamāno itarītarena;\\
parissayānaṃ sahitā achambhī, eko care khaggavisāṇakappo. %\hfill\textcolor{gray}{\footnotesize 8}

\begin{enumerate}\item 缘起为何?据说,先前在迦叶世尊的教法内,有五位辟支菩萨出家后,履行了二万年往还的义务,投生到天界。从此堕后,其中的长者成了波罗奈的国王,其余为当地的王。他们四人执取业处后,舍弃王位而出家,渐次成了辟支佛,住在欢喜之源山坡上,一天,从等至出起后,如竹笋一颂所说的方法,转向于自身的业及朋友,了知已,寻找机会用方法向波罗奈王显示所缘。
\item 国王在夜里三次惊觉,怖畏悲号,跑上屋顶平台。祭司按时起来,询问睡眠安乐,他说「老师!我如何能安乐」,便告知了一切经过。祭司想「这病无法以任何催吐等的药物治疗,但正好是我的稻粱之谋」,进一步恐吓国王「大王!这是亡国丧命的前兆」,为安抚他,教他举行献牲的祭祀「应布施如许如许象、马、车等以及货币、黄金等,举行献牲」。
\item 随后,众辟支佛看到数千生类为献牲而遭集结,「造了这业,他将难以觉悟,噫!我们早一步去看看」,如竹笋一颂所说的方法前往乞食,渐渐去到了王家庭院。国王立于窗前,俯眺王家庭院,便看见了他们,且一见之下,便对他们生起爱执。随后,命人召唤他们,请他们于设在屋顶的坐处入坐,恭敬地授食,当食事已毕,便问:「你们是谁?」「大王!我们名为游行四方者。」「尊者!游行四方者,其义为何?」「于四方的任何处,我们不因任何而有怖畏或心的恐惧,大王!」「尊者!你们因何而无怖畏?」「大王!我们修习慈、修习悲、修习喜、修习舍,因此而无怖畏。」说毕,从坐起,回到自己的住处。
\item 随后,国王想:「这些沙门说『修习了慈等而无怖畏』,而众婆罗门却称赏杀戮数千生类,谁的言语真实?」于是,他想到:「沙门以清净清洗不净,而婆罗门以不净清洗不净,但不能以不净清洗不净,出家人的言语确乎真实。」他以「愿一切有情快乐」等方法修习慈等四梵住,便以遍满利益的心命令大臣们:「释放所有生类!让它们饮清凉的水、吃新鲜的草,让凉风吹拂它们!」他们便照做。
\item 随后,国王想「正因善知识的言语,我才从恶业解脱」,即于此坐而修观,证得辟支菩提。当大臣们在食时说「大王!食时已到」时,他以先前的方法说「我不是国王」等一切已,便说了这慨叹与解答之颂。
\item 这里,\textbf{游行四方者},即于四方随所乐而住者,或者以「遍满一方而住」等方式,梵住修习遍满的四方对其成为寂静,亦为游行四方者。在彼诸方的任何处,于诸有情或诸行不遭怖畏,即\textbf{无有障碍}。\textbf{知足},即以十二种知足而知足。\textbf{随所遇},即以或优或劣的资具。\textbf{忍受危难,而不惊惧},这里的危难即令身心衰损,或因彼等的成就而安逸为危难,为外在的狮虎等及内在的欲贪等身心祸害的同义语。于此危难,以忍受、忍耐及以精进等法忍受,为忍受危难,以无引起僵硬的怖畏为不惊惧。
\item 这说的是什么?好比这四位沙门,如是随所遇的资具而知足,于此,立于行道之足处的知足,以于四方修习慈等而为游行四方者,且由对有情、诸行不遭怖畏而无有障碍。他由作为游行四方者而忍受所说种类的危难,并由无有障碍而不惊惧,如是在见到行道的功德、如理行道后,我即证得辟支菩提。或者,了知到「如同这些沙门,随所遇而知足者以所说的方法而成游行四方者」,希求如是游行四方的状态,经如理行道,我即得证。所以,其他希求如此状态者,忍受游行四方的危难,且以无有障碍而不惊惧,应当独自游行,像犀牛角一样。其余仍如前述。\end{enumerate}

\subsection\*{\textbf{43}}

\textbf{即便有些出家人也难以摄受,还有居家的在家人,\\}
\textbf{不再操心别人的孩子们,他应当独自游行,像犀牛角一样。}

Dussaṅgahā pabbajitā pi eke, atho gahaṭṭhā gharam āvasantā;\\
appossukko paraputtesu hutvā, eko care khaggavisāṇakappo. %\hfill\textcolor{gray}{\footnotesize 9}

\begin{enumerate}\item 缘起为何?据说,波罗奈国王的正妃去世。随后,在度过忧伤的时日后,一天,大臣们便请求:「对于国王,在种种义务中,正妃不可或缺,善哉!愿陛下迎娶别的王后!」国王说:「那么,我说,你们找吧!」他们正寻找时,邻国的国王去世,他的王后在训政,且怀着胎。大臣们了知「她适合国王」后,便向她请求。她说:「怀胎者不为众人所喜,若能等待至我分娩,那便如是,否则,请另觅他人!」他们便将此事告知国王。国王说「即便怀胎也可迎娶」,他们便娶了来。国王给她灌顶后,给予所有王后的财物,并以各种礼物摄受其仆从。她按时诞下一子,国王也对其视如己出,于一切威仪中置之腿上或怀中而住。随后,王后的仆从想「国王极摄受王子,国王之心甚可信赖\footnote{甚可信赖:PTS 本作「不可信赖」。},噫!让我们背叛他」,便对王子说:「亲爱的!你是我们国王的孩子,不是这个国王的,莫要信赖他!」
\item 于是,王子被国王召唤「来!孩子」而握手相牵时,不像先前那样亲近国王。国王思索道「这是为何」,了知其经过后,生起厌离「咄!他们经我如是摄受,仍行事违逆」,便舍弃王位而出家。因「国王已出家」,于是大臣、仆从们也纷纷出家,因「国王带着仆从已出家」,于是人们带来上好的资具。国王按年资分配上好的资具。于此,那些得了胜妙者便满足,其他人讥嫌道:「我们做着洒扫僧房等所有义务,却只得鄙食粗服。」他了知后,想「咄!即便按年资分配,他们仍然讥嫌,哎!这众人难以摄受」,持了衣钵,独自入林野后作观,证得辟支菩提。于此,当被来者问及业处时,说了此颂。
\item 此颂语义明了。其章句为:\textbf{即便有些出家人也难以摄受},他们为不知足所胜,\textbf{还有居家的在家人}也同样,我嫌厌着这难以摄受的状态,作观而证得辟支菩提。其余当知仍如前述。\end{enumerate}

\subsection\*{\textbf{44}}

\textbf{除去了俗家相,好比树叶落尽的黑檀,\\}
\textbf{英雄斩断了俗家束缚,他应当独自游行,像犀牛角一样。}

Oropayitvā gihibyañjanāni, sañchinnapatto yathā koviḷāro;\\
chetvāna vīro gihibandhanāni, eko care khaggavisāṇakappo. %\hfill\textcolor{gray}{\footnotesize 10}

\begin{enumerate}\item 缘起为何?据说,在波罗奈,有国王名四月梵赐,当热季的初月去到园林。于此,在一片喜人的地上,看到覆以浓密绿叶的黑檀树,说「请在黑檀树下铺设我的卧处」,在园林嬉戏已,晡时便于此安卧。当热季之中,再次去到园林,此时黑檀已经开花,他仍照旧行事。当热季的末月,再次前去,此时黑檀树叶落尽,好似枯木,他当时未及见此树,便仍按先前的惯例命人于此铺床。大臣们虽然知晓,但因怖畏于「国王所命」,便教人在此铺设卧处。
\item 他在园林嬉戏已,晡时于此安卧,看到此树,想「咄!它先前覆以树叶,如摩尼所造,见之胜妙,而后在摩尼色泽的枝叉上间以珊瑚芽般的花朵,见之光彩照人,且其下的地面铺以珍珠般的砂粒,覆以挣脱的花朵,如敷以染色的毛毯般,而它现今好似枯木,唯余枝干而立,哎!黑檀因老而败坏」,而得了无常想「甚至无执受者\footnote{无执受者即无情,执受者即有情。}也为老所害,何况执受者」。
\item 据此,对一切诸行从无常、无我修观,希求「哎!愿我也如树叶落尽的黑檀般,离去俗家相」,于此卧处以右胁而卧,渐次证得了辟支菩提。随后当返,当大臣们说「大王!归时已到」时,答以「我不是国王」等,仍以先前之法说了此颂。
\item 这里,\textbf{除去},即离去。\textbf{俗家相},即发、须、白衣、装饰、花鬘、芳香、涂油、妇、孺、女奴、男奴等,因为此等标志着在家的状态,故被称为俗家相。\textbf{斩断},即以道智斩断。\textbf{英雄},即具足道之精进。\textbf{俗家束缚},即爱欲之束缚,因为爱欲为俗家的束缚。以上是词义。
\item 而其旨趣为:因为想着「哎!愿我也除去了俗家相,如树叶落尽的黑檀般」而作观,我证得了辟支菩提。其余当知仍如前述。\end{enumerate}

\subsection\*{\textbf{45}}

\textbf{若可得贤明、同行、善住、坚定的朋友,\\}
\textbf{征服了一切危难,他满意、具念,应当与其同行。}

Sace labhetha nipakaṃ sahāyaṃ, saddhiṃcaraṃ sādhuvihāri dhīraṃ;\\
abhibhuyya sabbāni parissayāni, careyya ten’attamano satīmā. %\hfill\textcolor{gray}{\footnotesize 11}

\begin{enumerate}\item 缘起为何?据说,先前在迦叶世尊的教法内,有两位辟支菩萨出家后,履行了二万年往还的义务,投生到天界。从此堕后,其中的长者成了波罗奈国王的儿子,幼者成了祭司的儿子。他们在同一天结生,同一天出母胎,成了一起玩泥巴的朋友。祭司之子有智慧,他对王子说:「兄弟!你在父亲身后将得王位,而我得祭司之职,若善加学习,则能易治国家,来!我们去学艺!」
\item 随后,二人经颈绕圣线\footnote{颈绕圣线 \textit{yaññopavītakaṇṭhā}:这里从 PTS 本译出,缅甸本作「先前积累的业 \textit{pubbopacitakammā}」。},行乞于村镇,到了边境的村庄,众辟支佛在行乞时也进入此村。于是,人们看到众辟支佛后,生起热情,设好坐处,以精美的硬食、软食手授、敬献、供养。他俩想:「没有与我们相等的上等家族了,然而这些人若愿意便给我们施食,若不愿就不给,却对这些出家人这样恭敬,他们肯定知晓什么技艺,噫!我们到他们跟前去学艺。」
\item 当人群退去时,他们得了机会,便请求道:「尊者!你们所知的技艺,也请教给我们吧!」众辟支佛说:「未出家者不堪受学。」他们便经乞请而出了家。随后,众辟支佛以「你们当如是著下衣、如是披上衣」等方法告知了等正行\footnote{等正行,见\textbf{清净道论}·说戒品第 27 段。},分别施与了茅蓬:「此技艺由乐于独处而成就,所以,唯应独坐,独自经行、立、卧。」
\item 随后,他们便进入各自的茅蓬而坐。祭司之子从入坐之时起,便心得等持而得了禅。王子片刻即烦躁,来到他的跟前。他见到后便问:「兄弟!怎么了?」他便说:「我烦躁。」「那么,请坐在这里!」他立刻于此落坐,说:「兄弟!据说,此技艺由乐于独处而成就。」祭司之子说:「那么,兄弟!你就回到自己的坐处,我将获得此艺的成就。」他去后,片刻又再次烦躁,仍如前述,三次前往。随后,祭司之子仍旧驱赶了他,当他去后,想「他退失了自己的业,还屡屡来此退失我的」,便离了茅蓬,进入林野。
\item 另一个坐在自己的茅蓬内,片刻又开始烦躁,前往他的茅蓬,处处寻找,也不见他,便想「当在家时,他带着礼物前来也见不到我,现在当我到来却不愿见而走开,哎!咄!心!你不知耻,四次引领我至此,现在我将不受你的控制,我反倒要让你受制于我」,进入自己的坐卧处后,开始作观,证得辟支菩提,从空中去到了欢喜之源山坡。而另一个进入林野后,开始作观,证得辟支菩提,也到了那里。他二人都坐在雄黄之原,各自说了这慨叹的二颂。
\item 这里,\textbf{贤明},即天性聪明、智慧、善巧于遍的预作等。\textbf{善住},即具足安止之住或近行。\textbf{坚定},即具足坚毅。这里已经以贤明提及坚毅之成就,但于此唯是具足坚毅之义。坚毅即不松弛之勇猛,是以「宁愿皮、腱」等转起之精进的同义语。而且厌憎于恶亦为坚定。\end{enumerate}

\subsection\*{\textbf{46}}

\textbf{若不可得贤明、同行、善住、坚定的朋友,\\}
\textbf{如国王舍弃了征服的国土,他应当独自游行,如林中的大象。\footnote{第 45~46 两颂为同一缘起,且全同\textbf{法句}·象品第 328~329 颂,唯 PTS 本此颂末句仍作「像犀牛角一样」。}}

No ce labhetha nipakaṃ sahāyaṃ, saddhiṃcaraṃ sādhuvihāri dhīraṃ;\\
rājā va raṭṭhaṃ vijitaṃ pahāya, eko care mātaṅg’araññe va nāgo. %\hfill\textcolor{gray}{\footnotesize 12}

\begin{enumerate}\item \textbf{如国王舍弃了征服的国土},即如同敌王了知「征服的国土会带来不利」后,舍弃国土而独自游行,如是,他应当舍弃愚友而独自游行。或说其义亦为:如同闻月王、大生王\footnote{闻月王及大生王事,均见于\textbf{本生}。}舍弃征服的国土而独自游行,如是,他应当独自游行。其余依前述可知,不饶详繁。\end{enumerate}

\subsection\*{\textbf{47}}

\textbf{当然,我们赞叹成就的朋友,应亲近更胜或同等的朋友,\\}
\textbf{得不到这些,无过地受用,他应当独自游行,像犀牛角一样。}

Addhā pasaṃsāma sahāyasampadaṃ, seṭṭhā samā sevitabbā sahāyā;\\
ete aladdhā anavajjabhojī, eko care khaggavisāṇakappo. %\hfill\textcolor{gray}{\footnotesize 13}

\begin{enumerate}\item 此颂的缘起,直至众辟支佛坐于设在屋顶的坐处,都与游行四方一颂的缘起相同。其差别为:他不像那国王在夜里三次惊觉,也不准备献牲,在屋顶请众辟支佛坐于设好的坐处后,问:「你们是谁?」「大王!我们名为无过受用者。」「尊者!所谓无过受用者,其义为何?」「所得或胜或劣,皆无变而受用,大王!」
\item 国王听后,便想:「我何不考察他们是否如此。」当天,以糠伴醋给食。众辟支佛如受用甘露般,无变而受用。国王想「他们今天因自认故无变,明天我将知晓」,便又邀以明日。然后在第二天仍旧照做。他们也仍如是受用。于是,国王想「现在我将施以胜妙而观察」,又再次邀请,二日内施以大恭敬,以种种精美的硬食、软食给食。他们仍如是无变受用,对国王祝以吉祥后离去。当他们离去后不久,国王想「这些沙门确实是无过受用者,哎!我也愿成为无过受用者」,舍弃了广大的国土,受持出家而作观,成了辟支佛后,在曼珠沙树下,在众辟支佛中阐明自己的所缘时,说了此颂。
\item 此颂词义明了。其中,仅\textbf{成就的朋友},当知具足无学之戒等蕴的朋友,即成就的朋友。而其章句为:即是说\textbf{当然,我们赞叹}、一向赞美所说的成就的朋友。如何?\textbf{应亲近更胜或同等的朋友}。为什么?因为对亲近较自己的戒等更胜者,未生起的戒等法生起,已生起的得至增长广大,对亲近同等者,以彼此等同的受持及以除去恶作,所得不至退失。然而,\textbf{得不到这些}更胜或同等的朋友,在避开诡诈\footnote{诡诈:即拒绝资具诡诈事、迂回之谈诡诈事、假肃威仪诡诈事,见\textbf{清净道论}·说戒品第 66 段及以下。}等邪命后,以平等之法受用所得的食物,且于此不令嗔恚随眠生起,而成\textbf{无过地受用},欲求义利的族姓子应当独自游行,像犀牛角一样,且因我也如是而行,故证得此成就。\end{enumerate}

\subsection\*{\textbf{48}}

\textbf{看到锻工之子善加打造的光彩的金(钏)\\}
\textbf{双双在臂上哐当作响,他应当独自游行,像犀牛角一样。}

Disvā suvaṇṇassa pabhassarāni, kammāraputtena suniṭṭhitāni;\\
saṅghaṭṭamānāni duve bhujasmiṃ, eko care khaggavisāṇakappo. %\hfill\textcolor{gray}{\footnotesize 14}

\begin{enumerate}\item 缘起为何?某位波罗奈王在热季午休,而他的仕女在近旁研磨牛顶旃檀。她的一条手臂上有一个金钏,一条手臂上有两个,两两发出撞击,而另一个则不响。国王见后,反复观察这仕女,想到:「如是,当聚居则撞击,而独居则不响。」此时,王后以一切庄严装饰,立而摇扇。她想「我猜,国王于仕女牵绊其心」,便让这仕女起来,开始自己研磨。她的双臂上有许多金钏,它们撞击而发出大声。国王愈加厌烦,便右胁而卧开始作观,证得了辟支菩提。他乐于无上之乐而卧,王后手持旃檀而来,说:「大王!我来涂油!」国王说:「走开!别涂!」她说:「为何?大王!」他说:「我不是国王。」大臣们听到他们如是交谈,便即前来。被他们称为大王时,他也说:「我说,我不是国王。」其余仍与第一颂所说的相同。
\item 其颂释为:\textbf{看到},即观察。\textbf{金},即金钏的省文,因为带上省略的文字方成此义。\textbf{光彩},即是说光彩自然、具有光辉。其余之义自明。其章句为:看到臂上的金钏,想到「当聚居则撞击,而独居则不响」,我开始作观,证得辟支菩提。其余仍如前述。\end{enumerate}

\subsection\*{\textbf{49}}

\textbf{如是,与伴侣一起,我也会有言谈或执著,\footnote{颂首的「如是」,表示此颂实接上颂金钏的比方而言。}\\}
\textbf{觉察着这未来的怖畏,他应当独自游行,像犀牛角一样。}

Evaṃ dutīyena sahā mam’assa, vācābhilāpo abhisajjanā vā;\\
etaṃ bhayaṃ āyatiṃ pekkhamāno, eko care khaggavisāṇakappo. %\hfill\textcolor{gray}{\footnotesize 15}

\begin{enumerate}\item 缘起为何?某位波罗奈王在少年时即欲出家,命令大臣们:「你们带上王后,守卫国土,我将出家。」大臣们劝道:「大王!我们无法守护没有国王的国土,周边的国家将会前来掠夺,哪怕等到一个孩子出生!」国王心软,便接受了。然后,王后怀了孕。国王再次命令他们:「王后已怀孕,你们给降生的孩子灌顶以王位,守卫国土,我将出家。」大臣们再次劝说:「大王!王后生男生女难以知晓,请等到分娩的时候!」然后,她生下了男孩。此时,国王又照旧命令众大臣。大臣们也再次用许多理由劝说国王:「大王!请等到堪任!」随后,当王子堪任,便召集众大臣「他已堪任,你们给他灌顶以王位,追随之」,不给众大臣机会,教人从市场带来袈裟衣等所有资具,当即于宫内出家,像大生王一样离开。所有仆从种种悲泣,跟随着国王。
\item 国王等到了自己国家的边界,以手杖划了记号,说:「不可越过这记号。」大众在记号处磕头,卧在地上悲泣,让王子越过记号:「现在,亲爱的!王国听命于你,他能奈何?」王子边跑边喊「亲爱的、亲爱的」,追上了国王。国王见到王子后,想「我让这大众守卫国土,为何我不能守卫一个孩子呢」,便带上王子,进入林野,于此见到了先前的辟支佛所居住的茅蓬,便与孩子一起修整了住处。
\item 随后,王子习惯了上等的卧处等,躺在草垫或绳床上哭泣。当为寒风等所触,便说:「冷,亲爱的!热,亲爱的!苍蝇叮咬,亲爱的!我饿,亲爱的!我渴,亲爱的!」国王唯有劝慰着他过夜。而在白天行乞后带给他的食物,也是稗、谷、豆等多种混杂的食物。王子即便不喜,也迫于饥饿吃下,不过几日,便如热季摘下的莲花般萎顿。辟支菩萨却以省思之力,无变而食。
\item 随后,他劝说王子道:「亲爱的!在城里可得上好的食物,我们去那里吧!」王子说:「好!亲爱的!」随后,便领着他原路返回。而王子之母、王后想「现在,国王带着王子,在林野里无法久住,不过几日就会回来」,让人在国王以手杖划记之处围起篱笆,修整了住处。随后,国王站在这篱笆不远处,遣送王子:「你的母亲坐在那里,亲爱的!去吧!」他立而观察「莫让人伤害他」,直到他到达那里。王子跑着到了母亲跟前。护卫们看见他后,告知了王后。王后为二万舞女围绕前去迎接,并问了国王的经历。当听到「从后赶来」后,便派了人去。国王立刻去到自己的住处。人们不见国王,便也返回。随后,王后便不抱希望,带着孩子回到城中,给他灌顶以王位。国王则在到达自己的住处后,于此坐而修观,证得了辟支菩提,在曼珠沙树下的众辟支佛中,说了这慨叹之颂。
\item 此颂词义明了。而其中的旨趣为:作为伴侣的王子以冷暖等相告而同住,我对他会有劝说的言谈,且因爱执而生执著,如果我不舍弃他,则未来也会如现在一般,\textbf{如是,与伴侣一起,我也会有言谈或执著},而这两者都是证得殊胜的障碍,\textbf{觉察着这未来的怖畏},舍弃他后,如理行道,我证得了辟支菩提。其余仍如前述。\end{enumerate}

\subsection\*{\textbf{50}}

\textbf{爱欲实在多彩、甜蜜而悦意,以各色形相搅乱着心,\\}
\textbf{看到了种种爱欲\footnote{种种爱欲 \textit{kāmaguṇa}:旧译作「五欲功德」,Norman 说 guṇa 一词有「种种」的意思,见 M.W. 氏之梵英词典。}中的过患,他应当独自游行,像犀牛角一样。}

Kāmā hi citrā madhurā manoramā, virūparūpena mathenti cittaṃ;\\
ādīnavaṃ kāmaguṇesu disvā, eko care khaggavisāṇakappo. %\hfill\textcolor{gray}{\footnotesize 16}

\begin{enumerate}\item 缘起为何?据说,在波罗奈的商人之子在少年时即得商人的地位。对应三个季节,他有三个楼阁。他在那里以一切财富自娱,如天界的童子一般。他仍在少年时,便向父母请求「我要出家」,他们遮止他。他就如是继续纠缠,父母再次以「亲爱的!你体质纤弱,而出家难为,如履刀刃」等种种方式遮止。他就如是继续纠缠,他们便想「如果他出家,则我们忧伤,如果我们遮止他,则他忧伤,还是让忧伤归我们吧,莫要归他」,便同意了。
\item 随后,他不顾所有人的悲泣,到了仙人堕处,在众辟支佛跟前出了家。未得到舒适的坐卧处,他在床上铺了垫子便睡了。他习惯了上等的卧处,整夜难捱。早上洗漱完毕,还要持衣钵与众辟支佛一起行乞。于此,耆宿们得到上等的坐处与上等的食物,新进们唯有随宜的坐处与粗鄙的食物。他又苦于粗鄙的食物。没过几天,他就容貌憔悴,且其沙门法未至成熟,便生嫌厌。随后,他派人至父母处,便还了俗。没过几天,他恢复了体力,便又想出家。随后,以同样的过程出家后,又再次还俗,在第三次出家时,以正当的行道证得了辟支菩提,说了这慨叹之颂后,在众辟支佛中又说了这解答之颂。
\item 这里,\textbf{爱欲},即两种爱欲,物欲和烦恼欲\footnote{物欲、烦恼欲:欲即所欲之物,欲即能欲之烦恼,参见\textbf{清净道论}·说地遍品第 83 段,叶均译物欲为「事欲」。}。这里的物欲为适意的色等法,烦恼欲为欲等一切贪染的种类,而这里指的是物欲。以色等多种行相而为\textbf{多彩},以世间之味为\textbf{甜蜜},以使愚痴凡夫之意悦乐为\textbf{悦意}。\textbf{各色形相},即是说各异的形相、各种的自性。因为它们以色等而多彩,且以色等中的青等而成种种色。如是,以各色形相如此这般显示其味后,\textbf{搅乱着心},不令其欢喜于出家。其余则自明。结语与二或三句连结,当知仍如先前几颂所述。\end{enumerate}

\subsection\*{\textbf{51}}

\textbf{「这对于我是灾、痈、祸害,是病、箭、怖畏」,\\}
\textbf{看到了这种种爱欲中的怖畏,他应当独自游行,像犀牛角一样。}

“Ītī ca gaṇḍo ca upaddavo ca, rogo ca sallañ ca bhayañ ca m’etaṃ”;\\
etaṃ bhayaṃ kāmaguṇesu disvā, eko care khaggavisāṇakappo. %\hfill\textcolor{gray}{\footnotesize 17}

\begin{enumerate}\item 缘起为何?据说,在波罗奈的国王患了痈,生起阵阵剧痛。医生们说:「除非手术,病不能瘥。」国王施以无畏,便让他们动手术。他们切开后,挤出脓血,止住疼痛,用布包扎了创口,并教导他正当的饮食起居。国王因粗鄙的食物而身形瘦削,但他的痈竟消去了。
\item 他想既然安乐,便吃了油腻的食物。且因之恢复气力,便追逐境域。他的痈便又复归先前的状态。如是经过三次手术后,遭到医生的拒绝,便生起嫌厌,舍弃王位而出家,进入林野作观,以七年证得辟支菩提,说了这慨叹之颂后,到了欢喜之源山坡。
\item 这里,以到来为\textbf{灾},是外来者、不善的部分、不幸之因的同义语。所以,种种爱欲以带来诸多不幸之义、以紧密结合之义而为灾。\textbf{痈}则漏泄不净,而有肿胀、成熟、破裂。所以,它们由烦恼不净的漏泄及由因生、老、坏的肿胀、成熟、破裂而为痈。以恼害为\textbf{祸害},即令生不利而征服、吞没之义,是国王的刑罚等的同义语。所以,种种爱欲以作为带来未知、沉重的不利之因及以作为一切祸害的依处而为祸害。
\item 因为它们令称为戒的无病生烦恼疾病,或令生起贪,而劫掠自然之无病,所以以此劫掠无病之义而为\textbf{病}。以进入内部之义、刺入内里之义、难以拔出之义而为\textbf{箭}。由带来现法、来世的怖畏而为\textbf{怖畏}。其余于此自明。结语当知仍如前述。\end{enumerate}

\subsection\*{\textbf{52}}

\textbf{寒热饥渴,以及风晒虻蛇,\\}
\textbf{克服了这一切,他应当独自游行,像犀牛角一样。}

Sītañ ca uṇhañ ca khudaṃ pipāsaṃ, vātātape ḍaṃsasarīsape ca;\\
sabbāni p’etāni abhisambhavitvā, eko care khaggavisāṇakappo. %\hfill\textcolor{gray}{\footnotesize 18}

\begin{enumerate}\item 缘起为何?据说,在波罗奈,有国王名畏寒梵赐。他出家后,住在林野中的寮房。而此地在寒季酷寒,在热季酷热,由地处露天之故。他在行处村落不得如意的食物,堪饮之水也难得,还有风晒虻蛇等逼恼。他便想:「离此仅半由旬处即有具足之地,那里没有这一切危难,我何不去到那里,藉安住便能证得殊胜?」他又想:「出家者,即不受诸缘的制约,且于制约下转起如是之心,而非于心的制约下转起,我不应去。」经省思后,便不去了。如是三次省思生起之心,予以阻止。随后即于此地住了七年,经正当的行道证得了辟支菩提,说了这慨叹之颂后,到了欢喜之源山坡。
\item 这里,\textbf{寒},有两种寒,缘自内界的扰动与缘自外界的扰动,\textbf{热}也一样。\textbf{虻},即褐蝇。\textbf{蛇},即凡长身而游行者。其余于此自明。结语当知仍如前述。\end{enumerate}

\subsection\*{\textbf{53}}

\textbf{如象离了群,肩背宽阔、莲色而伟岸,\\}
\textbf{随所欢喜地住于林野,他应当独自游行,像犀牛角一样。}

Nāgo va yūthāni vivajjayitvā, sañjātakhandho padumī uḷāro;\\
yathābhirantaṃ viharaṃ araññe, eko care khaggavisāṇakappo. %\hfill\textcolor{gray}{\footnotesize 19}

\begin{enumerate}\item 缘起为何?据说,在波罗奈,某位国王在统治了二十年后死去,在地狱煎熬了二十年后,于喜马拉雅地区的象胎内投生,成了肩背宽阔、全身莲色、伟岸的群主大象。幼象们吃了他的截截断枝,而在涉水时,母象们也用泥甩它,一切都如 Pālileyyaka 象\footnote{Pālileyyaka 象,见\textbf{自说}第 4:5 经。}一般。它嫌厌了群队而离开,而群队却追随着它的足迹。如是三次离开,仍被追随,随后,它便想:「现在,我的孙子统治着波罗奈,我何不去到自己前生的园林?在那里,他会保护我。」
\item 随后,在晚上,当群队入睡,它便舍弃群队,进到那园林。护园人见后,告知了国王。国王想「我要捕象」,便带上军队。象朝着国王走去。国王想「它朝我走来」,便端箭而立。随后,象想「他竟然会射我」,以人类的语言说:「梵赐!别射我!我是你祖父。」国王说「你说什么」,便问了一切。象也将王位、地狱、象胎等的所有本末告知。国王说「妙哉!别害怕!但也别恐吓任何人」,命人为象准备了场地、守卫和物品。
\item 于是,某天,国王上到象背,想「他在统治了二十年后,在地狱煎熬,以剩余的异熟投生在畜生的胎内,在那里不堪忍受群居的接触而来至此,哎!群居甚苦,唯独一为乐」,即于此作观,证得了辟支菩提。他乐于出世间之乐,大臣们前来跪拜后说:「该走了,大王!」随后,他说「我不是国王」,以先前的方式说了此颂。
\item 此颂词义明了,而此处为其旨趣与章句,且此唯以理应,而非以传闻。好比这象,由于人类所悦的戒调伏而不至于未调伏之地,或者由身量巨大而为\textbf{象}\footnote{义注对「象」的解释,下文的「不至、不再造罪、不来此间」等均为语源上的拼凑。},如是,我何时也能由于圣者所悦的戒调伏,以不至于未调伏之地、以不再造罪,且以不来此间而为象,或者由功德身量巨大而为象。又好比它\textbf{离了群},以独行之乐\textbf{随所欢喜地住于林野,独自游行,像犀牛角一样},我何时也能如是,离开群体,以独住之乐、禅那之乐而随所欢喜地住于林野,或随己所乐、依我所愿地在林野居住,独自像犀牛角一样游行之义。
\item 又好比它由善伸展的肩背而\textbf{肩背宽阔},我何时也能如是,由无学戒蕴\footnote{肩背与蕴的巴利文均为 khandha。}的广大而成肩背宽阔。又好比它由莲花般的肢体,或由投生于莲花家族而为\textbf{莲色},我何时也能如是,由莲花般正直的身形,或由投生于圣生的莲花而成莲色。又好比它由强壮、力量、速度等而\textbf{伟岸},我何时也能如是,由遍净的身正行等,或由戒、定、抉择、慧等而成伟岸。如是思量时,开始作观,我得证辟支菩提。\end{enumerate}

\subsection\*{\textbf{54}}

\textbf{「对耽乐聚会者,不可能证得暂时的解脱」,\\}
\textbf{留意到日种的话语,他应当独自游行,像犀牛角一样。}

Aṭṭhāna taṃ saṅgaṇikāratassa, yaṃ phassaye sāmayikaṃ vimuttiṃ;\\
Ādiccabandhussa vaco nisamma, eko care khaggavisāṇakappo. %\hfill\textcolor{gray}{\footnotesize 20}

\begin{enumerate}\item 缘起为何?据说,波罗奈国王的儿子尚在少年便欲出家,便向提出父母请求,父母遮止他。虽被遮止,他仍执意「我要出家」。随后,如先前所说的商人之子一般,在对他说完一切后,他们便同意了。并让他答应在出家后仍旧住在园林,他便照做。他的母亲早上为二万舞女围绕,去到园林,让儿子喝了粥,期间又让他吃了硬食等,同他交谈,直到中午才进城。而父亲中午到来,让他吃了饭,自己也吃了之后,白天同他交谈,晡时安置了警卫后进城。他如是日夜不得独处而住。
\item 尔时,名为「日种」的辟支佛住于欢喜之源山坡。他在转向时便看到了他:「这童子已能出家,却不能斩断结缚。」随后,更转向于「他是否能以自己的法性而厌离」。于是,在了知了「以法性而厌离将要很久」后,想「我要向他显示所缘」,以先前所说的方法从雄黄之原出发,立于园林。王臣见后便告知国王:「辟支佛来了,大王!」国王心生欢喜「现在,我儿将与辟支佛一起无烦而住」,恭敬地给侍辟支佛,请求即于此住下,命人建造了茅蓬、昼住处、经行道等一切,请他住下。
\item 他住于此时,一天,得了机会,便问童子:「你是谁?」他说:「我是出家人。」「出家人没有这样的。」「那么,尊者!他们是怎样的?我有何不当之处?」「你尚不能觉察自己的不当之处:你母亲是否与二万女子一起在上午前来,使园林不得远离?父亲是否在晡时与大军一起?警卫是否整夜?出家人没有如你这般的,而是这样的」,便在所立之处以神变显现雪山中的某寺院。他见到众辟支佛在那里或倚栏杆而立,或经行,或染衣缝补,便说:「你们没来此处,也允许出家?」「唯!允许出家,从出家之时起,即名沙门,可以行自己的出离、去到所愿或希求之处,唯此为适当」,言毕,站在空中,说了这前半颂后,在注视下举身去到欢喜之源山坡。
\item 如是,当辟支佛去后,他便进入自己的茅蓬卧下。守卫想「童子躺着,现在能去哪里」,便也放逸入睡。他在了知其放逸后,拿了衣钵,便进入林野。且在那里独处而开始作观,证得辟支菩提后,到了辟支佛的所在。那里,当被问及「如何证得」时,补全了日种所说的半颂,说了(此颂)。
\item 其义为:\textbf{不可能},即是说无原由。\textbf{耽乐聚会},即乐于群居。\textbf{暂时的解脱},即世间的等至。因为唯有每每安止之时才能从诸敌对中解脱,故说「暂时的解脱」。\textbf{留意到}这\textbf{日种}辟支佛的\textbf{话语}「对耽乐聚会者,这暂时的解脱也不可能,无有以之能证的原由」,我舍弃了聚会之乐,经如理行道而证得。其余仍如前述。\end{enumerate}

\subsection\*{\textbf{55}}

\textbf{「已超越见的蠢动,已达决定,已获得道,\\}
\textbf{「我生起智,不由他人引导」,他应当独自游行,像犀牛角一样。}

“Diṭṭhīvisūkāni upātivatto, patto niyāmaṃ paṭiladdhamaggo;\\
uppannañāṇo’mhi anaññaneyyo”, eko care khaggavisāṇakappo. %\hfill\textcolor{gray}{\footnotesize 21}

\begin{enumerate}\item 缘起为何?据说,在波罗奈,某位国王在幽居时想:「好比对寒冷等,有热等的防御,那么有没有作为流转之防御的还灭呢?」他问大臣们:「你们知道还灭吗?」他们说:「我们知道,大王!」国王问「那是什么」,随后,他们以「世间有边」等方法谈论常断。于是,国王自己见到其中的分歧与不当「他们不知道,他们全都是具见者」,想「存在作为流转之防御的还灭,应予寻求」,便舍弃王位,出家后修观,证得了辟支菩提。在辟支佛中间说了这慨叹之颂与解答之颂。
\item 其义为:\textbf{见的蠢动},即具六十二见。因为它们对道的正见以相违之义、刺穿之义、分歧之义而为蠢动。如是以见而蠢动,或者见即蠢动,为见的蠢动。\textbf{超越},即以见道而越过。\textbf{已达决定},即已证得非堕处法性的、以菩提为彼岸的决定状态,或即称为正性决定的初道。至此是说初道作用的完成及其获得。
\item 现在,\textbf{已获得道},即以此显示获得其余的道。\textbf{我生起智},即我已生起辟支菩提之智,以此显示果。\textbf{不由他人引导},即不受他人「此为真实、此为真实」等的引导,以此显明自成,或于已达的辟支菩提智,由无有他人引导而显明自在。
\item 或者,以止观而超越见的蠢动,以初道而已达决定,以其余而已获得道,以果智而生起智,且由自己证得这一切为不由他人引导。其余当知仍如前述。\end{enumerate}

\subsection\*{\textbf{56}}

\textbf{离贪求、离诡诈、离渴望,离覆藏,除去恶浊、愚痴,\\}
\textbf{于一切世间无意乐,他应当独自游行,像犀牛角一样。}

Nillolupo nikkuho nippipāso, nimmakkho niddhanta-kasāva-moho;\\
nirāsayo sabbaloke bhavitvā, eko care khaggavisāṇakappo. %\hfill\textcolor{gray}{\footnotesize 22}

\begin{enumerate}\item 缘起为何?据说,波罗奈国王的厨师在煮了悦目、美味的点心后便献上:「兴许国王会赐我财富。」仅凭香味,即令国王对其生起食欲,口内生津,而当往嘴里投下第一口时,七千味蕾便如为甘露所触。厨师想:「现在他要赐我了、现在他要赐我了。」国王也认为「厨师应得奖赏」,但想:「尝味后即行奖赏,会对我产生坏名声『这国王贪求、重味』。」便什么也没说。如是,直到食事终了,厨师还在想:「现在要赐了、现在要赐了。」国王因畏惧不名誉,仍什么也没说。
\item 随后,厨师想「这国王没有舌识」,第二天便献上无味的食物。国王吃时,即便知道「厨师今天该罚」,如先前一样省察后,因畏惧不名誉,什么也没说。随后,厨师想「国王既不知美味,也不知恶味」,自己拿了薪水,随便煮点什么就给国王。国王嫌厌道「哎!真是贪,我食邑二万,因他的贪,竟不得一餐」,便舍弃王位,出家后修观,证得了辟支菩提,并仍以先前的方法说了此颂。
\item 这里,\textbf{离贪求},即无贪求。因为被味爱征服者会强烈贪求且反复贪求,因此被称为「贪求」,所以他拒绝之而说「离贪求」。此中的\textbf{离诡诈},凡是无有三种诡诈事者被称为「离诡诈」,而在此颂中,于美食等不至惊叹的「离诡诈」是其意。此中的\textbf{离渴望},欲饮为渴,以其无有而为离渴,即除去因贪图美味的欲食之义。此中的\textbf{离覆藏},以灭失他人功德之相为覆藏,以无有此而为离覆藏,这是就自己在家时覆藏厨师的功德而说。
\item 此中的\textbf{除去恶浊、愚痴},当知贪等三与身恶行等三的六法,分别以不净喜之义、以令舍弃自性执取他性之义、以恶秽之义为恶浊。如说:\begin{quoting}这里,什么是三恶浊?贪恶浊、嗔恶浊、痴恶浊,这些即三恶浊。这里,更有什么是三恶浊?身恶浊、语恶浊、意恶浊。(分别论第 924 段)\end{quoting}其中,由对除了痴的五恶浊及作为这一切根本之愚痴的除去,为除去恶浊、愚痴,或者,只是身语意恶浊及愚痴之除去为除去恶浊、愚痴,而于其它,以离贪求等为贪恶浊除去的成就,以离覆藏为嗔恶浊(除去的成就)。\textbf{无意乐},即无渴爱,\textbf{一切世间},即整个世间,即于三有或十二处已无有有爱、无有爱之义。其余当知仍如前述。或者,在说了三句之后,此中的\textbf{他应当独自游行}也可作「他堪能独自游行」解。\end{enumerate}

\subsection\*{\textbf{57}}

\textbf{他应回避示以非义、住于不正的恶友,\\}
\textbf{莫自己亲近执著而放逸者,他应当独自游行,像犀牛角一样。}

Pāpaṃ sahāyaṃ parivajjayetha, anatthadassiṃ visame niviṭṭhaṃ;\\
sayaṃ na seve pasutaṃ pamattaṃ, eko care khaggavisāṇakappo. %\hfill\textcolor{gray}{\footnotesize 23}

\begin{enumerate}\item 缘起为何?据说,在波罗奈,某位国王以极盛大的王家仪仗巡城时,看到人们将陈谷从粮仓搬出,便问大臣们:「我说,这是什么?」「现在,大王!新谷将熟,这些人为它们腾出地方,丢掉陈谷。」国王说:「我说,难道后宫、军队等的场地也满了吗?」「唯!大王!已满。」「既然如此,我说,命人建布施堂,我将布施,莫让这些谷物无益亡失!」
\item 随后,某具见的大臣开始对他说「大王!并无所施」,乃至以「无论贤愚,在转世、轮回之后都能尽苦边」来遮止。他第二次、第三次看到劫掠粮仓者后,仍如是下令。第三次,大臣对他说「大王!布施乃是愚人的说教」等来遮止。他生起嫌厌「咄!我甚至不能布施自己的财产,这些恶友于我何有」,便舍弃王位,出家后修观,证得了辟支菩提,且叱责那恶友而说了这慨叹之颂。
\item 其略义为:对于由具足十事恶见\footnote{十事恶见,同十事邪见,见\textbf{法集论}第 1221 段:「什么是见取?无所施,无所献,无供养,无善恶业之果异熟,无此世间,无他世间,无母,无父,无化生有情,世上无正行、正行道的沙门、婆罗门——他们以自身的证智得证此世他世而宣说。」}而为\textbf{恶}、乃至向他人展示非义为\textbf{示以非义}、\textbf{住于}身恶行等的\textbf{不正}者,欲求义利的族姓子应回避这示以非义、住于不正的恶友。\textbf{莫自己亲近},即是说不应自愿地亲近,但若有他人的权势,又能奈何。\textbf{执著},即布满,以见而固著于彼彼之义。\textbf{放逸},即醉心于种种爱欲,或无有善的修习。于此等人,不应亲近、不应结交、不应承事,而应当独自游行,像犀牛角一样。\end{enumerate}

\subsection\*{\textbf{58}}

\textbf{他应结交多闻、持法、高尚、富有辩才的朋友,\\}
\textbf{知晓了义利,调伏了疑惑,他应当独自游行,像犀牛角一样。}

Bahussutaṃ dhammadharaṃ bhajetha, mittaṃ uḷāraṃ paṭibhānavantaṃ;\\
aññāya atthāni vineyya kaṅkhaṃ, eko care khaggavisāṇakappo. %\hfill\textcolor{gray}{\footnotesize 24}

\begin{enumerate}\item 缘起为何?据说,先前在迦叶世尊的教法内,八位辟支菩萨出家后,履行了往还的义务而投生天界,一切与无过受用一颂所说的相同。其差别为:让众辟支佛落坐后,国王说:「你们是谁?」他们说:「大王!我们名为多闻者。」国王想「我名为所闻梵赐,无厌足于所闻,噫!在他们跟前,我将听闻丰富多彩的善法教」,心满意足,施与供养与水并施食,在食事终了,取了僧中上座的钵,礼拜后坐在前面说:「尊者!请说法!」他说了「愿你快乐!大王!愿贪尽」后便起身。国王想「这不算多闻,第二位也许多闻,明天我将听闻丰富的法教」,便以明日相招请。如是招请,直至全都顺次而来,全都仅以「愿嗔尽、痴尽、趣尽、流转尽、依持尽、渴爱尽」等差别以一字,其余均同第一位,说后便起身。
\item 随后,国王想「他们说『我们多闻』,但其所说却并不丰富,他们说了什么」,开始审思其语义。于是,当审思「愿贪尽」时,了知到「当贪尽时,嗔、痴及其余烦恼都灭尽」,便心满意足,想「这些沙门是不重方法\footnote{不重方法 \textit{nippariyāya},这里也可译作「直接」,说详下文第 67 颂注。}的多闻者,好比有人以指指向大地或天空时,不是只示以指尖之量的部分,而是示以地或天,如是,他们在说一一义时,是示以无量之义」。随后,他想「我何时也能成为如是多闻」,希求着这样的多闻,便舍弃王位,出家后修观,证得辟支菩提后,说了这慨叹之颂。
\item 这里,其略义为:\textbf{多闻},有两种多闻,即藉由三藏语义的完整的教法多闻,以及由通达道、果、明、神通等的通达多闻。通晓阿含为\textbf{持法}。而具足高尚的身语意业为\textbf{高尚}。正确应对、迅速应对、正确且迅速应对为\textbf{富有辩才}\footnote{辩才 \textit{paṭibhāna},这里也译作「应对」。正确应对、迅速应对等等,见\textbf{增支部}第 4:132 经。}。或当知以教法、遍问、证得等为三种辩才。因为教法对其显露者,即富有教法辩才,对遍问义、智、相、理、非理者,遍问显露,即富有遍问辩才,以之而有道等的通达者,即富有证得辩才。他应结交如此多闻、持法、高尚、富有辩才的朋友。
\item 随后,以其势力,从自义、他义、两者之义等,或从现法、来世、第一义等,\textbf{知晓了}多种品类的\textbf{义利}。随后,于「过去世我是否曾存在」等的疑惑处\textbf{调伏了疑惑},调伏、消除了疑虑,如是已作了一切应作者,他堪能独自游行,像犀牛角一样。\end{enumerate}

\subsection\*{\textbf{59}}

\textbf{不满于世间的嬉戏、喜乐和欲乐,便不再关切,\\}
\textbf{戒离严饰,言语真实,他应当独自游行,像犀牛角一样。}

Khiḍḍaṃ ratiṃ kāmasukhañ ca loke, analaṅkaritvā anapekkhamāno;\\
vibhūsanaṭṭhānā virato saccavādī, eko care khaggavisāṇakappo. %\hfill\textcolor{gray}{\footnotesize 25}

\begin{enumerate}\item 缘起为何?在波罗奈,名为严饰梵赐的国王早上吃了粥或饭,命人以种种饰品严饰自己后,在大镜子前检视全身,除去不想要的,再命人以别的饰品严饰。一天,他正这么做时,就到了中午用餐的时间。于是,尚未严饰,以白布包了头,吃了后便去午休。当再起身仍这么做时,已到日落。第二天、第三天也如是。于是,他如是热衷装饰,便患了背疾。他便想「哎!咄!我用了全部力气来严饰,仍不满足于这般严饰,令贪生起,而这贪是趣向苦处之法,噫!我要抑止贪!」便舍弃王位,出家后修观,证得辟支菩提后,说了这慨叹之颂。
\item 这里,\textbf{嬉戏、喜乐}如先前所说。\textbf{欲乐},即物欲之乐。因为物欲因作为乐的境域等故称为乐。如说:\begin{quoting}色是乐,沦于乐……(相应部第 22:60 经)\end{quoting}如是,\textbf{不满于}这器\textbf{世间}的嬉戏、喜乐和欲乐、不以为足、不执之为「这即满足或有价值」。\textbf{便不再关切},即因这不满而不习于关切、不渴望、离渴爱。\textbf{戒离严饰,言语真实,他应当独自游行},这里,有两种严饰,即在家严饰与出家严饰,在家严饰即衣服、头巾、花鬘、芳香等,出家严饰即钵、装饰等,所以,以三种离\footnote{三种离:即既得离、受持离、正断离,见\textbf{吉祥经}·第 267 颂注。}而戒离严饰。由无虚妄之语为言语真实。\end{enumerate}

\subsection\*{\textbf{60}}

\textbf{孩子、妻子、父亲、母亲,财富、谷物及众眷属,\\}
\textbf{舍弃了如其限度的爱欲,他应当独自游行,像犀牛角一样。}

Puttañ ca dāraṃ pitarañ ca mātaraṃ, dhanāni dhaññāni ca bandhavāni;\\
hitvāna kāmāni yathodhikāni, eko care khaggavisāṇakappo. %\hfill\textcolor{gray}{\footnotesize 26}

\begin{enumerate}\item 缘起为何?据说,波罗奈国王的儿子尚在少年时便受灌顶,统治了王国。他与第一颂中所说的辟支菩萨\footnote{第一颂中所说的辟支菩萨,见第 35 颂注第 16 段及以下。}一样,享受着王室的光耀,一天,他便想:「我统治着王国,给众人造了许多苦,难道以此罪恶只为了一餐?噫!我要让乐生起!」便舍弃王位,出家后修观,证得辟支菩提后,说了这慨叹之颂。
\item 这里,\textbf{财富},即珍珠、摩尼、琉璃、螺贝、珊瑚、银、金等宝物。\textbf{谷物},即稻、米、大麦、小麦、黍、豆、稗等七类与其余谷类。\textbf{众眷属},即亲属、族属、友属、职属等四类眷属。\textbf{如其限度}\footnote{如其限度 \textit{yathodhi} 一词亦见于\textbf{中部}·布经第 75 段,\textbf{义释}说「以须陀洹道舍弃之烦恼不会再来、不会回来、不会返回,以斯陀含道……阿那含道……阿罗汉道舍弃之烦恼不会再来、不会回来、不会返回」。},即以各自的限度而存者。其余仍如前述。\end{enumerate}

\subsection\*{\textbf{61}}

\textbf{「这是染著,于此幸福有限,乐味些许,而苦于此更多,\\}
\textbf{「这是钓钩」,具慧者如此了知已,他应当独自游行,像犀牛角一样。}

“Saṅgo eso parittam ettha sokhyaṃ, app’assādo dukkham ettha bhiyyo;\\
gaḷo eso” iti ñatvā matīmā, eko care khaggavisāṇakappo. %\hfill\textcolor{gray}{\footnotesize 27}

\begin{enumerate}\item 缘起为何?据说,在波罗奈,有国王名为游步梵赐。他早上吃了粥或饭后,到三处宫殿去看三辈舞女。据说,三辈舞女来自先前的国王、上任国王及成长于自己的时代。一天,他在早上去到青年舞女的宫殿。那些舞女想「让我们取悦国王」,如诸天因陀帝释的天女一般,表演了极迷人的歌舞器乐。国王却不满足「这对青年并不希有」,去到中年舞女的宫殿。那些舞女也这样表演。他于此仍不满足,去到老年舞女的宫殿。那些舞女也这样表演。他看了以流传了二三代国王的老态(表演的)如枯骨嬉戏般的舞蹈,听了并不甜蜜的歌曲,重又来回徘徊于青年舞女的宫殿、中年舞女的宫殿间,无处可得满足,便想:「这些舞女如诸天因陀帝释的天女一般,为取悦我,使出全部力气来表演歌舞器乐,而我却无处满足,徒然令贪增长,而这贪是趣向苦处之法,噫!我要抑止贪!」便舍弃王位,出家后修观,证得辟支菩提后,说了这慨叹之颂。
\item 其义为:\textbf{这是染著},即显示自己的受用。因为群生羁绊于此,如象陷入泥潭一般,故为「染著」。\textbf{于此幸福有限},即于此受用种种五欲时,由以颠倒想而生,或由系属于欲界法,幸福以低劣之义而为有限,即是说如因电光的照亮得见舞蹈之乐般短暂、暂时。\textbf{乐味些许,而苦于此更多},此中,这如\begin{quoting}诸比丘!这些缘于种种五欲生起的乐与喜,即为爱欲的乐味。(中部·大苦蕴经第 166 段)\end{quoting}所说,它即以\begin{quoting}诸比丘!什么是爱欲的过患?此处,诸比丘!族姓子以技艺而谋生者,若以计算、若以算数……(中部·大苦蕴经第 167 段)\end{quoting}等方法说为苦,相较于此,些许为水滴之量,而苦却多出许多,如四大海中的水一般,因此说「乐味些许,而苦于此更多」。\textbf{这是钓钩},即示以乐味后,以牵引之势如同钩针,「这」即此种种五欲。\textbf{具慧者如此了知已},即具觉者、有智之士如是了知已,舍弃了这一切,他应当独自游行,像犀牛角一样。\end{enumerate}

\subsection\*{\textbf{62}}

\textbf{摆脱了结缚,如水中游鱼冲破了网,\\}
\textbf{如火不返余烬,他应当独自游行,像犀牛角一样。}

Sandālayitvāna saṃyojanāni, jālaṃ va bhetvā salil’ambucārī;\\
aggīva daḍḍhaṃ anivattamāno, eko care khaggavisāṇakappo. %\hfill\textcolor{gray}{\footnotesize 28}

\begin{enumerate}\item 缘起为何?据说,在波罗奈,有国王名为不返梵赐。当他进入战场,不胜不返,或当开始别的工作,不令完成不返,所以人们这样称呼他。一天,他去到园林。此时,林火生起。这火烧着干草与新草等,不返而前。国王见到后,生起了与之相似的相:「好比这林火,如是十一种火\footnote{十一种火:即\textbf{燃烧经}中所说的「贪、嗔、痴、生、老、死、愁、悲、苦、忧、恼」等火。}燃烧着一切有情,不返而前,令生大苦,我何时也能如这火一般,为了折返此苦,以圣道智之火烧却烦恼,不返而前?」
\item 随后,走了片刻,他看见渔夫们在河边捉鱼。一条进入网中的大鱼冲破了网而逃脱。他们作声:「鱼冲破网走了。」国王听到这话后,生起了与之相似的相:「我何时也能以圣道智冲破爱与见之网,无所羁绊而行?」他便舍弃王位,出家后开始作观,证得了辟支菩提,并说了这慨叹之颂。
\item 这第二句中,\textbf{网},即绳线所造者。第三句中,\textbf{余烬},即所烧之处。即是说好比火不再返回所烧之处,不更来此处,如是不再返回为道智之火所烧的种种爱欲之处,不更来此处。其余仍如前述。\end{enumerate}

\subsection\*{\textbf{63}}

\textbf{目光下视,且不游步,防护诸根,守护于意,\\}
\textbf{不漏泄,不为所烧,他应当独自游行,像犀牛角一样。}

Okkhittacakkhu na ca pādalolo, guttindriyo rakkhitamānasāno;\\
anavassuto apariḍayhamāno, eko care khaggavisāṇakappo. %\hfill\textcolor{gray}{\footnotesize 29}

\begin{enumerate}\item 缘起为何?据说,在波罗奈,有国王名为游目梵赐,与游步梵赐一样,碌碌于观看舞女。其差别为:他不满时便处处寻访,看到各个舞女后,便极其欢喜,为见舞女围绕令渴爱增长而巡游。据说,他见到某个为看舞女而来的地主之妻,便生起贪染。随后,又陷入悚惧「我令这渴爱增长,将成满苦处,噫!我要抑止它」,出家后修观,证得辟支菩提后,为谴责自己先前的行径,说了这显明其对治之功德的慨叹之颂。
\item 这里,\textbf{目光下视},即向下注目,是说顺次安放七节颈椎后,为显示应以回避舍断\footnote{应以回避舍断(的诸漏),见\textbf{中部}·一切漏经第 25 段,这里据 PTS 本译出,原本作「应以回避执取 \textit{parivajja-gahetabba}」。}而眼见一寻之地的意思。但不应以颚骨接触胸骨,因为这样的目光下视不适合沙门。\textbf{且不游步},即不会因想要进入群体之中,如「成为单人之第二者、二人之第三者」等而脚底发痒,或不作远游、无目的的游行等。\textbf{防护诸根},即对于六根,分别以此处的所及之余\footnote{所及之余 \textit{vuttāvasesa}:菩提比丘认为是意根,但这里似指非当下守护的诸根,待考。}来守卫诸根。\textbf{守护于意},即是说守护于心,使不为烦恼所侵蚀。
\item \textbf{不漏泄},即以此修习,于彼彼所缘无有烦恼的漏入。\textbf{不为所烧},即由如是无有漏入而不为烦恼之火所烧。或于外不漏泄,于内不为所烧。其余仍如前述。\end{enumerate}

\subsection\*{\textbf{64}}

\textbf{除去了俗家相,好比树叶密覆\footnote{树叶密覆 \textit{sañchannapatto}:PTS 本作「树叶落尽 \textit{sañchinnapatto}」,同第 44 颂,说详菩提比丘注 503。}的波利质多树\footnote{波利质多树:也作昼度树、珊瑚树、香遍树等,即刺桐,慧苑华严音义「此云香遍树,谓此树根、茎、枝、叶、花、实皆能遍熏忉利天宫」。},\\}
\textbf{著袈裟衣而出家后,他应当独自游行,像犀牛角一样。}

Ohārayitvā gihibyañjanāni, sañchannapatto yathā pārichatto;\\
kāsāyavattho abhinikkhamitvā, eko care khaggavisāṇakappo. %\hfill\textcolor{gray}{\footnotesize 30}

\begin{enumerate}\item 缘起为何?据说,在波罗奈,另有国王名为四月梵赐,每四月去到园林游玩。热季中月的一天,当他进园林时,在园林入口看见树叶密覆、花满枝头的波利质多黑檀,摘了朵花便进了园林。随后,某大臣想「最好的花被国王摘了」,便站在象背上摘了一朵。以此方法,整个大军都摘了,未享受到花的人甚至把叶子也摘了。那树便无花无叶,徒余枝干。当国王在晡时离开园林时见后,想着「这树怎么了?我来之时,摩尼色的枝干间还缀以珊瑚般的花朵,现在已成无花无叶」,而就在这不远处,他看到尚未开花的树,树叶密覆。
\item 见后,他便想到:「那树因花满枝条而为众人所贪,须臾之间便至零落,而这树由无可贪,便如是而存。这王位也如开花之树般可贪,而比丘的身份则如未开花之树般无可贪,所以,他也应当在如这树未被侵蚀、像另一树叶密覆的波利质多树一样时,披覆袈裟而出家。」他便舍弃王位,出家后修观,证得辟支菩提后,说了这慨叹之颂。
\item 这里,\textbf{著袈裟衣而出家后},此句当知即「从家出家而著袈裟衣」之义。其余仍如前述可知,不饶详繁。\end{enumerate}

\subsection\*{\textbf{65}}

\textbf{于众味不贪图、不动摇,不养育他人,次第行乞,\\}
\textbf{于各家不牵绊其心,他应当独自游行,像犀牛角一样。}

Rasesu gedhaṃ akaraṃ alolo, anaññaposī sapadānacārī;\\
kule kule appaṭibaddhacitto, eko care khaggavisāṇakappo. %\hfill\textcolor{gray}{\footnotesize 31}

\begin{enumerate}\item 缘起为何?据说,某位波罗奈的国王在园林中为众大臣之子围绕,在石板池里嬉戏。他的厨师采集众肉之味,煮了极善烹饪、如甘露般的点心后进献。他贪图于此,未给任何人任何食物,唯自己享用。且当从水中出来时已经很晚,他便速速吃完,竟未念及先前与其共享用者。事后,他生起反省「哎!作恶在我,我为味爱所胜,忘失了众仆从,竟独自享用,噫!我要抑止味爱」,便舍弃王位,出家后修观,证得辟支菩提后,为谴责自己先前的行径,说了这显明其对治之功德的慨叹之颂。
\item 这里,\textbf{众味},即酸、甜、苦、辣、咸、碱、涩等可尝者。\textbf{不贪图},即是说不令生起渴爱。\textbf{不动摇},即不以「我要尝这、我要尝这」等纠结于特殊之味。\textbf{不养育他人},即无有需被养育者、常住者等,即是说仅需维持此身便已满足。
\item 或者,这显示:好比先前在园林中,我于众味贪图、动摇而不养育他人\footnote{不养育他人:据 PTS 本翻译,原本作「养育他人」。},现已不如是,舍弃了因之而起动摇、于众味贪图的贪已,以未来不再生起以渴爱为根本的另一自体而不养育他人\footnote{另一 \textit{añña} 与他人的巴利相同,这里的义注是解他人为「另一自体」。}。或者,以毁坏义利之义称烦恼为「他人」,这里即以不养育彼等为「不养育他人」之义。
\item \textbf{次第行乞}\footnote{次第行乞,见\textbf{清净道论}·说头陀支品第 31~34 段。},即不偏离而行、顺次而行,即不遗漏俗家的顺序,于富家、穷家连续行乞之义。\textbf{于各家不牵绊其心},即于刹帝利家等的任一处,不以烦恼而固著其心,而为明月般的常新者之义。其余仍如前述。\end{enumerate}

\subsection\*{\textbf{66}}

\textbf{舍弃了心之五盖,除去了一切随烦恼,\\}
\textbf{无依止者斩断了爱执之过,他应当独自游行,像犀牛角一样。}

Pahāya pañcāvaraṇāni cetaso, upakkilese byapanujja sabbe;\\
anissito chetva sinehadosaṃ, eko care khaggavisāṇakappo. %\hfill\textcolor{gray}{\footnotesize 32}

\begin{enumerate}\item 缘起为何?据说,在波罗奈,某位国王证得了初禅。他为了守护禅那,便舍弃王位,出家后修观,证得辟支菩提后,为显明自己行道的成就,说了这慨叹之颂。
\item 这里,「盖」的语义在蛇经(第 17 颂)已述。且因为它们如黑云障蔽日月一般而障蔽心,所以被称为\textbf{心之盖}——以近行或安止\textbf{舍弃了}彼等。\textbf{随烦恼},即跟随并迫害心的诸不善法,或如在(中部)布喻等中所说的贪等。\textbf{除去},即除去、消除、以毗婆舍那之道舍弃之义。\textbf{一切},即无余。
\item 如是,具足止观者由以初道舍断见之依止为\textbf{无依止者}。以其余诸道\textbf{斩断了}三界的\textbf{爱执之过}\footnote{爱执之过 \textit{sinehadosa}:或可译作「爱执与嗔恨」,这里为贴合义注而作此译,说详菩提比丘注 507。},即是说渴爱与贪染。因为爱执由为功德的敌对,故称为「爱执之过」。其余仍如前述。\end{enumerate}

\subsection\*{\textbf{67}}

\textbf{背离了先前的苦乐与喜忧,\\}
\textbf{获得了舍、止息、清净,他应当独自游行,像犀牛角一样。}

Vipiṭṭhikatvāna sukhaṃ dukhañ ca, pubbe va ca somanassadomanassaṃ;\\
laddhān’upekkhaṃ samathaṃ visuddhaṃ, eko care khaggavisāṇakappo. %\hfill\textcolor{gray}{\footnotesize 33}

\begin{enumerate}\item 缘起为何?据说,在波罗奈,某位国王证得了第四禅。他为了守护禅那,便舍弃王位,出家后修观,证得辟支菩提后,为显明自己行道的成就,说了这慨叹之颂。
\item 这里,\textbf{背离},即置之背后、丢弃、舍断之义。\textbf{苦乐},即身的可意、不可意。\textbf{喜忧},即心的可意、不可意。\textbf{舍},即第四禅之舍。\textbf{止息},即第四禅之止息。\textbf{清净},即由解脱于被称为五盖、寻、伺、喜、乐的九敌对法而清净,如去滓的金子般,离去随烦恼之义。
\item 其章句为:背离了先前的苦乐,指初禅近行地中的苦、第三禅近行地中的乐。再把开头所说的「与」字带到后面,即「背离了喜忧」,以「先前的」来统摄,以此显示第四禅近行中的喜与第二禅近行中的忧。这些是它们在经上的舍断处。而在离经上\footnote{经 \textit{pariyāya} 与离经 \textit{nippariyāya} 是一对经常出现的概念,菩提比丘分别英译作 expository sense 和 direct sense,见其注 508。DOP 的解释分别为「间接的、迂回的、譬喻的」和「直接的」等等。兹以 pariyāya 即旧译「方便、法门」者,为「经」的另称,而义注的意思是先按颂中的顺序解释「苦乐喜忧」的舍断,再按「离经」或阿毗达摩的顺序来解释,故作此译。},初禅是苦的舍断处,第二禅是忧的,第三禅是乐的,第四禅是喜的。如说\begin{quoting}他具足而住于初禅,在此无余地灭去已生起之苦根(相应部第 48:40 经)\end{quoting}等,这一切如于\textbf{殊胜义·法集义注}(第 165 段)中所说。因为先前于初禅等三背离了苦、忧、乐等,更于此第四禅背离了喜,以此行道获得了舍、止息、清净,他应当独自游行。其余一切则自明。\end{enumerate}

\subsection\*{\textbf{68}}

\textbf{为得第一义而勇猛精进,心不沉滞,行不怠堕,\\}
\textbf{坚持不懈,具足强力,他应当独自游行,像犀牛角一样。}

Āraddhaviriyo paramatthapattiyā, alīnacitto akusītavutti;\\
daḷhanikkamo thāmabalūpapanno, eko care khaggavisāṇakappo. %\hfill\textcolor{gray}{\footnotesize 34}

\begin{enumerate}\item 缘起为何?据说,某位边地的国王有支千人的军队,王国虽小,智慧却大。一天,他想「虽然我是小国,却能以智慧夺取整个阎浮提」,便向邻国的国王派遣使者「七日之内,要么把王国交给我,要么交战」。随后,他召集了己方的大臣说:「我未问你们便仓促行事,已如是向某某国王遣使,该当如何?」他们说:「大王!这使者还能召回吗?」「不能,他肯定去了。」「要是这样,我们可被你毁了,因为死在他人的剑下实苦,噫!让我们互相残杀而死!让我们自残而死、让我们上吊、让我们服毒!」如是,他们各各都赞叹死亡。随后,国王便说:「他们于我何有?我说,我有战士。」于是,这一千战士奋起:「大王!我是战士,大王!我是战士。」
\item 国王想「我来考察他们」,准备了火葬的柴堆,说:「我说,此事由我仓促而就,众大臣以此指责我,我将赴火,谁与我同赴?我命与谁同捐?」如是说已,五百战士便奋起:「大王!我等赴火!」随后,国王对另外五百战士说:「你们现在,亲爱的!将何所为?」他们便说:「大王!此非男子所为,而是妇人行事,况且大王已向敌王遣使,因此,我们将与大王一起战而后死。」随后,国王道「我命与尔等同捐」,部署了四军,随以这一千战士而出发,在国界驻扎。
\item 那敌王听闻了这经过,怒道「咄!这小王连做我的奴隶都不够格」,带上全部军队出去迎战。小王见他们攻来,便对军队说:「亲爱的!你们寡不敌众,全体聚拢一处,带上剑与盾,速速直冲向那国王!」他们依令而行。于是,那军队被分割成两块,露出豁口。他们便活捉了那国王,余下的士兵作鸟兽散。小王冲上来「我要杀了你」,敌王便向他求饶。随后,在赦免他后,让他起誓,便把他当作自己人,和他一起征讨另一国王,在其国界驻扎后,遣使「要么把王国交给我,要么交战」。他想「我不堪一战」,即献出王国。以此方法,他擒拿了所有国王,最后竟捉了波罗奈王。
\item 他为一百位国王围绕,治理着整个阎浮提,便想:「我先前只是小王,以一己智慧的成就成为整个阎浮提的主宰,但我这智慧只与世间之精进相应,并不转起厌离或离贪,善哉!我将以此智慧寻求出世间法!」随后,将统治交给波罗奈王,把妻儿送回自己的国家,受持出家后开始作观,证得辟支菩提后,为显明自己精进的成就,说了这慨叹之颂。
\item 这里,\textbf{勇猛精进},以此显示自己的精进勇猛、最初之精进。涅槃被称为第一义,其获得为\textbf{得第一义},以此显示精进勇猛当得之果。\textbf{心不沉滞},以此显示以强力精进支持下心、心所的不沉滞。\textbf{行不怠堕},以此(显示)于站、坐、经行等中身的不沉滞。
\item \textbf{坚持不懈},以此显示以「宁愿皮、腱」等转起的精勤、精进,举凡于次第之学等精勤者即被称为「他以身证得最高之谛,且以慧通达而见」。或者,以此显示与道相应之精进。因为由至于修习的圆满为坚持,由出离于一切敌对为不懈,所以具足此之人为坚持、不懈,故称为「坚持不懈」。
\item \textbf{具足强力},即在道的刹那具足身之强、智之力,或者,具足强的力为具足强力,即是说具足坚强的智力。以此显示这精进与观智相应,成就为如理的精勤。或者,三句当以前分、中、后之精进相连结。其余仍如前述。\end{enumerate}

\subsection\*{\textbf{69}}

\textbf{不疏忽于宴坐与禅那,于诸法始终随法行,\\}
\textbf{于诸有思惟过患,他应当独自游行,像犀牛角一样。}

Paṭisallānaṃ jhānam ariñcamāno, dhammesu niccaṃ anudhammacārī;\\
ādīnavaṃ sammasitā bhavesu, eko care khaggavisāṇakappo. %\hfill\textcolor{gray}{\footnotesize 35}

\begin{enumerate}\item 缘起为何?此颂的缘起与舍弃五盖一颂的缘起正相同,无任何差别。
\item 而在其释义上,\textbf{宴坐},即回避了彼彼有情与诸行而独自隐遁、独一,即身远离之义。\textbf{禅那},即由燃烧敌对及省虑所缘与相而被称为心远离。这里,八等至由燃烧盖等的敌对及省虑所缘而被称为禅那,观、道、果由燃烧有情想等的敌对及省虑相而被称为禅那\footnote{省虑相而被称为禅那:原文恐误,据\textbf{义释}义注及\textbf{譬喻}义注改,说详菩提比丘注 511。},但此处唯是省虑所缘的意思。如是,于此宴坐与禅那\textbf{不疏忽}、不舍、不弃。
\item \textbf{诸法},即毗婆舍那所经验之蕴等诸法。\textbf{始终},即常常、连续、不断。\textbf{随法行},即以就彼法之转起而行随行的毗婆舍那之法。或者,法即九出世间法,彼法的随顺之法为随法,即毗婆舍那的同义语。这里,应说「对诸法始终随法行」,为易于结颂,以格的转换而说成「于诸法」。
\item \textbf{于诸有思惟过患},即以此被称为随法行的观而随观三有中的无常相等之过失。如是,应说「不疏忽身远离与心远离,以被称为到达顶点之观的行道而证得」,当知如是与「他应当独自游行」相连结。\end{enumerate}

\subsection\*{\textbf{70}}

\textbf{希求着渴爱的灭尽,不放逸,聪明,多闻,具念,\\}
\textbf{已察知法,已入决定,具足精勤,他应当独自游行,像犀牛角一样。}

Taṇhakkhayaṃ patthayam appamatto, aneḷamūgo sutavā satīmā;\\
saṅkhātadhammo niyato padhānavā, eko care khaggavisāṇakappo. %\hfill\textcolor{gray}{\footnotesize 36}

\begin{enumerate}\item 缘起为何?据说,某位波罗奈王以极盛大的王家仪仗巡城。因其身体的光辉,有情的心被吸引,走在前面的便回头仰视,走在后面、两侧的也如是。因为出于天性,世间于见佛以及见满月、大海、国王无有厌足。
\item 于是,某地主之妻也上到楼阁,打开窗子,站而俯视。国王一见到她,便心被牵绊,命令大臣:「马上查查,我说,这女人有主还是无主?」他去后,便告知「有主」。于是,国王便想:「那二万舞女如天女一般,只为取悦我一人,而我现在却仍于她们不满足,对别的女人生起渴爱,这生起的只会引我至苦处。」见到渴爱的过患后,想「噫!我要抑止它」,便舍弃王位,出家后修观,证得辟支菩提后,说了这慨叹之颂。
\item 这里,\textbf{渴爱的灭尽},即涅槃,如是而不转起得见过患的渴爱。\textbf{不放逸},即常恒行、恭敬行。\textbf{聪明},即不愚笨。或者即不聋、不哑,即是说有智、堪任。其所闻能导向利益、快乐者为\textbf{多闻},即是说具足阿含。\textbf{具念},即于长久所作而能忆念。\textbf{已察知法},即以考察法而遍知法。\textbf{已入决定},即以圣道而证得决定。\textbf{具足精勤},即具足正精勤与精进。
\item 此文当以错序相连。如是,具足此不放逸等者,以导向决定的精勤而具足精勤,以此精勤,由证得决定而为已入决定,随后,由证得阿罗汉而已察知法。因为阿罗汉由无有再需察知者,故被称为「已察知法」。如说:\begin{quoting}于此,那些已察知法者,与种种有学……(经集·阿耆多学童问第 1045 颂)\end{quoting}其余仍如前述。\end{enumerate}

\subsection\*{\textbf{71}}

\textbf{好比狮子不惊怖于声响,好比清风不羁绊于罗网,\\}
\textbf{好比莲花不著于水,他应当独自游行,像犀牛角一样。}

Sīho va saddesu asantasanto, vāto va jālamhi asajjamāno;\\
padumaṃ va toyena alippamāno, eko care khaggavisāṇakappo. %\hfill\textcolor{gray}{\footnotesize 37}

\begin{enumerate}\item 缘起为何?据说,某位波罗奈王在远处有一园林。他早早起来便去往园林,中途下车,行到水边「我要洗脸」。而在此地,母狮产下幼崽后便去到领地。王臣见到便告知「陛下!有狮崽」。国王想「据说狮子什么也不怕」,便命人擂鼓以考察之。狮崽听到声响后,仍照旧躺着。国王便命人擂了三次,它在第三次时抬头看了所有会众,仍旧躺下。于是,国王说「趁它妈妈没来,我们赶紧走」,走时便想:「狮崽在出生当天就不惊怖、不怖畏,我何时也能斩断爱与见的恐惧而不惊怖、不怖畏?」
\item 他执此所缘正行时,又看到渔夫捕了鱼,在枝头结了网,而清风却毫无阻滞地穿过铺展开的网,便又执取此相:「何时我也能撕开爱与见之网或愚痴之网,如是无羁绊而行?」
\item 于是,他到达园林后,坐在石板池岸,看到莲花为风所偃,弯折触水,当风停时,又立于原处,不著于水,便又执取此相:「何时我也能像它们一样,生于水中,却不著于水而立,如是生于世间,却不著于世间而立?」他再再地想「好比狮子、清风、莲花,如是当以不惊怖、不羁绊、不著而存」,便舍弃王位,出家后修观,证得辟支菩提后,说了这慨叹之颂。
\item 这里,\textbf{狮子},有四种狮子,即草狮、黄狮、黑狮、鬃狮,其中以鬃狮为最上,此处即指它。\textbf{清风}以东方等而有多种,\textbf{莲花}则以红白等有多种,其中任何风及莲花都合适。
\item 这里,因为惊怖以对我的爱执而存,而对我的爱执即渴爱之沾染,它以与见相应或不相应的贪而存,而贪即渴爱。这里的羁绊以无考察之痴而存,而痴即无明。这里,以奢摩他舍弃渴爱,以毗婆舍那舍弃无明。所以,以奢摩他舍弃了对我的爱执,\textbf{好比狮子不惊怖于}无常等的\textbf{声响},以毗婆舍那舍弃了痴,\textbf{好比清风不羁绊于}蕴处等的\textbf{罗网},更以奢摩他舍弃了贪及与贪相应的见,\textbf{好比莲花不著于}一切有与财之贪的\textbf{水}。
\item 且此中,戒是奢摩他的足处,奢摩他即定,毗婆舍那即慧,如是当此二法成就时,三蕴亦得成就。这里,以戒蕴而调柔,他好比狮子之于声响,不为嫌恨事触怒而惊怖。以慧蕴而通达自性,好比清风之于罗网,于蕴等法不羁绊。以定蕴而离贪,好比莲花之于水,于贪染不著。如是,当知以止观及戒定慧蕴分别舍弃了无明、渴爱及三不善根而不惊怖、不羁绊、不著。其余仍如前述。\end{enumerate}

\subsection\*{\textbf{72}}

\textbf{好比狮子齿牙有力,众兽之王制胜、征服而行,\\}
\textbf{应亲近边鄙的坐卧处,他应当独自游行,像犀牛角一样。}

Sīho yathā dāṭhabalī pasayha, rājā migānaṃ abhibhuyya cārī;\\
sevetha pantāni senāsanāni, eko care khaggavisāṇakappo. %\hfill\textcolor{gray}{\footnotesize 38}

\begin{enumerate}\item 缘起为何?据说,某位波罗奈王为平息边地的动乱,舍了沿村的驿路而取林间的直道,带领大军前行。这时,在某处山麓,狮子正躺着享受初日的阳光。王臣见后,便告知国王。国王想「据说狮子不惊怖于声响」,便命人以鼓、螺、钹等作声,狮子仍旧躺着。第二次命人,狮子仍旧躺着。第三次命人,狮子想「我有敌人了」,以四足稳稳站定,站定后便作狮子吼。骑象等的人听到后,便从象上摔下,跌入草丛,成群的象马四散。
\item 国王的象也带着国王撞入密林而逃。他无法制止,挂上树枝后摔到地上,沿着一足宽的小径而行,到了众辟支佛的住处,便在此问众辟支佛:「尊者!你们也听到声响了吗?」「唯!大王!」「什么的声响?尊者!」「最开始是鼓、螺等的,后来是狮子的。」「你们不害怕吗?尊者!」「大王!我们不畏惧任何声响。」「那么,尊者!我也能做到吗?」「能!大王!只要你出家。」「我愿出家!尊者!」随后,他们便度他出家,以先前所说的方法让他修学等正行。他也以先前所说的方法修观,证得辟支菩提后,说了这慨叹之颂。
\item 这里,由忍耐、击杀及疾速而为\textbf{狮子}\footnote{忍耐、击杀及疾速:这是从语源学上解释「狮子」。},这里唯指鬃狮。\textbf{制胜、征服}二者均与\textbf{行}字相连,作「制胜而行、征服而行」,这里,以制胜、折服、清除而行为制胜而行,以征服、威吓、控制而行为征服而行。它以体力制胜而行,以光辉征服而行。这里,若有人说「制胜、征服何者而行」,则应以属格作业格,答以「制胜、征服\textbf{众兽}而行」。\textbf{边鄙},即远处。\textbf{坐卧处},即住处。其余仍如先前所述可知,不饶详繁。\end{enumerate}

\subsection\*{\textbf{73}}

\textbf{适时习行慈、舍、悲、喜解脱,\\}
\textbf{不被一切世间妨碍,他应当独自游行,像犀牛角一样。}

Mettaṃ upekkhaṃ karuṇaṃ vimuttiṃ, āsevamāno muditañ ca kāle;\\
sabbena lokena avirujjhamāno, eko care khaggavisāṇakappo. %\hfill\textcolor{gray}{\footnotesize 39}

\begin{enumerate}\item 缘起为何?据说,某位国王证得了慈等禅那。他想「王位是禅那之乐的障碍」,为了守护禅那,便舍弃王位,出家后修观,证得辟支菩提后,说了这慨叹之颂。
\item 这里,以「愿一切有情快乐」等方法欲带来利益快乐为\textbf{慈},以「哎!愿他们诚能从此苦解脱」等方法欲离去不利与苦为\textbf{悲},以「善极!诸尊喜悦,有情喜悦」等方法欲不离利益快乐为\textbf{喜},以「将以自己的业被体认」等于苦乐超然视之为\textbf{舍},为了易于结颂,以错序在说了慈后说舍,喜为最后。\textbf{解脱},亦即四者,因为它们从自身的敌对法中解脱,故为解脱。因此说「适时习行慈、舍、悲、喜解脱」。
\item 这里,\textbf{习行},即以四种禅中的三种修习三者,而以第四禅修习舍。\textbf{适时},即习行慈,从其出起后习行悲,从其出起后习行喜,从其出起,或从其它离喜的禅那出起后习行舍,故说「适时习行」,或在安乐之时习行。\textbf{不被一切世间妨碍},即不被十方一切有情世间妨碍。因为由修习慈等,众有情皆成为非违逆者,且于有情作为违逆的嗔恚止息,故说「不被一切世间妨碍」。这于此是略说,详说见\textbf{殊胜义·法集义注·说慈等}(第 251 段)。其余仍与先前所说的相同。\end{enumerate}

\subsection\*{\textbf{74}}

\textbf{舍弃了贪、嗔、痴,摆脱了结缚,\\}
\textbf{不惊怖于生命的灭尽,他应当独自游行,像犀牛角一样。}

Rāgañ ca dosañ ca pahāya mohaṃ, sandālayitvāna saṃyojanāni;\\
asantasaṃ jīvitasaṅkhayamhi, eko care khaggavisāṇakappo. %\hfill\textcolor{gray}{\footnotesize 40}

\begin{enumerate}\item 缘起为何?据说,名为摩登伽的辟支佛依王舍城而住,是所有辟支佛中的最后一位。于是,当我们的菩萨出世时,诸天为供养菩萨而来,见到他后便说:「先生!先生!佛陀已经出世。」他从灭出起,听闻了此声,并见到自己生命的灭尽,便从空中去到雪山里名为大崖的山上,那里是众辟支佛的般涅槃处,把先前入灭的辟支佛的骨聚抛下山崖,坐在石坂上,说了这慨叹之颂。
\item 这里,\textbf{贪、嗔、痴}在蛇经中已说。\textbf{结缚},即十种结缚,且已以彼彼道\textbf{摆脱}之。\textbf{不惊怖于生命的灭尽},死心的遍断被称为生命的灭尽,且于此生命的灭尽,由舍弃对生命的欣求而不惊怖。至此,已显示自己的有余依涅槃界,而在偈颂终了,他便于无余依涅槃界般涅槃。\end{enumerate}

\subsection\*{\textbf{75}}

\textbf{他们结交、亲近缘于利益,当今难得没有缘由的朋友,\\}
\textbf{不纯洁的人们谋图自利,他应当独自游行,像犀牛角一样。}

Bhajanti sevanti ca kāraṇatthā, nikkāraṇā dullabhā ajja mittā;\\
attaṭṭhapaññā asucī manussā, eko care khaggavisāṇakappo. %\hfill\textcolor{gray}{\footnotesize 41}

\begin{enumerate}\item 缘起为何?据说,在波罗奈,某位国王治理着也如初颂中所说般繁荣富裕的王国。他得了重病,苦受转起。二万女子围绕,按摩手足等处。大臣们想「这国王现在活不成了,噫!我们各寻归处吧」,去到另一国王跟前,请求侍奉。但他们虽在那里侍奉,却毫无所得。国王病瘥后,便问:「某某人、某某人在哪里?」随后,听闻了这经过,便摇头沉默。
\item 那些大臣也听闻了「国王病瘥」,在那里又毫无所得,陷入极度贫困,便重新回去,礼拜国王后,站在一边。当被国王问到「亲爱的!你们哪里去了」时,便说:「看到陛下羸弱,我们害怕生计,便去到某某国土。」国王摇头,便想:「我何不考察他们是否会再这么行事?」
\item 他便如先前患病般,表现出自己疼痛剧烈,假装生病。女子们也如先前一般周匝围绕,忙着一切。那些大臣仍又离开,带走更多的人。如是,国王三次一切照旧行事,他们仍旧离开。随后,见到他们第四次回来时,生起嫌厌「哎!这些人为难为之事,舍弃生病的我,无所关切而离开」,便舍弃王位,出家后修观,证得辟支菩提后,说了这慨叹之颂。
\item 这里,\textbf{结交},即以身接近而承事。\textbf{亲近},即以行合掌等及以乐意侍奉而敬事。缘由即他们的利益,为\textbf{缘于利益},即是说结交、亲近非有别的缘由,唯利益是他们的缘由,因利益而亲近。\textbf{当今难得没有缘由的朋友},即无「我们从中能得什么」等获取利益的缘由,仅具足以\begin{quoting}助益之友,苦乐中友,\\示利之友,同情之友(长部·尸迦罗经第 265 段)\end{quoting}等所说的圣友之性的朋友,当今为难得。他们的智慧停留于自己,唯关照自己,而非他人,为\textbf{谋图自利}。此处古本也作「谋图现利」,即是说他们的智慧唯于现今所见的利益,觉察不到未来。\textbf{不纯洁},即具足不纯洁、非圣的身语意业。其余当知仍如前述。
\item 如是,这四十一颂之量的犀牛角经,当以唯于某处所说的连结之法于一切处随适地连结已而知其后续与语义。为免过繁,我们未于一切处连结。\end{enumerate}

\begin{center}\vspace{1em}犀牛角经第三\\Khaggavisāṇasuttaṃ tatiyaṃ.\end{center}

%\begin{flushright}癸卯十一月初四二稿\\甲辰耶诞三稿\end{flushright}