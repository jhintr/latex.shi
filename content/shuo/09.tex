\chapter{品藻第九}

\subsection*{1}

\textbf{汝南陳仲舉、潁川李元禮二人共論其功德,不能定先後,蔡伯喈}{\footnotesize \textbf{續漢書}曰蔡伯喈,陳留圉人,通達有儁才,博學善屬文,伎藝術數,無不精綜,仕至左中郎將,為王允所誅。}\textbf{評之曰:「陳仲舉彊於犯上,李元禮嚴於攝下,犯上難,攝下易。」}{\footnotesize \textbf{張璠漢紀}曰時人為之語曰「不畏彊禦陳仲舉,天下模楷李元禮」。}\textbf{仲舉遂在三君之下,}{\footnotesize \textbf{謝沈漢書}曰三君者,一時之所貴也,竇武、劉淑、陳蕃,少有高操,海內尊而稱之,故得因以為目,}\textbf{元禮居八俊之上。}{\footnotesize \textbf{薛瑩漢書}曰李膺、王暢、荀緄、朱㝢、魏朗、劉佑、杜楷、趙典為八俊。\textbf{英雄記}曰先是,張儉等相與作衣冠糺彈,彈中人相調,言「我彈中誠有八俊八乂,猶古之八元八凱也」。\textbf{謝沈書}曰俊者,卓出之名也。\textbf{姚信士緯}曰陳仲舉體氣高烈,有王臣之節,李元禮忠壯正直,有社稷之能,海內論之未決,蔡伯喈抑一言以變之,疑論乃定也。}

\subsection*{2}

\textbf{龐士元至吳,吳人並友之,}{\footnotesize \textbf{蜀志}曰周瑜領南郡,士元為功曹,瑜卒,士元送喪至吳,吳人多聞其名,及當還西,並會閶門與士元言。}\textbf{見陸績、}{\footnotesize \textbf{文士傳}曰績,字公紀,幼有儁朗才數,博學多通,龐士元年長於績,共為交友,仕至鬱林太守,自知亡日,年三十二而卒。}\textbf{顧劭、全琮,}{\footnotesize \textbf{環濟吳紀}曰琮,字子黃,吳郡錢塘人,有德行義概,為大司馬。}\textbf{而為之目曰:「陸子所謂駑馬有逸足之用,顧子所謂駑牛可以負重致遠。」或問:「如所目,陸為勝邪?」曰:「駑馬雖精速,能致一人耳,駑牛一日行百里,所致豈一人哉?」吳人無以難,「全子好聲名,似汝南樊子昭。」}{\footnotesize \textbf{蔣濟萬機論}曰許子將褒貶不平,以拔樊子昭而抑許文休,劉曄難曰「子昭拔自賈豎,年至七十,退能守靜,進不苟競」,濟答曰「子昭誠自幼至長,容貌完潔,然觀其插齒牙、樹頰頦、吐脣吻,自非文休之敵」。}

\subsection*{3}

\textbf{顧劭嘗與龐士元宿語,問曰:「聞子名知人,吾與足下孰愈?」曰:「陶冶世俗,與時浮沈,吾不如子,}{\footnotesize \textbf{吳志}曰劭好樂人倫,自州郡庶幾及四方人事,往來相見,或諷議而去,或結友而別,風聲流聞,遠近稱之。}\textbf{論王霸之餘策,覽倚伏之要害,吾似有一日之長。」劭亦安其言。}{\footnotesize \textbf{吳錄}曰劭安其言,更親之。}

\subsection*{4}

\textbf{諸葛瑾、弟亮及從弟誕,}{\footnotesize \textbf{吳書}曰瑾,字子瑜,其先葛氏,琅邪諸縣人,後徙陽都,陽都先有姓葛者,時人謂之諸葛,因以為氏,瑾少以至孝稱,累遷豫州牧,六十八卒。\textbf{魏志}曰誕,字公休,為吏部郎,人有所屬託,輒顯其言而亟用之,後有得失當不,則公議其得失,以為褒貶,自是群寮莫不慎其所舉,累遷揚州刺史、鎮東將軍、司空,謀逆伏誅。}\textbf{並有盛名,各在一國,于時以為「蜀得其龍,吳得其虎,魏得其狗」,誕在魏與夏侯玄齊名,瑾在吳,吳朝服其弘量。}{\footnotesize \textbf{吳書}曰瑾避亂渡江,大皇帝取為長史,遣使蜀,但與弟亮公會相見,反無私面,而又有容貌思度,時人服其弘量。}

\subsection*{5}

\textbf{司馬文王問武陔:「陳玄伯何如其父司空?」陔曰:「通雅博暢,能以天下聲教為己任者,不如也,明練簡至,立功立事,過之。」}{\footnotesize \textbf{魏志}曰陔與泰善,故文王問之。}

\subsection*{6}

\textbf{正始中,人士比論,以五荀方五陳,荀淑方陳寔,荀靖方陳諶,}{\footnotesize \textbf{逸士傳}曰靖,字叔慈,潁川人,有儁才,以孝著名,兄弟八人,號八龍,隱身修學,動止合禮,弟爽,亦有才學,顯名當世,或問汝南許章「爽與靖孰賢」,章曰「二人皆玉也,慈明外朗,叔慈內潤」,太尉辟,不就,年五十終,時人惜之,號玄行先生。}\textbf{荀爽方陳紀,荀彧方陳群,}{\footnotesize \textbf{典略}曰彧,字文若,潁川人,為漢侍中、守尚書令,彧為人英偉,折節待士,坐不累席,其在臺閣間,不以私欲撓意,年五十薨,諡曰敬侯,以其德高,追贈太尉。}\textbf{荀顗方陳泰,}{\footnotesize \textbf{晉諸公贊}曰顗,字景倩,彧之子,蹈禮立德,思義溫雅,加深識國體,累遷光祿大夫,晉受禪,封臨淮公,典朝儀,刊正國式,為一代之制,轉太尉,為台輔,德望清重,留心禮教,卒,諡康公。}\textbf{又以八裴方八王,裴徽方王祥,裴楷方王夷甫,裴康方王綏,}{\footnotesize \textbf{晉百官名}曰康,字仲豫,徽之子。\textbf{晉諸公贊}曰康有弘量,歷太子左率。}\textbf{裴綽方王澄,}{\footnotesize \textbf{王朝目錄}曰綽,字仲舒,楷弟也,名亞於楷,歷中書、黃門侍郎。}\textbf{裴瓚方王敦,}{\footnotesize \textbf{晉諸公贊}曰瓚,字國寶,楷之子,才氣爽儁,終中書郎。}\textbf{裴遐方王導,裴頠方王戎,裴邈方王玄。}

\subsection*{7}

\textbf{冀州刺史楊準二子喬與髦,俱總角為成器,準與裴頠、樂廣友善,遣見之,頠性弘方,愛喬之有高韻,謂準曰:「喬當及卿,髦小減也。」廣性清湻,愛髦之有神檢,謂準曰:「喬自及卿,然髦尤精出。」準笑曰:「我二兒之優劣,乃裴、樂之優劣。」論者評之,以為喬雖高韻而檢不匝,樂言為得,然並為後出之儁。}{\footnotesize \textbf{荀綽冀州記}曰喬,字國彥,爽朗有遠意,髦,字士彥,清平有貴識,並為後出之儁,為裴頠、樂廣所重。\textbf{晉諸公贊}曰喬似準而疎,皆為二千石,髦為石勒所害。}

\subsection*{8}

\textbf{劉令言始入洛,}{\footnotesize \textbf{劉氏譜}曰訥,字令言,彭城叢亭人,祖瑾,樂安長,父甝,魏洛陽令,訥歷司隸校尉。}\textbf{見諸名士而歎曰:「王夷甫太解明,樂彥輔我所敬,張茂先我所不解,周弘武巧於用短,}{\footnotesize \textbf{王隱晉書}曰周恢,字弘武,汝南人,祖斐,永寧少府,父隆,州從事,恢仕至秦相,秩中二千石。}\textbf{杜方叔拙於用長。」}{\footnotesize \textbf{晉諸公贊}曰杜育,字方叔,襄城鄧陵人,杜襲孫也,育幼便岐嶷,號神童,及長,美風姿,有才藻,時人號曰杜聖,累遷國子祭酒,洛陽將沒,為賊所殺。}

\subsection*{9}

\textbf{王夷甫云:「閭丘沖,}{\footnotesize \textbf{荀綽兗州記}曰沖,字賓卿,高平人,家世二千石,沖清平有鑒識,博學有文義,累遷太傅長史,雖不能立功蓋世,然聞義不惑,當世蒞事,務於平允,操持文案,必引經誥,飾以文采,未嘗有滯,性尤通達,不矜不假,好音樂,侍婢在側,不釋弦管,出入乘四望車,居之甚夷,不以虧損恭素之行,淡然肆其心志,論者不以為侈、不以為僭,至於白首而清名令望,不渝於始,為光祿勳,京邑未潰,乘車出,為賊所害,時人皆痛惜之。}\textbf{優於滿奮、郝隆,}{\footnotesize \textbf{晉諸公贊}曰隆,字弘始,高平人,為人通亮清識,為吏部郎、楊州刺史,齊王冏起義,隆應檄稽留,為參軍王邃所殺。}\textbf{此三人並是高才,沖最先達。」}{\footnotesize \textbf{兗州記}曰于時高平人士偶盛,滿奮、郝隆達在沖前,名位已顯,而劉寶、王夷甫猶以沖之虛貴,足先二人。}

\subsection*{10}

\textbf{王夷甫以王東海比樂令,}{\footnotesize \textbf{江左名士傳}曰承言理辯物,但明其旨要,不為辭費,有識伏其約而能通,太尉王夷甫一世龍門,見而雅重之,以比南陽樂廣。}\textbf{故王中郎作碑云:「當時標榜,為樂廣之儷。」}

\subsection*{11}

\textbf{庾中郎與王平子鴈行。}{\footnotesize \textbf{晉陽秋}曰初,王澄有通朗稱,而輕薄無行,兄夷甫有盛名,時人許以人倫鑒識,常為天下士目曰「阿平第一,子嵩第二,處仲第三」,敳以澄、敦莫己若也,及澄喪、敦敗,敳世譽如初。}

\subsection*{12}

\textbf{王大將軍在西朝時,見周侯輒扇障面不得住,}{\footnotesize 敦性彊梁,自少及長,季倫斬妓,曾無異色,若斯傲狠,豈憚於周顗乎?其言不然也。}\textbf{後度江左,不能復爾,王歎曰:「不知我進,伯仁退?」}{\footnotesize \textbf{沈約晉書}曰周顗,王敦素憚之,見輒面熱,雖復臘月亦扇面不休,其憚如此。}

\subsection*{13}

\textbf{會稽虞騑,元皇時與桓宣武同俠,其人有才理勝望,}{\footnotesize \textbf{虞光祿傳}曰騑,字思行,會稽餘姚人,虞翻曾孫,右光祿潭兄子也,雖機幹不及潭,而至行過之,歷吏部郎、吳興守,徵為金紫光祿大夫,卒。}\textbf{王丞相嘗謂騑曰:「孔愉有公才而無公望,丁潭有公望而無公才,}{\footnotesize 愉已見。\textbf{會稽後賢記}曰潭,字世康,山陰人,吳司徒固曾孫也,沈婉有雅望,少與孔愉齊名,仕至光祿大夫。\textbf{晉陽秋}曰孔敬康、丁世康、張偉康俱著名,時謂會稽三康,偉康名茂,嘗夢得大象,以問萬雅,雅曰「君當為大郡而不善也,象,大獸也,取其音狩,故為大郡,然象以齒喪身」,後為吳郡,果為沈充所殺。}\textbf{兼之者其在卿乎?」騑未達而喪。}{\footnotesize \textbf{虞光祿傳}曰騑未登台鼎,時論稱屈。}

\subsection*{14}

\textbf{明帝問周伯仁:「卿自謂何如郗鑒?」周曰:「鑒方臣,如有功夫。」復問郗,郗曰:「周顗比臣,有國士門風。」}{\footnotesize \textbf{鄧粲晉紀}曰伯仁清正嶷然,以德望稱之。}

\subsection*{15}

\textbf{王大將軍下,庾公問:「卿有四友,何者是?」答曰:「君家中郎,我家太尉、阿平,胡毋彥國,}{\footnotesize \textbf{八王故事}曰胡毋輔之少有雅俗鑒識,與王澄、庾敳、王敦、王夷甫為四友。今故答也。}\textbf{阿平故當最劣。」庾曰:「似未肯劣。」庾又問:「何者居其右?」王曰:「自有人。」又問:「何者是?」王曰:「噫!其自有公論。」左右躡公,公乃止。}{\footnotesize 敦自謂右者在己也。}

\subsection*{16}

\textbf{人問丞相:「周侯何如和嶠?」答曰:「長輿嵯櫱。」}{\footnotesize \textbf{虞預晉書}曰嶠厚自封植,嶷然不群。}

\subsection*{17}

\textbf{明帝問謝鯤:「君自謂何如庾亮?」答曰:「端委廟堂,使百僚準則,臣不如亮,一丘一壑,自謂過之。」}{\footnotesize \textbf{晉陽秋}曰鯤隨王敦下,入朝,見太子於東宮,語及夕,太子從容問鯤曰「論者以君方庾亮,自謂孰愈」,對曰「宗廟之美,百官之富,臣不如亮,縱意丘壑,自謂過之」。\textbf{鄧粲晉紀}曰鯤與王澄之徒慕竹林諸人,散首披髮,裸袒箕踞,謂之八達,故鄰家之女折其兩齒,世為謠曰「任達不已,幼輿折齒」,鯤有勝情遠概,為朝廷之望,故時以庾亮方焉。}

\subsection*{18}

\textbf{王丞相二弟不過江,曰潁、曰敞,時論以潁比鄧伯道、敞比溫忠武,議郎、祭酒者也。}{\footnotesize \textbf{王氏譜}曰潁,字茂英,位至議郎,年二十卒。敞,字茂平,丞相祭酒,不就,襲爵堂邑公,年二十有二而卒。}

\subsection*{19}

\textbf{明帝問周侯:「論者以卿比郗鑒,云何?」周曰:「陛下不須牽顗比。」}{\footnotesize \textbf{按}顗死彌年,明帝乃即位,世說此言妄矣。}

\subsection*{20}

\textbf{王丞相云:「頃下論以我比安期、千里,亦推此二人,唯共推太尉,此君特秀。」}{\footnotesize \textbf{晉諸公贊}曰夷甫性矜峻,少為同志所推。}

\subsection*{21}

\textbf{宋禕曾為王大將軍妾,後屬謝鎮西,鎮西問禕:「我何如王?」答曰:「王比使君,田舍、貴人耳。」鎮西妖冶故也。}{\footnotesize 未詳宋禕。}

\subsection*{22}

\textbf{明帝問周伯仁:「卿自謂何如庾元規?」對曰:「蕭條方外,亮不如臣,從容廊廟,臣不如亮。」}{\footnotesize \textbf{按}諸書皆以謝鯤比亮,不聞周顗。}

\subsection*{23}

\textbf{王丞相辟王藍田為掾,庾公問丞相:「藍田何似?」王曰:「真獨簡貴,不減父祖,然曠澹處故當不如爾。」}{\footnotesize 王述狷隘故也。}

\subsection*{24}

\textbf{卞望之云:「郗公體中有三反,方於事上,好下佞己,一反,治身清貞,大脩計校,二反,自好讀書,憎人學問,三反。」}{\footnotesize \textbf{按}太尉劉寔論王肅,方於事上,好下佞己,性嗜榮貴,不求苟合,治身不穢,尤惜財物,王、郗志性儻亦同乎?}

\subsection*{25}

\textbf{世論溫太真是過江第二流之高者,時名輩共說人物,第一將盡之間,溫常失色。}{\footnotesize \textbf{溫氏譜序}曰晉大夫郤至封於溫,子孫因氏,居太原祁縣,為郡著姓。}

\subsection*{26}

\textbf{王丞相云:「見謝仁祖恆令人得上,與何次道語,唯舉手指地曰『正自爾馨』。」}{\footnotesize 前篇及諸書皆云王公重何充,謂必代己相,而此章以手指地,意如輕詆,或清言析理,何不逮謝故邪?}

\subsection*{27}

\textbf{何次道為宰相,人有譏其信任不得其人,}{\footnotesize \textbf{晉陽秋}曰充所暱庸雜,以此損名。}\textbf{阮思曠慨然曰:「次道自不至此,但布衣超居宰相之位,可恨唯此一條而已。」}{\footnotesize \textbf{語林}曰阮光祿聞何次道為宰相,歎曰「我當何處生活」。此則阮未許何為鼎輔,二說便相符也。}

\subsection*{28}

\textbf{王右軍少時,丞相云:「逸少何緣復減萬安邪?」}{\footnotesize 劉綏已見。}

\subsection*{29}

\textbf{郗司空家有傖奴,知及文章,事事有意,王右軍向劉尹稱之,劉問:「何如方回?」}{\footnotesize \textbf{郗愔別傳}曰愔,字方回,高平金鄉人,太宰鑒長子也,淵靖純素,無執無競,簡私暱,罕交遊,歷會稽內史、侍中、司徒。}\textbf{王曰:「此正小人有意向耳,何得便比方回?」劉曰:「若不如方回,故是常奴耳。」}

\subsection*{30}

\textbf{時人道阮思曠:「骨氣不及右軍,簡秀不如真長,韶潤不如仲祖,思致不如淵源,而兼有諸人之美。」}{\footnotesize \textbf{中興書}曰裕以人不須廣學,正應以禮讓為先,故終日頹然,無所修綜,而物自宗之。}

\subsection*{31}

\textbf{簡文云:「何平叔巧累於理,嵇叔夜儁傷其道。」}{\footnotesize 理本真率,巧則乖其致,道唯虛澹,儁則違其宗,所以二子不免也。}

\subsection*{32}

\textbf{時人共論晉武帝出齊王之與立惠帝,其失孰多,}{\footnotesize \textbf{晉陽秋}曰齊王攸,字大猷,文帝第二子,孝敬忠肅,清和平允,親賢下士,仁惠好施,能屬文,善尺牘,初,荀勖、馮紞為武帝親幸,攸惡勖之佞,勖懼攸或嗣立,必誅己,且攸甚得眾心,朝賢景附,會帝有疾,攸及皇太子入問訊,朝士皆屬目於攸,而不在太子,至是,勖從容曰「陛下萬年後,太子不得立也」,帝曰「何故」,勖曰「百寮內外皆歸心於齊王,太子安得立乎?陛下試詔齊王歸國,必舉朝謂之不可,若然,則臣言徵矣」,侍中馮紞又曰「陛下必欲建諸侯,成五等,宜從親始,親莫若齊王」,帝從之,於是下詔,使攸之國,攸聞勖、紞間己,憂忿不知所為,入辭,出,嘔血薨,帝哭之慟,馮紞侍曰「齊王名過其實,而天下歸之,今自薨殞,陛下何哀之甚」,帝乃止,劉毅聞之,故終身稱疾焉。}\textbf{多謂立惠帝為重,桓溫曰:「不然,使子繼父業,弟承家祀,有何不可?」}{\footnotesize 武帝兆禍亂、覆神州,在斯而已,輿隸且知其若此,況宣武之弘儁乎?此言非也。}

\subsection*{33}

\textbf{人問殷淵源:「當世王公以卿比裴叔道,云何?」殷曰:「故當以識通暗處。」}{\footnotesize 遐與浩並能清言。}

\subsection*{34}

\textbf{撫軍問殷浩:「卿定何如裴逸民?」良久答曰:「故當勝耳。」}

\subsection*{35}

\textbf{桓公少與殷侯齊名,常有競心,桓問殷:「卿何如我?」殷云:「我與我周旋久,寧作我。」}

\subsection*{36}

\textbf{撫軍問孫興公:「劉真長何如?」曰:「清蔚簡令。」「王仲祖何如?」曰:「溫潤恬和。」}{\footnotesize \textbf{徐廣晉紀}曰凡稱風流者,皆舉王、劉為宗焉。}\textbf{「桓溫何如?」曰:「高爽邁出。」「謝仁祖何如?」曰:「清易令達。」「阮思曠何如?」曰:「弘潤通長。」「袁羊何如?」曰:「洮洮清便。」「殷洪遠何如?」曰:「遠有致思。」「卿自謂何如?」曰:「下官才能所經,悉不如諸賢,至於斟酌時宜,籠罩當世,亦多所不及,然以不才,時復託懷玄勝,遠詠老莊,蕭條高寄,不與時務經懷,自謂此心無所與讓也。」}

\subsection*{37}

\textbf{桓大司馬下都,問真長曰:「聞會稽王語奇進,爾邪?」}{\footnotesize \textbf{桓溫別傳}曰興寧九年,以溫克復舊京,肅靜華夏,進都督中外諸軍事、侍中、大司馬,加黃鉞,使入參朝政。}\textbf{劉曰:「極進,然故是第二流中人耳。」桓曰:「第一流復是誰?」劉曰:「正是我輩耳。」}

\subsection*{38}

\textbf{殷侯既廢,桓公語諸人曰:「少時與淵源共騎竹馬,我棄去,已輒取之,故當出我下。」}{\footnotesize \textbf{續晉陽秋}曰簡文輔政,引殷浩為揚州,欲以抗桓,桓素輕浩,未之憚也。}

\subsection*{39}

\textbf{人問撫軍:「殷浩談竟何如?」答曰:「不能勝人,差可獻酬群心。」}

\subsection*{40}

\textbf{簡文云:「謝安南清令不如其弟,}{\footnotesize 安南,謝奉也,已見。\textbf{謝氏譜}曰奉弟聘,字弘遠,歷侍中、廷尉卿。}\textbf{學義不及孔巖,}{\footnotesize \textbf{中興書}曰巖,字彭祖,會稽山陰人,父倫,黃門侍郎,巖有才學,歷丹陽尹、尚書、西陽侯,在朝多所匡正,為吳興太守,大得民和,後卒於家。}\textbf{居然自勝。」}{\footnotesize 言奉任天真也。}

\subsection*{41}

\textbf{未廢海西公時,王元琳問桓元子:「箕子、比干迹異心同,不審明公孰是孰非?」曰:「仁稱不異,寧為管仲。」}{\footnotesize \textbf{論語}曰微子去之,箕子為之奴,比干諫而死,子曰「殷有三仁焉」。子路曰「桓公殺公子糾,召忽死之,管仲不死,曰未仁乎」,子曰「桓公九合諸侯,一匡天下,不以兵車,管仲之力,如其仁,如其仁」。}

\subsection*{42}

\textbf{劉丹陽、王長史在瓦官寺集,桓護軍亦在坐,}{\footnotesize 桓伊已見。}\textbf{共商略西朝及江左人物,或問:「杜弘治何如衛虎?」桓答曰:「弘治膚清,衛虎奕奕神令。」王、劉善其言。}{\footnotesize 虎,衛玠小字。\textbf{玠別傳}曰永和中,劉真長、謝仁祖共商略中朝人,或問「杜弘治可方衛洗馬不」,謝曰「安得比,其間可容數人」。\textbf{江左名士傳}曰劉真長曰「吾請評之,弘治膚清,叔寶神清」,論者謂為知言。}

\subsection*{43}

\textbf{劉尹撫王長史背曰:「阿奴比丞相,但有都長。」}{\footnotesize 阿奴,濛小字也。都,美也。\textbf{司馬相如傳}曰閑雅甚都。\textbf{語林}曰劉真長與丞相不相得,每曰「阿奴比丞相,條達清長」。}

\subsection*{44}

\textbf{劉尹、王長史同坐,長史酒酣起舞,劉尹曰:「阿奴今日不復減向子期。」}{\footnotesize 類秀之任率也。}

\subsection*{45}

\textbf{桓公問孔西陽:「安石何如仲文?」}{\footnotesize 西陽,即孔巖也。}\textbf{孔思未對,反問公曰:「何如?」答曰:「安石居然不可陵踐,其處故勝也。」}

\subsection*{46}

\textbf{謝公與時賢共賞說,遏、胡兒並在坐,公問李弘度曰:「卿家平陽,何如樂令?」}{\footnotesize \textbf{晉諸公贊}曰李重,字茂曾,江夏鍾武人,少以清尚見稱,歷吏部郎、平陽太守。}\textbf{於是李潸然流涕曰:「趙王篡逆,樂令親授璽綬,}{\footnotesize \textbf{晉陽秋}曰趙王倫篡位,樂廣與滿奮、崔隨進璽綬。}\textbf{亡伯雅正,恥處亂朝,遂至仰藥,恐難以相比,此自顯於事實,非私親之言。」}{\footnotesize \textbf{晉諸公贊}曰趙王為相國,取重為左司馬,重以倫將篡,辭疾不就,敦喻之,重不復自治,至於篤甚,扶曳受拜,數日卒,時人惜之,贈散騎常侍。}\textbf{謝公語胡兒曰:「有識者果不異人意。」}

\subsection*{47}

\textbf{王脩齡問王長史:「我家臨川,何如卿家宛陵?」長史未答,脩齡曰:「臨川譽貴。」長史曰:「宛陵未為不貴。」}{\footnotesize \textbf{中興書}曰羲之自會稽王友改授臨川太守,王述從驃騎功曹出為宛陵令,述之為宛陵,多脩為家之具,初有勞苦之聲,丞相王導使人謂之曰「名父之子,屈臨小縣,甚不宜爾」,述答曰「足自當止」,時人未之達也,後屢臨州郡,無所造作,世始歎服之。}

\subsection*{48}

\textbf{劉尹至王長史許清言,時苟子年十三,倚牀邊聽,既去,問父曰:「劉尹語何如尊?」長史曰:「韶音令辭,不如我,往輒破的,勝我。」}{\footnotesize \textbf{劉惔別傳}曰惔有儁才,其談詠虛勝,理會所歸,王濛略同,而敘致過之,其詞當也。}

\subsection*{49}

\textbf{謝萬壽春敗後,簡文問郗超:「萬自可敗,那得乃爾失士卒情?」超曰:「伊以率任之性,欲區別智勇。」}{\footnotesize \textbf{中興書}曰萬之為豫州,氐羌暴掠司豫,鮮卑屯結并冀,萬既受方任,自率眾入潁,以援洛陽,萬矜豪傲物,失士眾之和,北中郎郗曇以疾還彭城,萬以為賊盛致退,便回還南,遂自潰亂,狼狽單歸,太宗責之,廢為庶人。}

\subsection*{50}

\textbf{劉尹謂謝仁祖曰:「自吾有四友,門人加親。」謂許玄度曰:「自吾有由,惡言不及於耳。」二人皆受而不恨。}{\footnotesize \textbf{尚書大傳}曰孔子曰「文王有四友,自吾得回也,門人加親,是非胥附邪?自吾得賜也,遠方之士至,是非奔走邪?自吾得師也,前有輝,後有光,是非先後邪?自吾得由也,惡言不入於耳,是非禦侮邪」。}

\subsection*{51}

\textbf{世目殷中軍:「思緯淹通,比羊叔子。」}{\footnotesize 羊祜德高一世,才經夷險,淵源蒸燭之曜,豈喻日月之明也。}

\subsection*{52}

\textbf{有人問謝安石、王坦之優劣於桓公,桓公停欲言,中悔,曰:「卿喜傳人語,不能復語卿。」}

\subsection*{53}

\textbf{王中郎嘗問劉長沙曰:「我何如苟子?」}{\footnotesize \textbf{大司馬官屬名}曰劉奭,字文時,彭城人。\textbf{劉氏譜}曰奭祖昶,彭城內史,父濟,臨海令,奭歷車騎咨議、長沙相、散騎常侍。}\textbf{劉答曰:「卿才乃當不勝苟子,然會名處多。」王笑曰:「癡。」}

\subsection*{54}

\textbf{支道林問孫興公:「君何如許掾?」孫曰:「高情遠致,弟子蚤已服膺,一吟一詠,許將北面。」}

\subsection*{55}

\textbf{王右軍問許玄度:「卿自言何如安、萬?」許未答,王因曰:「安石故相為雄,阿萬當裂眼爭邪?」}{\footnotesize \textbf{中興書}曰萬器量不及安石,雖居藩任,安在私門之時,名稱居萬上也。}

\subsection*{56}

\textbf{劉尹云:「人言江虨田舍,江乃自田宅屯。」}{\footnotesize 謂能多出有也。}

\subsection*{57}

\textbf{謝公云:「金谷中蘇紹最勝。」紹是石崇姊夫、蘇則孫、愉子也。}{\footnotesize \textbf{石崇金谷詩敘}曰余以元康六年,從太僕卿出為使持節、監青徐諸軍事、征虜將軍,有別廬在河南縣界金谷澗中,或高或下,有清泉茂林,眾果竹柏、藥草之屬,莫不畢備,又有水碓、魚池、土窟,其為娛目歡心之物備矣,時征西大將軍祭酒王詡當還長安,余與眾賢共送往澗中,晝夜遊宴,屢遷其坐,或登高臨下,或列坐水濱,時琴瑟笙筑,合載車中,道路並作,及住,令與鼓吹遞奏,遂各賦詩,以敘中懷,或不能者,罰酒三斗,感性命之不永,懼凋落之無期,故具列時人官號、姓名、年紀,又寫詩著後,後之好事者,其覽之哉!凡三十人,吳王師、議郎、關中侯始平武功蘇紹,字世嗣,年五十,為首。\textbf{魏書}曰蘇則,字文師,扶風武功人,剛直疾惡,常慕汲黯之為人,仕至侍中、河東相。\textbf{晉百官名}曰愉,字休豫,則次子。\textbf{山濤啓事}曰愉忠義有智意,位至光祿大夫。}

\subsection*{58}

\textbf{劉尹目庾中郎:「雖言不愔愔似道,突兀差可以擬道。」}{\footnotesize \textbf{名士傳}曰敳頹然淵放,莫有動其聽者。}

\subsection*{59}

\textbf{孫承公云:「謝公清於無奕,}{\footnotesize \textbf{中興書}曰孫統,字承公,太原人,善屬文,時人謂其有祖楚風,仕至餘姚令。}\textbf{潤於林道。」}{\footnotesize \textbf{陳逵別傳}曰逵,字林道,潁川許昌人,祖淮,太尉,父畛,光祿大夫,逵少有幹,以清敏立名,襲封廣陵公、黃門郎、西中郎將,領梁、淮南二郡太守。}

\subsection*{60}

\textbf{或問林公:「司州何如二謝?」林公曰:「故當攀安提萬。」}{\footnotesize \textbf{王胡之別傳}曰胡之好談諧,善屬文辭,為當世所重。}

\subsection*{61}

\textbf{孫興公、許玄度皆一時名流,或重許高情,則鄙孫穢行,或愛孫才藻,而無取於許。}{\footnotesize \textbf{宋明帝文章志}曰綽博涉經史,長於屬文,與許詢俱有負俗之談,詢卒不降志,而綽嬰綸世務焉。\textbf{續晉陽秋}曰綽雖有文才,而誕縱多穢行,時人鄙之。}

\subsection*{62}

\textbf{郗嘉賓道謝公:「造厀雖不深徹,而纏綿綸至。」又曰:「右軍詣嘉賓。」嘉賓聞之云:「不得稱詣,政得謂之朋耳。」謝公以嘉賓言為得。}{\footnotesize 凡徹、詣者,蓋深覈之名也,謝不徹,王亦不詣,謝、王於理相與為朋儔也。}

\subsection*{63}

\textbf{庾道季云:「思理倫和,吾愧康伯,志力彊正,吾愧文度,自此以還,吾皆百之。」}{\footnotesize 庾龢已見。}

\subsection*{64}

\textbf{王僧恩輕林公,藍田曰:「勿學汝兄,汝兄自不如伊。」}{\footnotesize 僧恩,王禕之小字也。\textbf{王氏世家}曰禕之,字文劭,述次子,少知名,尚尋陽公主,仕至中書郎,未三十而卒,坦之悼念,與桓溫稱之,贈散騎常侍。}

\subsection*{65}

\textbf{簡文問孫興公:「袁羊何似?」答曰:「不知者不負其才,知之者無取其體。」}{\footnotesize 言其有才而無德也。}

\subsection*{66}

\textbf{蔡叔子云:「韓康伯雖無骨幹,然亦膚立。」}

\subsection*{67}

\textbf{郗嘉賓問謝太傅曰:「林公談何如嵇公?」謝云:「嵇公勤著腳,裁可得去耳。」}{\footnotesize \textbf{支遁傳}曰遁神悟機發,風期所得,自然超邁也。}\textbf{又問:「殷何如支?」謝曰:「正爾有超拔,支乃過殷,然亹亹論辯,恐殷欲制支。」}

\subsection*{68}

\textbf{庾道季云:「廉頗、藺相如雖千載上死人,懍懍恆如有生氣,}{\footnotesize \textbf{史記}曰廉頗者,趙良將也,以勇氣聞諸侯,藺相如者,趙人也,趙惠文王時,得楚和氏璧,秦昭王請以十五城易之,趙遣相如送璧,秦受之,無還城意,相如請璧示其瑕,因持璧卻立倚柱,怒髮上衝冠,曰「王欲急臣,臣頭今與璧俱碎」,秦王謝之,後秦王使趙王鼓瑟,相如請秦王擊筑,趙以相如功大,拜上卿,位在廉頗上。}\textbf{曹蜍、}{\footnotesize 蜍,曹茂之小字也。\textbf{曹氏譜}曰茂之,字永世,彭城人也,祖韶,鎮東將軍司馬,父曼,少府卿,茂之仕至尚書郎。}\textbf{李志}{\footnotesize \textbf{晉百官名}曰志,字溫祖,江夏鍾武人。\textbf{李氏譜}曰志祖重,散騎常侍,父慕,純陽令,志仕至員外常侍、南康相。}\textbf{雖見在,厭厭如九泉下人,人皆如此,便可結繩而治,但恐狐狸猯狢噉盡。」}{\footnotesize 言人皆如曹、李質魯湻慤,則天下無姦民,可結繩致治,然才智無聞,功迹俱滅,身盡於狐狸,無擅世之名也。}

\subsection*{69}

\textbf{衛君長是蕭祖周婦兄,謝公問孫僧奴:}{\footnotesize 僧奴,孫騰小字也。\textbf{晉百官名}曰騰,字伯海,太原人。\textbf{中興書}曰騰,紞子也,博學,歷中庶子、廷尉。}\textbf{「君家道衛君長云何?」孫曰:「云是世業人。」謝曰:「殊不爾,衛自是理義人。」于時以比殷洪遠。}

\subsection*{70}

\textbf{王子敬問謝公:「林公何如庾公?」謝殊不受,答曰:「先輩初無論,庾公自足沒林公。」}{\footnotesize \textbf{殷羡言行}曰時有人稱庾太尉理者,羡曰「此公好舉宗本槌人」。}

\subsection*{71}

\textbf{謝遏諸人共道竹林優劣,謝公云:「先輩初不臧貶七賢。」}{\footnotesize \textbf{魏氏春秋}曰山濤通簡有德,秀、咸、戎、伶朗達有儁才,於時之談,以阮為首,王戎次之,山、向之徒,皆其倫也。若如盛言,則非無臧貶,此言謬也。}

\subsection*{72}

\textbf{有人以王中郎比車騎,車騎聞之曰:「伊窟窟成就。」}{\footnotesize \textbf{續晉陽秋}曰坦之雅貴有識量,風格峻整。}

\subsection*{73}

\textbf{謝太傅謂王孝伯:「劉尹亦奇自知,然不言勝長史。」}

\subsection*{74}

\textbf{王黃門兄弟三人俱詣謝公,子猷、子重多說俗事,}{\footnotesize \textbf{王氏譜}曰操之,字子重,羲之第六子,歷祕書監、侍中、尚書、豫章太守。}\textbf{子敬寒溫而已,既出,坐客問謝公:「向三賢孰愈?」謝公曰:「小者最勝。」客曰:「何以知之?」謝公曰:「吉人之辭寡,躁人之辭多,推此知之。」}

\subsection*{75}

\textbf{謝公問王子敬:「君書何如君家尊?」答曰:「固當不同。」公曰:「外人論殊不爾。」王曰:「外人那得知?」}{\footnotesize \textbf{宋明帝文章志}曰獻之善隸書,變右軍法為今體,字畫秀媚,妙絕時倫,與父俱得名,其章草疎弱,殊不及父,或訊獻之云「羲之書勝不」,「莫能判」,有問羲之云「世論卿書不逮獻之」,答曰「殊不爾也」,它日見獻之,問「尊君書何如」,獻之不答,又問「論者云,君固當不如」,獻之笑而答曰「人那得知之也」。}

\subsection*{76}

\textbf{王孝伯問謝太傅:「林公何如長史?」太傅曰:「長史韶興。」問:「何如劉尹?」謝曰:「噫!劉尹秀。」王曰:「若如公言,並不如此二人邪?」謝云:「身意正爾也。」}

\subsection*{77}

\textbf{人有問太傅:「子敬可是先輩誰比?」謝曰:「阿敬近撮王、劉之標。」}{\footnotesize \textbf{續晉陽秋}曰獻之文義並非所長,而能撮其勝會,故擅名一時,為風流之冠也。}

\subsection*{78}

\textbf{謝公語孝伯:「君祖比劉尹,故為得逮。」孝伯云:「劉尹非不能逮,直不逮。」}{\footnotesize 言濛質而惔文也。}

\subsection*{79}

\textbf{袁彥伯為吏部郎,子敬與郗嘉賓書曰:「彥伯已入,殊足頓興往之氣,故知捶撻自難為人,冀小卻,當復差耳。」}

\subsection*{80}

\textbf{王子猷、子敬兄弟共賞高士傳人及贊,子敬賞「井丹高潔」,子猷云:「未若長卿慢世。」}{\footnotesize \textbf{嵇康高士傳}曰丹,字大春,扶風郿人,博學高論,京師為之語曰「五經紛綸井大春,未嘗書刺謁一人」,北宮五王更請,莫能致,新陽侯陰就使人要之,不得已而行,侯設麥飯、蔥菜,以觀其意,丹推卻曰「以君侯能供美膳,故來相過,何謂如此」,乃出盛饌,侯起,左右進輦,丹笑曰「聞桀紂駕人車,此所謂人車者邪」,侯即去輦,越騎梁松,貴震朝廷,請交丹,丹不肯見,後丹得時疾,松自將醫視之,病愈,久之,松失大男磊,丹一往弔之,時賓客滿廷,丹裘褐不完,入門,坐者皆悚,望其顏色,丹四向長揖,前與松語,客主禮畢後,長揖徑坐,莫得與語,不肯為吏,徑出,後遂隱遁,其贊曰「井丹高潔,不慕榮貴,抗節五王,不交非類,顯譏輦車,左右失氣,披褐長揖,義陵群萃」。司馬相如者,蜀郡成都人,字長卿,初為郎,事景帝,梁孝王來朝,從遊說士鄒陽等,相如說之,因病免遊梁,後過臨邛,富人卓王孫女文君新寡,好音,相如以琴心挑之,文君奔之,俱歸成都,後居貧,至臨邛買酒舍,文君當壚,相如著犢鼻褌,滌器市中,為人口吃,善屬文,仕宦不慕高爵,常託疾不與公卿大事,終於家,其贊曰「長卿慢世,越禮自放,犢鼻居市,不恥其狀,託疾避官,蔑此卿相,乃賦大人,超然莫尚」。}

\subsection*{81}

\textbf{有人問袁侍中}{\footnotesize \textbf{袁氏譜}曰恪之,字元祖,陳郡陽夏人,祖王孫,司徒從事中郎,父綸,臨汝令,恪之仕黃門侍郎,義熙初為侍中。}\textbf{曰:「殷仲堪何如韓康伯?」答曰:「理義所得,優劣乃復未辨,然門庭蕭寂,居然有名士風流,殷不及韓。」故殷作誄云:「荊門晝掩,閑庭晏然。」}

\subsection*{82}

\textbf{王子敬問謝公:「嘉賓何如道季?」答曰:「道季誠復鈔撮清悟,嘉賓故自上。」}{\footnotesize 謂超拔也。}

\subsection*{83}

\textbf{王珣疾,臨困,問王武岡曰:}{\footnotesize \textbf{中興書}曰謐,字雅遠,丞相導孫,車騎劭子,有才器,襲爵武岡侯,位至司徒。}\textbf{「世論以我家領軍比誰?」武岡曰:「世以比王北中郎。」東亭轉臥向壁,歎曰:「人固不可以無年。」}{\footnotesize 領軍王洽,珣之父也,年二十六卒,珣意以其父名德過坦之而無年,故致此論。}

\subsection*{84}

\textbf{王孝伯道謝公「濃至」,又曰:「長史虛,劉尹秀,謝公融。」}{\footnotesize 謂條暢也。}

\subsection*{85}

\textbf{王孝伯問謝公:「林公何如右軍?」謝曰:「右軍勝林公,林公在司州前亦貴徹。」}{\footnotesize 不言若羲之,而言勝胡之。}

\subsection*{86}

\textbf{桓玄為太傅,大會,朝臣畢集,坐裁竟,問王楨之曰:「我何如卿第七叔?」}{\footnotesize \textbf{王氏譜}曰楨之,字公幹,琅邪人,徽之子,歷侍中、大司馬長史。第七叔,獻之也。}\textbf{于時賓客為之咽氣,王徐徐答曰:「亡叔是一時之標,公是千載之英。」一坐懽然。}

\subsection*{87}

\textbf{桓玄問劉太常曰:「我何如謝太傅?」}{\footnotesize \textbf{劉瑾集敘}曰瑾,字仲璋,南陽人,祖遐,父暢,暢娶王羲之女,生瑾,瑾有才力,歷尚書、太常卿。}\textbf{劉答曰:「公高,太傅深。」又曰:「何如賢舅子敬?」答曰:「樝梨橘柚,各有其美。」}{\footnotesize \textbf{莊子}曰樝梨橘柚,其味相反,皆可於口也。}

\subsection*{88}

\textbf{舊以桓謙比殷仲文,}{\footnotesize \textbf{中興書}曰謙,字敬祖,沖第三子,尚書僕射、中軍將軍。\textbf{晉安帝紀}曰仲文有器貌才思。}\textbf{桓玄時,仲文入,桓於庭中望見之,謂同坐曰:「我家中軍那得及此也。」}