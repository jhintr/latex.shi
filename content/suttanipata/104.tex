\section{耕田婆罗豆婆遮经}

\begin{center}Kasibhāradvāja Sutta\end{center}\vspace{1em}

\textbf{如是我闻\footnote{此经旧译见杂阿含经第 98 经、别译杂阿含经第 264 经。耕田婆罗豆婆遮、一那罗等译名皆从杂阿含经。}。一时世尊住摩竭陀南山的一那罗婆罗门村。}

Evaṃ me sutaṃ— ekaṃ samayaṃ Bhagavā Magadhesu viharati Dakkhiṇāgirismiṃ Ekanāḷāyaṃ brāhmaṇagāme.

\begin{enumerate}\item 缘起为何?世尊住摩竭陀南山的一那罗婆罗门村时,于饭前义务、饭后义务等二种佛陀的义务中完成了饭前义务,在饭后义务的终了,以佛眼观察世间,便见到耕田婆罗豆婆遮\footnote{婆罗豆婆遮 \textit{Bhārad-vāja} 是古印度大仙 Bharadvāja 的后裔,该大仙及其弟子们被认为是梨俱吠陀第六卷的作者,而在史诗摩诃婆罗多中,大仙是德罗纳 \textit{Droṇa} 的父亲,后者则是俱卢族和般度族的老师。}婆罗门具足阿罗汉的近依,且了知到「当我到达那里,将发生谈话,随后,在谈话终了,听闻了法的开示,这婆罗门将出家而圆满阿罗汉」,便去到那里,引起谈话,说了此经。
\item 这里,若问「什么是诸佛的饭前义务、什么是饭后义务」?当答:佛世尊清早起来后,为摄受侍者及身体安泰,作好洗脸等的洗漱,直至行乞之时,于远离的坐处等待,到了行乞之时,便著下衣、扎腰带、披覆上衣、持钵,时而独自,时而为比丘僧团围绕,进入村镇行乞,有时自然前去,有时则现起种种神变。
\item 例如,在世间的依怙进入行乞时,柔风在正前方开道,扫净地面。雨云洒下水珠,让路上的尘埃落定,在上方形成华盖。又一阵风收集了花朵,撒到路上。隆起的地面下沉,而下沉者隆起,使落足之时地面平整,或有乐触且大如车轮的莲花承受双足。当右足踏入因陀柱\footnote{因陀柱:即村口的标志。}内,六色光芒从身上放出,使楼阁、尖顶等有如金黄的色泽,又如被彩布包裹,处处散射。象马飞鸟等于各自的所在发出甜美的声音,鼓、琵琶等乐器以及人们随身的璎珞也同样,人们以此征兆了知「今天世尊来此乞食」。
\item 他们整理衣裳,持了香、花等,从家出来,行到路边,以香、花等恭敬地供养世尊,顶礼后请求道「尊者!给我们十个比丘,二十个、一百个比丘」,乃至取了世尊的钵,设好座位,以食物恭敬地敬候。世尊食事已毕,观察了他们的相续,开示以相应的法,以使有些人住于皈依,有些于五戒,有些于须陀洹、斯陀含、阿那含中的任一果,有些出家已于最上的阿罗汉果。如是,如此摄受众人后,便从坐起,去往寺庙。
\item 于此,他坐在圆亭中设好的最上佛座,等待诸比丘食事终了。随后,当诸比丘食事终了,侍者便报告世尊。于是,世尊便进入香房。以上即\textbf{饭前义务}。且此中未述及者,当依(中部)梵摩经中所说的把握。
\item 于是,世尊如是完成了饭前义务,坐于香房的前厅,清洗了双足,置于足凳,教诫比丘僧团「诸比丘!应以不放逸而成就,在世间佛陀出世难得、获得人身难得、信之成就难得\footnote{信之成就难得:PTS 本作「时机之成就难得」。}、出家难得、听闻善法难得」。随后,众比丘礼拜了世尊,问取业处。于是,世尊依众比丘的性行授予业处。他们取了业处后,礼敬了世尊,去到各自的住处,有些到林野,有些到树下,有些到山中等的某处,有些到四大王天的居处……有些到自在天的居处。随后,世尊进入香房,倘若需要,便具念正知,以右胁作片刻狮子卧。
\item 于是,恢复体力起来后,他在(饭后的)第二分中观察世间。第三分中,在他所依于居住的村镇,人们在饭前作了布施后,在饭后整理衣裳,持了香、花等,在寺庙集合。随后,世尊以适当的神变去到聚集的会众中,坐在法堂设好的最上佛座,开示适时、适量的法。然后,知时已到,即遣散会众。
\item 随后,若欲擦拭肢体,即从佛座起,去到侍者准备好水的地方,从侍者手中接过水浴衣,进入浴室。侍者则取了佛座,在香房的隔间设好。世尊擦拭完肢体,著了善加染色的双层下衣,扎了腰带,穿了上衣,便回到那里坐下,独自宴坐片刻。然后,众比丘从各处赶来给侍世尊。于此,有些人提出问题,有些请求业处,有些请求闻法。世尊遂其所愿,即度过初夜。
\item 在中夜,整个一万世界的天人获得机会,去到世尊处,提出如其准备、乃至四音节\footnote{四音节:即半句。}的问题。世尊解答这些天人的问题,即度过中夜。
\item 随后,分后夜为四分,以一分决意于经行,在第二分进入香房,具念正知,以右胁作狮子卧,在第三分以果定度过,在第四分进入大悲等至,为见少尘、多尘等的有情,以佛眼观察世间。此即\textbf{饭后义务}。
\item 如是,在这饭后义务之被称为观察世间的第四分的终了,在为得见于佛法僧及施、戒、布萨、羯磨等未作或已作增上行、未具足或已具足近依的有情而以佛眼观察世间时,便见到耕田婆罗豆婆遮婆罗门具足阿罗汉的近依,了知到「当我到达那里,将发生谈话,随后,在谈话终了,听闻了法的开示,这婆罗门将出家而圆满阿罗汉」,便去到那里,引起谈话,说了此经。
\item 这里,「如是我闻」等由尊者阿难在第一次大结集中,当在作法的结集时,被尊者大迦叶长老问及而对五百阿罗汉所说,「沙门!我耕作、播种」由耕田婆罗豆婆遮所说,「婆罗门!我也耕作、播种」等由世尊所说,这一切汇集后,即被称为\textbf{耕田婆罗豆婆遮经}。
\item 这里,\textbf{如是},此如是之词即行相、显示、强调之义。因为以此行相之义,即显明此义:彼世尊的话语具种种理法之微妙,等起多种意乐,具足义文\footnote{文 \textit{vyañjana}:与义 \textit{attha} 相对而言,旧译作「味」,详见\textbf{施罗经}义注。},去除种种(敌对)\footnote{去除 \textit{pāṭihāriya}:通常译作「神变」,如\textbf{长部}说有变化神变、记说神变、教诫神变等三种神变,菩提比丘据长部疏等,依字面作「去除敌对」解,详见其注 538。},于法、义、开示、通达等甚深,以各自的语言随适于一切有情,自性可辨,谁能以一切行相了知此?故说「如是我闻,我唯以一种行相得闻」。以显示之义,即以「我非自成者,此非由我证」为己开脱,显示现在当说的全部经文为「如是我闻,我如是得闻」。以强调之义,即以\begin{quoting}诸比丘!我的声闻比丘中,多闻之最上,即此阿难,智慧\footnote{智慧 \textit{gati} 一词多义,详见\textbf{施罗经}义注。}、忆念、坚定、给侍之最上,即此阿难。(增支部第 1:219~223 经)\end{quoting}等显明随适于世尊所赞叹的自身的忆持力,令有情愿欲听闻,故说「如是我闻,且其于义、于文不增不减,唯当如是而见,而非其它」。\textbf{我闻},此中的我字为由我之义,闻字为耳门识之义,所以「如是我闻」即是说,如是由我以耳识为先导的识路所理解。
\item \textbf{世尊},即是说吉祥者、破坏者、应敬者\footnote{对「世尊」的解释详见\textbf{清净道论}·说六随念品第 53 段及以下,这里的译文略有不同。}。\textbf{住摩竭陀},摩竭陀本指国中的众王子,他们的住处乃至一国也被俗称作「摩揭陀」,即于此摩揭陀国。然而,有人以\begin{quoting}因为支提王\footnote{支提王事,见\textbf{本生}第 8:45~59 颂。}说了妄语,正进入地下时,被告知「莫进深处 \textit{mā gadhaṃ pavisa}」,或者因为人们搜寻这国王而掘地时,被告知「莫挖深处 \textit{mā gadhaṃ karotha}」,所以叫摩揭陀。\end{quoting}等多种方式解释,可随喜好把握。
\item \textbf{住},即是说以另一威仪打断一威仪的限制,持运、转起自体,使不失坠。或者,以天、梵、圣之住为有情持运种种利益为住。持运,即是说收摄、提供、令生、生起。因为如此,当有情于爱欲邪行道时,世尊便以天住而住,令其生起不贪善根,「或许见到这行道,对此生起好乐,他们能于爱欲离染」。当他们为了统治,对有情邪行道时,便以梵住而住,令其生起不嗔善根,「或许见到这行道,对此生起好乐,他们能以无嗔止息嗔恨」。当出家众诤论于法,便以圣住而住,令其生起不痴善根,「或许见到这行道,对此生起好乐,他们能以无痴止息愚痴」。然而,无时不以威仪住而住,因为舍此便无有自体之维持。这于此只是略说,我们将在吉祥经注中详说。
\item \textbf{南山},有山环绕王舍城而立,在其南侧的国土被称为「南山」,即是说于此国土。这里同时也是寺庙的名称。\textbf{一那罗婆罗门村},一那罗是这村名,且于此有众多婆罗门居住,或此是婆罗门的财产,所以被称为「婆罗门村」。\end{enumerate}

\textbf{尔时,耕田婆罗豆婆遮婆罗门的五百张犁已经上轭,正值播种时节。}

Tena kho pana samayena Kasibhāradvājassa brāhmaṇassa pañcamattāni naṅgalasatāni payuttāni honti vappakāle.

\begin{enumerate}\item \textbf{尔时},即是说世尊在结不可战胜之跏趺坐、证无上正等正觉已,转起无上法轮,于摩揭陀国依一那罗婆罗门村、住南山大寺,期待着婆罗门的根成熟之时。而其中的 \textbf{kho pana} 是一对不变词,用于补足语句,或者可被视作用于显示新的场景。\textbf{耕田婆罗豆婆遮婆罗门},即此婆罗门以耕田为生,且婆罗豆婆遮是他的族姓,所以如是得称。\textbf{已经上轭},即已连结,把轭置于耕牛的背上,以轭带连结好之义。
\item \textbf{播种时节},即是说抛撒种子的时节。这里,有两种播种:播于泥中、播于土中,这里指的是播于土中,且其为第一天的仪式性播种。这里,其器具的成就有:三千头供驱使的耕牛,全都系上金制的角、银制的蹄,全都饰以具一切香中妙香且五指宽的白色花鬘,肢体健全,具一切相,有些漆色般黑,有些水晶色白,有些珊瑚色红,有些杂色如琥珀。五百耕夫全都著了崭新的白下衣,饰以花鬘,右肩套着花圈,闪耀着雌黄、雄黄的印记,每十张犁成一组而行。犁头、轭与刺棒都贴了金。八头耕牛套在头一张犁上,四头四头套在其余上,剩余的随行,以供疲累者轮换。每组有一车种子,一人耕作,一人播种。
\item 婆罗门一早就剃了须,洗了澡,涂了妙香,著了价值五百的下衣,把价值一千(的衣)偏覆一肩,每指各二,戴了二十个戒指,双耳戴了狮耳环,头上缠了梵头巾,项上戴了金花鬘,为众婆罗门围绕,指挥作业。他的婆罗门尼命人煮了数百锅粥,再命人装上大车,以香水沐浴,以一切庄严装饰,为众婆罗门尼围绕,去往农场。他的家也处处涂了妙香,以鲜花善作供奉,田地也处处竖了幢幡。到访农场的会众与仆从、农工一起有二千五百人,全都著了崭新的白下衣,且粥食已为全体作了准备。
\item 于是,婆罗门在自己的饮食之处,让人洗了金钵,装满了粥,调和了酥、蜜、糖等,命人去供奉犁。婆罗门尼则取了金、银、青铜、黄铜制成的盘子,拿了金勺,为坐着的五百耕夫施粥。婆罗门在命人作了供奉后,穿了赤金的鞋子,拿了赤金的权杖,周行指挥「这里给粥,这里给酥、砂糖」。\end{enumerate}

\textbf{于是,世尊晨朝著了下衣,持了衣钵,便往耕田婆罗豆婆遮婆罗门的农场走去。尔时,耕田婆罗豆婆遮婆罗门的食物分发正在进行。于是,世尊便往食物分发处走去,走到后,站在一边。}

Atha kho Bhagavā pubbaṇhasamayaṃ nivāsetvā pattacīvaram ādāya yena Kasibhāradvājassa brāhmaṇassa kammanto ten’upasaṅkami. Tena kho pana samayena Kasibhāradvājassa brāhmaṇassa parivesanā vattati. Atha kho Bhagavā yena parivesanā ten’upasaṅkami, upasaṅkamitvā ekamantaṃ aṭṭhāsi.

\begin{enumerate}\item 于是,世尊正坐在香房,了知到婆罗门的食物分发正在进行,想「这是调御婆罗门的时机」,便著下衣、扎腰带、披覆僧伽梨、持钵后出了香房,如无上调御丈夫一般,因此尊者阿难说「于是,世尊晨朝著了下衣」。
\item 这里,\textbf{于是},即发起另一话题之语的不变词。\textbf{Kho} 补足语句。\textbf{世尊}已如前述。\textbf{晨朝},即一天的前半天,意为在晨朝,或者上午的某时为晨朝,即是说在上午的一时,如是在延续的连接处可用业格。\textbf{著了下衣},即围了下衣,当知这是替换寺庙内的下衣,而非世尊此前未著下衣。\textbf{持了衣钵},即以手持钵、以身持衣,领受、保持之义。据说,世尊欲行乞食时,他的蓝宝石色的石钵来至一对双手之间,如蜜蜂来至一对盛开的莲花之间般。所以,当知其意即以双手领受如是来至之钵,且以身受持齐整披覆之衣。或者,以某某方式把握即为「持」,如「受持而行」。
\item 那为什么诸比丘不随从世尊呢?当答:当世尊想要独自去往某处,便在行乞之时关上房门,进入香房内。随后,诸比丘以此标记了知「今天世尊想独自入村,他肯定见到某个可调伏之人」,他们便拿了自己的衣钵,右绕香房后,前去行乞。此时,世尊便是这么做的,所以诸比丘未随从世尊。
\item \textbf{尔时},即世尊去往农场之时,这\textbf{婆罗门的食物分发正在进行},即食物分配正在进行之义。我们先前曾说「婆罗门尼则取了金、银、青铜、黄铜制成的盘子,拿了金勺,为坐着的五百耕夫施粥」。
\item \textbf{于是,世尊便往食物分发处走去}。为什么缘由?为了摄受婆罗门的缘由。因为世尊不会像穷人一样,为了食物而往食物分发处走去。因为世尊有八万二千之数的释迦、拘利王族亲戚,他们为了自身的成就都热衷于施以固定食,但世尊并非为了食物而出家,而以「遍舍五大遍舍\footnote{五大遍舍:据\textbf{长部}义注等,为遍舍肢体、眼睛、财产、王位、妻儿,菩提比丘注据\textbf{如是语}义注,作肢体、自体、财产、妻子、王位。},在圆满了数阿僧祇的波罗蜜后,已解脱,欲令解脱,已调御,欲令调御,已寂静,欲令寂静,已般涅槃,欲令般涅槃」而出家,所以,当知由自己的解脱……般涅槃及令他人解脱……般涅槃而周行世间,为摄受婆罗门的缘由,往食物分发处走去。
\item \textbf{走到后,站在一边},「一边」是表示状态的中性词,即是说一处、一侧。或者是在依格的意义上使用业格:在其可见的近处、可听闻交谈之处,站立于婆罗门可见的高地上。且站立后,便放出金黄色泽的身光,光芒超过一千个日月,量及周围八十肘,由为其覆盖,婆罗门的农场、会堂、墙壁、树木、翻起的土块等都成了像金子打造的一般。\end{enumerate}

\textbf{耕田婆罗豆婆遮婆罗门看到世尊站着乞食。看到后,对世尊说:「沙门!我耕作、播种,耕作、播种后我才吃饭,沙门!你也该(为自己)耕作、播种,耕作、播种后你才该吃饭。」「婆罗门!我也耕作、播种,耕作、播种后我才吃饭。」}

Addasā kho Kasibhāradvājo brāhmaṇo Bhagavantaṃ piṇḍāya ṭhitaṃ. Disvāna Bhagavantaṃ etad avoca: “ahaṃ kho, samaṇa, kasāmi ca vapāmi ca, kasitvā ca vapitvā ca bhuñjāmi, tvam pi, samaṇa, kasassu ca vapassu ca, kasitvā ca vapitvā ca bhuñjassū” ti. “Aham pi kho, brāhmaṇa, kasāmi ca vapāmi ca, kasitvā ca vapitvā ca bhuñjāmī” ti.

\begin{enumerate}\item 于是,喝粥的人们看到站在一边的正等正觉者,身俱伴随八十随行好的三十二胜相,双臂饰以环绕一寻的光明,光鬘视之辉煌祥瑞,如流动的莲池,如天空光网闪耀的星团,如以光芒照亮金山之巅的太阳,便洗了手脚后合掌,环绕而立。\textbf{耕田婆罗豆婆遮婆罗门看到}如是为众人环绕的\textbf{世尊站着乞食}。\textbf{看到后,对世尊说:「沙门!我耕作、播种……」}
\item 那他为什么要这样说?是因为对一切善见、应生净喜、证得最上调御与止息的世尊没有净喜,还是即便已为二千五百人准备了粥,却吝于一匙之食?两者皆非,而是见到人们观看世尊不厌足而丢下工作后,心生不满「前来打断我的工作」,所以如是说。
\item 且在见到世尊相的成就后,便想「如果他能从事农务,将会成为整个阎浮提中人们顶上的宝髻摩尼一般,还有什么义利不能成就?却这般这般出于懒惰,不从事农务,于播种等仪式中游行乞食,享用而身体强健,周行四方」,因此说「沙门!我耕作、播种,耕作、播种后我才吃饭」,意即不要妨碍我的农务,且我不如你一般具足相的意思。\textbf{沙门!你也该(为自己)耕作、播种,耕作、播种后你才该吃饭},意即具足如是之相者,还有什么义利不能成就的意思。
\item 且他曾听闻:据说,童子出生在释迦王族中,他舍弃了转轮王位而出家。所以,了知到「眼下这位就是」,便作非难「舍弃了转轮王位,你疲于奔命」而说「沙门!我……」。
\item 且这婆罗门黠慧,并未辱没世尊而说,而是见到世尊容貌的成就后,敬重慧的成就,为了引起谈话而如是说「沙门!我……」。
\item 随后,世尊以可调伏的方式,为显明自己在俱有天的世间为最上的耕作者、播种者的身份,而说「\textbf{婆罗门!我也……}」。\end{enumerate}

\textbf{「但我们没看见乔达摩君的轭、犁、犁铧、刺棒或是耕牛,然而乔达摩君却如是说『婆罗门!我也耕作、播种,耕作、播种后我才吃饭』。」于是,耕田婆罗豆婆遮婆罗门以偈颂对世尊说:}

“Na kho pana mayaṃ passāma bhoto Gotamassa yugaṃ vā naṅgalaṃ vā phālaṃ vā pācanaṃ vā balībadde vā, atha ca pana bhavaṃ Gotamo evam āha: ‘aham pi kho, brāhmaṇa, kasāmi ca vapāmi ca, kasitvā ca vapitvā ca bhuñjāmī’” ti. Atha kho Kasibhāradvājo brāhmaṇo Bhagavantaṃ gāthāya ajjhabhāsi:

\begin{enumerate}\item 于是,婆罗门便生起思惟:「这沙门说『我耕作、播种』,我却没有看见他的轭、犁等耕田的器具,他是否妄语?」当对世尊从脚掌开始直到发端彻底打量时,由于对相术曾作研究,便了知到他三十二胜相的成就,想「像这样的人妄语绝不可能、不可得」,对世尊生出敬意,舍弃了「沙门」的称谓,而以族姓体面地对待世尊,说「\textbf{但我们没看见乔达摩君的……}」。
\item 且如是说已,黠慧的婆罗门了知「这是就甚深之义而说」,欲知此义,便提问而以偈颂对世尊说。因此,尊者阿难说「\textbf{于是,耕田婆罗豆婆遮婆罗门以偈颂对世尊说}」。这里,\textbf{偈颂},即音节与句式限定的言语。\end{enumerate}

\subsection\*{\textbf{76} \textcolor{gray}{\footnotesize 〔76〕}}

\textbf{「您自称是耕者,但我们却没看见您的耕作,\\}
\textbf{「既然问到,请对我们说说耕作,好让我们知道您的耕作。」}

“Kassako paṭijānāsi, na ca passāma te kasiṃ;\\
kasiṃ no pucchito brūhi, yathā jānemu te kasiṃ”. %\hfill\textcolor{gray}{\footnotesize 1}

\subsection\*{\textbf{77} \textcolor{gray}{\footnotesize 〔77〕}}

\textbf{「信是种子,苦行是雨水,慧是我的轭与犁,\\}
\textbf{「惭是辕,意是轭带,念是我的犁铧与刺棒。}

“Saddhā bījaṃ tapo vuṭṭhi, paññā me yuganaṅgalaṃ;\\
hirī īsā mano yottaṃ, sati me phālapācanaṃ. %\hfill\textcolor{gray}{\footnotesize 2}

\begin{enumerate}\item 这里,婆罗门以轭、犁等耕作用具的组合来说「耕作」,然而,因为揭示与前法相似者的谈论为诸佛的威力,所以世尊为显明诸佛的威力、揭示与前法相似者,而说「信是种子」。那此中什么是与前法相似者?难道世尊不是被婆罗门问到轭、犁等耕作用具的组合,但却去揭示未被问及之种子的相似者而说「信是种子」,以致此谈论成不适用?
\item 当答:诸佛的谈论无有不适用者,诸佛也不会不揭示与前法相似者而谈论。且此中的适用当知如是:因为世尊被此婆罗门以轭、犁等耕作用具问及耕作,出于对他的怜悯,并不以「这未被问」而忽略之,而是为令了知连同根本、连同资助、连同用具、连同果报的耕作,从根本开始显明耕作而说「信是种子」。因为种子是耕作的根本,有此方可为,无则不可为,且应依其量而为。因为当有种子,人们方耕作,无则不为,且善巧的耕者依种子的量而耕田,以「莫让我们的收成减损」而无减,以「莫让我们白白努力」而无增。且因为唯有种子是根本,所以,世尊为从根本开始显明耕作,揭示自己耕作的前法与此婆罗门耕作前法的种子相似,而说「信是种子」。此中的与前法相似当知如是。
\item 若问「为什么不在说了被问的之后,再说未被问的」?出于对他的资助以及出于连接法的可能。因为此婆罗门有慧,但由生于邪见之家故无信。且无信有慧者由对他人起信,于自身的境域未行道而未证殊胜。且即便其弱力的信具有离去烦恼污浊的净喜之量,当与强力之慧一起转起时,也不得成就义利,好比与象连于同一轭上的公牛一般,所以对他而言,信是资助。如是出于对此婆罗门的资助,以令此婆罗门建立起信,由开示的善巧,先说原本应后说的,如别处的「信束结行资」、「信是人的伴侣」、「信是人在此世最上的财富」、「以信度过暴流」、「大象以信为鼻」、「诸比丘!圣弟子以信为支柱」等。且种子的资助是雨水,故能在其后即被说及。如是出于连接法的可能,先说原本应后说的,且其它如辕、轭带等也如是。
\item 这里,\textbf{信}以净喜为相,或以信任为相,以跃入为味,以信解为现起,或以不污浊为现起,以预流支为足处,或以可信之法为足处,心净喜的状态如镜子、水面等的澄净一般,使诸相应法明净,如净水摩尼(明净)水一般。\textbf{种子}有五种:根种子、茎种子、节种子、插种子,而种子种子为第五,一切皆以生长之义得称,如说「这以生长之义为种子」。
\item 这里,好比在婆罗门的耕田中,作为根本的种子起两种作用,在下建立根基,在上令发嫩芽,如是在世尊的耕田中,作为根本的信在下建立戒的根基,在上令发止观的嫩芽。又好比这(种子)以根摄取了地味、水味,以茎增长,以期获得谷物的成熟,如是这(信)以戒之根摄取了止观之味,以圣道之茎增长,以期获得圣果之谷物的成熟。又好比这(种子)在善地中建立后,以根、芽、叶、茎、干、籽等增长、生长、广大,令生汁液,结出许多稻粒充盈的稻穗,如是这(信)在心相续中建立后,以戒、心、见、度疑、道非道智见、行道智见清净等增长、生长、广大,令生智见清净的汁液,结出许多无碍解、神通充盈的阿罗汉果。因此世尊说「信是种子」。
\item 这里,若问「在五十多种一起生起的善法中,为什么只说信是种子」?当答:由执行种子的作用之故。因为好比在彼等中,唯有识执行了别的作用,如是,信执行种子的作用,且其为一切善之根本。如说:\begin{quoting}起信者前往,前往者承事,承事者侧耳,侧耳者闻法,闻法已忆持,考察所忆持诸法之义,考察其义者认可诸法,当认可法时,生起欲,生欲者热衷,热衷已衡量,衡量已精勤,精勤者以身证得最胜谛,且以慧通达而得见之。(中部·翅吒山经第 183 段)\end{quoting}
\item 消磨诸不善法及身体为\textbf{苦行},即根律仪、精进、头陀支、难为之事等的同义语,但这里指的是根律仪的意思。\textbf{雨水},即以雨季之雨、风雨等有多种,这里指的是雨季之雨。因为好比对于婆罗门,为雨水照料的种子及以种子为根本的谷物生长、不凋、趋于成熟,如是对于世尊,为根律仪照料的信及以信为根本的戒等法生长、不凋、趋于成熟,因此说「苦行是雨水」。
\item 且「慧是我的」中所说的「我的」一词,亦应用于此句中,作「信是我的种子,苦行是我的雨水」。以此显明什么?好比,婆罗门!你所播的种子,若有雨水则善哉,若无,则应灌溉以水,同样,我将惭之辕、慧之轭与犁以意之轭带捆缚一处,连结于精进之耕牛,以念之刺棒击打,于自身心相续之田所播的信之种子则无缺雨之虞,这恒常、持续的苦行即是我的雨水。
\item 人以此了知,或由自身了知为\textbf{慧},它依欲界等类而有多种,而这里指的是与毗婆舍那俱的道慧。\textbf{轭与犁},因为好比婆罗门有轭与犁,如是世尊也有两种慧。这里,好比轭是辕的支撑,为其前导,与辕绑定,是轭带的所依,保持众耕牛同行,如是慧是以惭为首的诸法的支撑,如说:\begin{quoting}慧是一切善法之最上。(增支部第 8:83 经)\\他们说慧是最胜的善,如群星中的星王。(本生第 17:81 经)\end{quoting}且以诸善法的先导之义为其前导,如说:\begin{quoting}戒、惭以及善法,为慧的随从。(本生第 17:81 经)\end{quoting}由与惭不相应则不生起,故与辕绑定,由作为被称为意的三摩地之轭带的依止缘,故是轭带的所依,由防止过度努力与过度懈怠,保持精进之众耕牛同行。且好比与犁铧结合的犁,在耕田时破碎了地的密集,断绝了根的相续,如是与念结合的慧,在修观时破碎了诸法相续、集合、作用、所缘的密集,断绝了一切烦恼之根的相续。且它唯是出世间的,但其它世间的兴许也是。因此说「慧是我的轭与犁」。
\item 人以此而惭,或由自身而惭,嫌厌于不善之转起为\textbf{惭}。由与其俱行,愧也被其包摄在内。\textbf{辕}是约束轭与犁的木棒。因为好比婆罗门的辕约束着轭与犁,如是世尊的惭约束着被称为世出世间慧之轭与犁,当无惭时,慧即不存。又好比与辕绑定的轭与犁在工作中不动摇、不松懈,如是与惭绑定的慧在工作中不动摇、不松懈,不掺杂以无惭。因此说「惭是辕」。
\item 觉知为\textbf{意},即心的同义语,但这里指的是以意为首、与其相应的三摩地。\textbf{轭带},即捆缚的绳子。这有三种:轭与辕的捆缚,耕牛与轭的捆缚,耕牛与御者的捆缚。这里,好比婆罗门的轭带将辕、轭、耕牛捆缚一处,使其发挥各自的作用,如是世尊的三摩地将这一切惭、慧、精进等法以不散乱的状态捆缚于同一所缘,使其发挥各自的作用,因此说「意是轭带」。
\item 人以此忆念长久所作等事,或由自身忆念为\textbf{念},它以不忘失为相。破土者为\textbf{犁铧},以之驱刺者为\textbf{刺棒}。因为好比婆罗门的犁铧与刺棒,如是世尊的念与毗婆舍那及道相应。这里,好比犁铧保护犁,且行于其前,如是念探究诸善法的趣向,或者令所缘现前而保护慧之犁,如\begin{quoting}以念守护之心而住。(增支部第 10:20 经)\end{quoting}等所说而为「守护」,且以不忘失而在其前。因为慧能了知念所惯习之法,而非忘失者。又好比刺棒示耕牛以戳刺的怖畏,使不消沉,且遮止邪路,如是念示精进之耕牛以恶趣的怖畏,使不懈怠消沉,防止行于被称为种种爱欲的非行处,促成业处并遮止邪路。因此说「念是我的犁铧与刺棒」。\end{enumerate}

\subsection\*{\textbf{78} \textcolor{gray}{\footnotesize 〔78〕}}

\textbf{「身防护,语防护,节制于食与腹,\\}
\textbf{「我用真实芟夷,调柔是我的解脱。}

Kāyagutto vacīgutto, āhāre udare yato;\\
saccaṃ karomi niddānaṃ, soraccaṃ me pamocanaṃ. %\hfill\textcolor{gray}{\footnotesize 3}

\begin{enumerate}\item \textbf{身防护}即以三种身善行防护,\textbf{语防护}即以四种语善行防护,以上是说别解脱律仪。\textbf{节制于食}与腹,此中以食为首而摄一切资具,即于四种资具节制、自制、离随烦恼之义,以此而说活命遍净戒。\textbf{节制于腹},即是说于腹节制、自制,度量而食,于食知量,以此于食知量为首而说资具受用戒。
\item 以此显明什么?好比你,婆罗门!播下种子后,为守护谷物而修建棘篱、树篱或墙围,因此那些牛、水牛、鹿群等不得而入,无法劫掠谷物,如是我播下信之种子后,为守护种种品类的善之谷物而修建由身、语、食的防护所造的三种围垣,因此,贪等不善法之牛、水牛、鹿群等不得而入,无法劫掠种种品类的善之谷物。
\item \textbf{我用真实芟夷},此中由二门的不欺骗为真实。芟夷即斩断、割断、除根,且当知此业格为具格之义,因为此中其义为「我以真实芟夷」。这说的是什么?好比你作了外在的耕作,对侵害谷物之杂草以手或以镰芟夷之,如是我也作了内在的耕作,对侵害善之谷物的欺骗之杂草以真实芟夷之。或者,当知此中的真实为智之真实,即被称为如实之智者,因此当连结作「我以此芟夷我想等的杂草」。
\item 或者,\textbf{芟夷}即斩草者、割草者、除根者之义。在此情况下,好比你让奴仆或工人芟夷「除去杂草」,而使之为杂草的斩草者、割草者、除根者,如是我用真实,则可以用业格来说。或者,\textbf{真实}即见之真实。我芟夷其应斩断者、应割断者、应根除者,如此也可以用业格来说。
\item \textbf{调柔是我的解脱},此中,凡以「身不违犯、语不违犯」等所说的「戒为调柔」并非这里的意思,这以「身防护」等方法已说,而指的是阿罗汉果。因为它由喜悦于善妙的涅槃而被称为调柔。\textbf{解脱},即免于努力。
\item 这说的是什么?好比你的解脱,由仍需在哺时、明日或来年努力,故并非解脱,而我不如是,我并无中途的解脱。因为我从燃灯十力时起,既已用慧之犁套上精进之耕牛,耕作着四阿僧祇又十万劫的大耕作,只要未证等正觉,便无解脱。且当我度过这所有时间,坐于菩提树下不可战胜之跏趺坐,生起了随以一切功德的阿罗汉果之时,以我证得一切热情之安息而解脱,现已不再有需要努力者。世尊就此义而说「调柔是我的解脱」。\end{enumerate}

\subsection\*{\textbf{79} \textcolor{gray}{\footnotesize 〔79〕}}

\textbf{「精进是我负重的牛,运载至离轭安稳,\\}
\textbf{「它前行而不退转,所到之处即不忧伤。}

Vīriyaṃ me dhuradhorayhaṃ, yogakkhemādhivāhanaṃ;\\
gacchati anivattantaṃ, yattha gantvā na socati. %\hfill\textcolor{gray}{\footnotesize 4}

\begin{enumerate}\item \textbf{精进是我负重的牛},此中「精进」即以「身或心发起精进」等方法所说的精勤。负重的牛,即运载负重之义。因为好比婆罗门的犁为负重的牛所牵引,破碎了地的密集,断绝了根的相续,如是世尊的慧之犁为精进所牵引,破碎了如前所说的密集,断绝了烦恼相续,因此说「精进是我负重的牛」。
\item 或者,运载前面的负荷为负重,运载根本的负荷为牛,负重与牛为\textbf{负重的牛}。这里,好比婆罗门的各个犁上有四头耕牛运载负重,破坏已生及未生的草根而成就谷物,如是世尊有四正勤的精进运载负重,破坏已生及未生的不善根而成就善,因此说「精进是我负重的牛」。
\item \textbf{运载至离轭安稳}\footnote{离轭安稳 \textit{yogakkhema}:即旧译「瑜伽安稳」者,Norman 英译作 rest-from-exertion,菩提比丘作 security from bondage。},此中以离于诸轭而安稳之「离轭安稳」来说涅槃,以之为目标而被运载,或向着而被运载为「运载至」。向着离轭安稳的运载为「运载至离轭安稳」。以此显明什么?好比你的负重的牛向着东方或西方等某一方被运载,如是我的负重的牛向着涅槃被运载。
\item 如是被运载,且\textbf{它前行而不退转}。好比你的负重的牛运载的犁到了田头又再转还,如是从燃灯时起,它唯前行而不退转。或者,因为被彼彼道所断的烦恼无需再再而断,而被你的犁斩断的草在其后某时需再斩断,所以,它以初道舍弃与见同义的烦恼,以第二舍弃粗重者,以第三舍弃微细俱行的烦恼,以第四舍弃一切烦恼,前行而不退转。或者,\textbf{前行而不退转}即成无退转而前行之义,\textbf{它}即此负重的牛,于此当知如是断句\footnote{如是断句:这里是将原文的 anivattantaṃ 断作 anivattan taṃ,译文取巧,勉为其难,读者知之。}。
\item 且好比你如是而行的负重的牛不得至彼处,所到之处耕者能无忧、离忧、离尘而不忧伤,但它得至彼处,\textbf{所到之处即不忧伤}。以念之刺棒督促着此精进之负重的牛得达之处,我这样的耕者便无忧、离忧、离尘而不忧伤,得达作为根除一切忧箭、被称为涅槃不死之处。\end{enumerate}

\subsection\*{\textbf{80} \textcolor{gray}{\footnotesize 〔80〕}}

\textbf{「如是,这耕作已作,那是不死之果,\\}
\textbf{「他耕作这耕作已,从一切苦中解脱。」}

Evam esā kasī kaṭṭhā, sā hoti amatapphalā;\\
etaṃ kasiṃ kasitvāna, sabbadukkhā pamuccatī” ti. %\hfill\textcolor{gray}{\footnotesize 5}

\begin{enumerate}\item 现在,世尊为作结语而说此颂。其略义为:婆罗门!我\textbf{这}为苦行之雨水所摄受的信之种子的\textbf{耕作},以意所造的轭带将慧所造的轭与犁及惭所造的辕捆缚一处,在慧犁上击打念之犁铧,拿了念之刺棒,以身、语、食之守护来守护,用真实芟夷,向调柔之解脱运载朝向离轭安稳的精进之负重的牛而不退转,故而\textbf{已作}、已圆满耕作终了的四种沙门果。
\item \textbf{那是不死之果},即那是这耕作的不死之果。不死被称为涅槃,即涅槃的功德之义。然而,这耕作的不死之果不是唯我一人的,而是任何刹帝利、婆罗门、吠舍、首陀罗、在家人或出家人若能耕作此耕作,\textbf{他}全都能\textbf{耕作这耕作已,从一切苦中解脱},从一切流转之苦、苦苦、行苦、坏苦中解脱。如是,世尊对婆罗门以阿罗汉为顶点,以涅槃为终了,完成了开示。\end{enumerate}

\textbf{于是,耕田婆罗豆婆遮婆罗门用大铜钵盛了粥,授予世尊:「请乔达摩君吃粥!您是耕者,因为乔达摩君耕作不死之果。」}

Atha kho Kasibhāradvājo brāhmaṇo mahatiyā kaṃsapātiyā pāyasaṃ vaḍḍhetvā Bhagavato upanāmesi: “bhuñjatu bhavaṃ Gotamo pāyasaṃ, kassako bhavaṃ, yañ hi bhavaṃ Gotamo amatapphalaṃ kasiṃ kasatī” ti.

\begin{enumerate}\item 随后,婆罗门听了甚深之义的开示,并了知到「吃了我的耕作之果,翌日便又饿了,而他的耕作为不死之果,吃了那果,能从一切苦中解脱」,净喜、现净喜状,始作施粥,因此说「\textbf{于是,耕田婆罗豆婆遮……}」。
\item 这里,\textbf{铜钵},即金钵,自己价值一万的金器皿。\textbf{授予世尊},即以酥、蜜、糖等调和后,覆以黄麻布,举起后恭敬地献给如来。为何?\textbf{请乔达摩君吃粥!您是耕者}。随后便说了成为耕者的原因,\textbf{因为乔达摩君耕作不死之果}。\end{enumerate}

\subsection\*{\textbf{81} \textcolor{gray}{\footnotesize 〔81〕}}

\textbf{「我不应受用吟颂之物,对诸正观者,婆罗门!此即非法,\\}
\textbf{「诸佛拒绝吟颂之物,法既存在,婆罗门!此即行事之道。}

“Gāthābhigītaṃ me abhojaneyyaṃ, sampassataṃ brāhmaṇa n’esa dhammo;\\
gāthābhigītaṃ panudanti buddhā, dhamme satī brāhmaṇa vuttir esā. %\hfill\textcolor{gray}{\footnotesize 6}

\begin{enumerate}\item 于是,世尊说了此颂。这里,\textbf{吟颂之物},即说了偈颂而得的意思。\textbf{诸正观者},即诸正确地观见活命遍净者,或普见为正观,即是说诸佛。\textbf{此即非法},即无「应受用吟颂之物」这样的法、这样的作持,所以\textbf{诸佛拒绝}、反对、不受用\textbf{吟颂之物}。
\item 那世尊为了粥而吟诵偈颂,为什么却这样说?非为此而吟诵,而且从清早起便站在田边,未曾得一匙之食,还显明了所有佛德,其所得等同于舞者歌舞后的所得,因此说是「吟颂之物」,且因为这样对诸佛不合适,所以说「不应受用」。且这也不随适于少欲,所以还出于对后人的怜悯而如是说。且于此,他们甚至反对藉由他人阐明自己的功德所得的利养,如少欲的陶匠伽吒迦罗\footnote{陶匠伽吒迦罗事,见\textbf{中部}第 81 经。},何况世尊具足到达顶点之少欲,反能受用藉由自己阐明自己的功德而得的利养?因此,世尊说此是合适的。
\item 至此已显明开示之遍净,为自己洗脱这世间的指责「对不净喜、不欲施的婆罗门,沙门乔达摩以吟诵偈颂而令欲施,便得了食物,对他开示是为了财物」,现在,为显明活命之遍净而说「\textbf{法既存在,婆罗门!此即行事之道}」。其义为:活命遍净之法,或十种善行之法,或诸佛作持之法既然存在、可得、未毁、现存,此即行事之道,诸佛的寻求、遍求、活命的行事之道一向洁净,如于空中伸开手掌一样,婆罗门!\end{enumerate}

\subsection\*{\textbf{82} \textcolor{gray}{\footnotesize 〔82〕}}

\textbf{「对整全者、大仙、漏尽者、恶作止息者,应以其它\\}
\textbf{「饮食给侍,因为他是希求福德者的良田。」}

Aññena ca kevalinaṃ mahesiṃ, khīṇāsavaṃ kukkuccavūpasantaṃ;\\
annena pānena upaṭṭhahassu, khettaṃ hi taṃ puññapekkhassa hotī” ti. %\hfill\textcolor{gray}{\footnotesize 7}

\begin{enumerate}\item 如是说已,婆罗门生起忧虑「他拒绝了我的粥,这食物据说已不合适,我真不幸,竟不得布施」,便想「兴许他会接受其它的」。世尊了知此后,想「我确定了行乞的时间而来『以这些时间令这婆罗门净喜』,婆罗门却生起忧虑,现在,心以此忧虑忧恼于我,将不能通达甘露胜法」,为令婆罗门生起净喜,以此圆满希求和心愿,说了此颂。
\item 这里,\textbf{整全者}\footnote{据菩提比丘注 572,整全者 \textit{kevalī} 的定义突出了这术语的两个含义:圆满与离系,且此术语在耆那教中更为重要。Jayawickrama 评论说:「这个概念的起源与产生肯定是前佛教的,但并不一定要通过任何耆那教的影响才能被佛教采纳。」},即圆满一切功德,或离于一切轭之义。寻求戒蕴等的大功德者为\textbf{大仙}。由遍尽一切漏为\textbf{漏尽者}。由止息了从手足不安开始的一切恶作为\textbf{恶作止息者}\footnote{据菩提比丘注 574,恶作 \textit{kukkucca} 有两层意思,即不安与后悔,见\textbf{前分离经}的\textbf{义释}。}。\textbf{给侍},即施食、敬候。当婆罗门生起如是之心时,方可说法,而非说「布施、拿来」等。其余于此自明。\end{enumerate}

\textbf{「那么,乔达摩君!我应把这粥给谁呢?」「我实不见,婆罗门!在这俱有天、魔、梵、沙门婆罗门、天人的人世间,有谁受用此粥而能正常消化的,除了如来或如来弟子,因此你,婆罗门!应把这粥丢到少草处,或投入无生类的水中。」于是,耕田婆罗豆婆遮婆罗门便把这粥投入无生类的水中。被投入水中的粥唧唧啾啾,冒着烟雾和水汽。好比白天晒热的犁铧被投入水中,唧唧啾啾,冒着烟雾和水汽,如是被投入水中的粥唧唧啾啾,冒着烟雾和水汽。}

“Atha kassa cāhaṃ, bho Gotama, imaṃ pāyasaṃ dammī” ti? “Na khvāhaṃ taṃ, brāhmaṇa, passāmi sadevake loke samārake sabrahmake sassamaṇabrāhmaṇiyā pajāya sadevamanussāya, yassa so pāyaso bhutto sammā pariṇāmaṃ gaccheyya, aññatra Tathāgatassa vā Tathāgatasāvakassa vā, tena hi tvaṃ, brāhmaṇa, taṃ pāyasaṃ appaharite vā chaḍḍehi appāṇake vā udake opilāpehī” ti. Atha kho Kasibhāradvājo brāhmaṇo taṃ pāyasaṃ appāṇake udake opilāpesi. Atha kho so pāyaso udake pakkhitto cicciṭāyati ciṭiciṭāyati sandhūpāyati sampadhūpāyati. Seyyathāpi nāma phālo divasaṃ santatto udake pakkhitto cicciṭāyati ciṭiciṭāyati sandhūpāyati sampadhūpāyati, evam evaṃ so pāyaso udake pakkhitto cicciṭāyati ciṭiciṭāyati sandhūpāyati sampadhūpāyati.

\begin{enumerate}\item 于是,婆罗门想「这粥是拿给世尊的,我不应把它随己欲去施给某人」,便说「\textbf{那么,我……}」。随后,世尊了知到「这粥除了如来及如来弟子,别人无法消化」,便说「\textbf{我实不见……}」。
\item 这里,以\textbf{俱有天}之语摄五欲界天,以\textbf{俱有魔}之语摄第六欲界天,以\textbf{俱有梵}之语摄色界梵,无色界则以「能受用」而未予考虑,以\textbf{俱有沙门婆罗门}之语摄教法之怨敌的沙门婆罗门与已止息恶、已驱除恶的沙门婆罗门,以\textbf{人}之语摄有情世间,以\textbf{俱有天人}之语摄共许的天\footnote{共许的天:即俗称为「天」的国王等。}与其余的人。如是,当知此中以三语摄器世间,以二「人」摄有情世间。此是略说,我们将在旷野经中详说。
\item 那为什么在俱有天等之处,无人能正常消化呢?由于在粗鄙之中投入了微细的食素。因为在为世尊准备的这粥中,诸天已投入食素,如在善生的粥中\footnote{善生的粥,见\textbf{本生}义注·序颂。}、在纯陀所煮的栴檀木耳中\footnote{纯陀所煮的栴檀木耳,见\textbf{长部}·大般涅槃经。}、在世尊于毗兰若所得的团食中、以及在药犍度里迦旃延的砂糖罐中残余的砂糖中\footnote{毗兰若及迦旃延事,均见\textbf{律藏}。}。由于在粗鄙之中投入了微细的食素,天人无法消化,因为诸天的身体微细,其中粗鄙的人类食物不能正常消化,而人类也无法消化,因为人类的身体粗鄙,其中微细的天之食素无法正常消化。但如来可以自然之火消化、完全吸收,有些说「以身力、智力的威力」。漏尽的如来弟子则以定力及知量消化,其余的人,即便是具神变者也不能消化。或者,此中的原因不可思议,唯是佛的境域。
\item \textbf{因此你},因为我未见他人,且不适合我,而不适合我者不适合我的弟子,所以说「你,婆罗门!……」。\textbf{少草处},即少量的草上,或如石背等鲜长草处。\textbf{无生类的水中},即不含生类,或不会因粥的散布而使生类致死的大水聚中。这是为保护草以及依草的生类与水中的生类而说的。
\item \textbf{唧唧啾啾},即发出如是的声响。为什么成了这样?以世尊的威力,而非水的、粥的、婆罗门的或其他天人夜叉等的。因为世尊为使婆罗门对法悚惧,便决意如是。\textbf{好比},即以譬喻表达,而说好比犁铧等等。\end{enumerate}

\textbf{于是,耕田婆罗豆婆遮婆罗门惊恐万状,身毛竖立,往世尊处走去,走到后,以头投于世尊的双足,对世尊说:「希有!乔达摩君!希有!乔达摩君!好比,乔达摩君!能扶正被倾倒的,能揭示被遮蔽的,能给迷者指路,能在黑暗中持油灯,以使『具眼者能见色』,如是乔达摩君以种种方法阐明法。我皈依乔达摩君、法与比丘僧,愿我能在乔达摩君跟前出家,愿我能受具足!」}

Atha kho Kasibhāradvājo brāhmaṇo saṃviggo lomahaṭṭhajāto yena Bhagavā ten’upasaṅkami, upasaṅkamitvā Bhagavato pādesu sirasā nipatitvā, Bhagavantaṃ etad avoca: “abhikkantaṃ, bho Gotama, abhikkantaṃ, bho Gotama, seyyathāpi, bho Gotama, nikkujjitaṃ vā ukkujjeyya, paṭicchannaṃ vā vivareyya, mūḷhassa vā maggaṃ ācikkheyya, andhakāre vā telapajjotaṃ dhāreyya ‘cakkhumanto rūpāni dakkhantī’ ti, evam evaṃ bhotā Gotamena anekapariyāyena dhammo pakāsito. Esāhaṃ bhavantaṃ Gotamaṃ saraṇaṃ gacchāmi dhammañ ca bhikkhusaṅghañ ca, labheyyāhaṃ bhoto Gotamassa santike pabbajjaṃ, labheyyaṃ upasampadan” ti.

\begin{enumerate}\item 心中\textbf{惊恐万状},身上\textbf{身毛竖立}。据说,他身上九万九千个毛孔如金壁上镶嵌了摩尼的象牙一般竖立。其余则自明。
\item 且投于双足已,为极大地随喜于世尊法的开示,\textbf{对世尊说:「希有!乔达摩君!希有!乔达摩君!……」}。此处的\textbf{希有}一词用于极大的随喜,其详细的释义将在吉祥经注中说明。且当知出于极大随喜之义,而说「善哉!善哉!乔达摩君」等。\begin{quoting}当怖畏、忿恨、称赏,当匆忙、兴奋、惊异,\\当欢笑、忧伤、净喜,智者便一唱三叹。\end{quoting}且以此相,当知此处是以净喜与称赏而说这两遍的。或者,\textbf{希有}即是说极优异、极可爱、极可意、极善妙。
\item 这里,以一句「希有」赞美开示,以一句赞美自己的净喜。此中的旨趣为:乔达摩君!乔达摩君法的开示希有,我对乔达摩君法的开示的净喜希有。或者,他就二二之义赞美乔达摩君之语:乔达摩君之语由破灭过失而希有、由证得功德而希有,同样,可以用「由生起信、生起慧,由有义、有文,由浅显之句、甚深之义,由悦耳、赏心,由不自赞、不毁他,由悲之清凉、慧之洁白,由感官悦意、堪耐驳斥,由被听闻之乐、被省察之利益」等相连。
\item 随后,他更以四个譬喻赞美开示。这里,\textbf{被倾倒的},即头朝下放置,或下部成了上部。\textbf{扶正},即令朝上。\textbf{被遮蔽的},即被草、叶等覆盖的。\textbf{揭示},即根除。\textbf{迷者},即迷失方向者。\textbf{指路},即把臂告之「此是路」。\textbf{在黑暗中},即在黑分之十四夜、夜半、林密、云遮等的四支黑暗之中。以上是词义。
\item 而其旨趣与章句为:好比有人能扶正被倾倒的,如是使忽于妙法、落入非妙法的我从非妙法中出起,好比能揭示被遮蔽的,如是从迦叶世尊的教法隐没以来,揭示由执取邪见而被覆蔽的教法,好比能给迷者指路,如是给行于错路、邪路的我指以天与解脱之路,好比能在黑暗中持油灯,如是为陷入愚痴黑暗、不见佛等宝之色的我持以摧破覆蔽此的愚痴黑暗的开示之灯,\textbf{乔达摩君}以这些方法为我开示,\textbf{以种种方法阐明法}。
\item 或者,依有些意见,因为这法,以见苦及舍弃以不净为净的颠倒,与扶正被倾倒的相似,以见集及舍弃以苦为乐的颠倒,与揭示被遮蔽的相似,以见灭及舍弃以无常为常的颠倒,与给迷者指路相似,以见道及舍弃以无我为我的颠倒,与黑暗中的灯相似,所以如是阐明:好比能扶正被倾倒的……持油灯,以使「具眼者能见色」。
\item 且此中,因为以信、苦行、身防护等阐明了戒蕴,以慧阐明了慧蕴,以惭、意等阐明了定蕴,以离轭安稳阐明了灭,如是,三蕴之圣道与灭明确地阐明了二圣谛,这里以「道为集的敌对,而灭为苦的」之敌对阐明了二,如此,以此方法阐明了四圣谛,所以当知为「以种种方法阐明」。
\item \textbf{皈依},即已经以投于双足、跪拜而皈依,现在则以言语受持而说。或者,以跪拜唯皈依佛,现在则以其为首,而说还皈依余下的法与僧。\textbf{从今起} \textit{ajjatagge}\footnote{从今起、尽寿命、请乔达摩君受持我皈依:这里的原文并无此句,当是为\textbf{贱民经}等作的解释。},即以今日为始,文本或作 ajjadagge,d 为连接字,即是说以今日为首。具有生命为\textbf{尽寿命},即是说只要我的生命转起,\textbf{请乔达摩君受持}、了知\textbf{我}无其他大师的藉由三皈依的\textbf{皈依}。
\item 至此,以此显明随适于所闻的行道。或以「被倾倒的」等显明大师的成就已,以「我皈依」等显明弟子的成就。或以彼显明慧的获得已,以此显明信的获得。
\item 现在,如是获得信的具慧者,欲行所当行,而向世尊请求「\textbf{愿我能……}」。这里,由对世尊的神变等心极净喜,「世尊甚至舍弃了转轮王位而出家,我又有什么理由呢」,以信请求出家,且为希求圆满,以慧请求具足。其余则自明。\end{enumerate}

\textbf{耕田婆罗豆婆遮婆罗门便在世尊跟前出了家,受了具足。受具后不久,尊者婆罗豆婆遮独一、远离、不放逸、热忱、自励而住,此后不久,就在今生,他以自身的证智证得并具足而住于无上梵行的终了,正是为此义利,族姓子们从家出至非家。他证知:「生已灭尽,梵行已立,应作已作,不更为此。」尊者婆罗豆婆遮便成了众阿罗汉中的某个。}

Alattha kho Kasibhāradvājo brāhmaṇo Bhagavato santike pabbajjaṃ, alattha upasampadaṃ. Acirūpasampanno kho panāyasmā Bhāradvājo eko vūpakaṭṭho appamatto ātāpī pahitatto viharanto nacirass’eva, yass’atthāya kulaputtā sammad eva agārasmā anagāriyaṃ pabbajanti, tad anuttaraṃ brahmacariya-pariyosānaṃ diṭṭhe va dhamme sayaṃ abhiññā sacchikatvā upasampajja vihāsi. “Khīṇā jāti, vusitaṃ brahmacariyaṃ, kataṃ karaṇīyaṃ, nāparaṃ itthattāyā” ti abbhaññāsi. Aññataro ca panāyasmā Bhāradvājo arahataṃ ahosī ti.

\begin{enumerate}\item 在「独一、远离」等中,以身独处为\textbf{独一},以心独处为\textbf{远离},以于业处不舍弃念为\textbf{不放逸},以被称为身心精进的热忱为\textbf{热忱},以不顾身命为\textbf{自励},以住于某种威仪为\textbf{住}。
\item \textbf{此后不久}是就出家来说。\textbf{族姓子},有两种族姓子:藉由出身的族姓子与藉由正行的族姓子,而他于两处都是族姓子。家的利益为家产,即耕田、护牛、地产、养育等事,无此家产者为\textbf{非家},即出家的同义语。\textbf{出至},即靠近、前往。\textbf{梵行的终了},即道梵行的终了,即是说阿罗汉果。因为为此义利,族姓子们才出家。\textbf{就在今生},即就在此自体中。\textbf{以自身的证智证得},即唯以自己的慧作现量,不缘他人而了知之义。\textbf{具足而住},即得达或成就而住。且当如是而住时,\textbf{他证知「生已灭尽……」},以此显明其省察地。
\item 那么他的什么生已灭尽,且他又如何证知此?当答:非其过去生已灭尽,由先前已灭尽故,非未来(生已灭尽),由尚未在未来努力故,非现在(生已灭尽),由正在发生故,而是由无有道的修习将于一、四、五蕴有中生起的一、四、五蕴等类之生,由道的修习,以得达不生法而灭尽。他以道的修习省察舍断的烦恼已,了知到「当无烦恼时,即便正在发生的业,在未来也成非结生者」,便了知此。
\item \textbf{已立},即已住、已遍住、已作、已行、已完成之义。\textbf{梵行},即道梵行。\textbf{应作已作},即对四谛,以四道的遍知、舍断、证得、修习等十六种应作已完成之义。\textbf{不更为此},即现在,为了如是十六应作这样的状态,或是为了烦恼之灭尽,无有道的修习。或者,「为此」即从这样的状态,即现在,从如是品类正在转起的蕴相续,无有次后的蕴相续。他证知:「这已被遍知的五蕴,如断根的树一般而存。」\textbf{某个},即一个。这里的旨趣是,据说,尊者婆罗豆婆遮成了众大弟子之一。\end{enumerate}

\begin{center}\vspace{1em}耕田婆罗豆婆遮经第四\\Kasibhāradvājasuttaṃ catutthaṃ.\end{center}

%\begin{flushright}癸卯十一月廿八二稿\end{flushright}