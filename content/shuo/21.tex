\chapter{巧蓺第二十一}

\subsection*{1}

\textbf{彈棊始自魏宮內,用妝匳戲,}{\footnotesize \textbf{傅玄彈棊賦敘}曰漢成帝好蹴踘,劉向以謂勞人體、竭人力,非至尊所宜御,乃因其體作彈棊,今觀其道,蹴踘道也。\textbf{按}玄此言,則彈棊之戲,其來久矣,且梁冀傳云「冀善彈棊,格五」,而此云起魏世,謬矣。}\textbf{文帝於此戲特妙,用手巾角拂之,無不中,有客自云能,帝使為之,客著葛巾角,低頭拂棊,妙踰於帝。}{\footnotesize \textbf{典論}帝自敘曰戲弄之事,少所喜,唯彈棊略盡其妙,少時嘗為之賦,昔京師少工有二焉,合鄉侯、東方世安、張公子,常恨不得與之對也。\textbf{博物志}曰帝善彈棊,能用手巾角,時有一書生,又能低頭以所冠葛巾角撇棊也。}

\subsection*{2}

\textbf{陵雲臺樓觀精巧,先稱平眾木輕重,然後造構,乃無錙銖相負揭,臺雖高峻,常隨風搖動,而終無傾倒之理,魏明帝登臺,懼其勢危,別以大材扶持之,樓即穨壞,論者謂輕重力偏故也。}{\footnotesize \textbf{洛陽宮殿簿}曰陵雲臺上壁方十三丈,高九尺,樓方四丈,高五丈,棟去地十三丈五尺七寸五分也。}

\subsection*{3}

\textbf{韋仲將能書,魏明帝起殿,欲安榜,使仲將登梯題之,既下,頭鬢皓然,因敕兒孫:「勿復學書。」}{\footnotesize \textbf{文章敘錄}曰韋誕,字仲將,京兆杜陵人,太僕端子,有文學,善屬辭,以光祿大夫卒。\textbf{衛恆四體書勢}曰誕善楷書,魏宮觀多誕所題,明帝立陵霄觀,誤先釘榜,乃籠盛誕,轆轤長絙引上,使就題之,去地二十五丈,誕甚危懼,乃戒子孫絕此楷法,著之家令。}

\subsection*{4}

\textbf{鍾會是荀濟北從舅,二人情好不協,荀有寶劍,可直百萬,常在母鍾夫人許,}{\footnotesize \textbf{孔氏志怪}曰勖以寶劍付妻。}\textbf{會善書,學荀手跡,作書與母取劍,仍竊去不還,}{\footnotesize \textbf{世語}曰會善學人書,伐蜀之役,於劍閣要鄧艾章表,皆約其言,令詞旨倨傲,多自矜伐,艾由此被收也。}\textbf{荀勖知是鍾而無由得也,思所以報之,後鍾兄弟以千萬起一宅,始成,甚精麗,未得移住,荀極善畫,乃潛往畫鍾門堂,作太傅形象,衣冠狀貌如平生,二鍾入門,便大感慟,宅遂空廢。}{\footnotesize \textbf{孔氏志怪}曰于時咸謂勖之報會,過於所失數十倍,彼此書畫,巧妙之極。}

\subsection*{5}

\textbf{羊長和博學工書,}{\footnotesize \textbf{文字志}曰忱性能草書,亦善行隸,有稱於一時。}\textbf{能騎射,善圍棊,諸羊後多知書,而射、弈餘蓺莫逮。}

\subsection*{6}

\textbf{戴安道就范宣學,}{\footnotesize \textbf{中興書}曰逵不遠千里,往豫章詣范宣,宣見逵,異之,以兄女妻焉。}\textbf{視范所為,范讀書亦讀書,范鈔書亦鈔書,唯獨好畫,范以為無用,不宜勞思於此,戴乃畫南都賦圖,范看畢咨嗟,甚以為有益,始重畫。}

\subsection*{7}

\textbf{謝太傅云:「顧長康畫,有蒼生來所無。」}{\footnotesize \textbf{續晉陽秋}曰愷之尤好丹青,妙絕於時,曾以一廚畫寄桓玄,皆其絕者,深所珍惜,悉糊題其前,桓乃發廚後取之,好加理,後愷之見封題如初,而畫並不存,直云「妙畫通靈,變化而去,如人之登仙矣」。}

\subsection*{8}

\textbf{戴安道中年畫行像甚精妙,庾道季看之,語戴云:「神明太俗,由卿世情未盡。」戴云:「唯務光當免卿此語耳。」}{\footnotesize \textbf{列仙傳}曰務光,夏時人也,耳長七寸,好鼓琴,服菖蒲韭根,湯將伐桀,謀於光,光曰「非吾事也」,湯曰「伊尹何如」,務光曰「彊力忍詬,不知其它」,湯克天下,讓於光,光曰「吾聞無道之世,不踐其土,況讓我乎」,負石自沈於盧水。}

\subsection*{9}

\textbf{顧長康畫裴叔則,頰上益三毛,人問其故,顧曰:「裴楷儁朗有識具,正此是其識具。」看畫者尋之,定覺益三毛如有神明,殊勝未安時。}{\footnotesize 愷之歷畫古賢,皆為之贊也。}

\subsection*{10}

\textbf{王中郎以圍棊是坐隱,支公以圍棊為手談。}{\footnotesize \textbf{博物志}曰堯作圍棊,以教丹朱。\textbf{語林}曰王以圍棊為手談,故其在哀制中,祥後客來,方幅會戲。}

\subsection*{11}

\textbf{顧長康好寫起人形,}{\footnotesize \textbf{續晉陽秋}曰愷之圖寫特妙。}\textbf{欲圖殷荊州,殷曰:「我形惡,不煩耳。」顧曰:「明府正為眼爾,}{\footnotesize 仲堪眇目故也。}\textbf{但明點童子,飛白拂其上,使如輕雲之蔽日。」}{\footnotesize 日,一作月。}

\subsection*{12}

\textbf{顧長康畫謝幼輿在巖石裏,人問其所以,顧曰:「謝云『一丘一壑,自謂過之』,此子宜置丘壑中。」}

\subsection*{13}

\textbf{顧長康畫人,或數年不點目精,人問其故,顧曰:「四體妍蚩,本無關於妙處,傳神寫照,正在阿堵中。」}

\subsection*{14}

\textbf{顧長康道:「畫手揮五弦易,目送歸鴻難。」}