\begin{center}
    Namo tassa Bhagavato Arahato Sammāsambuddhassa
\end{center}

\section*{773}

\begin{quoting}
    \textbf{对于欲求着爱欲的人,若他于此成功,\\有死者既得了所希望的,必然有喜意。}
    \footnote{经文及义注的翻译,可访问 \url{https://ehipassa.org/khuddaka/suttanipata/401/}。}
\end{quoting}

%\begin{quoting}\textbf{Kāmaṃ kāmayamānassa, tassa ce taṃ samijjhati;\\Addhā pītimano hoti, laddhā macco yad icchati.}\end{quoting}

\textbf{对于欲求着爱欲的人}。\textbf{爱欲},总的来说有两种爱欲:物欲及烦恼欲。什么是\textbf{物欲}?适意之色、适意之声、适意之香、适意之味、适意之触,床单、毯子、奴婢、山羊、鸡猪、象牛马骡、田、物、钱、金、村镇王畿、王国、国土、宝藏、仓库,举凡可贪之物,即物欲。

%\textbf{Kāmaṃ kāmayamānassā} ti \textbf{kāmā} ti uddānato dve kāmā— vatthukāmā ca kilesakāmā ca. Katame \textbf{vatthukāmā}? Manāpikā rūpā manāpikā saddā manāpikā gandhā manāpikā rasā manāpikā phoṭṭhabbā; attharaṇā pāvuraṇā dāsidāsā ajeḷakā kukkuṭa-sūkarā hatthi-gavāssa-vaḷavā khettaṃ vatthu hiraññaṃ suvaṇṇaṃ gāma-nigama-rājadhāniyo raṭṭhañ ca janapado ca koso ca koṭṭhāgārañ ca, yaṃ kiñci rajanīyaṃ vatthu— vatthukāmā.

还有过去的爱欲、未来的爱欲、现在的爱欲,内在的爱欲、外在的爱欲、内在外在的爱欲,低贱的爱欲、中等的爱欲、胜妙的爱欲,苦处的爱欲、人间的爱欲、天界的爱欲,现前的爱欲、化生的爱欲、由他人化生的爱欲,占有的爱欲、非占有的爱欲,执为我的爱欲、非执为我的爱欲,一切欲界之法、一切色界之法、一切无色界之法、渴爱之所依、渴爱之所缘,以可欲之义、可贪之义、可醉之义而为爱欲,这些被称为物欲。

%Api ca atītā kāmā anāgatā kāmā paccuppannā kāmā; ajjhattā kāmā bahiddhā kāmā ajjhattabahiddhā kāmā; hīnā kāmā majjhimā kāmā paṇītā kāmā; āpāyikā kāmā mānusikā kāmā dibbā kāmā; paccupaṭṭhitā kāmā nimmitā kāmā animmitā kāmā paranimmitā kāmā; pariggahitā kāmā apariggahitā kāmā mamāyitā kāmā amamāyitā kāmā; sabbe pi kāmāvacarā dhammā, sabbe pi rūpāvacarā dhammā, sabbe pi arūpāvacarā dhammā, taṇhāvatthukā taṇhārammaṇā kāmanīyaṭṭhena rajanīyaṭṭhena madanīyaṭṭhena kāmā— ime vuccanti vatthukāmā.

什么是\textbf{烦恼欲}?欲即爱欲、贪染即爱欲、贪染之欲即爱欲,思惟即爱欲、贪染即爱欲、贪染之思惟即爱欲,诸爱欲中的欲贪、欲染、欲喜、欲爱、欲腻、欲恼、欲痴、欲耽、欲暴流、欲轭、欲取、欲贪之盖,

%Katame \textbf{kilesakāmā}? Chando kāmo rāgo kāmo chandarāgo kāmo; saṅkappo kāmo rāgo kāmo saṅkapparāgo kāmo; yo kāmesu kāmacchando kāmarāgo kāmanandī kāmataṇhā kāmasneho kāmapariḷāho kāmamucchā kāmajjhosānaṃ kāmogho kāmayogo kāmupādānaṃ kāmacchandanīvaraṇaṃ.

\begin{quoting}我看到你的爱欲之因,爱欲从思惟生起,\\而我将不再思惟它,如是爱欲则将不存。\end{quoting}

%\begin{quoting}“Addasaṃ kāma te mūlaṃ, saṅkappā kāma jāyasi;\\Na taṃ saṅkappayissāmi, evaṃ kāma na hehisī” ti.—\end{quoting}

这些被称为烦恼欲。\textbf{欲求着},即欲求着、希望着、受用着、希求着、艳羡着、祈求着。

%Ime vuccanti kilesakāmā. \textbf{Kāmayamānassā} ti kāmayamānassa icchamānassa sādiyamānassa patthayamānassa pihayamānassa abhijappamānassā ti— kāmaṃ kāmayamānassa.

\textbf{若他于此成功}。\textbf{若他},即刹帝利、婆罗门、吠舍、首陀罗、在家人、出家人、天或人。\textbf{此},即被称为物欲的适意之色、适意之声、适意之香、适意之味、适意之触。\textbf{成功},即成就、成功、得到、获得、证得、求得。

%\textbf{Tassa ce taṃ samijjhatī} ti. \textbf{Tassa ce} ti tassa khattiyassa vā brāhmaṇassa vā vessassa vā suddassa vā gahaṭṭhassa vā pabbajitassa vā devassa vā manussassa vā. \textbf{Tan} ti vatthukāmā vuccanti— manāpikā rūpā manāpikā saddā manāpikā gandhā manāpikā rasā manāpikā phoṭṭhabbā. \textbf{Samijjhatī} ti ijjhati samijjhati labhati paṭilabhati adhigacchati vindatī ti— tassa ce taṃ samijjhati.

\textbf{必然有喜意}。\textbf{必然},即决定、无疑、无惑、不二、不歧、必定、无误、确立之辞。\textbf{喜},即与五欲相关的喜、愉悦、喜悦、极喜悦、欢笑、极欢笑、幸福、满足、踊跃、悦意、心之遍满。\textbf{意},心即意,意向、心脏之净白即意,意处、意根、识、识蕴、所起的意识界被称为意。此意与此喜俱行、俱生、交际、相应、同生、同灭、同所依、同所缘。\textbf{有喜意},即有喜意、满足、欣喜、极欣喜、悦意、踊跃、喜悦、极喜悦。

%\textbf{Addhā pītimano hotī} ti. \textbf{Addhā} ti ekaṃsavacanaṃ nissaṃsayavacanaṃ nikkaṅkhāvacanaṃ advejjhavacanaṃ adveḷhakavacanaṃ niyogavacanaṃ apaṇṇakavacanaṃ avatthāpanavacanam etaṃ— addhā ti. \textbf{Pītī} ti yā pañcakāmaguṇa-paṭisaññuttā pīti pāmujjaṃ āmodanā pamodanā hāso pahāso vitti tuṭṭhi odagyaṃ attamanatā abhipharaṇatā cittassa. \textbf{Mano} ti yaṃ cittaṃ mano mānasaṃ hadayaṃ paṇḍaraṃ mano manāyatanaṃ manindriyaṃ viññāṇaṃ viññāṇakkhandho tajjā manoviññāṇadhātu, ayaṃ vuccati mano. Ayaṃ mano imāya pītiyā sahagato hoti sahajāto saṃsaṭṭho sampayutto ekuppādo ekanirodho ekavatthuko ekārammaṇo. \textbf{Pītimano hotī} ti pītimano hoti tuṭṭhamano haṭṭhamano pahaṭṭhamano attamano udaggamano muditamano pamoditamano hotī ti— addhā pītimano hoti.

\textbf{有死者既得了所希望的}。\textbf{既得了},即得到了、获得了、证得了、求得了。\textbf{有死者},即有情、人类、学童、男子、个体、生命、营生者、人、自在行者、摩奴所生者。\textbf{所希望的},即所希望、所受用、所希求、所艳羡、所祈求的色、声、香、味、触。是故世尊说:

%\textbf{Laddhā macco yad icchatī} ti. \textbf{Laddhā} ti labhitvā paṭilabhitvā adhigantvā vinditvā. \textbf{Macco} ti satto naro mānavo poso puggalo jīvo jāgu jantu indagu manujo. \textbf{Yad icchatī} ti yaṃ icchati yaṃ sādiyati yaṃ pattheti yaṃ piheti yaṃ abhijappati, rūpaṃ vā saddaṃ vā gandhaṃ vā rasaṃ vā phoṭṭhabbaṃ vā ti, laddhā macco yad icchati. Tenāha Bhagavā—

\begin{quoting}对于欲求着爱欲的人,若他于此成功,\\有死者既得了所希望的,必然有喜意。\end{quoting}

%\begin{quoting}“Kāmaṃ kāmayamānassa, tassa ce taṃ samijjhati;\\Addhā pītimano hoti, laddhā macco yad icchatī” ti\end{quoting}

\section*{774}

\begin{quoting}\textbf{若对这欲求着、生起欲望的人,\\那些爱欲消逝,他如被箭射穿般恼害。}\end{quoting}

%\begin{quoting}\textbf{Tassa ce kāmayānassa, chandajātassa jantuno;\\Te kāmā parihāyanti, sallaviddho va ruppati.}\end{quoting}

\textbf{若对这欲求着}。\textbf{若对这},即对这刹帝利、婆罗门、吠舍、首陀罗、在家人、出家人、天或人。\textbf{欲求着},即欲求着、希望着、受用着、希求着、艳羡着、祈求着。或说\footnote{菩提比丘:「\textbf{或说……}」之后是误把分词 \textit{kāmayāna} 当作复合词 \textit{kāma-yāna} 来解释。},被欲爱驱使、引领、裹挟、征服。好比被象车、马车、牛车、山羊车、绵羊车、骆驼车、驴车驱使、引领、裹挟、征服,如是被欲爱驱使、引领、裹挟、征服。

%\textbf{Tassa ce kāmayānassā} ti. \textbf{Tassa ce} ti tassa khattiyassa vā brāhmaṇassa vā vessassa vā suddassa vā gahaṭṭhassa vā pabbajitassa vā devassa vā manussassa vā. \textbf{Kāmayānassā} ti kāme icchamānassa sādiyamānassa patthayamānassa pihayamānassa abhijappamānassa. Atha vā kāmataṇhāya yāyati niyyati vuyhati saṃharīyati. Yathā hatthiyānena vā assayānena vā goyānena vā ajayānena vā meṇḍayānena vā oṭṭhayānena vā kharayānena vā yāyati niyyati vuyhati saṃharīyati; evam evaṃ kāmataṇhāya yāyati niyyati vuyhati saṃharīyatī ti— tassa ce kāmayānassa.

\textbf{生起欲望的人}。\textbf{欲望},诸爱欲中的欲贪、欲染、欲喜、欲爱、欲腻、欲恼、欲痴、欲耽、欲暴流、欲轭、欲取、欲贪之盖,对于他,这欲贪生起、产生、出现、生成、显现。\textbf{人},即有情、人类、学童、男子、个体、生命、营生者、人、自在行者、摩奴所生者。

%\textbf{Chandajātassa jantuno} ti. \textbf{Chando} ti yo kāmesu kāmacchando kāmarāgo kāmanandī kāmataṇhā kāmasneho kāmapariḷāho kāmamucchā kāmajjhosānaṃ kāmogho kāmayogo kāmupādānaṃ kāmacchandanīvaraṇaṃ, tassa so kāmacchando jāto hoti sañjāto nibbatto abhinibbatto pātubhūto. \textbf{Jantuno} ti sattassa narassa mānavassa posassa puggalassa jīvassa jāgussa jantussa indagussa manujassā ti— chandajātassa jantuno.

\textbf{那些爱欲消逝}。或是那些爱欲消逝,或是他从爱欲中消逝。那些爱欲如何消逝?即对这存在者,王家们夺走那些财物,或盗贼们夺走,或火烧,或水漂,或不可爱的继承者夺走,或丧失积贮,或难营的产业被破坏,或家族中出现败家子,散布、毁坏、败坏财物,无常为第八\footnote{菩提比丘:\textbf{无常为第八},其实文中给出了九种。}。如是那些爱欲减损、消逝、丧失、败落、消失、分离。如何他从爱欲中消逝?即那些财物存在,而他去世、死亡、分离。如是他从爱欲中减损、消逝、丧失、败落、消失、分离。

%\textbf{Te kāmā parihāyantī} ti— te vā kāmā parihāyanti, so vā kāmehi parihāyati. Kathaṃ te kāmā parihāyanti? Tassa tiṭṭhantass’eva te bhoge rājāno vā haranti, corā vā haranti, aggi vā dahati, udakaṃ vā vahati, appiyā vā dāyādā haranti, nihitaṃ vā nādhigacchati, duppayuttā vā kammantā bhijjanti, kule vā kulaṅgāro uppajjati, yo te bhoge vikirati vidhamati viddhaṃseti aniccatā yeva aṭṭhamī. Evaṃ te kāmā hāyanti parihāyanti paridhaṃsenti paripatanti antaradhāyanti vippalujjanti. Kathaṃ so kāmehi parihāyati? Tiṭṭhant’eva te bhoge so cavati marati vippalujjati. Evaṃ so kāmehi hāyati parihāyati paridhaṃsati paripatati antaradhāyati vippalujjati.

\begin{quoting}盗贼、王家夺走,火烧且消亡,\\然后,在临终舍弃身体及所有,\\智者了知此后,应享用并布施。\end{quoting}

%\begin{quoting}Corā haranti rājāno, aggi dahati nassati;\\Atha antena jahati, sarīraṃ sapariggahaṃ;\\Etad aññāya medhāvī, bhuñjetha ca dadetha ca\end{quoting}

布施并享用后,随其势力,无罪咎者登于天处。

%Datvā ca bhutvā ca yathānubhāvaṃ, anindito saggam upeti ṭhānan ti— te kāmā parihāyanti.

\textbf{他如被箭射穿般恼害}。好比被铁制的箭射穿,或被骨制的箭、牙制的箭、角制的箭、木制的箭射穿而恼害、焦扰、触犯、压迫,而成不安、忧虑,如是由物欲的变易、改变而生起忧悲苦恼愁。他被爱欲之箭及忧伤之箭射穿,而恼害、焦扰、触犯、压迫,而成不安、忧虑。是故世尊说:

%\textbf{Sallaviddho va ruppatī} ti. Yathā ayomayena vā sallena viddho, aṭṭhimayena vā sallena dantamayena vā sallena visāṇamayena vā sallena kaṭṭhamayena vā sallena viddho ruppati kuppati ghaṭṭīyati pīḷīyati, byādhito domanassito hoti, evam evaṃ vatthukāmānaṃ vipariṇāmaññathābhāvā uppajjanti sokaparidevadukkhadomanassupāyāsā. So kāmasallena ca sokasallena ca viddho, ruppati kuppati ghaṭṭīyati pīḷīyati byādhito domanassito hotī ti— sallaviddho va ruppati. Tenāha Bhagavā—

\begin{quoting}若对这欲求着、生起欲望的人,\\那些爱欲消逝,他如被箭射穿般恼害。\end{quoting}

%\begin{quoting}“Tassa ce kāmayānassa, chandajātassa jantuno;\\Te kāmā parihāyanti, sallaviddho va ruppatī” ti\end{quoting}

\section*{775}

\begin{quoting}\textbf{若避开爱欲,如以足(避开)蛇头,\\此具念者超越这对世间的爱著。}\end{quoting}

%\begin{quoting}\textbf{Yo kāme parivajjeti, sappasseva padā siro;\\So’maṃ visattikaṃ loke, sato samativattati.}\end{quoting}

%\textbf{Yo kāme parivajjetī} ti. \textbf{Yo} ti yo yādiso yathāyutto yathāvihito yathāpakāro yaṃṭhānappatto yaṃdhammasamannāgato khattiyo vā brāhmaṇo vā vesso vā suddo vā gahaṭṭho vā pabbajito vā devo vā manusso vā. \textbf{Kāme parivajjetī} ti. \textbf{Kāmā} ti uddānato dve kāmā— vatthukāmā ca kilesakāmā ca …pe… ime vuccanti vatthukāmā …pe… ime vuccanti kilesakāmā.

\textbf{避开爱欲},即通过两种途径避开爱欲,或由镇伏,或由正断。如何由镇伏避开爱欲?

\begin{enumerate}
    \item 看到「以少味之义,爱欲譬如骨架」,由镇伏避开爱欲,
    \item 看到「以众所共通之义,爱欲譬如肉片」……
    \item 看到「以销烬之义,爱欲譬如草炬」……
    \item 看到「以大热恼之义,爱欲譬如火坑」……
    \item 看到「以暂时现起之义,爱欲譬如梦」……
    \item 看到「以片刻之义,爱欲譬如借用」……
    \item 看到「以熟落之义,爱欲譬如树上果实」……
    \item 看到「以断头台之义,爱欲譬如屠场」……
    \item 看到「以刺穿之义,爱欲譬如戟矛」……
    \item 看到「以有害之义,爱欲譬如蛇头」……
    \item 看到「以极热之义,爱欲譬如火聚」,由镇伏避开爱欲,
\end{enumerate}

%\textbf{Kāme parivajjetī} ti dvīhi kāraṇehi kāme parivajjeti— vikkhambhanato vā samucchedato vā. Kathaṃ vikkhambhanato kāme parivajjeti?

%\begin{enumerate}
%    \item “Aṭṭhikaṅkalūpamā kāmā appassādaṭṭhenā” ti passanto vikkhambhanato kāme parivajjeti,
%    \item “maṃsapesūpamā kāmā bahusādhāraṇaṭṭhenā” ti passanto vikkhambhanato kāme parivajjeti,
%    \item “tiṇukkūpamā kāmā anudahanaṭṭhenā” ti passanto vikkhambhanato kāme parivajjeti,
%    \item “aṅgārakāsūpamā kāmā mahāpariḷāhaṭṭhenā” ti passanto vikkhambhanato kāme parivajjeti,
%    \item “supinakūpamā kāmā ittarapaccupaṭṭhānaṭṭhenā” ti passanto vikkhambhanato kāme parivajjeti,
%    \item “yācitakūpamā kāmā tāvakālikaṭṭhenā” ti passanto vikkhambhanato kāme parivajjeti,
%    \item “rukkhaphalūpamā kāmā sambhañjanaparibhañjanaṭṭhenā” ti passanto vikkhambhanato kāme parivajjeti,
%    \item “asisūnūpamā kāmā adhikuṭṭanaṭṭhenā” ti passanto vikkhambhanato kāme parivajjeti,
%    \item “sattisūlūpamā kāmā vinivijjhanaṭṭhenā” ti passanto vikkhambhanato kāme parivajjeti,
%    \item “sappasirūpamā kāmā sappaṭibhayaṭṭhenā” ti passanto vikkhambhanato kāme parivajjeti,
%    \item “aggikkhandhūpamā kāmā mahābhitāpanaṭṭhenā” ti passanto vikkhambhanato kāme parivajjeti.
%\end{enumerate}

修习佛随念者由镇伏避开爱欲,修习法随念者⋯修习僧随念者⋯修习戒随念者⋯修习舍随念者⋯修习天随念者⋯修习入出息念者⋯修习死随念者⋯修习身至念者⋯修习寂止随念者由镇伏避开爱欲。

%Buddhānussatiṃ bhāvento pi vikkhambhanato kāme parivajjeti, dhammānussatiṃ bhāvento pi …pe… saṅghānussatiṃ bhāvento pi … sīlānussatiṃ bhāvento pi … cāgānussatiṃ bhāvento pi … devatānussatiṃ bhāvento pi … ānāpānassatiṃ bhāvento pi … maraṇassatiṃ bhāvento pi … kāyagatāsatiṃ bhāvento pi … upasamānussatiṃ bhāvento pi vikkhambhanato kāme parivajjeti.

修习初禅者由镇伏避开爱欲,修习二禅者⋯修习三禅者⋯修习四禅者⋯修习空无边处等至者⋯修习识无边处等至者⋯修习无所有处等至者⋯修习非想非非想处等至者由镇伏避开爱欲。如是,由镇伏避开爱欲。

%Paṭhamaṃ jhānaṃ bhāvento pi vikkhambhanato kāme parivajjeti …pe… dutiyaṃ jhānaṃ bhāvento pi … tatiyaṃ jhānaṃ bhāvento pi … catutthaṃ jhānaṃ bhāvento pi … ākāsānañcāyatanasamāpattiṃ bhāvento pi … viññāṇañcāyatanasamāpattiṃ bhāvento pi … ākiñcaññāyatanasamāpattiṃ bhāvento pi … nevasaññānāsaññāyatanasamāpattiṃ bhāvento pi vikkhambhanato kāme parivajjeti. Evaṃ vikkhambhanato kāme parivajjeti.

如何由正断避开爱欲?修习须陀洹道者由正断避开导向苦趣的爱欲,修习斯陀含道者由正断避开粗重的爱欲,修习阿那含道者由正断避开残余的爱欲,修习阿罗汉道者完全地、于一切处无余地由正断避开爱欲。如是,由正断避开爱欲。

%Kathaṃ samucchedato kāme parivajjeti? Sotāpattimaggaṃ bhāvento pi apāyagamanīye kāme samucchedato parivajjeti, sakadāgāmimaggaṃ bhāvento pi oḷārike kāme samucchedato parivajjeti, anāgāmimaggaṃ bhāvento pi anusahagate kāme samucchedato parivajjeti, arahattamaggaṃ bhāvento pi sabbena sabbaṃ sabbathā sabbaṃ asesaṃ nissesaṃ samucchedato kāme parivajjeti. Evaṃ samucchedato kāme parivajjetī ti— yo kāme parivajjeti.

\textbf{如以足(避开)蛇头}。\textbf{蛇}即毒蛇。以何义为蛇?匍匐而行为蛇,蜿蜒而行为蛇,以腹而行为蛇,低头而行为蛇,以头而眠为蛇,卧于洞穴为蛇,卧于洞窟为蛇,以牙作其武器为蛇,以毒作其恐吓为蛇,以舌作其副二为蛇,以二舌尝味为蛇。好比男子欲生、欲不死,欲乐、厌苦,以足避开、避免、回避、除去蛇头,如是欲乐厌苦者避开、避免、回避、除去爱欲。

%\textbf{Sappasseva padā siro} ti. \textbf{Sappo} vuccati ahi. Kenaṭṭhena sappo? Saṃsappanto gacchatī ti sappo; bhujanto gacchatī ti bhujago; urena gacchatī ti urago; pannasiro gacchatī ti pannago; sirena supatī ti sarīsapo; bile sayatī ti bilāsayo; guhāyaṃ sayatī ti guhāsayo; dāṭhā tassa āvudho ti dāṭhāvudho; visaṃ tassa ghoran ti ghoraviso; jivhā tassa duvidhā ti dvijivho; dvīhi jivhāhi rasaṃ sāyatī ti dvirasaññū. Yathā puriso jīvitukāmo amaritukāmo sukhakāmo dukkhapaṭikkūlo pādena sappasiraṃ vajjeyya vivajjeyya parivajjeyya abhinivajjeyya; evam evaṃ sukhakāmo dukkhapaṭikkūlo kāme vajjeyya vivajjeyya parivajjeyya abhinivajjeyyā ti— sappasseva padā siro.

\textbf{此具念者超越这对世间的爱著}。\textbf{此}即此避开爱欲者。爱著即渴爱。贪染、贪恋、跟随、顺从、喜、喜染,心的贪恋、希望、痴迷、耽著、贪求、遍贪求、执著、沦陷,动摇、幻惑、产生者、制造者、缝合者、罗网、湍流、爱著,缠绕、执著、积累、伴侣、誓愿、导向于有,欲望、欲念、连结、黏著、希求、关联,意欲、意愿、意望,意欲色、意欲声、意欲香、意欲味、意欲触,意欲利养、意欲财富、意欲子嗣、意欲活命,热望、贪婪、欲求善好,贪染非法、贪于非理、欣求,欲爱、有爱、无有爱,色爱、无色爱、灭爱,色爱、声爱、香爱、味爱、触爱、法爱,暴流、轭、系缚、取、障碍、盖、蔽覆、束缚,随烦恼、随眠、缠、蔓、悭贪,苦根、苦因、苦起、魔缚、魔钩、魔境,爱河、爱网、爱纽、爱海、贪⋯不善根。

%\textbf{So’maṃ visattikaṃ loke, sato samativattatī} ti. \textbf{So} ti yo kāme parivajjeti. Visattikā vuccati taṇhā. Yo rāgo sārāgo anunayo anurodho nandī nandirāgo, cittassa sārāgo icchā mucchā ajjhosānaṃ gedho paligedho saṅgo paṅko, ejā māyā janikā sañjananī sibbinī jālinī saritā visattikā, suttaṃ visatā āyūhinī dutiyā paṇidhi bhavanetti, vanaṃ vanatho sandhavo sneho apekkhā paṭibandhu, āsā āsīsanā āsīsitattaṃ, rūpāsā saddāsā gandhāsā rasāsā phoṭṭhabbāsā, lābhāsā dhanāsā puttāsā jīvitāsā, jappā ~~pajappā abhijappā jappanā jappitattaṃ~~ loluppaṃ ~~loluppāyanā loluppāyitattaṃ pucchañjikatā~~ sādhukamyatā, adhammarāgo visamalobho nikanti nikāmanā patthanā pihanā sampatthanā, kāmataṇhā bhavataṇhā vibhavataṇhā, rūpataṇhā arūpataṇhā nirodhataṇhā, rūpataṇhā saddataṇhā gandhataṇhā rasataṇhā phoṭṭhabbataṇhā dhammataṇhā, ogho yogo gantho upādānaṃ āvaraṇaṃ nīvaraṇaṃ chadanaṃ bandhanaṃ, upakkileso anusayo pariyuṭṭhānaṃ latā vevicchaṃ, dukkhamūlaṃ dukkhanidānaṃ dukkhappabhavo mārapāso mārabaḷisaṃ māravisayo, taṇhānadī taṇhājālaṃ taṇhāgaddūlaṃ taṇhāsamuddo abhijjhā ~~lobho~~ akusalamūlaṃ.

\textbf{爱著}。以何义为爱著?抑或,这广大、广布、散布于色声香味触、家族、团体、住所、利养、名闻、称誉、快乐、衣食住药、欲界、色界、无色界、欲有、色有、无色有、想有、无想有、非想非非想有、一蕴有、四蕴有、五蕴有、过去、未来、现在、可得见闻觉知之诸法的渴爱为爱著。

%\textbf{Visattikā} ti. Kenaṭṭhena visattikā? ~~Visatā ti visattikā; visālā ti visattikā; visaṭā ti visattikā; visakkatī ti visattikā; visaṃharatī ti visattikā; visaṃvādikā ti visattikā; visamūlā ti visattikā; visaphalā ti visattikā; visaparibhogo ti visattikā;~~ visālā vā pana sā taṇhā rūpe sadde gandhe rase phoṭṭhabbe, kule gaṇe āvāse lābhe yase, pasaṃsāya sukhe cīvare piṇḍapāte senāsane gilānapaccayabhesajjaparikkhāre, kāmadhātuyā rūpadhātuyā arūpadhātuyā, kāmabhave rūpabhave arūpabhave, saññābhave asaññābhave nevasaññānāsaññābhave, ekavokārabhave catuvokārabhave pañcavokārabhave, atīte anāgate paccuppanne, diṭṭhasutamutaviññātabbesu dhammesu visaṭā vitthatā ti visattikā.

\textbf{世间},即苦趣世间、人世间、天世间,蕴世间、界世间、处世间。\textbf{具念者},即以四种方式具念:于身修习身随观念处而具念,于受……于心……于法修习法随观念处而具念。

%\textbf{Loke} ti apāyaloke manussaloke devaloke, khandhaloke dhātuloke āyatanaloke. \textbf{Sato} ti catūhi kāraṇehi sato— kāye kāyānupassanāsatipaṭṭhānaṃ bhāvento sato, vedanāsu … citte … dhammesu dhammānupassanāsatipaṭṭhānaṃ bhāvento sato.

另以四种方式具念:由避免无念而具念,由已作念所应作之法而具念,由已毁障碍于念之法而具念,由不忘失念相之法而具念。

%Aparehi pi catūhi kāraṇehi sato— asatiparivajjanāya sato, satikaraṇīyānaṃ dhammānaṃ katattā sato, satiparibandhānaṃ dhammānaṃ hatattā sato, satinimittānaṃ dhammānaṃ asammuṭṭhattā sato.

另以四种方式具念:由具足念而具念,由自在于念而具念,由熟习于念而具念,由不衰退于念而具念。

%Aparehi pi catūhi kāraṇehi sato— satiyā samannāgatattā sato, satiyā vasitattā sato, satiyā pāguññatāya sato, satiyā apaccorohaṇatāya sato.

另以四种方式具念:由执著而具念,由惊怖而具念,由寂止而具念,由具足善法而具念。

%Aparehi pi catūhi kāraṇehi sato— sattattā sato, santattā sato, samitattā sato, santadhammasamannāgatattā sato.

由佛随念而具念,由法随念而具念,由僧随念而具念,由戒随念而具念,由舍随念而具念,由天随念而具念,由入出息念而具念,由死随念而具念,由身至念而具念,由寂止随念而具念。

%Buddhānussatiyā sato, dhammānussatiyā sato, saṅghānussatiyā sato, sīlānussatiyā sato, cāgānussatiyā sato, devatānussatiyā sato, ānāpānassatiyā sato, maraṇassatiyā sato, kāyagatāsatiyā sato, upasamānussatiyā sato.

念、随念、忆念,念、忆持、受持、不漂浮、不忘失,念、念根、念力、正念、念觉支、一乘道,这被称为念。具有、具足此念,这被称为具念。

%Yā sati anussati paṭissati sati saraṇatā dhāraṇatā apilāpanatā asammussanatā sati satindriyaṃ satibalaṃ sammāsati satisambojjhaṅgo ekāyanamaggo, ayaṃ vuccati sati. Imāya satiyā upeto hoti ~~samupeto upagato samupagato upapanno samupapanno~~ samannāgato, so vuccati sato.

\textbf{此具念者超越这对世间的爱著}。具念者度过、超过、通过、越过、超越这对世间的爱著。是故世尊说:

%\textbf{So’maṃ visattikaṃ loke, sato samativattatī} ti. ~~Loke vā sā visattikā,~~ loke vā taṃ visattikaṃ sato tarati uttarati patarati samatikkamati vītivattatī ti— so’maṃ visattikaṃ loke, sato samativattati. Tenāha Bhagavā—

\begin{quoting}若避开爱欲,如以足(避开)蛇头,\\此具念者超越这对世间的爱著。\end{quoting}

%\begin{quoting}“Yo kāme parivajjeti, sappasseva padā siro;\\So’maṃ visattikaṃ loke, sato samativattatī” ti\end{quoting}

\section*{776}

\begin{quoting}\textbf{田地、物品、金钱,或牛马、奴仆、\\女人、亲眷等种种爱欲,若人贪求,}\end{quoting}

%\begin{quoting}\textbf{Khettaṃ vatthuṃ hiraññaṃ vā, Gavāssaṃ dāsaporisaṃ;\\Thiyo bandhū puthu kāme, Yo naro anugijjhati.}\end{quoting}

\textbf{田地、物品、金钱}。\textbf{田地},即粳田、稻田、豌豆田、绿豆田、大麦田、小麦田、芝麻田。\textbf{物品},即居家之物、仓库之物、前方之物、后方之物、园林之物、寺庙之物。\textbf{金钱},即指货币。

%\textbf{Khettaṃ vatthuṃ hiraññaṃ vā} ti. \textbf{Khettan} ti sālikkhettaṃ vīhikkhettaṃ muggakkhettaṃ māsakkhettaṃ yavakkhettaṃ godhumakkhettaṃ tilakkhettaṃ. \textbf{Vatthun} ti gharavatthuṃ koṭṭhakavatthuṃ purevatthuṃ pacchāvatthuṃ ārāmavatthuṃ vihāravatthuṃ. \textbf{Hiraññan} ti hiraññaṃ vuccati kahāpaṇo ti— khettaṃ vatthuṃ hiraññaṃ vā.

\textbf{牛马、奴仆}。\textbf{牛},即指牛。\textbf{马},即指畜生等。\textbf{奴},即四种奴:家生奴,以财购得之奴,或自愿沦落为奴,或不愿亦沦为奴。

%\textbf{Gavāssaṃ dāsaporisan} ti. \textbf{Gavan} ti gavā vuccanti. \textbf{Assā} ti pasukādayo vuccanti. \textbf{Dāsā} ti cattāro dāsā— antojātako dāso, dhanakkītako dāso, sāmaṃ vā dāsabyaṃ upeti, akāmako vā dāsavisayaṃ upeti.

\begin{quoting}有些生而即为奴,以财购得亦为奴,\\有些自愿沦为奴,怖畏逼迫亦为奴。\end{quoting}

%\begin{quoting}“Āmāya dāsā pi bhavanti h’eke, dhanena kītā pi bhavanti dāsā;\\Sāmañ ca eke upayanti dāsyaṃ, bhayāpanuṇṇā pi bhavanti dāsā” ti\end{quoting}

\textbf{仆},即三种仆:雇工、工人、佣人。

%\textbf{Purisā} ti tayo purisā— bhatakā, kammakarā, upajīvino ti— gavāssaṃ dāsaporisaṃ.

\textbf{女人、亲眷等种种爱欲}。\textbf{女人},即指女性伴侣。\textbf{亲眷},即四种亲眷:亲戚关系而为亲眷、家族关系而为亲眷、颂诗(学习)关系而为亲眷、职业关系而为亲眷。\textbf{种种爱欲},即许多爱欲,这些种种爱欲即适意之色……适意之触。

%\textbf{Thiyo bandhū puthu kāme} ti. \textbf{Thiyo} ti itthipariggaho vuccati. \textbf{Bandhū} ti cattāro bandhū— ñātibandhavāpi bandhu, gottabandhavāpi bandhu, mantabandhavāpi bandhu, sippabandhavāpi bandhu. \textbf{Puthu kāme} ti bahū kāme. Ete puthu kāmā manāpikā rūpā …pe… manāpikā phoṭṭhabbā ti— thiyo bandhū puthu kāme.

\textbf{若人贪求}。\textbf{贪求},即以烦恼欲,对于物欲贪求、随贪、遍贪。是故世尊说:

%\textbf{Yo naro anugijjhatī} ti. ~~\textbf{Yo} ti yo yādiso yathāyutto yathāvihito yathāpakāro yaṃṭhānappatto yaṃdhammasamannāgato khattiyo vā brāhmaṇo vā vesso vā suddo vā gahaṭṭho vā pabbajito vā devo vā manusso vā. \textbf{Naro} ti satto naro mānavo poso puggalo jīvo jāgu jantu indagu manujo.~~ \textbf{Anugijjhatī} ti kilesakāmena vatthukāmesu gijjhati anugijjhati paligijjhati ~~palibajjhatī~~ ti— yo naro anugijjhati. Tenāha Bhagavā—

\begin{quoting}田地、物品、金钱,或牛马、奴仆、\\女人、亲眷等种种爱欲,若人贪求,\end{quoting}

%\begin{quoting}“Khettaṃ vatthuṃ hiraññaṃ vā, gavāssaṃ dāsaporisaṃ;\\Thiyo bandhū puthu kāme, yo naro anugijjhatī” ti\end{quoting}

\section*{777}

\begin{quoting}\textbf{则诸多无力征服他,诸多危难压迫他,\\随后,苦追随他,如水之于漏船。}\end{quoting}

%\begin{quoting}\textbf{Abalā naṃ balīyanti, maddante naṃ parissayā;\\Tato naṃ dukkham anveti, nāvaṃ bhinnam ivodakaṃ.}\end{quoting}

\textbf{诸多无力征服他}。\textbf{诸多无力},即无力、弱力、少力、少势力、劣、卑劣、劣等、低劣、少量的诸多烦恼。这些烦恼征服、征胜、战胜、淹没、占据、压迫此人,如是即诸多无力征服他。或者说,无力、弱力、少力、少势力、劣、卑劣、劣等、低劣、少量之人,无信力、精进力、念力、定力、慧力、惭力、愧力,而这些烦恼征服、征胜、战胜、淹没、占据、压迫此人,如是即诸多无力征服他。

%\textbf{Abalā naṃ balīyantī} ti. \textbf{Abalā} ti abalā kilesā dubbalā appabalā appathāmakā hīnā nihīnā omakā lāmakā ~~chatukkā~~ parittā. Te kilesā taṃ puggalaṃ sahanti parisahanti abhibhavanti ajjhottharanti pariyādiyanti maddantī ti, evam pi abalā naṃ balīyanti. Atha vā abalaṃ puggalaṃ dubbalaṃ appabalaṃ appathāmakaṃ hīnaṃ nihīnaṃ omakaṃ lāmakaṃ ~~chatukkaṃ~~ parittaṃ, yassa natthi saddhābalaṃ vīriyabalaṃ satibalaṃ samādhibalaṃ paññābalaṃ hiribalaṃ ottappabalaṃ. Te kilesā taṃ puggalaṃ sahanti parisahanti abhibhavanti ajjhottharanti pariyādiyanti maddantī ti— evam pi abalā naṃ balīyantī ti.

\textbf{诸多危难压迫他}。两种危难:显明的危难和隐蔽的危难。什么是\textbf{显明的危难}?狮、虎、豹、熊、鬣狗、狼、水牛、象、蛇、蝎、蜈蚣,或为盗贼、暴徒、已作之业、未作之业,眼病、耳病、鼻病、舌病、身病、头病、耳朵病、口病、齿病,咳、喘、感冒、热病、疟疾,胃病、昏迷、痢疾、腹痛、霍乱,癞、疖、疮、痨、癫痫,癣、疥、疥癣、癞病、疥疮、血胆病,糖尿病、肩病、疹、痔,胆汁所起病、痰所起病、风所起病、和合之病、季节变化所生之病、护理不周所生之病,急性病、业异熟所生之病,寒、暑、饥、渴、粪、尿,或虻、蚊、风、炎、爬虫之触,这些被称为显明的危难。

%\textbf{Maddante naṃ parissayā} ti. Dve parissayā— pākaṭaparissayā ca paṭicchannaparissayā ca. Katame \textbf{pākaṭaparissayā}? Sīhā byagghā dīpī acchā taracchā kokā mahiṃsā hatthī ahī vicchikā satapadī, corā vā assu mānavā vā katakammā vā akatakammā vā, cakkhurogo sotarogo ghānarogo jivhārogo kāyarogo sīsarogo kaṇṇarogo mukharogo dantarogo, kāso sāso pināso ḍāho jaro, kucchirogo mucchā pakkhandikā sūlā visūcikā, kuṭṭhaṃ gaṇḍo kilāso soso apamāro, daddu kaṇḍu kacchu rakhasā vitacchikā lohitapittaṃ, madhumeho aṃsā piḷakā bhagandalā, pittasamuṭṭhānā ābādhā semhasamuṭṭhānā ābādhā vātasamuṭṭhānā ābādhā sannipātikā ābādhā utupariṇāmajā ābādhā visamaparihārajā ābādhā, opakkamikā ābādhā kammavipākajā ābādhā, sītaṃ uṇhaṃ jighacchā pipāsā uccāro passāvo ḍaṃsamakasavātātapasarīsapasamphassā iti vā— ime vuccanti pākaṭaparissayā.

什么是\textbf{隐蔽的危难}?身恶行、语恶行、意恶行,欲贪盖、嗔恚盖、昏沉睡眠盖、掉举恶作盖、疑盖,贪、嗔、痴、忿、恨、覆、恼、嫉、悭,伪善、谄曲、固执、愤激、慢、过慢、㤭、放逸,一切烦恼、一切恶行、一切不安、一切热恼、一切热患、一切不善之行作,这些被称为隐蔽的危难。

%Katame \textbf{paṭicchannaparissayā}? Kāyaduccaritaṃ vacīduccaritaṃ manoduccaritaṃ, kāmacchandanīvaraṇaṃ byāpādanīvaraṇaṃ thinamiddhanīvaraṇaṃ uddhaccakukkuccanīvaraṇaṃ vicikicchānīvaraṇaṃ, rāgo doso moho kodho upanāho makkho paḷāso issā macchariyaṃ, māyā sāṭheyyaṃ thambho sārambho māno atimāno mado pamādo, sabbe kilesā sabbe duccaritā sabbe darathā sabbe pariḷāhā sabbe santāpā sabbākusalābhisaṅkhārā— ime vuccanti paṭicchannaparissayā.

\textbf{危难},以何义为危难?征胜故为危难,导致衰退故为危难,依止于彼故为危难。如何征胜故为危难?那些危难征服、征胜、战胜、淹没、占据、压迫此人,如是即征胜故为危难。

%\textbf{Parissayā} ti kenaṭṭhena parissayā? Parisahantī ti parissayā, parihānāya saṃvattantī ti parissayā, tatrāsayā ti parissayā. Kathaṃ parisahantī ti parissayā? Te parissayā taṃ puggalaṃ sahanti parisahanti abhibhavanti ajjhottharanti pariyādiyanti maddanti, evaṃ parisahantī ti parissayā.

如何导致衰退故为危难?那些危难导致善法的妨难、衰退。哪些善法?正行道、随顺行道、不逆行道、不相违行道、随顺义利行道、法次法行道,于诸戒能使圆满,于诸根守护其门,于食知量,践行醒觉,念正知,践行四念处的修习,践行四正勤的修习,践行四神足的修习,践行五根的修习,践行五力的修习,践行七觉支的修习,践行八支圣道的修习,导致这些善法的妨难、衰退。如是即导致衰退故为危难。

%Kathaṃ parihānāya saṃvattantī ti parissayā? Te parissayā kusalānaṃ dhammānaṃ antarāyāya parihānāya saṃvattanti. Katamesaṃ kusalānaṃ dhammānaṃ? Sammāpaṭipadāya anulomapaṭipadāya apaccanīkapaṭipadāya aviruddhapaṭipadāya anvatthapaṭipadāya dhammānudhammapaṭipadāya, sīlesu paripūrikāritāya indriyesu guttadvāratāya bhojane mattaññutāya, jāgariyānuyogassa satisampajaññassa, catunnaṃ satipaṭṭhānānaṃ bhāvanānuyogassa catunnaṃ sammappadhānānaṃ bhāvanānuyogassa catunnaṃ iddhipādānaṃ bhāvanānuyogassa, pañcannaṃ indriyānaṃ bhāvanānuyogassa pañcannaṃ balānaṃ bhāvanānuyogassa, sattannaṃ bojjhaṅgānaṃ bhāvanānuyogassa, ariyassa aṭṭhaṅgikassa maggassa bhāvanānuyogassa— imesaṃ kusalānaṃ dhammānaṃ antarāyāya parihānāya saṃvattanti, evaṃ parihānāya saṃvattantī ti parissayā.

如何依止于彼故为危难?于此,这些恶不善法依止于自性生起。好比穴居的生物睡在洞穴,水栖的生物睡在水中,林居的生物睡在林中,树棲的生物睡在树上,如是,于此,这些恶不善法依止于自性生起,如是即依止于彼故为危难。

%Kathaṃ tatrāsayā ti parissayā? Tatth’ete pāpakā akusalā dhammā uppajjanti attabhāvasannissayā. Yathā bile bilāsayā pāṇā sayanti, dake dakāsayā pāṇā sayanti, vane vanāsayā pāṇā sayanti, rukkhe rukkhāsayā pāṇā sayanti, evam evaṃ tatth’ete pāpakā akusalā dhammā uppajjanti attabhāvasannissayā, evam pi tatrāsayā ti parissayā.

因此世尊说:\footnote{所引即相应部·六入处相应·内住者经。\textbf{有内住者、有出入者} \textit{sāntevāsiko, sācariyako} 也可作「有弟子者、有阿阇黎者」,这里联系下文的「住于其内 \textit{anto vasanti}、出入其中 \textit{samudācaranti}」来翻译,菩提比丘对此亦有注释。}

%Vuttañ h’etaṃ Bhagavatā—

\begin{quoting}诸比丘!有内住者、有出入者的比丘苦住、不安住。诸比丘!如何有内住者、有出入者的比丘苦住、不安住?诸比丘!于此,对于比丘,以眼见色已,恶不善法、念与思惟之结缚生起。它们住于其内,恶不善法漏入其内,所以被称为有内住者。它们出入其中,恶不善法出入其中,所以被称为有出入者。\end{quoting}

%\begin{quoting}“Sāntevāsiko, bhikkhave, bhikkhu sācariyako dukkhaṃ na phāsu viharati. Kathañ ca, bhikkhave, bhikkhu sāntevāsiko sācariyako dukkhaṃ na phāsu viharati? Idha, bhikkhave, bhikkhuno cakkhunā rūpaṃ disvā uppajjanti ye pāpakā akusalā dhammā sarasaṅkappā saṃyojaniyā, tyassa anto vasanti \underline{anvāsavanti} pāpakā akusalā dhammā ti— tasmā sāntevāsiko ti vuccati. Te naṃ samudācaranti, samudācaranti naṃ pāpakā akusalā dhammā ti— tasmā sācariyako ti vuccati.\end{quoting}

\begin{quoting}复次,诸比丘!对于比丘,以耳闻声已,以鼻嗅香已,以舌尝味已,以身触触已,以意识法已,恶不善法、念与思惟之结缚生起。它们住于其内,恶不善法漏入其内,所以被称为有内住者。它们出入其中,恶不善法出入其中,所以被称为有出入者。如是,诸比丘!有内住者、有出入者的比丘苦住、不安住。\end{quoting}

%\begin{quoting}Puna c’aparaṃ, bhikkhave, bhikkhuno sotena saddaṃ sutvā, ghānena gandhaṃ ghāyitvā, jivhāya rasaṃ sāyitvā, kāyena phoṭṭhabbaṃ phusitvā, manasā dhammaṃ viññāya uppajjanti ye pāpakā akusalā dhammā sarasaṅkappā saṃyojaniyā, tyassa anto vasanti \underline{anvāsavanti} pāpakā akusalā dhammā ti— tasmā sāntevāsiko ti vuccati. Te naṃ samudācaranti, samudācaranti naṃ pāpakā akusalā dhammā ti— tasmā sācariyako ti vuccati. Evaṃ kho, bhikkhave, bhikkhu sāntevāsiko sācariyako dukkhaṃ na phāsu viharatī” ti.\end{quoting}

如是即依止于彼故为危难。因此世尊说:\footnote{所引即如是语·三集·第四品·内垢经。}

%Evam pi tatrāsayā ti parissayā. Vuttañ h’etaṃ Bhagavatā—

\begin{quoting}诸比丘!这三种是内垢、内在非友、内在敌手、内在凶手、内在怨敌。哪三种?诸比丘!贪是内垢、内在非友、内在敌手、内在凶手、内在怨敌。诸比丘!嗔……痴是内垢、内在非友、内在敌手、内在凶手、内在怨敌。诸比丘!这三种是内垢、内在非友、内在敌手、内在凶手、内在怨敌。\end{quoting}

%\begin{quoting}“Tayo’me, bhikkhave, antarāmalā antarāamittā antarāsapattā antarāvadhakā antarāpaccatthikā. Katame tayo? Lobho, bhikkhave, antarāmalo antarāamitto antarāsapatto antarāvadhako antarāpaccatthiko. Doso …pe… moho, bhikkhave, antarāmalo antarāamitto antarāsapatto antarāvadhako antarāpaccatthiko. Ime kho, bhikkhave, tayo antarāmalā antarāamittā antarāsapattā antarāvadhakā antarāpaccatthikā.\end{quoting}

\begin{quoting}
    \begin{quoting}
        贪是不利所生,贪是心的动摇,\\
        而人不了悟这从内所生的怖畏。

        贪人不晓义利,贪人未得见法,\\
        贪所征服之人,随后即成暗冥。

        %Anatthajanano lobho, lobho cittappakopano;\\
        %Bhayam antarato jātaṃ, taṃ jano nāvabujjhati.

        %Luddho atthaṃ na jānāti, luddho dhammaṃ na passati;\\
        %Andhantamaṃ tadā hoti, yaṃ lobho sahate naraṃ.

        嗔是不利所生,嗔是心的动摇,\\
        而人不了悟这从内所生的怖畏。

        嗔人不晓义利,嗔人未得见法,\\
        嗔所征服之人,随后即成暗冥。

        %Anatthajanano doso, doso cittappakopano;\\
        %Bhayam antarato jātaṃ, taṃ jano nāvabujjhati.

        %Kuddho atthaṃ na jānāti, kuddho dhammaṃ na passati;\\
        %Andhantamaṃ tadā hoti, yaṃ doso sahate naraṃ.

        痴是不利所生,痴是心的动摇,\\
        而人不了悟这从内所生的怖畏。

        痴人不晓义利,痴人未得见法,\\
        痴所征服之人,随后即成暗冥。

        %Anatthajanano moho, moho cittappakopano;\\
        %Bhayam antarato jātaṃ, taṃ jano nāvabujjhati.

        %Mūḷho atthaṃ na jānāti, mūḷho dhammaṃ na passati;\\
        %Andhantamaṃ tadā hoti, yaṃ moho sahate naran” ti
    \end{quoting}
\end{quoting}

如是即依止于彼故为危难。因此世尊又说:\footnote{所引即相应部·㤭萨罗相应·男子经。}

%Evam pi tatrāsayā ti parissayā. Vuttam pi h’etaṃ Bhagavatā—

\begin{quoting}大王!对于男子,有三种法生起于内在,生起后导致不利、苦、不安住。哪三种?大王!对于男子,贪生起于内在,生起后导致不利、苦、不安住。大王!对于男子,嗔……痴生起于内在,生起后导致不利、苦、不安住。大王!对于男子,这三种法生起于内在,生起后导致不利、苦、不安住。\end{quoting}

%\begin{quoting}“Tayo kho, mahārāja, purisassa dhammā ajjhattaṃ uppajjamānā uppajjanti, ahitāya dukkhāya aphāsuvihārāya. Katame tayo? Lobho kho, mahārāja, purisassa dhammo ajjhattaṃ uppajjamāno uppajjati, ahitāya dukkhāya aphāsuvihārāya. Doso kho, mahārāja …pe… moho kho, mahārāja, purisassa dhammo ajjhattaṃ uppajjamāno uppajjati, ahitāya dukkhāya aphāsuvihārāya. Ime kho, mahārāja, tayo purisassa dhammā ajjhattaṃ uppajjamānā uppajjanti, ahitāya dukkhāya aphāsuvihārāya.\end{quoting}

\begin{quoting}
    \begin{quoting}
        贪、嗔、痴,从自我生起,伤害\\
        恶心的男子,好比果实之于竹芦。

        %Lobho doso ca moho ca, purisaṃ pāpacetasaṃ;\\
        %Hiṃsanti attasambhūtā, tacasāraṃ va samphalan” ti
    \end{quoting}
\end{quoting}

如是即依止于彼故为危难。因此世尊又说:\footnote{所引即经集·针毛经·第 274 颂。}

%Evam pi tatrāsayā ti parissayā. Vuttam pi c’etaṃ Bhagavatā—

\begin{quoting}贪嗔以此为因,乐、不乐与汗毛竖立从此而生,\\寻从此起(驱散)意,如同儿童驱散乌鸦。\end{quoting}

%\begin{quoting}“Rāgo ca doso ca itonidānā, arati rati lomahaṃso itojā;\\Ito samuṭṭhāya mano vitakkā, kumārakā dhaṅkam iv’ossajantī” ti\end{quoting}

如是即依止于彼故为危难。

%Evam pi tatrāsayā ti parissayā.

\textbf{诸多危难压迫他}。这些危难征服、征胜、战胜、淹没、占据、压迫此人。

%\textbf{Maddante naṃ parissayā} ti. Te parissayā taṃ puggalaṃ sahanti parisahanti abhibhavanti ajjhottharanti pariyādiyanti maddantī ti— maddante naṃ parissayā.

\textbf{随后,苦追随他}。\textbf{随后},即危难之后,苦追随、跟随此人而为随从,

\begin{enumerate}
    \item 生苦、病苦、死苦、忧悲苦恼愁之苦……,
    \item 地狱之苦、畜生界之苦、饿鬼境之苦……,
    \item 为人之苦、入胎为因之苦、怀胎为因之苦、分娩为因之苦、出生有关之苦、出生听命于人之苦、自我残害之苦、他人侵犯之苦……,
    \item 苦苦、行苦、坏苦……,
    \item 眼病……虻、蚊、风、炎、爬虫之触,
    \item 丧母之苦、丧父之苦、丧兄弟之苦、丧姊妹之苦、丧子之苦、丧女之苦、亲戚厄难之苦、财物损失之苦、疾病衰损之苦、破戒之苦、破见之苦追随、跟随而为随从。
\end{enumerate}

%\textbf{Tato naṃ dukkham anvetī} ti. \textbf{Tato} ti tato tato parissayato taṃ puggalaṃ dukkhaṃ anveti anugacchati anvāyikaṃ hoti,

%\begin{enumerate}
%    \item jātidukkhaṃ anveti anugacchati anvāyikaṃ hoti, jarādukkhaṃ anveti anugacchati anvāyikaṃ hoti, byādhidukkhaṃ anveti anugacchati anvāyikaṃ hoti, maraṇadukkhaṃ anveti anugacchati anvāyikaṃ hoti, sokaparidevadukkhadomanassupāyāsadukkhaṃ anveti anugacchati anvāyikaṃ hoti,
%    \item nerayikaṃ dukkhaṃ, tiracchānayonikaṃ dukkhaṃ, pettivisayikaṃ dukkhaṃ anveti anugacchati anvāyikaṃ hoti,
%    \item mānusikaṃ dukkhaṃ … gabbhokkantimūlakaṃ dukkhaṃ … gabbhe ṭhitimūlakaṃ dukkhaṃ … gabbhā vuṭṭhānamūlakaṃ dukkhaṃ … jātassūpanibandhakaṃ dukkhaṃ … jātassa parādheyyakaṃ dukkhaṃ … attūpakkamaṃ dukkhaṃ … parūpakkamaṃ dukkhaṃ anveti anugacchati anvāyikaṃ hoti,
%    \item dukkhadukkhaṃ anveti anugacchati anvāyikaṃ hoti, saṅkhāradukkhaṃ … vipariṇāmadukkhaṃ …
%    \item cakkhurogo sotarogo ghānarogo jivhārogo kāyarogo sīsarogo kaṇṇarogo mukharogo dantarogo, kāso sāso pināso ḍāho jaro, kucchirogo mucchā pakkhandikā sūlā visūcikā, kuṭṭhaṃ gaṇḍo kilāso soso apamāro, daddu kaṇḍu kacchu rakhasā vitacchikā lohitapittaṃ, madhumeho aṃsā piḷakā bhagandalā pittasamuṭṭhānā ābādhā semhasamuṭṭhānā ābādhā vātasamuṭṭhānā ābādhā sannipātikā ābādhā utupariṇāmajā ābādhā visamaparihārajā ābādhā, opakkamikā ābādhā kammavipākajā ābādhā, sītaṃ uṇhaṃ jighacchā pipāsā uccāro passāvo ḍaṃsamakasavātātapasarīsapasamphassadukkhaṃ …
%    \item mātumaraṇaṃ dukkhaṃ … pitumaraṇaṃ dukkhaṃ … bhātumaraṇaṃ dukkhaṃ … bhaginimaraṇaṃ dukkhaṃ … puttamaraṇaṃ dukkhaṃ … dhītumaraṇaṃ dukkhaṃ … ñātibyasanaṃ dukkhaṃ … bhogabyasanaṃ dukkhaṃ … rogabyasanaṃ dukkhaṃ … sīlabyasanaṃ dukkhaṃ … diṭṭhibyasanaṃ dukkhaṃ anveti anugacchati anvāyikaṃ hotī ti— tato naṃ dukkham anveti.
%\end{enumerate}

\textbf{如水之于漏船}。好比入水的漏船,随后,水追随、跟随而为随从,水从前方、后方、下方、侧方追随、跟随而为随从,如是在危难之后,苦追随、跟随此人而为随从,生苦……破见之苦追随、跟随而为随从。是故世尊说:

%\textbf{Nāvaṃ bhinnam ivodakan} ti. Yathā bhinnaṃ nāvaṃ dakam esiṃ tato tato udakaṃ anveti anugacchati anvāyikaṃ hoti, purato pi udakaṃ anveti anugacchati anvāyikaṃ hoti, pacchato pi … heṭṭhato pi … passato pi udakaṃ anveti anugacchati anvāyikaṃ hoti; evam evaṃ tato tato parissayato taṃ puggalaṃ dukkhaṃ anveti anugacchati anvāyikaṃ hoti, jātidukkhaṃ anveti anugacchati anvāyikaṃ hoti …pe… diṭṭhibyasanaṃ dukkhaṃ anveti anugacchati anvāyikaṃ hotī ti— nāvaṃ bhinnam ivodakaṃ. Tenāha Bhagavā—

\begin{quoting}则诸多无力征服他,诸多危难压迫他,\\随后,苦追随他,如水之于漏船。\end{quoting}

%\begin{quoting}“Abalā naṃ balīyanti, maddante naṃ parissayā;\\Tato naṃ dukkham anveti, nāvaṃ bhinnam ivodakan” ti\end{quoting}

\section*{778}

\begin{quoting}\textbf{所以,常常具念之人应避开爱欲,\\舍弃了它们,则能度过暴流,如汲水出船,到达彼岸。}\end{quoting}

%\begin{quoting}\textbf{Tasmā jantu sadā sato, kāmāni parivajjaye;\\Te pahāya tare oghaṃ, nāvaṃ sitvā va pāragū.}\end{quoting}

\textbf{所以,常常具念之人}。\textbf{所以},即以此为由、以此为因、以此为缘、以此为因缘,看到了爱欲中的过患。\textbf{人},即有情……摩奴所生者。\textbf{常常},即饭前、饭后、前夜、中夜、后夜,黑月、白月、雨季、冬季、热季,寿蕴前期、寿蕴中期、寿蕴后期。\textbf{具念},即以四种方式具念:于身修习身随观念处而具念,于受……于心……于法修习法随观念处而具念。另以四种方式具念……这被称为具念。

%\textbf{Tasmā jantu sadā sato} ti. \textbf{Tasmā} ti tasmā taṅkāraṇā taṃhetu tappaccayā taṃnidānā etaṃ ādīnavaṃ sampassamāno kāmesū ti— tasmā. \textbf{Jantū} ti satto naro mānavo poso puggalo jīvo jāgu jantu indagu manujo. \textbf{Sadā} ti ~~sadā sabbadā sabbakālaṃ niccakālaṃ dhuvakālaṃ satataṃ samitaṃ abbokiṇṇaṃ poṅkhānupoṅkhaṃ udakūmikajātaṃ avīci santati sahitaṃ phassitaṃ,~~ purebhattaṃ pacchābhattaṃ purimayāmaṃ majjhimayāmaṃ pacchimayāmaṃ, kāḷe juṇhe vasse hemante gimhe, purime vayokhandhe majjhime vayokhandhe pacchime vayokhandhe. \textbf{Sato} ti catūhi kāraṇehi sato— kāye kāyānupassanāsatipaṭṭhānaṃ bhāvento sato, vedanāsu … citte … dhammesu dhammānupassanāsatipaṭṭhānaṃ bhāvento sato. Aparehi catūhi kāraṇehi sato …pe… so vuccati sato ti— tasmā jantu sadā sato.

\textbf{应避开爱欲}。\textbf{爱欲},总的来说有两种爱欲:物欲及烦恼欲……。\textbf{应避开爱欲},即通过两种途径避开爱欲,或由镇伏,或由正断……。

%\textbf{Kāmāni parivajjaye} ti. \textbf{Kāmā} ti uddānato dve kāmā— vatthukāmā ca kilesakāmā ca …pe… ime vuccanti vatthukāmā …pe… ime vuccanti kilesakāmā. \textbf{Kāmāni parivajjaye} ti dvīhi kāraṇehi kāme parivajjeyya— vikkhambhanato vā samucchedato vā. Kathaṃ vikkhambhanato kāme parivajjeyya? “Aṭṭhikaṅkalūpamā kāmā appassādaṭṭhenā” ti passanto vikkhambhanato kāme parivajjeyya, “maṃsapesūpamā kāmā bahusādhāraṇaṭṭhenā” ti passanto vikkhambhanato kāme parivajjeyya, “tiṇukkūpamā kāmā anudahanaṭṭhenā” ti passanto vikkhambhanato kāme parivajjeyya …pe… nevasaññānāsaññāyatanasamāpattiṃ bhāvento vikkhambhanato kāme parivajjeyya. Evaṃ vikkhambhanato kāme parivajjeyya …pe… evaṃ samucchedato kāme parivajjeyyā ti— kāmāni parivajjaye.

\textbf{舍弃了它们,则能度过暴流}。遍知了这些物欲,舍弃、除遣、去除了烦恼欲而成空无,舍弃、除遣、去除了欲贪盖、嗔恚盖、昏沉睡眠盖、掉举恶作盖、疑盖而成空无,则能度过欲流、有流、见流、无明流。

%\textbf{Te pahāya tare oghan} ti. \textbf{Te} ti vatthukāme parijānitvā kilesakāme pahāya pajahitvā vinodetvā byantiṃ karitvā anabhāvaṃ gamitvā; kāmacchandanīvaraṇaṃ pahāya pajahitvā vinodetvā byantiṃ karitvā anabhāvaṃ gamitvā; byāpādanīvaraṇaṃ …pe… thinamiddhanīvaraṇaṃ … uddhaccakukkuccanīvaraṇaṃ … vicikicchānīvaraṇaṃ pahāya pajahitvā vinodetvā byantiṃ karitvā anabhāvaṃ gamitvā kāmoghaṃ bhavoghaṃ diṭṭhoghaṃ avijjoghaṃ tareyya ~~uttareyya patareyya samatikkameyya vītivatteyyā~~ ti— te pahāya tare oghaṃ.

\textbf{如汲水出船,到达彼岸}。好比汲出、舀出、抛开了重船中所载之水,能以轻快之船快速、轻便、容易地去到对岸,如是遍知了这些物欲,舍弃、除遣、去除了烦恼欲而成空无,舍弃、除遣、去除了欲贪盖、嗔恚盖、昏沉睡眠盖、掉举恶作盖、疑盖而成空无,能快速、轻便、容易地去到对岸。\textbf{对岸}指不死、涅槃,即一切行的止息、一切所依的舍遣、爱尽、离贪、灭、涅槃。\textbf{去到对岸},即到达对岸、触到对岸、证得对岸。\textbf{到达彼岸},欲往对岸者为到达彼岸,去向对岸者为到达彼岸,已到对岸者为到达彼岸。

%\textbf{Nāvaṃ sitvā va pāragū} ti. Yathā garukaṃ nāvaṃ bhārikaṃ udakaṃ sitvā osiñcitvā chaḍḍetvā lahukāya nāvāya khippaṃ lahuṃ appakasiren’eva pāraṃ gaccheyya; evam evaṃ vatthukāme parijānitvā kilesakāme pahāya pajahitvā vinodetvā byantiṃ karitvā anabhāvaṃ gamitvā; kāmacchandanīvaraṇaṃ … byāpādanīvaraṇaṃ … thinamiddhanīvaraṇaṃ … uddhaccakukkuccanīvaraṇaṃ … vicikicchānīvaraṇaṃ pahāya pajahitvā vinodetvā byantiṃ karitvā anabhāvaṃ gamitvā khippaṃ lahuṃ appakasiren’eva pāraṃ gaccheyya. \textbf{Pāraṃ} vuccati amataṃ nibbānaṃ. Yo so sabbasaṅkhārasamatho sabbūpadhipaṭinissaggo taṇhakkhayo virāgo nirodho nibbānaṃ. \textbf{Pāraṃ gaccheyyā} ti pāraṃ adhigaccheyya, pāraṃ phuseyya, pāraṃ sacchikareyya. \textbf{Pāragū} ti yo pi pāraṃ gantukāmo so pi pāragū; yo pi pāraṃ gacchati so pi pāragū; yo pi pāraṅgato, so pi pāragū.

因此世尊又说:

%Vuttam pi h’etaṃ Bhagavatā—

\begin{quoting}诸比丘!婆罗门已度、去到对岸、立于高地,此即阿罗汉的同义语。他以证知到达彼岸,以遍知到达彼岸,以舍弃到达彼岸,以修习到达彼岸,以证得到达彼岸,以等至到达彼岸。以证知一切法到达彼岸,以遍知一切苦到达彼岸,以舍弃一切烦恼到达彼岸,以修习四圣道到达彼岸,以证得灭到达彼岸,以等至一切等至到达彼岸。\end{quoting}

%\begin{quoting}“Tiṇṇo pāraṅgato thale tiṭṭhati brāhmaṇo ti kho, bhikkhave, arahato etaṃ adhivacanaṃ. So abhiññāpāragū pariññāpāragū pahānapāragū bhāvanāpāragū sacchikiriyāpāragū samāpattipāragū. Abhiññāpāragū sabbadhammānaṃ, pariññāpāragū sabbadukkhānaṃ, pahānapāragū sabbakilesānaṃ, bhāvanāpāragū catunnaṃ ariyamaggānaṃ, sacchikiriyāpāragū nirodhassa, samāpattipāragū sabbasamāpattīnaṃ.\end{quoting}

\begin{quoting}他于圣戒得达自在、得达圆满,于圣定得达自在、得达圆满,于圣慧得达自在、得达圆满,于圣解脱得达自在、得达圆满。\end{quoting}

%\begin{quoting}So vasippatto pāramippatto ariyasmiṃ sīlasmiṃ, vasippatto pāramippatto ariyasmiṃ samādhismiṃ, vasippatto pāramippatto ariyāya paññāya, vasippatto pāramippatto ariyāya vimuttiyā.\end{quoting}

\begin{quoting}他已到彼岸、得达彼岸,已到边际、得达边际,已到顶点、得达顶点,已到边界、得达边界,已到终点、得达终点,已到救护、得达救护,已到庇护、得达庇护,已到皈依、得达皈依,已到无畏、得达无畏,已到不灭、得达不灭,已到不死、得达不死,已到涅槃、得达涅槃。\end{quoting}

%\begin{quoting}So pāraṅgato pārappatto antagato antappatto koṭigato koṭippatto pariyantagato pariyantappatto vosānagato vosānappatto tāṇagato tāṇappatto leṇagato leṇappatto saraṇagato saraṇappatto abhayagato abhayappatto accutagato accutappatto amatagato amatappatto nibbānagato nibbānappatto.\end{quoting}

\begin{quoting}他成满而住、习练而行、已尽旅途、已尽怨敌、已到终点、守护梵行、达最上见、已修习道、舍弃烦恼、通达不动、已证得灭,他已遍知苦、已舍断集、已修习道、已证得灭,已证知应证知者、已遍知应遍知者、已舍断应舍断者、已修习应修习者、已证得应证得者。\end{quoting}

%\begin{quoting}So vuṭṭhavāso ciṇṇacaraṇo gataddho gatadiso gatakoṭiko pālitabrahmacariyo uttamadiṭṭhippatto bhāvitamaggo pahīnakileso paṭividdhākuppo sacchikatanirodho, dukkhaṃ tassa pariññātaṃ, samudayo pahīno, maggo bhāvito, nirodho sacchikato, abhiññeyyaṃ abhiññātaṃ, pariññeyyaṃ pariññātaṃ, pahātabbaṃ pahīnaṃ, bhāvetabbaṃ bhāvitaṃ, sacchikātabbaṃ sacchikataṃ.\end{quoting}

\begin{quoting}他已除去障碍、填满沟堑、拔出门柱、无有闩塞、圣、降下旗帜、落下荷担、离于轭缚、舍断五支、具足六支、守护于一、依倚于四、除遣各别的谛、终结邪求、思惟无浊、身行轻安、心善解脱、慧善解脱、整全、已立、最高人、最上人、得最上利。\end{quoting}

%\begin{quoting}So ukkhittapaligho saṅkiṇṇaparikkho abbuḷhesiko niraggaḷo ariyo pannaddhajo pannabhāro visaññutto pañcaṅgavippahīno chaḷaṅgasamannāgato ekārakkho caturāpasseno panuṇṇapaccekasacco samavayasaṭṭhesano anāvilasaṅkappo passaddhakāyasaṅkhāro suvimuttacitto suvimuttapañño kevalī vusitavā uttamapuriso paramapuriso paramapattippatto.\end{quoting}

\begin{quoting}他既不积集,也不舍离,舍离已而立,既不舍弃,也不执取,舍弃已而立,既不纠缠,也不亲近,离开已而立,既不熏散,也不聚集,熏散已而立。具足无学之戒蕴而立,具足无学之定蕴、无学之慧蕴、无学之解脱蕴、无学之解脱智见蕴而立。正行道于谛已而立,超越动摇已而立,完全熄灭烦恼之火已而立,无所行而立,受持已作已而立,受用解脱而立,以慈、悲、喜、舍遍净而立,极遍净而立,由无渴爱遍净而立,由解脱而立,由满足而立,立于蕴之边界,立于界之边界,立于处之边界,立于趣之边界,立于转生之边界,立于结生之边界,立于有之边界,立于轮回之边界,立于轮转之边界,立于最后之有,立于最后之积集,立于最后之身,为阿罗汉。\end{quoting}

%\begin{quoting}So nevācinati nāpacinati, apacinitvā ṭhito. Neva pajahati na upādiyati, pajahitvā ṭhito. Neva saṃsibbati na ussineti, visinitvā ṭhito. Neva vidhūpeti na sandhūpeti, vidhūpetvā ṭhito. Asekkhena sīlakkhandhena samannāgatattā ṭhito. Asekkhena samādhikkhandhena … asekkhena paññākkhandhena … asekkhena vimuttikkhandhena … asekkhena vimuttiñāṇadassanakkhandhena samannāgatattā ṭhito. Saccaṃ sampaṭipādiyitvā ṭhito. Ejaṃ samatikkamitvā ṭhito. Kilesaggiṃ pariyādiyitvā ṭhito, aparigamanatāya ṭhito, kaṭaṃ samādāya ṭhito, muttipaṭisevanatāya ṭhito, mettāya pārisuddhiyā ṭhito, karuṇāya … muditāya … upekkhāya pārisuddhiyā ṭhito, accantapārisuddhiyā ṭhito, atammayatāya pārisuddhiyā ṭhito, vimuttattā ṭhito, santussitattā ṭhito, khandhapariyante ṭhito, dhātupariyante ṭhito, āyatanapariyante ṭhito, gatipariyante ṭhito, upapattipariyante ṭhito, paṭisandhipariyante ṭhito, bhavapariyante ṭhito, saṃsārapariyante ṭhito, vaṭṭapariyante ṭhito, antime bhave ṭhito, antime samussaye ṭhito, antimadehadharo arahā.\end{quoting}

\begin{quoting}
    \begin{quoting}
        对他,这是最后之有,这是最后之积集,\\
        对他,已无生死的轮回,已无再有。

        %Tassāyaṃ pacchimako bhavo, carimoyaṃ samussayo;\\
        %Jātimaraṇasaṃsāro, natthi tassa punabbhavo” ti.
    \end{quoting}
\end{quoting}

是故世尊说:

%Nāvaṃ sitvā va pāragū ti. Tenāha Bhagavā—

\begin{quoting}所以,常常具念之人应避开爱欲,\\舍弃了它们,则能度过暴流,如汲水出船,到达彼岸。\end{quoting}

%\begin{quoting}“Tasmā jantu sadā sato, kāmāni parivajjaye;\\Te pahāya tare oghaṃ, nāvaṃ sitvā va pāragū” ti\end{quoting}

\begin{center}\vspace{2em}
    爱欲经义释第一\\
    Kāmasuttaniddeso paṭhamo.
\end{center}