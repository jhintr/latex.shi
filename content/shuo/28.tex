\chapter{黜免第二十八}

\subsection*{1}

\textbf{諸葛厷在西朝,少有清譽,為王夷甫所重,時論亦以擬王,後為繼母族黨所讒,誣之為狂逆,將遠徙,友人王夷甫之徒詣檻車與別,厷問:「朝廷何以徙我?」王曰:「言卿狂逆。」厷曰:「逆則應殺,狂何所徙?」}{\footnotesize 厷已見。}

\subsection*{2}

\textbf{桓公入蜀,至三峽中,部伍中有得猿子者,}{\footnotesize \textbf{荊州記}曰峽長七百里,兩岸連山,略無絕處,重巖疊障,隱天蔽日,常有高猿長嘯,屬引清遠,漁者歌曰「巴東三峽巫峽長,猿鳴三聲淚沾裳」。}\textbf{其母緣岸哀號,行百餘里不去,遂跳上船,至便即絕,破視其腹中,腸皆寸寸斷,公聞之,怒,命黜其人。}

\subsection*{3}

\textbf{殷中軍被廢,在信安,終日恆書空作字,揚州吏民尋義逐之,竊視,唯作「咄咄怪事」四字而已。}{\footnotesize \textbf{晉陽秋}曰初,浩以中軍將軍鎮壽陽,羌姚襄上書歸降,後有罪,浩陰圖誅之,會關中有變,苻健死,浩偽率軍而行,云「修復山陵」,襄前驅,恐,遂反,軍至山桑,聞襄將至,棄輜重馳保譙,襄至,據山桑,焚其舟實,至壽陽,略流民而還,浩士卒多叛,征西溫乃上表黜浩,撫軍大將軍奏免浩,除名為民,浩馳還謝罪,既而遷于東陽信安縣。}

\subsection*{4}

\textbf{桓公坐有參軍椅烝薤不時解,共食者又不助,而椅終不放,舉坐皆笑,桓公曰:「同盤尚不相助,況復危難乎?」敕令免官。}

\subsection*{5}

\textbf{殷中軍廢後,恨簡文曰:「上人著百尺樓上,儋梯將去。」}{\footnotesize \textbf{續晉陽秋}曰浩雖廢黜,夷神委命,雅詠不輟,雖家人不見其有流放之戚,外生韓伯始隨至徙所,周年還都,浩素愛之,送至水側,乃詠曹顏遠詩曰「富貴它人合,貧賤親戚離」,因泣下。其悲見于外者,唯此一事而已,則書空、去梯之言,未必皆實也。}

\subsection*{6}

\textbf{鄧竟陵免官後赴山陵,過見大司馬桓公,公問之曰:「卿何以更瘦?」}{\footnotesize \textbf{大司馬寮屬名}曰鄧遐,字應玄,陳郡人,平南將軍岳之子,勇力絕人,氣蓋當世,時人方之樊噲,為桓溫參軍,數從溫征伐,歷竟陵太守,枋頭之役,溫既懷恥忿,且憚遐,因免遐官,病卒。}\textbf{鄧曰:「有愧於叔達,不能不恨於破甑。」}{\footnotesize \textbf{郭林宗別傳}曰鉅鹿孟敏,字叔達,敦朴質直,客居太原,雜處凡俗,未有所名,嘗至市買甑,荷儋墮地壞之,徑去不顧,適遇林宗,見而異之,因問曰「壞甑可惜,何以不顧」,客曰「甑既已破,視之何益」,林宗賞其介決,因以知其德性,謂必為美士,勸令讀書,遊學十年,遂知名,三府並辟,不就,東夏以為美賢。}

\subsection*{7}

\textbf{桓宣武既廢太宰父子,仍上表曰:「應割近情,以存遠計,若除太宰父子,可無後憂。」簡文手答表曰:「所不忍言,況過於言?」宣武又重表,辭轉苦切,簡文更答曰:「若晉室靈長,明公便宜奉行此詔,如大運去矣,請避賢路。」桓公讀詔,手戰流汗,於此乃止,太宰父子,遠徙新安。}{\footnotesize \textbf{司馬晞傳}曰晞,字道升,元帝第四子,初封武陵王,拜太宰,少不好學,尚武凶恣,時太宗輔政,晞以宗長不得執權,常懷憤慨,欲因桓溫入朝殺之,太宗即位,新蔡王晃首辭,引與晞及子綜謀逆,有司奏晞等斬刑,詔原之,徙新安,晞未敗四五年中,喜為挽歌,自搖大鈴,使左右習和之,又燕會,使人作新安人歌舞離別之辭,其聲甚悲,後果徙新安。}

\subsection*{8}

\textbf{桓玄敗後,殷仲文還為大司馬咨議,意似二三,非復往日,大司馬府聽前有一老槐,甚扶疎,殷因月朔,與眾在聽,視槐良久,歎曰:「槐樹婆娑,無復生意。」}{\footnotesize \textbf{晉安帝紀}曰桓玄敗,殷仲文歸京師,高祖以其衛從二后,且以大信宣令,引為鎮軍長史,自以名輩先達,位遇至重,而後來謝混之徒,皆疇昔之所附也,今比肩同列,常怏然自失,後果徙信安。}

\subsection*{9}

\textbf{殷仲文既素有名望,自謂必當阿衡朝政,忽作東陽太守,意甚不平,}{\footnotesize \textbf{晉安帝紀}曰仲文後為東陽,愈憤怨,乃與桓胤謀反,遂伏誅,仲文嘗照鏡不見頭,俄而難及。}\textbf{及之郡,至富陽,慨然歎曰:「看此山川形勢,當復出一孫伯符。」}{\footnotesize 孫策,富春人,故及此而歎。}