\section{雪山经}

\begin{center}Hemavata Sutta\end{center}\vspace{1em}

\subsection\*{\textbf{153}}

\textbf{「今天是十五布萨日,」七岳夜叉\footnote{此经旧译见杂阿含经第 1329 经、别译杂阿含经第 328 经。七岳、雪山等译名从别译杂阿含经。}说,「圣洁的夜晚降临,\\}
\textbf{「噫!我们去见乔达摩,享有盛名的大师!」}

“Ajja pannaraso uposatho, \textit{(iti Sātāgiro yakkho)} dibbā ratti upaṭṭhitā;\\
anomanāmaṃ satthāraṃ, handa passāma Gotamaṃ”. %\hfill\textcolor{gray}{\footnotesize 1}

\begin{enumerate}\item 缘起为何?缘起为问题的主导。世尊为雪山所问,说了「世间在六中生起」等。这里,「今天是十五」等为七岳所说,「七岳夜叉说」等为结集者所说,「心意是否」等为雪山所说,「世间在六中」等为世尊所说,这一切汇集后,即被称为「雪山经」。有些人则称为「七岳经」。
\item 这里,「今天是十五」等为初颂。其缘起为:即此贤劫中,迦叶世尊等正觉在人寿二万岁时投生,住世一万六千年后般涅槃,人们便以大供养而殓葬。其舍利未作散布,如金块般聚在一起,因为这是多寿诸佛的法性。而少寿的诸佛因为不为更多的世人所见便入般涅槃,所以作了舍利的供养后,为怜悯「各处的世人将生福德」而决意「散布舍利」,因此他们的舍利如金粉般散布,比如我们的世尊。
\item 人们为彼世尊作了一个舍利盒后,便教人建了支提,高一由旬,围亦如之。每隔一牛呼便有四门。国王松鸦把持一门,他名为「持地」的儿子把持一门,以将军为首的众大臣把持一门,以商人为首的国人把持一门。通体赤金所造,砖瓦也由种种宝物所造,与黄金的色泽相似,均价值百千。他们教人以雌黄、雄黄作土,以香油作水,建了这支提。
\item 如是,当支提建好时,两个族姓子好友便离家,在(迦叶世尊)亲传弟子的长老们跟前出了家。因为对于多寿的诸佛,唯有亲传弟子能给予出家、具足、依止,而非其他。随后,二族姓子便问:「尊者!教法内有几种责任?」长老们说:「两种责任,居住的责任与学习的责任。」
\item 这里,已出家的族姓子在阿阇黎与和尚跟前住满五年,履行了种种义务,练达于波罗提木叉及二三诵的经文,获取了业处,以除去对俗家及人群的执著而进入林野,为证得阿罗汉而努力、精进,此即\textbf{居住的责任}。而以自身的能力,或精通一部,或二部乃至五部后,从圣典及义理致力于令教法极明晰,此即\textbf{学习的责任}。
\item 于是,这二族姓子说「两种责任中,唯居住的责任为胜,但我们还是孩童,等年长时再圆满居住的责任,先履行学习的责任吧」,便开始学习。二人天性聪慧,不久就完成了对所有佛语的知晓,且于律极具裁断的善巧。依于二人的学习便聚集了随从,依于随从聚集了利养,便一一有了五百比丘的随从。二人阐明大师的教法而住,宛若又值佛时。
\item 那时,有二比丘住于村落,一为法说者,一为非法说者。非法说者凶恶、粗鄙、饶舌,其过行为另一人闻知。他便责备道:「朋友!你这行为于教法失当。」他打断说:「你何所见、何所闻?」另一个说:「持律者自会知晓。」非法说者知道「如果持律者来裁断此事,那我在教法内将无立足之处」,欲建立自己的派别,立刻带了资具,去到那两位长老处,布施了沙门的资具后,开始依他们而住,并对他们作着一切给侍,像是要恭敬地履行种种义务一般。
\item 之后一天,他在前去给侍、顶礼,当被二人遣散时,仍然站着。二长老便问他:「是否有什么要说的?」他说:「唯!尊者!我与一比丘就过行存在诤论,他如果来此告发此事,请不要按裁断来裁断。」二长老说:「不按裁断来裁断发起之事不妥。」他说:「若如是做,尊者!我在教法内无处立足,让此恶归我,你们莫作裁断!」二人受他逼迫,便同意了。
\item 他得了二人的认可,便又回到原来的住处,想「一切都在持律者跟前了结了」,愈加蔑视那法说者,举止粗鄙。法说者想「此事无疑」,立刻出发,去到二长老的一千随从比丘处,说:「朋友!难道不应如法裁断发起之事吗?或未令发起,则应宣示彼此的罪过而和合。然而,这二长老既未裁断此事,又不令和合,这算什么?」他们听后,也保持默然,想「阿阇黎们肯定知晓些什么」。随后,非法说者得了机会,逼迫法说者道「你先前说『持律者自会知晓』,现在去向这二持律者宣告此事吧」,并说「从今以后,你就输了,别再回这住处」,便离开了。
\item 随后,法说者去到二长老处,说「你们罔顾教法,却顾念『给侍、满足我等』之人,不守护教法,却守护人,从今以后,你们不适于作出裁断,今天,迦叶世尊已般涅槃」,大声号哭「大师的教法已灭」,悲泣着离开。
\item 于是,这些比丘意有悚惧,便生起恶作:「我们守护着人,却把教法之宝弃诸沟渠了。」他们即因此恶作,由意乐的败坏,死后不能转生于天界,一个阿阇黎转生于喜马拉雅的雪山中,名为\textbf{雪山}夜叉,第二个阿阇黎转生于中国的七山之中,名为\textbf{七岳}。二人的那些随从比丘仍追随着二者,无法转生于天界,成为二者的随从夜叉而转生。但布施他们资具的在家人转生到了天界。雪山与七岳位列二十八夜叉大将之中,有大威力,为夜叉王。
\item 这是夜叉大将的法性:每月之中的八天,诸天会为裁断法而聚集于喜马拉雅雄黄之原的具龙亭,应于此集会。于是,七岳与雪山在那集会中见到了对方,便即相认,问了各自的投生处「亲爱的!你投生在哪里」,便起追悔:「亲爱的!我们毁了!先前行了二万年的沙门法,依于一恶友而投生于夜叉的胎中,而我们的资具布施者倒是转生于欲界天中。」于是,七岳便说:「先生!喜马拉雅素称奇异希有,若看见或听闻什么希有之事,请告知我!」雪山也说:「先生!中国素称奇异希有,若看见或听闻什么希有之事,请也告知我!」如是,这二友人彼此订下规约,不放弃\footnote{不放弃 \textit{avivajjetvā}:PTS 本作「不疏忽 \textit{ariñcitvā}」。}这次出生,当如是住时,便度过一佛的间隔,大地隆起一由旬又三牛呼。
\item 于是,我们的菩萨在燃灯的足下已立下誓愿,直至毗输安多罗本生圆满了波罗蜜,投生到兜率天的居处,在那里住至寿尽,如法句的因缘中所说的方式为诸天祈请,观察了五大观察\footnote{五大观察:据菩提比丘注 727,见于\textbf{佛种姓}义注,即观察时间、洲、地方、家族、母亲等。},在向诸天宣告后,当三十二种征兆\footnote{三十二种征兆:据菩提比丘注 728,见于\textbf{自说}义注及\textbf{如是语}义注等,即一万世界大地震动、放大光明、盲者能视、聋者能听、哑者能言、偻者背直、跛者能行等等。}转起时,便于此获取结生,震动了一万个世界。而这些夜叉王见到后,竟不知道「以何原因发生这些」,有人说「因忙于游戏,甚至都没看见」。在出生、出离、觉悟时也如是。但在法轮转起,世尊与五人相谈,转起三转十二行相之最上法轮时,他们中唯有七岳首先见到了大地震动的征兆及诸神变。了知了发生这些的原因后,便与随从一起去到世尊处,听了法的开示,却未证得任何殊胜。为什么?因为他在听闻法时,忆念起雪山「我的朋友来了没有」,便观察会众,当未见到他时,想到「我的朋友被蒙蔽了,没有听闻世尊如是多彩、辩才的法的开示」而心生散乱。而世尊在日落时分也未令开示完结。
\item 于是,七岳想「带上朋友后,和他一起来听闻法的开示」,造了象车、马车、金翅鸟车等,为五百夜叉随从,朝着喜马拉雅进发。此时,雪山亦然。因为在结生、出生、出离、觉悟、涅槃时,三十二种征兆现已即逝,并不久住,但在法轮转起时,这些殊胜现已,久住后方灭,所以他在喜马拉雅见到这奇异现起后,想「自我出生,此山从未如是喜人、如往常一样,噫!现在我带上我的朋友,和他一起领略这花的灿烂」,也如是朝着中国出发。
\item 他们二人在王舍城的上方遇到后,便问起彼此前来的原因。雪山说:「先生!自我出生,此山从未如是喜人、如往常一样,树上盛开非时之花,所以想和你一起领略这花的灿烂,我即前来。」七岳便说:「可你,先生!知道什么原因而生此非时花盛开的神变吗?」「我并不知,先生!」「先生!这神变不独于雪山,而是生起于一万世界之中,等正觉者出现世间,今天转起法轮,以此原因。」如是,七岳对雪山谈起佛陀出世后,欲带他到世尊跟前,便说了此颂。有些人则以「当世尊住乔达摩支提时,他如是说」而说此颂。
\item 这里,\textbf{今天},即此日夜,从半月数来为\textbf{十五},由应当遵守为\textbf{布萨}。或于三布萨日中,今天是十五布萨日,非十四布萨日,非和合布萨日。或者,因为「布萨」一词表示诵波罗提木叉、八支、斋戒、概念、日期等众多意义,如\begin{quoting}来!朋友 Kappina!我们去布萨。(律藏)\end{quoting}等处「布萨」一词表示诵波罗提木叉,\begin{quoting}如是,毗舍佉!遵守具足八支的布萨……(增支部第 8:43 经)\end{quoting}等处表示离杀生等的八支,\begin{quoting}清净者总在禁食,\\清净者总在布萨。(中部·布经第 79 段)\end{quoting}等处表示斋戒,\begin{quoting}象王名为布萨。(中部·贤愚经第 258 段)\end{quoting}等处表示概念,\begin{quoting}在十五布萨那日,沐头者……(中部·贤愚经第 256 段)\end{quoting}等处表示日期,所以,在遮止其它意义后,唯限定为阿沙陀月\footnote{阿沙陀月 \textit{Āsāḷhī}:即现在的六~七月。}的满月之日而说「今天是十五布萨日」,即以「半月的第一天、第二天」来计数时,今天是第十五天之义。
\item 存在于天中者为圣洁者,于此有圣洁者即\textbf{圣洁}。这是什么?色。因为当晚,整个阎浮提都被从一万世界前来集会的诸天之身、衣、璎珞、宫殿的光辉以及除去云翳等遮染的月光所庄严,且尤其为最清净之天的世尊的身光所庄严,因此说「圣洁的夜晚降临」。
\item 如是,以称赞、说明夜的功德令朋友之心生起净喜,在谈起佛陀出世后,便说「噫!我们去见乔达摩,享有盛名的大师」。这里,以非低、非劣、圆满一切行相的功德所得之名为\textbf{享有盛名}。因为他如\begin{quoting}觉悟诸谛为佛陀,令人类觉悟为佛陀。(大义释)\end{quoting}等方法,以非低之功德而得名「佛陀」,且如\begin{quoting}断除贪为世尊,断除嗔为世尊。(大义释)\end{quoting}等方法,以非低之功德而得名,于「阿罗汉、正等正觉者、明行足」等亦同。以教授天、人现法等的义利「应舍弃此、应受持此而行持」为\textbf{大师},且以\begin{quoting}大师即世尊、商队的领队,好比商队的领队度脱有情于荒漠。(大义释)\end{quoting}等在义释中所说的方法为大师。此即享有盛名的大师。
\item \textbf{噫},即确定之义的不变词。\textbf{我们去见},即连带自己与他一起的现在时。\textbf{乔达摩},即乔达摩族姓。这说的是什么?莫作疑惑「是大师、不是大师」,完全确定后,来!我们去见乔达摩!\end{enumerate}

\subsection\*{\textbf{154}}

\textbf{「如此之人的心意,」雪山夜叉说,「是否善待一切生命?\\}
\textbf{「于诸可意和不可意,他的思惟是否受控?」}

“Kacci mano supaṇihito, \textit{(iti Hemavato yakkho)} sabbabhūtesu tādino;\\
kacci iṭṭhe aniṭṭhe ca, saṅkapp’assa vasīkatā”. %\hfill\textcolor{gray}{\footnotesize 2}

\begin{enumerate}\item 如是说已,雪山想「这七岳在说『享有盛名的大师』时,阐明了他的一切知性,而在世间,一切知者为难得,世间恰是由被认为是一切知者的富楼那等辈所祸害,他若是一切知者,则必证得如此之相,因此,我将如是检查之」,为问如此之相,说了此颂。
\item 这里,\textbf{是否},即问。\textbf{心意},即心。\textbf{善待},即善加安置、不动摇、不动荡。\textbf{如此之人}\footnote{如此之人 \textit{tādino}:菩提比丘英译作 impartial one,即平等之人。义注在说「如此,抑或不然」时,是将 tādino 一词拆解为「如此、不然 \textit{tādī no}」两词。},即证得如此之相的善人。或者,这即是问:「你的这位大师对一切生命都如此,抑或不然?」\textbf{可意和不可意},即如是样类的所缘。\textbf{思惟},即寻。
\item 这说的是什么?你所说的这位大师,你的这位大师、证得如此之相的善人的心意是否善待一切生命?抑或只是在不得动摇之缘时看起来像似善待?或者,你的这位大师是否以平等心对一切生命都如此,抑或不然?且于诸可意和不可意的所缘,这些会以贪嗔之力生起的思惟,它们是否受控?抑或有时也依彼等之力转起?\end{enumerate}

\subsection\*{\textbf{155}}

\textbf{「如此之人的心意,」七岳夜叉说,「善待一切生命,\\}
\textbf{「并且于诸可意和不可意,他的思惟受控。」}

“Mano c’assa supaṇihito, \textit{(iti Sātāgiro yakkho)} sabbabhūtesu tādino;\\
atho iṭṭhe aniṭṭhe ca, saṅkapp’assa vasīkatā”. %\hfill\textcolor{gray}{\footnotesize 3}

\begin{enumerate}\item 随后,七岳由对世尊一切知性的确定,为认可所有一切知的功德,说了此颂。
\item 这里,\textbf{善待},即善加安置,以不妨碍之义如大地,以善住立、不动摇之义如须弥山,以不为四种魔罗\footnote{四种魔罗:即作为天子、烦恼、蕴、死的四种魔罗。}及其他论众所撼动之义如因陀柱。且现在,由具足一切行相,住于一切知性之世尊的心意能够善待、不动摇,这并非异事。即便在俱贪等时生为畜生,投生于六牙象族之中,为毒箭射中亦能不动摇,对此杀戮者亦不冒犯,而是截断自己的牙后布施给他。同样,生为大猿,即便为巨石击中头部,仍为其指路。同样,生为 Vidhura 智者,即便被抓住双脚掷于六十由旬的黑山崖下,仍为此夜叉的义利而开示法。所以,七岳说「他的心意善待」完全恰当。
\item \textbf{如此之人},即证得如此之相的善人的心意,善待\textbf{一切生命}、一切有情,非仅当不得缘之义。这里,当知世尊有五种如此之相,如说:\begin{quoting}世尊以五行相而如此,于可意、不可意如此,以已舍弃而如此,以已解脱而如此,以已度而如此,以彼义释而如此。如何世尊于可意、不可意如此?当有利养时,世尊如此……(大义释)\end{quoting}如是等一切,当以在义释中所说的方法把握,且利养等当以其在「大义注」\footnote{大义注 \textit{Mahā-aṭṭhakathā}:当为古注。}中所详述的方法了知。
\item 且在这「这即是问:你的这位大师对一切生命都如此,抑或不然」的选择中,我们的大师对一切生命以平等心都如此之义。因为这世尊以欲摄乐、欲除苦而于一切有情心怀平等,如待自身,如此待他人,如待母亲摩诃摩耶,如此待学童女罗望子\footnote{学童女罗望子 \textit{Ciñcamāṇavikā} 事,见于\textbf{法句}义注。},如待父亲净饭,如此待善觉\footnote{善觉 \textit{Suppabuddha} 事,见于\textbf{法句}义注。},如待儿子罗睺罗,如此待提婆达多、护财、央掘魔罗等众杀戮者,于俱有天的世间也如此。所以,七岳说「如此之人……一切生命」完全恰当。
\item 而在\textbf{并且于诸可意和不可意}中,当如是显示其义:任何可意或不可意的所缘,以一切行相,于此会以贪嗔之力生起的思惟,由以无上的道舍断贪等之故而受控,不再依彼等之力转起。因为彼世尊为思惟不污浊者、心善解脱者、慧善解脱者。
\item 且此中,以「善待之心意」来说无有不如理作意,以「于一切生命、可意和不可意」来说存在于任何处的有情与行两种所缘,以「思惟受控的状态」来说由对此所缘无有此作意而舍断烦恼。且以「善待之心意」来说意正行之清净,以「于一切生命如此」来说身正行之清净,以「思惟受控的状态」来说以寻为根本之言语的语正行之清净。同样,当知以「善待之心意」来说无有贪等一切过失,以「于一切生命如此」来说慈等功德的善性,以「思惟受控的状态」来说于厌逆中作不厌逆想等的圣神变,且以此来说一切知性。\end{enumerate}

\subsection\*{\textbf{156}}

\textbf{「是否不取不与物?」雪山夜叉说,「是否于生命自制?\\}
\textbf{「是否离于放逸?是否不疏忽禅那?」}

“Kacci adinnaṃ nādiyati, \textit{(iti Hemavato yakkho)} kacci pāṇesu saññato;\\
kacci ārā pamādamhā, kacci jhānaṃ na riñcati”. %\hfill\textcolor{gray}{\footnotesize 4}

\begin{enumerate}\item 如是,雪山先前只以意门问如此的状态并得到他的肯定,为求确认,现在则以三门,或者先前只简略地问身语意门的清净并得到他的肯定,仍是为求确认,现在则详细地发问,说了此颂。
\item 这里,为了结颂的方便,首先问戒离不与取。\textbf{离于放逸},即由心对种种五欲的舍遣,以远离的状态来问戒离非梵行。文本或作 ārā pamadamhā,即是说离于女人。\textbf{不疏忽禅那},则以此来问这三种身恶行戒离的力量。因为从事禅那者的戒离是强有力的。\end{enumerate}

\subsection\*{\textbf{157}}

\textbf{「他不取不与物,」七岳夜叉说,「并于生命自制,\\}
\textbf{「并且离于放逸,佛陀不疏忽禅那。」}

“Na so adinnaṃ ādiyati, \textit{(iti Sātāgiro yakkho)} atho pāṇesu saññato;\\
atho ārā pamādamhā, Buddho jhānaṃ na riñcati”. %\hfill\textcolor{gray}{\footnotesize 5}

\begin{enumerate}\item 于是,因为世尊不仅于现在,而是于过去已长时戒离不与取等,且以彼彼戒离之威力而得彼彼大人相,俱有天的世间也以「沙门乔达摩戒离不与取」等方法表示赞叹,所以,七岳以清晰的言语作狮子吼,说了此颂。
\item 其语义明了。此颂的第三句也有 pamādamhā 和 pamadamhā 两种文本。第四句中,\textbf{不疏忽禅那},当知为不令禅那空无、不舍弃之义。\end{enumerate}

\subsection\*{\textbf{158}}

\textbf{「是否不妄语?」雪山夜叉说,「是否言路不粗鲁?\\}
\textbf{「是否不语中伤?是否不说绮语?」}

“Kacci musā na bhaṇati, \textit{(iti Hemavato yakkho)} kacci na khīṇabyappatho;\\
kacci vebhūtiyaṃ nāha, kacci samphaṃ na bhāsati”. %\hfill\textcolor{gray}{\footnotesize 6}

\begin{enumerate}\item 如是,在得知身门清净后,现在,为问语门清净而说此颂。
\item 此中,以散布为粗鲁\footnote{不粗鲁 \textit{na khīṇa°} 的原文费解,义注给出的异读为 nākhīṇa°,Norman 据此认为 CPD 的解释可从,即作 na ākhīṇa°,而 ākhīṇa/āskīrṇa < ā + √stṛ,故此处译作「散布」。},即伤害、迫害之义,言语之方式为言路,粗鲁的言路即\textbf{言路粗鲁},在以「不」字遮止后问「言路不粗鲁」,即是说「不恶口」。文本或作 nākhīṇabyappatho,即言路非不灭尽之义——因为恶语于他人心中不灭尽而住——即是说他是否不是如此之语者。以毁坏为亡失,以告知或造成毁坏为\textbf{中伤},即两舌的同义语,因为它以有情间背离彼此而造成亡失。其余之义自明。\end{enumerate}

\subsection\*{\textbf{159}}

\textbf{「他不妄语,」七岳夜叉说,「且言路不粗鲁,\\}
\textbf{「并且不语中伤,以智慧讲说义利。」}

“Musā ca so na bhaṇati, \textit{(iti Sātāgiro yakkho)} atho na khīṇabyappatho;\\
atho vebhūtiyaṃ nāha, mantā atthañ ca bhāsati”. %\hfill\textcolor{gray}{\footnotesize 7}

\begin{enumerate}\item 于是,因为世尊不仅于现在,而是于过去已长时戒离妄语等,且以彼彼戒离之威力而得彼彼大人相,俱有天的世间也以「沙门乔达摩戒离妄语」等方法表示赞叹,所以,七岳以清晰的言语作狮子吼,说了此颂。
\item 这里,\textbf{妄},即误陈所见等欺骗他人之语,他不语此。而第二句中,文本以第一种方式作「言路不粗鲁 \textit{na khīṇabyappatho}」,以第二种方式作「言路非不灭尽 \textit{nākhīṇabyappatho}」。第四句中,\textbf{智慧}即慧。因为世尊以此智慧确定已,唯\textbf{讲说义利}、不离义利之语,而非绮语——诸佛无有以无智为前导的离义之语——所以说「以智慧讲说义利」。其余之义自明。\end{enumerate}

\subsection\*{\textbf{160}}

\textbf{「是否不味著爱欲?」雪山夜叉说,「是否心不污浊?\\}
\textbf{「是否超越愚痴?是否于法具眼?」}

“Kacci na rajjati kāmesu, \textit{(iti Hemavato yakkho)} kacci cittaṃ anāvilaṃ;\\
kacci mohaṃ atikkanto, kacci dhammesu cakkhumā”. %\hfill\textcolor{gray}{\footnotesize 8}

\begin{enumerate}\item 如是,在得知语门也清净后,现在,为问意门清净而说此颂。
\item 这里,\textbf{爱欲},即物欲,以问「于此不以烦恼欲而\textbf{味著}」来问不贪求性。以问「\textbf{不污浊}」,就以嗔恚而起的污浊性,来问不嗔恚性。以问「\textbf{超越愚痴}」来问超越愚人以之持守邪见之愚痴的正见性。以问「\textbf{于法具眼}」,是想到「即便三门清净,也非一切知者」,来问于一切法不受限之智眼的一切知性,或以五眼来问于五眼境域诸法的一切知性。\end{enumerate}

\subsection\*{\textbf{161}}

\textbf{「他不味著爱欲,」七岳夜叉说,「且心不污浊,\\}
\textbf{「超越一切愚痴,佛陀于法具眼。」}

“Na so rajjati kāmesu, \textit{(iti Sātāgiro yakkho)} atho cittaṃ anāvilaṃ;\\
sabbamohaṃ atikkanto, Buddho dhammesu cakkhumā”. %\hfill\textcolor{gray}{\footnotesize 9}

\begin{enumerate}\item 于是,因为世尊尚未证得阿罗汉,即由以阿那含道舍断了对爱欲的贪染与嗔,既不味著爱欲,也不以嗔恚污浊其心,且由以须陀洹道舍断了作为邪见之缘的覆蔽真谛的愚痴而超越愚痴,并依自身于诸谛现等觉已,便得究竟解脱之名「佛陀」及所说的诸眼,所以,七岳为布告其意门清净及一切知性,说了此颂。\end{enumerate}

\subsection\*{\textbf{162}}

\textbf{「是否具足明?」雪山夜叉说,「是否行为清净?\\}
\textbf{「是否诸漏已尽?是否无有再有?」}

“Kacci vijjāya sampanno, \textit{(iti Hemavato yakkho)} kacci saṃsuddhacāraṇo;\\
kacci’ssa āsavā khīṇā, kacci n’atthi punabbhavo”. %\hfill\textcolor{gray}{\footnotesize 10}

\begin{enumerate}\item 如是,雪山在得知了世尊的三门清净及一切知性后,欢喜踊跃,以过去生中多闻明晰之慧而不羁绊于言路,欲闻奇异希有的一切知之功德,说了此颂。
\item 这里,\textbf{具足明},即以此问知见的成就。\textbf{行为清净},即以此问行的成就。且此中为协韵律,拉长而说 cā 字,即 saṃsuddhacaraṇo 之义。\textbf{诸漏已尽},即以此问当以知行之成就而证得的被称为漏尽的第一涅槃界的成就。\textbf{无有再有},即以此问第二涅槃界成就之能力,或在以省察智了知了最上安息的成就后而问住立性。\end{enumerate}

\subsection\*{\textbf{163}}

\textbf{「既具足明,」七岳夜叉说,「且行为清净,\\}
\textbf{「一切漏已尽,他无有再有。」}

“Vijjāya c’eva sampanno, \textit{(iti Sātāgiro yakkho)} atho saṃsuddhacāraṇo;\\
sabba’ssa āsavā khīṇā, n’atthi tassa punabbhavo”. %\hfill\textcolor{gray}{\footnotesize 11}

\begin{enumerate}\item 随后,这些以\begin{quoting}我(忆念)种种前世……\end{quoting}等方法在(中部)怖畏经等中所说的三种明,以\begin{quoting}当心如是等持……至于不动时,他将心导向知见。\end{quoting}等方法在(长部)阿摩昼经等中所说的八种明,因为世尊以具足一切行相而具足此等,且具足在如是指出\begin{quoting}于此,摩诃男!圣弟子具足戒,守护根门,饮食知量,常事醒觉,具足七善法,于增上心、现法乐住的四种禅那随欲可得。\end{quoting}后,以\begin{quoting}摩诃男!如何圣弟子具足戒……\end{quoting}等方法在(中部)有学经中说明的十五类行,且其对于世尊,因为以舍断一切随烦恼而极清净,又因为此欲漏等的四漏,彼等一切与其随从、习气对于世尊都已灭尽,又因为以此明行之成就而成漏尽已,世尊经省察「现在无有再有」而住,所以,七岳对世尊之一切知性因确定而振奋其心,并为认可一切功德,说了此颂。\end{enumerate}

\subsection\*{\textbf{164} \textcolor{gray}{\footnotesize 〔PTS 163A〕}}

\textbf{「对具足业与言路的牟尼的心,以及\\}
\textbf{「明行足,你如法地赞叹他。」\footnote{义注以为 164 为雪山所说,165~166 为七岳所说,167~168 为雪山所说,169 为二夜叉同说。}}

“Sampannaṃ munino cittaṃ, kammunā byappathena ca;\\
vijjācaraṇasampannaṃ, dhammato naṃ pasaṃsati”. %\hfill\textcolor{gray}{\footnotesize 12}

\begin{enumerate}\item 随后,雪山于世尊为「等正觉者、世尊」已无疑惑,仍立于空中,为赞叹世尊且为认同七岳,说了此颂。
\item 其义为:\textbf{具足的牟尼的心},即是说圆满、具足以「心意善待」中所说的如此之性,且圆满、具足以「他不取不与物」中所说的身\textbf{业},以「他不味著爱欲」中所说的意\textbf{业},以「他不妄语」中所说的\textbf{言路}及语\textbf{业}等。如是具足之心,\textbf{以及}由具足无上明行之成就的\textbf{明行足},你依这些功德以「心意善待」等方法\textbf{如法地赞叹他},表明唯从自性、真实、实在去赞叹他,而不仅是出于信。\end{enumerate}

\subsection\*{\textbf{165} \textcolor{gray}{\footnotesize 〔PTS 163B〕}}

\textbf{「对具足业与言路的牟尼的心,以及\\}
\textbf{「明行足,你如法地随喜。}

“Sampannaṃ munino cittaṃ, kammunā byappathena ca;\\
vijjācaraṇasampannaṃ, dhammato anumodasi. %\hfill\textcolor{gray}{\footnotesize 13}

\begin{enumerate}\item 随后,七岳也以「如是,先生!你对他已善加了知并随喜」之意趣,为认同他而说此颂。\end{enumerate}

\subsection\*{\textbf{166} \textcolor{gray}{\footnotesize 〔PTS 164〕}}

\textbf{「对具足业与言路的牟尼的心,以及\\}
\textbf{「明行足,噫!我们去见乔达摩!」}

Sampannaṃ munino cittaṃ, kammunā byappathena ca;\\
vijjācaraṇasampannaṃ, handa passāma Gotamaṃ”. %\hfill\textcolor{gray}{\footnotesize 14}

\begin{enumerate}\item 且如是说已,为再次催促他去见世尊,说了此颂。\end{enumerate}

\subsection\*{\textbf{167} \textcolor{gray}{\footnotesize 〔PTS 165〕}}

\textbf{「具羚羊之腿肚者,瘦削的英雄,少食而无贪求,\\}
\textbf{「在林中禅修的牟尼,来!我们去见乔达摩!}

“Eṇijaṅghaṃ kisaṃ vīraṃ, appāhāraṃ alolupaṃ;\\
muniṃ vanasmiṃ jhāyantaṃ, ehi passāma Gotamaṃ. %\hfill\textcolor{gray}{\footnotesize 15}

\begin{enumerate}\item 于是,雪山出于悦己,以自己过去生的多闻之力,为赞叹世尊,对七岳说了此颂。
\item 其义为:\textbf{具羚羊之腿肚者},因为诸佛的腿肚如羚羊般渐次丰满,而非在前无肉,在后如鳄鱼之腹般鼓起。且诸佛都以肢体修短合度之成就而\textbf{瘦削},非如肥胖之人般臃肿,或由以慧刮去烦恼而瘦削。由摧毁内外之敌为\textbf{英雄}。由一坐食及限量食为\textbf{少食},而非由仅二三团饭之食,如说:\begin{quoting}而我,优陀夷!有时以满钵而食,或食过此,则那些会以「沙门乔达摩是少食者,且赞叹少食」而恭敬、尊重、奉事、供养我,恭敬、尊重已依止而住的弟子们,优陀夷!我的仅食一杯之量、半杯之量、一木瓜、半木瓜的弟子们,便不会以此法恭敬我……依止而住。(中部·大舍拘罗陀夷经)\end{quoting}以于食物无有欲贪为\textbf{无贪求},即食用八支具足的食物。以牟尼性之成就为\textbf{牟尼}。以非家及心意倾向于远离而\textbf{在林中禅修}。因此,雪山夜叉说「具羚羊之腿肚者……来!我们去见乔达摩」。\end{enumerate}

\subsection\*{\textbf{168} \textcolor{gray}{\footnotesize 〔PTS 166〕}}

\textbf{「如同狮子,龙象独行,不关切爱欲,\\}
\textbf{「去到后,我们问问解脱死亡之网!」}

Sīhaṃ v’ekacaraṃ nāgaṃ, kāmesu anapekkhinaṃ;\\
upasaṅkamma pucchāma, maccupāsappamocanaṃ”. %\hfill\textcolor{gray}{\footnotesize 16}

\begin{enumerate}\item 且如是说已,再次为欲在彼世尊跟前听闻法,说了此颂。
\item 其义为:\textbf{如同狮子},即以难以接近、忍耐及无畏之义如同鬃狮。由渴爱而说「以爱为侣的人」,以无有此为\textbf{独行},亦由对一世界无两位佛陀生起为独行,且此中之义当如犀牛角经中所述\footnote{独行:详见\textbf{犀牛角经}第 35 颂的义注。}。\textbf{龙象},即于再有既不去、也不来,或以不犯罪为龙象\footnote{既不去、也不来,或以不犯罪:这是从语源学上解释「龙象」。},亦以有力为龙象。\textbf{不关切爱欲},即于两种爱欲以无有欲贪而不关切。\textbf{去到后,我们问问解脱死亡之网},即去到如此的大仙后,我们问问三界流转的死亡之网的解脱、还灭、涅槃。或者,以之从被称为苦之集的死亡之网解脱的方法,我们问问此死亡之网的解脱。雪山对七岳、七岳的随从及自己的随从说了此颂。
\item 尔时,阿沙陀月的节日来临。于是,王舍城四处都已庄严准备,如散发着光辉的天城一般,鱼鹰家\footnote{鱼鹰家 \textit{Kuraraghara}:词典中说为阿槃提国 \textit{Avanti} 的某城,但这里恐只是王舍城内的某家。}名为迦梨的优婆夷登上楼阁,打开窗子,正除遣幽居的烦闷,站在有风之处纳凉,便从头到尾听到了这两位夜叉大将关于佛陀功德的谈话。且听闻后,以「诸佛具足种种功德」生起以佛陀为所缘的喜,由此镇伏诸盖,即于此处便住立于须陀洹果。之后,被世尊置于上首:\begin{quoting}诸比丘!我的声闻优婆夷中,以传闻而净喜的上首,即鱼鹰家的迦梨优婆夷是。(增支部第 1:267 经)\end{quoting}\end{enumerate}

\subsection\*{\textbf{169} \textcolor{gray}{\footnotesize 〔PTS 167〕}}

\textbf{「宣说者,转起者,已度一切法者,\\}
\textbf{「佛陀,超越敌对怖畏者,我们问问乔达摩!」}

“Akkhātāraṃ pavattāraṃ, sabbadhammāna pāraguṃ;\\
buddhaṃ verabhayātītaṃ, mayaṃ pucchāma Gotamaṃ”. %\hfill\textcolor{gray}{\footnotesize 17}

\begin{enumerate}\item 这两位夜叉大将为一千夜叉所随从,在中夜时分到达仙人堕处,去到仍以转法轮之跏趺而坐的世尊处后,顶礼已,以此颂赞美世尊,提出请求。
\item 其义为:对除渴爱的三界之法,以\begin{quoting}诸比丘!此是苦圣谛。\end{quoting}等方法谈论诸谛的差别为\textbf{宣说},以\begin{quoting}「此苦圣谛应遍知」,诸比丘!我……(转法轮经)\end{quoting}等方法于彼等转起应作之智与已作之智为\textbf{转起}。或者,于应如是表达的诸法中,即以如是的表达来谈论为宣说,而于此等法以随适于有情来谈论为转起。或对敏知者、广知者的开示为宣说,对可予引导者的支持为转起\footnote{敏知者、广知者、可予引导者及仅通文字者,见\textbf{纯陀经}第 87 颂的说明。}。或以总说为宣说,以从彼彼行相的言说来分别为转起。或以对菩提分俱相的谈论为宣说,以对有情心相续的转起为转起。或简略地从三转谈论诸谛为宣说,详细地为转起,如以\begin{quoting}信根是法,转起此法即法轮。(无碍解道 2.40)\end{quoting}等无碍解的方法转起详细的法轮为转起。
\item \textbf{一切法},即四地之法。\textbf{已度},即以证知、遍知、舍断、修习、证得、等至等六种行相到达彼岸。因为彼世尊以证知一切法而度为证知已度,以遍知五取蕴而度为遍知已度,以舍断一切烦恼而度为舍断已度,以修习四道而度为修习已度,以证得灭而度为证得已度,以等至一切等至而度为等至已度。如是已度一切法。
\item \textbf{佛陀,超越敌对怖畏者},由从无智之睡眠觉醒为佛陀,或以在「皈依注」\footnote{皈依注:即\textbf{小诵}·三皈依之义注。}中所说的一切之义为佛陀,由超越五种敌对怖畏\footnote{五种敌对怖畏:据菩提比丘注,即对五戒的违犯,见\textbf{相应部}第 12:41 经。}为超越敌对怖畏者。当如是赞美世尊时,他们提出请求「我们问问乔达摩」。\end{enumerate}

\subsection\*{\textbf{170} \textcolor{gray}{\footnotesize 〔PTS 168〕}}

\textbf{「于何世间生起?」雪山夜叉说,「于何产生亲密?\\}
\textbf{「取著何者而有世间?于何世间遘难?」}

“Kismiṃ loko samuppanno, \textit{(iti Hemavato yakkho)} kismiṃ kubbati santhavaṃ;\\
kissa loko upādāya, kismiṃ loko vihaññati”. %\hfill\textcolor{gray}{\footnotesize 18}

\begin{enumerate}\item 于是,在这些夜叉中以光辉和智慧为上首的雪山,为问如其意趣所当问者,说了此颂。
\item 在其首句中,\textbf{于何}是依格独立式,这里是「当生起何者,\textbf{世间生起}」的意思,是就有情世间与行世间\footnote{有情世间与行世间,见\textbf{清净道论}·说六随念品第 37 段及以下。}发问。\textbf{于何产生亲密},即「我」或「我所」的爱、见之亲密于何产生,是表起因的依格。\textbf{何者而有世间}是表业格的属格,这里是\textbf{取著}何者而得「世间」之定义的意思。\textbf{于何世间}为依格独立式及表原因的依格,这里是「存在什么、以何原因,世间遘难、受逼迫、受恼害」的意思。\end{enumerate}

\subsection\*{\textbf{171} \textcolor{gray}{\footnotesize 〔PTS 169〕}}

\textbf{「于六世间生起,雪山!」世尊说,「于六产生亲密,\\}
\textbf{「唯取著六(而有世间),于六世间遘难。」}

“Chasu loko samuppanno, \textit{(Hemavatā ti Bhagavā)} chasu kubbati santhavaṃ;\\
channam eva upādāya, chasu loko vihaññati”. %\hfill\textcolor{gray}{\footnotesize 19}

\begin{enumerate}\item 于是,因为当六内外处生起,有情世间及财富、谷物等的行世间生起。又因为此中,有情世间即于彼等六中产生两种亲密,由其把握眼处或其余某处为「我、我所」,如说:\begin{quoting}若说「眼是我」者,此不应理。(中部·六六经第 422 段)\end{quoting}等。又因为唯取著此六而得两种世间之定义。又因为即存在此六,有情世间由苦的显现而遘难,如说:\begin{quoting}诸比丘!有手的存在而有取舍,有脚的存在而有进退,有关节的存在而有屈伸,有腹的存在而有饥渴,如是,诸比丘!有眼的存在,以眼触为缘,内在的苦乐生起。(相应部第 35:236 经)\end{quoting}等。同样,以彼等作为支持,受妨碍的行世间遘难,如说:\begin{quoting}(有见有对之色)在无见有对之眼处受妨碍……(法集论第 597 段)\end{quoting}以及\begin{quoting}诸比丘!眼在适意、不适意的色中受妨碍。(相应部第 35:238 经)\end{quoting}等。同样,即以彼等作为原因,两种世间遘难,如说:\begin{quoting}眼在适意、不适意的色中遘难。(相应部第 35:238 经)\end{quoting}以及\begin{quoting}诸比丘!眼在燃烧,色在燃烧……因何燃烧?因贪之火……(相应部第 35:28 经)\end{quoting}等。所以,世尊为以六内外处解答此问,说了此颂。\end{enumerate}

\subsection\*{\textbf{172} \textcolor{gray}{\footnotesize 〔PTS 170〕}}

\textbf{「什么是这世间在其中遘难的取著?\\}
\textbf{「问及出离,请说如何从苦解脱!」}

“Katamaṃ taṃ upādānaṃ, yattha loko vihaññati;\\
niyyānaṃ pucchito brūhi, kathaṃ dukkhā pamuccati”. %\hfill\textcolor{gray}{\footnotesize 20}

\begin{enumerate}\item 于是,这夜叉对世尊以十二处简略地回答自己以流转所提的问题未能善加辨别,欲了知其义与其对治,仍简略地问及流转与还灭,说了此颂。
\item 这里,以可被取著之义为\textbf{取著},为苦谛的同义语。\textbf{世间在其中遘难},即世尊以「于六世间遘难」所说的「世间在六种取著之中遘难」,\textbf{什么是这}取著?如是他以半颂明确地问了苦谛,而集谛作为其原因也被包摄。\textbf{问及出离},则以此半颂问了道谛。因为圣弟子以道谛知苦、断集、证灭、修道而从世间出离,所以被称为出离。\textbf{如何},即以何方式。\textbf{从苦解脱},即从所说为「取著」的流转之苦证得解脱。如是于此他明确地问了道谛,而灭谛作为其境域也被包摄。\end{enumerate}

\subsection\*{\textbf{173} \textcolor{gray}{\footnotesize 〔PTS 171〕}}

\textbf{「种种五欲被宣告于世间,意为第六,\\}
\textbf{「除去此中的欲,如是从苦解脱。}

“Pañca kāmaguṇā loke, manochaṭṭhā paveditā;\\
ettha chandaṃ virājetvā, evaṃ dukkhā pamuccati. %\hfill\textcolor{gray}{\footnotesize 21}

\begin{enumerate}\item 如是,当被夜叉以明示及未明示地问及四谛,世尊也以此方法解答,说了此颂。
\item 这里,以被称为\textbf{种种五欲}的行处所摄,作为其行处的五处即被包摄。以意为彼等之第六,即\textbf{意为第六}。\textbf{被宣告},即被阐明。此中,以内处中第六意处所摄,作为其境域的法处即被包摄。如是,为解答此问「什么是这取著」,又再次依十二处阐明了苦谛。
\item 或者,由七识界被意包摄,其中,以前五识界所摄,作为彼等依处的眼等五处即被包摄,以意界、意识界所摄,作为彼等依处与行处等类的法处即被包摄,如是,也依十二处阐明了苦谛。且此中,由于是对「世间在其中遘难」的说明,出世间的意处及法处一分未被包摄。
\item \textbf{除去此中的欲},即\textbf{此}十二处等类的苦谛\textbf{中},从蕴、界、名色等如此如此分别了这些处后,引入三相,修观者以阿罗汉道的终了之观,完全\textbf{除去}、调伏、摧毁了称为渴爱的\textbf{欲}之义。\textbf{如是从苦解脱},即以此方式,从此流转之苦解脱。如是,以此半颂,此问「问及出离,请说如何从苦解脱」得以解答,且道谛得以阐明。而此中集谛、灭谛由以先前的方法所摄,当知也已阐明。
\item 或者,当知以半颂阐明苦谛,以「欲」阐明集谛,以此中「除去」之离贪阐明灭谛,由「离贪则解脱」之语或以「如是」所示的方法阐明道谛,以「苦灭」之语或「从苦解脱」之苦的解脱阐明灭谛,如是于此得以阐明四谛。\end{enumerate}

\subsection\*{\textbf{174} \textcolor{gray}{\footnotesize 〔PTS 172〕}}

\textbf{「此即世间的出离,已如实对你们宣说,\\}
\textbf{「我将对你们宣说此,如是从苦解脱。」}

Etaṃ lokassa niyyānaṃ, akkhātaṃ vo yathātathaṃ;\\
etaṃ vo aham akkhāmi, evaṃ dukkhā pamuccati”. %\hfill\textcolor{gray}{\footnotesize 22}

\begin{enumerate}\item 如是,以包含四谛之颂,从相阐明了出离,为再次对此以自己的言辞总结,说了此颂。
\item 此中,\textbf{此},即先前所说的义释。\textbf{世间},即三界之世间。\textbf{如实},即无颠倒。\textbf{我将对你们宣说此},即便你们问我千遍,我仍将对你们宣说此,而非其它。为什么?因为\textbf{如是从苦解脱},而非其它之义。
\item 或者,即便对以此出离已作了一次、二次、三次的离去者,我仍将对你们宣说此,即为证更上的殊胜,我仍将宣说此之义。为什么?因为如是从无余、全体之苦解脱,即以阿罗汉为顶点完成了开示。
\item 当开示终了,二夜叉大将与一千夜叉便住于须陀洹果。\end{enumerate}

\subsection\*{\textbf{175} \textcolor{gray}{\footnotesize 〔PTS 173〕}}

\textbf{「谁于此度过暴流?谁于此度过海洋?\\}
\textbf{「于无落足、无攀援之深,谁不沉没?」}

“Ko sū’dha tarati oghaṃ, ko’dha tarati aṇṇavaṃ;\\
appatiṭṭhe anālambe, ko gambhīre na sīdati”. %\hfill\textcolor{gray}{\footnotesize 23}

\begin{enumerate}\item 于是,虽然雪山天性就敬重法,现在则因住于圣地,于世尊多样辩才的开示更无厌足,为问世尊有学、无学之地,说了此颂。
\item 这里,\textbf{谁于此度过暴流},即以此「谁度过四暴流」无差别地问有学地。因为海洋不仅仅广或不仅仅深,而是既甚广且甚深,故得是称,轮回之海洋也如此,因为它以周匝无有边界而广,以下无落足、上无攀援而深,所以,\textbf{谁于此度过海洋}?且于此\textbf{无落足、无攀援之深}海,\textbf{谁不沉没}?即问无学地。\end{enumerate}

\subsection\*{\textbf{176} \textcolor{gray}{\footnotesize 〔PTS 174〕}}

\textbf{「始终具足戒,具慧,善等持,\\}
\textbf{「内省,具念,他度过难度的暴流。}

“Sabbadā sīlasampanno, paññavā susamāhito;\\
ajjhattacintī satimā, oghaṃ tarati duttaraṃ. %\hfill\textcolor{gray}{\footnotesize 24}

\begin{enumerate}\item 若比丘宁舍性命也不行违犯而\textbf{始终具足戒},且以世出世间慧而\textbf{具慧},以近行、安止定及威仪、下三道果而\textbf{善等持},习于引入三相、以毗婆舍那\textbf{内省}自身,\textbf{具}足导向坚持的不放逸之\textbf{念},因为他以第四道无余地\textbf{度过}这极\textbf{难度的暴流},所以,世尊为解答有学地,说了这包含三学之颂。
\item 此中,以戒的成就为增上戒学,以念与定为增上心学,以内省与慧为增上慧学,如是说了三学及其资助与利益。因为世间慧与念为三学的资助,而沙门果为其利益。\end{enumerate}

\subsection\*{\textbf{177} \textcolor{gray}{\footnotesize 〔PTS 175〕}}

\textbf{「戒离爱欲想,越过一切结缚,\\}
\textbf{「灭尽喜与有,他不沉没于深。」}

Virato kāmasaññāya, sabbasaṃyojanātigo;\\
nandībhavaparikkhīṇo, so gambhīre na sīdati”. %\hfill\textcolor{gray}{\footnotesize 25}

\begin{enumerate}\item 如是,在以第一颂显明了有学地后,现在,为显明无学地而说第二颂。
\item 其义为:\textbf{戒离爱欲想},即以与第四道相应的正断离戒离任何爱欲之想。文本也作「离染 \textit{viratto}」。此处「爱欲想」为依格,而在(相应部)有偈品中文本也作复数 \textit{kāmasaññāsu}。亦由以第四道越过十种结缚而\textbf{越过一切结缚},或仅以第四道越过一切上分结缚。由灭尽被称为彼彼乐著之渴爱的喜与三有而\textbf{灭尽喜与有}。像\textbf{他}这样的漏尽比丘\textbf{不沉没于}轮回之\textbf{深}海,因喜的灭尽而至有余依涅槃之陆地,因有的灭尽而至无余依涅槃之陆地,以至最上的安息。\end{enumerate}

\subsection\*{\textbf{178} \textcolor{gray}{\footnotesize 〔PTS 176〕}}

\textbf{「深慧,见微妙义,无所牵绊,不取著爱欲与有,\\}
\textbf{「你们看这解脱于一切处、行走在天路上的大仙!}

“Gambhīrapaññaṃ nipuṇatthadassiṃ, akiñcanaṃ kāmabhave asattaṃ;\\
taṃ passatha sabbadhi vippamuttaṃ, dibbe pathe kamamānaṃ mahesiṃ. %\hfill\textcolor{gray}{\footnotesize 26}

\begin{enumerate}\item 于是,雪山观察了朋友与夜叉会众,生起喜悦,以「深慧」等颂称赏了世尊,与全体会众及朋友一起顶礼并右绕后,便回到自己的住处。
\item 这些颂的释义为:\textbf{深慧},即具足甚深之慧,这里当以无碍解道中所说的方法来理解深慧,因为彼处说:\begin{quoting}于甚深诸蕴转起的智为深慧(无碍解道 3.4)\end{quoting}等等。\textbf{见微妙义},即能见由微妙的刹帝利智者等所提问题之义,或以能见他人难以通达的微妙原因为见微妙义。以无有任何贪染等为\textbf{无所牵绊}。以不执著于二种爱欲与三种有为\textbf{不取著爱欲与有}。以于蕴等类的一切所缘无有欲贪的束缚为\textbf{解脱于一切处}。
\item \textbf{行走在天路上},即以入定经行于八等至等类的天路。这里,虽然世尊并非在此时行走于天路,但依先前的行走,以具备行走的能力,以于此所得的势力而如是说。或者,以行走于任何清净天阿罗汉的路与寂静的住处而说此。以寻求大功德为\textbf{大仙}。\end{enumerate}

\subsection\*{\textbf{179} \textcolor{gray}{\footnotesize 〔PTS 177〕}}

\textbf{「享有盛名,见微妙义,给予智慧,不取著于欲执,\\}
\textbf{「你们看这知晓一切、善慧、行走在圣路上的大仙!}

Anomanāmaṃ nipuṇatthadassiṃ, paññādadaṃ kāmālaye asattaṃ;\\
taṃ passatha sabbaviduṃ sumedhaṃ, ariye pathe kamamānaṃ mahesiṃ. %\hfill\textcolor{gray}{\footnotesize 27}

\begin{enumerate}\item 在第二颂中,以别的方法作了赞赏,再次显示\textbf{见微妙义},或者,即「显示微妙义者」之义。\textbf{给予智慧},即以讲论导向获得智慧的行道为施与智慧者。\textbf{不取著于欲执},即于爱欲不以爱、见两种执著而取著。
\item \textbf{知晓一切},即知晓一切法,即是说一切知者。\textbf{善慧},即具足成为一切知性之道的称为波罗蜜慧的慧。\textbf{圣路},即八支圣道,或果定。\textbf{行走},即以慧潜入,由了知了道之相而开示故,或即进入,由刹那刹那入于果定故,或以被称为四种道之修习的行走的能力而先前曾行。\end{enumerate}

\subsection\*{\textbf{180} \textcolor{gray}{\footnotesize 〔PTS 178〕}}

\textbf{「我们今天确实有好的所见、好的早晨、好的起身,\\}
\textbf{「因为我们见到了已度过暴流、无漏的等正觉。}

Sudiṭṭhaṃ vata no ajja, suppabhātaṃ suhuṭṭhitaṃ;\\
yaṃ addasāma Sambuddhaṃ, oghatiṇṇam anāsavaṃ. %\hfill\textcolor{gray}{\footnotesize 28}

\begin{enumerate}\item \textbf{我们今天确实有好的所见},即今天我们得见善妙的所见,或今天我们有善妙的所见,即知见之义。\textbf{好的早晨、好的起身},即今天我们有好的早晨或净美的早晨,且今天我们有善妙的起身,无障碍地从睡眠中起身。什么原因?\textbf{因为我们见到了等正觉},就自身利益的成就而宣告愉悦。\end{enumerate}

\subsection\*{\textbf{181} \textcolor{gray}{\footnotesize 〔PTS 179〕}}

\textbf{「这一千个夜叉具有神变、具有名望,\\}
\textbf{「全都皈依你,你是我们无上的大师。}

Ime dasasatā yakkhā, iddhimanto yasassino;\\
sabbe taṃ saraṇaṃ yanti, tvaṃ no satthā anuttaro. %\hfill\textcolor{gray}{\footnotesize 29}

\begin{enumerate}\item \textbf{具有神变},即具足业异熟所生的神变。\textbf{具有名望},即具足最上的利养、最上的眷属。\textbf{皈依},虽然已经以道而至,仍如是为说明须陀洹的状态及为表明净喜而诉之于言。\end{enumerate}

\subsection\*{\textbf{182} \textcolor{gray}{\footnotesize 〔PTS 180〕}}

\textbf{「我们将从村到村、从山到山地游行,\\}
\textbf{「礼敬着等正觉,以及法的善法性。」}

Te mayaṃ vicarissāma, gāmā gāmaṃ nagā nagaṃ;\\
namassamānā Sambuddhaṃ, dhammassa ca sudhammatan” ti. %\hfill\textcolor{gray}{\footnotesize 30}

\begin{enumerate}\item \textbf{从村到村},即从天村到天村。\textbf{从山到山},即从天山到天山。\textbf{礼敬着等正觉,以及法的善法性},即是说以「世尊确实是正等正觉者、世尊的法确实是善说」等方法称赏佛的善觉性及法的善法性,并以「世尊的声闻众确实是善行道者」等称赏僧的善行道,我们将礼敬、宣扬法而游行。余义于此自明。\end{enumerate}

\begin{center}\vspace{1em}雪山经第九\\Hemavatasuttaṃ navamaṃ.\end{center}

%\begin{flushright}甲辰清明二稿\end{flushright}