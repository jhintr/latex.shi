\chapter{豪爽第十三}

\subsection*{1}

\textbf{王大將軍年少時,舊有田舍名,語音亦楚,武帝喚時賢共言伎蓺事,人皆多有所知,唯王都無所關,意色殊惡,自言知打鼓吹,帝令取鼓與之,於坐振袖而起,揚槌奮擊,音節諧捷,神氣豪上,傍若無人,舉坐歎其雄爽。}{\footnotesize 或曰敦嘗坐武昌釣臺,聞行船打鼓,嗟稱其能,俄而一槌小異,敦以扇柄撞几曰「可恨」,應侍側曰「不然,此是回颿槌」,使視之,云「船人入夾口」,應知鼓又善於敦也。}

\subsection*{2}

\textbf{王處仲世許高尚之目,嘗荒恣於色,體為之敝,左右諫之,處仲曰:「吾乃不覺爾,如此者,甚易耳。」乃開後閤,驅諸婢妾數十人出路,任其所之,時人歎焉。}{\footnotesize \textbf{鄧粲晉紀}曰敦性簡脫,口不言財,其存尚如此。}

\subsection*{3}

\textbf{王大將軍自目「高朗疎率,學通左氏」。}{\footnotesize \textbf{晉陽秋}曰敦少稱高率通朗,有鑒裁。}

\subsection*{4}

\textbf{王處仲每酒後輒詠「老驥伏櫪,志在千里,烈士暮年,壯心不已」,}{\footnotesize 魏武帝樂府詩。}\textbf{以如意打唾壺,壺口盡缺。}

\subsection*{5}

\textbf{晉明帝欲起池臺,元帝不許,帝時為太子,好養武士,一夕中作池,比曉便成,今太子西池是也。}{\footnotesize \textbf{丹陽記}曰西池,孫登所創,吳史所稱西苑也,明帝修復之耳。}

\subsection*{6}

\textbf{王大將軍始欲下都更分樹置,先遣參軍告朝廷,諷旨時賢,祖車騎尚未鎮壽春,瞋目厲聲語使人曰:「卿語阿黑,}{\footnotesize 敦小字也。}\textbf{何敢不遜!催攝面去,須臾不爾,我將三千兵槊腳令上。」王聞之而止。}

\subsection*{7}

\textbf{庾穉恭既常有中原之志,文康時,權重未在己,及季堅作相,忌兵畏禍,與穉恭歷同異者久之,乃果行,傾荊漢之力,窮舟車之勢,師次于襄陽,}{\footnotesize \textbf{漢晉春秋}曰翼風儀美劭,才能豐贍,少有經緯大略,及繼兄亮居方州之任,有匡維內外、埽蕩群凶之志,是時,杜乂、殷浩諸人盛名冠世,翼未之貴也,常曰「此輩宜束之高閣,俟天下清定,然後議其所任耳」,其意氣如此,唯與桓溫友善,相期以寧濟宇宙之事,初,翼輒發所部奴及車馬萬數,率大軍入沔,將謀伐狄,遂次于襄陽。\textbf{翼別傳}曰翼為荊州,雅有正志,每以門地威重,兄弟寵授,不陳力竭誠,何以報國,雖蜀阻險塞,胡負凶力,然皆無道酷虐,易可乘滅,當此時,不能掃除二寇,以復王業,非丈夫也,於是徵役三州,悉其帑實,成眾五萬,兼率荒附,治戎大舉,直指魏趙,軍次襄陽,耀威漢北也。}\textbf{大會參佐,陳其旌甲,親授弧矢曰:「我之此行,若此射矣。」遂三起三疊,徒眾屬目,其氣十倍。}

\subsection*{8}

\textbf{桓宣武平蜀,集參僚置酒於李勢殿,巴蜀縉紳莫不來萃,桓既素有雄情爽氣,加爾日音調英發,敘古今成敗由人、存亡繫才,其狀磊落,一坐歎賞,既散,諸人追味餘言,于時尋陽周馥曰:「恨卿輩不見王大將軍。」}{\footnotesize \textbf{中興書}曰馥,周撫孫也,字濟隱,有將略,曾作敦掾。}

\subsection*{9}

\textbf{桓公讀高士傳,至於陵仲子,便擲去曰:「誰能作此溪刻自處?」}{\footnotesize \textbf{皇甫謐高士傳}曰陳仲子,字子終,齊人,兄戴相齊,食祿萬鍾,仲子以兄祿為不義,乃適楚,居於陵,曾乏糧三日,匍匐而食井李之實,三咽而後能視,身自織屨,令妻擗纑,以易衣食,嘗歸省母,有饋其兄生鵝者,仲子嚬顣曰「惡用此鶂鶂為哉」,後母殺鵝,仲子不知而食之,兄自外入曰「鶂鶂肉邪」,仲子出門,哇而吐之,楚王聞其名,聘以為相,乃夫婦逃去,為人灌園。}

\subsection*{10}

\textbf{桓石虔,司空豁之長庶也,}{\footnotesize \textbf{豁別傳}曰豁,字朗子,溫之弟,累遷荊州刺史,贈司空。}\textbf{小字鎮惡,年十七八未被舉,而童隸已呼為鎮惡郎,嘗住宣武齋頭,從征枋頭,車騎沖沒陳,左右莫能先救,宣武謂曰:「汝叔落賊,汝知不?」石虔聞,氣甚奮,命朱辟為副,策馬於數萬眾中,莫有抗者,徑致沖還,三軍歎服,河朔後以其名斷瘧。}{\footnotesize \textbf{中興書}曰石虔有才幹,有史學,累有戰功,仕至豫州刺史,贈後軍將軍。}

\subsection*{11}

\textbf{陳林道在西岸,}{\footnotesize \textbf{晉陽秋}曰逵為西中郎將,領淮南太守,戍歷陽。}\textbf{都下諸人共要至牛渚會,陳理既佳,人欲共言折,陳以如意拄頰,望雞籠山,歎曰:「孫伯符志業不遂。」}{\footnotesize \textbf{吳錄}曰長沙桓王諱策,字伯符,吳郡富春人,少有雄姿風氣,年十九而襲業,眾號孫郎,平定江東,為許貢客射破其面,引鏡自照,謂左右曰「面如此,豈可復立功乎」,乃謂張昭曰「中國方亂,夫以吳越之眾、三江之固,足以觀成敗,公等善相吾弟」,呼大皇帝,授以印綬曰「舉江東之眾,決機於兩陳之間,卿不如我,任賢使能,各盡其心,我不如卿,慎勿北渡」,語畢而薨,年二十有六。}\textbf{於是竟坐不得談。}

\subsection*{12}

\textbf{王司州在謝公坐,詠「入不言兮出不辭,乘回風兮載雲旗」,}{\footnotesize 離騷九歌少司命之辭。}\textbf{語人云:「當爾時,覺一坐無人。」}

\subsection*{13}

\textbf{桓玄西下,入石頭,外白「司馬梁王奔叛」,}{\footnotesize \textbf{續晉陽秋}曰梁王珍之,字景度。\textbf{中興書}曰初,桓玄篡位,國人有孔璞者,奉珍之奔尋陽,義旗既興,歸朝廷,仕至太常卿,以罪誅。}\textbf{玄時事形已濟,在平乘上笳鼓並作,直高詠云:「簫管有遺音,梁王安在哉?」}{\footnotesize 阮籍詠懷詩也。}