\chapter{仇隟第三十六}

\subsection*{1}

\textbf{孫秀既恨石崇不與綠珠,}{\footnotesize \textbf{干寶晉紀}曰石崇有妓人綠珠,美而工笛,孫秀使人求之,崇別館北邙下,方登涼觀,臨清水,使者以告,崇出其婢妾數十人以示之曰「任所以擇」,使者曰「本受命者,指綠珠也,未識孰是」,崇勃然曰「綠珠,吾所愛,不可得也」,使者曰「君侯博古知今,察遠照邇,願加三思」,崇不然,使者已出又反,崇竟不許。}\textbf{又憾潘岳昔遇之不以禮,後秀為中書令,岳省內見之,因喚曰:「孫令,憶疇昔周旋不?」秀曰:「中心藏之,何日忘之?」岳於是始知必不免,}{\footnotesize \textbf{王隱晉書}曰岳父文德,為琅邪太守,孫秀為小吏給使,岳數蹴蹋秀,而不以人遇之也。}\textbf{後收石崇、歐陽堅石,同日收岳,}{\footnotesize \textbf{晉陽秋}曰歐陽建,字堅石,渤海人,有才藻,時人為之語曰「渤海赫赫,歐陽堅石」,初,建為馮翊太守,趙王倫為征西將軍,孫秀為腹心,撓亂關中,建每匡正,由是有隟。\textbf{王隱晉書}曰石崇、潘岳與賈謐相友善,及謐廢,懼終見危,與淮南王謀誅倫,事泄,收崇及親朞以上皆斬之,初,岳母誡岳以止足之道,及收,與母別曰「負阿母」,崇家河北,收者至,曰「吾不過流徙交、廣耳」,及車載東市,始歎曰「奴輩利吾家之財」,收崇人曰「知財為害,何不蚤散」,崇不能答。}\textbf{石先送市,亦不相知,潘後至,石謂潘曰:「安仁,卿亦復爾邪?」潘曰:「可謂『白首同所歸』。」}{\footnotesize \textbf{語林}曰潘、石同刑東市,石謂潘曰「天下殺英雄,卿復何為」,潘曰「俊士填溝壑,餘波來及人」。}\textbf{潘金谷集詩云「投分寄石友,白首同所歸」,乃成其讖。}

\subsection*{2}

\textbf{劉璵兄弟少時為王愷所憎,嘗召二人宿,欲默除之,令作阬,阬畢,垂加害矣,石崇素與璵、琨善,聞就愷宿,知當有變,便夜往詣愷,問二劉所在,愷卒迫不得諱,答云:「在後齋中眠。」石便徑入,自牽出,同車而去,語曰:「少年,何以輕就人宿?」}{\footnotesize \textbf{劉璨晉紀}曰琨與兄璵俱知名,遊權貴之間,當世以為豪傑。}

\subsection*{3}

\textbf{王大將軍執司馬愍王,夜遣世將載王於車而殺之,當時不盡知也,}{\footnotesize \textbf{晉陽秋}曰司馬丞,字元敬,譙王遜子也,為中宗相州刺史,路過武昌,王敦與燕會,酒酣,謂丞曰「大王篤實佳士,非將御之才」,對曰「焉知鉛刀不能一割乎」,敦將謀逆,召丞為軍司馬,丞歎曰「吾其死矣,地荒民解,勢孤援絕,赴君難,忠也,死王事,義也,死忠與義,又何求焉」,乃馳檄諸郡丞赴義,敦遣從母弟魏乂攻丞,王廙使賊迎之,薨於車,敦既滅,追贈驃騎,諡曰愍王。}\textbf{雖愍王家亦未之皆悉,而無忌兄弟皆穉,}{\footnotesize \textbf{無忌別傳}曰無忌,字公壽,丞子也,才器兼濟,有文武幹,襲封譙王,衛軍將軍。}\textbf{王胡之與無忌長甚相暱,胡之嘗共遊,無忌入告母,請為饌,母流涕曰:「王敦昔肆酷汝父,假手世將,}{\footnotesize \textbf{司馬氏譜}曰丞娶南陽趙氏女。\textbf{王廙別傳}曰廙,字世將,祖覽,父正,廙高朗豪率,王導、庾亮遊于石頭,會廙至,爾日迅風飛颿,廙倚船樓長嘯,神氣甚逸,導謂亮曰「世將為復識事」,亮曰「正足舒其逸耳」,性倨傲,不合己者面拒之,故為物所疾,加平南將軍,薨。}\textbf{吾所以積年不告汝者,王氏門彊,汝兄弟尚幼,不欲使此聲著,蓋以避禍耳。」無忌驚號,抽刃而出,胡之去已遠。}

\subsection*{4}

\textbf{應鎮南作荊州,}{\footnotesize \textbf{王隱晉書}曰應詹,字思遠,汝南南頓人,璩曾孫也,為人弘長有淹度,飾之以文才,司徒何充歎曰「所謂文質之士」,累遷江州刺史、鎮南將軍。}\textbf{王脩載、譙王子無忌同至新亭與別,坐上賓甚多,不悟二人俱到,有一客道:「譙王丞致禍,非大將軍意,正是平南所為耳。」無忌因奪直兵參軍刀,便欲斫,脩載走投水,舸上人接取,得免。}{\footnotesize \textbf{中興書}曰褚裒為江州,無忌於坐拔刀斫耆之,裒與桓景共免之,御史奏無忌欲專殺害,詔以贖論。前章既言無忌母告之,而此章復云客敘其事,且王廙之害司馬丞,遐邇共悉,脩齡兄弟豈容不知?法盛之言,皆實錄也。}

\subsection*{5}

\textbf{王右軍素輕藍田,藍田晚節論譽轉重,右軍尤不平,藍田於會稽丁艱,停山陰治喪,右軍代為郡,屢言出弔,連日不果,後詣門自通,主人既哭,不前而去,以陵辱之,於是彼此嫌隟大搆,後藍田臨揚州,右軍尚在郡,初得消息,遣一參軍詣朝廷,求分會稽為越州,使人受意失旨,大為時賢所笑,藍田密令從事數其郡諸不法,以先有隟,令自為其宜,右軍遂稱疾去郡,以憤慨致終。}{\footnotesize \textbf{中興書}曰羲之與述志尚不同,而兩不相能,述為會稽,艱居郡境,王羲之後為郡,申慰而已,初不重詣,述深以為恨,喪除,徵拜揚州,就徵,周行郡境,而不歷羲之,臨發,一別而去,羲之初語其友曰「王懷祖免喪,正可當尚書,投老可得為僕射,更望會稽,便自邈然」,述既顯授,又檢校會稽郡,求其得失,主者疲於課對,羲之恥慨,遂稱疾去郡,墓前自誓不復仕,朝廷以其誓苦,不復徵也。}

\subsection*{6}

\textbf{王東亭與孝伯語,後漸異,孝伯謂東亭曰:「卿便不可復測。」答曰:「王陵廷爭,陳平從默,但問克終云何耳。」}{\footnotesize \textbf{漢書}曰呂后欲王諸呂,問右相王陵,以為不可,問左丞相陳平,平曰「可」,陵出讓平,平曰「面折廷爭,臣不如君,全社稷,定劉氏,君不如臣」。\textbf{晉安帝紀}曰初,王恭赴山陵,欲斬國寶,王珣固諫之,乃止,既而恭謂珣曰「此日視君,一似胡廣」,珣曰「王陵廷爭,陳平從默,但問克終如何也」。}

\subsection*{7}

\textbf{王孝伯死,縣其首於大桁,司馬太傅命駕出至標所,孰視首,曰:「卿何故趣欲殺我邪?」}{\footnotesize \textbf{續晉陽秋}曰王恭深懼禍難,抗表起兵,於是遣左將軍謝琰討恭,恭敗,走曲阿,為湖浦尉所擒,初,道子與恭善,欲載出都,面相折數,聞西軍之逼,乃令於兒塘斬之,梟首於東桁也。}

\subsection*{8}

\textbf{桓玄將篡,桓脩欲因玄在脩母許襲之,庾夫人云:「汝等近過我餘年,我養之,不忍見行此事。」}{\footnotesize \textbf{桓氏譜}曰桓沖後娶潁川庾蔑女,字姚。\textbf{晉安帝紀}曰脩少為玄所侮,言論常鄙之,脩深憾焉,密有圖玄之意,脩母曰「靈寶視我如母,汝等何忍骨肉相圖」,脩乃止。}