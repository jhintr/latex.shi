\chapter{衛淇奧詁訓傳第五}

\begin{quoting}\textbf{釋文}鄭、王俱云「紂都之東也」。\end{quoting}

\section{淇奧}

%{\footnotesize 三章、章九句}

\textbf{淇奧,美武公之德也。有文章,又能聽其規諫、以禮自防,故能入相于周,美而作是詩也。}

\textbf{瞻彼淇奧,綠竹猗猗。}{\footnotesize 興也。奧,隈也。綠,王芻也。竹,萹竹也。猗猗,美盛貌。武公質美德盛,有康叔之餘烈。}\textbf{有匪君子,如切如磋,如琢如磨。}{\footnotesize 匪,文章貌。治骨曰切,象曰磋,玉曰琢,石曰磨。道其斈而成也,聽其規諫以自修,如玉石之見琢磨。}\textbf{瑟兮僴兮,赫兮咺兮。}{\footnotesize 瑟,矜莊貌。僴,寬大也。赫,有明德赫赫然。咺,威儀容止宣著也。}\textbf{有匪君子,終不可諼兮。}{\footnotesize 諼,忘也。}

\begin{quoting}奧,三家詩作澳、隩。綠,魯詩作菉。匪,同斐。說文「僴 \texttt{xiàn},武貌」。咺 \texttt{xuān},韓詩作宣。\end{quoting}

\textbf{瞻彼淇奧,綠竹青青。}{\footnotesize 青青,茂盛貌。}\textbf{有匪君子,充耳琇瑩,會弁如星。}{\footnotesize 充耳謂之瑱。琇瑩,美石也。天子玉瑱,諸侯以石。弁,皮弁,所以會髮。箋云會,謂弁之縫中。飾之以玉,皪皪而處,狀似星也。天子之朝服皮弁以日視朝。}\textbf{瑟兮僴兮,赫兮咺兮。有匪君子,終不可諼兮。}

\begin{quoting}青青,唐風杕杜釋文「青,本或作菁」。充耳,見君子偕老注。會 \texttt{kuài}。\end{quoting}

\textbf{瞻彼淇奧,綠竹如簀。}{\footnotesize 簀,積也。}\textbf{有匪君子,如金如錫,如圭如璧。}{\footnotesize 金錫鍊而精,圭璧性有質。箋云圭璧亦琢磨,四者亦道其斈而成也。}\textbf{寬兮綽兮,猗重較兮。}{\footnotesize 寬能容眾。綽,緩也。重較,卿士之車。箋云綽兮,謂仁於施舍。}\textbf{善戲謔兮,不為虐兮。}{\footnotesize 寬緩弘大,雖則戲謔,不為虐矣。箋云君子之德有張有弛,故不常矜莊而時戲謔。}

\begin{quoting}猗,三家詩作倚,依也。\textbf{胡承珙}較在兩旁可倚,人直立稍後,一手可以憑較,俛躬向前,兩手可以憑式。\textbf{馬瑞辰}虐之言劇,謂甚也。\end{quoting}

\section{考槃}

%{\footnotesize 三章、章四句}

\textbf{考槃,刺莊公也。不能繼先公之業,使賢者退而窮處。}{\footnotesize 窮,猶終也。}

\textbf{考槃在澗,碩人之寬。}{\footnotesize 考成、槃樂也。山夾水曰澗。箋云碩,大也。有窮處成樂在於此澗者,形貌大人而寬然有虛乏之色。}\textbf{獨寐寤言,永矢弗諼。}{\footnotesize 箋云寤覺、永長、矢誓、諼忘也。在澗獨寐,覺而獨言,長自誓以不忘君之惡,志在窮處,故云然。}

\begin{quoting}\textbf{朱熹}陳傅良曰「考,扣也,槃,器名」,蓋扣之以節歌,如鼓盆拊缶之為樂也。澗,韓詩作干。\textbf{嚴粲}詩緝曰既寐而寤,既寤而言,皆獨自耳。\end{quoting}

\textbf{考槃在阿,碩人之薖。}{\footnotesize 曲陵曰阿。薖,寬大貌。箋云薖,飢意。}\textbf{獨寐寤歌,永矢弗過。}{\footnotesize 箋云弗過者,不復入君之朝也。}

\begin{quoting}薖 \texttt{kē},同窠。\end{quoting}

\textbf{考槃在陸,碩人之軸。}{\footnotesize 軸,進也。箋云軸,病也。}\textbf{獨寐寤宿,永矢弗告。}{\footnotesize 無所告語也。箋云不復告君以善道。}

\begin{quoting}軸,魯詩作逐,亦云病也。案軸其行也,宿其止也,不告其所之也。\end{quoting}

\section{碩人}

%{\footnotesize 四章、章七句}

\textbf{碩人,閔莊姜也。莊公惑於嬖妾,使驕上僭,莊姜賢而不答,終以無子,國人閔而憂之。}

\textbf{碩人其頎,衣錦褧衣。}{\footnotesize 頎,長貌。錦,文衣也。夫人德盛而尊,嫁則錦衣加褧襜。箋云碩,大也。言莊姜儀表長麗佼好頎頎然。褧,禪也。國君夫人翟衣而嫁,今衣錦者,在塗之所服也,尚之以禪衣,為其文之大著。}\textbf{齊侯之子,衛侯之妻,東宮之妹,邢侯之姨,譚公維私。}{\footnotesize 東宮,齊大子也。女子後生曰妹。妻之姊妹曰姨。姊妹之夫曰私。箋云陳此者,言莊姜容貌既美,兄弟皆正大。}

\begin{quoting}褧 \texttt{jiǒng},韓詩作檾,段注「檾者,枲屬,績檾為衣,是為褧也」。\end{quoting}

\textbf{手如柔荑,}{\footnotesize 如荑之新生。}\textbf{膚如凝脂,}{\footnotesize 如脂之凝。}\textbf{領如蝤蠐,}{\footnotesize 領,頸也。蝤蠐,蝎蟲也。}\textbf{齒如瓠犀,}{\footnotesize 瓠犀,瓠瓣。}\textbf{螓首蛾眉。}{\footnotesize 螓首,顙廣而方。箋云螓,謂蜻蜻也。}\textbf{巧笑倩兮,}{\footnotesize 倩,好口輔。}\textbf{美目盻兮。}{\footnotesize 盻,白黑分。箋云此章說莊姜容貌之美,所宜親幸。}

\begin{quoting}蝤蠐 \texttt{qiú qí}。犀,魯詩作棲。\textbf{陳奐}口輔即靨輔也。\end{quoting}

\textbf{碩人敖敖,說于農郊。}{\footnotesize 敖敖,長貌。農郊,近郊。箋云敖敖,猶頎頎也。說,當作禭,禮、春秋之禭讀皆宜同,衣服曰禭,今俗語然。此言莊姜始來,更正衣服于衛近郊。}\textbf{四牡有驕,朱幩鑣鑣,翟茀以朝。}{\footnotesize 驕,牡貌。幩,飾也,人君以朱纏鑣扇汗,且以為飾。鑣鑣,盛貌。翟,翟車也,夫人以翟羽飾車。茀,蔽也。箋云此又言莊姜自近郊既正衣服,乘是車馬以入君之朝,皆用嫡夫人之正禮,今而不答。}\textbf{大夫夙退,無使君勞。}{\footnotesize 大夫未退,君聽朝於路寢,夫人聽內事於正寢,大夫退,然後罷。箋云莊姜始來時,衛諸大夫朝夕者皆早退,無使君之勞倦者,以君夫人新為妃耦,宜親親之故也。}

\begin{quoting}說,魯詩作稅。茀 \texttt{fú},韓詩作蔽,爾雅釋器「輿革前謂之鞎,後謂之茀,竹前謂之禦,後謂之蔽」。\end{quoting}

\textbf{河水洋洋,北流活活。施罛濊濊,鱣鮪發發,葭菼揭揭。庶姜孽孽,庶士有朅。}{\footnotesize 洋洋,盛大也。活活,流也。罛,魚罟。濊濊,施之水中。鱣,鯉也。鮪,鮥也。發發,盛貌。葭蘆、菼薍也。揭揭,長也。孽孽,盛飾。庶士,齊大夫送女者。朅,武壯貌。箋云庶姜,謂姪娣。此章言齊地廣饒,士女佼好,禮儀之備,而君何為不答夫人。}

\begin{quoting}活活 \texttt{guō}。罛 \texttt{gū},魯詩作罟。濊濊 \texttt{huò}。發發 \texttt{bō}。菼 \texttt{tǎn}。庶,眾也。朅 \texttt{qiè},韓詩作桀,訓健。\end{quoting}

\section{氓}

%{\footnotesize 六章、章十句}

\textbf{氓,刺時也。宣公之時,禮義消亡,淫風大行,男女無別,遂相奔誘,華落色衰,復相棄背,或乃困而自悔,喪其妃耦,故序其事以風焉,美反正、刺淫泆也。}

\textbf{氓之蚩蚩,抱布貿絲。}{\footnotesize 氓,民也。蚩蚩,敦厚之貌。布,幣也。箋云幣者,所以貿買物也。季春始蠶,孟夏賣絲。}\textbf{匪來貿絲,來即我謀。}{\footnotesize 箋云匪非、即就也。此民非來買絲,但來就我,欲與我謀為室家也。}\textbf{送子涉淇,至于頓丘。}{\footnotesize 丘一成為頓丘。箋云子者,男子之通稱。言民誘己,己乃送之,涉淇水至此頓丘,定室家之謀,且為會期。}\textbf{匪我愆期,子無良媒。}{\footnotesize 愆,過也。箋云良,善也。非我心欲過子之期,子無善媒來告期時。}\textbf{將子無怒,秋以為期。}{\footnotesize 將,願也。箋云將,請也。民欲為近期,故語之曰「請子無怒,秋以與子為期」。}

\begin{quoting}氓,周禮注「新徙來者也」。鹽鐵輪「古者市朝而無刀幣,各以其所有易無,抱布貿絲而已」,孔疏「此布非泉,泉不宜抱之也」。\textbf{馬瑞辰}詩當與男子不相識之初則稱氓,約為婚姻則稱子,子者男子美稱也,嫁則稱士,士者夫也,荀子非相篇「處女莫不願得以為士」,是足見立言之序。\end{quoting}

\textbf{乘彼垝垣,以望復關。}{\footnotesize 垝,毀也。復關,君子之近也。箋云前既與民以秋為期,期至,故登毀垣,鄉其所近而望之,猶有廉耻之心,故因復關以託號民,云此時始秋也。}\textbf{不見復關,泣涕漣漣。}{\footnotesize 言其有一心乎君子,故能自悔。箋云用心專者怨必深。}\textbf{既見復關,載笑載言。}{\footnotesize 箋云則笑則言,喜之甚。}\textbf{爾卜爾筮,體無咎言。}{\footnotesize 龜曰卜,蓍曰筮。體,兆卦之體。箋云爾,女也。復關既見此婦人,告之曰「我卜女筮,女宜為室家矣,兆卦之繇無凶咎之辭」,言其皆吉,又誘定之。}\textbf{以爾車來,以我賄遷。}{\footnotesize 賄財、遷徙也。箋云女,女復關也。信其卜筮皆吉,故答之曰「徑以女車來迎我,我以所有財賄徙就女也」。}

\begin{quoting}垝 \texttt{guǐ}。體,齊詩韓詩作履。\end{quoting}

\textbf{桑之未落,其葉沃若。于嗟鳩兮,無食桑葚。于嗟女兮,無與士耽。}{\footnotesize 桑,女功之所起。沃若,猶沃沃然。鳩,鶻鳩也。食桑葚過則醉而傷其性。耽,樂也,女與士耽則傷禮義。箋云桑之未落,謂其時仲秋也,於是時,國之賢者刺此婦人見誘,故于嗟而戒之。鳩以非時食葚,猶女子嫁不以禮,耽非禮之樂。}\textbf{士之耽兮,猶可說也。女之耽兮,不可說也。}{\footnotesize 箋云說,解也。士有百行,可以功過相除,至於婦人無外事,維以貞信為節。}

\begin{quoting}\textbf{陳奐}傳以然字代若字,旄丘傳又以然字代如字。于,韓詩作吁。耽,同酖,\textbf{陳奐}凡樂過其節謂之酖。說,同脫。\end{quoting}

\textbf{桑之落矣,其黃而隕。自我徂爾,三歲食貧。淇水湯湯,漸車帷裳。}{\footnotesize 隕,隋也。湯湯,水盛貌。帷裳,婦人之車也。箋云桑之落矣,謂其時季秋也,復關以此時車來迎己。徂,往也。我自是往之女家,女家乏穀食已三歲貧矣,言此者,明己之悔不以女今貧故也。帷裳,童容也。我乃渡深水,至漸車童容,猶冒此難而往,又明己專心於女。}\textbf{女也不爽,士貳其行。}{\footnotesize 爽,差也。箋云我心於女故無差貳,而復關之行有二意。}\textbf{士也罔極,二三其德。}{\footnotesize 極,中也。}

\begin{quoting}湯 \texttt{shāng}。廣雅「漸 \texttt{jiān},漬也」。\end{quoting}

\textbf{三歲為婦,靡室勞矣。}{\footnotesize 箋云靡,無也。無居室之勞,言不以婦事見困苦。有舅姑曰婦。}\textbf{夙興夜寐,靡有朝矣。}{\footnotesize 箋云無有朝者,常早起夜臥,非一朝然,言己亦不解惰。}\textbf{言既遂矣,至于暴矣。}{\footnotesize 箋云言,我也。遂,猶久也。我既久矣,謂三歲之後見遇浸薄,乃至見酷暴。}\textbf{兄弟不知,咥其笑矣。}{\footnotesize 咥咥然笑。箋云兄弟在家,不知我之見酷暴,若其知之,則咥咥然笑我。}\textbf{靜言思之,躬自悼矣。}{\footnotesize 悼,傷也。箋云靜安、躬身也。我安思君子之遇己無終,則身自哀傷。}

\begin{quoting}言,語詞也。遂,安也。咥 \texttt{xì}。\end{quoting}

\textbf{及爾偕老,老使我怨。}{\footnotesize 箋云及,與也。我欲與女俱至於老,老乎女反薄我,使我怨也。}\textbf{淇則有岸,隰則有泮。}{\footnotesize 泮,坡也。箋云泮,讀為畔,畔,厓也。言淇與隰皆有厓岸以自拱持,今君子放恣心意,曾無所拘制。}\textbf{緫角之宴,言笑晏晏,信誓旦旦。}{\footnotesize 緫角,結髮也。晏晏,和柔也。信誓旦旦然。箋云我為童女未筓結髮晏然之時,女與我言笑晏晏然而和柔,我其以信相誓旦旦耳,言其懇惻款誠。}\textbf{不思其反。}{\footnotesize 箋云反,復也。今老而使我怨,曾不復念其前言。}\textbf{反是不思,亦已焉哉。}{\footnotesize 箋云已焉哉,謂此不可奈何,死生自決之辭。}

\begin{quoting}旦旦,同怛怛,誠懇貌。\end{quoting}

\section{竹竿}

%{\footnotesize 四章、章四句}

\textbf{竹竿,衛女思歸也。適異國而不見答,思而能以禮者也。}

\textbf{籊籊竹竿,以釣于淇。}{\footnotesize 興也。籊籊,長而殺也。釣以得魚,如婦人待禮以成為室家。}\textbf{豈不爾思,遠莫致之。}{\footnotesize 箋云我豈不思與君子為室家乎,君子疏遠己,己無由致此道。}

\begin{quoting}籊 \texttt{dí}。\textbf{陳奐}殺者,纖小之稱。\textbf{王先謙}淇水衛地,此女身在異國,思昔日釣遊之樂而遠莫能致,此賦意。\end{quoting}

\textbf{泉源在左,淇水在右。}{\footnotesize 泉源,小水之源。淇水,大水也。箋云小水有流入大水之道,猶婦人有嫁於君子之禮,今水相與為左右而已,亦以喻己不見答。}\textbf{女子有行,遠兄弟父母。}{\footnotesize 箋云行,道也。女子有道當嫁耳,不以不答而違婦禮。}

\begin{quoting}\textbf{馬瑞辰}按古音右與母為韻,當從唐石經及明監本作「遠兄弟父母」。\end{quoting}

\textbf{淇水在右,泉源在左。巧笑之瑳,佩玉之儺。}{\footnotesize 瑳,巧笑貌。儺,行有節度。箋云己雖不見答,猶不惡君子,美其容貌與禮儀也。}

\begin{quoting}\textbf{何楷}毛詩世本古義曰瑳 \texttt{cuō},說文云「玉色鮮白也」,笑而見齒,其色似之。儺 \texttt{nuó}。\end{quoting}

\textbf{淇水滺滺,檜楫松舟。}{\footnotesize 滺滺,流貌。檜,柏葉松身。楫,所以櫂舟。舟楫相配,得水而行,男女相配,得禮而備。箋云此傷己今不得夫婦之禮。}\textbf{駕言出遊,以寫我憂。}{\footnotesize 出遊,思鄉衛之道。箋云適異國而不見答,其除此憂,維有歸耳。}

\begin{quoting}檜 \texttt{guì}。\textbf{王先謙}前三章衛之淇水,末章則異國之淇水也。\end{quoting}

\section{芄蘭}

%{\footnotesize 二章、章六句}

\textbf{芄蘭,刺惠公也。驕而無禮,大夫刺之。}{\footnotesize 惠公以幼童即位,自謂有才能而驕慢,於大臣但習威儀,不知為政之禮。}

\begin{quoting}\textbf{錢澄之}田間詩學曰觿所以解結,以象智也,智不足則虛佩觿矣,韘所以發矢,以象武也,武不足則虛佩韘矣。\end{quoting}

\textbf{芄蘭之支,}{\footnotesize 興也。芄蘭,草也。君子之德當柔潤溫良。箋云芄蘭柔弱,恆蔓延於地,有所依緣則起,興者,喻幼稚之君任用大臣乃能成其政。}\textbf{童子佩觿。}{\footnotesize 觿,所以解結,成人之佩也。人君治成人之事,雖童子猶佩觿,早成其德。}\textbf{雖則佩觿,能不我知。}{\footnotesize 不自謂無知,以驕慢人也。箋云此幼稚之君雖佩觿與,其才能實不如我眾臣之所知為也。惠公自謂有才能而驕慢,所以見刺。}\textbf{容兮遂兮,垂帶悸兮。}{\footnotesize 容儀可觀,佩玉遂遂然,垂其紳帶悸悸然有節度。箋云容,容刀也。遂,瑞也。言惠公佩容刀與瑞及垂紳帶三尺,則悸悸然行止有節度,然其德不稱服。}

\begin{quoting}芄 \texttt{wán}。支,魯詩作枝。\textbf{王引之}古字多借能為而。\end{quoting}

\textbf{芄蘭之葉,}{\footnotesize 箋云葉,猶支也。}\textbf{童子佩韘。}{\footnotesize 韘,玦,能射御則佩韘。箋云韘之言沓,所以彄沓手指。}\textbf{雖則佩韘,能不我甲。}{\footnotesize 甲,狎也。箋云此君雖佩韘與,其才能實不如我眾臣之所狎習。}\textbf{容兮遂兮,垂帶悸兮。}

\begin{quoting}韘 \texttt{shè}。甲,韓詩作狎。\end{quoting}

\section{河廣}

%{\footnotesize 二章、章四句}

\textbf{河廣,宋襄公母歸于衛,思而不止,故作是詩也。}{\footnotesize 宋桓公夫人,衛文公之妹,生襄公而出,襄公即位,夫人思宋,義不可往,故作詩以自止。}

\textbf{誰謂河廣,一葦杭之。}{\footnotesize 杭,渡也。箋云誰謂河水廣與,一葦加之則可以渡之,喻狹也。今我之不渡,直自不往耳,非為其廣。}\textbf{誰謂宋遠,跂予望之。}{\footnotesize 箋云予,我也。誰謂宋國遠與,我跂足則可以望見之,亦喻近也。今我之不往,直以義不往耳,非為其遠。}

\begin{quoting}杭,魯詩作斻,說文「斻,方舟也」。跂,魯詩齊詩作企,說文「企,舉踵也」。\end{quoting}

\textbf{誰謂河廣,曾不容刀。}{\footnotesize 箋云不容刀,亦喻狹小。船曰刀。}\textbf{誰謂宋遠,曾不崇朝。}{\footnotesize 箋云崇,終也。行不終朝,亦喻近。}

\section{伯兮}

%{\footnotesize 四章、章四句}

\textbf{伯兮,刺時也。言君子行役,為王前驅,過時而不反焉。}{\footnotesize 衛宣公之時,蔡人、衛人、陳人從王伐鄭,伯也為王前驅久,故家人思之。}

\textbf{伯兮朅兮,邦之桀兮。}{\footnotesize 伯,州伯也。朅,武貌。桀,特立也。箋云伯,君子字也。桀,英桀,言賢也。}\textbf{伯也執殳,為王前驅。}{\footnotesize 殳長丈二而無刃。箋云兵車六等,軫也、戈也、人也、殳也、車戟也、酋矛也,皆以四尺為差。}

\begin{quoting}朅 \texttt{qiè},魯詩作偈,見碩人注。\textbf{馬瑞辰}執殳先驅,為旅賁之職。\end{quoting}

\textbf{自伯之東,首如飛蓬。}{\footnotesize 婦人夫不在無容飾。}\textbf{豈無膏沐,誰適為容。}{\footnotesize 適,主也。}

\begin{quoting}適,悦也。\end{quoting}

\textbf{其雨其雨,杲杲出日。}{\footnotesize 杲杲然日復出矣。箋云人言其雨其雨而杲杲然日復出,猶我言伯且來伯且來復不來。}\textbf{願言思伯,甘心首疾。}{\footnotesize 甘,厭也。箋云願,念也。我念思伯,心不能已,如人心嗜欲所貪,口味不能絕也,我憂思以生首疾。}

\begin{quoting}\textbf{王引之}其,猶庶幾也。\textbf{馬瑞辰}杲 \texttt{gǎo} 對杳言,說文「杳,冥也,从日在木下,杲,明也,从日在木上」。又曰甘與苦古以相反為義,故甘草爾雅名為大苦,則甘心亦得訓為苦心,猶言憂心、勞心、痛心也。\end{quoting}

\textbf{焉得諼草,言樹之背。}{\footnotesize 諼草令人志憂。背,北堂也。箋云憂以生疾,恐將危身,欲忘之。}\textbf{願言思伯,使我心痗。}{\footnotesize 痗,病也。}

\begin{quoting}樹,植也。痗 \texttt{mèi}。\end{quoting}

\section{有狐}

%{\footnotesize 三章、章四句}

\textbf{有狐,刺時也。衛之男女失時,喪其妃耦焉。古者國有凶荒,則殺禮而多昬,會男女之無夫家者,所以育人民也。}{\footnotesize 育,生長也。}

\textbf{有狐綏綏,在彼淇梁。}{\footnotesize 興也。綏綏,匹行貌。石絕水曰梁。}\textbf{心之憂矣,之子無裳。}{\footnotesize 之子,無室家者。在下曰裳,所以配衣也。箋云之子,是子也。時婦人喪其妃耦,寡而憂是子無裳,無為作裳者,欲與為室家。}

\begin{quoting}綏綏,韓詩作夊夊,說文「行遲曳夊夊也」。\end{quoting}

\textbf{有狐綏綏,在彼淇厲。}{\footnotesize 厲,深可厲之旁。}\textbf{心之憂矣,之子無帶。}{\footnotesize 帶,所以申束衣。}

\begin{quoting}\textbf{胡承珙}此厲當為瀨之借字,說文「瀨,水流沙上也」,楚辭「石瀨兮淺淺」,是瀨為水流沙石間,上章「石絕水曰梁」為水深之所,次章言厲為水淺之所,三章言側,則在岸矣,立言次序如此。帶,即紳也。\end{quoting}

\textbf{有狐綏綏,在彼淇側。心之憂矣,之子無服。}{\footnotesize 言無室家,若人無衣服。}

\begin{quoting}伐檀傳「側,猶厓也」。\end{quoting}

\section{木瓜}

%{\footnotesize 三章、章四句}

\textbf{木瓜,美齊桓公也。衛國有狄人之敗,出處于漕,齊桓公救而封之,遺之車馬器服焉,衛人思之,欲厚報之而作是詩也。}

\textbf{投我以木瓜,報之以瓊琚。}{\footnotesize 木瓜,楙木也,可食之木。瓊,玉之美者。琚,佩玉名。}\textbf{匪報也,永以為好也。}{\footnotesize 箋云匪,非也。我非敢以瓊琚為報木瓜之惠,欲令齊長以為翫好,結己國之恩也。}

\begin{quoting}好 \texttt{hào}。\end{quoting}

\textbf{投我以木桃,報之以瓊瑤。}{\footnotesize 瓊瑤,美玉。}\textbf{匪報也,永以為好也。}

\textbf{投我以木李,報之以瓊玖。}{\footnotesize 瓊玖,玉名。}\textbf{匪報也,永以為好也。}{\footnotesize 孔子曰「吾於木瓜,見苞苴之禮行」。箋云以果實相遺者必苞苴之,尚書曰「厥苞橘柚」。}

%\begin{flushright}衛國十篇、三十四章、二百三句\end{flushright}