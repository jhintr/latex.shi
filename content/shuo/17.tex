\chapter{傷逝第十七}

\subsection*{1}

\textbf{王仲宣好驢鳴,}{\footnotesize \textbf{魏志}曰王粲,字仲宣,山陽高平人,曾祖龔、父暢皆為漢三公,粲至長安見蔡邕,邕奇之,倒屣迎之曰「此王公孫,有異才,吾不及也,吾家書籍,盡當與之」,避亂荊州,依劉表,以粲貌寢通脫,不甚重之,太祖以從征吳,道中卒。}\textbf{既葬,文帝臨其喪,顧語同遊曰:「王好驢鳴,可各作一聲以送之。」赴客皆一作驢鳴。}{\footnotesize \textbf{按}戴叔鸞母好驢鳴,叔鸞每為驢鳴以說其母,人之所好,儻亦同之。}

\subsection*{2}

\textbf{王濬沖為尚書令,著公服,乘軺車,經黃公酒壚下過,}{\footnotesize \textbf{韋昭漢書注}曰壚,酒肆也,以土為墮,四邊高似壚也。}\textbf{顧謂後車客:「吾昔與嵇叔夜、阮嗣宗共酣飲於此壚,竹林之遊,亦預其末,自嵇生夭、阮公亡以來,便為時所羈紲,今日視此雖近,邈若山河。」}{\footnotesize \textbf{竹林七賢論}曰俗傳若此,潁川庾爰之嘗以問其伯文康,文康云「中朝所不聞,江左忽有此論,皆好事者為之也」。}

\subsection*{3}

\textbf{孫子荊以有才,少所推服,唯雅敬王武子,武子喪時,名士無不至者,子荊後來,臨屍慟哭,賓客莫不垂涕,哭畢,向靈牀曰:「卿常好我作驢鳴,今我為卿作。」體似真聲,賓客皆笑,孫舉頭曰:「使君輩存,令此人死。」}{\footnotesize \textbf{語林}曰王武子葬,孫子荊哭之甚悲,賓客莫不垂涕,既作驢鳴,賓客皆笑,孫曰「諸君不死,而令武子死乎」,賓客皆怒。}

\subsection*{4}

\textbf{王戎喪兒萬子,山簡往省之,王悲不自勝,簡曰:「孩抱中物,何至於此?」王曰:「聖人忘情,最下不及情,情之所鍾,正在我輩。」}{\footnotesize \textbf{王隱晉書}曰戎子綏,欲取裴遁女,綏既蚤亡,戎過傷痛,不許人求之,遂至老無敢取者。}\textbf{簡服其言,更為之慟。}{\footnotesize 一說是王夷甫喪子,山簡弔之。}

\subsection*{5}

\textbf{有人哭和長輿曰:「峨峨若千丈松崩。」}

\subsection*{6}

\textbf{衛洗馬以永嘉六年喪,謝鯤哭之,感動路人,}{\footnotesize \textbf{永嘉流人名}曰玠以六年六月二十日亡,葬南昌城許徵墓東,玠之薨,謝幼輿發哀於武昌,感慟不自勝,人問「子何卹而致哀如是」,答曰「棟梁折矣,何得不哀」。}\textbf{咸和中,丞相王公教曰:「衛洗馬當改葬,此君風流名士,海內所瞻,可脩薄祭,以敦舊好。」}{\footnotesize \textbf{玠別傳}曰玠咸和中改遷於江寧,丞相王公教曰「洗馬明當改葬,此君風流名士,海內民望,可脩三牲之祭,以敦舊好」。}

\subsection*{7}

\textbf{顧彥先平生好琴,及喪,家人常以琴置靈牀上,張季鷹往哭之,不勝其慟,遂徑上牀,鼓琴,作數曲竟,撫琴曰:「顧彥先頗復賞此不?」因又大慟,遂不執孝子手而出。}

\subsection*{8}

\textbf{庾亮兒遭蘇峻難遇害,諸葛道明女為庾兒婦,既寡,將改適,}{\footnotesize 亮子會、會妻文彪並已見上。}\textbf{與亮書及之,亮答曰:「賢女尚少,故其宜也,感念亡兒,若在初沒。」}

\subsection*{9}

\textbf{庾文康亡,何揚州臨葬云:「埋玉樹著土中,使人情何能已已。」}{\footnotesize \textbf{搜神記}曰初,庾亮病,術士戴洋曰「昔蘇峻事,公於白石祠中許賽車下牛,從來未解,為此鬼所考,不可救也」,明年,亮果亡。\textbf{靈鬼志謠徵}曰文康初鎮武昌,出石頭,百姓看者於岸歌曰「庾公上武昌、翩翩如飛鳥,庾公還揚州,白馬牽旒旐」,又曰「庾公初上時,翩翩如飛鵶,庾公還揚州,白馬牽旐車」,後連徵不入,尋薨,下都葬焉。}

\subsection*{10}

\textbf{王長史病篤,寢臥鐙下,轉麈尾視之,歎曰:「如此人,曾不得四十。」及亡,劉尹臨殯,以犀柄麈尾著柩中,因慟絕。}{\footnotesize \textbf{濛別傳}曰濛以永和初卒,年三十九,沛國劉惔與濛至交,及卒,惔深悼之,雖友于之愛,不能過也。}

\subsection*{11}

\textbf{支道林喪法虔之後,精神霣喪,風味轉墜,}{\footnotesize \textbf{支遁傳}曰法虔,道林同學也,儁朗有理義,遁甚重之。}\textbf{常謂人曰:「昔匠石廢斤於郢人,}{\footnotesize \textbf{莊子}曰郢人堊漫其鼻端若蠅翼,使匠石運斤斲之,堊盡而鼻不傷,郢人立不失容。}\textbf{牙生輟絃於鍾子,}{\footnotesize \textbf{韓詩外傳}曰伯牙鼓琴,鍾子期聽之,方鼓琴,志在太山,子期曰「善哉乎鼓琴,巍巍乎若太山」,莫景之間,志在流水,子期曰「善哉乎鼓琴,洋洋乎若流水」,鍾子期死,伯牙擗琴絕絃,終身不復鼓之,以為在者無足為之鼓琴也。}\textbf{推己外求,良不虛也,冥契既逝,發言莫賞,中心蘊結,余其亡矣。」卻後一年,支遂殞。}

\subsection*{12}

\textbf{郗嘉賓喪,左右白郗公「郎喪」,既聞,不悲,因語左右:「殯時可道。」公往臨殯,一慟幾絕。}{\footnotesize \textbf{中興書}曰超年四十一,先愔卒,超所交友,皆一時俊乂,及死之日,貴賤為誄者四十餘人。\textbf{續晉陽秋}曰超黨戴桓氏,為其謀主,以父愔忠於王室,不令知之,將亡,出一小書箱付門生,云「本欲焚此,恐官年尊,必以傷愍為斃,我亡後,若大損眠食,則呈此箱」,愔後果慟悼成疾,門生乃如超旨,則與桓溫往反密計,愔見即大怒曰「小子死恨晚」,後不復哭。}

\subsection*{13}

\textbf{戴公見林法師墓,}{\footnotesize \textbf{支遁傳}曰遁太和元年終于剡之石城山,因葬焉。}\textbf{曰:「德音未遠,而拱木已積,冀神理緜緜,不與氣運俱盡耳。」}{\footnotesize \textbf{王珣法師墓下詩序}曰余以寧康二年,命駕之剡石城山,即法師之丘也,高墳鬱為荒楚,丘隴化為宿莽,遺跡未滅,而其人已遠,感想平昔,觸物悽懷。其為時賢所惜如此。}

\subsection*{14}

\textbf{王子敬與羊綏善,綏清淳簡貴,為中書郎,少亡,}{\footnotesize 綏已見。}\textbf{王深相痛悼,語東亭云:「是國家可惜人。」}

\subsection*{15}

\textbf{王東亭與謝公交惡,}{\footnotesize \textbf{中興書}曰珣兄弟皆壻謝氏,以猜嫌離婚,太傅既與珣絕婚,又離妻,由是二族遂成仇釁。}\textbf{王在東聞謝喪,便出都詣子敬,道欲哭謝公,子敬始臥,聞其言,便驚起曰:「所望於法護。」}{\footnotesize 法護,珣小字。}\textbf{王於是往哭,督帥刁約不聽前,曰:「官平生在時,不見此客。」王亦不與語,直前,哭甚慟,不執末婢手而退。}{\footnotesize 末婢,謝琰小字。琰,字瑗度,安少子,開率有大度,為孫恩所害,贈侍中、司空。}

\subsection*{16}

\textbf{王子猷、子敬俱病篤,而子敬先亡,}{\footnotesize 獻之以泰元十三年卒,年四十五。}\textbf{子猷問左右:「何以都不聞消息,此已喪矣。」語時了不悲,便索輿來奔喪,都不哭,子敬素好琴,便徑入坐靈牀上,取子敬琴彈,弦既不調,擲地云:「子敬,子敬!人琴俱亡。」因慟絕良久,月餘亦卒。}{\footnotesize \textbf{幽明錄}曰泰元中,有一師從遠來,莫知所出,云「人命應終,有生樂代者,則死者可生,若逼人求代,亦復不過少時」,人聞此,咸怪其虛誕,王子猷、子敬兄弟特相和睦,子敬疾屬纊,子猷謂之曰「吾才不如弟,位亦通塞,請以餘年代弟」,師曰「夫生代死者,以己年限有餘,得以足亡者耳,今賢弟命既應終,君侯算亦當盡,復何所代」,子猷先有背疾,子敬疾篤,恆禁來往,聞亡,便撫心悲惋,都不得一聲,背即潰裂,推師之言,信而有實。}

\subsection*{17}

\textbf{孝武山陵夕,王孝伯入臨,告其諸弟曰:「雖榱桷惟新,便自有黍離之哀。」}{\footnotesize \textbf{中興書}曰烈宗喪,會稽王道子執政,寵幸王國寶,委以機任,王恭入赴山陵,故有此歎。}

\subsection*{18}

\textbf{羊孚年三十一卒,桓玄與羊欣書曰:「賢從情所信寄,暴疾而殞,}{\footnotesize 孚已見。\textbf{宋書}曰欣,字敬元,太山南城人,少懷靜默,秉操無競,美姿容,善笑言,長於草隸。\textbf{羊氏譜}曰孚即欣從祖。}\textbf{祝予之歎,如何可言。」}{\footnotesize \textbf{公羊傳}曰顏淵死,子曰「噫!天喪予」,子路亡,子曰「噫!天祝予」。\textbf{何休}曰祝者,斷也,天將亡夫子耳。}

\subsection*{19}

\textbf{桓玄當篡位,語卞鞠云:}{\footnotesize 卞範已見。}\textbf{「昔羊子道恆禁吾此意,今腹心喪羊孚,爪牙失索元,}{\footnotesize \textbf{索氏譜}曰元,字天保,燉煌人,父緒,散騎常侍,元歷征虜將軍、歷陽太守。\textbf{幽明錄}曰元在歷陽,疾病,西界一年少女子姓某,自言為神所降,來與元相聞,許為治護,元性剛直,以為妖惑,收以付獄,戮之於市中,女臨死曰「卻後十七日,當令索元知其罪」,如期,元果亡。}\textbf{而悤悤作此詆突,詎允天心?」}