\section{雪山经}

\begin{center}Hemavata Sutta\end{center}\vspace{1em}

\subsection\*{\textbf{153}}

\textbf{「今天是十五布萨日,」七岳夜叉\footnote{此经旧译见杂阿含经第 1329 经、别译杂阿含经第 328 经。七岳、雪山等译名从别译杂阿含经。}说,「圣洁的夜晚降临,\\}
\textbf{「噫!我们去见乔达摩,享有盛名的大师!」}

“Ajja pannaraso uposatho, \textit{(iti Sātāgiro yakkho)} dibbā ratti upaṭṭhitā;\\
anomanāmaṃ satthāraṃ, handa passāma Gotamaṃ”. %\hfill\textcolor{gray}{\footnotesize 1}

\begin{enumerate}\item 缘起为何?缘起为问题的主导。世尊为雪山所问,说了「世间在六中生起」等。这里,「今天是十五」等为七岳所说,「七岳夜叉说」等为结集者所说,「心意是否」等为雪山所说,「世间在六中」等为世尊所说,这一切汇集后,即被称为「雪山经」。有些人则称为「七岳经」。
\item 这里,「今天是十五」等为初颂,其缘起为:即此贤劫中,迦叶世尊等正觉在人寿二万岁时投生,住世一万六千年后般涅槃,人们便以大供养而殓葬。其舍利未作散布,如金块般聚在一起。因为这是多寿诸佛的法性。而少寿的诸佛因为不为更多的世人所见便入般涅槃,所以作了舍利的供养后,为怜悯「各处的世人将生福德」而决意「散布舍利」,因此他们的舍利如金粉般散布,比如我们的世尊。
\item 人们将彼世尊的舍利聚在一起后,教人建了支提,高一由旬,围亦如之。每隔一牛呼便有四门。国王松鸦把持一门,他的儿子名为「持地」把持一门,以将军为首的众大臣把持一门,以商人为首的国人把持一门。通体赤金所造,砖瓦也由种种宝物所造,与黄金的色泽相似,均价值百千。他们教人以雌黄、雄黄作土,以香油作水,建了这支提。
\item 如是,当支提建好时,两个族姓子好友便离家,在作为(迦叶世尊)亲传弟子的长老们跟前出了家。因为对于多寿的诸佛,唯有亲传弟子能给予出家、具足、依止,而非其他。其后,二族姓子便问:「尊者!教法内有几种责任?」长老们说:「两种责任,居住的责任与学习的责任。」
\item 这里,已出家的族姓子在阿阇黎与和尚跟前住满五年,履行了种种义务,练达于波罗提木叉及二三诵的经文,获取了业处,以除去对俗家及人群的执著而进入林野,为证得阿罗汉而努力、精进,此即\textbf{居住的责任}。而以自身的能力,或精通一部,或二部乃至五部后,从圣典及义理致力于令教法极明晰,此即\textbf{学习的责任}。
\item 于是,这二族姓子说「两种责任中,唯居住的责任为胜,但我们还是孩童,等年长时再圆满居住的责任,先履行学习的责任吧」,便开始学习。二人天性聪慧,不久就完成了对所有佛语的知晓,且于律极具裁断的善巧。依于二人的学习便聚集了随从,而依于随从则聚集了利养,便一一有了五百比丘的随从。二人阐明大师的教法而住,宛若佛时。
\item 那时,有二比丘住于村落,一为法说者,一为非法说者。非法说者凶恶、粗鄙、饶舌,其过行为另一人闻知。他便即责备道:「朋友!你这行为于教法失当。」他打断说:「你何所见、何所闻?」另一个说:「持律者自会知晓。」非法说者知道「如果持律者来裁断此事,那我在教法内将无立足之处」,欲建立自己的派别,立刻带了资具,去到那两位长老处,布施了沙门的资具后,开始依他们而住,并对他们作着一切给侍,像是要恭敬地履行种种义务一般。
\item 之后一天,他在前去给侍、顶礼,当被二人遣散时,仍然站着。二长老问他:「是否有什么要说的?」他说:「唯!尊者!我与一比丘就过行存在诤论,他如果来此告发此事,请不要按裁断来裁断。」二长老说:「不按裁断来裁断发起之事不妥。」他说:「若如是做,尊者!我在教法内无处立足,让此恶归我,你们莫要裁断!」二人受他逼迫,便同意了。
\item 他得了二人的认可,便又回到原来的住处,想「一切都在持律者跟前了结了」,愈加蔑视那法说者,举止粗鄙。法说者想「此事无疑」,立刻出发,去到二长老的一千随从比丘处,说:「朋友!难道不应如法裁断发起之事吗?或未经发起,则应宣示彼此的罪过而和合。然而,这二长老既未裁断此事,又不令和合,这算什么?」他们听后,也保持默然,想「阿阇黎们肯定知晓些什么」。随后,非法说者得了机会,逼迫法说者:「你先前说『持律者自会知晓』,现在去向这二持律者宣告此事吧!」并说「从今以后,你就输了,别再回这住处」,便离开了。
\item 随后,法说者去到二长老处,说「你们罔顾教法,却顾念『给侍、满足我等』之人,不守护教法,却守护人,从今以后,你们不适于作出裁断,今天,迦叶世尊已般涅槃」,大声号哭「大师的教法已灭」,悲泣着离开。
\item 于是,这些比丘意有悚惧,生起恶作:「我们守护着人,却把教法之宝弃诸沟渠了。」他们即因此恶作,由意乐的败坏,死后不能投生于天界,一个阿阇黎投生于喜马拉雅的雪山中,名为\textbf{雪山}夜叉,第二个阿阇黎投生于中国的七山之中,名为\textbf{七岳}。二人的那些随从比丘仍追随着二者,无法投生于天界,成为二者的随从夜叉而投生。但布施他们资具的在家人投生到了天界。雪山与七岳位列二十八夜叉大将之中,有大威力,为夜叉王。
\item 这是夜叉大将的法性:每月之中的八天,诸天会为裁断法而聚集于喜马拉雅雄黄之原的具龙亭,他们应于此集会。于是,七岳与雪山在那集会中见到了对方,便即相认,问了各自的投生处「亲爱的!你投生在哪里」,便起追悔:「亲爱的!我们毁了!先前行了二万年的沙门法,一旦依于恶友,投生于夜叉的胎中,而我们的资具布施者倒生于欲界天中。」于是,七岳说:「先生!喜马拉雅素称奇异希有,若看见或听闻什么希有之事,请告知我!」雪山也说:「先生!中国素称奇异希有,若看见或听闻什么希有之事,请也告知我!」如是,这二友人彼此订下规约,唯独不回避这番缘起,当如是住时,便度过了一佛的间隔,大地隆起一由旬又三牛呼。
\item 于是,我们的菩萨在燃灯的足下已立下誓愿,直至毗输安多罗本生圆满了波罗蜜,投生到兜率天的居处,在那里住至寿尽,如法句的因缘中所说的方式为诸天祈请,观察了五大观察\footnote{五大观察:据菩提比丘注 727,见于\textbf{佛种姓}义注,即观察时间、洲、地方、家族、母亲等。},在向诸天宣告后,当三十二种征兆\footnote{三十二种征兆:据菩提比丘注 728,见于\textbf{自说}义注及\textbf{如是语}义注等,即一万世界大地震动、放大光明、盲者能视、聋者能听、哑者能言、偻者背直、跛者能行等等。}转起时,便获取结生,震动了一万个世界。而这些夜叉王见到后,竟不知道发生这些是何原因,有些因忙于游戏,甚至都没看见。在出生、离家、觉悟时也如是。但在法轮转起,世尊与五人相谈,转起三转十二行相之最上法轮时,他们中唯有七岳首先见到了大地震动之征兆及诸神变。了知了发生这些的原因后,便与随从一起去到世尊处,听了法的开示,却未证得任何殊胜。为什么?因为他在听闻法时,忆念起雪山「我的朋友来了没有」,观察会众,当未见到他时,想到「我的朋友被蒙蔽了,没有听闻世尊如是多彩、辩才的法的开示」,便心生散乱。而世尊在日落时分也未令开示完结。
\item 于是,七岳想「带上朋友后,和他一起来听闻法的开示」,造了象车、马车、金翅鸟车等,为五百夜叉随从,朝着喜马拉雅进发。此时,雪山亦然。因为在结生、出生、出离、觉悟、涅槃时,三十二种征兆现已即逝,并未久住,但在法轮转起时,这些殊胜现已,久住后方灭,所以他在喜马拉雅见到这奇异现起后,想「自我出生,此山从未如是喜人、如往常一样,噫!现在我带上我的朋友,和他一起领纳这花的灿烂」,也如是朝着中国出发。
\item 他们二人在王舍城的上方遇到后,问起彼此前来的原因。雪山说:「先生!自我出生,此山从未如是喜人、如往常一样,树上盛开非时之花,所以想和你一起领纳这花的灿烂,我即前来。」七岳说:「可你,先生!知道什么原因而生此非时花盛开的神变吗?」「我并不知,先生!」「先生!这神变不独于雪山,而是生起于一万世界之中。等正觉者出现世间,今天转起法轮,以此原因。」如是,七岳对雪山谈起佛陀出世后,想带他到世尊跟前,便说了此颂。有些人则以「世尊住乔达摩支提时,他如是说」而说此颂。
\item 这里,\textbf{今天},即此日夜,从半月数来为\textbf{十五},由应当遵守为\textbf{布萨}。或于三布萨日中,今天是十五布萨日,非十四布萨日,非和合布萨日。或因为「布萨」一词表示诵波罗提木叉、八支、斋戒、概念、日期等众多意义,如\begin{quoting}来!朋友 Kappina!我们将去布萨。(律藏)\end{quoting}等处「布萨」一词表示诵波罗提木叉,\begin{quoting}毗舍佉!遵守如是具足八支的布萨……(增支部第 8:43 经)\end{quoting}等处表示离杀生等的八支,\begin{quoting}清净者总在禁食,\\清净者总在布萨。(中部·布经第 79 段)\end{quoting}等处表示斋戒,\begin{quoting}象王名为布萨。(中部·贤愚经第 258 段)\end{quoting}等处表示概念,\begin{quoting}在十五布萨日,沐头者……(中部·贤愚经第 256 段)\end{quoting}等处表示日期,所以,在遮止其它意义后,唯限定为阿沙陀月\footnote{阿沙陀月 \textit{Āsāḷhī}:即现在的六~七月。}的满月之日而说「今天是十五布萨日」,即以「半月的第一天、第二天」来计数时,今天是第十五天之义。
\item 神圣的存在为圣洁,于此存有圣洁即\textbf{圣洁}之义。这是什么?色。因为当晚,整个阎浮提都被从一万世界前来集会的诸天之身、衣、璎珞、宫殿的光辉,以及除去云翳等遮染的月光所庄严,且尤其为最清净之天的世尊的身光所庄严,因此说「圣洁的夜晚降临」。
\item 如是,以称赞、说明夜的功德令朋友之心生起净喜,在谈起佛陀出世后,便说「噫!我们去见乔达摩,享有盛名的大师」。这里,以非低、非劣、圆满一切行相的功德所得之名为\textbf{享有盛名}。因为他如\begin{quoting}觉悟诸谛为佛陀,令人类觉悟为佛陀。(大义释)\end{quoting}等方法,以非低之功德而得名「佛陀」,且如\begin{quoting}断除贪为世尊,断除嗔为世尊。(大义释)\end{quoting}等方法,以非低之功德而得名。「阿罗汉、正等正觉者、明行足」等亦同。以教授天人现法等的义利「应舍弃此,应受持此而行持」为\textbf{大师}。且以\begin{quoting}大师即世尊、商队的领队,好比商队的领队度脱有情于沙漠。(大义释)\end{quoting}等在义释中所说的方法为大师。此即享有盛名的大师。
\item \textbf{噫},即表确定的不变词。\textbf{我们去见},即连带自己与他一起的现在时。\textbf{乔达摩},即乔达摩家族。这说的是什么?莫作疑惑「是大师、不是大师」,完全确定后,来!我们去见乔达摩!\end{enumerate}

\subsection\*{\textbf{154}}

\textbf{「如此之人的心意,」雪山夜叉说,「是否善待一切生命?\\}
\textbf{「于诸可意和不可意,他的思惟是否受控?」}

“Kacci mano supaṇihito, \textit{(iti Hemavato yakkho)} sabbabhūtesu tādino;\\
kacci iṭṭhe aniṭṭhe ca, saṅkapp’assa vasīkatā”. %\hfill\textcolor{gray}{\footnotesize 2}

\begin{enumerate}\item 如是说已,雪山想「这七岳在说『享有盛名的大师』时,阐明了他的一切知性,而在世间,一切知者为难得,世间恰是由被认为是一切知者的富楼那等辈所祸害,他若是一切知者,则将证如此之相,因此,我将如是检查之」,为问如此之相,说了此颂。
\item 这里,\textbf{是否},即问。\textbf{心意},即心。\textbf{善待},即善加安置、不动摇、不动荡。\textbf{如此之人}\footnote{如此之人:菩提比丘英译作 impartial one,即平等之人。义注在说「抑或不然」时,是将「如此之人的 \textit{tādino}」一词拆解为「如此、不然 \textit{tādi no}」两词。},即证得如此之相的善人。或者,此即是所问:「你的这位大师对一切生命都如此,抑或不然?」\textbf{可意和不可意},即如是样类的所缘。\textbf{思惟},即寻。
\item 这说的是什么?你所说的这位大师,这证得如此之相的善人的心意是否善待一切生命?抑或只是在不得动摇之缘时看起来像似善待?或者,你的这位大师是否以平等心对一切生命都如此,抑或不然?且于诸可意和不可意的所缘,这些会以贪嗔之力生起的思惟,它们是否受控?抑或有时也依彼等之力转起?\end{enumerate}

\subsection\*{\textbf{155}}

\textbf{「如此之人的心意,」七岳夜叉说,「善待一切生命,\\}
\textbf{「并且于诸可意和不可意,他的思惟受控。」}

“Mano c’assa supaṇihito, \textit{(iti Sātāgiro yakkho)} sabbabhūtesu tādino;\\
atho iṭṭhe aniṭṭhe ca, saṅkapp’assa vasīkatā”. %\hfill\textcolor{gray}{\footnotesize 3}

\begin{enumerate}\item 随后,七岳由对世尊一切知性的确定,为认可所有一切知的功德,说了此颂。
\item 这里,\textbf{善待},即善加安置,以不妨碍之义如大地,以善住、不动摇之义如须弥山,以不为四种魔罗\footnote{四种魔罗:即作为天子、烦恼、蕴、死的四种魔罗。}及其他论说之众所撼动,则如因陀柱。且现在,由具足一切行相,住于一切知性之世尊的心意能够善待、不动摇,这并非异事。即便在俱贪等时生为畜生,投生于六牙象族之中,为毒箭射中亦能不动摇,对此杀戮者亦不冒犯,而是截断它自己的牙后布施。同样,生为大猿,即便为巨石击中头部,仍为其指路。同样,生为 Vidhura 智者,即便被抓住双脚掷于六十由旬的黑山崖下,仍为此夜叉的义利而开示法。所以,七岳说「他的心意善待」完全恰当。
\item \textbf{如此之人},即证得如此之相的善人的心意,善待\textbf{一切生命},即一切有情,非仅当不得缘之义。这里,当知对于世尊有五种如此之相,如说:\begin{quoting}世尊以五行相而如此,于可意、不可意如此,以已舍弃而如此,以已解脱而如此,以已度而如此,以彼义释而如此。世尊如何于可意、不可意如此?当有利养时,世尊如此……(大义释)\end{quoting}如是等一切,当以在义释中所说的方法把握,且利养等当以其「大义注」\footnote{大义注 \textit{Mahā-aṭṭhakathā}:当为古注。}中所详述的方法了知。
\item 且在这「此即是所问:你的这位大师对一切生命都如此,抑或不然」的选择中,我们的大师对一切生命以平等心都如此之义。因为这世尊以欲为集乐除苦而于一切有情心怀平等,如待自身,如此待他人,如待母亲摩诃摩耶,如此待学童女罗望子,如待父亲净饭,如此待 Suppabuddha\footnote{学童女罗望子、Suppabuddha 事,见于\textbf{法句}义注。},如待儿子罗睺罗,如此待众杀戮者提婆达多、护财、央掘魔罗等,待俱有天的世间也如此。所以,七岳说「如此之人……一切生命」完全恰当。
\item 而在\textbf{并且于诸可意和不可意}中,其义当如是见:任何可意或不可意的所缘,以一切行相,于此会以贪嗔之力生起的思惟,由以无上的道舍断之故而受控,不再依彼等之力转起。因为彼世尊为思惟不污浊者、心善解脱者、慧善解脱者。
\item 且此中,以善待之意来说无有不如理作意。他应于一切生命以可意不可意修习,是从有情与行来说两种所缘。以思惟的控制,是从他于此所缘无有作意来说烦恼的舍断。且以善待之意来说意正行之清净,以于一切生命如此来说身正行之清净,以思惟之控制来说以寻为根本之言语的语正行之清净。同样,当知以善待之意来说无有贪等一切过失,以于一切生命如此来说慈等功德的善性,以思惟之控制来说于厌逆中作不厌逆想等类的圣神变,且以此来说一切知性。\end{enumerate}

\subsection\*{\textbf{156}}

\textbf{「是否不取不与物?」雪山夜叉说,「是否于生命自制?\\}
\textbf{「是否离于放逸?是否不疏忽禅那?」}

“Kacci adinnaṃ nādiyati, \textit{(iti Hemavato yakkho)} kacci pāṇesu saññato;\\
kacci ārā pamādamhā, kacci jhānaṃ na riñcati”. %\hfill\textcolor{gray}{\footnotesize 4}

\begin{enumerate}\item 如是,雪山先前只以意门问如此之性并得到他的肯定,为求确认,现在则以三门发问,或者先前只简略地问身语意门的清净并得到他的肯定,仍是为求确认,现在则详细地发问,说了此颂。
\item 这里,为了结颂的方便,首先问远离不与取。\textbf{离于放逸},即从心对种种五欲的舍遣来问远离非梵行。文本或作 ārā pamadamhā,即是说离于女人。\textbf{不疏忽禅那},即以此来问这三种身恶行远离的力量。因为修习禅那者的远离是强有力的。\end{enumerate}

\subsection\*{\textbf{157}}

\textbf{「他不取不与物,」七岳夜叉说,「并于生命自制,\\}
\textbf{「离于放逸,佛陀不疏忽禅那。」}

“Na so adinnaṃ ādiyati, \textit{(iti Sātāgiro yakkho)} atho pāṇesu saññato;\\
atho ārā pamādamhā, Buddho jhānaṃ na riñcati”. %\hfill\textcolor{gray}{\footnotesize 5}

\begin{enumerate}\item 于是,因为世尊不仅于现在,而是于过去长时已远离不与取等,且以彼彼远离之威力而得彼彼大人相,俱有天的世间也以「沙门乔达摩远离不与取」等方法表示赞叹,所以,七岳以清晰的言语作狮子吼,说了此颂。
\item 其语义明了。此颂的第三句也有两种文本。而第四句中,\textbf{不疏忽禅那},当知为于禅那不令空无、不舍弃之义。\end{enumerate}

\subsection\*{\textbf{158}}

\textbf{「是否不妄语?」雪山夜叉说,「是否言路不粗鲁?\\}
\textbf{「是否不语中伤?是否不说绮语?」}

“Kacci musā na bhaṇati, \textit{(iti Hemavato yakkho)} kacci na khīṇabyappatho;\\
kacci vebhūtiyaṃ nāha, kacci samphaṃ na bhāsati”. %\hfill\textcolor{gray}{\footnotesize 6}

\begin{enumerate}\item 如是,在得知身门清净后,现在,为问语门清净而说此颂。
\item 此中,以散布\footnote{不粗鲁 \textit{na khīṇa°} 的原文费解,义注给出的异读为 nākhīṇa°,Norman 据此认为 CPD 的解释可从,即作 na ākhīṇa°,而 ākhīṇa/āskīrṇa < ā + √stṛ,故此处译作「散布」。}为粗鲁,即伤害、迫害之义。言语之方式为言路,粗鲁的言路即\textbf{言路粗鲁},在以「不」字遮止后问「言路不粗鲁」,即是说「不恶口」。文本或作 nākhīṇabyappatho,即言语非不灭尽之义,因为恶语于他人心中不灭尽而住,即是说他是否不是如此之语者。以毁坏为亡失,以造成毁坏为\textbf{中伤},即两舌的同义语,因为它以有情间彼此分裂造成亡失。其余之义自明。\end{enumerate}

\subsection\*{\textbf{159}}

\textbf{「他不妄语,」七岳夜叉说,「且言路不粗鲁,\\}
\textbf{「不语中伤,以智慧讲说义利。」}

“Musā ca so na bhaṇati, \textit{(iti Sātāgiro yakkho)} atho na khīṇabyappatho;\\
atho vebhūtiyaṃ nāha, mantā atthañ ca bhāsati”. %\hfill\textcolor{gray}{\footnotesize 7}

\begin{enumerate}\item 于是,因为世尊不仅于现在,而是于过去长时已远离妄语等,且以彼彼远离之威力而得彼彼大人相,俱有天的世间也以「沙门乔达摩远离妄语」等方法表示赞叹,所以,七岳以清晰的言语作狮子吼,说了此颂。
\item 这里,\textbf{妄},即误陈所见等欺骗他人之语,他不语此。而第二句中,文本以第一种方式作 na khīṇabyappatho,以第二种方式作 nākhīṇabyappatho。第四句中,\textbf{智慧}即慧。因为世尊以此智慧确定已,唯\textbf{讲说义利}、不离义利之语,不说绮语,诸佛无有以无智为前导的离义之语,所以说「以智慧讲说义利」。其余之义自明。\end{enumerate}

\subsection\*{\textbf{160}}

\textbf{「是否不味著爱欲?」雪山夜叉说,「是否心不污浊?\\}
\textbf{「是否超越愚痴?是否于法具眼?」}

“Kacci na rajjati kāmesu, \textit{(iti Hemavato yakkho)} kacci cittaṃ anāvilaṃ;\\
kacci mohaṃ atikkanto, kacci dhammesu cakkhumā”. %\hfill\textcolor{gray}{\footnotesize 8}

\begin{enumerate}\item 如是,在得知语门清净后,现在,为问意门清净而说此颂。
\item 这里,\textbf{爱欲},即事欲,以问「于此不以烦恼欲而味著」来问不贪求性。以问\textbf{不污浊},就以嗔恚而起的污浊性,来问不嗔恚性。以问\textbf{超越愚痴},来问超越愚人以之持守邪见之愚痴的正见性。以问\textbf{于法具眼},是想到「即便三门清净,也非一切知者」,来问于一切法不受限之智眼的一切知性,或问五眼各自于五眼境域诸法的一切知性。\end{enumerate}

\subsection\*{\textbf{161}}

\textbf{「他不味著爱欲,」七岳夜叉说,「且心不污浊,\\}
\textbf{「超越一切愚痴,佛陀于法具眼。」}

“Na so rajjati kāmesu, \textit{(iti Sātāgiro yakkho)} atho cittaṃ anāvilaṃ;\\
sabbamohaṃ atikkanto, Buddho dhammesu cakkhumā”. %\hfill\textcolor{gray}{\footnotesize 9}

\begin{enumerate}\item 于是,因为世尊尚未证得阿罗汉,即由以阿那含道舍断了对爱欲的贪染与嗔,既不味著爱欲,也不以嗔恚污浊其心,且由以须陀洹道舍断了作为邪见之缘的覆蔽正见的愚痴,而超越愚痴,并依自身于诸谛现等觉已,便得究竟解脱之名「佛陀」及所说的诸眼,所以,七岳为布告其意门清净及一切知性,说了此颂。\end{enumerate}

\subsection\*{\textbf{162}}

\textbf{「是否具足明?」雪山夜叉说,「是否行为清净?\\}
\textbf{「是否诸漏已尽?是否无有再有?」}

“Kacci vijjāya sampanno, \textit{(iti Hemavato yakkho)} kacci saṃsuddhacāraṇo;\\
kacci’ssa āsavā khīṇā, kacci n’atthi punabbhavo”. %\hfill\textcolor{gray}{\footnotesize 10}

\begin{enumerate}\item 如是,雪山在得知了世尊的三门清净及一切知性后,欢喜踊跃,以过去生中多闻明晰之慧而不羁绊于言路,欲闻奇异希有的一切知之功德,说了此颂。
\item 这里,\textbf{具足明},即以此问知见的成就。\textbf{行为清净},即以此问行的成就。且此中为协韵律,拉长而说 cā 字,即 saṃsuddhacaraṇo 之义。\textbf{诸漏已尽},即以此问当以知行之成就而证得的被称为漏尽的第一涅槃界的成就。\textbf{无有再有},即以此问第二涅槃界成就之能力,或在以省察智了知了最上安息的成就后,问其住立性。\end{enumerate}

\subsection\*{\textbf{163}}

\textbf{「既具足明,」七岳夜叉说,「且行为清净,\\}
\textbf{「一切漏已尽,他无有再有。」}

“Vijjāya c’eva sampanno, \textit{(iti Sātāgiro yakkho)} atho saṃsuddhacāraṇo;\\
sabba’ssa āsavā khīṇā, n’atthi tassa punabbhavo”. %\hfill\textcolor{gray}{\footnotesize 11}

\begin{enumerate}\item 随后,这些以\begin{quoting}我(忆念)种种前世……\end{quoting}等方法在(中部)怖畏经中所说的三种明,以\begin{quoting}当心如是等持……至于不动时,他将心导向知见。\end{quoting}等方法在(长部)阿摩昼经中所说的八种明,因为世尊以具足一切行相而具足此等,且具足在如是指出\begin{quoting}于此,摩诃男!圣弟子具足戒,守护根门,饮食知量,常事醒觉,具足七善法,于增上心、现法乐住的四种禅那随欲可得。\end{quoting}后,以\begin{quoting}摩诃男!如何圣弟子具足戒……\end{quoting}等方法在(中部)有学经中说明的十五类行,且其对于世尊,因为以舍断一切随烦恼而极清净。又因为此欲漏等的四漏,彼等一切与其随从、习气对于世尊都已灭尽。又因为以此明行之成就而成漏尽已,世尊经省察「现在无有再有」而住,所以,七岳对世尊之一切知性因确定而振奋其心,且为认可一切功德,说了此颂。\end{enumerate}

\subsection\*{\textbf{164} \textcolor{gray}{\footnotesize 〔163A〕}}

\textbf{「对具足业与言路的牟尼的心,以及\\}
\textbf{「明行足,你如法地赞叹他。」\footnote{义注以为 164 为雪山所说,165~166 为七岳所说,167~168 为雪山所说,169 为二夜叉同说。}}

“Sampannaṃ munino cittaṃ, kammunā byappathena ca;\\
vijjācaraṇasampannaṃ, dhammato naṃ pasaṃsati”. %\hfill\textcolor{gray}{\footnotesize 12}

\begin{enumerate}\item 随后,雪山于世尊为「等正觉者、世尊」已无疑惑,仍立于空中,为赞叹世尊且为认同七岳,说了此颂。
\item 其义为:\textbf{具足的牟尼的心},即是说圆满、具足以「心意善待」中所说的如此之性,且圆满、具足以「他不取不与物」中所说的身业,以「他不味著爱欲」中所说的意业,以「他不妄语」中所说的言路及语业等。如是具足之心,\textbf{以及}由具足无上明行之成就的\textbf{明行足},你依这些功德以「心意善待」等方法\textbf{如法地赞叹他},表明唯从自性、真实、实在去赞叹他,不仅仅出于信。\end{enumerate}

\subsection\*{\textbf{165} \textcolor{gray}{\footnotesize 〔163B〕}}

\textbf{「对具足业与言路的牟尼的心,以及\\}
\textbf{「明行足,你如法地随喜。}

“Sampannaṃ munino cittaṃ, kammunā byappathena ca;\\
vijjācaraṇasampannaṃ, dhammato anumodasi. %\hfill\textcolor{gray}{\footnotesize 13}

\begin{enumerate}\item 随后,七岳也以「如是,先生!你已善加了知并随喜」之意趣,为认同他,说了此颂。\end{enumerate}

\subsection\*{\textbf{166} \textcolor{gray}{\footnotesize 〔164〕}}

\textbf{「对具足业与言路的牟尼的心,以及\\}
\textbf{「明行足,噫!我们去见乔达摩!」}

Sampannaṃ munino cittaṃ, kammunā byappathena ca;\\
vijjācaraṇasampannaṃ, handa passāma Gotamaṃ”. %\hfill\textcolor{gray}{\footnotesize 14}

\begin{enumerate}\item 且如是说已,为再次催促他去见世尊,说了此颂。\end{enumerate}

\subsection\*{\textbf{167} \textcolor{gray}{\footnotesize 〔165〕}}

\textbf{「具羚羊之腿肚者,瘦削的英雄,少食而无贪求,\\}
\textbf{「在林中禅修的牟尼,来!我们去见乔达摩!}

“Eṇijaṅghaṃ kisaṃ vīraṃ, appāhāraṃ alolupaṃ;\\
muniṃ vanasmiṃ jhāyantaṃ, ehi passāma Gotamaṃ. %\hfill\textcolor{gray}{\footnotesize 15}

\begin{enumerate}\item 于是,雪山依悦己之功德,以自己过去生的多闻之力,为赞叹世尊,对七岳说了此颂。
\item 其义为:\textbf{具羚羊之腿肚者},因为诸佛的腿肚如羚羊般渐次丰满,而非在前无肉,在后如鳄鱼之腹般鼓起。且诸佛都以肢体修短合度之成就而\textbf{瘦削},非如肥胖之人般臃肿,或由以慧刮去烦恼而瘦削。由摧毁内外之敌为\textbf{英雄}。由一坐食及限量食为\textbf{少食},而非由仅二三团饭之食,如说:\begin{quoting}而我,优陀夷!有时以满钵而食,或食过此,则那些会以「沙门乔达摩是少食者,且赞叹少食」恭敬、尊重、奉事、供养我,恭敬、尊重已依止而住的弟子们,优陀夷!我的仅食一杯之量、半杯之量、一木瓜、半木瓜的弟子们,便不会以此法恭敬我……依止而住。(中部·大舍拘罗陀夷经)\end{quoting}以于食物无有欲贪为\textbf{无贪求},他们食用八支具足的食物。以牟尼性之成就为\textbf{牟尼}。以非家及心意倾向于远离而\textbf{在林中禅修}。因此,雪山夜叉说「具羚羊之腿肚者……来!我们去见乔达摩」。\end{enumerate}

\subsection\*{\textbf{168} \textcolor{gray}{\footnotesize 〔166〕}}

\textbf{「如同狮子,龙象独行,不关切爱欲,\\}
\textbf{「去到后,我们问问解脱死亡之网!」}

Sīhaṃ v’ekacaraṃ nāgaṃ, kāmesu anapekkhinaṃ;\\
upasaṅkamma pucchāma, maccupāsappamocanaṃ”. %\hfill\textcolor{gray}{\footnotesize 16}

\begin{enumerate}\item 且如是说已,再次为欲在彼世尊跟前听闻法,说了此颂。
\item 其义为:\textbf{如同狮子},即以难以接近、忍耐及无畏之义如同鬃狮。由渴爱被称为「以爱为侣的人」,以无有此为\textbf{独行},亦由对一世界无两位佛陀生起为独行,且此中之义当如犀牛角经中所述\footnote{独行:详见\textbf{犀牛角经}第 35 颂的义注。}。\textbf{龙象},即于再有既不去、也不来,或以不犯罪为龙象\footnote{既不去、也不来,或以不犯罪:这是从语源学上解释「龙象」。},亦以有力为龙象。\textbf{不关切爱欲},即于两种爱欲以无有欲贪而不关切。\textbf{去到后,我们问问解脱死亡之网},即去到如此的大仙后,我们问问三界流转的死亡之网的解脱、还灭、涅槃。或者,以之从被称为苦之集的死亡之网解脱的方法,我们问问其对死亡之网的解脱。雪山对七岳、七岳的随从及自己的随从说了此颂。
\item 尔时,阿沙陀月的节日来临。于是,王舍城四处都已庄严准备,如散发着光辉的天城一般,名为 Kāḷī Kuraragharikā 的优婆夷登上楼阁,打开窗子,正除遣幽居的烦闷,站在有风之处纳凉,便从头到尾听到了这两位夜叉大将关于佛陀功德的谈话。且听闻后,以「诸佛具足种种功德」生起以佛陀为所缘的喜,由此镇伏诸盖,即于此处便住立于须陀洹果。之后,被世尊置于上首:\begin{quoting}诸比丘!我的声闻优婆夷中,以传闻而净喜的上首,即 Kāḷī Kuraragharikā 优婆夷是。(增支部第 1:267 经)\end{quoting}\end{enumerate}

\subsection\*{\textbf{169} \textcolor{gray}{\footnotesize 〔167〕}}

\textbf{「宣说者,转起者,已度一切法者,\\}
\textbf{「佛陀,超越敌对怖畏者,我们问问乔达摩!」}

“Akkhātāraṃ pavattāraṃ, sabbadhammāna pāraguṃ;\\
buddhaṃ verabhayātītaṃ, mayaṃ pucchāma Gotamaṃ”. %\hfill\textcolor{gray}{\footnotesize 17}

\begin{enumerate}\item 这两位夜叉大将为一千夜叉所随从,在中夜时分到达仙人堕处,去到仍以转法轮之跏趺而坐的世尊处后,顶礼已,以此颂赞美世尊,提出请求。
\item 其义为:在除渴爱的三界之法中,以\begin{quoting}诸比丘!此是苦圣谛。\end{quoting}等方法谈论诸谛的差别为\textbf{宣说},以\begin{quoting}「此苦圣谛应遍知」,诸比丘!我……(转法轮经)\end{quoting}等方法于彼等转起应作之智与已作之智为\textbf{转起}。或者,于应如是表达的诸法中,即以如是的表达来谈论为宣说,即于此等法,以随适于有情来谈论为转起。或对敏知者、广知者的开示为宣说,对可予引导者的支持为转起\footnote{敏知者、广知者、可予引导者及仅通文字者,见\textbf{纯陀经}第 87 颂的说明。}。或以总说为宣说,以从彼彼行相的言说来分别为转起。或以对菩提分俱相的谈论为宣说,以于有情的心相续处转起为转起。或简略地从三转谈论诸谛为宣说,详细地为转起,如以\begin{quoting}信根为法,转起此法者为法轮。(无碍解道)\end{quoting}等无碍解的方法转起详细的法轮为转起。
\item \textbf{一切法},即四地之法。\textbf{已度},即以证知、遍知、舍断、修习、证得、等至等六种行相到达彼岸。因为彼世尊以证知一切法而度为证知已度,以遍知五取蕴而度为遍知已度,以舍断一切烦恼而度为舍断已度,以修习四道而度为修习已度,以证得灭而度为证得已度,以等至一切等至而度为等至已度。如是已度一切法。
\item \textbf{佛陀,超越敌对怖畏者},由从无智之睡眠觉醒为佛陀,或以在「皈依注」\footnote{皈依注:即\textbf{小诵}·三皈依之义注。}中所说的一切之义为佛陀,由超越五种敌对怖畏\footnote{五种敌对怖畏:据菩提比丘注,即对五戒的违犯,见\textbf{相应部}第 12:41 经。}为超越敌对怖畏者。当如是赞美世尊时,他们提出请求「我们问问乔达摩」。\end{enumerate}

\subsection\*{\textbf{170} \textcolor{gray}{\footnotesize 〔168〕}}

\textbf{「于何世间生起?」雪山夜叉说,「于何产生亲密?\\}
\textbf{「取著何者而有世间?于何世间遘难?」}

“Kismiṃ loko samuppanno, \textit{(iti Hemavato yakkho)} kismiṃ kubbati santhavaṃ;\\
kissa loko upādāya, kismiṃ loko vihaññati”. %\hfill\textcolor{gray}{\footnotesize 18}

\begin{enumerate}\item 于是,在这些夜叉中以光辉和智慧为上首的雪山,为问如其意趣所当问者,说了此颂。
\item 在其第一句中,\textbf{于何}是依格独立式,这里是「当生起何者,世间生起」的意思,是就有情世间与行世间\footnote{有情世间与行世间,见\textbf{清净道论}·说六随念品。}而发问。\textbf{于何产生亲密},即「我」或「我所」的爱、见之亲密于何产生,是表起因的依格。\textbf{何者而有世间}是表业格的属格,这里是\textbf{取著}何者而得「世间」之定义的意思。\textbf{于何世间}为依格独立式及表原因的依格,这里是「存在什么、以何原因,世间遘难、受逼迫、受恼害」的意思。\end{enumerate}

\subsection\*{\textbf{171} \textcolor{gray}{\footnotesize 〔169〕}}

\textbf{「于六世间生起,雪山!」世尊说,「于六产生亲密,\\}
\textbf{「即取著六(而有世间),于六世间遘难。」}

“Chasu loko samuppanno, \textit{(Hemavatā ti Bhagavā)} chasu kubbati santhavaṃ;\\
channam eva upādāya, chasu loko vihaññati”. %\hfill\textcolor{gray}{\footnotesize 19}

\begin{enumerate}\item 于是,因为当六内外处生起,有情世间及以财富、谷物等的行世间生起。又因为此中,有情世间即于彼等六中产生两种亲密,由其把握眼处或其它某处为「我、我所」,如说:\begin{quoting}若说「眼是我」者,这不合理。(中部·六六经第 422 段)\end{quoting}等。又因为即取著此六而得两种世间之定义。又因为即存在此六,有情世间由苦的显现而遘难,如说:\begin{quoting}诸比丘!有手的存在而有取舍,有脚的存在而有进退,有关节的存在而有屈伸,有腹的存在而有饥渴,如是,诸比丘!有眼的存在,以眼触为缘,内在的苦乐生起。(相应部第 35:236 经)\end{quoting}等。同样,存在彼等作为支持,受妨碍的行世间遘难,如说:\begin{quoting}(有见有对之色)在无见有对之眼处受到妨碍……(法集论第 597 段)\end{quoting}以及\begin{quoting}诸比丘!眼在适意、不适意的色中受妨碍。(相应部第 35:238 经)\end{quoting}等。同样,即以彼等作为原因,两种世间遘难,如说:\begin{quoting}眼在适意、不适意的色中遘难。(相应部第 35:238 经)\end{quoting}以及\begin{quoting}诸比丘!眼在燃烧,色在燃烧……因何燃烧?因贪之火……(相应部第 35:28 经)\end{quoting}等。所以,世尊为以六内外处解答此问,说了此颂。\end{enumerate}

\subsection\*{\textbf{172} \textcolor{gray}{\footnotesize 〔170〕}}

\textbf{「什么是这世间在其中遘难的取著?\\}
\textbf{「问及出离,请说如何从苦解脱!」}

“Katamaṃ taṃ upādānaṃ, yattha loko vihaññati;\\
niyyānaṃ pucchito brūhi, kathaṃ dukkhā pamuccati”. %\hfill\textcolor{gray}{\footnotesize 20}

\begin{enumerate}\item 于是,这夜叉自己以流转提出问题,对世尊以十二处简略的回答未能善加辨别,欲了知其义与其对治,仍简略地问及流转与还灭,说了此颂。
\item 这里,以可被取著之义为\textbf{取著},为苦谛的同义语。\textbf{世间在其中遘难},即世尊以「于六世间遘难」所说的「世间在六种取著之中遘难」,\textbf{什么是这}取著?如是他以半颂明确地问了苦谛,而集谛作为其原因也被包摄。\textbf{问及出离},则以此半颂问了道谛。因为圣弟子以道谛知苦、断集、证灭、修道而从世间出离,所以被称为出离。\textbf{如何},即以何方式。\textbf{从苦解脱},即从所说为「取著」的流转之苦证得解脱。如是于此他明确地问了道谛,而灭谛作为其境域也被包摄。\end{enumerate}

\subsection\*{\textbf{173} \textcolor{gray}{\footnotesize 〔171〕}}

\textbf{「种种五欲被宣告于世间,意为第六,\\}
\textbf{「除去此中的欲,如是从苦解脱。}

“Pañca kāmaguṇā loke, manochaṭṭhā paveditā;\\
ettha chandaṃ virājetvā, evaṃ dukkhā pamuccati. %\hfill\textcolor{gray}{\footnotesize 21}

\begin{enumerate}\item 如是,当被夜叉以明示及未明示地问及四谛,世尊也以此方法解答,说了此颂。
\item 这里,以被称为\textbf{种种五欲}的行处所摄,作为其行处的五处即被包摄。以意为彼等之第六,即\textbf{意为第六}。\textbf{被宣告},即被阐明。此中,以内处中第六意处所摄,作为其境域的法处即被包摄。如是,为解答此问「什么是这取著」,又再次依十二处阐明了苦谛。
\item 或者,由七识界被意包摄,其中,以前五识界所摄,作为彼等依处的眼等五处即被包摄,以意界、意识界所摄,作为彼等依处与行处等类的法处即被包摄,如是,也依十二处阐明了苦谛。且此中,由于是对「世间在其中遘难」的说明,出世间的意处及法处一分未被包摄。
\item \textbf{除去此中的欲},即\textbf{此}十二处等类的苦谛\textbf{中},从蕴、界、名色等如此如此分别了这些处后,引入三相,修观者以阿罗汉道为终了之观,彻底\textbf{除去}、调伏、摧毁了称为渴爱的\textbf{欲}之义。\textbf{如是从苦解脱},即以此行相,从此流转之苦解脱。如是,以此半颂,此问「问及出离,请说如何从苦解脱」得以解答,且道谛得以阐明,而此中集谛、灭谛由以先前的方法所摄,当知也已阐明。
\item 或者,当知以半颂阐明苦谛,以「欲」阐明集谛,以此中「除去」之离贪阐明灭谛,由「离贪则解脱」之语或以「如是」所示的方法阐明道谛,以「苦灭」之语或「从苦解脱」之苦的解脱阐明灭谛,如是于此得以阐明四谛。\end{enumerate}

\subsection\*{\textbf{174} \textcolor{gray}{\footnotesize 〔172〕}}

\textbf{「此即世间的出离,已如实对你们宣说,\\}
\textbf{「我将对你们宣说此,如是从苦解脱。」}

Etaṃ lokassa niyyānaṃ, akkhātaṃ vo yathātathaṃ;\\
etaṃ vo aham akkhāmi, evaṃ dukkhā pamuccati”. %\hfill\textcolor{gray}{\footnotesize 22}

\begin{enumerate}\item 如是,以包含四谛之颂,从相阐明了出离,为再次对此以自己的言辞总结,说了此颂。
\item 此中,\textbf{此},即先前所说的义释。\textbf{世间},即三界之世间。\textbf{如实},即无颠倒。\textbf{我将对你们宣说此},即便你们问我千遍,我仍将对你们宣说此,而非其它。为什么?因为\textbf{如是从苦解脱},而非其它。
\item 或者,即便对以此出离已作了一次、二次、三次的离去者,我仍将对你们宣说此。为什么?因为如是从无余之苦解脱,即以阿罗汉为顶点完成了开示。
\item 当开示终了,二夜叉大将与一千夜叉便住于须陀洹果。\end{enumerate}

\subsection\*{\textbf{175} \textcolor{gray}{\footnotesize 〔173〕}}

\textbf{「谁于此度过暴流?谁于此度过海洋?\\}
\textbf{「于无落足、无攀援之深,谁不沉没?」}

“Ko sū’dha tarati oghaṃ, ko’dha tarati aṇṇavaṃ;\\
appatiṭṭhe anālambe, ko gambhīre na sīdati”. %\hfill\textcolor{gray}{\footnotesize 23}

\begin{enumerate}\item 于是,虽然雪山天性就敬重法,现在则以住于圣地,于世尊多样辩才的开示更无厌足,为问世尊有学、无学之地,说了此颂。
\item 这里,\textbf{谁于此度过暴流},即以此「谁度过四暴流」无差别地问有学地。因为海洋不仅仅广或不仅仅深,而是既甚广且甚深,故得是称。轮回之海洋也如此,因为它以处处无有边界而广,以下无落足、上无攀援而深,所以,\textbf{谁于此度过海洋}?且于此\textbf{无落足、无攀援之深}海,\textbf{谁不沉没}?即问无学地。\end{enumerate}

\subsection\*{\textbf{176} \textcolor{gray}{\footnotesize 〔174〕}}

\textbf{「始终具足戒,具慧,善等持,\\}
\textbf{「内省,具念,他度过难度的暴流。}

“Sabbadā sīlasampanno, paññavā susamāhito;\\
ajjhattacintī satimā, oghaṃ tarati duttaraṃ. %\hfill\textcolor{gray}{\footnotesize 24}

\begin{enumerate}\item 若比丘宁舍性命也不行违犯而\textbf{始终具足戒},且以世出世间慧而\textbf{具慧},以近行、安止定及威仪、下三道果而\textbf{善等持},习于引入三相,以毗婆舍那\textbf{内省}自身,\textbf{具}足导向坚持的不放逸之\textbf{念},因为他以第四道无余地\textbf{度过}极\textbf{难度的暴流},所以,世尊为解答有学地,说了这包含三学之颂。
\item 此中,以戒的成就为增上戒学,以念与定为增上心学,以内省与慧为增上慧学,如是说了三学及其资助与利益。因为世间慧与念为三学的资助,而沙门果为其利益。\end{enumerate}

\subsection\*{\textbf{177} \textcolor{gray}{\footnotesize 〔175〕}}

\textbf{「离于爱欲想,越过一切结缚,\\}
\textbf{「灭尽喜与有,他不沉没于深。」}

Virato kāmasaññāya, sabbasaṃyojanātigo;\\
nandībhavaparikkhīṇo, so gambhīre na sīdati”. %\hfill\textcolor{gray}{\footnotesize 25}

\begin{enumerate}\item 如是,在以第一颂显明了有学地后,现在,为显明无学地,说了第二颂。
\item 其义为:\textbf{离于爱欲想},即以与第四道相应的正断离离于一切爱欲之想。文本也作「离染 \textit{viratto}」。此处,爱欲想为依格,而在有偈品中也作复数 \textit{kāmasaññāsu}。亦由以第四道越过十种结缚而\textbf{越过一切结缚},或仅以第四道越过一切上分结缚。由灭尽被称为彼彼乐著的渴爱与三有而\textbf{灭尽喜与有}。像\textbf{他}这样的漏尽比丘\textbf{不沉没于}轮回之\textbf{深}海,因喜的灭尽而至有余依涅槃之陆地,因有的灭尽而至无余依涅槃之陆地,以至最上的安息。\end{enumerate}

\subsection\*{\textbf{178} \textcolor{gray}{\footnotesize 〔176〕}}

\textbf{「深慧,见微妙义,无所牵绊,不取著于欲与有,\\}
\textbf{「你们看这解脱于一切处、行走在天路上的大仙!}

“Gambhīrapaññaṃ nipuṇatthadassiṃ, akiñcanaṃ kāmabhave asattaṃ;\\
taṃ passatha sabbadhi vippamuttaṃ, dibbe pathe kamamānaṃ mahesiṃ. %\hfill\textcolor{gray}{\footnotesize 26}

\begin{enumerate}\item 于是,雪山观察了朋友与夜叉会众,生起喜悦,以「深慧」等颂称赏了世尊,与全体会众及朋友一起顶礼并右绕后,便回到自己的住处。
\item 这些颂的释义为:\textbf{深慧},即具足甚深之慧,这里当以无碍解道中所说的方法来理解深慧,因为那里说:\begin{quoting}于甚深诸蕴转起的智为深慧,等等。(无碍解道·慧品第 4 段)\end{quoting}\textbf{见微妙义},即能见由微妙的刹帝利智者等所提问题之义,或以能见他人难以通达的诸义之微妙原因为见微妙义。以无有任何贪等为\textbf{无所牵绊}。以不执著于二种爱欲及三种有为\textbf{不取著于欲与有}。以于蕴等类的一切所缘无有欲贪的束缚为\textbf{解脱于一切处}。
\item \textbf{行走在天路上},即以等至经行于八等至等类的天路。这里,虽然世尊并非在此时行走于天路,但依先前的行走,以具备行走的能力或于此所得的势力而如是说。或者,以行走于任何清净天阿罗汉之路、寂静的住处而说此。以寻求大功德为\textbf{大仙}。\end{enumerate}

\subsection\*{\textbf{179} \textcolor{gray}{\footnotesize 〔177〕}}

\textbf{「享有盛名,见微妙义,给予智慧,不取著于欲执,\\}
\textbf{「你们看这知晓一切、善慧、行走在圣路上的大仙!}

Anomanāmaṃ nipuṇatthadassiṃ, paññādadaṃ kāmālaye asattaṃ;\\
taṃ passatha sabbaviduṃ sumedhaṃ, ariye pathe kamamānaṃ mahesiṃ. %\hfill\textcolor{gray}{\footnotesize 27}

\begin{enumerate}\item 在第二颂中,以别的方法赞赏后,再次显示\textbf{见微妙义},或者,即显示微妙义者之义。\textbf{给予智慧},即以讲论导向获得智慧的行道为施与智慧者。\textbf{不取著于欲执},即不取著于爱欲中的爱、见两种执著。
\item \textbf{知晓一切},即知晓一切法,即是说一切知者。\textbf{善慧},即具足成为一切知性之道的称为波罗蜜慧的慧。\textbf{圣路},即八支圣道,或果定。\textbf{行走},即由了知道之相后的开示而以慧潜入,或由刹那刹那等至于果定而进入,或以被称为四种道之修习的行走的能力而先前曾行。\end{enumerate}

\subsection\*{\textbf{180} \textcolor{gray}{\footnotesize 〔178〕}}

\textbf{「我们今天确实有好的所见、好的早晨、好的起身,\\}
\textbf{「因为我们见到了已度过暴流、无漏的等正觉。}

Sudiṭṭhaṃ vata no ajja, suppabhātaṃ suhuṭṭhitaṃ;\\
yaṃ addasāma Sambuddhaṃ, oghatiṇṇam anāsavaṃ. %\hfill\textcolor{gray}{\footnotesize 28}

\begin{enumerate}\item \textbf{我们今天确实有好的所见},即今天我们得见善妙的所见,或今天我们有善妙的所见之义。\textbf{好的早晨、好的起身},即今天我们有好的早晨或净美的早晨,且今天我们有善妙的起身,无障碍地从睡眠中起身。什么原因?\textbf{因为我们见到了等正觉},就自身利益的成就而宣告愉悦。\end{enumerate}

\subsection\*{\textbf{181} \textcolor{gray}{\footnotesize 〔179〕}}

\textbf{「这一千个夜叉具有神变、具有名望,\\}
\textbf{「全都皈依你,你是我们无上的大师。}

Ime dasasatā yakkhā, iddhimanto yasassino;\\
sabbe taṃ saraṇaṃ yanti, tvaṃ no satthā anuttaro. %\hfill\textcolor{gray}{\footnotesize 29}

\begin{enumerate}\item \textbf{具有神变},即具足业异熟所生的神变。\textbf{具有名望},即具足最上的利养、最上的眷属。\textbf{皈依},虽然已经以道而行,仍如是为说明须陀洹性及为表明净喜而诉之于言。\end{enumerate}

\subsection\*{\textbf{182} \textcolor{gray}{\footnotesize 〔180〕}}

\textbf{「我们将从村到村、从山到山地游行,\\}
\textbf{「礼敬着等正觉,以及法的善法性。」}

Te mayaṃ vicarissāma, gāmā gāmaṃ nagā nagaṃ;\\
namassamānā Sambuddhaṃ, dhammassa ca sudhammatan” ti. %\hfill\textcolor{gray}{\footnotesize 30}

\begin{enumerate}\item \textbf{从村到村},即从天村到天村。\textbf{从山到山},即从天山到天山。\textbf{礼敬着等正觉,以及法的善法性},即是说以「世尊确实是正等正觉、法确实是世尊所善说」等方法称赏佛的善觉性及法的善法性,并以「世尊的声闻众确实是善行道者」等称赏僧的善行道,礼敬着,宣扬法而游行。余义于此自明。\end{enumerate}

\begin{center}\vspace{1em}雪山经第九\\Hemavatasuttaṃ navamaṃ.\end{center}

%\begin{flushright}甲辰清明二稿\end{flushright}