\chapter{輕詆第二十六}

\subsection*{1}

\textbf{王太尉問眉子:「汝叔名士,何以不相推重?」}{\footnotesize 眉子已見。叔,王澄也。}\textbf{眉子曰:「何有名士終日妄語?」}

\subsection*{2}

\textbf{庾元規語周伯仁:「諸人皆以君方樂。」周曰:「何樂?謂樂毅邪?」}{\footnotesize \textbf{史記}曰樂毅,中山人,賢而為燕昭王將軍,率諸侯伐齊,終於趙。}\textbf{庾曰:「不爾,樂令耳。」周曰:「何乃刻畫無鹽,以唐突西子也?」}{\footnotesize \textbf{列女傳}曰鍾離春者,齊無鹽之女也,其醜無雙,黃頭深目,長壯大節,鼻昂結喉,肥項少髮,折腰出胸,皮膚若漆,行年三十,無所容入,衒嫁不售,乃自詣齊宣王,乞備後宮,因說王以四殆,王拜為正后。\textbf{吳越春秋}曰越王句踐得山中採薪女子,名曰西施,獻之吳王。}

\subsection*{3}

\textbf{深公云:「人謂庾元規名士,胸中柴棘三斗許。」}

\subsection*{4}

\textbf{庾公權重,足傾王公,庾在石頭,王在冶城坐,大風揚塵,王以扇拂塵曰:「元規塵汙人。」}{\footnotesize \textbf{按}王公雅量通濟,庾亮之在武昌,傳其應下,公以識度裁之,囂言自息,豈或回貳有扇塵之事乎?\textbf{王隱晉書戴洋傳}曰丹陽太守王導,問洋得病七年,洋曰「君侯命在申,為土地之主,而於申上冶,火光昭天,此為金火相鑠,水火相炒,以故相害」,導呼冶令奕遜,使啓鎮東徙,今東冶是也。\textbf{丹陽記}曰丹陽冶城,去宮三里,吳時鼓鑄之所,吳平猶不廢。又云孫權築冶城,為鼓鑄之所。既立石頭大塢,不容近立此小城,當是徙縣治空城而置冶爾,冶城疑是金陵本治,漢高六年,令天下縣邑,秣陵不應獨無。}

\subsection*{5}

\textbf{王右軍少時甚澀訥,在大將軍許,王、庾二公後來,右軍便起欲去,大將軍留之曰:「爾家司空、}{\footnotesize 王丞相已見。}\textbf{元規,復可所難?」}

\subsection*{6}

\textbf{王丞相輕蔡公,曰:「我與安期、千里共遊洛水邊,何處聞有蔡充兒?」}{\footnotesize \textbf{晉諸公贊}曰充,字子尼,陳留雍丘人。\textbf{充別傳}曰充祖睦,蔡邕孫也,充少好學,有雅尚,體貌尊嚴,莫有媟慢於其前者,高平劉整有雋才,而車服奢麗,謂人曰「紗縠,人常服耳,嘗遇蔡子尼在坐,終日不自安」,見憚如此,是時陳留為大郡,多人士,琅邪王澄嘗經郡,入境,問「此郡多士,有誰乎」,吏曰「有江應元、蔡子尼」,時陳留多居大位者,澄問「何以但稱此二人」,吏曰「向謂君侯問人,不謂位也」,澄笑而止,充歷成都王東曹掾,故稱東曹。\textbf{妒記}曰丞相曹夫人性甚忌,禁制丞相,不得有侍御,乃至左右小人亦被檢簡,時有妍妙,皆加誚責,王公不能久堪,乃密營別館,眾妾羅列,兒女成行,後元會日,夫人於青疎臺中望見兩三兒騎羊,皆端正可念,夫人遙見,甚憐愛之,語婢「汝出問,是誰家兒」,給使不達旨,乃答云「是第四五等諸郎」,曹氏聞,驚愕大恚,命車駕,將黃門及婢二十人,人持食刀,自出尋討,王公亦遽命駕,飛轡出門,猶患牛遲,乃以左手攀車蘭,右手捉麈尾,以柄助御者打牛,狼狽奔馳,劣得先至,蔡司徒聞而笑之,乃故詣王公,謂曰「朝廷欲加公九錫,公知不」,王謂信然,自敘謙志,蔡曰「不聞餘物,唯聞有短轅犢車,長柄麈尾」,王大愧,後貶蔡曰「吾與安期、千里共在洛水」。}

\subsection*{7}

\textbf{褚太傅初渡江,嘗入東,至金昌亭,吳中豪右,燕集亭中,}{\footnotesize \textbf{謝歆金昌亭詩敘}曰余尋師,來入經吳,行達昌門,忽覩斯亭,傍川帶河,其榜題曰「金昌」,訪之耆老,曰「昔朱買臣仕漢,還為會稽內史,逢其迎吏,逆旅比舍,與買臣爭席,買臣出其印綬,群吏慚服自裁,因事建亭,號曰金傷,失其字義耳」。}\textbf{褚公雖素有重名,于時造次不相識別,敕左右多與茗汁,少著粽,汁盡輒益,使終不得食,褚公飲訖,徐舉手共語云:「褚季野。」於是四座驚散,無不狼狽。}

\subsection*{8}

\textbf{王右軍在南,丞相與書,每歎子姪不令,云:「虎㹠、虎犢,還其所如。」}{\footnotesize 虎㹠,王彭之小字也。\textbf{王氏譜}曰彭之,字安壽,琅邪人,祖正,尚書郎,父彬,衛將軍,彭之仕至黃門郎。虎犢,彪之小字也。彪之,字叔虎,彭之第三弟,年二十而頭鬚皓白,時人謂之王白鬚,少有局幹之稱,累遷至左光祿大夫。}

\subsection*{9}

\textbf{褚太傅南下,孫長樂於船中視之,}{\footnotesize 長樂,孫綽。}\textbf{言次及劉真長死,孫流涕,因諷詠曰:「人之云亡,邦國殄瘁。」}{\footnotesize 大雅詩。\textbf{毛公}注曰殄盡、瘁病也。}\textbf{褚大怒曰:「真長平生,何嘗相比數,而卿今日作此面向人。」孫回泣向褚曰:「卿當念我。」時咸笑其才而性鄙。}

\subsection*{10}

\textbf{謝鎮西書與殷揚州,為真長求會稽,殷答曰:「真長標同伐異,俠之大者,常謂使君降階為甚,乃復為之驅馳邪?」}

\subsection*{11}

\textbf{桓公入洛,過淮泗,踐北境,與諸僚屬登平乘樓,眺矚中原,慨然曰:「遂使神州陸沈,百年丘墟,王夷甫諸人,不得不任其責。」}{\footnotesize \textbf{八王故事}曰夷甫雖居台司,不以事物自嬰,當世化之,羞言名教,自臺郎以下,皆雅崇拱默,以遺事為高,四海尚寧,而識者知其將亂。\textbf{晉陽秋}曰夷甫將為石勒所殺,謂人曰「吾等若不祖尚浮虛,不至於此」。}\textbf{袁虎率爾對曰:「運自有廢興,豈必諸人之過?」桓公懍然作色,顧謂四坐曰:「諸君頗聞劉景升不?}{\footnotesize \textbf{劉鎮南銘}曰表,字景升,山陽高平人,黃中通理,博識多聞,仕至鎮南將軍、荊州刺史。}\textbf{有大牛重千斤,噉芻豆十倍於常牛,負重致遠,曾不若一羸牸,魏武入荊州,烹以饗士卒,于時莫不稱快。」意以況袁,四坐既駭,袁亦失色。}

\subsection*{12}

\textbf{袁虎、伏滔同在桓公府,桓公每遊燕,輒命袁、伏,袁甚恥之,恆歎曰:「公之厚意,未足以榮國士,與伏滔比肩,亦何辱如之。」}

\subsection*{13}

\textbf{高柔在東,甚為謝仁祖所重,既出,不為王、劉所知,仁祖曰:「近見高柔,大自敷奏,然未有所得。」真長云:「故不可在偏地居,輕在角䚥}{\footnotesize 奴角反。}\textbf{中,為人作議論。」高柔聞之,云:「我就伊無所求。」人有向真長學此言者,真長曰:「我寔亦無可與伊者。」然遊燕猶與諸人書:「可要安固。」安固者,高柔也。}{\footnotesize \textbf{孫統}為柔集敘曰柔,字世遠,樂安人,才理清鮮,安行仁義,婚泰山胡毋氏女,年二十,既有倍年之覺,而姿色清惠,近是上流婦人,柔家道隆崇,既罷司空參軍、安固令,營宅於伏川,馳動之情既薄,又愛翫賢妻,便有終焉之志,尚書令何充取為冠軍參軍,僶俛應命,眷戀綢繆,不能相舍,相贈詩書,清婉新切。}

\subsection*{14}

\textbf{劉尹、江虨、王叔虎、孫興公同坐,江、王有相輕色,虨以手歙叔虎云:「酷吏。」詞色甚彊,劉尹顧謂:「此是瞋邪?非特是醜言聲,拙視瞻。」}{\footnotesize 言江此言非是醜拙,似有忿於王也。}

\subsection*{15}

\textbf{孫綽作列仙商丘子贊曰:「所牧何物,殆非真豬,儻遇風雲,為我龍攄。」}{\footnotesize \textbf{列仙傳}曰商丘子晉者,商邑人,好吹竽牧豕,年七十,不娶妻而不老,問其道要,言「但食老朮、昌蒲根、飲水,如此便不飢不老耳」,貴戚富室,聞而服之,不能終歲輒止,謂將有匿術。\textbf{孫綽}為贊曰商丘卓犖,執策吹竽,渴飲寒泉,飢食菖蒲,所牧何物,殆非真豬,儻逢風雲,為我龍攄。}\textbf{時人多以為能,王藍田語人云:「近見孫家兒作文,道『何物真豬』也。」}

\subsection*{16}

\textbf{桓公欲遷都,以張拓定之業,孫長樂上表,諫此議甚有理,桓見表心服,而忿其為異,令人致意孫云:「君何不尋遂初賦,而彊知人家國事?」}{\footnotesize \textbf{孫綽表}諫曰中宗龍飛,實賴萬里長江,畫而守之耳,不然,胡馬久已踐建康之地,江東為豺狼之場矣。綽賦遂初,陳止足之道。}

\subsection*{17}

\textbf{孫長樂兄弟就謝公宿,言至款雜,劉夫人在壁後聽之,具聞其語,謝公明日還,問:「昨客何似?」劉對曰:「亡兄門未有如此賓客。」}{\footnotesize 夫人,劉惔之妹。}\textbf{謝深有愧色。}

\subsection*{18}

\textbf{簡文與許玄度共語,許云:「舉君親以為難。」簡文便不復答,許去後而言曰:「玄度故可不至於此。」}{\footnotesize \textbf{按}邴原別傳「魏五官中郎將嘗與群賢共論曰『今有一丸藥,得濟一人疾,而君父俱病,與君邪,與父邪』,諸人紛葩,或父或君,原勃然曰『父子,一本也』,亦不復難」,君親相校,自古如此,未解簡文誚許意。}

\subsection*{19}

\textbf{謝萬壽春敗後還,書與王右軍云:「慙負宿顧。」右軍推書曰:「此禹湯之戒。」}{\footnotesize \textbf{春秋傳}曰禹湯罪己,其興也勃焉。言禹湯以聖德自罪,所以能興,今萬失律致敗,雖復自咎,其可濟焉?故王嘉萬也。}

\subsection*{20}

\textbf{蔡伯喈睹睞笛椽,孫興公聽妓,振且擺折,}{\footnotesize \textbf{伏滔長笛賦敘}曰余同寮桓子野有故長笛,傳之耆老云「蔡邕伯喈之所製也」,初,邕避難江南,宿於柯亭之館,以竹為椽,邕仰眄之,曰「良竹也」,取以為笛,音聲獨絕,歷代傳之至於今。}\textbf{王右軍聞,大嗔曰:「三祖壽}{\footnotesize 一作臺。}\textbf{樂器,虺瓦}{\footnotesize 一作尫凡。}\textbf{弔孫家兒打折。」}

\subsection*{21}

\textbf{王中郎與林公絕不相得,王謂林公詭辯,林公道王云:「著膩顏帢,翕布單衣,挾左傳,逐鄭康成車後,問是何物塵垢囊?」}{\footnotesize 中郎,坦之。帢,帽也。\textbf{裴子}曰林公云「文度著膩顏,挾左傳,逐鄭康成,自為高足弟子,篤而論之,不離塵垢囊也」。}

\subsection*{22}

\textbf{孫長樂作王長史誄云:「余與夫子,交非勢利,心猶澄水,同此玄味。」}{\footnotesize \textbf{禮記}曰君子之交淡若水,小人之交甘若醴。}\textbf{王孝伯見曰:「才士不遜,亡祖何至與此人周旋?」}

\subsection*{23}

\textbf{謝太傅謂子姪曰:「中郎始是獨有千載。」車騎曰:「中郎衿抱未虛,復那得獨有?」}{\footnotesize 中郎,謝萬。}

\subsection*{24}

\textbf{庾道季詫謝公曰:「裴郎云『謝安謂裴郎乃可不惡,何得為復飲酒』,}{\footnotesize 庾龢、裴啓已見。}\textbf{裴郎又云『謝安目支道林,如九方皋之相馬,略其玄黃,取其儁逸』。」}{\footnotesize \textbf{支遁傳}曰遁每標舉會宗,而不留心象喻,解釋章句,或有所漏,文字之徒,多以為疑,謝安石聞而善之曰「此九方皋之相馬也,略其玄黃,而取其儁逸」。\textbf{列子}曰伯樂謂秦穆公曰「臣所與共儋纆薪菜者,有九方皋,此其於馬,非臣之下也」,公使行求馬,反,曰「得矣!牡而黃」,使人取之,牝而驪,公曰「毛物牝牡之不知,何馬之能知也」,伯樂曰「若皋之觀馬者,天機也,得其精,亡其粗,在其內,亡其外,見其所見,不見其所不見,視其所視,遺其所不視,若彼之所相,有貴於馬也」,既而馬果千里足。}\textbf{謝公云:「都無此二語,裴自為此辭耳。」庾意甚不以為好,因陳東亭經酒壚下賦,讀畢,都不下賞裁,直云:「君乃復作裴氏學。」於此語林遂廢,今時有者,皆是先寫,無復謝語。}{\footnotesize \textbf{續晉陽秋}曰晉隆和中,河東裴啓撰漢魏以來迄于今時言語應對之可稱者,謂之語林,時人多好其事,文遂流行,後說太傅事不實,而有人於謝坐敘其黃公酒壚,司徒王珣為之賦,謝公加以與王不平,乃云「君遂復作裴郎學」,自是眾咸鄙其事矣,安鄉人有罷中宿縣詣安者,安問其歸資,答曰「嶺南凋弊,惟有五萬蒲葵扇,又以非時為滯貨」,安乃取其中者捉之,於是京師士庶競慕而服焉,價增數倍,旬月無賣,夫所好生羽毛,所惡成瘡痏,謝相一言,挫成美於千載,及其所與,崇虛價於百金,上之愛憎與奪,可不慎哉。}

\subsection*{25}

\textbf{王北中郎不為林公所知,乃著論沙門不得為高士論,大略云:「高士必在於縱心調暢,沙門雖云俗外,反更束於教,非情性自得之謂也。」}

\subsection*{26}

\textbf{人問顧長康:「何以不作洛生詠?」答曰:「何至作老婢聲?」}{\footnotesize 洛下書生詠,音重濁,故云老婢聲。}

\subsection*{27}

\textbf{殷顗、庾恆並是謝鎮西外孫,}{\footnotesize \textbf{謝氏譜}曰尚長女僧要適庾龢,次女僧韶適殷歆。}\textbf{殷少而率悟,庾每不推,嘗俱詣謝公,謝公熟視殷曰:「阿巢故似鎮西。」}{\footnotesize 巢,殷顗小字也。}\textbf{於是庾下聲語曰:「定何似?」謝公續復云:「巢頰似鎮西。」庾復云:「頰似,足作健不?」}{\footnotesize \textbf{庾氏譜}曰恆,字敬則,祖亮,父龢,恆仕至尚書僕射。}

\subsection*{28}

\textbf{舊目韓康伯:「將肘無風骨。」}{\footnotesize \textbf{說林}曰范啓云「韓康伯似肉鴨」。}

\subsection*{29}

\textbf{苻宏叛來歸國,謝太傅每加接引,宏自以有才,多好上人,坐上無折之者,適王子猷來,太傅使共語,子猷直孰視良久,回語太傅云:「亦復竟不異人。」宏大慚而退。}{\footnotesize \textbf{續晉陽秋}曰宏,苻堅太子也,堅為姚萇所殺,宏將母妻來投,詔賜田宅,桓玄以宏為將,玄敗,寇湘中,伏誅。}

\subsection*{30}

\textbf{支道林入東,見王子猷兄弟,還,人問:「見諸王何如?」答曰:「見一群白頸烏,但聞喚啞啞聲。」}

\subsection*{31}

\textbf{王中郎舉許玄度為吏部郎,郗重熙曰:「相王好事,不可使阿訥在坐。」}{\footnotesize 訥,詢小字。}

\subsection*{32}

\textbf{王興道謂謝望蔡:「霍霍如失鷹師。」}{\footnotesize \textbf{永嘉記}曰王和之,字興道,琅琊人,祖翼,平南將軍,父胡之,司州刺史,和之歷永嘉太守、正員常侍。望蔡,謝琰小字也。}

\subsection*{33}

\textbf{桓南郡每見人不快,輒嗔云:「君得哀家梨,當復不烝食不?」}{\footnotesize 舊語,秣陵有哀仲家梨甚美,大如升,入口消釋,言愚人不別味,得好梨烝食之也。}