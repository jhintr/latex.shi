\chapter{Lesson 4}

\section*{Reading 1}

Evameva kho, bhikkhave, cattāro’me samaṇabrāhmaṇānaṃ upakkilesā, yehi upakkilesehi upakkiliṭṭhā eke samaṇabrāhmaṇā na tapanti, na bhāsanti, na virocanti.

Katame cattāro?

Santi, bhikkhave, eke samaṇabrāhmaṇā suraṃ pivanti merayaṃ, surāmerayapānā appaṭiviratā. Ayaṃ, bhikkhave, paṭhamo samaṇabrāhmaṇānaṃ upakkileso, yena upakkilesena upakkiliṭṭhā eke samaṇabrāhmaṇā na tapanti, na bhāsanti, na virocanti.

Santi, bhikkhave, eke samaṇabrāhmaṇā methunaṃ dhammaṃ patisevanti, methunasmā dhammā appaṭiviratā. Ayaṃ, bhikkhave, dutiyo samaṇabrāhmaṇānaṃ upakkileso yena upakkilesena upakkiliṭṭhā eke samaṇabrāhmaṇā na tapanti, na bhāsanti, na virocanti.

Santi, bhikkhave, eke samaṇabrāhmaṇā jātarūparajataṃ sādiyanti, jātarūparajatapaṭiggahaṇā appaṭiviratā. Ayaṃ, bhikkhave, tatiyo samaṇabrāhmaṇānaṃ upakkileso yena upakkilesena upakkiliṭṭhā eke samaṇabrāhmaṇā na tapanti, na bhāsanti, na virocanti.

Santi, bhikkhave, eke samaṇabrāhmaṇā micchājīvena jīvanti, micchājīvā appaṭiviratā. Ayaṃ, bhikkhave, catuttho samaṇabrāhmaṇānaṃ upakkileso yena upakkilesena upakkiliṭṭhā eke samaṇabrāhmaṇā na tapanti, na bhāsanti, na virocanti.

Ime kho, bhikkhave, cattāro samaṇabrāhmaṇānaṃ upakkilesā, yehi upakkilesehi upakkiliṭṭhā eke samaṇabrāhmaṇā na tapanti, na bhāsanti, na virocantīti.

Suraṃ pivanti merayaṃ,\\
paṭisevanti methunaṃ,\\
Rajataṃ jātarūpaṃ ca,\\
sādiyanti aviddasū,\\
Micchājīvena jīvanti,\\
eke samaṇabrāhmaṇā. \hfill(A 4.5.10)

\section*{Reading 2}

Bhojanaṃ, Suppavāse, dentī ariyasāvikā paṭiggāhakānaṃ cattāri ṭhānāni deti.

Katamāni cattāri?

Āyuṃ deti, vaṇṇaṃ deti, sukhaṃ deti, balaṃ deti. Āyuṃ kho pana datvā āyussa bhāginī hoti dibbassa vā mānusassa vā. Vaṇṇaṃ datvā vaṇṇassa bhāginī hoti dibbassa vā mānusassa vā. Sukhaṃ datvā sukhassa bhāginī hoti dibbassa vā mānusassa vā. Balaṃ datvā balassa bhāginī hoti dibbassa vā mānusassa vā.

Bhojanaṃ, Suppavāse, dentī ariyasāvikā paṭiggāhakānaṃ imāni cattāri ṭhānāni detī ti. \hfill(A 4.6.7)

\section*{Reading 3}

Na bhaje pāpake mitte,\\
na bhaje purisādhame,\\
bhajetha mitte kalyāṇe,\\
bhajetha purisuttame. \hfill(Dhp 6)

Sabbe tasanti daṇdassa,\\
sabbe bhāyanti maccuno,\\
attānaṃ upamaṃ katvā,\\
na haneyya na ghātaye.

Sabbe tasanti daṇdassa,\\
sabbesaṃ jīvitaṃ piyaṃ,\\
attānaṃ upamaṃ katvā,\\
na haneyya na ghātaye. \hfill(Dhp 10)

Bahuṃ pi ce sahitaṃ bhāsamāno,\\
na takkaro hoti naro pamatto,\\
gopo va gāvo gaṇayaṃ paresaṃ,\\
na bhāgavā sāmaññassa hoti.

Appaṃ pi ce sahitaṃ bhāsamāno,\\
dhammassa hoti anudhammacārī,\\
rāgañ ca dosañ ca pahāya mohaṃ,\\
sammappajāno suvimuttacitto,\\
anupādiyāno idha vā huraṃ vā,\\
sa bhāgavā sāmaññassa hoti. \hfill(Dhp 1)

Piyato jāyatī soko,\\
piyato jāyatī bhayaṃ,\\
piyato vippamuttassa,\\
natthi soko, kuto bhayaṃ?

Pemato jāyatī soko,\\
pemato jāyatī bhayaṃ,\\
pemato vippamuttassa,\\
natthi soko, kuto bhayaṃ?

Ratiyā jāyatī soko,\\
ratiyā jāyatī bhayaṃ,\\
ratiyā vippamuttassa,\\
natthi soko, kuto bhayaṃ?

Kāmato jāyatī soko,\\
kāmato jāyatī bhayaṃ,\\
kāmato vippamuttassa,\\
natthi soko, kuto bhayaṃ?

Taṇhāya jāyatī soko,\\
taṇhāya jāyatī bhayaṃ,\\
taṇhāya vippamuttassa,\\
natthi soko, kuto bhayaṃ? \hfill(Dhp 16)

\section*{Further Reading 1}

Dve’mā, bhikkhave, parisā. Katamā dve? Uttānā ca parisā gambhīrā ca parisā.

Katamā ca, bhikkhave, uttānā parisā?

Idha, bhikkhave, yassaṃ parisāyaṃ bhikkhū uddhatā honti unnaḷā capalā mukharā vikiṇṇavācā… asampajānā asamāhitā vibbhantacittā pākat’indriyā. Ayaṃ vuccati, bhikkhave, uttānā parisā.

Katamā ca, bhikkhave, gambhīrā parisā?

Idha, bhikkhave, yassaṃ parisāyaṃ bhikkhū anuddhatā honti anunnaḷā acapalā amukharā avikiṇṇavācā… sampajānā samāhitā ekaggacittā saṃvut’indriyā. Ayaṃ vuccati, bhikkhave, gambhīrā parisā. Imā kho, bhikkhave, dve parisā. \hfill(A 2.5.1)

Dve’mā, bhikkhave, parisā. Katamā dve? Vaggā ca parisā samaggā ca parisā.

Katamā ca, bhikkhave, vaggā parisā?

Idha, bhikkhave, yassaṃ parisāyaṃ bhikkhū bhaṇḍanajātā kalahajātā vivādāpannā… viharanti. Ayaṃ vuccati, bhikkhave, vaggā parisā.

Katamā ca, bhikkhave, samaggā parisā?

Idha, bhikkhave, yassaṃ parisāyaṃ bhikkhū samaggā sammodamānā avivadamānā khīrodakībhūtā… viharanti. Ayaṃ vuccati, bhikkhave, samaggā parisā.

Imā kho, bhikkhave, dve parisā. \hfill(A 2.5.2)

Dve’mā, bhikkhave, parisā. Katamā dve? Visamā ca parisā samā ca parisā.

Katamā ca, bhikkhave, visamā parisā?

Idha, bhikkhave, yassaṃ parisāyaṃ adhammakammāni pavattanti dhammakammāni nappavattanti, avinayakammāni pavattanti vinayakammāni nappavattanti, adhammakammāni dippanti dhammakammāni na dippanti, avinayakammāni dippanti vinayakammāni na dippanti. Ayaṃ vuccati, bhikkhave, visamā parisā.

Katamā ca, bhikkhave, samā parisā?

Idha, bhikkhave, yassaṃ parisāyaṃ dhammakammāni pavattanti adhammakammāni nappavattanti, vinayakammāni pavattanti avinayakammāni nappavattanti, dhammakammāni dippanti adhammakammāni na dippanti, vinayakammāni dippanti avinayakammāni na dippanti. Ayaṃ vuccati, bhikkhave, samā parisā.

Imā kho, bhikkhave, dve parisā. \hfill(A 2.5.8)

\section*{Further Reading 2}

Appamādo amatapadaṃ,\\
pamādo maccuno padaṃ,\\
appamattā na mīyanti,\\
ye pamattā yathā matā.

Etaṃ visesato ñatvā,\\
appamādamhi paṇḍitā,\\
appamāde pamodanti,\\
ariyānaṃ gocare ratā. \hfill(Dhp 2)

Yathā pi rahado gambhīro,\\
vippasanno anāvilo,\\
evaṃ dhammāni sutvāna,\\
vippasīdanti paṇḍitā.

Selo yathā ekaghano,\\
vātena na samīrati,\\
evaṃ nindāpasaṃsāsu,\\
na samiñjanti paṇḍitā. \hfill(Dhp 6)

Andhabhūto ayaṃ loko,\\
tanuk’ettha vipassati,\\
sakuṇo jālamutto va,\\
appo saggāya gacchati. \hfill(Dhp 13)

Udakaṃ hi nayanti nettikā,\\
usukārā namayanti tejanaṃ,\\
dāruṃ namayanti tacchakā,\\
attānaṃ damayanti paṇḍitā. \hfill(Dhp 6)

\section*{Further Reading 3}

Dve’māni, bhikkhave, sukhāni. Katamāni dve? Gihisukhaṃ ca pabbajitasukhaṃ ca. Imāni kho, bhikkhave, dve sukhāni.

Etadaggaṃ, bhikkhave, imesaṃ dvinnaṃ sukhānaṃ yadidaṃ pabbajitasukhan ti. \hfill(A 2.65)

Dve’māni, bhikkhave, sukhāni. Katamāni dve? Kāmasukhaṃ ca nekkhammasukhaṃ ca. Imāni kho, bhikkhave, dve sukhāni.

Etadaggaṃ, bhikkhave, imesaṃ dvinnaṃ sukhānaṃ yadidaṃ nekkhammasukhan ti. \hfill(A 2.66)

Dve’māni, bhikkhave, sukhāni. Katamāni dve? Upadhisukhaṃ ca nirupadhisukhaṃ ca. Imāni kho, bhikkhave, dve sukhāni.

Etadaggaṃ, bhikkhave, imesaṃ dvinnaṃ sukhānaṃ yadidaṃ nirupadhisukhan ti. \hfill(A 2.67)

Dve’māni, bhikkhave, sukhāni. Katamāni dve? Sāsavasukhañ ca anāsavasukhañ ca. Imāni kho, bhikkhave, dve sukhāni.

Etadaggaṃ, bhikkhave, imesaṃ dvinnaṃ sukhānaṃ yadidaṃ anāsavasukhan ti. \hfill(A 2.68)

Dve’māni, bhikkhave, sukhāni. Katamāni dve? Sāmisaṃ ca sukhaṃ nirāmisaṃ ca sukhaṃ. Imāni kho, bhikkhave, dve sukhāni.

Etadaggaṃ, bhikkhave, imesaṃ dvinnaṃ sukhānaṃ yadidaṃ nirāmisaṃ sukhan ti. \hfill(A 2.69)

Dve’māni, bhikkhave, sukhāni. Katamāni dve? Ariyasukhaṃ ca anariyasukhaṃ ca. Imāni kho, bhikkhave, dve sukhāni.

Etadaggaṃ, bhikkhave, imesaṃ dvinnaṃ sukhānaṃ yadidaṃ ariyasukhan ti. \hfill(A 2.70)

Dve’māni, bhikkhave, sukhāni. Katamāni dve? Kāyikaṃ ca sukhaṃ cetasikaṃ ca sukhaṃ. Imāni kho, bhikkhave, dve sukhāni.

Etadaggaṃ, bhikkhave, imesaṃ dvinnaṃ sukhānaṃ yadidaṃ cetasikaṃ sukhaṃ ti. \hfill(A 2.71)

\section*{Further Reading 4}

Pañcahi, bhikkhave, aṅgehi samannāgato rājā cakkavattī dhammen’eva cakkaṃ pavatteti, taṃ hoti cakkaṃ appaṭivattiyaṃ kenaci manussabhūtena paccatthikena pāṇinā.

Katamehi pañcahi?

Idha, bhikkhave, rājā cakkavattī atthaññū ca hoti, dhammaññū ca, mattaññū ca, kālaññū ca, parisaññū ca. Imehi kho, bhikkhave, pañcahi aṅgehi samannāgato rājā cakkavattī dhammeneva cakkaṃ pavatteti, taṃ hoti cakkaṃ appaṭivattiyaṃ kenaci manussabhūtena paccatthikena pāṇinā.

Evameva kho, bhikkhave, pañcahi dhammehi samannāgato tathāgato arahaṃ sammāsambuddho dhammeneva anuttaraṃ dhammacakkaṃ pavatteti, taṃ hoti cakkaṃ appaṭivattiyaṃ samaṇena vā brāhmaṇena vā devena vā mārena vā brahmunā vā kenaci vā lokasmiṃ.

Katamehi pañcahi?

Idha, bhikkhave, tathāgato arahaṃ sammāsambuddho atthaññū, dhammaññū, mattaññū, kālaññū, parisaññū. Imehi kho, bhikkhave, pañcahi dhammehi samannāgato tathāgato arahaṃ sammāsambuddho dhammeneva anuttaraṃ dhammacakkaṃ pavatteti, taṃ hoti dhammacakkaṃ appaṭivattiyaṃ samaṇena vā brāhmaṇena vā devena vā mārena vā brahmunā vā kenaci vā lokasmin ti. \hfill(A 5.14.1)