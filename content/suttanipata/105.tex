\section{纯陀经}

\begin{center}Cunda Sutta\end{center}\vspace{1em}

\subsection\*{\textbf{83}}

\textbf{「我问牟尼、广慧者、」锻工之子纯陀\footnote{此经旧译见长阿含经·游行经(大正藏第一册一八页)。纯陀,长阿含作「周那」。}说,「佛陀、法主、离爱者、\\}
\textbf{「两足尊、御者之最胜,世间有几种沙门?请快说说这个!」}

“Pucchāmi muniṃ pahūtapaññaṃ, \textit{(iti Cundo kammāraputto)}\\
\makebox[2em]{} buddhaṃ dhammassāmiṃ vītataṇhaṃ;\\
dvipaduttamaṃ sārathīnaṃ pavaraṃ, kati loke samaṇā tad iṅgha brūhi”. %\hfill\textcolor{gray}{\footnotesize 1}

\begin{enumerate}\item 缘起为何?先从略,在自身的意乐、他人的意乐、事件的发生、问题的主导等四种缘起中\footnote{四种缘起 \textit{uppatti},见\textbf{犀牛角经}义注。},此经的缘起为问题的主导。而从详,则自\begin{quoting}一时,世尊在末罗游行,与大比丘僧团一起到了波婆。于此,世尊住波婆锻工之子纯陀的芒果林中。\end{quoting}直至\begin{quoting}于是,世尊晨朝著下衣,持衣钵,与比丘僧团一起往锻工之子纯陀的住处走去,走到后,坐在设好的坐处。(长部·大般涅槃经第 189 段)\end{quoting}唯当以经中所述的方法详绎。
\item 如是,当世尊与比丘僧团一起坐下,锻工之子纯陀为以世尊为首的比丘僧团施食时,为盛咖喱等,便把金器皿授予众比丘。在学处尚未施设时,有些比丘便接受了金器皿,有些则未接受。而世尊唯有一个器皿,即他自己石制的钵,诸佛不接受第二个器皿。于此,某恶比丘将价值一千、呈给自己饮食用的金器皿,以盗心放入了钥匙袋。
\item 纯陀施食后,洗了手足,在礼敬世尊并观察比丘僧团时,便看到了这比丘,但看到后,出于对世尊及诸长老的尊重,且以「莫对邪见者有所言说」,便像没看到一样,什么也没说。他欲了知「是否唯有正当防护者是沙门,抑或如这般疏于防护者也是沙门」,便在晡时前往世尊处,说了此颂。
\item 这里,\textbf{问},以「有三种问:解释所未见之问……」等方法如义释中所说。\textbf{牟尼},也以「智被称为寂默,即慧、了知……正见,具足此智者为牟尼、证得寂默,三种寂默,即身寂默……」等方法如彼(义释)中所说。
\item 而此中,其略说为:\textbf{问},即创造机会。\textbf{牟尼},即称呼世尊为牟尼中的牟尼。「广慧者」等等为称赞之语,即以这些来称赞这牟尼。这里,\textbf{广慧者},即广大之慧者,且当知此广大是由已至能知的边界。\textbf{锻工之子纯陀说},此二词已如有财者经中所述。且从此以后,连这些(如……所述)也不再说,我们将抛开所有已述者,只解释未述者。
\item \textbf{佛陀},即三种佛陀中的第三种。\textbf{法主},由令生道法,如父之于子、老师之于自己创造的工巧处等一般,而为法之主、法之主宰、法之王、法之自在天之义。且如说:\begin{quoting}婆罗门!彼世尊令未生起的道生起,令未产生的道产生,宣说未曾宣说的道,知道、明道、熟习于道。且众弟子现今追随道而住,将来得具足之。(中部·牧者目揵连经第 79 段)\end{quoting}\textbf{离爱者},即离欲爱、有爱、无有爱者。
\item \textbf{两足尊},即两足者中的最上。然而,世尊何止是两足尊,而是无足、两足……非想非非想等一切有情中的最上,只是以高贵的部分而称为两足尊。因为两足者为一切有情中的高贵者,转轮王、大声闻、辟支佛投生于此,且在说为彼等的最上时,即是说为一切有情的最上。
\item \textbf{御者之最胜},以引导为「御者」,即调象者等的同义语,且世尊由能以无上调御调御可调御之人而为彼等之最上。如说:\begin{quoting}诸比丘!可调御之象受调象者的引导,唯往一个方向跑去,东方、西方、北方或南方。诸比丘!可调御之马受调马者……诸比丘!可调御之牛受调牛者……或南方。诸比丘!可调御之人受如来、阿罗汉、正等正觉者的引导,分奔至八个方向:具色者见色,这是一个方向……具足想受灭而住,这是第八个方向。(中部·六入处分别经第 312 段)\end{quoting}
\item \textbf{几种},即问事物的品类。\textbf{世间},即有情世间。\textbf{沙门},即表明所问之事。\textbf{快},即请求之义的不变词。\textbf{请说说},即请宣说、请谈论。\end{enumerate}

\subsection\*{\textbf{84}}

\textbf{「有四种沙门,没有第五种,纯陀!」世尊说,「作为见证,我向你解释这些,\\}
\textbf{「胜道者,示道者,依道生活者,以及污道者\footnote{四种沙门:长阿含作「行道殊胜、善说道义、依道生活、为道作秽」。}。」}

“Caturo samaṇā na pañcam’atthi, \textit{(Cundā ti Bhagavā)}\\
\makebox[2em]{} te te āvikaromi sakkhipuṭṭho;\\
maggajino maggadesako ca, magge jīvati yo ca maggadūsī”. %\hfill\textcolor{gray}{\footnotesize 2}

\begin{enumerate}\item 如是说已,世尊看到锻工之子纯陀不以「尊者!什么是善、什么是不善」等方法问在家的问题,却问关于沙门的问题,经转向便了知到「他就那恶比丘而问」,为显明他除俗称外的非沙门性,说了此颂。
\item 这里,\textbf{四种}是数量的限定。\textbf{沙门},世尊有时以沙门之语说外道,如说:\begin{quoting}凡是种种沙门、婆罗门的禁戒、庆典、祥瑞……(中部·大爱尽经第 407 段)\end{quoting}有时说凡夫,如说:\begin{quoting}「沙门、沙门」,诸比丘!人们如是认为。(中部·小马邑经第 435 段)\end{quoting}有时说有学,如说:\begin{quoting}于此有沙门,于此有第二沙门……(中部·小狮子吼经第 139 段)\end{quoting}有时说漏尽者,如说:\begin{quoting}由诸漏已尽而为沙门。(中部·小马邑经第 438 段)\end{quoting}有时则说自己,如说:\begin{quoting}诸比丘!沙门即如来的同义语。(增支部·第 8:85 经)\end{quoting}而这里,以三句摄一切圣者与具戒的凡夫,以第四摄其他仅以秃头、袈裟绕颈而俗称为沙门的非沙门,而说「四种沙门」。\textbf{没有第五种},即在此法律中,以俗称乃至以认可而言,没有第五种沙门。\textbf{我向你解释这些},即我为你厘清这四种沙门。\textbf{作为见证},即当面被问。
\item \textbf{胜道者},即以道战胜一切烦恼者之义。\textbf{示道者},即向他人开示道者。\textbf{依道生活者},即凡七有学中由未终了而住于出世间道的学人,以及依世间道生活的具戒凡夫,或者,具戒的凡夫以出世间道为目标而生活,当知为依道生活。\textbf{污道者},即恶戒者、邪见者、以逆道而行的污道者之义。\end{enumerate}

\subsection\*{\textbf{85}}

\textbf{「诸佛说谁是胜道者?」锻工之子纯陀说,「宣道者如何无等伦?\\}
\textbf{「既然问到,请对我说说依道生活者,然后,请对我解释污道者!」}

“Kaṃ maggajinaṃ vadanti buddhā, \textit{(iti Cundo kammāraputto)}\\
\makebox[2em]{} maggakkhāyī kathaṃ atulyo hoti;\\
magge jīvati me brūhi puṭṭho, atha me āvikarohi maggadūsiṃ”. %\hfill\textcolor{gray}{\footnotesize 3}

\begin{enumerate}\item 如是,世尊简略地指出「这些即四种沙门」,而于四种沙门中,纯陀尚不能通达「此中,此为胜道者,此为示道者,此为依道生活者,此为污道者」,便再发问,说了此颂。\end{enumerate}

\subsection\*{\textbf{86}}

\textbf{「度诸犹疑,离于箭刺,喜于涅槃,无有贪求,\\}
\textbf{「俱有天的世间的导师,诸佛说此等是胜道者。}

“Yo tiṇṇakathaṅkatho visallo, nibbānābhirato anānugiddho;\\
lokassa sadevakassa netā, tādiṃ maggajinaṃ vadanti buddhā. %\hfill\textcolor{gray}{\footnotesize 4}

\begin{enumerate}\item 现在,世尊为他以四颂说明四种沙门,说了此颂。这里,\textbf{度诸犹疑、离于箭刺}已如蛇经(第 17 颂)中所述,其差别处如下。因为此颂旨在以「胜道者」说明佛沙门,所以当知由以一切知智度过与犹疑相适的、于一切法的无智而为「度诸犹疑」。而以先前所说的方法,度诸犹疑的须陀洹等乃至辟支佛,于斯陀含的境域等乃至佛的境域,由受限之智的影响,依次仍有未度的犹疑,而世尊则以一切方式度诸犹疑。
\item \textbf{喜于涅槃},即心总是以果定倾向涅槃之义,如世尊等。如说:\begin{quoting}阿奇舍那!我在此谈话终了,仍于此先前的三摩地之相在内安置、静置、专注、等持其心。(中部·大萨遮经第 387 段)\end{quoting}\textbf{无有贪求},即不以渴爱之贪求贪求任何法。
\item \textbf{俱有天的世间的导师},即以随顺于意乐与倾向而开示法,在彼岸道\footnote{彼岸道:即\textbf{经集}第五品的内容。}、大集会\footnote{大集会 \textit{Mahāsamaya},见\textbf{正游行经}义注。}等的诸多经中,使无量天、人成就谛的通达,而为俱有天的世间的导师、遣送者、度脱者、令达彼岸者之义。\textbf{此等},即如此者,无变于所说品类的世间法者之义。其余于此自明。\end{enumerate}

\subsection\*{\textbf{87}}

\textbf{「于此了知了最上为最上,即于此宣说、分别法,\\}
\textbf{「他们说这断疑者、牟尼、不动者是第二种比丘、示道者。}

Paramaṃ paraman ti yo’dha ñatvā, akkhāti vibhajate idh’eva dhammaṃ;\\
taṃ kaṅkhachidaṃ muniṃ anejaṃ, dutiyaṃ bhikkhunam āhu maggadesiṃ. %\hfill\textcolor{gray}{\footnotesize 5}

\begin{enumerate}\item 如是,世尊以前颂说明了佛沙门为「胜道者」,现在为说明漏尽沙门而说此颂。这里,\textbf{最上}名为涅槃,即一切法中的顶点、至高之义。\textbf{于此了知了为最上},即以省察智了知了此最上于此教法内确为最上。
\item \textbf{即于此宣说、分别法},即宣说涅槃法,由自己通达而向他人表明「此是涅槃」,分别道法「这些是四念处……八支圣道」。或者,以略开示对众敏知者宣说两者,以详开示对众广知者分别两者\footnote{敏知者 \textit{ugghaṭitaññū}、广知者 \textit{vipañcitaññū},见\textbf{增支部}第 4:133 经,另两种人为可予引导者 \textit{neyya}、仅通文字者 \textit{padaparama}。}。且当如是宣说、分别时作狮子吼「此法唯于此教法中,非此以外」而宣说、分别,因此说「即于此宣说、分别法」。
\item \textbf{这断疑者、牟尼、不动者},即这般以通达四谛而断自身的、并以开示而断他人的疑惑的断疑者,以具足寂默的牟尼,以无有称为动摇的渴爱的不动者,\textbf{他们说是第二种比丘、示道者}。\end{enumerate}

\subsection\*{\textbf{88}}

\textbf{「于善开示的法句依道生活,自制,具念,\\}
\textbf{「从事着无过的行迹,他们说第三种比丘是活道者。}

Yo dhammapade sudesite, magge jīvati saññato satīmā;\\
anavajjapadāni sevamāno, tatiyaṃ bhikkhunam āhu maggajīviṃ. %\hfill\textcolor{gray}{\footnotesize 6}

\begin{enumerate}\item 如是,虽然自身令无上之道生起,并以开示而成为无上的示道者,却仍以前颂说明了传播、光耀自己教法的漏尽沙门为「示道者」,如同国王的使者、宣旨者一般,现在为说明有学沙门及具戒凡夫沙门而说此颂。
\item 这里的释词自明。而此中的释义为:由作为涅槃法的行迹而为\textbf{法句}\footnote{此颂第一句中「法句」之「句」与第三句中的「行迹」为同一词 pada,义注以「部分 \textit{koṭṭhāsa}」解之,Norman 英译作 path,菩提比丘则于两处分别作 trail, ways of conduct,且 Norman 在其法句英译中总结有「足、足迹,词语、诗行、教义,状态」等义。}。由避开两端而开示,或由顺适意乐,以念处等种种品类而开示为\textbf{善开示}。即便具有道,但由未完成道的作用而\textbf{依道生活}。以戒制御为\textbf{自制}。以于身等善安置之念,或以忆念长久所作之念为\textbf{具念}。由无有细微之量的罪过为无过,且由以部分为行迹,称三十七菩提分法为\textbf{无过的行迹}。从坏(随观)智\footnote{坏随观智,见\textbf{清净道论}·说行道智见清净品。}开始的修习、从事为\textbf{从事}。\textbf{他们说}这\textbf{第三种比丘是活道者}。\end{enumerate}

\subsection\*{\textbf{89}}

\textbf{「披了善行者的外衣,唐突,污家,鲁莽,\\}
\textbf{「欺瞒、不自制、伪装,以模仿而行,那是污道者。}

Chadanaṃ katvāna subbatānaṃ, pakkhandī kuladūsako pagabbho;\\
māyāvī asaññato palāpo, patirūpena caraṃ sa maggadūsī. %\hfill\textcolor{gray}{\footnotesize 7}

\begin{enumerate}\item 如是,世尊以前颂说明了有学沙门及具戒凡夫沙门为「活道者」,现在为说明这仅以秃头、袈裟绕颈的俗称沙门而说此颂。
\item 这里,\textbf{披了外衣},即模仿、伪装、持有外表之义。\textbf{善行者},即佛、辟支佛、声闻众等。因为彼等行为善妙,所以被称为善行者。\textbf{唐突},即突入、进入内部之义。因为恶戒者如为了掩盖粪便而以草叶等遮盖一般,为了掩盖自己恶戒的状态而披了善行者的外衣,以「我也是比丘」突入比丘中间,当以「如许年资的比丘当得此」分配利养时,他以「我即如许年资」突入而得,因此说「披了善行者的外衣,唐突」。
\item 以不适当的行道污染刹帝利等四家中生起的净喜为\textbf{污家}。\textbf{鲁莽},即具足八处身鲁莽、四处语鲁莽及多处意鲁莽之义。这于此是略说,我们将在慈经注中详述。
\item 由具足以覆藏所作为相的伪善为\textbf{欺瞒}。以无有戒等制御为\textbf{不自制}。由与糠相似为\textbf{伪装}。因为好比糠,虽然内无粒实,以外稃而被看作稻米,如是,于此有人虽然内无戒等功德之实,以外披善行者的沙门衣装而被看作沙门,他由如是与糠相似,得称为「伪装」。但在入出息念经中说:\begin{quoting}诸比丘!此集会无糠,诸比丘!此集会离糠,清净而住于坚实。(中部·入出息念经第 146 段)\end{quoting}如是良善凡夫也被称为糠,而于此及于法行经(第 285 颂)中,犯波罗夷者被称为糠。
\item \textbf{以模仿而行,那是污道者},即披了善行者的外衣,好比世人认为「那是林野住者、树下住者、粪扫衣者、常乞食者、少欲者、知足者」等的行者般,如是以模仿、样貌正当、外表鲜洁的正行而行,且污染自己的出世间道与他人的善趣之道,当知此人为「污道者」。\end{enumerate}

\subsection\*{\textbf{90}}

\textbf{「若在家人通达了这些,便为多闻、圣人弟子、有慧,\\}
\textbf{「了知到一切并非这般,见到如此,他的信便不退失,\\}
\textbf{「因为他如何能把败坏与未败坏、清净与不清净等同起来?」}

Ete ca paṭivijjhi yo gahaṭṭho, sutavā ariyasāvako sapañño;\\
sabbe n’etādisā ti ñatvā, iti disvā na hāpeti tassa saddhā;\\
kathaṃ hi duṭṭhena asampaduṭṭhaṃ, suddhaṃ asuddhena samaṃ kareyyā” ti. %\hfill\textcolor{gray}{\footnotesize 8}

\begin{enumerate}\item 如是,以前颂说明了恶戒的俗称沙门为「污道者」,现在为显明他们彼此的不调和性而说此颂。
\item 其义为:\textbf{若}刹帝利、婆罗门或其他任何\textbf{在家人}以所说之相\textbf{通达了}、了知了、证得了\textbf{这些}四种沙门,\textbf{便}以听闻这四种沙门之相\textbf{为多闻},由在圣人跟前听闻此相为\textbf{圣人弟子},以能了知彼等沙门「彼彼为如是之相」为\textbf{有慧},\textbf{了知到}其余\textbf{一切并非}如最后所说的污道者\textbf{这般},\textbf{见到如此},即见到如是作恶的这恶比丘。
\item 这里,其章句为:若在家人通达了这些,便为多闻、圣人弟子而有慧,如是而住者,以此慧了知到一切并非这般,见到如此,信便不退失,即便见到作恶的恶比丘,信也不退失、不减损、不灭失。
\item 如是,以此颂显明了他们彼此的不调和性,现在为赞赏见到如此而了知「一切并非这般」的圣人弟子而说末二句。其连结为:对多闻的圣人弟子,这是适当的,即见到有些人作恶,了知一切并非这般。什么原因?\textbf{因为他如何能把败坏与未败坏、清净与不清净等同起来}?
\item 其义为:多闻有慧的圣人弟子,如何能把欠缺戒而败坏的污道者与未败坏的其余三种沙门,清净的三种沙门与如是身正行等不遍净、不清净的最后一种的俗称沙门等同起来?岂能如是了知?
\item 在经的终了,未谈论优婆塞的道果,他只是舍断了疑惑。\end{enumerate}

\begin{center}\vspace{1em}纯陀经第五\\Cundasuttaṃ pañcamaṃ.\end{center}

%\begin{flushright}癸卯十二月初五二稿\end{flushright}