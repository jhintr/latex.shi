\chapter{臣工之什詁訓傳第二十七}

\section{臣工}

%{\footnotesize 一章、十五句}

\textbf{臣工,諸侯助祭遣於廟也。}

\textbf{嗟嗟臣工,敬爾在公,王釐爾成,來咨來茹。}{\footnotesize 嗟嗟,敕之也。工,官也。公,君也。箋云臣,謂諸侯也。釐理、咨謀、茹度也。諸侯來朝天子,有不純臣之義,於其將歸,故於廟中正君臣之禮,敕其諸官卿大夫云,敬女在君之事,王乃平理女之成功,女有事當來謀之、來度之於王之朝,無自專。}\textbf{嗟嗟保介,維莫之春,亦又何求,如何新畬。}{\footnotesize 田二歲曰新,三歲曰畬。箋云保介,車右也,月令「孟春,天子親載耒耜,措之于參保介之御間」。莫,晚也,周之季春,於夏為孟春,諸侯朝周之春,故晚春遣之。敕其車右以時事,女歸當何求於民,將如新田、畬田何急,其教農趨時也。介,甲也。車右,勇力之士,被甲執兵也。}\textbf{於皇來牟,將受厥明,明昭上帝,迄用康年。}{\footnotesize 康,樂也。箋云將大、迄至也。於美乎,赤烏以牟麥俱來,故我周家大受其光明,謂為珍瑞,天下所休慶也,此瑞乃明見於天,至今用之,有樂歲,五穀豐熟。}\textbf{命我眾人,庤乃錢鎛,奄觀銍艾。}{\footnotesize 庤具、錢銚、鎛耨、銍穫也。箋云奄久、觀多也。教我庶民,具女田器,終久必多銍艾,勸之也。}

\begin{quoting}嗟嗟 \texttt{jué},發語詞。釐,同賚,賜也。\textbf{馬瑞辰}來者,詞之是也,來咨來茹,猶言是咨是茹。保介,田畯也。明,收成。迄至,\textbf{馬瑞辰}至,猶致也,迄用康年,猶云用致康年。庤 \texttt{zhì},準備。錢 \texttt{jiǎn},鍬也。鎛 \texttt{bó},鋤也。銍 \texttt{zhì},說文「獲禾短鐮也」。艾,同乂,說文「芟草也」。\end{quoting}

\section{噫嘻}

%{\footnotesize 一章、八句}

\textbf{噫嘻,春夏祈穀于上帝也。}{\footnotesize 祈,猶禱也、求也,月令「孟春,祈穀于上帝,夏則龍見而雩」是與。}

\textbf{噫嘻成王,既昭假爾。率時農夫,播厥百穀。}{\footnotesize 噫,歎也。嘻,敕也。成王,成是王事也。箋云噫嘻,有所多大之聲也。假,至也。播,猶種也。噫嘻乎能成周王之功,其德已著至矣,謂「光被四表,格于上下」也,又能率是主田之吏農夫,使民耕田而種百穀也。}\textbf{駿發爾私,終三十里。亦服爾耕,十千維耦。}{\footnotesize 私,民田也。言上欲富其民而讓於下,欲民之大發其私田耳。終三十里,言各極其望也。箋云駿,疾也。發,伐也。亦大、服事也。使民疾耕,發其私田,竟三十里者,一部一吏主之,於是民大事耕其私田,萬耦同時舉也。周禮曰「凡治野田,夫間有遂,遂上有徑,十夫有溝,溝上有畛,百夫有洫,洫上有塗,千夫有澮,澮上有道,萬夫有川,川上有路」,計此萬夫之地,方三十三里少半里也,耜廣五寸,二耜為耦,一川之間萬夫,故有萬耦。耕言三十里者,舉其成數。}

\section{振鷺}

%{\footnotesize 一章、八句}

\textbf{振鷺,二王之後來助祭也。}{\footnotesize 二王,夏殷也,其後,杞也宋也。}

\textbf{振鷺于飛,于彼西雝。我客戾止,亦有斯容。}{\footnotesize 興也。振振,群飛貌。鷺,白鳥也。雝,澤也。客,二王之後。箋云白鳥集于西雝之澤,言所集得其處也,興者,喻杞宋之君有絜白之德,來助祭於周之廟,得禮之宜也。其至止亦有此容,言威儀之善如鷺然。}\textbf{在彼無惡,在此無斁。庶幾夙夜,以永終譽。}{\footnotesize 箋云在彼,謂居其國無怨惡之者,在此,謂其來朝人皆愛敬之,無厭之者。永,長也。譽,聲美也。}

\begin{quoting}雝,同邕,典籍中多作雍。戾,段注「訓為至,皆於曲義引伸之」。終,韓詩、魯詩作眾,\textbf{馬瑞辰}終與眾古通用,後漢書崔駰傳「豈可不庶幾夙夜,以永眾譽」,義本三家詩。眾譽,即盛譽。\end{quoting}

\section{豐年}

%{\footnotesize 一章、七句}

\textbf{豐年,秋冬報也。}{\footnotesize 報者,謂嘗也、烝也。}

\textbf{豐年多黍多稌,亦有高廪,萬億及秭。}{\footnotesize 豐大、稌稻也。廪,所以藏齍盛之穗也。數萬至萬曰億,數億至億曰秭。箋云豐年,大有年也。亦,大也。萬億及秭,以言穀數多。}\textbf{為酒為醴,烝畀祖妣。以洽百禮,降福孔皆。}{\footnotesize 皆,徧也。箋云烝進、畀予也。}

\begin{quoting}稌 \texttt{tú}。十萬為億,十億為秭。洽,合也。百禮,謂牲玉幣帛等祭品。\end{quoting}

\section{有瞽}

%{\footnotesize 一章、十三句}

\textbf{有瞽,始作樂而合乎祖也。}{\footnotesize 王者治定制禮,功成作樂。合者,大合諸樂而奏之。}

\textbf{有瞽有瞽,在周之庭。設業設虡,崇牙樹羽。應田縣鼓,鞉磬柷圉。}{\footnotesize 瞽,樂官也。業,大板也,所以飾栒為縣也,捷業如鋸齒,或曰畫之。植者為虡,衡者為栒。崇牙上飾卷然,可以縣也。樹羽,置羽也。應,小鞞也。田,大鼓也。縣鼓,周鼓也。鞉,鞉鼓也。柷,木椌也。圉,楬也。箋云瞽,矇也,以為樂官者,目無所見,於音聲審也,周禮「上瞽四十人,中瞽百人,下瞽百六十人」,有視瞭者相之,又設縣鼓。田,當作朄,朄,小鼓,在大鼓傍,應鞞之屬也,聲轉字誤,變而作田。}\textbf{既備乃奏,簫管備舉。喤喤厥聲,肅雝和鳴,先祖是聽。}{\footnotesize 箋云既備者,縣也朄也皆畢已也,乃奏,謂樂作也。簫,編小竹管,如今賣餳者所吹也。管如篴,併而吹之。}\textbf{我客戾止,永觀厥成。}{\footnotesize 箋云我客,二王之後也。長多其成功,謂深感於和樂,遂入善道,終無愆過。}

\begin{quoting}崇牙,樅 \texttt{cōng} 也,見靈臺注。朄 \texttt{yǐng}。鞉 \texttt{táo}。柷圉 \texttt{zhù yǔ},又作敔,釋名「柷以作樂,敔以止樂」。篴 \texttt{dí}。\textbf{朱熹}成,樂闋也。\end{quoting}

\section{潛}

%{\footnotesize 一章、六句}

\textbf{潛,季冬薦魚、春獻鮪也。}{\footnotesize 冬魚之性定,春鮪新來,薦獻之者,謂於宗廟也。}

\textbf{猗與漆沮,潛有多魚。有鱣有鮪,鰷鱨鰋鯉。}{\footnotesize 漆沮,岐周之二水也。潛,糝也。箋云猗與,歎美之言也。鱣,大鯉也。鮪,鮥也。鰷,白鰷也。鰋,鮎也。}\textbf{以享以祀,以介景福。}{\footnotesize 箋云介助、景大也。}

\begin{quoting}潛,魯詩、韓詩作涔,\textbf{王先謙}案列木水中,魚得藏隱,有若池然,故曰魚池。\end{quoting}

\section{雝}

%{\footnotesize 一章、十六句}

\textbf{雝,禘大祖也。}{\footnotesize 禘,大祭也,大於四時而小於祫。大祖,謂文王。}

\begin{quoting}\textbf{朱熹}此但為武王祭文王而徹俎之詩,而後通用於他廟耳。\end{quoting}

\textbf{有來雝雝,至止肅肅。相維辟公,天子穆穆。於薦廣牡,相予肆祀。}{\footnotesize 相助、廣大也。箋云雝雝,和也。肅肅,敬也。有是來時雝雝然,既至止而肅肅然者,乃助王禘祭百辟與諸侯也,天子是時則穆穆然,於進大牡之牲,百辟與諸侯又助我陳祭祀之饌,言得天下之歡心。}\textbf{假哉皇考,綏予孝子。宣哲維人,文武維后。}{\footnotesize 假,嘉也。箋云宣,徧也。嘉哉君考,斥文王也,文王之德乃安我孝子,謂受命定其基業也,又徧使天下之人有才知,以文德武功為之君故。}\textbf{燕及皇天,克昌厥後。綏我眉壽,介以繁祉。}{\footnotesize 燕,安也。箋云繁,多也。文王之德安及皇天,謂降瑞應、無變異也,又能昌大其子孫,安助之以考壽與多福祿。}\textbf{既右烈考,亦右文母。}{\footnotesize 烈考,武王也。文母,大姒也。箋云烈,光也。子孫所以得考壽與多福者,乃以見右助於光明之考與文德之母,歸美焉。}

\begin{quoting}\textbf{馬瑞辰}按宣哲與文武對舉,二字平列,朱子集傳訓「宣為通、哲為知」是也,宣之言顯,顯,明也,宣哲,猶言明哲也。史記燕世家索引「人,猶臣也,文王以一身兼盡君臣之道,故言維人、維后」。綏,通賚,賜也。右,通侑,勸也。\end{quoting}

\section{載見}

%{\footnotesize 一章、十四句}

\textbf{載見,諸侯始見乎武王廟也。}

\begin{quoting}\textbf{陳奐}成王之世,武王廟為禰廟,武王主喪畢入禰廟,而諸侯於是乎始見之,此其樂歌也。\end{quoting}

\textbf{載見辟王,曰求厥章。龍旂陽陽,和鈴央央。鞗革有鶬,休有烈光。}{\footnotesize 載,始也。龍旂陽陽,言有文章也。和在軾前,鈴在旂上。鞗革有鶬,言有法度也。箋云諸侯始見君王,謂見成王也,曰求其章者,求車服禮儀之文章制度也。交龍為旂。鞗革,轡首也。鶬,金飾貌。休者,休然盛壯。}\textbf{率見昭考,以孝以享,以介眉壽。永言保之,思皇多祜。}{\footnotesize 昭考,武王也。享,獻也。箋云言我、皇君也。諸侯既以朝禮見於成王,至祭時,伯又率之見於武王廟,使助祭也,以致孝子之事,以獻祭祀之禮,以助考壽之福,長我安行此道,思使成王之多福。}\textbf{烈文辟公,綏以多福。俾緝熙于純嘏。}{\footnotesize 箋云俾使、純大也。祭有十倫之義,成王乃光文百辟與諸侯,安之以多福,使光明於大嘏之意。天子受福曰大嘏,辭有福祚之言。}

\begin{quoting}和鈴,即和鑾,見蓼蕭注。文王為穆,武王為昭。\textbf{馬瑞辰}孝與享同義,故享祀亦曰孝祀,此詩「以孝以享」,猶潛詩「以享以祀」,皆二字同義,合言之則曰孝享。思,發語詞。綏,安也賜也,段注「論語曰『升車必正立執綏』,周生烈曰『正立執綏,所以為安』,按引申為凡安之偁」。\end{quoting}

\section{有客}

%{\footnotesize 一章、十二句}

\textbf{有客,微子來見祖廟也。}{\footnotesize 成王既黜殷命,殺武庚,命微子代殷後,既受命,來朝而見也。}

\textbf{有客有客,亦白其馬。有萋有且,敦琢其旅。}{\footnotesize 殷尚白也。亦,亦周也。萋且,敬慎貌。箋云有客有客,重言之者,異之也。亦,亦武庚也,武庚為二王後,乘殷之馬,乃叛而誅,不肖之甚也,今微子代之,亦乘殷之馬,獨賢而見尊異,故言亦駮而美之。其來威儀萋萋且且,盡心力於其事,又選擇眾臣卿大夫之賢者,與之朝王。言敦琢者,以賢美之,故至言之。}\textbf{有客宿宿,有客信信。言授之縶,以縶其馬。}{\footnotesize 一宿曰宿,再宿曰信。欲縶其馬而留之。箋云縶,絆也。周之君臣皆愛微子,其所館宿,可以去矣,而言絆其馬,意各殷勤。}\textbf{薄言追之,左右綏之。}{\footnotesize 箋云追,送也。於微子去,王始言餞送之,左右之臣又欲從而安樂之,厚之無已。}\textbf{既有淫威,降福孔夷。}{\footnotesize 淫大、威則、夷易也。箋云既有大則,謂用殷正朔行其禮樂如天子也,神與之福,又甚易也,言動作而有度。}

\begin{quoting}左傳僖二十四年,皇武子曰「宋,先代之後,於周為客」。\textbf{馬瑞辰}萋且雙聲字,皆狀從者之盛。敦 \texttt{duī} 琢,即雕琢。旅,通侶,隨從也。\textbf{馬瑞辰}既有淫威,猶云既有大德耳。又曰說文「夷」从大从弓,古夷字必有大訓。\end{quoting}

\section{武}

%{\footnotesize 一章、七句}

\textbf{武,奏大武也。}{\footnotesize 大武,周公作樂所為舞也。}

\begin{quoting}禮記樂記「武樂有六」,\textbf{王國維}大武樂章考以武、桓、賚、酌、般、我將為六成。左傳宣十二年,楚子曰「武王克商,作武,其卒章曰『耆定爾功』,其三曰『鋪時繹思,我徂惟求定』,其六曰『綏萬邦,屢豐年』。」\end{quoting}

\textbf{於皇武王,無競維烈。允文文王,克開厥後。}{\footnotesize 烈,業也。箋云皇,君也。於乎君哉,武王也,無彊乎其克商之功業,言其彊也,信有文德哉,文王也,能開其子孫之基緒。}\textbf{嗣武受之,勝殷遏劉,耆定爾功。}{\footnotesize 武迹、劉殺、耆致也。箋云遏止、耆老也。嗣子武王受文王之業,舉兵伐殷而勝之,以止天下之暴虐而殺人者,年老乃定女之此功,言不汲汲於誅紂,須暇五年。}

\begin{quoting}耆 \texttt{zhǐ},同厎,致使。\end{quoting}

%\begin{flushright}臣工之什十篇、十章、一百六句\end{flushright}