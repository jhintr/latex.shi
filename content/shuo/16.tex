\chapter{企羨第十六}

\subsection*{1}

\textbf{王丞相拜司空,桓廷尉作兩髻、葛帬、策杖,路邊窺之,歎曰:「人言阿龍超,阿龍故自超。」}{\footnotesize 阿龍,丞相小字。}\textbf{不覺至臺門。}

\subsection*{2}

\textbf{王丞相過江,自說昔在洛水邊,數與裴成公、阮千里諸賢共談道,羊曼曰:「人久以此許卿,何須復爾?」王曰:「亦不言我須此,但欲爾時不可得耳。」}{\footnotesize 欲,一作歎。}

\subsection*{3}

\textbf{王右軍得人以蘭亭集序方金谷詩序,又以己敵石崇,甚有欣色。}{\footnotesize \textbf{王羲之臨河敘}曰永和九年,歲在癸丑,莫春之初,會于會稽山陰之蘭亭,修禊事也,群賢畢至,少長咸集,此地有崇山峻嶺,茂林修竹,又有清流激湍,映帶左右,引以為流觴曲水,列坐其次,是日也,天朗氣清,惠風和暢,娛目騁懷,信可樂也,雖無絲竹管絃之盛,一觴一詠,亦足以暢敘幽情矣,故列序時人,錄其所述,右將軍司馬太原孫丞公等二十六人,賦詩如左,前餘姚令會稽謝勝等十五人不能賦詩,罰酒各三斗。}

\subsection*{4}

\textbf{王司州先為庾公記室參軍,後取殷浩為長史,始到,庾公欲遣王使下都,王自啓求住曰:「下官希見盛德,淵源始至,猶貪與少日周旋。」}

\subsection*{5}

\textbf{郗嘉賓得人以己比苻堅,大喜。}

\subsection*{6}

\textbf{孟昶未達時,家在京口,}{\footnotesize \textbf{晉安帝紀}曰昶,字彥達,平昌人,父馥,中護軍,昶矜嚴有志局,少為王恭所知,豫義旗之勳,遷丹陽尹,盧循既下,昶慮事不濟,仰藥而死。}\textbf{嘗見王恭乘高輿,被鶴氅裘,于時微雪,昶於籬間窺之,歎曰:「此真神仙中人。」}