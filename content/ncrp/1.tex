\chapter{Lesson 1}

\section*{Reading 1}

Buddhaṃ saraṇaṃ gacchāmi.\\
Dhammaṃ saraṇaṃ gacchāmi.\\
Saṅghaṃ saraṇaṃ gacchāmi.

Dutiyaṃ pi buddhaṃ saraṇaṃ gacchāmi.\\
Dutiyaṃ pi dhammaṃ saraṇaṃ gacchāmi.\\
Dutiyaṃ pi saṅghasaraṇaṃ gacchāmi.

Tatiyaṃ pi buddhaṃ saraṇaṃ gacchāmi.\\
Tatiyaṃ pi dhammaṃ saraṇaṃ gacchāmi.\\
Tatiyaṃ pi saṅghaṃ saraṇaṃ gacchāmi. \hfill(Khp 1, Saraṇattayaṃ)

\section*{Reading 2}

Cittaṃ, bhikkhave, adantaṃ mahato anatthāya saṃvattatī ti. Cittaṃ, bhikkhave, dantaṃ mahato atthāya saṃvattatī ti.

Cittaṃ, bhikkhave, aguttaṃ mahato anatthāya saṃvattatī ti. Cittaṃ, bhikkhave, guttaṃ mahato atthāya saṃvattatī ti.

Cittaṃ, bhikkhave, arakkhitaṃ mahato anatthāya saṃvattatī ti. Cittaṃ, bhikkhave, rakkhitaṃ mahato atthāya saṃvattatī ti.

Cittaṃ, bhikkhave, asaṃvutaṃ mahato anatthāya saṃvattatī ti. Cittaṃ, bhikkhave, saṃvutaṃ mahato atthāya saṃvattatī ti.

Nāhaṃ, bhikkhave, aññaṃ ekadhammaṃ pi samanupassāmi, yaṃ evaṃ adantaṃ aguttaṃ arakkhitaṃ asaṃvutaṃ mahato anatthāya saṃvattati, yathayidaṃ cittaṃ. Cittaṃ, bhikkhave, adantaṃ aguttaṃ arakkhitaṃ asaṃvutaṃ mahato anatthāya saṃvattatī ti. \hfill(A 1.4)

\section*{Reading 3}

Micchādiṭṭhikassa, bhikkhave, anuppannā c’eva akusalā dhammā uppajjanti, uppannā ca akusalā dhammā bhiyyobhāvāya vepullāya saṃvattanti.

Nāhaṃ, bhikkhave, aññaṃ ekadhammaṃ pi samanupassāmi, yena anuppannā vā kusalā dhammā nūppajjanti, uppannā vā kusalā dhammā parihāyanti, yathayidaṃ, bhikkhave, micchādiṭṭhi.

Micchādiṭṭhikassa, bhikkhave, anuppannā c’eva kusalā dhammā nūppajjanti, uppannā ca kusalā dhammā parihāyanti.

Nāhaṃ, bhikkhave, aññaṃ ekadhammaṃ pi samanupassāmi, yena anuppannā vā akusalā dhammā nūppajjanti, uppannā vā akusalā dhammā parihāyanti, yathayidaṃ, bhikkhave, sammādiṭṭhi.

Sammādiṭṭhikassa, bhikkhave, anuppannā c’eva akusalā dhammā nūppajjanti, uppannā ca akusalā dhammā parihāyanti. \hfill(A 1.16.2)

\section*{Reading 4}

Bhikkhu… anuppannānaṃ pāpakānaṃ akusalānaṃ dhammānaṃ anuppādāya chandaṃ janeti; vāyamati; viriyaṃ ārabhati, cittaṃ paggaṇhāti; padahati.

… uppannānaṃ pāpakānaṃ akusalānaṃ dhammānaṃ pahānāya chandaṃ janeti; vāyamati; viriyaṃ ārabhati, cittaṃ paggaṇhāti; padahati.

… anuppannānaṃ kusalānaṃ dhammānaṃ uppādāya chandaṃ janeti; vāyamati; viriyaṃ ārabhati, cittaṃ paggaṇhāti; padahati.

… uppannānaṃ kusalānaṃ dhammānaṃ ṭhitiyā asammosāya bhiyyobhāvāya vepullāya bhāvanāya pāripūriyā chandaṃ janeti; vāyamati; viriyaṃ ārabhati, cittaṃ paggaṇhāti; padahati. \hfill(A 1.18)

\section*{Further Reading 1}

Nāhaṃ, bhikkhave, aññaṃ ekadhammaṃ pi samanupassāmi, yo evaṃ saddhammassa sammosāya antaradhānāya saṃvattati, yathayidaṃ, bhikkhave, pamādo. Pamādo, bhikkhave, saddhammassa sammosāya antaradhānāya saṃvattatī ti.

Nāhaṃ, bhikkhave, aññaṃ ekadhammaṃ pi samanupassāmi, yo evaṃ saddhammassa ṭhitiyā asammosāya anantaradhānāya saṃvattati, yathayidaṃ, bhikkhave, appamādo. Appamādo, bhikkhave, saddhammassa ṭhitiyā asammosāya anantaradhānāya saṃvattatī ti.

Nāhaṃ, bhikkhave, aññaṃ ekadhammaṃ pi samanupassāmi, yaṃ evaṃ saddhammassa sammosāya antaradhānāya saṃvattati, yathayidaṃ, bhikkhave, kosajjaṃ. Kosajjaṃ, bhikkhave, saddhammassa sammosāya antaradhānāya saṃvattatī ti.

Nāhaṃ, bhikkhave, aññaṃ ekadhammaṃ pi samanupassāmi, yo evaṃ saddhammassa ṭhitiyā asammosāya anantaradhānāya sạṃvattati, yathayidaṃ, bhikkhave, viriyārambho. Viriyārambho, bhikkhave, saddhammassa ṭhitiyā asammosāya anantaradhānāya saṃvattatī ti.

Nāhaṃ, bhikkhave, aññaṃ ekadhammaṃ pi samanupassāmi, yo evaṃ saddhammassa ṭhitiyā asammosāya anantaradhānāya, yathayidaṃ, bhikkhave, anuyogo kusalānaṃ dhammānaṃ, ananuyogo akusalānaṃ dhammānaṃ. Anuyogo, bhikkhave, kusalānaṃ dhammānaṃ, ananuyogo akusalānaṃ dhammānaṃ saddhammassa ṭhitiyā asammosāya anantaradhānāya saṃvattatī ti.

\section*{Further Reading 2}

Nāhaṃ, bhikkhave, aññaṃ ekarūpaṃ pi samanupassāmi, yaṃ evaṃ purisassa cittaṃ pariyādāya tiṭṭhati, yathayidaṃ, bhikkhave, itthirūpaṃ. Itthirūpaṃ, bhikkhave, purisassa cittaṃ pariyādāya tiṭṭhatī ti.

Nāhaṃ, bhikkhave, aññaṃ ekasaddaṃ pi samanupassāmi, yaṃ evaṃ purisassa cittaṃ pariyādāya tiṭṭhati, yathayidaṃ, bhikkhave, itthisaddo. Itthisaddo, bhikkhave, purisassa cittaṃ pariyādāya tiṭṭhatī ti.

Nāhaṃ, bhikkhave, aññaṃ ekagandhaṃ pi samanupassāmi, yaṃ evaṃ purisassa cittaṃ pariyādāya tiṭṭhati, yathayidaṃ, bhikkhave, itthigandho. Itthigandho, bhikkhave, purisassa cittaṃ pariyādāya tiṭṭhatī ti.

Nāhaṃ, bhikkhave, aññaṃ ekarasaṃ pi samanupassāmi, yaṃ evaṃ purisassa cittaṃ pariyādāya tiṭṭhati, yathayidaṃ, bhikkhave, itthiraso. Itthiraso, bhikkhave, purisassa cittaṃ pariyādāya tiṭṭhatī ti.

Nāhaṃ, bhikkhave, aññaṃ ekaphoṭṭhabbaṃ pi samanupassāmi, yaṃ evaṃ purisassa cittaṃ pariyādāya tiṭṭhati, yathayidaṃ, bhikkhave, itthiphoṭṭhabbo. Itthiphoṭṭhabbo, bhikkhave, purisassa cittaṃ pariyādāya tiṭṭhatī ti.

Nāhaṃ, bhikkhave, aññaṃ ekarūpaṃ pi samanupassāmi, yaṃ evaṃ itthiyā cittaṃ pariyādāya tiṭṭhati, yathayidaṃ, bhikkhave, purisarūpaṃ. Purisarūpaṃ, bhikkhave, itthiyā cittaṃ pariyādāya tiṭṭhatī ti.

Nāhaṃ, bhikkhave, aññaṃ ekasaddaṃ pi samanupassāmi, yaṃ evaṃ itthiyā cittaṃ pariyādāya tiṭṭhati, yathayidaṃ, bhikkhave, purisasaddo. Purisasaddo, bhikkhave, itthiyā cittaṃ pariyādāya tiṭṭhatī ti.

Nāhaṃ, bhikkhave, aññaṃ ekagandhaṃ pi samanupassāmi, yaṃ evaṃ itthiyā cittaṃ pariyādāya tiṭṭhati, yathayidaṃ, bhikkhave, purisagandho. Purisagandho, bhikkhave, itthiyā cittaṃ pariyādāya tiṭṭhatī ti.

Nāhaṃ, bhikkhave, aññaṃ ekarasaṃ pi samanupassāmi, yaṃ evaṃ itthiyā cittaṃ pariyādāya tiṭṭhati, yathayidaṃ, bhikkhave, purisaraso. Purisaraso, bhikkhave, itthiyā cittaṃ pariyādāya tiṭṭhatī ti.

Nāhaṃ, bhikkhave, aññaṃ ekaphoṭṭhabbaṃ pi samanupassāmi, yaṃ evaṃ itthiyā cittaṃ pariyādāya tiṭṭhati, yathayidaṃ, bhikkhave, purisaphoṭṭhabbaṃ. Purisaphoṭṭhabbaṃ, bhikkhave, itthiyā cittaṃ pariyādāya tiṭṭhatī ti.