\chapter{王黍離詁訓傳第六}

\begin{quoting}\textbf{釋文}王國者,周室東都王城畿內之地,在豫州,今之洛陽是也,幽王滅,平王東遷,政遂微弱,詩不能復雅,下列稱風,以王當國,猶春秋稱王人。\end{quoting}

\section{黍離}

%{\footnotesize 三章、章十句}

\textbf{黍離,閔宗周也。周大夫行役至于宗周,過故宗廟宮室,盡為禾黍,閔周室之顛覆,彷徨不忍去,而作是詩也。}{\footnotesize 宗周,鎬京也,謂之西周,周王城也,謂之東周,幽王之亂而宗周滅,平王東遷,政遂微弱,下列於諸侯,其詩不能復雅而同於國風焉。}

\textbf{彼黍離離,彼稷之苗。}{\footnotesize 彼,彼宗廟宮室。箋云宗廟宮室毀壞而其地盡為禾黍,我以黍離離時至,稷則尚苗。}\textbf{行邁靡靡,中心搖搖。}{\footnotesize 邁,行也。靡靡,猶遲遲也。搖搖,憂無所愬。箋云行,道也,道行,猶行道也。}\textbf{知我者,謂我心憂。}{\footnotesize 箋云知我者,知我之情。}\textbf{不知我者,謂我何求。}{\footnotesize 箋云謂我何求,怪我久留不去。}\textbf{悠悠蒼天,此何人哉。}{\footnotesize 悠悠,遠意。蒼天,以體言之,尊而君之則稱皇天,元氣廣大則稱昊天,仁覆閔下則稱旻天,自上降監則稱上天,據遠視之蒼蒼然則稱蒼天。箋云遠乎蒼天,仰愬欲其察己言也,此亡國之君何等人哉,疾之甚。}

\begin{quoting}\textbf{馬瑞辰}程瑤田九穀考謂黍今之黃米,稷今之高粱,其說是也,稷以春種,黍以夏種,而詩言黍離離、稷尚苗者,稷種在黍先,秀在黍後故也。搖,三家詩作愮,爾雅「愮愮,憂無吿也」。\end{quoting}

\textbf{彼黍離離,彼稷之穗。}{\footnotesize 穗,秀也。詩人自黍離離見稷之穗,故歷道其所更見。}\textbf{行邁靡靡,中心如醉。}{\footnotesize 醉於憂也。}\textbf{知我者,謂我心憂。不知我者,謂我何求。悠悠蒼天,此何人哉。}

\textbf{彼黍離離,彼稷之實。}{\footnotesize 自黍離離見稷之實。}\textbf{行邁靡靡,中心如噎。}{\footnotesize 噎,憂不能息也。}\textbf{知我者,謂我心憂。不知我者,謂我何求。悠悠蒼天,此何人哉。}

\section{君子于役}

%{\footnotesize 二章、章八句}

\textbf{君子于役,刺平王也。君子行役無期度,大夫思其危難以風焉。}

\begin{quoting}\textbf{許瑤光}再讀詩經四十二首之十四「雞棲于桀下牛羊,饑渴縈懷對夕陽,已啟唐人閨怨句,最難消遣是昏黃」。\end{quoting}

\textbf{君子于役,不知其期,曷至哉。}{\footnotesize 箋云曷,何也。君子往行役,我不知其反期,何時當來至哉,思之甚。}\textbf{雞棲于塒,日之夕矣,羊牛下來。}{\footnotesize 鑿牆而棲曰塒。箋云雞之將棲,日則夕矣,牛羊從下牧地而來,言畜產出入尚使有期節,至於行役者乃反不也。}\textbf{君子于役,如之何勿思。}{\footnotesize 箋云行役多危難,我誠思之。}

\textbf{君子于役,不日不月,曷其有佸。}{\footnotesize 佸,會也。箋云行役反無日月,何時而有來會期。}\textbf{雞棲于桀,日之夕矣,羊牛下括。}{\footnotesize 雞棲于杙為桀。括,至也。}\textbf{君子于役,苟無飢渴。}{\footnotesize 箋云苟,且也。且得無飢渴,憂其飢渴也。}

\begin{quoting}佸 \texttt{huó}。釋文「括,本亦作佸」。\end{quoting}

\section{君子陽陽}

%{\footnotesize 二章、章四句}

\textbf{君子陽陽,閔周也。君子遭亂,相招為祿仕,全身遠害而已。}{\footnotesize 祿仕者,苟得祿而已,不求道行。}

\textbf{君子陽陽,左執簧,右招我由房。}{\footnotesize 陽陽,無所用其心也。簧,笙也。由,用也。國君有房中之樂。箋云由,從也。君子祿仕在樂官,左手持笙,右手招我,欲使我從之於房中,俱在樂官也。我者,君子之友自謂也,時在位有官職也。}\textbf{其樂只且。}{\footnotesize 箋云君子遭亂,道不行,其且樂此而已。}

\begin{quoting}\textbf{陳奐}荀子儒效篇「則揚揚如也」,楊注云「得意之貌」,陽即揚之假借。\textbf{胡承珙}由房者,房中,對廟朝言之,人君燕息時所奏之樂,非廟朝之樂,故曰房中。只且,語詞。\end{quoting}

\textbf{君子陶陶,左執翿,右招我由敖。}{\footnotesize 陶陶,和樂貌。翿,纛也、翳也。箋云陶陶,猶陽陽也。翳,舞者所持,謂羽舞也。君子左手持羽,右手招我,欲使我從之於燕舞之位,亦俱在樂官也。}\textbf{其樂只且。}

\begin{quoting}翿 \texttt{dào}。\textbf{陳奐}翳者,謂以翳覆頭也。\textbf{馬瑞辰}敖,疑當讀如鷔夏之鷔,周官鍾師,奏九夏,其九為鷔夏。\end{quoting}

\section{揚之水}

%{\footnotesize 三章、章六句}

\textbf{揚之水,刺平王也。不撫其民而遠屯戍于母家,周人怨思焉。}{\footnotesize 怨平王恩澤不行於民,而久令屯戍,不得歸,思其鄉里之處者。言周人者,時諸侯亦有使人戍焉,平王母家申國,在陳鄭之南,迫近彊楚,王室微弱而數見侵伐,王是以戍之。}

\textbf{揚之水,不流束薪。}{\footnotesize 興也。揚,激揚也。箋云激揚之水至湍迅而不能流移束薪,興者,喻平王政教煩急而恩澤之令不行于下民。}\textbf{彼其之子,不與我戍申。}{\footnotesize 戍,守也。申,姜姓之國,平王之舅。箋云之子,是子也。彼其是子獨處鄉里,不與我來守申,是思之言也。其,或作記,或作己,讀聲相似。}\textbf{懷哉懷哉,曷月予還歸哉。}{\footnotesize 箋云懷,安也。思鄉里處者,故曰今亦安不哉安不哉,何月我得還歸見之哉,思之甚。}

\begin{quoting}\textbf{朱熹}揚,悠揚也,水緩流之貌。\textbf{聞一多}析薪、束薪蓋上世婚禮中實有之儀式,非泛泛舉譬也。其,語詞。還,旋也。\end{quoting}

\textbf{揚之水,不流束楚。}{\footnotesize 楚,木也。}\textbf{彼其之子,不與我戍甫。}{\footnotesize 甫,諸姜也。}\textbf{懷哉懷哉,曷月予還歸哉。}

\begin{quoting}楚,較薪更小者。\textbf{陳奐}甫,即呂國,詩及孝經、禮記皆作甫,尚書、左傳、國語皆作呂,甫呂古同聲。\end{quoting}

\textbf{揚之水,不流束蒲。}{\footnotesize 蒲,草也。箋云蒲,蒲柳。}\textbf{彼其之子,不與我戍許。}{\footnotesize 許,諸姜也。}\textbf{懷哉懷哉,曷月予還歸哉。}

\begin{quoting}案薪也楚也蒲也漸輕,申也甫也許也漸近洛陽。\end{quoting}

\section{中谷有蓷}

%{\footnotesize 三章、章六句}

\textbf{中谷有蓷,閔周也。夫婦日以衰薄,凶年饑饉,室家相棄爾。}

\textbf{中谷有蓷,暵其乾矣。}{\footnotesize 興也。蓷,鵻也。暵,菸貌。陸草生於谷中,傷於水。箋云興者,喻人居平安之世,猶鵻之生於陸,自然也,遇衰亂凶年,猶鵻之生谷中,得水則病將死。}\textbf{有女仳離,嘅其嘆矣。}{\footnotesize 仳,別也。箋云有女遇凶年而見棄,與其君子別離,嘅然而嘆,傷己見棄,其恩薄。}\textbf{嘅其嘆矣,遇人之艱難矣。}{\footnotesize 艱,亦難也。箋云所以嘅然而嘆者,自傷遇君子之窮厄。}

\begin{quoting}蓷 \texttt{tuī},益母草,其性宜濕。暵 \texttt{hàn},蔫貌。仳 \texttt{pǐ}。嘅,同慨。\end{quoting}

\textbf{中谷有蓷,暵其脩矣。}{\footnotesize 脩,且乾也。}\textbf{有女仳離,條其歗矣。}{\footnotesize 條條然歗也。}\textbf{條其歗矣,遇人之不淑矣。}{\footnotesize 箋云淑,善也。君子於己不善也。}

\begin{quoting}\textbf{陳奐}乾肉謂之脯,亦謂之脩,因之凡乾皆曰脩矣。條,長也。\end{quoting}

\textbf{中谷有蓷,暵其濕矣。}{\footnotesize 鵻遇水則濕。箋云鵻之傷於水,始則濕,中而脩,久而乾,有似君子於己之恩,徒用凶年深淺為厚薄。}\textbf{有女仳離,啜其泣矣。}{\footnotesize 啜,泣貌。}\textbf{啜其泣矣,何嗟及矣。}{\footnotesize 箋云及,與也。泣者傷其君子棄己,嗟乎,將復何與為室家乎,此其有餘厚於君子也。}

\begin{quoting}\textbf{王念孫}濕,當讀為㬤 \texttt{qī},㬤亦乾也。啜,韓詩作惙。\end{quoting}

\section{兔爰}

%{\footnotesize 三章、章七句}

\textbf{兔爰,閔周也。桓王失信,諸侯背叛,構怨連禍,王師傷敗,君子不樂其生焉。}{\footnotesize 不樂其生者,寐不欲覺之謂也。}

\textbf{有兔爰爰,雉離于羅。}{\footnotesize 興也。爰爰,緩意。鳥網為羅。言為政有緩有急,用心之不均。箋云有緩者,有所聽縱也,有急者,有所躁蹙也。}\textbf{我生之初,尚無為。}{\footnotesize 尚無成人為也。箋云尚,庶幾也。言我幼稚之時,庶幾於無所為,謂軍役之事也。}\textbf{我生之後,逢此百罹,尚寐無吪。}{\footnotesize 罹憂、吪動也。箋云我長大之後,乃遇此軍役之多憂,今但庶幾於寐,不欲見動,無所樂生之甚。}

\begin{quoting}\textbf{馬瑞辰}韓詩「爰,發蹤之貌」,胡承珙曰「蹤,當作縱,發縱,謂解放之,即鄭箋聽縱之義」,其說是也。\textbf{朱熹}言張羅本以取兔,今兔狡得脫,而雉以耿介反離于羅。\end{quoting}

\textbf{有兔爰爰,雉離于罦。}{\footnotesize 罦,覆車也。}\textbf{我生之初,尚無造。}{\footnotesize 造,為也。}\textbf{我生之後,逢此百憂,尚寐無覺。}

\begin{quoting}罦 \texttt{fú}。\end{quoting}

\textbf{有兔爰爰,雉離于罿。}{\footnotesize 罿,罬也。}\textbf{我生之初,尚無庸。}{\footnotesize 庸,用也。箋云庸,勞也。}\textbf{我生之後,逢此百凶,尚寐無聦。}{\footnotesize 聦,聞也。箋云百凶者,王構怨連禍之凶。}

\begin{quoting}罿 \texttt{tóng}。\textbf{陳奐}無用者,謂無用師之苦。\end{quoting}

\section{葛藟}

%{\footnotesize 三章、章六句}

\textbf{葛藟,王族刺平王也。周室道衰,棄其九族焉。}{\footnotesize 九族者,據己上至高祖,下及玄孫之親。}

\textbf{緜緜葛藟,在河之滸。}{\footnotesize 興也。緜緜,長不絕之貌。水厓曰滸。箋云葛也藟也生於河之厓,得其潤澤以長大而不絕,興者,喻王之同姓得王之恩施,以生長其子孫。}\textbf{終遠兄弟,謂他人父。}{\footnotesize 兄弟之道已相遠矣。箋云兄弟,猶言族親也。王寡於恩施,今已遠棄族親矣,是我謂他人為己父,族人尚親親之辭。}\textbf{謂他人父,亦莫我顧。}{\footnotesize 箋云謂他人為己父,無恩於我,亦無顧眷我之意。}

\begin{quoting}\textbf{陳奐}傳云「兄弟之道已相遠矣」者,以已釋終,為全詩終字通訓,既醉傳又以終字釋既字,終、既、已三字同義。遠 \texttt{yuàn}。\end{quoting}

\textbf{緜緜葛藟,在河之涘。}{\footnotesize 涘,厓也。}\textbf{終遠兄弟,謂他人母。}{\footnotesize 王又無母恩。}\textbf{謂他人母,亦莫我有。}{\footnotesize 箋云有,識有也。}

\begin{quoting}有,同友,左傳昭二十年「是不有寡君也」,杜注「有,相親有也」。\end{quoting}

\textbf{緜緜葛藟,在河之漘。}{\footnotesize 漘,水隒也。}\textbf{終遠兄弟,謂他人昆。}{\footnotesize 昆,兄也。}\textbf{謂他人昆,亦莫我聞。}{\footnotesize 箋云不與我相聞命也。}

\begin{quoting}爾雅釋丘「夷上洒下不漘」,郭注「厓上平坦而下水深者為漘,不,發聲」。\textbf{王引之}聞,猶問也,謂相恤問也。\end{quoting}

\section{采葛}

%{\footnotesize 三章、章三句}

\textbf{采葛,懼讒也。}{\footnotesize 桓王之時,政事不明,臣無大小使出者則為讒人所毀,故懼之。}

\textbf{彼采葛兮,一日不見,如三月兮。}{\footnotesize 興也。葛,所以為絺綌也。事雖小,一日不見於君,憂懼於讒矣。箋云興者,以采葛喻臣以小事使出。}

\textbf{彼采蕭兮,一日不見,如三秋兮。}{\footnotesize 蕭,所以共祭祀。箋云彼采蕭者,喻臣以大事使出。}

\begin{quoting}周禮甸師「祭祀共蕭茅」,杜子春注「蕭,香蒿也」。\end{quoting}

\textbf{彼采艾兮,一日不見,如三歲兮。}{\footnotesize 艾,所以療疾。箋云彼采艾者,喻臣以急事使出。}

\begin{quoting}\textbf{朱熹}艾,蒿屬,乾之可以灸,故采之。\end{quoting}

\section{大車}

%{\footnotesize 三章、章四句}

\textbf{大車,刺周大夫也。禮義陵遲,男女淫奔,故陳古以刺今大夫不能聽男女之訟焉。}

\textbf{大車檻檻,毳衣如菼。}{\footnotesize 大車,大夫之車。檻檻,車行聲也。毳衣,大夫之服。菼,鵻也,蘆之初生者也。天子大夫四命,其出封五命,如子男之服,乘其大車檻檻然,服毳冕以決訟。箋云菼,薍也。古者天子大夫服毳冕以巡行邦國而決男女之訟,則是子男入為大夫者。毳衣之屬,衣繢而裳繡,皆有五色焉,其青者如鵻。}\textbf{豈不爾思,畏子不敢。}{\footnotesize 畏子大夫之政終不敢。箋云此二句者,古之欲淫奔者之辭,我豈不思與女以為無禮與,畏子大夫來聽訟,將罪我,故不敢也。子者,稱所尊敬之辭。}

\begin{quoting}檻 \texttt{kǎn}。說文「毳 \texttt{cuì},獸細毛也」。菼 \texttt{tǎn},色綠也。\end{quoting}

\textbf{大車啍啍,毳衣如璊。}{\footnotesize 啍啍,重遲之貌。璊,赬也。}\textbf{豈不爾思,畏子不奔。}

\begin{quoting}啍 \texttt{tūn}。璊 \texttt{mén},色赤也。\end{quoting}

\textbf{穀則異室,死則同穴。謂予不信,有如皦日。}{\footnotesize 穀生、皦白也。生在於室則外內異,死則神合,同為一也。箋云穴,謂塚壙中也。此章言古之大夫聽訟之政,非但不敢淫奔,乃使夫婦之禮有別,今之大夫不能然,反謂我言不信,我言之信如白日也,刺其闇於古禮。}

\section{丘中有麻}

%{\footnotesize 三章、章四句}

\textbf{丘中有麻,思賢也。莊王不明,賢人放逐,國人思之而作是詩也。}{\footnotesize 思之者,思其來,己得見之。}

\textbf{丘中有麻,彼畱子嗟。}{\footnotesize 留,大夫氏。子嗟,字也。丘中磽埆之處,盡有麻麥草木,乃彼子嗟之所治。箋云子嗟放逐於朝,去治卑賤之職而有功,所在則治理,所以為賢。}\textbf{彼畱子嗟,將其來施施。}{\footnotesize 施施,難進之意。箋云施施,舒行,伺間獨來見己之貌。}

\begin{quoting}\textbf{馬瑞辰}留、劉古通用。將,請也。\textbf{顏之推}書證篇曰詩云「將其來施施」,河北毛詩皆云施施,江南舊本悉單為施。\textbf{馬瑞辰}施,亦為也助也。\end{quoting}

\textbf{丘中有麥,彼畱子國。}{\footnotesize 子國,子嗟父。箋云言子國使丘中有麥,著其世賢。}\textbf{彼畱子國,將其來食。}{\footnotesize 子國復來,我乃得食。箋云言其將來食,庶其親己,己得厚待之。}

\textbf{丘中有李,彼畱之子。}{\footnotesize 箋云丘中而有李,又留氏之子所治。}\textbf{彼畱之子,貽我佩玖。}{\footnotesize 玖,石次玉者。言能遺我美寶。箋云留氏之子於思者則朋友之子,庶其敬己而遺己也。}

%\begin{flushright}王國十篇、二十八章、百六十二句\end{flushright}