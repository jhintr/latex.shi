\chapter{卷之七\hspace{1ex}疏祭文}

\section{與子儼等疏}

\textbf{告儼、俟、份、佚、佟:}

\textbf{天地賦命,生必有死,自古賢聖,誰獨能免,子夏有言曰「死生有命,富貴在天」,四友之人,親受音旨,發斯談者,將非窮達不可妄求、壽夭永無外請故耶?}

\textbf{吾年過五十,少而窮苦,每以家弊,東西遊走,性剛才拙,與物多忤,自量為己,必貽俗患,僶俛辭世,使汝等幼而飢寒。余嘗感孺仲賢妻之言\footnote{後漢書王霸傳、列女傳:霸,字孺仲,光武時連徵不仕,與同郡令狐子伯為友,後子伯為楚相,而其子為郡功曹,子伯乃令子奉書於霸,客去而久臥不起,妻怪問其故,曰「吾與子伯素不相若,向見其子容服甚光,舉措有適,而我兒曹蓬髮歷齒,未知禮則,見客而有慚色,父子恩深,不覺自失耳」,妻曰「君少修清節,不顧榮祿,今子伯之貴孰與君之高?君躬勤苦,子女安得不耕以養?既耕安得不黃頭歷齒?奈何忘宿志而慚兒女子乎」,霸屈起而笑曰「有是哉」,遂共終身隱遁。},敗絮自擁,何慚兒子,此既一事矣。但恨鄰靡二仲\footnote{高士傳:求仲、羊仲皆治車為業,挫廉逃名,蔣元卿之去兗州,還杜陵,荊棘塞門,舍中有三徑,不出,惟二人從之遊,時人謂之二仲。},室無萊婦\footnote{列女傳:萊子逃世,耕於蒙山之陽,王欲聘以璧帛,妻曰「妾聞之,可食以酒肉者,可隨以鞭捶,可授以官祿者,可隨以鈇鉞,今先生食人酒肉,受人官祿,能免於患乎」。},抱茲苦心,良獨內愧。}

\textbf{少學琴書,偶愛閑靜,開卷有得,便欣然忘食,見樹木交蔭,時鳥變聲,亦復歡然有喜。常言「五六月中,北窗下臥,遇涼風暫至,自謂是羲皇上人」,意淺識罕,謂斯言可保,日月遂往,機巧好疏,緬求在昔,眇然如何。}

\textbf{疾患以來,漸就衰損,親舊不遺,每以藥石見救,自恐大分將有限也。汝輩稚小家貧,每役柴水之勞,何時可免?念之在心,若何可言。然汝等雖不同生,當思四海皆兄弟之義,鮑叔、管仲分財無猜,歸生、伍舉班荊道舊,遂能以敗為成,因喪立功,他人尚爾,況同父之人哉?}

\textbf{潁川韓元長,漢末名士,身處卿佐,八十而終,兄弟同居,至於沒齒。濟北氾稚春,晉時操行人也,七世同財,家人無怨色。詩曰「高山仰止,景行行止」,雖不能爾,至心尚之,汝其慎哉,吾復何言。}

\section{祭程氏妹文}

\textbf{維晉義熙三年,五月甲辰,程氏妹服制\footnote{已嫁姊妹,應服大功,為期九月。}再周,淵明以少牢之奠,俯而酹之:}

\textbf{嗚呼哀哉!寒往暑來,日月寖疏。梁塵委積,庭草荒蕪。寥寥空室,哀哀遺孤。肴觴虛奠,人逝焉如。}

\textbf{誰無兄弟,人亦同生。嗟我與爾,特百常情。慈妣早世,時尚孺嬰。我年二六,爾纔九齡。爰從靡識,撫髫相成。咨爾令妹,有德有操。靖恭鮮言,聞善則樂。能正能和,惟友惟孝。行止中閨,可象可效。我聞為善,慶自己蹈。彼蒼何偏,而不斯報。}

\textbf{昔在江陵,重罹天罰。兄弟索居,乖隔楚越。伊我與爾,百哀是切。黯黯高雲,蕭蕭冬月。白雪掩晨,長風悲節。感惟崩號,興言泣血。}

\textbf{尋念平昔,觸事未遠。書疏猶存,遺孤滿眼。如何一往,終天不返。寂寂高堂,何時復踐。藐藐孤女,曷依曷恃。煢煢遊魂,誰主誰祀。奈何程妹,于此永已。死如有知,相見蒿里。嗚呼哀哉!}

\section{祭從弟敬遠文}

\textbf{歲在辛亥,月惟仲秋,旬有九日,從弟敬遠,卜辰云窆\footnote{窆 \texttt{biǎn}:葬下棺也。},永寧后土。感平生之遊處,悲一往之不返,情惻惻以摧心,淚愍愍而盈眼,乃以園果時醪,祖其將行。}

\textbf{嗚呼哀哉!於鑠我弟,有操有概。孝發幼齡,友自天愛。少思寡欲,靡執靡介。後己先人,臨財思惠。心遺得失,情不依世。其色能溫,其言則厲。樂勝朋高,好是文藝。遙遙帝鄉,爰感奇心。絕粒委務,考槃山陰。淙淙懸溜,曖曖荒林。晨採上藥,夕閑素琴。}

\textbf{曰仁者壽,竊獨信之。如何斯言,徒能見欺。年甫過立,奄與世辭。長歸蒿里,邈無還期。惟我與爾,匪但親友。父則同生,母則從母\footnote{豫章書:孟嘉以二女妻陶侃子茂之二子,一生淵明,一生敬遠,是敬遠之母為先生從母也。}。相及齠齒\footnote{韓詩外傳:男子八月生齒,八歲而齠齒。},並罹偏咎\footnote{言二人俱遭父喪。}。斯情實深,斯愛實厚。念疇昔日,同房之歡。冬無縕褐,夏渴瓢簞。相將以道,相開以顏。豈不多乏,忽忘飢寒。}

\textbf{余嘗學仕,纏綿人事。流浪無成,懼負素志。斂策歸來,爾知我意。常願攜手,寘彼眾議。每憶有秋,我將其刈。與汝偕行,舫舟同濟。三宿水濱,樂飲川界。}

\textbf{靜月澄高,溫風始逝。撫杯而言,物久人脆。奈何吾弟,先我離世。事不可尋,思亦何極。日徂月流,寒暑代息。死生異方,存亡有域。候晨永歸,指塗載陟。}

\textbf{呱呱遺稚,未能正言。哀哀嫠人,禮儀孔閑。庭樹如故,齋宇廓然。孰云敬遠,何時復還。}

\textbf{余惟人斯,昧茲近情。蓍龜有吉,制我祖行。望旐翩翩,執筆涕盈。神其有知,昭余中誠。嗚呼哀哉!}

\section{自祭文}

\textbf{歲惟丁卯,律中無射\footnote{禮記月令:季秋之月,其音商,律中無射。},天寒夜長,風氣蕭索,鴻雁于征,草本黃落,陶子將辭逆旅之館,永歸於本宅,故人悽其相悲,同祖行於今夕,羞以嘉蔬,薦以清酌,候顏已冥,聆音愈漠。}

\textbf{嗚呼哀哉!茫茫大塊,悠悠高旻。是生萬物,余得為人。自余為人,逢運之貧。簞瓢屢罄,絺綌冬陳。含歡谷汲,行歌負薪。翳翳柴門,事我宵晨。春秋代謝,有務中園。載耘載耔,迺育迺繁。欣以素牘,和以七弦。冬曝其日,夏濯其泉。勤靡餘勞,心有常閑。樂天委分,以至百年。}

\textbf{惟此百年,夫人愛之。懼彼無成,愒\footnote{爾雅釋言:愒 \texttt{qì},貪也。}日惜時。存為世珍,歿亦見思。嗟我獨邁,曾是異茲。寵非己榮,湼豈吾緇。捽兀窮廬,酣飲賦詩。識運知命,疇能罔眷?余今斯化,可以無恨。壽涉百齡,身慕肥遁。從老得終,奚所復戀。寒暑逾邁,亡既異存。外姻晨來,良友宵奔。葬之中野,以安其魂。窅窅我行,蕭蕭墓門。奢恥宋臣,儉笑王孫\footnote{楊王孫:見飲酒之十一。}。}

\textbf{廓兮已滅,慨焉已遐。不封不樹,日月遂過。匪貴前譽,孰重後歌。人生實難,死如之何。嗚呼哀哉!}

