\chapter{卷之六\hspace{1ex}記傳贊述}

\section{桃花源記\hspace{1ex}{\footnotesize 并詩}}

\textbf{晉太元中,武陵人捕魚為業。緣溪行,忘路之遠近,忽逢桃花林,夾岸數百步,中無雜樹,芳草鮮美,落英繽紛,漁人甚異之,復前行,欲窮其林。林盡水源,便得一山,山有小口,髣髴若有光,便捨船從口入。初極狹,纔通人,復行數十步,豁然開朗,土地平曠,屋舍儼然,有良田美池桑竹之屬,阡陌交通,雞犬相聞,其中往來種作,男女衣著悉如外人,黃髮垂髫並怡然自樂。見漁人,乃大驚,問所從來,具答之,便要還家,為設酒殺雞作食。村中聞有此人,咸來問訊,自云先世避秦時亂,率妻子邑人來此絕境,不復出焉,遂與外人間隔,問今是何世,乃不知有漢,無論魏晉,此人一一為具言所聞,皆歎惋。餘人各復延至其家,皆出酒食。停數日,辭去,此中人語云「不足為外人道也」。既出,得其船,便扶向路,處處志之,及郡下,詣太守說如此。太守即遣人隨其往,尋向所志,遂迷不復得路。南陽劉子驥,高尚士也,聞之,欣然規往,未果,尋病終。後遂無問津者。}

\begin{quoting}嬴氏亂天紀,賢者避其世。黃綺之商山,伊人亦云逝。往迹浸復湮,來徑遂蕪廢。相命肆農耕,日入從所憩。桑竹垂餘蔭,菽稷隨時藝。春蠶收長絲,秋熟靡王稅。荒路曖交通,雞犬互鳴吠。俎豆猶古法,衣裳無新製。童孺縱行歌,斑白歡遊詣。草榮識節和,木衰知風厲。雖無紀曆誌,四時自成歲。怡然有餘樂,于何勞智慧。奇蹤隱五百,一朝敞神界。淳薄既異源,旋復還幽蔽。借問游方士,焉測塵囂外。願言躡輕風,高舉尋吾契。\end{quoting}

\section{晉故征西大將軍長史孟府君傳}

\textbf{君諱嘉,字萬年,江夏鄂人也。曾祖父宗,以孝行稱,仕吳司空。祖父揖,元康中為廬陵太守。宗葬武昌陽新縣,子孫家焉,遂為縣人也。}

\textbf{君少失父,奉母二弟居,娶大司馬長沙桓公陶侃第十女,閨門孝友,人無能間,鄉閭稱之。沖默有遠量,弱冠,儔類咸敬之。同郡郭遜,以清操知名,時在君右,常歎君溫雅平曠,自以為不及,遜從弟立,亦有才志,與君同時齊譽,每推服焉,由是名冠州里,聲流京邑。}

\textbf{太尉潁川庾亮,以帝舅民望,受分陜之重,鎮武昌,并領江州,辟君部廬陵從事。下郡還,亮引見,問風俗得失,對曰「嘉不知,還傳\footnote{傳:傳舍、驛舍。}當問從吏」,亮以麈尾掩口而笑,諸從事既去,喚弟翼語之曰「孟嘉故是盛德人也」。君既辭出外,自除吏名,便步歸家,母在堂,兄弟共相歡樂,怡怡如也。旬有餘日,更版\footnote{版:授職、任命。}為勸學從事。時亮崇修學校,高選儒官,以君望實,故應尚德之舉。}

\textbf{太傅河南褚裒,簡穆有器識,時為豫章太守,出朝宗亮,正旦大會州府人士,率多時彥,君坐次甚遠,裒問亮「江州有孟嘉,其人何在」,亮云「在坐,卿但自覓」,裒歷觀,遂指君謂亮曰「將無是耶」,亮欣然而笑,喜裒之得君,奇君為裒之所得,乃益器焉。}

\textbf{舉秀才,又為安西將軍庾翼府功曹,再為江州別駕、巴丘令、征西大將軍譙國桓溫參軍。君色和而正,溫甚重之。九月九日,溫遊龍山,參佐畢集,四弟二甥咸在坐,時佐吏並著戎服,有風吹君帽墮落,溫目左右及賓客勿言,以觀其舉止。君初不自覺,良久如廁,溫命取以還之,廷尉太原孫盛為諮議參軍,時在坐,溫命紙筆令嘲之,文成示溫,溫以著坐處。君歸,見嘲笑而請筆作答,了不容思,文辭超卓,四座歎之。}

\textbf{奉使京師,除尚書刪定郎,不拜。孝宗穆皇帝聞其名,賜見東堂,君辭以腳疾,不任拜起,詔使人扶入。}

\textbf{君嘗為刺史謝永別駕,永,會稽人,喪亡,君求赴義\footnote{赴義:弔喪。},路由永興。高陽許詢,有雋才,辭榮不仕,每縱心獨往,客居縣界,嘗乘船近行,適逢君過,歎曰「都邑美士,吾盡識之,獨不識此人,唯聞中州有孟嘉者,將非是乎,然亦何由來此」,使問君之從者。君謂其使曰「本心相過,今先赴義,尋還就君」。及歸,遂止信宿,雅相知得,有若舊交。}

\textbf{還至,轉從事中郎,俄遷長史。在朝隤然,仗正順而已,門無雜賓。常會神情獨得,便超然命駕,徑之龍山,顧景酣宴,造夕乃歸。溫從容謂君曰「人不可無勢,我乃能駕御卿」。}

\textbf{後以疾終於家,年五十一。始自總髮,至于知命,行不茍合,言無夸矜,未嘗有喜慍之容,好酣飲,逾多不亂,至於任懷得意,融然遠寄,傍若無人。溫嘗問君「酒有何好而卿嗜之」,君笑而答曰「明公但不得酒中趣爾」。又問聽妓,絲不如竹,竹不如肉,答曰「漸近自然」。中散大夫桂陽羅含賦之曰「孟生善酣,不愆其意」。光祿大夫南陽劉耽,昔與君同在溫府,淵明從父太常夔嘗問耽「君若在,當已作公不」,答云「此本是三司人」。為時所重如此。淵明先親,君之第四女也,凱風寒泉之思,實鍾厥心,謹按採行事,撰為此傳,懼或乖謬,有虧大雅君子之德,所以戰戰兢兢,若履深薄云爾。}

\textbf{贊曰:孔子稱「進德修業,以及時也」,君清蹈衡門則令聞孔昭,振纓公朝則德音允集,道悠運促,不終遠業。惜哉!仁者必壽,豈斯言之謬乎。}

\section{五柳先生傳}

\textbf{先生不知何許人也,亦不詳其姓字,宅邊有五柳樹,因以為號焉。閑靜少言,不慕榮利,好讀書,不求甚解,每有會意,便欣然忘食。性嗜酒,家貧不能常得,親舊知其如此,或置酒而招之,造飲輒盡,期在必醉,既醉而退,曾不吝情去留。環堵蕭然,不蔽風日,短褐穿結,簞瓢屢空,晏如也。常著文章自娛,頗示己志,忘懷得失,以此自終。}

\textbf{贊曰:黔婁之妻有言「不戚戚於貧賤,不汲汲於富貴」,極其言,茲若人之儔乎?酣觴賦詩,以樂其志,無懷氏之民歟、葛天氏之民歟?}

\section{扇上畫贊}

\begin{quoting}荷蓧丈人\hspace{1ex}長沮桀溺\hspace{1ex}於陵仲子\hspace{1ex}張長公\hspace{1ex}丙曼容\hspace{1ex}鄭次都\hspace{1ex}薛孟嘗\hspace{1ex}周陽珪\end{quoting}

\textbf{三五道邈,淳風日盡。九流參差,互相推隕。形逐物遷,心無常準。是以達人,有時而隱。}

\textbf{四體不勤,五穀不分。超超丈人,日夕在耘。遼遼沮溺,耦耕自欣。入鳥不駭,雜獸斯群。}

\textbf{至矣於陵\footnote{高士傳:陳仲子居於於陵,楚王聞其賢,遣使聘之,欲以為相,仲子入告其妻,妻曰「夫子左琴右書,樂在其中矣,結駟連騎,所甘不過一肉,而懷楚國之憂,可乎」,於是謝使者,遂相與逃而為人灌園。},養氣浩然。蔑彼結駟,甘此灌園。張生\footnote{張長公:見飲酒之十二。}一仕,曾以事還。顧我不能,高謝人間。}

\textbf{岧岧\footnote{岧岧 \texttt{tiáo}:山高貌。}丙公\footnote{何注:漢邴漢兄曼容,養志自修,為官不肯過六百石,輒自免去,其名過出於漢。},望崖輒歸。匪驕匪吝,前路威夷。鄭叟\footnote{何注:後漢鄭敬,字次都,都尉逼為功曹,辭病去,隱處精學。}不合,垂釣川湄。交酌林下,清言究微。}

\textbf{孟嘗\footnote{何注:後漢汝南薛包,字孟嘗。}遊學,天網時疏。眷言哲友,振褐偕徂。美哉周子,稱疾閑居。寄心清尚,悠然自娛。}

\textbf{翳翳衡門,洋洋泌流\footnote{詩陳風衡門:衡門之下,可以棲遲,泌之洋洋,可以樂飢。}。曰琴曰書,顧盻有儔。飲河既足,自外皆休。緬懷千載,託契孤遊。}

\section{讀史述九章\hspace{1ex}{\footnotesize 余讀史記有所感而述之}}

\begin{quoting}\textbf{夷齊}\end{quoting}

\textbf{二子讓國,相將海隅\footnote{孟子盡心:伯夷避紂,居北海之濱。}。天人革命,絕景窮居。采薇高歌,慨想黃虞。貞風凌俗,爰感懦夫\footnote{孟子萬章:故聞伯夷之風者,頑夫廉,懦夫有立志。}。}

\begin{quoting}\textbf{箕子}\end{quoting}

\textbf{去鄉之感,猶有遲遲。矧伊代謝,觸物皆非。哀哀箕子,云胡能夷。狡童之歌,悽矣其悲。}

\begin{quoting}\textbf{管鮑}\end{quoting}

\textbf{知人未易,相知實難。淡美初交,利乖歲寒。管生稱心,鮑叔必安。奇情雙亮,令名俱完。}

\begin{quoting}\textbf{程杵}\end{quoting}

\textbf{遺生良難,士為知己。望義如歸,允伊二子。程生揮劍,懼茲餘恥。令德永聞,百代見紀。}

\begin{quoting}\textbf{七十二弟子}\end{quoting}

\textbf{恂恂舞雩,莫曰匪賢。俱映日月,共飡至言。慟由才難,感為情牽。回也早夭,賜獨長年。}

\begin{quoting}\textbf{屈賈}\end{quoting}

\textbf{進德修業,將以及時。如彼稷契,孰不願之。嗟乎二賢,逢世多疑。候詹寫志,感鵩獻辭。}

\begin{quoting}\textbf{韓非}\end{quoting}

\textbf{豐狐隱穴,以文自殘\footnote{言豐狐以文美招禍,見韓非子喻老。}。君子失時,白首抱關。巧行居災,忮辯召患。哀矣韓生,竟死說難。}

\begin{quoting}\textbf{魯二儒}\end{quoting}

\textbf{易大隨時,迷變則愚。介介若人,特為貞夫\footnote{見史記劉敬叔孫通列傳。}。德不百年,污我詩書。逝然不顧,被褐幽居。}

\begin{quoting}\textbf{張長公}\end{quoting}

\textbf{遠哉長公,蕭然何事。世路多端,皆為我異。斂轡朅\footnote{朅 \texttt{qiè}:離去。}來,獨養其志。寢跡窮年,誰知斯意。}

