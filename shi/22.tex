\chapter{魚藻之什詁訓傳第二十二}

\section{魚藻}

%{\footnotesize 三章、章四句}

\textbf{魚藻,刺幽王也。言萬物失其性,王居鎬京,將不能以自樂,故君子思古之武王焉。}{\footnotesize 萬物失其性者,王政教衰,陰陽不和,群生不得其所也。將不能以自樂,言必自是有危亡之禍也。}

\textbf{魚在在藻,有頒其首。}{\footnotesize 頒,大首貌。魚以依蒲藻為得其性。箋云藻,水草也。魚之依水草,猶人之依明王也,明王之時,魚何所處乎,處於藻,既得其性則肥充,其首頒然,此時人物皆得其所,正言魚者,以潛逃之類,信其著見。}\textbf{王在在鎬,豈樂飲酒。}{\footnotesize 箋云豈亦樂也。天下平安,萬物得其性,武王何所處乎,處于鎬京,樂八音之樂,與群臣飲酒而已,今幽王惑于褒姒,萬物失其性,方有危亡之禍,而亦豈樂飲酒於鎬京,而無悛心,故以此刺焉。}

\begin{quoting}頒 \texttt{fén},魯詩作賁。都鎬始自武王。\end{quoting}

\textbf{魚在在藻,有莘其尾。}{\footnotesize 莘,長貌。}\textbf{王在在鎬,飲酒樂豈。}

\textbf{魚在在藻,依于其蒲。王在在鎬,有那其居。}{\footnotesize 箋云那,安貌。天下平安,王無四方之虞,故其居處那然安也。}

\section{采菽}

%{\footnotesize 五章、章八句}

\textbf{采菽,刺幽王也。侮慢諸侯,諸侯來朝,不能錫命,以禮數徵會之而無信義,君子見微而思古焉。}{\footnotesize 幽王徵會諸侯,為合義兵征討有罪,既往而無之,是於義事不信也。君子見其如此,知其後必見攻伐,將無救也。}

\begin{quoting}釋文「菽,本亦作叔」,左傳昭十七年「賦采叔」。\end{quoting}

\textbf{采菽采菽,筐之筥之。}{\footnotesize 興也。菽,所以芼大牢而待君子也。羊則苦,豕則薇。箋云菽,大豆也,采之者,采其葉以為藿,三牲牛羊豕芼以藿,王饗賓客,有生俎,乃用鉶羹,故使采之。}\textbf{君子來朝,何錫予之。雖無予之,路車乘馬。}{\footnotesize 君子,謂諸侯也。箋云賜諸侯以車馬,言雖無予之,尚以為薄。}\textbf{又何予之,玄袞及黼。}{\footnotesize 玄袞,卷龍也。白與黑謂之黼。箋云及,與也。玄袞,玄衣而畫以卷龍也。黼,黼黻,謂絺衣也,諸公之服自袞冕而下,侯伯自鷩冕而下,子男自毳冕而下,王之賜,維用有文章者。}

\begin{quoting}\textbf{陳奐}芼大牢本以待君子之禮,而言興者,因所事而興也。\end{quoting}

\textbf{觱沸檻泉,言采其芹。}{\footnotesize 觱沸,泉出貌。檻泉,正出也。箋云言,我也。芹,菜也,可以為菹,亦所用待君子也。我使采其水中芹者,尚潔清也,周禮「芹菹鴈醢」。}\textbf{君子來朝,言觀其旂。其旂淠淠,鸞聲嘒嘒。載驂載駟,君子所屆。}{\footnotesize 淠淠,動也。嘒嘒,中節也。箋云屆,極也。諸侯來朝,王使人迎之,因觀其衣服車乘之威儀,所以為敬,且省禍福也。諸侯將朝于王,則驂乘乘四馬而往。此之服飾,君子法制之極也,言其尊,而王今不尊也。}

\begin{quoting}觱 \texttt{bì} 沸,疊韻。檻泉,爾雅「濫泉,正出,正出,涌出也」。\textbf{馬瑞辰}周官「上公建旂九旒,侯伯七旒,子男五旒」,觀其所建旌旂,則諸侯之尊卑等級判焉。淠 \texttt{pèi}。驂,一車駕三馬,駟,一車駕四馬。\end{quoting}

\textbf{赤芾在股,邪幅在下。彼交匪紓,天子所予。}{\footnotesize 諸侯赤芾。邪幅,幅,偪也,所以自偪束也。紓,緩也。箋云芾,大古蔽膝之象也,冕服謂之芾,其他服謂之韠,以韋為之,其制上廣一尺,下廣二尺,長三尺,其頸五寸,肩革帶博二寸。脛本曰股。邪幅,如今行縢也,偪束其脛,自足至膝,故曰在下。彼與人交接,自偪束如此,則非有解怠紓緩之心,天子以是故賜予之。}\textbf{樂只君子,天子命之。樂只君子,福祿申之。}{\footnotesize 申,重也。箋云只之言是也。古者天子賜諸侯也,以禮樂樂之,乃後命予之也。天子賜之,神則以福祿申重之,所謂神謀鬼謀也,刺今王不然。}

\begin{quoting}邪幅,即今綁腿。彼交匪紓,荀子勸學篇引作「匪交匪紓」,交,同絞,急躁也,紓,怠慢也。\end{quoting}

\textbf{維柞之枝,其葉蓬蓬。}{\footnotesize 蓬蓬,盛貌。箋云此興也,柞之幹猶先祖也,枝猶子孫也,其葉蓬蓬,喻賢才也。正以柞為興者,柞之葉新,將生,故,乃落於地,以喻繼世以德相承者明也。}\textbf{樂只君子,殿天子之邦。樂只君子,萬福攸同。}{\footnotesize 殿,鎮也。}\textbf{平平左右,亦是率從。}{\footnotesize 平平,辯治也。箋云率,循也。諸侯之有賢才之德,能辯治其連屬之國,使得其所,則連屬之國亦循順之。}

\begin{quoting}蓬蓬,同芃芃。平平 \texttt{pián},釋文引韓詩作便便,古平、便同聲,即堯典平章、平秩之平。\end{quoting}

\textbf{汎汎楊舟,紼纚維之。}{\footnotesize 紼,繂也。纚,緌也。明王能維持諸侯也。箋云楊木之舟浮於水上,汎汎然東西無所定,舟人以紼繫其緌以制行之,猶諸侯之治民,御之以禮法。}\textbf{樂只君子,天子葵之。樂只君子,福祿膍之。}{\footnotesize 葵,揆也。膍,厚也。}\textbf{優哉游哉,亦是戾矣。}{\footnotesize 戾,至也。諸侯有盛德者亦優游,自安止於是,言思不出其位。箋云戾,止也。}

\begin{quoting}\textbf{馬瑞辰}紼蓋以麻為索,纚蓋以竹為索,皆所以維舟也。膍 \texttt{pí},釋文引韓詩作肶。優哉游哉,韓詩外傳引作「優哉柔哉」。左傳襄二十九年杜注「戾,定也」。末句為留客之詞,故章首以紼纚維繫楊舟起興。\end{quoting}

\section{角弓}

%{\footnotesize 八章、章四句}

\textbf{角弓,父兄刺幽王也。不親九族而好讒佞,骨肉相怨,故作是詩也。}

\textbf{騂騂角弓,翩其反矣。}{\footnotesize 興也。騂騂,調利也。不善紲檠巧用則翩然而反。箋云興者,喻王與九族不以恩禮御待之,則使之多怨也。}\textbf{兄弟昬姻,無胥遠矣。}{\footnotesize 箋云胥,相也。骨肉之親當相親信,無相疏遠,相疏遠則以親親之望易以成怨。弓之為物,張之則內向而來,弛之則外反而去,有似兄弟昏姻親疏遠近之意。又云騂騂角弓既翩然而反矣,兄弟昏姻則豈可以相遠哉。}

\begin{quoting}翩,同偏,說文「偏,頗也」,段注「頗,頭偏也,引伸為凡偏之稱」。\end{quoting}

\textbf{爾之遠矣,民胥然矣。爾之敎矣,民胥傚矣。}{\footnotesize 箋云爾,女,女幽王也。胥,皆也。言王,女不親骨肉則天下之人皆如之,見女之教令,無善無惡,所尚者,天下之人皆學之,言上之化下不可不慎。}

\begin{quoting}\textbf{馬瑞辰}詩以「教」與「遠」對言,遠為不善,則教當為善。\end{quoting}

\textbf{此令兄弟,綽綽有裕。不令兄弟,交相為瘉。}{\footnotesize 綽綽,寬也。裕饒、瘉病也。箋云令,善也。}

\textbf{民之無良,相怨一方。}{\footnotesize 箋云良,善也。民之意不獲,當反責之於身,思彼所以然者而恕之,無善心之人則徙居一處,怨恚之。}\textbf{受爵不讓,至于己斯亡。}{\footnotesize 爵祿不以相讓,故怨禍及之。比周而黨愈少,鄙爭而名愈辱,求安而身愈危。箋云斯,此也。}

\begin{quoting}\textbf{馬瑞辰}人之無良,一方之人皆怨之,至于己受爵不讓,亦為無良,則忘之也。斯,語詞。\end{quoting}

\textbf{老馬反為駒,不顧其後。}{\footnotesize 已老矣而孩童慢之。箋云此喻幽王見老人反侮慢之,遇之如幼稚,不自顧念後至年老,人之遇己亦將然。}\textbf{如食宜饇,如酌孔取。}{\footnotesize 饇,飽也。箋云王如食老者,則宜令之飽,如飲老者,則當孔取。孔取,謂度其所勝多少,凡器之孔,其量大小不同,老者氣力弱,故取義焉。王有族食、族燕之禮。}

\begin{quoting}食 \texttt{sì}。饇 \texttt{yù}。\end{quoting}

\textbf{毋敎猱升木,如塗塗附。}{\footnotesize 猱,猿屬。塗泥、附著也。箋云毋,禁辭。猱之性善登木,若教使其為之,必也。附,木桴也。塗之性善著,若以塗附,其著亦必也。以喻人之心皆有仁義,教之則進。}\textbf{君子有徽猷,小人與屬。}{\footnotesize 徽,美也。箋云猷,道也。君子有美道以得聲譽,則小人亦樂與之而自連屬焉,今無良之人相怨,王不教之。}

\begin{quoting}猱 \texttt{náo}。如,而也。初句意為勿教猱上樹,反用泥塗不使之升。\end{quoting}

\textbf{雨雪瀌瀌,見晛曰消。}{\footnotesize 晛,日氣也。箋云雨雪之盛瀌瀌然,至日將出,其氣始見,人則皆稱曰雪今消釋矣,喻小人雖多,王若欲興善政,則天下聞之,莫不曰小人今誅滅矣,其所以然者,人心皆樂善,王不啟教之。}\textbf{莫肯下遺,式居婁驕。}{\footnotesize 箋云莫,無也。遺,讀曰隨。式,用也。婁,斂也。今王不以善政啟小人之心,則無肯謙虛以禮相卑下,先人而後己,用此自居處,斂其驕慢之過者。}

\begin{quoting}見晛 \texttt{xiàn},疊韻,日出清明貌。\textbf{馬瑞辰}古者以雪喻小人,以雪之遇日氣而消喻小人之遇王政之清明而將敗也。遺,魯詩作遂。\textbf{陳奐}北門傳「遺,加也」,此遺字當亦訓加,婁,數也,「莫肯下遺,式居婁驕」,言小人之行不肯卑下加禮于人,唯數數驕慢好自用也。\end{quoting}

\textbf{雨雪浮浮,見晛曰流。}{\footnotesize 浮浮,猶瀌瀌也。流,流而去也。}\textbf{如蠻如髦,我是用憂。}{\footnotesize 蠻,南蠻也。髦,夷髦也。箋云今小人之行如夷狄,而王不能變化之,我用是為大憂也。髦,西夷別名,武王伐紂,其等有八國從焉。}

\section{菀柳}

%{\footnotesize 三章、章六句}

\textbf{菀柳,刺幽王也。暴虐無親而刑罰不中,諸侯皆不欲朝,言王者之不可朝事也。}

\textbf{有菀者柳,不尚息焉。}{\footnotesize 興也。菀,茂木也。箋云尚,庶幾也。有菀然枝葉茂盛之柳,行路之人豈有不庶幾欲就之止息乎,興者,喻王有盛德則天下皆庶幾願往朝焉,憂今不然。}\textbf{上帝甚蹈,無自暱焉。}{\footnotesize 蹈動、暱近也。箋云蹈,讀曰悼。上帝乎者,愬之也。今幽王暴虐,不可以朝事,甚使我心中悼病,是以不從而近之,釋己所以不朝之意。}\textbf{俾予靖之,後予極焉。}{\footnotesize 靖治、極至也。箋云靖謀、俾使、極誅也。假使我朝王,王留我,使我謀政事,王信讒,不察功考績,後反誅放我,是言王刑罰不中,不可朝事也。}

\begin{quoting}菀 \texttt{yù},淮南子「形苑而神壯」,高注「苑,枯病也」。\textbf{馬瑞辰}詩蓋以枯柳之不可止息,興王朝之不可依倚也。又云動者,言其喜怒變動無常。極,同殛,放逐。\end{quoting}

\textbf{有菀者柳,不尚愒焉。}{\footnotesize 愒,息也。}\textbf{上帝甚蹈,無自瘵焉。}{\footnotesize 瘵,病也。箋云瘵,接也。}\textbf{俾予靖之,後予邁焉。}{\footnotesize 箋云邁,行也,行亦放也,春秋傳曰「子將行之」。}

\begin{quoting}愒 \texttt{qì}。瘵 \texttt{zhài}。\end{quoting}

\textbf{有鳥高飛,亦傅于天。彼人之心,于何其臻。}{\footnotesize 箋云傅、臻皆至也。彼人,斥幽王也。鳥之高飛,極至於天耳,幽王之心,於何所至乎,言其轉側無常,人不知其所屆。}\textbf{曷予靖之,居以凶矜。}{\footnotesize 曷害、矜危也。箋云王何為使我謀之,隨而罪我,居我以凶危之地,謂四裔也。}

\section{都人士}

%{\footnotesize 五章、章六句}

\textbf{都人士,周人刺衣服無常也。古者長民衣服不貳,從容有常以齊其民則民德歸壹,傷今不復見古人也。}{\footnotesize 服,謂冠弁衣裳也。古者,明王時也。長民,謂凡在民上倡率者也。變易無常謂之貳。從容,謂休燕也,休燕猶有常,則朝夕明矣。壹者,專也、同也。}

\begin{quoting}\textbf{王先謙}此詩毛氏五章,三家皆止四章,孔疏云「左襄十四年傳引此詩『行歸于周,萬民所望』二句,服虔曰逸詩也」。熹平石經魯詩殘石亦無首章。\end{quoting}

\textbf{彼都人士,狐裘黃黃。其容不改,出言有章。}{\footnotesize 彼,彼明王也。箋云城郭之域曰都。古明王時,都人之有士行者,冬則衣狐裘,黃黃然取溫裕而已,其動作容貌既有常,吐口言語又有法度文章,疾今奢淫,不自責以過差。}\textbf{行歸于周,萬民所望。}{\footnotesize 周,忠信也。箋云于,於也。都人之士所行要歸於忠信,其餘萬民寡識者,咸瞻望而法俲之,又疾今不然。}

\begin{quoting}\textbf{馬瑞辰}逸周書大匡解「士惟都人,孝悌子孫」,是都人乃美士之稱,鄭風「洵美且都、不見子都」,都皆訓美,美色謂之都,都人猶言美人也。白虎通衣裳篇「諸侯狐黃」,禮玉藻「狐裘黃衣以裼之」,裼 \texttt{xī} 蓋罩衫披風之屬。行,將。周,鎬京。\end{quoting}

\textbf{彼都人士,臺笠緇撮。}{\footnotesize 臺所以禦暑,笠所以禦雨也。緇撮,緇布冠也。箋云臺,夫須也。都人之士以臺皮為笠,緇布為冠。古明王之時儉且節也。}\textbf{彼君子女,綢直如髮。}{\footnotesize 密直如髮也。箋云彼君子女者,謂都人之家女也,其情性密緻,操行正直,如髮之本末無隆殺也。}\textbf{我不見兮,我心不說。}{\footnotesize 箋云疾時皆奢淫,我不復見今士女之然者,心思之而憂也。}

\begin{quoting}臺,通薹,莎草。綢,同䯾,說文「䯾,髮多也」。如,其也。\end{quoting}

\textbf{彼都人士,充耳琇實。}{\footnotesize 琇,美石也。箋云言以美石為瑱,瑱,塞耳。}\textbf{彼君子女,謂之尹吉。}{\footnotesize 尹,正也。箋云吉,讀為姞,尹氏、姞氏,周室昏姻之舊姓也。人見都人之家女,咸謂之尹氏、姞氏之女,言有禮法。}\textbf{我不見兮,我心苑結。}{\footnotesize 箋云苑,猶屈也、積也。}

\begin{quoting}苑,音義同鬱。\end{quoting}

\textbf{彼都人士,垂帶而厲。彼君子女,卷髮如蠆。}{\footnotesize 厲,帶之垂者。箋云而,亦如也,而厲,如鞶厲也,鞶必垂厲以為飾,厲,字當作裂。蠆,螫蟲,尾末揵然,似婦人髮末曲上卷然者也。}\textbf{我不見兮,言從之邁。}{\footnotesize 箋云言,亦我也。邁,行也。我今不見士女此飾,心思之,欲從之行,言己憂悶,欲自殺求從古人。}

\begin{quoting}卷 \texttt{quán}。蠆 \texttt{chài},通俗文「長尾為蠆,短尾為蠍」。\end{quoting}

\textbf{匪伊垂之,帶則有餘。匪伊卷之,髮則有旟。}{\footnotesize 旟,揚也。箋云伊,辭也。此言士非故垂此帶也,帶於禮自當有餘也,女非故卷此髮也,髮於禮自當有旟也。旟,枝旟。揚,起也。}\textbf{我不見兮,云何盱矣。}{\footnotesize 箋云盱,病也。思之甚,云「何乎,我今已病也」。}

\begin{quoting}\textbf{朱熹}言其自然閒美,不假修飾也。\end{quoting}

\section{采綠}

%{\footnotesize 四章、章四句}

\textbf{采綠,刺怨曠也。幽王之時多怨曠者也。}{\footnotesize 怨曠者,君子行役過時之所由也,而刺之者,譏其不但憂思而已,欲從君子於外,非禮也。}

\textbf{終朝采綠,不盈一匊。}{\footnotesize 興也。自旦及食時為終朝。兩手曰匊。箋云綠,王芻也,易得之菜也,終朝采之而不滿手,怨曠之深,憂思不專於事。}\textbf{予髮曲局,薄言歸沐。}{\footnotesize 局,卷也。婦人夫不在則不容飾。箋云言,我也。禮,婦人在夫家筓象筓,今曲卷其髮,憂思之甚也。有云君子將歸者,我則沐以待之。}

\begin{quoting}綠,同菉。\end{quoting}

\textbf{終朝采藍,不盈一襜。}{\footnotesize 衣蔽前謂之襜。箋云藍,染草也。}\textbf{五日為期,六日不詹。}{\footnotesize 詹,至也。婦人五日一御。箋云婦人過於時乃怨曠。五日六日者,五月之日、六月之日也,期至五月而歸,今六月猶不至,是以憂思。}

\begin{quoting}襜 \texttt{chān}。\end{quoting}

\textbf{之子于狩,言韔其弓。之子于釣,言綸之繩。}{\footnotesize 箋云之子,是子也,謂其君子也。于,往也。綸,釣繳也。君子往狩與,我當從之,為之韔弓,其往釣與,我當從之,為之繩繳,今怨曠,自恨初行時不然。}

\begin{quoting}韔 \texttt{chàng},弓袋。\end{quoting}

\textbf{其釣維何,維魴及鱮。維魴及鱮,薄言觀者。}{\footnotesize 箋云觀,多也。此美其君子之有技藝也,釣必得魴鱮,魴鱮是云其多者耳,其眾雜魚乃眾多矣。}

\section{黍苗}

%{\footnotesize 五章、章四句}

\textbf{黍苗,刺幽王也。不能膏潤天下,卿士不能行召伯之職焉。}{\footnotesize 陳宣王之德、召伯之功,以刺幽王及其群臣廢此恩澤事業也。}

\begin{quoting}\textbf{劉玉汝}詩纘緒曰黍苗為營謝方畢而歸之詩,崧高為營謝既成,申伯出封之詩。\end{quoting}

\textbf{芃芃黍苗,陰雨膏之。}{\footnotesize 興也。芃芃,長大貌。箋云興者,喻天下之民如黍苗然,宣王能以恩澤育養之,亦如天之有陰雨之潤。}\textbf{悠悠南行,召伯勞之。}{\footnotesize 悠悠,行貌。箋云宣王之時,使召伯營謝邑,以定申伯之國,將徒役南行,眾多悠悠然,召伯則能勞來勸說以先之。}

\begin{quoting}\textbf{陳奐}謝在周南也。召伯,召穆公虎,\textbf{陳啟源}稽古編曰穆公諫厲王親兄弟,又脫宣王於難而以子代之,及王立,復為平淮夷,城謝邑。\end{quoting}

\textbf{我任我輦,我車我牛。我行既集,蓋云歸哉。}{\footnotesize 任者,輦者,車者,牛者。箋云集,猶成也。蓋,猶皆也。營謝轉餫之役,有負任者,有輓輦者,有將車者,有牽傍牛者,其所為南行之事既成,召伯則皆告之云「可歸哉」,刺今王使民行役,曾無休止時。}

\begin{quoting}蓋,通盍。\end{quoting}

\textbf{我徒我御,我師我旅。我行既集,蓋云歸處。}{\footnotesize 徒行者,御車者,師者,旅者。箋云步行曰徒。召伯營謝邑,以兵眾行,其士卒有步行者,有御兵車者。五百人為旅,五旅為師,春秋傳曰「諸侯之制,君行師從,卿行旅從」。}

\textbf{肅肅謝功,召伯營之。烈烈征師,召伯成之。}{\footnotesize 謝,邑也。箋云肅肅,嚴正之貌。營,治也。烈烈,威武貌。征,行也。美召伯治謝邑,則使之嚴正,將師旅行則有威武。}

\textbf{原隰既平,泉流既清。召伯有成,王心則寧。}{\footnotesize 土治曰平,水治曰清。箋云召伯營謝邑,相其原隰之宜,通其水泉之利,此功既成,宣王之心則安也。又刺今王臣無成功而亦心安。}

\section{隰桑}

%{\footnotesize 四章、章四句}

\textbf{隰桑,刺幽王也。小人在位,君子在野,思見君子,盡心以事之。}

\textbf{隰桑有阿,其葉有難。}{\footnotesize 興也。阿然美貌,難然盛貌,有以利人也。箋云隰中之桑,枝條阿阿然長美,其葉又茂盛可以庇廕人。興者,喻時賢人君子不用而野處,有覆養之德也,正以隰桑興者,反求此義,則原上之桑,枝葉不能然,以刺時小人在位,無德於民。}\textbf{既見君子,其樂如何。}{\footnotesize 箋云思在野之君子而得見其在位,喜樂無度。}

\begin{quoting}案低濕處宜桑。\textbf{王先謙}案有阿即阿阿也,故箋讀為阿阿,字亦變為猗猗,見淇奧傳,經中凡累字多參用「有」字,與累字無異。\textbf{陳奐}難之為言那也,其讀同那,桑扈、那傳「那,多也」,盛與多同義,阿難連緜字,萇楚曰猗儺,那曰猗那,聲義皆同也。\end{quoting}

\textbf{隰桑有阿,其葉有沃。}{\footnotesize 沃,柔也。}\textbf{既見君子,云何不樂。}

\textbf{隰桑有阿,其葉有幽。}{\footnotesize 幽,黑色也。}\textbf{既見君子,德音孔膠。}{\footnotesize 膠,固也。箋云君子在位,民附仰之,其教令之行甚堅固也。}

\begin{quoting}\textbf{馬瑞辰}孔膠,猶云甚盛耳。\end{quoting}

\textbf{心乎愛矣,遐不謂矣。中心藏之,何日忘之。}{\footnotesize 箋云遐遠、謂勤、藏善也。我心愛此君子,君子雖遠在野,豈能不勤思之乎,宜思之也,我心善此君子,又誠不能忘也,孔子曰「愛之能勿勞乎,忠焉能勿誨乎」。}

\begin{quoting}遐不,即何不、胡不,古遐與何、胡雙聲,故通用。謂,吿也。\end{quoting}

\section{白華}

%{\footnotesize 八章、章四句}

\textbf{白華,周人刺幽后也。幽王取申女以為后,又得褒姒而黜申后,故下國化之,以妾為妻、以孽代宗而王弗能治,周人為之作是詩也。}{\footnotesize 申,姜姓之國也。褒姒,褒人所入之女,姒其字也,是謂幽后。孽,支庶也。宗,適子也。王不能治,己不正故也。}

\textbf{白華菅兮,白茅束兮。}{\footnotesize 興也。白華,野菅也,已漚為菅。箋云白華於野,已漚名之為菅。菅柔忍中用矣,而更取白茅收束之,茅比於白華為脆,興者,喻王取於申,申后禮儀備,任妃后之事,而更納褒姒,褒姒為孽,將至滅國。}\textbf{之子之遠,俾我獨兮。}{\footnotesize 箋云之子,斥幽王也。俾,使也。王之遠外我,不復答耦我,意欲使我獨也。老而無子曰獨。後褒姒譖申后之子宜咎,宜咎奔申。}

\begin{quoting}\textbf{朱熹}蓋言白華與茅尚能相依,而我與子乃相去如此之遠。\end{quoting}

\textbf{英英白雲,露彼菅茅。}{\footnotesize 英英,白雲貌。露亦有雲,言天地之氣無微不著,無不覆養。箋云白雲下露,養彼可以為菅之茅,使與白華之菅相亂易,猶天下妖氣生褒姒,使申后見黜。}\textbf{天步艱難,之子不猶。}{\footnotesize 步行、猶可也。箋云猶,圖也。天行此艱難之妖久矣,王不圖其變之所由爾,昔夏之衰,有二龍之妖,卜藏其漦,周厲王發而觀之,化為玄黿,童女遇之,當宣王時而生女,懼而棄之,後褒人有獄而入之幽王,幽王嬖之,是謂褒姒。}

\begin{quoting}\textbf{馬瑞辰}六月云「白旆英英」,英英是白貌,則知此詩英英亦雲之白貌。孔疏引侯苞云「天行艱難於我身,不我可也」。\end{quoting}

\textbf{滮池北流,浸彼稻田。}{\footnotesize 滮,流貌。箋云池水之澤浸潤稻田,使之生殖,喻王無恩意於申后,滮池之不如也。豐鎬之間水北流。}\textbf{嘯歌傷懷,念彼碩人。}{\footnotesize 箋云碩,大也。妖大之人,謂褒姒也。申后見黜,褒姒之所為,故憂傷而念之。}

\begin{quoting}滮 \texttt{biāo} 池,水經註渭水「鎬水又北流,西北注與滮池水合,水出鄗池西而北流入於鎬」。\end{quoting}

\textbf{樵彼桑薪,卬烘于煁。}{\footnotesize 卬我、烘燎也。煁,烓竈也。桑薪,宜以養人者也。箋云人之樵取彼桑薪,宜以炊饔膳之爨以養食人,桑薪,薪之善者也,我反以燎於烓竈,用炤事物而已,喻王始以禮取申后,申后禮儀備,今反黜之,使為卑賤之事亦猶是。}\textbf{維彼碩人,實勞我心。}

\begin{quoting}煁 \texttt{chén}。\end{quoting}

\textbf{鼓鍾于宮,聲聞于外。}{\footnotesize 有諸宮中,必形見於外。箋云王失禮於內而下國聞知而化之,王弗能治,如鳴鼓鍾於宮中,而欲外人不聞,亦不可止。}\textbf{念子懆懆,視我邁邁。}{\footnotesize 邁邁,不說也。箋云此言申后之忠於王也,念之懆懆然,欲諫正之,王反不說於其所言。}

\begin{quoting}懆懆,釋文引說文云「愁不申也」,今本作「愁不安也」。邁邁,釋文「韓詩及說文並作㤄㤄,韓詩云意不說好也,許云很怒也」。\end{quoting}

\textbf{有鶖在梁,有鶴在林。}{\footnotesize 鶖,禿鶖也。箋云鶖也鶴也,皆以魚為美食者也,鶖之性貪惡而今在梁,鶴絜白而反在林,興王養褒姒而餒申后,近惡而遠善。}\textbf{維彼碩人,實勞我心。}

\textbf{鴛鴦在梁,戢其左翼。}{\footnotesize 箋云戢,斂也。斂左翼者,謂右掩左也。鳥之雌雄不可別者,以翼右掩左雄,左掩右雌,陰陽相下之義也,夫婦之道,亦以禮義相下,以成家道。}\textbf{之子無良,二三其德。}{\footnotesize 箋云良,善也。王無答耦己之善意,而變移其心志,令我怨曠。}

\begin{quoting}\textbf{馬瑞辰}詩蓋以鴛鴦匹鳥得其所止,能不貳其偶,以興幽王二三其德,為匹鳥之不若也。\end{quoting}

\textbf{有扁斯石,履之卑兮。}{\footnotesize 扁扁,乘石貌。王乘車履石。箋云王后出入之禮與王同,其行登車亦履石,申后始時亦然,今見黜而卑賤也。}\textbf{之子之遠,俾我疧兮。}{\footnotesize 疧,病也。箋云王之遠外我,欲使我困病。}

\begin{quoting}\textbf{胡承珙}卑字當屬石言,何氏古義云「履之卑兮是倒文,言乘石卑下,猶得蒙王踐履」。\end{quoting}

\section{綿蠻}

%{\footnotesize 三章、章八句}

\textbf{綿蠻,微臣刺亂也。大臣不用仁心,遺忘微賤,不肯飲食教載之,故作是詩也。}{\footnotesize 微臣,謂士也。古者卿大夫出行,士為末介,士之祿薄,或困乏於資財,則當賙贍之。幽王之時,國亂禮廢恩薄,大不念小,尊不恤賤,故本其亂而刺之。}

\begin{quoting}\textbf{王質}詩總聞曰重臣出行,而下士冗役吿勞者也,聞其吿勞而旋生憫心。\end{quoting}

\textbf{綿蠻黃鳥,止于丘阿。}{\footnotesize 興也。綿蠻,小鳥貌。丘阿,曲阿也。鳥止於阿,人止於仁。箋云止,謂飛行所止託也。興者,小鳥知止於丘之曲阿靜安之處而託息焉,喻小臣擇卿大夫有仁厚之德者而依屬焉。}\textbf{道之云遠,我勞如何。飲之食之,敎之誨之。命彼後車,謂之載之。}{\footnotesize 箋云在國依屬於卿大夫之仁者,至於為末介,從而行,道路遠矣,我罷勞則卿大夫之恩宜如何乎,渴則予之飲,飢則予之食,事未至則豫教之,臨事則誨之,車敗則命後車載之。後車,倅車也。}

\begin{quoting}綿蠻,文選李注引韓詩薛君章句「文貌」。謂,吿也。\end{quoting}

\textbf{綿蠻黃鳥,止于丘隅。}{\footnotesize 箋云丘隅,丘角也。}\textbf{豈敢憚行,畏不能趨。}{\footnotesize 箋云憚,難也。我罷勞,車又敗,豈敢難徒行乎,畏不能及爾疾至也。}\textbf{飲之食之,敎之誨之。命彼後車,謂之載之。}

\begin{quoting}趨,疾行。\end{quoting}

\textbf{綿蠻黃鳥,止于丘側。}{\footnotesize 箋云丘側,丘傍也。}\textbf{豈敢憚行,畏不能極。}{\footnotesize 箋云極,至也。}\textbf{飲之食之,敎之誨之。命彼後車,謂之載之。}

\section{瓠葉}

%{\footnotesize 四章、章四句}

\textbf{瓠葉,大夫刺幽王也。上棄禮而不能行,雖有牲牢饔餼不肯用也,故思古之人不以微薄廢禮焉。}{\footnotesize 牛羊豕為牲,繫養者曰牢,熟曰饔,腥曰餼,生曰牽。不肯用者,自養厚而薄於賓客。}

\begin{quoting}\textbf{王質}詩總聞曰當為在野君子相見為禮。\end{quoting}

\textbf{幡幡瓠葉,采之亨之。君子有酒,酌言嘗之。}{\footnotesize 幡幡,瓠葉貌。庶人之菜也。箋云亨,熟也,熟瓠葉者,以為飲酒之菹也。此君子謂庶人之有賢行者也,其農功畢,乃為酒漿,以合朋友習禮講道藝也,酒食成,先與父兄室人亨瓠葉而飲之,所以急和親親也。飲酒而曰嘗者,以其為之主於賓客,賓客則加之以羞,易兌象曰「君子以朋友講習」。}

\begin{quoting}\textbf{王先謙}主人未獻于賓,先自嘗之也。\end{quoting}

\textbf{有兔斯首,炮之燔之。君子有酒,酌言獻之。}{\footnotesize 毛曰炮,加火曰燔。獻,奏也。箋云斯,白也,今俗語斯白之字作鮮,齊魯之間聲近斯,有兔白首者,兔之小者也。炮之燔之者,將以為飲酒之羞也。飲酒之禮,既奏酒於賓,乃薦羞,每酌言言者,禮不下庶人,庶人依士禮立賓主為酌名。}

\begin{quoting}\textbf{朱熹}有兔斯首,一兔也,猶數魚以尾也。獻,主人敬賓也,合下二章酢、醻為一獻之禮。\end{quoting}

\textbf{有兔斯首,燔之炙之。君子有酒,酌言酢之。}{\footnotesize 炕火曰炙。酢,報也。箋云報者,賓既卒爵,洗而酌主人也。凡治兔之宜,鮮者毛炮之,柔者炙之,乾者燔之。}

\textbf{有兔斯首,燔之炮之。君子有酒,酌言醻之。}{\footnotesize 醻,道飲也。箋云主人既卒酢爵,又酌自飲,卒爵,復酌進賓,猶今俗人勸酒。}

\section{漸漸之石}

%{\footnotesize 三章、章六句}

\textbf{漸漸之石,下國刺幽王也。戎狄叛之,荊舒不至,乃命將率東征,役久病於外,故作是詩也。}{\footnotesize 荊,謂楚也。舒,舒鳩、舒鄝、舒庸之屬。役,謂士卒也。}

\textbf{漸漸之石,維其高矣。山川悠遠,維其勞矣。}{\footnotesize 漸漸,山石高峻。箋云山石漸漸然高峻,不可登而上,喻戎狄眾彊而無禮義,不可得而伐也。山川者,荊舒之國所處也,其道里長遠,邦域又勞勞廣闊,言不可卒服。}\textbf{武人東征,不皇朝矣。}{\footnotesize 箋云武人,謂將帥也。皇,正也。將率受王命東行而征伐,役人罷病,必不能正荊舒,使之朝於王。}

\begin{quoting}漸漸 \texttt{zhǎn},同嶄。勞,同遼。\textbf{馬瑞辰}古者戰多以朝,詩言不遑朝者,甚言其東征急迫,言不暇至朝也。\end{quoting}

\textbf{漸漸之石,維其卒矣。山川悠遠,曷其沒矣。}{\footnotesize 卒竟、沒盡也。箋云卒者,崔嵬也,謂山巔之末也。曷,何也。廣闊之處,何時其可盡服。}\textbf{武人東征,不皇出矣。}{\footnotesize 箋云不能正之,令出使聘問於王。}

\begin{quoting}卒,同崒,說文「崒,危高也」。\end{quoting}

\textbf{有豕白蹢,烝涉波矣。}{\footnotesize 豕,豬也。蹢,蹄也。將久雨,則豕進涉水波。箋云烝,眾也。豕之性能水,又唐突難禁制,四蹄皆白曰駭,則白蹄其尤躁疾者,今離其繒牧之處,與眾豕涉入水之波漣矣,喻荊舒之人勇悍捷敏,其君猶白蹄之豕也,乃率民去禮義之安,而居亂亡之危,賤之,故比方於豕。}\textbf{月離于畢,俾滂沱矣。}{\footnotesize 畢,噣也。月離陰星則雨。箋云將有大雨,徵氣先見於天,以言荊舒之叛,萌漸亦由王出也。豕既涉波,今又雨使之滂沱,疾王甚也。}\textbf{武人東征,不皇他矣。}{\footnotesize 箋云不能正之,令其守職,不干王命。}

\begin{quoting}豬本好泥而今者白蹄,言水潦之盛也。離,麗也。\end{quoting}

\section{苕之華}

%{\footnotesize 三章、章四句}

\textbf{苕之華,大夫閔時也。幽王之時,西戎東夷交侵中國,師旅並起,因之以饑饉,君子閔周室之將亡,傷己逢之,故作是詩也。}{\footnotesize 師旅並起者,諸侯或出師或出旅,以助王距戎與夷也,大夫將師出,見戎夷之侵周而閔之,今當其難,自傷近危亡。}

\textbf{苕之華,芸其黃矣。}{\footnotesize 興也。苕,陵苕也,將落則黃。箋云陵苕之華,紫赤而繁。興者,陵苕之幹喻如京師也,其華猶諸夏也,故或謂諸夏為諸華,華衰則黃,猶諸侯之師旅罷病將敗,則京師孤弱。}\textbf{心之憂矣,維其傷矣。}{\footnotesize 箋云傷者,謂國日見侵削。}

\begin{quoting}苕,凌霄花也,五六月中花盛黃色。\end{quoting}

\textbf{苕之華,其葉靑靑。}{\footnotesize 華落,葉青青然。箋云京師以諸夏為障蔽,今陵苕之華衰而葉見青青然,喻諸侯微弱而王之臣當出見也。}\textbf{知我如此,不如無生。}{\footnotesize 箋云我,我王也。知王之為政如此,則己之生不如不生也,自傷逢今世之難,憂閔之甚。}

\begin{quoting}靑靑,同菁菁。\end{quoting}

\textbf{牂羊墳首,三星在罶。}{\footnotesize 牂羊,牝羊也。墳,大也。罶,曲梁也,寡婦之笱也。牂羊墳首,言無是道也,三星在罶,言不可久也。箋云無是道者,喻周已衰,求其復興不可得也,不可久者,喻周將亡,如心星之光耀,見於魚笱之中,其去須臾也。}\textbf{人可以食,鮮可以飽。}{\footnotesize 治日少而亂日多。箋云今者士卒人人於晏早皆可以食矣,時饑饉,軍興乏少,無可以飽之者。}

\begin{quoting}\textbf{朱熹}羊瘠則首大也。又曰罶中無魚而水靜,但見三星之光而已,言饑饉之餘,百物彫耗如此。\end{quoting}

\section{何草不黃}

%{\footnotesize 四章、章四句}

\textbf{何草不黃,下國刺幽王也。四夷交侵,中國背叛,用兵不息,視民如禽獸,君子憂之,故作是詩也。}

\textbf{何草不黃,何日不行。}{\footnotesize 箋云用兵不息,軍旅自歲始草生而出,至歲晚矣,何草而不黃乎,言草皆黃也,於是之間,將率何日不行乎,言常行,勞苦之甚。}\textbf{何人不將,經營四方。}{\footnotesize 言萬民無不從役。}

\begin{quoting}\textbf{馬瑞辰}周頌敬之篇「日就月將」,毛傳「將,行也」,此詩何人不將與何日不行同義,何日不行言日日行也,何人不將言人人行也,集傳「將,亦行也」是也。經營,往來也。\end{quoting}

\textbf{何草不玄,何人不矜。}{\footnotesize 箋云玄,赤黑色。始春之時,草牙孽者將生,必玄於此時也,兵猶復行。無妻曰矜,從役者皆過時不得歸,故謂之矜。}\textbf{哀我征夫,獨為匪民。}{\footnotesize 箋云征夫,從役者也。古者師出不踰時,所以厚民之性也,今則草玄至於黃,黃至於玄,此豈非民乎。}

\begin{quoting}\textbf{馬瑞辰}爾雅釋詁「玄黃,病也」,馬病謂之玄黃,草病亦謂之玄黃,其義一也。矜 \texttt{guān},通鰥。\end{quoting}

\textbf{匪兕匪虎,率彼曠野。}{\footnotesize 兕虎,野獸也。曠,空也。箋云兕虎,比戰士也。}\textbf{哀我征夫,朝夕不暇。}

\begin{quoting}匪,通彼。\end{quoting}

\textbf{有芃者狐,率彼幽草。有棧之車,行彼周道。}{\footnotesize 芃,小獸貌。棧車,役車也。箋云狐草行草止,故以比棧車輦者。}

\begin{quoting}\textbf{馬瑞辰}有棧之車與有芃者狐皆形容之詞,據說文「嶘,尤高也,从山,棧聲」,則棧當為車高之貌。\end{quoting}

%\begin{flushright}魚藻之什十四篇、六十二章、三百二句\end{flushright}