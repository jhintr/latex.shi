\chapter{閔予小子之什詁訓傳第二十八}

\section{閔予小子}

%{\footnotesize 一章、十一句}

\textbf{閔予小子,嗣王朝於廟也。}{\footnotesize 嗣王者,謂成王也。除武王之喪,將始即政,朝於廟也。}

\textbf{閔予小子,遭家不造,嬛嬛在疚。}{\footnotesize 閔病、造為、疚病也。箋云閔,悼傷之言也。造,猶成也。可悼傷乎我小子耳,遭武王崩,家道未成,嬛嬛然孤特在憂病之中。}\textbf{於乎皇考,永世克孝。念茲皇祖,陟降庭止。}{\footnotesize 庭,直也。箋云茲,此也。陟降,上下也。於乎我君考武王,長世能孝,謂能以孝行為子孫法度,使長見行也,念此君祖文王,上以直道事天,下以直道治民,言無私枉。}\textbf{維予小子,夙夜敬止。於乎皇王,繼序思不忘。}{\footnotesize 序,緒也。箋云夙早、敬慎也。我小子早夜慎行祖考之道,言不敢解倦也,於乎君王,歎文王武王也,我繼其緒,思其所行不忘也。}

\begin{quoting}\textbf{馬瑞辰}不造,猶不善,不善,猶不淑也,不淑,猶云不祥,謂遭凶喪也。嬛嬛 \texttt{qióng},說文引詩作煢煢。\textbf{馬瑞辰}庭,直也,蓋謂文王陟降群臣皆以直道。\textbf{陳奐}繼緒,猶纘緒,閟宮「纘禹之緒」,傳「緒,業也」。\end{quoting}

\section{訪落}

%{\footnotesize 一章、十二句}

\textbf{訪落,嗣王謀於廟也。}{\footnotesize 謀者,謀政事也。}

\textbf{訪予落止,率時昭考。於乎悠哉,朕未有艾。將予就之,繼猶判渙。}{\footnotesize 訪謀、落始、率循、時是、悠遠、猶道、判分、渙散也。箋云昭明、艾數、猶圖也。成王始即政,自以承聖父之業,懼不能遵其道德,故於廟中與群臣謀我始即政之事,群臣曰當循是明德之考所施行,故答之以謙曰,於乎遠哉,我於是未有數,言遠不可及也,女扶將我,就其典法而行之,繼續其業,圖我所失,分散者收斂之。}\textbf{維予小子,未堪家多難。}{\footnotesize 箋云多,眾也。我小子耳,未任統理國家眾難成之事,心有任賢待年長大之志。難成之事,謂諸政有業未平者。}\textbf{紹庭上下,陟降厥家。休矣皇考,以保明其身。}{\footnotesize 箋云紹,繼也。厥家,謂群臣也。繼文王陟降庭止之道,上下群臣之職以次序者,美矣我君考武王,能以此道尊安其身,謂定天下,居天子之位。}

\begin{quoting}爾雅釋詁「艾,歷也」。\textbf{馬瑞辰}判渙疊韻,字當讀與卷阿詩「伴奐爾游矣」同,伴、奐皆大也,說文「伴,大貌」,奐字注「一曰大也」,繼猶判渙,言當謀其大者。孔疏「上言昭考,此言皇考,皆指武王也」。\textbf{馬瑞辰}此詩保明宜訓保勉。\end{quoting}

\section{敬之}

%{\footnotesize 一章、十二句}

\textbf{敬之,群臣進戒嗣王也。}

\begin{quoting}\textbf{林義光}按詩言「維予小子」,又言「示我顯德行」,則是嗣王吿群臣,非群臣戒嗣王也。\end{quoting}

\textbf{敬之敬之,天維顯思,命不易哉。無曰高高在上,陟降厥士,日監在茲。}{\footnotesize 顯見、士事也。箋云顯光、監視也。群臣見王謀即政之事,故因時戒之曰,敬之哉敬之哉,天乃光明,去惡與善,其命吉凶,不變易也,無謂天高又高在上,遠人而不畏也,天上下其事,謂轉運日月,施其所行,日日瞻視,近在此也。}\textbf{維予小子,不聦敬止。日就月將,學有緝熙于光明。佛時仔肩,示我顯德行。}{\footnotesize 小子,嗣王也。將,行也。光,廣也。佛,大也。仔肩,克也。箋云緝熙,光明也。佛,輔也。時,是也。仔肩,任也。群臣戒成王以敬之敬之,故承之以謙云,我小子耳,不聦達於敬之之意,日就月行,言當習之以積漸也,且欲學於有光明之光明者,謂賢中之賢也,輔佛是任,示道我以顯明之德行。是時自知未能成文武之功,周公始有居攝之志。}

\begin{quoting}\textbf{馬瑞辰}按廣雅「聦,聽也」,不為語詞,不聦敬止,謂聽而警戒也,承上「敬之敬之」而言。淮南子修務篇引詩,高注云「言為善者日有所成就,月有所奉行」。佛 \texttt{bì},同弼。\end{quoting}

\section{小毖}

%{\footnotesize 一章、八句}

\textbf{小毖,嗣王求助也。}{\footnotesize 毖,慎也。天下之事當慎其小,小時而不慎,後為禍大,故成王求忠臣早輔助己為政,以救患難。}

\textbf{予其懲而,毖後患。莫予荓蜂,自求辛螫。}{\footnotesize 毖,慎也。荓蜂,掣曳也。箋云懲,艾也。始者,管叔及其群弟流言於國,成王信之,而疑周公,至後三監叛而作亂,周公以王命舉兵誅之,歷年乃已,故今周公歸政,成王受之,而求賢臣以自輔助也。曰我其創艾於往時矣,畏慎後復有禍難,群臣小人無敢我掣曳,謂為譎詐誑欺不可信也,女如是,徒自求辛苦毒螫之害耳,謂將有刑誅。}\textbf{肇允彼桃蟲,拚飛維鳥。}{\footnotesize 桃蟲,鷦也,鳥之始小終大者。箋云肇始、允信也。始者信以彼管蔡之屬,雖有流言之罪如鷦鳥之小,不登誅之,後反叛而作亂,猶鷦之翻飛為大鳥也。鷦之所為鳥,題肩也,或曰鴞,皆惡聲之鳥。}\textbf{未堪家多難,予又集于蓼。}{\footnotesize 堪任、予我也。我又集于蓼,言辛苦也。箋云集,會也。未任統理我國家眾難成之事,謂使周公居攝時也,我又會於辛苦,遇三監及淮夷之難也。}

\begin{quoting}唐石經作「予其懲而,毖彼後患」。荓 \texttt{píng} 蜂,雙聲,意為牽引扶助。\textbf{陳奐}辛螫 \texttt{shì},釋文引韓詩作「辛赦」,云「赦,事也」,辛事,謂辛苦之事。又曰拚 \texttt{fān},疑當作翻。\end{quoting}

\section{載芟}

%{\footnotesize 一章、三十一句}

\textbf{載芟,春籍田而祈社稷也。}{\footnotesize 籍田,甸師氏所掌,王載耒耜所耕之田,天子千畝,諸侯百畝。籍之言借也,借民力治之,故謂之籍田。}

\textbf{載芟載柞,其耕澤澤。千耦其耘,徂隰徂畛。侯主侯伯,侯亞侯旅,侯彊侯以。}{\footnotesize 除草曰芟,除木曰柞。畛,埸也。主,家長也。伯,長子也。亞,仲叔也。旅,子弟也。彊,彊力也。以,用也。箋云載,始也。隰,謂新發田也。畛,謂舊田有徑路者。彊,有餘力者,周禮曰「以彊予任民」。以,謂閒民,今時傭賃也,春秋之義,能東西之曰以。成王之時,萬民樂治田業,將耕,先始芟柞其草木,土氣烝達而和,耕之則澤澤然解散,於是耘除其根株,輩作者千耦,言趨時也,或往之隰,或往之畛,父子餘夫俱行,彊有餘力者相助,又取傭賃,務疾畢已當種也。}\textbf{有嗿其饁,思媚其婦,有依其士。}{\footnotesize 嗿,眾貌。士,子弟也。箋云饁,饋饟也。依之言愛也。婦子來饋饟其農人於田野,乃逆而媚愛之,言勸其事勞,不自苦。}\textbf{有略其耜,俶載南畝。播厥百穀,實函斯活。}{\footnotesize 略,利也。箋云俶載,當作熾菑。播,猶種也。實,種子也。函,含也。活,生也。農夫既耘除草木根株,乃更以利耜熾菑之,而後種其種,皆成好含生氣。}\textbf{驛驛其達,有厭其傑。厭厭其苗,緜緜其麃。}{\footnotesize 達,射也。有厭其傑,言傑苗厭然特美也。麃,耘也。箋云達,出地也。傑,先長者。厭厭其苗,眾齊等也。}\textbf{載穫濟濟,有實其積,萬億及秭。}{\footnotesize 濟濟,難也。箋云難者,穗眾難進也。有實,實成也。其積之乃萬億及秭,言得多也。}\textbf{為酒為醴,烝畀祖妣,以洽百禮。}{\footnotesize 箋云烝進、畀予、洽合也。進予祖妣,謂祭先祖先妣也。以洽百禮,謂饗燕之屬。}\textbf{有飶其香,邦家之光。}{\footnotesize 飶,芬香也。箋云芬香之酒醴饗燕賓客,則多得其歡心,於國家有榮譽。}\textbf{有椒其馨,胡考之寧。}{\footnotesize 椒,猶飶也。胡,壽也。考,成也。箋云寧,安也。以芬香之酒醴祭於祖妣,則多得其福右。}\textbf{匪且有且,匪今斯今,振古如茲。}{\footnotesize 且,此也。振,自也。箋云匪,非也。振,亦古也。饗燕祭祀,心非云且而有且,謂將有嘉慶禎祥先來見也,心非云今而有此今,謂嘉慶之事不聞而至也,言脩德行禮莫不獲報,乃古古而如此,所由來者久,非適今時。}

\begin{quoting}柞 \texttt{zé},通斮。嗿 \texttt{tǎn}。思媚,見思齊注,\textbf{馬瑞辰}思媚其婦,亦形容美盛之詞。\textbf{王引之}依之言殷也,馬融易注「殷,盛也」,有殷,為壯盛之貌。孔疏「婦、士俱是行饟之人」。略,魯詩作㗉。俶 \texttt{chù},始也,意即起土。驛驛,魯詩作繹繹。麃,魯詩作䅺,即穗。\textbf{王引之}實,廣大貌,有實其積,謂露積之庾其形實實然廣大也。\end{quoting}

\section{良耜}

%{\footnotesize 一章、二十三句}

\textbf{良耜,秋報社稷也。}

\begin{quoting}周禮春官「祭祀有二時,謂春祈、秋報,報者,報其成熟之功」,故李迂仲云「祈之詩則詳耕種之事,報之詩則詳收成之事」。\end{quoting}

\textbf{畟畟良耜,俶載南畝。播厥百穀,實函斯活。}{\footnotesize 畟畟,猶測測也。箋云良,善也。農人測測以利善之耜,熾菑是南畝也,種此百穀,其種皆成好含生氣,言得其時。}\textbf{或來瞻女,載筐及筥,其饟伊黍。其笠伊糾,其鎛斯趙,以薅荼蓼。}{\footnotesize 笠,所以禦暑雨也。趙,刺也。蓼,水草也。箋云瞻,視也。有來視女,謂婦子來饁者也。筐筥,所以盛黍也,豐年之時,雖賤者猶食黍。饁者見戴糾然之笠,以田器刺地,薅去荼蓼之事,言閔其勤苦。}\textbf{荼蓼朽止,黍稷茂止。穫之挃挃,積之栗栗。其崇如墉,其比如櫛,以開百室。}{\footnotesize 挃挃,穫聲也。栗栗,眾多也。墉,城也。箋云百室,一族也。草穢既除而禾稼茂,禾稼茂而穀成孰,穀成孰而積聚多,如墉也,如櫛也,以言積之高大,且相比迫也,其已治之,則百家開戶納之。千耦其耘,輩作尚眾也,一族同時納穀,親親也。百室者,出必共洫間而耕,入必共族中而居,又有祭酺合醵之歡。}\textbf{百室盈止,婦子寧止。殺時犉牡,有捄其角。以似以續,續古之人。}{\footnotesize 黃牛黑脣曰犉。社稷之牛角尺。以似以續,嗣前歲、續往事也。箋云捄,角貌。五穀畢入,婦子則安,無行饁之事,於是殺牲報祭社稷,嗣前歲者,復求有豐年也,續往事者,復以養人也,續古之人,求有良司嗇也。}

\begin{quoting}畟畟 \texttt{cè},\textbf{胡承珙}凡入深者,必以漸而進,爾雅「深,測也」,說文「測,深所至也」,畟畟、測測皆狀農人深耕之貌。說文「周人謂餉曰饟 \texttt{xiǎng}」,段注「周頌曰其饟伊黍,正周人語也,釋詁曰饁、饟,饋也」。趙,三家詩作㨄,同㨖,說文「刺也」。薅 \texttt{hāo}。比,排列。\textbf{馬瑞辰}爾雅釋畜「黑脣犉 \texttt{chún}」,又「牛七尺為犉」,此詩及無羊詩「九十其犉」,皆當以「牛七尺為犉」釋之,犉謂牛之大者,犉牡,猶言廣牡,廣亦大也。\end{quoting}

\section{絲衣}

%{\footnotesize 一章、九句}

\textbf{絲衣,繹賓尸也。高子曰「靈星之尸也」。}{\footnotesize 繹,又祭也,天子諸侯曰繹,以祭之明日,卿大夫曰賓尸,與祭同日。周曰繹,商謂之肜。}

\textbf{絲衣其紑,載弁俅俅。自堂徂基,自羊徂牛,鼐鼎及鼒。}{\footnotesize 絲衣,祭服也。紑,絜鮮貌。俅俅,恭順貌。基,門塾之基。自羊徂牛,言先小後大也。大鼎謂之鼐,小鼎謂之鼒。箋云載,猶戴也。弁,爵弁也,爵弁而祭於王,士服也。繹禮輕,使士升門堂,視壺濯及籩豆之屬,降往於基,告濯具,又視牲從羊之牛,反告充已,乃舉鼎冪告絜,禮之次也。鼎圜弇上謂之鼒。}\textbf{兕觥其觩,旨酒思柔。不吳不敖,胡考之休。}{\footnotesize 吳,譁也。考,成也。箋云柔,安也。繹之旅士用兕觥,變於祭也,飲美酒者皆思自安,不讙譁,不敖慢也,此得壽考之休徵。}

\begin{quoting}紑 \texttt{fǒu}。載,魯詩、韓詩作戴。俅,說文「冠飾貌」。基,同畿,\textbf{馬瑞辰}畿之言期限也,期、朞、基古同音,故畿可借作基。鼒 \texttt{zī}。\end{quoting}

\section{酌}

%{\footnotesize 一章、九句}

\textbf{酌,告成大武也。言能酌先祖之道以養天下也。}{\footnotesize 周公居攝六年,制禮作樂,歸政成王,乃後祭於廟而奏之,其始成告之而已。}

\begin{quoting}此又一大武也。\end{quoting}

\textbf{於鑠王師,遵養時晦。時純熙矣,是用大介。}{\footnotesize 鑠美、遵率、養取、晦昧也。箋云純大、熙興、介助也。於美乎文王之用師,率殷之叛國以事紂,養是闇昧之君以老其惡,是周道大興而天下歸往矣,故有致死之士助之。}\textbf{我龍受之,蹻蹻王之造,載用有嗣。}{\footnotesize 龍,和也。蹻蹻,武貌。造,為也。箋云龍,寵也。來助我者,我寵而受用之,蹻蹻之士皆爭來造王,王則用之有嗣,傳相致。}\textbf{實維爾公,允師。}{\footnotesize 公,事也。箋云允,信也。王之事所以舉兵克勝者,實維女之事信,得用師之道。}

\begin{quoting}\textbf{馬瑞辰}按純熙,謂大光明也,武王既攻取晦昧,于時遂大光明。爾雅釋詁「介,善也」。寵,榮也。蹻 \texttt{jué}。\textbf{王先謙}王之所用有相續不絕者,言周得人之盛也,爾既荷天寵,又得人和,信可為後世師法矣。\end{quoting}

\section{桓}

%{\footnotesize 一章、九句}

\textbf{桓,講武類禡也。桓,武志也。}{\footnotesize 類也禡也,皆師祭也。}

\begin{quoting}此大武其六也。孔疏「謂武王時欲伐殷,陳列六軍,講習武事,又為類祭於上帝,為禡祭於所在之地,治兵祭神,然後克紂」。\end{quoting}

\textbf{綏萬邦,婁豐年。}{\footnotesize 箋云綏,安也。婁,亟也。誅無道,安天下,則亟有豐熟之年,陰陽和也。}\textbf{天命匪解,桓桓武王,保有厥士,于以四方,克定厥家。}{\footnotesize 士,事也。箋云天命為善不解倦者,以為天子我桓桓有威武之武王,則能安有天下之事,此言其當天意也,於是用武事於四方,能定其家先王之業,遂有天下。}\textbf{於昭于天,皇以間之。}{\footnotesize 間,代也。箋云于,曰也。皇,君也。於明乎曰天也,紂為天下之君,但由為惡,天以武王代之。}

\begin{quoting}婁,屢也。左傳僖十九年「昔周饑,克殷而年豐」。\textbf{馬瑞辰}保土,猶言保邦也,作士者,蓋以形近而譌。\textbf{陳奐}言武王之德昭著於天,故天以武王代殷也。\end{quoting}

\section{賚}

%{\footnotesize 一章、六句}

\textbf{賚,大封於廟也。賚,予也,言所以錫予善人也。}{\footnotesize 大封,武王伐紂時,封諸臣有功者。}

\begin{quoting}此大武其三也。\end{quoting}

\textbf{文王既勤止,我應受之。敷時繹思,我徂維求定。}{\footnotesize 勤勞、應當、繹陳也。箋云敷,猶徧也。文王既勞心於政事,以有天下之業,我當而受之,敷是文王之勞心,能陳繹而行之,今我往以此求定,謂安天下也。}\textbf{時周之命,於繹思。}{\footnotesize 箋云勞心者,是周之所以受天命而王之所由也,於女諸臣受封者,陳繹而思行之,以文王之功業敕勸之。}

\begin{quoting}敷,左傳引詩作鋪。\textbf{姚際恆}敷,布也施也,布施是政,使之續而不絕,不敢倦而中止也。思,語詞。\textbf{馬瑞辰}按時與承一聲之轉,古亦通用,楚策「仰承甘露而用之」,新序承作時,是其證也,周受天命,而諸侯受封於廟者又將受命於周,時周之命,即承周之命也,般詩「時周之命」同義,此謂諸侯受命於廟,彼謂巡守而諸侯受命於方嶽也。\textbf{姚際恆}於繹思,又重申己與諸侯始終無倦勤之意。\end{quoting}

\section{般}

%{\footnotesize 一章、七句}

\textbf{般,巡守而祀四嶽河海也。般,樂也。}

\begin{quoting}此又一大武也。\end{quoting}

\textbf{於皇時周,陟其高山,嶞山喬嶽,允猶翕河。}{\footnotesize 高山,四嶽也。嶞山,山之嶞嶞小者也。翕,合也。箋云皇君、喬高、猶圖也。於乎美哉,君是周邦而巡守,其所至則登其高山而祭之,望秩於山川,小山及高嶽皆信案山川之圖而次序祭之。河言合者,河自大陸之北敷為九,祭者合為一。}\textbf{敷天之下,裒時之對,時周之命。}{\footnotesize 裒,聚也。箋云裒眾、對配也。徧天之下眾山川之神,皆如是配而祭之,是周之所以受天命而王也。}

\begin{quoting}\textbf{朱熹}高山,泛言山耳,嶞則其狹而長者,喬,高也,嶽則其高而大者。\textbf{馬瑞辰}允猶,即允若,允若,即允順也,河以順軌而合流。\end{quoting}

%\begin{flushright}閔予小子之什十一篇、十一章、百三十七句\end{flushright}