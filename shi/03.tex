\chapter{邶柏舟詁訓傳第三}

\begin{quoting}\textbf{釋文}鄭云「邶鄘衛者,殷紂畿內地名,屬古冀州,自紂城而北曰邶,南曰鄘,東曰衛,衛在汲郡朝歌縣,時康叔正封于衛,其末子孫稍并兼彼二國,混其地而名之,作者各有所傷,從其本國而異之,故有邶鄘衛之詩」,王肅同,從此訖豳七月十三國並變風也。\end{quoting}

\section{柏舟}

%{\footnotesize 五章、章六句}

\textbf{柏舟,言仁而不遇也。衛頃公之時,仁人不遇,小人在側。}{\footnotesize 不遇者,君不受己之志也,君近小人,則賢者見侵害。}

\textbf{汎彼柏舟,亦汎其流。}{\footnotesize 興也。汎,汎流貌。柏,木所以宜為舟也。亦汎汎其流,不以濟渡也。箋云舟,載渡物者,今不用而與眾物汎汎然,俱流水中。興者,喻仁人之不見用而與群小人並列,亦猶是也。}\textbf{耿耿不寐,如有隱憂。}{\footnotesize 耿耿,猶儆儆也。隱,痛也。箋云仁人既不遇,憂在見侵害。}\textbf{微我無酒,以敖以遊。}{\footnotesize 非我無酒,可以敖遊忘憂也。}

\begin{quoting}耿,魯詩作炯,說文「炯,光也」,古人每以火比憂,如小雅節南山「憂心如惔」,采薇「憂心烈烈」等。如、而古通用。隱,齊詩韓詩作殷,深也。\end{quoting}

\textbf{我心匪鑒,不可以茹。}{\footnotesize 鑒所以察形也。茹,度也。箋云鑒之察形,但知方圓白黑,不能度其真偽,我心非如是鑒,我於眾人之善惡外內心度知之。}\textbf{亦有兄弟,不可以據。}{\footnotesize 據,依也。箋云兄弟至親,當相據依,言亦有不相據依以為是者希耳,責之以兄弟之道,謂同姓之臣也。}\textbf{薄言往愬,逢彼之怒。}{\footnotesize 彼,彼兄弟。}

\begin{quoting}\textbf{嚴粲}詩緝曰鑒雖明而不擇妍醜,皆納其影,我心有知善惡,善則從之,惡則拒之,不能混雜而容納之也。愬,訴也。\end{quoting}

\textbf{我心匪石,不可轉也。我心匪席,不可卷也。}{\footnotesize 石雖堅,尚可轉,席雖平,尚可卷。箋云言己心志堅平,過於石席。}\textbf{威儀棣棣,不可選也。}{\footnotesize 君子望之儼然可畏,禮容俯仰各有威儀耳。棣棣,富而閑習也。物有其容,不可數也。箋云稱己威儀如此者,言己德備而不遇,所以慍也。}

\begin{quoting}選,三家詩作算。\end{quoting}

\textbf{憂心悄悄,慍于群小。}{\footnotesize 慍,怒也。悄悄,憂貌。箋云群小,眾小人在君側者。}\textbf{覯閔既多,受侮不少。}{\footnotesize 閔,病也。}\textbf{靜言思之,寤擗有摽。}{\footnotesize 靜,安也。擗,拊心也。摽,拊心貌。箋云言,我也。}

\begin{quoting}\textbf{馬瑞辰}釋文及正義本傳皆作怒,蓋怨字形近之譌。覯閔,三家詩作遘愍。\textbf{馬瑞辰}此詩靜字宜用本義,訓宷,言為語詞,靜言思之,猶云審思之也。\end{quoting}

\textbf{日居月諸,胡迭而微。}{\footnotesize 箋云日,君象也,月,臣象也。微,謂虧傷也。君道當常明如日而月有虧盈,今君失道而任小人,大臣專恣,則日如月然。}\textbf{心之憂矣,如匪澣衣。}{\footnotesize 如衣之不澣矣。箋云衣之不澣,則憒辱無照察。}\textbf{靜言思之,不能奮飛。}{\footnotesize 不能如鳥奮翼而飛去。箋云臣不遇於君,猶不忍去,厚之至也。}

\begin{quoting}日月,喻夫也。居、諸疊韻,並語尾助詞。\textbf{陳喬樅}廣雅「迭,代也」,毛詩迭微,當訓為更迭而食。\end{quoting}

\section{綠衣}

%{\footnotesize 四章、章四句}

\textbf{綠衣,衛莊姜傷己也。妾上僭,夫人失位而作是詩也。}{\footnotesize 綠,當為褖,故作褖,轉作綠,字之誤也。莊姜,莊公夫人,齊女,姓姜氏。妾上僭者,謂公子州吁之母,母嬖而州吁驕。}

\textbf{綠兮衣兮,綠衣黃裏。}{\footnotesize 興也。綠,間色,黃,正色。箋云褖兮衣兮者,言褖衣自有禮制也,諸侯夫人祭服之下,鞠衣為上,展衣次之,褖衣次之,次之者,眾妾亦以貴賤之等服之,鞠衣黃,展衣白,褖衣黑,皆以素紗為裏,今褖衣反以黃為裏,非其禮制也,故以喻妾上僭。}\textbf{心之憂矣,曷維其已。}{\footnotesize 憂雖欲自止,何時能止也。}

\textbf{綠兮衣兮,綠衣黃裳。}{\footnotesize 上曰衣,下曰裳。箋云婦人之服不殊衣裳,上下同色,今衣黑而裳黃,喻亂嫡妾之禮。}\textbf{心之憂矣,曷維其亡。}{\footnotesize 箋云亡之言忘也。}

\textbf{綠兮絲兮,女所治兮。}{\footnotesize 綠,末也,絲,本也。箋云女,女妾上僭者。先染絲,後制衣,皆女之所治為也,而女反亂之,亦喻亂嫡妾之禮,責以本末之行。禮,大夫以上衣織,故本於絲也。}\textbf{我思古人,俾無訧兮。}{\footnotesize 俾使、訧過也。箋云古人,謂制禮者。我思此人定尊卑,使人無過差之行,心善之也。}

\begin{quoting}釋文「訧,本或作尤,過也」,孟子梁惠王下「畜君何尤」,注「何尤者,無過也」。\end{quoting}

\textbf{絺兮綌兮,淒其以風。}{\footnotesize 淒,寒風也。箋云絺綌所以當暑,今以待寒,喻其失所也。}\textbf{我思古人,實獲我心。}{\footnotesize 古之君子實得我之心也。箋云古之聖人制禮者,使夫婦有道,妻妾貴賤各有次序。}

\section{燕燕}

%{\footnotesize 四章、章六句}

\textbf{燕燕,衛莊姜送歸妾也。}{\footnotesize 莊姜無子,陳女戴媯生子名完,莊姜以為己子,莊公薨,完立而州吁殺之,戴媯於是大歸,莊姜遠送之于野,作詩見己志。}

\begin{quoting}\textbf{王士禎}分甘餘話曰為萬古送別之祖。\end{quoting}

\textbf{燕燕于飛,差池其羽。}{\footnotesize 燕燕,鳦也。燕之于飛,必差池其羽。箋云差池其羽,謂張舒其尾翼,興戴媯將歸,顧視其衣服。}\textbf{之子于歸,遠送于野。}{\footnotesize 之子,去者也。歸,歸宗也。遠送,過禮。于,於也。郊外曰野。箋云婦人之禮,送迎不出門,今我送是子乃至于野者,舒己憤,盡己情。}\textbf{瞻望弗及,泣涕如雨。}{\footnotesize 瞻,視也。}

\begin{quoting}\textbf{馬瑞辰}差 \texttt{cī} 池二字疊韻,義與參差同,皆不齊之貌。\end{quoting}

\textbf{燕燕于飛,頡之頏之。}{\footnotesize 飛而上曰頡,飛而下曰頏。箋云頡頏,興戴媯將歸,出入前却。}\textbf{之子于歸,遠于將之。}{\footnotesize 將,行也。箋云將,亦送也。}\textbf{瞻望弗及,佇立以泣。}{\footnotesize 佇立,久立也。}

\begin{quoting}說文段注「當作『飛而下曰頡,飛而上曰頏』,傳寫互譌久矣」。\end{quoting}

\textbf{燕燕于飛,下上其音。}{\footnotesize 飛而上曰上音,飛而下曰下音。箋云下上其音,興戴媯將歸,言語感激,聲有小大也。}\textbf{之子于歸,遠送于南。}{\footnotesize 陳在衛南。}\textbf{瞻望弗及,實勞我心。}{\footnotesize 實,是也。}

\textbf{仲氏任只,其心塞淵。}{\footnotesize 仲,戴媯字也。任大、塞瘞、淵深也。箋云任者,以恩相親信也,周禮「六行,孝友睦姻任恤」。}\textbf{終溫且惠,淑慎其身。}{\footnotesize 惠,順也。箋云溫,謂顏色和也。淑,善也。}\textbf{先君之思,以勗寡人。}{\footnotesize 勗,勉也。箋云戴媯思先君莊公之故,故將歸猶勸勉寡人以禮義。寡人,莊姜自謂也。}

\begin{quoting}只,語詞。終,既也。\end{quoting}

\section{日月}

%{\footnotesize 四章、章六句}

\textbf{日月,衛莊姜傷己也。遭州吁之難,傷己不見答於先君,以至困窮之詩也。}

\textbf{日居月諸,照臨下土。}{\footnotesize 日乎月乎,照臨之也。箋云日月,喻國君與夫人也,當同德齊意以治國者,常道也。}\textbf{乃如之人兮,逝不古處。}{\footnotesize 逝逮、古故也。箋云之人,是人也,謂莊公也。其所以接及我者,不以故處,甚違其初時。}\textbf{胡能有定,寧不我顧。}{\footnotesize 胡何、定止也。箋云寧,猶曾也。君之行如是,何能有所定乎,曾不顧念我之言,是其所以不能定完也。}

\textbf{日居月諸,下土是冒。}{\footnotesize 冒,覆也。箋云覆,猶照臨也。}\textbf{乃如之人兮,逝不相好。}{\footnotesize 不及我以相好。箋云其所以接及我者,不以相好之恩情,甚於己薄也。}\textbf{胡能有定,寧不我報。}{\footnotesize 盡婦道而不得報。}

\begin{quoting}\textbf{陳奐}不報,即不答也。\end{quoting}

\textbf{日居月諸,出自東方。}{\footnotesize 日始月盛,皆出東方。箋云自,從也。言夫人當盛之時,與君同位。}\textbf{乃如之人兮,德音無良。}{\footnotesize 音聲、良善也。箋云無善恩意之聲語於我也。}\textbf{胡能有定,俾也可忘。}{\footnotesize 箋云俾,使也。君之行如此,何能有所定,使是無良可忘也。}

\textbf{日居月諸,東方自出。父兮母兮,畜我不卒。}{\footnotesize 箋云畜養、卒終也。父兮母兮者,言己尊之如父,又親之如母,乃反養遇我不終也。}\textbf{胡能有定,報我不述。}{\footnotesize 述,循也。箋云不循,不循禮也。}

\begin{quoting}畜,同慉,孟子「畜君者,好君也」。述,魯詩作遹,韓詩作術,孫炎「遹,古述字」。\end{quoting}

\section{終風}

%{\footnotesize 四章、章四句}

\textbf{終風,衛莊姜傷己也。遭州吁之暴,見侮慢而不能正也。}{\footnotesize 正,猶止也。}

\textbf{終風且暴,顧我則笑。}{\footnotesize 興也。終日風為終風。暴,疾也。笑,侮之也。箋云既竟日風矣而又暴疾,興者,喻州吁之為不善,如終風之無休止,而其間又有甚惡,其在莊姜之旁,視莊姜則反笑之,是無敬心之甚。}\textbf{謔浪笑敖,}{\footnotesize 言戲謔不敬。}\textbf{中心是悼。}{\footnotesize 箋云悼者,傷其如是,然而己不能得而止之。}

\begin{quoting}\textbf{王引之}則,猶而也,文二年左傳「周志有之,勇則害上,不登於明堂」,言勇而害上也。\textbf{王先謙}蓋謔非不可謔,而浪則狂,笑非不可笑,而敖則縱。說文「悼,懼也,陳楚謂懼曰悼」。\end{quoting}

\textbf{終風且霾,}{\footnotesize 霾,雨土也。}\textbf{惠然肯來。}{\footnotesize 言時有順心也。箋云肯,可也。有順心然後可以來至我旁,不欲見其戲謔。}\textbf{莫往莫來,悠悠我思。}{\footnotesize 人無子道以來事己,己亦不得以母道往加之。箋云我思其如是,心悠悠然。}

\textbf{終風且曀,不日有曀。}{\footnotesize 陰而風曰曀。箋云有,又也。既竟日風且復曀不見日矣,而又曀者,喻州吁闇亂甚也。}\textbf{寤言不寐,願言則嚏。}{\footnotesize 嚏,跲也。箋云言我、願思也。嚏,讀當為不敢嚏咳之嚏。我其憂悼而不能寐,女思我心如是,我則嚏也。今俗人嚏,云人道我,此古之遺語也。}

\begin{quoting}\textbf{馬瑞辰}據考槃詩「獨寐寤言」傳云「在澗獨寐,覺而獨言」,則此言「寤言不寐」亦當訓為覺而有言。\end{quoting}

\textbf{曀曀其陰,}{\footnotesize 如常陰曀曀然。}\textbf{虺虺其靁。}{\footnotesize 暴若震靁之聲虺虺然。}\textbf{寤言不寐,願言則懷。}{\footnotesize 懷,傷也。箋云懷,安也。女思我心如是,我則安也。}

\section{擊鼓}

%{\footnotesize 五章、章四句}

\textbf{擊鼓,怨州吁也。衛州吁用兵暴亂,使公孫文仲將而平陳與宋,國人怨其勇而無禮也。}{\footnotesize 將者,將兵以伐鄭也。平,成也。將伐鄭,先告陳與宋,以成其伐事。春秋傳曰「宋殤公之即位也,公子馮出奔鄭,鄭人欲納之,及衛州吁立,將修先君之怨於鄭而求寵於諸侯,以和其民,使告於宋曰『君若伐鄭以除君害,君為主,敝邑以賦與陳蔡從,則衛國之願也』,宋人許之,於是陳蔡方睦於衛,故宋公、陳侯、蔡人、衛人伐鄭」是也,伐鄭在魯隱四年。}

\textbf{擊鼓其鏜,踊躍用兵。}{\footnotesize 鏜然,擊鼓聲也,使眾皆踊躍用兵也。箋云此用兵謂治兵時。}\textbf{土國城漕,我獨南行。}{\footnotesize 漕,衛邑也。箋云此言眾民皆勞苦也,或役土功於國,或修理漕城,而我獨見使從軍,南行伐鄭,是尤勞苦之甚。}

\textbf{從孫子仲,平陳與宋。}{\footnotesize 孫子仲,謂公孫文仲也,平陳與宋。箋云子仲,字也。平陳於宋,謂使告宋曰「君為主,敝邑以賦與陳蔡從」。}\textbf{不我以歸,憂心有忡。}{\footnotesize 憂心忡忡然。箋云以,猶與也。與我南行,不與我歸期。兵,凶事,懼不得歸,豫憂之。}

\begin{quoting}左傳隱六年杜注「和而不盟曰平」。\end{quoting}

\textbf{爰居爰處,爰喪其馬。}{\footnotesize 有不還者,有亡其馬者。箋云爰,於也。不還,謂死也傷也病也。今於何居乎,於何處乎,於何喪其馬乎。}\textbf{于以求之,于林之下。}{\footnotesize 山木曰林。箋云于,於也。求不還者及亡其馬者,當於山林之下。軍行必依山林,求其故處,近得之。}

\textbf{死生契闊,與子成說。}{\footnotesize 契闊,勤苦也。說,數也。箋云從軍之士與其伍約,死也生也相與處勤苦之中,我與子成相說愛之恩,志在相存救也。}\textbf{執子之手,與子偕老。}{\footnotesize 偕,俱也。箋云執其手,與之約誓示信也。言俱老者,庶幾俱免於難。}

\begin{quoting}契合、闊離也,疊韻,為偏義複詞。\end{quoting}

\textbf{于嗟闊兮,不我活兮。}{\footnotesize 不與我生活也。箋云州吁「阻兵安忍,阻兵無眾,安忍無親,眾叛親離」,軍士棄其約,離散相遠,故吁嗟歎之「闊兮,女不與我相救活」,傷之。}\textbf{于嗟洵兮,不我信兮。}{\footnotesize 洵遠、信極也。箋云歎其棄約,不與我相親信,亦傷之。}

\begin{quoting}爾雅「闊,遠也」。\textbf{馬瑞辰}活,當讀為「曷其有佸」之佸,毛傳「佸,會也」,佸為會至之會,又為聚會之會,承上闊兮為言,故云不我會耳。洵,魯詩韓詩作夐 \texttt{xiòng},廣雅「夐,遠也」。\end{quoting}

\section{凱風}

%{\footnotesize 四章、章四句}

\textbf{凱風,美孝子也。衛之淫風流行,雖有七子之母猶不能安其室,故美七子能盡其孝道,以慰其母心而成其志爾。}{\footnotesize 不安其室,欲去嫁也。成其志者,成言孝子自責之意。}

\textbf{凱風自南,吹彼棘心。}{\footnotesize 興也。南風謂之凱風,樂夏之長養。棘,難長養者。箋云興者,以凱風喻寬仁之母,棘,猶七子也。}\textbf{棘心夭夭,母氏劬勞。}{\footnotesize 夭夭,盛貌。劬勞,病苦也。箋云夭夭,以喻七子少長。母養之病苦也。}

\textbf{凱風自南,吹彼棘薪。}{\footnotesize 棘薪,其成就者。}\textbf{母氏聖善,我無令人。}{\footnotesize 聖,叡也。箋云叡作聖。令,善也。母乃有叡知之善德,我七子無善人能報之者,故母不安我室,欲去嫁也。}

\textbf{爰有寒泉,在浚之下。}{\footnotesize 浚,衛邑也。在浚之下,言有益於浚。箋云爰,曰也。曰有寒泉者,在浚之下浸潤之,使浚之民逸樂,以興七子不能如也。}\textbf{有子七人,母氏勞苦。}

\begin{quoting}爰,發語詞。\end{quoting}

\textbf{睍睆黃鳥,載好其音。}{\footnotesize 睍睆,好貌。箋云睍睆,以興顏色說也。好其音者,興其辭令順也,以言七子不能如也。}\textbf{有子七人,莫慰母心。}{\footnotesize 慰,安也。}

\begin{quoting}睍睆 \texttt{xiàn huǎn},韓詩作簡簡。\textbf{陳奐}後二章以寒泉之益於浚、黃鳥之好其音,喻七子不能事悦其母,泉鳥之不如也。\end{quoting}

\section{雄雉}

%{\footnotesize 四章、章四句}

\textbf{雄雉,刺衛宣公也。淫亂不恤國事,軍旅數起,大夫久役,男女怨曠,國人患之而作是詩。}{\footnotesize 淫亂者,荒放於妻妾、烝於夷姜之等。國人久處軍役之事,故男多曠、女多怨也,男曠而苦其事,女怨而望其君子。}

\textbf{雄雉于飛,泄泄其羽。}{\footnotesize 興也。雄雉見雌雉飛而鼓其翼泄泄然。箋云興者,喻宣公整其衣服而起,奮訊其形貌,志在婦人而已,不恤國之政事。}\textbf{我之懷矣,自詒伊阻。}{\footnotesize 詒遺、伊維、阻難也。箋云懷,安也。伊,當作繄,繄,猶是也。君之行如是,我安其朝而不去,今從軍旅,久役不得歸,此自遺以是患難。}

\begin{quoting}\textbf{馬瑞辰}前二章睹物起興,以雄雉之在目前,羽可得見,音可得聞,以興君子久役,不見其人,不聞其聲也。玉篇「阻,憂也」。\end{quoting}

\textbf{雄雉于飛,下上其音。}{\footnotesize 箋云下上其音,興宣公小大其聲,怡悅婦人。}\textbf{展矣君子,實勞我心。}{\footnotesize 展,誠也。箋云誠矣君子,愬於君子也。君之行如是,實使我心勞矣,君若不然,則我無軍役之事。}

\begin{quoting}說文段注「展與慎音近假借」。實,同寔。\end{quoting}

\textbf{瞻彼日月,悠悠我思。}{\footnotesize 瞻,視也。箋云視日月之行,迭往迭來,今君子獨久行役而不來,使我心悠悠然思之,女怨之辭。}\textbf{道之云遠,曷云能來。}{\footnotesize 箋云曷,何也。何時能來望之也。}

\textbf{百爾君子,不知德行。}{\footnotesize 箋云爾,女也。女眾君子,我不知人之德行何如者可謂為德行,而君或有所留。女怨,故問此焉。}\textbf{不忮不求,何用不臧。}{\footnotesize 忮害、臧善也。箋云我君子之行不疾害,不求備於一人,其行何用為不善,而君獨遠使之在外,不得來歸,亦女怨之辭。}

\begin{quoting}論語子罕馬融注「忮 \texttt{zhì},害也,不疾害,不貪求,何用不為善也」。\textbf{馬瑞辰}末章則推其君子久役之故,皆由有所忮求,若知脩其德行,無所忮求,則可以全身遠害,復何用而不臧乎,此以責君子之仕於亂世也。\end{quoting}

\section{匏有苦葉}

%{\footnotesize 四章、章四句}

\textbf{匏有苦葉,刺衛宣公也。公與夫人並為淫亂。}{\footnotesize 夫人,謂夷姜。}

\textbf{匏有苦葉,濟有深涉。}{\footnotesize 興也。匏謂之瓠,瓠葉苦不可食也。濟,渡也。由膝以上為涉。箋云瓠葉苦而渡處深,謂八月之時,陰陽交會,始可以為昏禮納采問名。}\textbf{深則厲,淺則揭。}{\footnotesize 以衣涉水為厲,謂由帶以上也。揭,褰衣也。遭時制宜,如遇水深則厲,淺則揭矣,男女之際,安可以無禮義,將無以自濟也。箋云既以深淺記時,因以水深淺喻男女之才性賢與不肖及長幼也,各順其人之宜,為之求妃耦。}

\begin{quoting}國語韋注「佩匏可以渡水也」。苦,枯也,齊說「枯瓠不朽,種以濟舟,渡渝江海,無有溺憂」。濟,水名。揭 \texttt{qì}。\end{quoting}

\textbf{有瀰濟盈,有鷕雉鳴。}{\footnotesize 瀰,深水也。盈,滿也。深水,人之所難也。鷕,雌雉聲也。衛夫人有淫泆之志,授人以色,假人以辭,不顧禮義之難,至使宣公有淫昏之行。箋云有瀰濟盈,謂過於厲,喻犯禮深也。}\textbf{濟盈不濡軌,雉鳴求其牡。}{\footnotesize 濡,漬也。由輈以上為軌。違禮義,不由其道,猶雉鳴而求其牡矣。飛曰雌雄,走曰牝牡。箋云渡深水者必濡其軌,言不濡者,喻夫人犯禮而不自知,雉鳴反求其牡,喻夫人所求非所求。}

\begin{quoting}鷕 \texttt{yǎo}。\end{quoting}

\textbf{雝雝鳴鴈,旭日始旦。}{\footnotesize 雝雝,鴈聲和也,納采用鴈。旭日始出,謂大昕之時。箋云鴈者隨陽而處,似婦人從夫,故昏禮用焉。自納采至請期用昕,親迎用昏。}\textbf{士如歸妻,迨冰未泮。}{\footnotesize 迨及、泮散也。箋云歸妻,使之來歸於己,謂請期也。冰未散,正月中以前也,二月可以昏矣。}

\begin{quoting}\textbf{姚際恆}古人行嫁娶必於秋冬農隙之際,故云迨冰未泮。荀子大略「霜降迎女,冰泮殺止」。\end{quoting}

\textbf{招招舟子,人涉卬否。}{\footnotesize 招招,號召之貌。舟子,舟人,主濟渡者。卬,我也。箋云舟人之子號召當渡者,猶媒人之會男女無夫家者使為妃匹,人皆從之而渡,我獨否。}\textbf{人涉卬否,卬須我友。}{\footnotesize 人皆涉,我友未至,我獨待之而不涉,以言室家之道,非得所適,貞女不行,非得禮義,昏姻不成。}

\begin{quoting}\textbf{馬瑞辰}按卬者,姎之假借,說文「姎,婦人自稱我也」,爾雅郭注「卬,猶姎也」,卬姎聲近通用,亦為我之通稱。\end{quoting}

\section{谷風}

%{\footnotesize 六章、章八句}

\textbf{谷風,刺夫婦失道也。衛人化其上淫於新昬而棄其舊室,夫婦離絕,國俗傷敗焉。}{\footnotesize 新昏者,新所與為昏禮。}

\textbf{習習谷風,以陰以雨。}{\footnotesize 興也。習習,和舒貌。東風謂之谷風。陰陽和而谷風至,夫婦和則室家成,室家成而繼嗣生。}\textbf{黽勉同心,不宜有怒。}{\footnotesize 言黽勉者,思與君子同心也。箋云所以黽勉者,以為見譴怒者,非夫婦之宜。}\textbf{采葑采菲,無以下體。}{\footnotesize 葑,須也。菲,芴也。下體,根莖也。箋云此二菜者,蔓菁與葍之類也,皆上下可食,然而其根有美時、有惡時,采之者不可以根惡時并棄其葉,喻夫婦以禮義合,顏色相親,亦不可以顏色衰,棄其相與之禮。}\textbf{德音莫違,及爾同死。}{\footnotesize 箋云莫無、及與也。夫婦之言無相違者,則可與女長相與處至死,顏色斯須之有。}

\begin{quoting}\textbf{嚴粲}來自山谷之風,大風也,盛怒之風也,又習習然連續不絕,又陰又雨,無清明開霽之意。以,為也。黽勉,魯詩韓詩作密勿。有,同又。\end{quoting}

\textbf{行道遲遲,中心有違。}{\footnotesize 遲遲,舒行貌。違,離也。箋云違,徘徊也。行於道路之人至將離別尚舒行,其心徘徊然,喻君子於己不能如也。}\textbf{不遠伊邇,薄送我畿。}{\footnotesize 畿,門內也。箋云邇,近也。言君子與己決別,不能遠,維近尒,送我裁於門內,無恩之甚。}\textbf{誰謂荼苦,其甘如薺。}{\footnotesize 荼,苦菜也。箋云荼誠苦矣,而君子於己之苦毒又甚於荼,此方之荼其甘如薺。}\textbf{宴爾新昬,如兄如弟。}{\footnotesize 宴,安也。}

\begin{quoting}\textbf{王夫之}方言「薄,勉也,秦晉曰薄,南楚之外曰薄努」,郭璞注曰「相勸勉也」。\end{quoting}

\textbf{涇以渭濁,湜湜其沚。}{\footnotesize 涇渭相入而清濁異。箋云小渚曰沚。涇水以有渭故見渭濁。湜湜,持正貌。喻君子得新昏故謂己惡也,己之持正守初如沚然不動搖,此絕去所經見,因取以自喻焉。}\textbf{宴爾新昬,不我屑以。}{\footnotesize 屑,潔也。箋云以,用也。言君子不復潔用我當室家。}\textbf{毋逝我梁,毋發我笱。}{\footnotesize 逝,之也。梁,魚梁。笱,所以捕魚也。箋云毋者,喻禁新昏也,女毋之我家,取我為室家之道。}\textbf{我躬不閱,遑恤我後。}{\footnotesize 閱,容也。箋云躬身、遑暇、恤憂也。我身尚不能自容,何暇憂我後所生子孫也。}

\begin{quoting}說文「湜 \texttt{shí},水清見底也」。\textbf{馬瑞辰}說文「止,下基也」,湜湜即狀水止之貌,故以為清可見底。發,同撥,韓詩云「發,亂也」。\end{quoting}

\textbf{就其深矣,方之舟之。就其淺矣,泳之游之。}{\footnotesize 舟,船也。箋云方,泭也。潛行為泳。言深淺者,喻君子之家事無難易,吾皆為之。}\textbf{何有何亡,黽勉求之。}{\footnotesize 有謂富也,亡謂貧也。箋云君子何所有乎,何所亡乎,吾其黽勉勤力為求之,有求多,亡求有。}\textbf{凡民有喪,匍匐救之。}{\footnotesize 箋云匍匐,言盡力也。凡於民有凶禍之事,鄰里尚盡力往救之,況我於君子家之事難易乎,固當黽勉,以疏喻親也。}

\textbf{不我能慉,反以我為讎。}{\footnotesize 慉,養也。箋云慉,驕也。君子不能以恩驕樂我,反憎惡我。}\textbf{既阻我德,賈用不售。}{\footnotesize 阻,難也。箋云既難却我,隱蔽我之善,我修婦道而事之,覬其察己,猶見疏外,如賣物之不售。}\textbf{昔育恐育鞫,及爾顛覆。}{\footnotesize 育長、鞫窮也。箋云昔育,育,稚也。及,與也。昔幼稚之時,恐至長老窮匱,故與女顛覆盡力於眾事,難易無所辟。}\textbf{既生既育,比予于毒。}{\footnotesize 箋云生,謂財業也,育,謂長老也。于,於也。既有財業矣,又既長老矣,其視我如毒螫,言惡之已甚也。}

\begin{quoting}\textbf{王念孫}能字古讀若耐,聲與乃相近,故義亦同。\textbf{馬瑞辰}慉與讎對,當讀如畜好之畜,孟子「畜君者,好君也」,文子亦云「善即吾畜也,不善即吾讎也」,並以畜與讎對舉,與詩文同,說文引詩「能不我慉」,與芄蘭詩「能不我知,能不我甲」句法相同,能不我慉承上章言,猶云「乃不我畜」也。于,如也。\end{quoting}

\textbf{我有旨蓄,亦以御冬。}{\footnotesize 旨美、御禦也。箋云蓄聚美菜者,以禦冬月乏無時也。}\textbf{宴爾新昬,以我御窮。}{\footnotesize 箋云君子亦但以我禦窮苦之時,至於富貴則棄我如旨蓄。}\textbf{有洸有潰,既詒我肄。}{\footnotesize 洸洸,武也。潰潰,怒也。肄,勞也。箋云詒,遺也。君子洸洸然潰潰然,無溫潤之色而盡遺我以勞苦之事,欲窮困我。}\textbf{不念昔者,伊余來塈。}{\footnotesize 塈,息也。箋云君子忘舊,不念往昔年稚我始來之時安息我。}

\begin{quoting}肄,同勩 \texttt{yì},爾雅釋詁「勩,勞也」。\textbf{王引之}來,詞之是也,全詩來字多與是同義。\textbf{馬瑞辰}愛,正字作㤅,說文「㤅,惠也」,伊余來塈,猶言「維余是愛」也,仍承「昔者」言之。\end{quoting}

\section{式微}

%{\footnotesize 二章、章四句}

\textbf{式微,黎侯寓于衛,其臣勸以歸也。}{\footnotesize 寓,寄也。黎侯為狄人所逐,棄其國而寄於衛,衛處之以二邑,因安之,可以歸而不歸,故其臣勸之。}

\textbf{式微式微,胡不歸。}{\footnotesize 式,用也。箋云式微式微者,微乎微者也,君何不歸乎,禁君留止於此之辭。式,發聲也。}\textbf{微君之故,胡為乎中露。}{\footnotesize 微,無也。中露,衛邑也。箋云我若無君,何為處此乎,臣又極諫之辭。}

\begin{quoting}\textbf{陳奐}微,非也,言非君之故。露,魯詩作路。\end{quoting}

\textbf{式微式微,胡不歸。微君之躬,胡為乎泥中。}{\footnotesize 泥中,衛邑也。}

\begin{quoting}說文「躬,身也」。\end{quoting}

\section{旄丘}

%{\footnotesize 四章、章四句}

\textbf{旄丘,責衛伯也。狄人迫逐黎侯,黎侯寓于衛,衛不能脩方伯連率之職,黎之臣子以責於衛也。}{\footnotesize 衛康叔之封爵稱侯,今曰伯者,時為州伯也。周之制,使伯佐牧,春秋傳曰「五侯九伯」,侯為牧也。}

\textbf{旄丘之葛兮,何誕之節兮。}{\footnotesize 興也。前高後下曰旄丘。諸侯以國相連屬,憂患相及,如葛之蔓延相連及也。誕,闊也。箋云土氣緩則葛生闊節,興者,喻此時衛伯不恤其職,故其臣於君事亦疏廢也。}\textbf{叔兮伯兮,何多日也。}{\footnotesize 日月以逝而不我憂。箋云叔伯,字也,呼衛之諸臣,叔與伯與,女期迎我君而復之,可來而不來,女日數何其多也。先叔後伯,臣之命不以齒。}

\begin{quoting}旄,三家詩作堥。誕,同覃。\textbf{馬瑞辰}之,猶其也,何誕之節,猶云何誕其節也。\end{quoting}

\textbf{何其處也,必有與也。}{\footnotesize 言與仁義也。箋云我君何以處於此乎,必以衛有仁義之道故也,責衛今不行仁義。}\textbf{何其久也,必有以也。}{\footnotesize 必以有功德。箋云我君何以久留於此乎,必以衛有功德故也,又責衛今不務功德也。}

\textbf{狐裘蒙戎,匪車不東。}{\footnotesize 大夫狐蒼裘。蒙戎,以言亂也。不東,言不來東也。箋云刺衛諸臣形貌蒙戎然,但為昏亂之行,女非有戎車乎,何不來東迎我君而復之。黎國在衛西,今所寓在衛東。}\textbf{叔兮伯兮,靡所與同。}{\footnotesize 無救患恤同也。箋云衛之諸臣行如是,不與諸伯之臣同,言其非之特甚。}

\begin{quoting}蒙戎,亦作尨茸。\textbf{陳奐}匪,彼也。\end{quoting}

\textbf{瑣兮尾兮,流離之子。}{\footnotesize 瑣尾,少好之貌。流離,鳥也,少好長醜,始而愉樂,終以微弱。箋云衛之諸臣初有小善,終無成功,似流離也。}\textbf{叔兮伯兮,褎如充耳。}{\footnotesize 褎,盛服也。充耳,盛飾也。大夫褎然有尊盛之服而不能稱也。箋云充耳,塞耳也。言衛之諸臣顏色褎然,如見塞耳無聞知也,人之耳聾,恆多笑而已。}

\begin{quoting}爾雅釋訓「瑣瑣,小也」。尾,通微。\textbf{王先謙}按然、如同訓,褎如,猶褎然也。\end{quoting}

\section{簡兮}

%{\footnotesize 三章、章六句}

\textbf{簡兮,刺不用賢也。衛之賢者仕於伶官,皆可以承事王者也。}{\footnotesize 伶官,樂官也,伶氏世掌樂官而善焉,故後世多號樂官為伶官。}

\textbf{簡兮簡兮,方將萬舞。}{\footnotesize 簡,大也。方,四方也。將,行也。以干羽為萬舞,用之宗廟山川,故言於四方。箋云簡擇、將且也。擇兮擇兮者,為且祭祀當萬舞也,萬舞,干舞者也。}\textbf{日之方中,在前上處。}{\footnotesize 教國子弟以日中為期。箋云在前上處者,在前列上頭也。周禮「大胥掌學士之版,以待致諸子,春入學,舍采合舞」。}\textbf{碩人俁俁,公庭萬舞。}{\footnotesize 碩人,大德也。俁俁,容貌大也。萬舞非但在四方,親在宗廟公庭。}

\begin{quoting}\textbf{馬瑞辰}方將二字連文,方,猶云將也,將,且也。俁 \texttt{yǔ},韓詩作扈,云「美貌」,\textbf{馬瑞辰}俁、扈音近,美與大亦同義,故扈扈訓美,又訓大。\end{quoting}

\textbf{有力如虎,執轡如組。}{\footnotesize 組,織組也。武力比於虎,可以御亂,御眾有文章,言能治眾,動於近,成於遠也。箋云碩人有御亂御眾之德,可任為王臣。}\textbf{左手執籥,右手秉翟。}{\footnotesize 籥,六孔。翟,翟羽也。箋云碩人多才多藝,又能籥舞,言文武道備。}\textbf{赫如渥赭,公言錫爵。}{\footnotesize 赫,赤貌。渥,厚潰也。祭有畀煇胞翟閽寺者,惠下之道,見惠不過一散。箋云碩人容色赫然,如厚傅丹,君徒賜其一爵而已,不知其賢而進用之。散受五升。}

\begin{quoting}有力如虎句,武舞也,左手執籥句,文舞也。\end{quoting}

\textbf{山有榛,隰有苓。}{\footnotesize 榛,木名。下濕曰隰。苓,大苦。箋云榛也苓也,生各得其所,以言碩人處非其位。}\textbf{云誰之思,西方美人。}{\footnotesize 箋云我誰思乎,思周室之賢者,以其宜薦碩人,與在王位。}\textbf{彼美人兮,西方之人兮。}{\footnotesize 乃宜在王室。箋云彼美人,謂碩人也。}

\section{泉水}

%{\footnotesize 四章、章六句}

\textbf{泉水,衛女思歸也。嫁於諸侯,父母終,思歸寧而不得,故作是詩以自見也。}{\footnotesize 以自見者,見己志也,國君夫人,父母在則歸寧,沒則使大夫寧於兄弟,衛女之思歸,雖非禮,思之至也。}

\textbf{毖彼泉水,亦流于淇。}{\footnotesize 興也。泉水始出,毖然流也。淇,水名也。箋云泉水流而入淇,猶婦人出嫁於異國。}\textbf{有懷于衛,靡日不思。}{\footnotesize 箋云懷至、靡無也。以言我有所至念於衛,無一日不思也,所至念者,謂諸姬諸姑伯姊。}\textbf{孌彼諸姬,聊與之謀。}{\footnotesize 孌,好貌。諸姬,同姓之女。聊,願也。箋云聊,且略之辭。諸姬者,未嫁之女。我且欲略與之謀婦人之禮,觀其志意,親親之恩也。}

\begin{quoting}毖 \texttt{bì},同泌,說文「泌,俠流也」。泉水,即卒章之肥泉也。\end{quoting}

\textbf{出宿于泲,飲餞于禰。}{\footnotesize 泲,地名。祖而舍軷,飲酒於其側曰餞,重始有事於道也。禰,地名。箋云泲禰者,所嫁國適衛之道所經,故思宿餞。}\textbf{女子有行,遠父母兄弟。}{\footnotesize 箋云行,道也。婦人有出嫁之道,遠於親親,故禮緣人情,使得歸寧。}\textbf{問我諸姑,遂及伯姊。}{\footnotesize 父之姊妹稱姑,先生曰姊。箋云寧則又問姑及姊,親其類也,先姑后姊,尊姑也。}

\begin{quoting}泲 \texttt{jǐ},魯詩作濟。禰 \texttt{nǐ},韓詩作坭。左傳桓九年「凡諸侯之女行」,杜注「行,嫁也」。\end{quoting}

\textbf{出宿于干,飲餞于言。}{\footnotesize 干言,所適國郊也。箋云干言,猶泲禰,未聞遠近同異。}\textbf{載脂載舝,還車言邁。}{\footnotesize 脂舝其車,以還我行也。箋云言還車者,嫁時乘來,今思乘以歸。}\textbf{遄臻于衛,不瑕有害。}{\footnotesize 遄疾、臻至、瑕遠也。箋云瑕,猶過也。害,何也。我還車疾至於衛而反,於行無過差,有何不可而止我。}

\begin{quoting}舝,古轄字。還,音義同旋。\textbf{馬瑞辰}瑕、遐古通用,遐之言胡,胡、無一聲之轉,凡詩言不遐有害、不遐有愆,不遐猶云不無,疑之之詞也。\end{quoting}

\textbf{我思肥泉,茲之永歎。}{\footnotesize 所出同、所歸異為肥泉。箋云茲,此也。自衛而來所渡水,故思此而長歎。}\textbf{思須與漕,我心悠悠。}{\footnotesize 須漕,衛邑也。箋云自衛而來所經邑,故又思之。}\textbf{駕言出遊,以寫我憂。}{\footnotesize 寫,除也。箋云既不得歸寧,且欲乘車出遊,以除我憂。}

\begin{quoting}\textbf{馬瑞辰}按茲即滋也,茲之永歎,猶常棣詩「況也永歎」,況亦滋也。\textbf{王先謙}說文湏下云「古文沬,从頁」,是湏即沬也,桑中「沬之鄉矣」是也,此詩思須之須當為湏。\end{quoting}

\section{北門}

%{\footnotesize 三章、章七句}

\textbf{北門,刺仕不得志也。言衛之忠臣不得其志爾。}{\footnotesize 不得其志者,君不知己志而遇困苦。}

\textbf{出自北門,憂心殷殷。}{\footnotesize 興也。北門背明鄉陰。箋云自,從也。興者,喻己仕於闇君,猶行而出北門,心為之憂殷殷然。}\textbf{終窶且貧,莫知我艱。}{\footnotesize 窶者,無禮也,貧者,困於財。箋云艱,難也。君於己祿薄,終不足以為禮,又近困於財,無知己以此為難者,言君既然矣,諸臣亦如之。}\textbf{已焉哉,天實為之,謂之何哉。}{\footnotesize 箋云謂,勤也。詩人事君無二志,故自決歸之於天,我勤身以事君,何哉,忠之至。}

\begin{quoting}釋文「殷,本有作慇」,爾雅釋訓「慇慇,憂也」。釋文「窶,謂貧無以為禮」。\textbf{陳奐}已焉,猶云既然。\textbf{馬瑞辰}齊策曰「雖惡於後王,吾獨謂先王何乎」,高注「謂,猶奈也」。\end{quoting}

\textbf{王事適我,政事一埤益我。}{\footnotesize 適之、埤厚也。箋云國有王命役使之事,不以之彼,必來之我,有賦稅之事則減彼一而以益我,言君政偏,己兼其苦。}\textbf{我入自外,室人交徧讁我。}{\footnotesize 讁,責也。箋云我從外而入,在室之人更迭徧來責我,使己去也,言室人亦不知己志也。}\textbf{已焉哉,天實為之,謂之何哉。}

\begin{quoting}適,擿也,說文「擿,搔也,一曰投也」。埤 \texttt{pí},同俾,段注「經傳之俾皆訓使也,無異解」。\end{quoting}

\textbf{王事敦我,政事一埤遺我。}{\footnotesize 敦厚、遺加也。箋云敦,猶投擿。}\textbf{我入自外,室人交徧摧我。}{\footnotesize 摧,沮也。箋云摧者,刺譏之言。}\textbf{已焉哉,天實為之,謂之何哉。}

\begin{quoting}敦,迫也。\end{quoting}

\section{北風}

%{\footnotesize 三章、章六句}

\textbf{北風,刺虐也。衛國並為威虐,百姓不親,莫不相攜持而去焉。}

\textbf{北風其涼,雨雪其雱。}{\footnotesize 興也。北風,寒涼之風。雱,盛貌。箋云寒涼之風病害萬物,興者,喻君政教酷暴,使民散亂。}\textbf{惠而好我,攜手同行。}{\footnotesize 惠愛、行道也。箋云性仁愛而又好我者,與我相攜持同道而去,疾時政也。}\textbf{其虛其邪,既亟只且。}{\footnotesize 虛,虛也。亟,急也。箋云邪,讀如徐。言今在位之人其故威儀虛徐寬仁者,今皆以為急刻之行矣,所以當去,以此也。}

\begin{quoting}\textbf{馬瑞辰}終風詩「惠然肯來」,傳「惠,順也」,此詩惠而猶惠然也。虛邪,即舒徐。只且 \texttt{jū},語尾助詞。\end{quoting}

\textbf{北風其喈,雨雪其霏。}{\footnotesize 喈,疾貌。霏,甚貌。}\textbf{惠而好我,攜手同歸。}{\footnotesize 歸有德也。}\textbf{其虛其邪,既亟只且。}

\begin{quoting}\textbf{馬瑞辰}喈,當作湝,說文「湝,水寒也」,引詩「風雨湝湝」即鄭風「風雨淒淒」之異文,邶風傳「淒,寒風也」,蓋水寒曰湝,風寒亦為湝。\end{quoting}

\textbf{莫赤匪狐,莫黑匪烏。}{\footnotesize 狐赤烏黑,莫能別也。箋云赤則狐也,黑則烏也,猶今君臣相承,為惡如一。}\textbf{惠而好我,攜手同車。}{\footnotesize 攜手就車。}\textbf{其虛其邪,既亟只且。}

\begin{quoting}\textbf{朱熹}同車,則貴者亦去矣。\end{quoting}

\section{靜女}

%{\footnotesize 三章、章四句}

\textbf{靜女,刺時也。衛君無道,夫人無德。}{\footnotesize 以君及夫人無道德,故陳靜女遺我以彤管之法,德如是,可以易之為人君之配。}

\textbf{靜女其姝,俟我於城隅。}{\footnotesize 靜,貞靜也,女德貞靜而有法度,乃可說也。姝,美色也。俟,待也。城隅,以言高而不可踰。箋云女德貞靜,然後可畜,美色,然後可安,又能服從,待禮而動,自防如城隅,故可愛也。}\textbf{愛而不見,搔首踟躕。}{\footnotesize 言志往而行止。箋云志往謂踟躕,行止謂愛之而不往見。}

\begin{quoting}\textbf{馬瑞辰}鄭詩「莫不靜好」,大雅「籩豆靜嘉」,皆以靜為靖之假借,此詩靜女亦當讀靖,謂善女。爾雅釋言「薆,隱也」,說文「僾,仿佛也,詩曰『僾而不見』」,\textbf{陳喬樅}離騷「眾薆然而蔽之」,薆而猶薆然也。踟躕,韓詩作躊躇,猶躑躅也。\end{quoting}

\textbf{靜女其孌,貽我彤管。}{\footnotesize 既有靜德,又有美色,又能遺我以古人之法,可以配人君也。古者后夫人必有女史彤管之法,史不記過,其罪殺之,后妃群妾以禮御於君所,女史書其日月,授之以環以進退之,生子月辰則以金環退之,當御者以銀環進之,著于左手,既御,著于右手,事無大小,記以成法。箋云彤管,筆赤管也。}\textbf{彤管有煒,說懌女美。}{\footnotesize 煒,赤貌。彤管以赤心正人也。箋云說懌,當作說釋。赤管煒煒然,女史以之說釋妃妾之德,美之。}

\textbf{自牧歸荑,洵美且異。}{\footnotesize 牧,田官也。荑,茅之始生也。本之於荑,取其有始有終。箋云洵,信也。茅,潔白之物也。自牧田歸荑,其信美而異者,可以供祭祀,猶貞女在窈窕之處,媒氏達之,可以配人君。}\textbf{匪女之為美,美人之貽。}{\footnotesize 非為其徒說美色而已,美其人能遺我法則。箋云遺我者,遺我以賢妃也。}

\begin{quoting}爾雅釋地「郊外謂之牧」。歸,同饋。異,韓詩作瘱,悦也。\end{quoting}

\section{新臺}

%{\footnotesize 三章、章四句}

\textbf{新臺,刺衛宣公也。納伋之妻,作新臺于河上而要之,國人惡之而作是詩也。}{\footnotesize 伋,宣公之世子。}

\textbf{新臺有泚,河水瀰瀰。}{\footnotesize 泚,鮮明貌。瀰瀰,盛貌。水所以潔汙穢,反于河上而為淫昏之行。}\textbf{燕婉之求,籧篨不鮮。}{\footnotesize 燕安、婉順也。籧篨,不能俯者。箋云鮮,善也。伋之妻齊女來嫁於衛,其心本求燕婉之人,謂伋也,反得籧篨不善,謂宣公也。籧篨口柔,常觀人顏色而為之辭,故不能俯也。}

\begin{quoting}泚,同玼,說文段注「玼 \texttt{cǐ} 本新玉色,引伸為凡新色,如詩『玼兮玼兮』言衣之鮮盛,『新臺有泚』言臺之鮮明」。籧篨、戚施皆醜惡之物。\end{quoting}

\textbf{新臺有洒,河水浼浼。}{\footnotesize 洒,高峻也。浼浼,平地也。}\textbf{燕婉之求,籧篨不殄。}{\footnotesize 殄,絕也。箋云殄,當作腆,腆,善也。}

\begin{quoting}洒 \texttt{cuǐ}。浼 \texttt{měi}。\end{quoting}

\textbf{魚網之設,鴻則離之。}{\footnotesize 言所得非所求也。箋云設魚網者宜得魚,鴻乃鳥也,反離焉,猶齊女以禮來求世子而得宣公。}\textbf{燕婉之求,得此戚施。}{\footnotesize 戚施,不能仰者。箋云戚施面柔,下人以色,故不能仰也。}

\begin{quoting}\textbf{王先謙}易序卦傳「離者,麗也」,附著之義。戚施,蟾蜍,喻醜惡。\end{quoting}

\section{二子乘舟}

%{\footnotesize 二章、章四句}

\textbf{二子乘舟,思伋、壽也。衛宣公之二子爭相為死,國人傷而思之,作是詩也。}

\textbf{二子乘舟,汎汎其景。}{\footnotesize 二子,伋壽也。宣公為伋取於齊女而美,公奪之,生壽及朔,朔與其母愬伋於公,公令伋之齊,使賊先待於隘而殺之,壽知之,以告伋,使去之,伋曰「君命也,不可以逃」,壽竊其節而先往,賊殺之,伋至,曰「君命殺我,壽有何罪」,賊又殺之,國人傷其涉危遂往,如乘舟而無所薄,汎汎然迅疾而不礙也。}\textbf{願言思子,中心養養。}{\footnotesize 願,每也。養養然,憂不知所定。箋云願,念也。念我思此二子,心為之憂養養然。}

\begin{quoting}\textbf{王引之}景讀如憬,憬,遠行貌。\textbf{陳奐}皇皇者華傳訓每為雖。養,同恙,說文「恙,憂也」。\end{quoting}

\textbf{二子乘舟,汎汎其逝。}{\footnotesize 逝,往也。}\textbf{願言思子,不瑕有害。}{\footnotesize 言二子之不遠害。箋云瑕,猶過也。我念思此二子之事,於行無過差,有何不可而不去也。}

%\begin{flushright}邶國十九篇、七十一章、三百六十三句\end{flushright}