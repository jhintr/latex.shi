\chapter{魏葛屨詁訓傳第九}

\begin{quoting}\textbf{釋文}案魏世家及左氏傳云「姬姓國也」,詩譜云「周以封同姓,其地虞舜夏禹所都之域也,在古冀州雷首之北、析城之西,南枕河曲,北涉汾水」。\end{quoting}

\section{葛屨}

%{\footnotesize 二章、一章六句、一章五句}

\textbf{葛屨,刺褊也。魏地陿隘,其民機巧趨利,其君儉嗇褊急而無德以將之。}{\footnotesize 儉嗇而無德,是其所以見侵削。}

\textbf{糾糾葛屨,可以履霜。}{\footnotesize 糾糾,猶繚繚也。夏葛屨,冬皮屨,葛屨非所以屨霜。箋云葛屨賤,皮屨貴,魏俗至冬猶謂葛屨可以屨霜,利其賤也。}\textbf{摻摻女手,可以縫裳。}{\footnotesize 摻摻,猶纖纖也。婦人三月廟見,然後執婦功。箋云言女手者,未三月未成為婦。裳,男子之下服,賤,又未可使縫,魏俗使未三月婦縫裳者,利其事也。}\textbf{要之襋之,好人服之。}{\footnotesize 要,䙅也。襋,領也。好人,好女手之人。箋云服,整也。䙅也領也在上,好人尚可使整治之,謂屬著之。}

\begin{quoting}\textbf{嚴粲}葛屨既敝而以繩糾纏之,糾而復糾,行於霜雪寒沍之地,言其苦也。可,同何。摻摻,韓詩作纖纖。䙅,繫衣之短帶也。\end{quoting}

\textbf{好人提提,宛然左辟,佩其象揥。}{\footnotesize 提提,安諦也。宛,辟貌。婦至門,夫揖而入,不敢當尊,宛然而左辟。象揥,所以為飾。箋云婦新至,慎於威儀,如是使之,非禮。}\textbf{維是褊心,是以為刺。}{\footnotesize 箋云魏俗所以然者,是君心褊急,無德教使之耳,我是以刺之。}

\begin{quoting}爾雅釋訓「媞媞,安也」。\textbf{朱熹}揥所以摘髮,用象為之,貴者之飾也。說文「褊,衣小也」,段注「引申為凡小之稱」。\end{quoting}

\section{汾沮洳}

%{\footnotesize 三章、章六句}

\textbf{汾沮洳,刺儉也。其君儉以能勤,刺不得禮也。}

\textbf{彼汾沮洳,言采其莫。}{\footnotesize 汾,水也。沮洳,其漸洳者。莫,菜也。箋云言,我也。於彼汾水漸洳之中,我采其莫以為菜,是儉以能勤。}\textbf{彼其之子,美無度。}{\footnotesize 箋云之子,是子也。是子之德美無有度,言不可尺寸。}\textbf{美無度,殊異乎公路。}{\footnotesize 路,車也。箋云是子之德美信無度矣,雖然,其采莫之事則非公路之禮也。公路,主君之軞車,庶子為之,晉趙盾為軞車之族是也。}

\begin{quoting}沮洳 \texttt{jù rù},水旁低濕處。殊,異也,說文「殊,死也」,段注「死罪者身首分離,故曰殊死,引伸為殊異」。\end{quoting}

\textbf{彼汾一方,言采其桑。}{\footnotesize 箋云采桑,親蠶事也。}\textbf{彼其之子,美如英。}{\footnotesize 萬人為英。}\textbf{美如英,殊異乎公行。}{\footnotesize 公行,從公之行也。箋云從公之行者,主君兵車之行列。}

\begin{quoting}方,同旁。\end{quoting}

\textbf{彼汾一曲,言采其藚。}{\footnotesize 藚,水舄也。}\textbf{彼其之子,美如玉。美如玉,殊異乎公族。}{\footnotesize 公族,公屬。箋云公族,主君同姓昭穆也。}

\begin{quoting}藚 \texttt{xù},即澤瀉。\end{quoting}

\section{園有桃}

%{\footnotesize 二章、章十二句}

\textbf{園有桃,刺時也。大夫憂其君國小而迫而儉以嗇,不能用其民而無德教,日以侵削,故作是詩也。}

\textbf{園有桃,其實之殽。}{\footnotesize 興也。園有桃,其實之食,國有民,得其力。箋云魏君薄公稅,省國用,不取於民,食園桃而已,不施德教民,無以戰,其侵削之由,由是也。}\textbf{心之憂矣,我歌且謠。}{\footnotesize 曲合樂曰歌,徒歌曰謠。箋云我心憂君之行如此,故歌謠以寫我憂矣。}\textbf{不知我者,謂我士也驕。}{\footnotesize 箋云士,事也。不知我所為歌謠之意者,反謂我於君事驕逸故。}\textbf{彼人是哉,子曰何其。}{\footnotesize 夫人謂我欲何為乎。箋云彼人,謂君也。曰,於也。不知我所為憂者,既非責我,又曰君儉而嗇,所行是其道哉,子於此憂之,何乎。}\textbf{心之憂矣,其誰知之。}{\footnotesize 箋云如是則眾臣無知我憂所為也。}\textbf{其誰知之,蓋亦勿思。}{\footnotesize 箋云無知我憂所為者,則宜無復思念之以自止也。眾不信我,或時謂我謗君,使我得罪也。}

\begin{quoting}\textbf{朱熹}殽,食也。\textbf{林義光}不知我者之言也,言彼在位者所行良是,而子譏之,果何故乎。\end{quoting}

\textbf{園有棘,其實之食。}{\footnotesize 棘,棗也。}\textbf{心之憂矣,聊以行國。}{\footnotesize 箋云聊,且略之辭也。聊出行於國中,觀民事以寫憂。}\textbf{不知我者,謂我士也罔極。}{\footnotesize 極,中也。箋云見我聊出行於國中,謂我於君事無中正。}\textbf{彼人是哉,子曰何其。心之憂矣,其誰知之。其誰知之,蓋亦勿思。}

\section{陟岵}

%{\footnotesize 三章、章六句}

\textbf{陟岵,孝子行役,思念父母也。國迫而數侵削,役乎大國,父母兄弟離散而作是詩也。}{\footnotesize 役乎大國者,為大國所徵發。}

\textbf{陟彼岵兮,瞻望父兮。}{\footnotesize 山無草木曰岵。箋云孝子行役,思其父之戒,乃登彼岵山,以遙瞻望其父所在之處。}\textbf{父曰嗟予子,行役夙夜無已。}{\footnotesize 箋云予我、夙早、夜莫也。無已,無懈倦。}\textbf{上慎旃哉,猶來無止。}{\footnotesize 旃之、猶可也。父尚義。箋云上者,謂在軍事作部列時。}

\begin{quoting}岵 \texttt{hù}。上,魯詩作尚。旃,之焉合音字。\end{quoting}

\textbf{陟彼屺兮,瞻望母兮。}{\footnotesize 山有草木曰屺。箋云此又思母之戒而登屺山而望之。}\textbf{母曰嗟予季,行役夙夜無寐。}{\footnotesize 季,少子也。無寐,無耆寐也。}\textbf{上慎旃哉,猶來無棄。}{\footnotesize 母尚恩也。}

\textbf{陟彼岡兮,瞻望兄兮。兄曰嗟予弟,行役夙夜必偕。}{\footnotesize 偕,俱也。}\textbf{上慎旃哉,猶來無死。}{\footnotesize 兄尚親也。}

\section{十畝之間}

%{\footnotesize 二章、章三句}

\textbf{十畝之間,刺時也。言其國削小,民無所居焉。}

\textbf{十畝之間兮,桑者閑閑兮,}{\footnotesize 閑閑然,男女無別往來之貌。箋云古者一夫百畝,今十畝之間往來者閑閑然,削小之甚。}\textbf{行與子還兮。}{\footnotesize 或行來者,或來還者。}

\begin{quoting}閑閑,釋文作閒閒。漢書顏注「行,且也」,\textbf{王引之}此詩「行與子還、行與子逝」,猶言且與子歸、且與子往也。\end{quoting}

\textbf{十畝之外兮,桑者泄泄兮,}{\footnotesize 泄泄,多人之貌。}\textbf{行與子逝兮。}{\footnotesize 箋云逝,逮也。}

\section{伐檀}

%{\footnotesize 三章、章九句}

\textbf{伐檀,刺貪也。在位貪鄙,無功而受祿,君子不得進仕爾。}

\textbf{坎坎伐檀兮,寘之河之干兮,河水清且漣猗。}{\footnotesize 坎坎,伐檀聲。寘,置也。干,厓也。風行水成文曰漣。伐檀以俟世用,若俟河水清且漣。箋云是謂君子之人不得進仕也。}\textbf{不稼不穡,胡取禾三百廛兮。不狩不獵,胡瞻爾庭有縣貆兮。}{\footnotesize 種之曰稼,斂之曰穡。一夫之居曰廛。貆,獸名。箋云是謂在位貪鄙,無功而受祿也。冬獵曰狩,宵田曰獵。胡,何也。貉子曰貆。}\textbf{彼君子兮,不素餐兮。}{\footnotesize 素,空也。箋云彼君子者,斥伐檀之人。仕有功乃肯受祿。}

\begin{quoting}干,同岸。猗,魯詩作兮,二字古通用。周官遂人「夫一廛 \texttt{chán},田百畝」,鄭注「廛,居也」。貆 \texttt{huán}。孟子盡心「無功而食謂之素餐」。\end{quoting}

\textbf{坎坎伐輻兮,寘之河之側兮,河水清且直猗。}{\footnotesize 輻,檀輻也。側,猶厓也。直,直波也。}\textbf{不稼不穡,胡取禾三百億兮。不狩不獵,胡瞻爾庭有縣特兮。}{\footnotesize 萬萬曰億。獸三歲曰特。箋云十萬曰億。三百億,禾秉之數。}\textbf{彼君子兮,不素食兮。}

\textbf{坎坎伐輪兮,寘之河之漘兮,河水清且淪猗。}{\footnotesize 檀可以為輪。漘,厓也。小風水成文轉如輪也。}\textbf{不稼不穡,胡取禾三百囷兮。不狩不獵,胡瞻爾庭有縣鶉兮。}{\footnotesize 圓者為囷。鶉,鳥也。}\textbf{彼君子兮,不素飧兮。}{\footnotesize 熟食曰飧。箋云飧,讀如魚飧之飧。}

\begin{quoting}說文「漘,水厓也」。爾雅「小波為淪」。囷 \texttt{qūn}。釋文引字林云「飧 \texttt{sūn},水澆飯也」。\end{quoting}

\section{碩鼠}

%{\footnotesize 三章、章八句}

\textbf{碩鼠,刺重斂也。國人刺其君重斂蠶食於民,不修其政,貪而畏人,若大鼠也。}

\textbf{碩鼠碩鼠,無食我黍。三歲貫女,莫我肯顧。}{\footnotesize 貫,事也。箋云碩,大也。大鼠大鼠者,斥其君也,女無復食我黍,疾其稅斂之多也,我事女三歲矣,曾無教令恩德來眷顧我,又疾其不修政也。古者三年大比,民或於是徙。}\textbf{逝將去女,適彼樂土。}{\footnotesize 箋云逝,往也。往矣將去女,與之訣別之辭。樂土,有德之國。}\textbf{樂土樂土,爰得我所。}{\footnotesize 箋云爰,曰也。}

\begin{quoting}貫,魯詩作宦,國語韋注「宦,為臣隸也」。女,韓詩作汝。\end{quoting}

\textbf{碩鼠碩鼠,無食我麥。三歲貫女,莫我肯德。}{\footnotesize 箋云不肯施德於我。}\textbf{逝將去女,適彼樂國。樂國樂國,爰得我直。}{\footnotesize 直,得其直道。箋云直,猶正也。}

\textbf{碩鼠碩鼠,無食我苗。}{\footnotesize 苗,嘉穀也。}\textbf{三歲貫女,莫我肯勞。}{\footnotesize 箋云不肯勞來我。}\textbf{逝將去女,適彼樂郊。}{\footnotesize 箋云郭外曰郊。}\textbf{樂郊樂郊,誰之永號。}{\footnotesize 號,呼也。箋云之,往也。永,歌也。樂郊之地,誰獨當往而歌號者,言皆喜說無憂苦。}

%\begin{flushright}魏國七篇、十八章、百二十八句\end{flushright}