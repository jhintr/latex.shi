\chapter{Lesson 11}

\section*{Reading 1}

Atīte Jambudīpe Ajitaraṭṭhe eko gopālako vasi. Tassa gehe eko Paccekabuddho nibaddhaṃ bhuñjati. Tasmiṃ gehe eko kukkuro ca ahosi. Paccekabuddho bhuñjanto tassa nibaddhaṃ ekaṃ bhattapiṇḍaṃ adāsi. So tena Paccekabuddhe sinehaṃ akāsi. Gopālako divasassa dve vāre Paccekabuddhass’upaṭṭhānaṃ gacchi. Sunakho pi tena saddhiṃ gacchi.

Gopālo ekadivasaṃ Paccekabuddhaṃ āha “bhante, yadā me okāso na bhavissati, tadā imaṃ sunakhaṃ pesessāmi, tena saññāṇena āgaccheyyāthā” ti. Tato paṭṭhāya anokāsadivase sunakhaṃ pesesi. So ekavacanen’eva pakkhanditvā Paccekabuddhassa vasanaṭṭhānaṃ gantvā tikkhattuṃ bhussitvā attano āgatabhāvaṃ jānāpetvā ekamantaṃ nipajji. Paccekabuddhe velaṃ sallakkhetvā nikkhante bhussanto purato gacchi. Paccekabuddho taṃ vīmaṃsanto ekadivasaṃ aññaṃ maggaṃ paṭipajji. Atha sunakho purato tiriyaṃ ṭhatvā bhussitvā itaramaggameva naṃ āropesi.

Ath’ekadivasaṃ aññaṃ maggaṃ paṭipajjitvā sunakhena tiriyaṃ ṭhatvā vāriyamāno pi anivattitvā taṃ pādena apanetvā pāyāsi. Sunakho tassa anivattanabhāvaṃ ñatvā nivāsanakaṇṇe ḍasitvā ākaḍḍhanto gantabbamaggam eva pāpesi. Evaṃ so sunakho tasmiṃ Paccekabuddhe balavasinehaṃ uppādesi.

Aparabhāge Paccekabuddhassa cīvaraṃ jīri. Ath’assa gopālako cīvaravatthāni adāsi. Paccekabuddho “phāsukaṭṭhānaṃ gantvā cīvaraṃ kāressāmī” ti gopālakaṃ āha. So pi “bhante, mā ciraṃ bahi vasitthā” ti avadi. Sunakho pi tesaṃ kathaṃ suṇanto aṭṭhāsi. Paccekabuddhe vehāsaṃ abbhuggantvā gacchante bhuṅkaritvā ṭhitassa sunakhassa hadayaṃ phali.

Tiracchānā nām’ete ujujātikā honti akuṭilā.

Manussā pana aññaṃ cintenti, aññaṃ vadanti. (Rasv)

\section*{Reading 2}

Evaṃ me sutaṃ ekaṃ samayaṃ Bhagavā Āḷaviyaṃ viharati Āḷavakassa yakkhassa bhavane. Atha kho Āḷavako yakkho yena Bhagavā ten’upasaṅkami, upasaṅkamitvā Bhagavantaṃ etadavoca “Nikkhama, samaṇā” ti. “Sādhāvuso” ti Bhagavā nikkhami. “Pavisa, samaṇā” ti. “Sādhāvuso” ti Bhagavā pāvisi.

Dutiyaṃ pi kho Āḷavako yakkho Bhagavantaṃ etadavoca “Nikkhama, samaṇā” ti. “Sādhāvuso” ti Bhagavā nikkhami. “Pavisa, samaṇā” ti. “Sādhāvuso” ti Bhagavā pāvisi.

Tatiyaṃ pi kho Āḷavako yakkho Bhagavantaṃ etadavoca “Nikkhama, samaṇā” ti. “Sādhāvuso” ti Bhagavā nikkhami. “Pavisa, samaṇā” ti. “Sādhāvuso” ti Bhagavā pāvisi.

Catutthaṃ pi kho Āḷavako yakkho Bhagavantaṃ etadavoca “Nikkhama, samaṇā” ti. “Na khvāhaṃ taṃ, āvuso, nikkhamissāmi. Yan te karaṇīyaṃ, taṃ karohī” ti.

“Pañhaṃ taṃ, samaṇa, pucchissāmi. Sace me na byākarissasi, cittaṃ vā te khipissāmi, hadayaṃ vā te phālessāmi, pādesu vā gahetvā pāragaṅgāya khipissāmī” ti.

“Na khvāhaṃ taṃ, āvuso, passāmi sadevake loke sabrahmake sassamaṇabrāhmaṇiyā pajāya sadevamanussāya yo me cittaṃ vā khipeyya hadayaṃ vā phāleyya pādesu vā gahetvā pāragaṅgāya khipeyya. Api ca tvaṃ, āvuso, puccha yad ākaṅkhasī” ti.

Atha kho Āḷavako yakkho Bhagavantaṃ gāthāya ajjhabhāsi

“Kiṃ sū’dha vittaṃ purisassa seṭṭhaṃ?\\
Kiṃ su suciṇṇaṃ sukham āvahāti?\\
Kiṃ su have sādutaraṃ rasānaṃ?\\
Kathaṃjīviṃ jīvitam āhu seṭṭhaṃ?”

“Saddhīdha vittaṃ purisassa seṭṭhaṃ,\\
Dhammo suciṇṇo sukhamāvahāti,\\
Saccaṃ have sādutaraṃ rasānaṃ,\\
Paññājīviṃ jīvitam āhu seṭṭhaṃ”. (Sn 1.10 Āḷavakasuttaṃ)

\section*{Reading 3}

Na antalikkhe na samuddam ajjhe,\\
na pabbatānaṃ vivaraṃ pavissa,\\
na vijjatī so jagatippadeso,\\
yatthaṭṭhito mucceyya pāpakammā.

Na antalikkhe na samuddam ajjhe,\\
na pabbatānaṃ vivaraṃ pavissa,\\
na vijjatī so jagatippadeso,\\
yatthaṭṭhitaṃ nappasaheyya maccu. (Dhp 9)

Sukhakāmāni bhūtāni,\\
yo daṇḍena vihiṃsati,\\
attano sukham esāno,\\
pecca so na labhate sukhaṃ.

Sukhakāmāni bhūtāni,\\
yo daṇḍena na hiṃsati,\\
attano sukham esāno,\\
pecca so labhate sukhaṃ. (Dhp 10)

Parijiṇṇam idaṃ rūpaṃ,\\
roganīḷaṃ pabhaṅguraṃ,\\
Bhijjati pūtisandeho,\\
maraṇantaṃ hi jīvitaṃ. (Dhp 11)

\section*{Reading 4}

Atha kho Bhagavā pañcavaggiye bhikkhū āmantesi “Rūpaṃ, bhikkhave, anattā. Rūpaṃ ca h’idaṃ, bhikkhave, attā abhavissa, nayidaṃ rūpaṃ ābādhāya saṃvatteyya, labbhetha ca rūpe ‘evaṃ me rūpaṃ hotu, evaṃ me rūpaṃ mā ahosī’ ti. Yasmā ca kho, bhikkhave, rūpaṃ anattā, tasmā rūpaṃ ābādhāya saṃvattati, na ca labbhati rūpe ‘evaṃ me rūpaṃ hotu, evaṃ me rūpaṃ mā ahosī’ ti.

Vedanā, bhikkhave, anattā. Vedanā ca h’idaṃ, bhikkhave, attā abhavissa, nayidaṃ vedanā ābādhāya saṃvatteyya, labbhetha ca vedanāya ‘evaṃ me vedanā hotu, evaṃ me vedanā mā ahosī’ ti. Yasmā ca kho, bhikkhave, vedanā anattā, tasmā vedanā ābādhāya saṃvattati, na ca labbhati vedanāya ‘evaṃ me vedanā hotu, evaṃ me vedanā mā ahosī’ ti.

Saññā, bhikkhave, anattā. Saññā ca h’idaṃ, bhikkhave, attā abhavissa, nayidaṃ saññā ābādhāya saṃvatteyya, labbhetha ca saññāya ‘evaṃ me saññā hotu, evaṃ me saññā mā ahosī’ ti. Yasmā ca kho, bhikkhave, saññā anattā, tasmā saññā ābādhāya saṃvattati, na ca labbhati saññāya ‘evaṃ me saññā hotu, evaṃ me saññā mā ahosī’ ti.

Saṅkhārā, bhikkhave, anattā. Saṅkhārā ca h’idaṃ, bhikkhave, attā abhavissaṃsu, nayidaṃ saṅkhārā ābādhāya saṃvatteyyuṃ, labbhetha ca saṅkhāresu ‘evaṃ me saṅkhārā hontu, evaṃ me saṅkhārā mā ahesun’ ti. Yasmā ca kho, bhikkhave, saṅkhārā anattā, tasmā saṅkhārā ābādhāya saṃvattanti, na ca labbhati saṅkhāresu ‘evaṃ me saṅkhārā hontu, evaṃ me saṅkhārā mā ahesun’ ti.

Viññāṇaṃ, bhikkhave, anattā. Viññāṇañ ca h’idaṃ, bhikkhave, attā abhavissa, nayidaṃ viññāṇaṃ ābādhāya saṃvatteyya, labbhetha ca viññāṇe ‘evaṃ me viññāṇaṃ hotu, evaṃ me viññāṇaṃ mā ahosī’ ti. Yasmā ca kho, bhikkhave, viññāṇaṃ anattā, tasmā viññāṇaṃ ābādhāya saṃvattati, na ca labbhati viññāṇe ‘evaṃ me viññāṇaṃ hotu, evaṃ me viññāṇaṃ mā ahosī’” ti. (Vin I 7-8)

\section*{Further Reading 1}

‘Na tvaṃ addasā manussesu itthiṃ vā purisaṃ vā āsītikaṃ vā nāvutikaṃ vā vassasatikaṃ vā jātiyā, jiṇṇaṃ gopānasivaṅkaṃ bhoggaṃ daṇḍaparāyaṇaṃ pavedhamānaṃ gacchantaṃ āturaṃ gatayobbanaṃ khaṇḍadantaṃ palitakesaṃ vilūnaṃ khallitaṃsiro valitaṃ tilakāhatagattan’ ti?

Tassa te viññussa sato mahallakassa na etadahosi ‘Aham pi kho ’mhi jarādhammo jaraṃ anatīto. Handāhaṃ kalyāṇaṃ karomi, kāyena vācāya manasā’ ti?

‘Na tvaṃ addasā manussesu itthiṃ vā purisaṃ vā ābādhikaṃ dukkhitaṃ bāḷhagilānaṃ, sake muttakarīse palipannaṃ semānaṃ, aññehi vuṭṭhāpiyamānaṃ, aññehi saṃvesiyamānan’ ti?

Tassa te viññussa sato mahallakassa na etadahosi ‘Aham pi kho ’mhi vyādhidhammo vyādhiṃ anatīto. Handāhaṃ kalyāṇaṃ karomi kāyena vācāya manasā’ ti?

‘Na tvaṃ addasā manussesu itthiṃ vā purisaṃ vā ekāhamataṃ vā dvīhamataṃ vā tīhamataṃ vā uddhumātakaṃ vinīlakaṃ vipubbakajātan’ ti?

Tassa te viññussa sato mahallakassa na etadahosi ‘Aham pi kho ’mhi maraṇadhammo maraṇaṃ anatīto. Handāhaṃ kalyāṇaṃ karomi kāyena vācāya manasā’ ti? (A 3.36 Devadūtasuttaṃ)

\section*{Further Reading 2}

Katamā ca, bhikkhave, sammādiṭṭhi? Yaṃ kho, bhikkhave, dukkhe ñāṇaṃ, dukkhasamudaye ñāṇaṃ, dukkhanirodhe ñāṇaṃ, dukkhanirodhagāminiyā paṭipadāya ñāṇaṃ. Ayaṃ vuccati, bhikkhave, sammādiṭṭhi ti. (D 22 Mahāsatipaṭṭhānasuttaṃ, M 141 Saccavibhangasuttaṃ)

Yato kho, āvuso, ariyasāvako akusalañ ca pajānāti, akusalamūlañ ca pajānāti, kusalañ ca pajānāti, kusalamūlañ ca pajānāti ettāvatā pi kho, āvuso, ariyasāvako sammādiṭṭhi hoti, dhamme aveccappasādena samannāgato, āgato imaṃ saddhammaṃ.

Katamaṃ panāvuso, akusalaṃ, katamaṃ akusalamūlaṃ, katamaṃ kusalaṃ, katamaṃ kusalamūlan ti? Pāṇātipāto kho, āvuso, akusalaṃ, adinnādānaṃ akusalaṃ, kāmesu micchācāro akusalaṃ, musāvādo akusalaṃ, pisuṇā vācā akusalaṃ, pharusā vācā akusalaṃ, samphappalāpo akusalaṃ, abhijjhā akusalaṃ, byāpādo akusalaṃ, micchādiṭṭhi akusalaṃ, idaṃ vuccatāvuso akusalaṃ. Ime dasa dhammā “akusalakammapathā” ti nāmena pi ñātabbā.

Katamañ cāvuso, akusalamūlaṃ? Lobho akusalamūlaṃ, doso akusalamūlaṃ, moho akusalamūlaṃ, idaṃ vuccatāvuso, akusalamūlaṃ.

Katamañ cāvuso, kusalaṃ? Pāṇātipātā veramaṇī kusalaṃ, adinnādānā veramaṇī kusalaṃ, kāmesu micchācārā veramaṇī kusalaṃ, musāvādā veramaṇī kusalaṃ, pisuṇāya vācāya veramaṇī kusalaṃ, pharusāya vācāya veramaṇī kusalaṃ, samphappalāpā veramaṇī kusalaṃ, anabhijjhā kusalaṃ, abyāpādo kusalaṃ, sammādiṭṭhi kusalaṃ, idaṃ vuccatāvuso kusalaṃ. Ime dasa dhammā “kusalakammapathā” ti nāmena pi ñātabbā.

Katamañ cāvuso, kusalamūlaṃ? Alobho kusalamūlaṃ, adoso kusalamūlaṃ, amoho kusalamūlaṃ, idaṃ vuccatāvuso, kusalamūlaṃ. (M 9 Sammādiṭṭhisuttaṃ)

\section*{Further Reading 3}

Pañcahi, bhikkhave, aṅgehi samannāgato mātugāmo ekantāmanāpo hoti purisassa. Katamehi pañcahi?

Na ca rūpavā hoti, na ca bhogavā hoti, na ca sīlavā hoti, alaso ca hoti, pajañ cassa na labhati, imehi kho, bhikkhave, pañcahi aṅgehi samannāgato mātugāmo ekantāmanāpo hoti purisassa.

Pañcahi, bhikkhave, aṅgehi samannāgato mātugāmo ekantamanāpo hoti purisassa. Katamehi pañcahi?

Rūpavā ca hoti, bhogavā ca hoti, sīlavā ca hoti, dakkho ca hoti analaso, pajañ cassa labhati, imehi kho, bhikkhave, pañcahi aṅgehi samannāgato mātugāmo ekantamanāpo hoti purisassā. (S 37.1 Mātugāmasuttaṃ)

Pañcahi, bhikkhave, aṅgehi samannāgato puriso ekantāmanāpo hoti mātugāmassa. Katamehi pañcahi?

Na ca rūpavā hoti, na ca bhogavā hoti, na ca sīlavā hoti, alaso ca hoti, pajañ cassa na labhati, imehi kho, bhikkhave, pañcahi aṅgehi samannāgato puriso ekantāmanāpo hoti mātugāmassa.

Pañcahi, bhikkhave, aṅgehi samannāgato puriso ekantamanāpo hoti mātugāmassa. Katamehi pañcahi?

Rūpavā ca hoti, bhogavā ca hoti, sīlavā ca hoti, dakkho ca hoti analaso, pajañcassa labhati, imehi kho, bhikkhave, pañcahi aṅgehi samannāgato puriso ekantamanāpo hoti mātugāmassā’ ti. (S 37.2 Purisasuttaṃ)

\section*{Further Reading 4}

Pañcimāni, bhikkhave, mātugāmassa āveṇikāni dukkhāni, yāni mātugāmo paccanubhoti, aññatr’eva purisehi. Katamāni pañca?

Idha, bhikkhave, mātugāmo daharo va samāno patikulaṃ gacchati, ñātakehi vinā hoti. Idaṃ, bhikkhave, mātugāmassa paṭhamaṃ āveṇikaṃ dukkhaṃ, yaṃ mātugāmo paccanubhoti, aññatr’eva purisehi.

Puna ca paraṃ, bhikkhave, mātugāmo utunī hoti. Idaṃ, bhikkhave, mātugāmassa dutiyaṃ āveṇikaṃ dukkhaṃ, yaṃ mātugāmo paccanubhoti, aññatr’eva purisehi.

Puna ca paraṃ, bhikkhave, mātugāmo gabbhinī hoti. Idaṃ, bhikkhave, mātugāmassa tatiyaṃ āveṇikaṃ dukkhaṃ, yaṃ mātugāmo paccanubhoti, aññatr’eva purisehi.

Puna ca paraṃ, bhikkhave, mātugāmo vijāyati. Idaṃ, bhikkhave, mātugāmassa catutthaṃ āveṇikaṃ dukkhaṃ, yaṃ mātugāmo paccanubhoti, aññatr’eva purisehi.

Puna ca paraṃ, bhikkhave, mātugāmo purisassa pāricariyaṃ upeti. Idaṃ kho, bhikkhave, mātugāmassa pañcamaṃ āveṇikaṃ dukkhaṃ, yaṃ mātugāmo paccanubhoti, aññatr’eva purisehi.

Imāni kho, bhikkhave, pañca mātugāmassa āveṇikāni dukkhāni, yāni mātugāmo paccanubhoti, aññatr’eva purisehī ti. (S 37.3 Āveṇikadukkhasuttaṃ)

\section*{Further Reading 5}

Atha kho rājā Pasenadikosalo yena Bhagavā ten’upasaṅkami, upasaṅkamitvā Bhagavantaṃ abhivādetvā ekamantaṃ nisīdi. Atha kho aññataro puriso yena rājā Pasenadikosalo ten’upasaṅkami, upasaṅkamitvā rañño Pasenadīkosalassa upakaṇṇake ārocesi “Mallikā, deva, devī dhītaraṃ vijātā” ti. Evaṃ vutte, rājā Pasenadikosalo anattamano ahosi.

Atha kho Bhagavā rājānaṃ Pasenadikosalaṃ anattamanataṃ viditvā tāyaṃ velāyaṃ imā gāthāyo abhāsi

Itthī pi hi ekacciyā,\\
seyyā posa janādhipa,\\
medhāvinī sīlavatī,\\
sassudevā patibbatā.

Tassā yo jāyati poso,\\
sūro hoti disampati,\\
tādisā subhariyā putto,\\
rajjam pi anusāsatī ti. (S 3.16 Mallikāsuttaṃ)