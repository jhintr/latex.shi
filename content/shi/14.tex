\chapter{曹蜉蝣詁訓傳第十四}

\begin{quoting}\textbf{釋文}曹者,武王之弟叔振鐸所封之國也,爵為伯,其封域在兗州陶丘之北、菏澤之野,今濟陰定陶是也。\end{quoting}

\section{蜉蝣}

%{\footnotesize 三章、章四句}

\textbf{蜉蝣,刺奢也。昭公國小而迫,無法以自守,好奢而任小人,將無所依焉。}

\textbf{蜉蝣之羽,衣裳楚楚。}{\footnotesize 興也。蜉蝣,渠略也,朝生夕死,猶有羽翼以自修飾。楚楚,鮮明貌。箋云興者,喻昭公之朝,其群臣皆小人也,徒整飾其衣裳,不知國之將迫脅,君臣死亡無日,如渠略然。}\textbf{心之憂矣,於我歸處。}{\footnotesize 箋云歸,依歸。君當於何依歸乎,言有危亡之難,將無所就往。}

\begin{quoting}\textbf{馬瑞辰}於之言與也,凡相於者,猶相與也,如孟子「麒麟之於走獸」之類,於即與也,憂蜉蝣之於我歸處,以言我將與浮游同歸也。\end{quoting}

\textbf{蜉蝣之翼,采采衣服。}{\footnotesize 采采,眾多也。}\textbf{心之憂矣,於我歸息。}{\footnotesize 息,止也。}

\textbf{蜉蝣掘閱,麻衣如雪。}{\footnotesize 掘閱,容閱也。如雪,言鮮潔。箋云掘閱,掘地解閱,謂其始生時也。以解閱喻君臣朝夕變易衣服也。麻衣,深衣,諸侯之朝服,朝夕則深衣也。}\textbf{心之憂矣,於我歸說。}{\footnotesize 箋云說,猶舍息也。}

\begin{quoting}掘,三家詩作堀,閱,通穴,掘閱,穿穴。麻衣,其羽翼也。\end{quoting}

\section{候人}

%{\footnotesize 四章、章四句}

\textbf{候人,刺近小人也。共公遠君子而好近小人焉。}

\textbf{彼候人兮,何戈與祋。}{\footnotesize 候人,道路送賓客者。何揭、祋殳也。言賢者之官不過候人。箋云是謂遠君子也。}\textbf{彼其之子,三百赤芾。}{\footnotesize 彼,彼曹朝也。芾,韠也。一命縕芾黝珩,再命赤芾黝珩,三命赤芾葱珩,大夫以上赤芾乘軒。箋云之子,是子也。佩赤芾者三百人。}

\begin{quoting}周官候人「若有方治,則帥而致于朝,及歸,送之于竟」,國語周語「敵國賓至,關尹以告,行理以節逆之,候人為導」。祋 \texttt{duì}。\end{quoting}

\textbf{維鵜在梁,不濡其翼。}{\footnotesize 鵜,洿澤鳥也。梁,水中之梁。鵜在梁,可謂不濡其翼乎。箋云鵜在梁當濡其翼,而不濡者,非其常也,以喻小人在朝亦非其常。}\textbf{彼其之子,不稱其服。}{\footnotesize 箋云不稱者,言德薄而服尊。}

\textbf{維鵜在梁,不濡其咮。}{\footnotesize 咮,喙也。}\textbf{彼其之子,不遂其媾。}{\footnotesize 媾,厚也。箋云遂,猶久也。不久其厚,言終將薄於君也。}

\begin{quoting}咮 \texttt{zhòu},韓詩作噣。\end{quoting}

\textbf{薈兮蔚兮,南山朝隮。}{\footnotesize 薈蔚,雲興貌。南山,曹南山也。隮,升雲也。箋云薈蔚之小雲,朝升於南山,不能為大雨,以喻小人雖見任於君,終不能成其德教。}\textbf{婉兮孌兮,季女斯飢。}{\footnotesize 婉,少貌。孌,好貌。季,人之少子也。女,民之弱者。箋云天無大雨則歲不熟而幼弱者飢,猶國之無政令則下民困病。}

\begin{quoting}朝隮,見蝃蝀注。\end{quoting}

\section{鳲鳩}

%{\footnotesize 四章、章六句}

\textbf{鳲鳩,刺不壹也。在位無君子,用心之不壹也。}{\footnotesize }

\textbf{鳲鳩在桑,其子七兮。}{\footnotesize 興也。鳲鳩,秸鞠也。鳲鳩之養其子,朝從上下,莫從下上,平均如一。箋云興者,喻人君之德當均一於下也,以刺今在位之人不如鳲鳩。}\textbf{淑人君子,其儀一兮。}{\footnotesize 箋云淑善、儀義也。善人君子,其執義當如一也。}\textbf{其儀一兮,心如結兮。}{\footnotesize 言執義一則用心同。}

\begin{quoting}鳲鳩,即布穀。\end{quoting}

\textbf{鳲鳩在桑,其子在梅。}{\footnotesize 飛在梅也。}\textbf{淑人君子,其帶伊絲。其帶伊絲,其弁伊騏。}{\footnotesize 騏,騏文也。弁,皮弁也。箋云其帶伊絲,謂大帶也,大帶用素絲,有雜色飾焉。騏,當作𤪌,以玉為之。言此帶弁者,刺不稱其服。}

\textbf{鳲鳩在桑,其子在棘。淑人君子,其儀不忒。}{\footnotesize 忒,疑也。}\textbf{其儀不忒,正是四國。}{\footnotesize 正,長也。箋云執義不疑,則可為四國之長,言任為侯伯。}

\begin{quoting}說文「忒,更也」,段注「凡人有過失改革謂之忒」。\end{quoting}

\textbf{鳲鳩在桑,其子在榛。淑人君子,正是國人。正是國人,胡不萬年。}{\footnotesize 箋云正,長也。能長人,則人欲其壽考。}

\section{下泉}

%{\footnotesize 四章、章四句}

\textbf{下泉,思治也。曹人疾共公侵刻下民,不得其所,憂而思明王賢伯也。}

\textbf{洌彼下泉,浸彼苞稂。}{\footnotesize 興也。洌,寒也。下泉,泉下流也。苞,本也。稂,童梁,非溉草,得水而病也。箋云興者,喻共公之施政教,徒困病其民。稂,當作涼,涼草,蕭蓍之屬。}\textbf{愾我寤嘆,念彼周京。}{\footnotesize 箋云愾,嘆息之意。寤,覺也。念周京者,思其先王之明者也。}

\begin{quoting}洌,當作冽,\textbf{嚴粲}列旁三點者,從水也,清也潔也,旁二點者,從冰也,寒也。苞,叢生。稂 \texttt{láng}。愾 \texttt{kài},魯詩作慨,韓詩作嘅。\end{quoting}

\textbf{洌彼下泉,浸彼苞蕭。}{\footnotesize 蕭,蒿也。}\textbf{愾我寤嘆,念彼京周。}

\textbf{洌彼下泉,浸彼苞蓍。}{\footnotesize 蓍,草也。}\textbf{愾我寤嘆,念彼京師。}

\textbf{芃芃黍苗,陰雨膏之。}{\footnotesize 芃芃,美貌。}\textbf{四國有王,郇伯勞之。}{\footnotesize 郇伯,郇侯也。諸侯有事,二伯述職。箋云有王,謂朝聘於天子也。郇侯,文王之子,為州伯,有治諸侯之功。}

\begin{quoting}芃 \texttt{péng},茂盛貌。郇伯,晉大夫荀躒也。\end{quoting}

%\begin{flushright}曹國四篇、十五章、六十八句\end{flushright}