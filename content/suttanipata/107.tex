\section{贱民经}

\begin{center}Vasala Sutta\end{center}\vspace{1em}

\textbf{如是我闻\footnote{此经旧译见杂阿含经第 102 经、别译杂阿含经第 268 经。贱民 \textit{vasala},杂阿含作「领群特」,别译杂阿含作「旃陀罗」。}。一时世尊住舍卫国祇树给孤独园。于是,世尊晨朝著了下衣,持了衣钵,便入舍卫国乞食。}

Evaṃ me sutaṃ— ekaṃ samayaṃ Bhagavā Sāvatthiyaṃ viharati Jetavane Anāthapiṇḍikassa ārāme. Atha kho Bhagavā pubbaṇhasamayaṃ nivāsetvā pattacīvaram ādāya Sāvatthiṃ piṇḍāya pāvisi.

\begin{enumerate}\item \textbf{事火婆罗豆婆遮经},也称\textbf{贱民经}。缘起为何?世尊住舍卫国祇树给孤独园。以在耕田婆罗豆婆遮经中所说的方法,在饭后义务的终了以佛眼观察世间,看到事火婆罗豆婆遮婆罗门具足皈依与学处的近依,了知到「当我到达那里,将发生谈话,随后,在谈话终了,听闻了法的开示,这婆罗门将在皈依后受持学处」,便去到那里,因发生谈话而受婆罗门祈请法的开示,说了此经。
\item 这里,我们将在吉祥经注中解释「\textbf{如是我闻}」等,而「\textbf{于是,世尊晨朝}」等当知如耕田婆罗豆婆遮经中所述。\end{enumerate}

\textbf{尔时,在事火婆罗豆婆遮婆罗门的住处,火已燃起,祭品已备好。于是,世尊在舍卫国次第行乞时,往事火婆罗豆婆遮婆罗门的住处走去。}

Tena kho pana samayena Aggikabhāradvājassa brāhmaṇassa nivesane aggi pajjalito hoti āhuti paggahitā. Atha kho Bhagavā Sāvatthiyaṃ sapadānaṃ piṇḍāya caramāno yena Aggikabhāradvājassa brāhmaṇassa nivesanaṃ ten’upasaṅkami.

\begin{enumerate}\item 对「尔时,在事火婆罗豆婆遮婆罗门的」等,我们将只解释先前所未说的。即:因为这婆罗门供奉、敬事火,以「事火」之名著称,而以族姓为婆罗豆婆遮,所以称作\textbf{事火婆罗豆婆遮婆罗门}。\textbf{住处},即家。据说,在这婆罗门住处门内的过道边即是火供堂,由此,本当说「住处门内」,由其地连属于住处,故说「住处」。或者,依格为附近义,即住处附近之义。
\item \textbf{火已燃起},即存于火盆中的火被取出,经喂柴扇风而燃烧,生起的火焰向上蹿动。\textbf{祭品已备好},即沐浴已,以极大的恭敬准备粥、酥、蜜、糖等之义。因为任何东西都可在火中供奉,故全都被称为「祭品」。\textbf{次第},即逐家。因为世尊为了摄受所有人,且以于食知足,不避贵贱之家而行乞,因此说「次第行乞」。\end{enumerate}

\textbf{事火婆罗豆婆遮婆罗门看见世尊从远处走来,看见后,对世尊说:「就那里,秃头!就那里,沙门!就那里,贱民!站住!」如是说已,世尊对事火婆罗豆婆遮婆罗门说:「婆罗门!那你知道贱民或成为贱民之法吗?」「乔达摩君!我并不知道贱民或成为贱民之法,善哉!请乔达摩君对我开示这样的法!好让我知道贱民或成为贱民之法。」「那么,婆罗门!谛听!善加作意!我将要说。」「如是,先生!」事火婆罗豆婆遮婆罗门答世尊。世尊说:}

Addasā kho Aggikabhāradvājo brāhmaṇo Bhagavantaṃ dūrato va āgacchantaṃ, disvāna Bhagavantaṃ etad avoca: “tatr’eva, muṇḍaka, tatr’eva, samaṇaka, tatr’eva, vasalaka, tiṭṭhāhī” ti. Evaṃ vutte, Bhagavā Aggikabhāradvājaṃ brāhmaṇaṃ etad avoca: “jānāsi pana tvaṃ, brāhmaṇa, vasalaṃ vā vasalakaraṇe vā dhamme” ti? “Na khvāhaṃ, bho Gotama, jānāmi vasalaṃ vā vasalakaraṇe vā dhamme, sādhu me bhavaṃ Gotamo tathā dhammaṃ desetu, yathāhaṃ jāneyyaṃ vasalaṃ vā vasalakaraṇe vā dhamme” ti. “Tena hi, brāhmaṇa, suṇāhi, sādhukaṃ manasi karohi, bhāsissāmī” ti. “Evaṃ, bho” ti kho Aggikabhāradvājo brāhmaṇo Bhagavato paccassosi. Bhagavā etad avoca:

\begin{enumerate}\item 那么,为什么见到具足一切行相、一切善见的世尊,婆罗门的心不净喜呢?且为什么他要以这么粗恶的语言攻击世尊呢?当答:据说,这婆罗门存如是之见「在行吉祥的仪式时,见到沙门不祥」,随后想到「在给大梵天供食时,一个黑耳的秃头沙门来到我的住处」,心便不净喜,势必受制于嗔。于是,他忿怒、心存不满,便发出「就那里,秃头」等不满之语。
\item 且于此,因为众婆罗门之见以「秃头为不洁」,所以嫌厌着「凡不洁者,不能与其供养诸天婆罗门」,或以「他由秃头故不全,不应来到此地」而说「\textbf{秃头}」。且以「即便成为沙门,他也无法美化这样的身染污」,嫌厌着沙门的状态而说「\textbf{沙门}」。不仅仅受制于嗔,还嫌厌着「他让贱民们出家后,因与他们一起共享、受用而沉沦,甚至比贱民还更差」而说「\textbf{贱民}」,或者思量「对生为贱民者,看见祭品、听闻颂诗是不好的」而如是说。
\item 即便被这样说,世尊仍以明净的面容、以甜美的语音、以出于对婆罗门的慈悲而清凉的心,为显明自己与一切有情不共的本性而说「\textbf{婆罗门!那你知道……}」。
\item 于是,婆罗门了知到世尊净喜的面容所示的本性,听到以慈悲而清凉的心发出的甜美的语音,心如灌以甘露般悦意,诸根明净,慢已折服,舍弃了以出身为自性、如吐毒般攻击性的语言,思量「我为何会认可出身卑贱者就是贱民呢?他在第一义上不是贱民,且出身卑贱也并非成为贱民之法」,便说「\textbf{乔达摩君!我并不……}」。因为这很自然,即便是具因者,由未得缘也会变得粗恶,而在得缘时则会变得柔和。
\item 这里,\textbf{善哉}一词用于请求、领受、高兴、善妙、努力等。用于请求,如\begin{quoting}善哉!尊者!请世尊为我略开示法!(相应部第 4:95 经)\end{quoting}等。用于领受,如\begin{quoting}「善哉!尊者!」那比丘欢喜、随喜于世尊之所说。(中部·大满月经第 86 段)\end{quoting}等。用于高兴,如\begin{quoting}善哉!善哉!舍利弗!(长部·结集经第 349 段)\end{quoting}等。用于善妙,如\begin{quoting}善哉乐法之王,善哉有慧之人,\\善哉不欺众友,不作恶即是乐。(本生第 18:101 颂)\end{quoting}等。用于努力,如\begin{quoting}汝等谛听!善加作意!(中部·根本方法经第 1 段)\end{quoting}等。而在此是用于请求。
\item \textbf{那么},即显示其意趣,即是说「如果你想知道」,或为表示原因之语,即「因为你想知道,所以,婆罗门!谛听!善加作意!如我对你所说,你将如是了知」,当知如是与其它句子连结。且此处,\textbf{谛听}遮止耳根的散乱,\textbf{善加作意}则以鼓励努力于作意,遮止意根的散乱。且此中,前者遮止文字的颠倒执取,后者遮止意义的颠倒执取。又以前者鼓励闻法,以后者鼓励对所闻之法的受持、考察其义等。又以前者显示「此法有文,所以应听闻」,以后者显示「有义,所以应作意」。或者两句都与「善加」一词连结,为显示此义「因为此法法甚深、开示甚深,所以请善加谛听,因为义甚深、证甚深,所以请善加作意」而说「谛听!善加作意」。
\item 随后,这婆罗门似乎以「于如是甚深中,我如何可得住立」而消沉,为令振奋而说「\textbf{我将要说}」。这里,当知其旨趣为「我将以齐整的文句、显明的方法来说,好让你了知」。
\item 随后,他生起勇猛,\textbf{「如是,先生!」事火婆罗豆婆遮婆罗门答世尊},即是说领受、受纳,或以如所教授而行,当面听闻。于是,\textbf{世尊}对他\textbf{说},即现在,就所当说者,说了如下诸颂。\end{enumerate}

\subsection\*{\textbf{116}}

\textbf{若人忿怒,怨恨,恶且覆藏,\\}
\textbf{破见,欺瞒,当知他是贱民。}

“Kodhano upanāhī ca, pāpamakkhī ca yo naro;\\
vipannadiṭṭhi māyāvī, taṃ jaññā vasalo iti. %\hfill\textcolor{gray}{\footnotesize 1}

\begin{enumerate}\item 这里,\textbf{忿怒},即惯于发怒。\textbf{怨恨},即具足此忿怒之强力的怨恨。覆盖、擦拭他人的功德为覆藏,恶与此覆藏为\textbf{恶且覆藏}。\textbf{破见},即亡失正见,或具足破损的、导致扭曲的十事邪见。\textbf{欺瞒},即具足以遮蔽自身现存的过失为相的伪善。
\item \textbf{当知他是贱民},即当知这样的人由此等低劣之法的落入\footnote{落入 \textit{vassanato}:据菩提比丘注 640,义注从「落入」来解释「贱民 \textit{vasala}」,PED 解释 vasala 源于吠陀语 vṛṣala < vṛṣan,意为「小人 \textit{little man}」。}、灌入、漏入而为「贱民」,即便是生于梵天的头中\footnote{此句原文费解,据 PTS 本译出。}。因为这才是第一义上的贱民,以自己内心的满足为量,而非其它。
\item 如是,此中,世尊仅以首句便折服了此婆罗门的忿怒,且以「忿怒等法为低劣之人」的基于人的开示开示忿怒等法,先以一种方法开示了贱民及成为贱民之法。且当如是开示时,未以「你、我」等自赞毁他,只是以平等的法理,置此婆罗门于贱民的状态,并置自己于婆罗门的状态。\end{enumerate}

\subsection\*{\textbf{117}}

\textbf{无论对一生者或对再生者,若于此杀害生命,\\}
\textbf{若于生命无有怜悯,当知他是贱民。}

Ekajaṃ vā dvijaṃ vā pi, yo’dha pāṇaṃ vihiṃsati;\\
yassa pāṇe dayā n’atthi, taṃ jaññā vasalo iti. %\hfill\textcolor{gray}{\footnotesize 2}

\begin{enumerate}\item 现在,为制止这婆罗门的见「有时即便行杀生、不与取等,仍是婆罗门」,或者,为向具足彼彼杀害等不善法而未见过患、不舍弃\footnote{不舍弃:据 PTS 本译出,原文作「令生起」。}彼等法的有情显示「此等法低劣,是成为贱民者」及此处的过患,更以其它方法开示贱民及成为贱民之法,说了如下诸颂。
\item 这里,\textbf{一生者}即除卵生的其余胎生者,因其唯出生一次,\textbf{再生者}即卵生者,因其从母胎及卵室出生两次\footnote{一生者、再生者:据菩提比丘注 643,除了上述意义外,再生者另指前三种姓,即婆罗门等雅利安人,他们以授予圣线获得精神上的出生,而一生者则指首陀罗。案,考虑到这里是将「一生者、再生者」作为杀害的宾语,义注的解释似较合理。}。\textbf{生命},即有情。\textbf{杀害},即以身门之思等起或语门之思等起的加行夺取生命。文本也作「杀害众多生命 \textit{pāṇāni hiṃsati}」,则此处当知如是连结:若于此杀害一生或再生等类的众多生命。\textbf{若于生命无有怜悯},即是说无有此同情之心意。
\item 其余仍如前述。且在此后诸颂中,连这些(仍如前述)也不再说,此后,我们将略去意义自明的句子,只解释未曾解释的句子。\end{enumerate}

\subsection\*{\textbf{118}}

\textbf{若摧毁、围攻村、镇等,\\}
\textbf{被认为是压迫者,当知他是贱民。}

Yo hanti parirundhati, gāmāni nigamāni ca;\\
niggāhako samaññāto, taṃ jaññā vasalo iti. %\hfill\textcolor{gray}{\footnotesize 3}

\begin{enumerate}\item \textbf{摧毁},即打击、消灭。\textbf{围攻},即以军队包围后驻守。\textbf{村、镇等},此中以「等」字表示还应说「城」。\textbf{被认为是压迫者},即以此摧毁、围攻,在世间被了知为毁灭村、镇、城者。\end{enumerate}

\subsection\*{\textbf{119}}

\textbf{于村或若林野,凡他人的所属,\\}
\textbf{出于盗窃而取未给予物,当知他是贱民。}

Gāme vā yadi vāraññe, yaṃ paresaṃ mamāyitaṃ;\\
theyyā adinnam ādeti, taṃ jaññā vasalo iti. %\hfill\textcolor{gray}{\footnotesize 4}

\begin{enumerate}\item \textbf{于村或若林野},村、镇、城等一切与其近郊一起,于此均作村,除此之外,其余为林野。于此村或若林野,\textbf{凡他人的所属},即凡其他有情拥有、未舍弃的有情或行。\textbf{出于盗窃而取未给予物},即以盗心取走他人未给予、未许可物,无论以何加行、无论如何窃取,使之成为自己的所得。\end{enumerate}

\subsection\*{\textbf{120}}

\textbf{若确实借了债,当被催促时却逃赖:\\}
\textbf{「没有欠你的债」,当知他是贱民。}

Yo have iṇam ādāya, cujjamāno palāyati;\\
“na hi te iṇam atthī” ti, taṃ jaññā vasalo iti. %\hfill\textcolor{gray}{\footnotesize 5}

\begin{enumerate}\item \textbf{借了债},或是典当了自己的财产,以持有典当的方式,或是未典当,「经若干时间,我将给予若干利息」,以持有利息的方式,或是如「从中的盈利是我的,本金则是你的」或「盈利两人分享」,以持有彼彼契约的方式,拿了贷款。\textbf{当被催促时却逃赖:「没有欠你的债」},当被此债主催促「还我贷款」时,却说「没有欠你的债,谁作证我拿了」,即便在家住着也逃赖。\end{enumerate}

\subsection\*{\textbf{121}}

\textbf{若出于对某物的欲求,对路上的行人\\}
\textbf{加以伤害,取走某物,当知他是贱民。}

Yo ve kiñcikkhakamyatā, panthasmiṃ vajantaṃ janaṃ;\\
hantvā kiñcikkham ādeti, taṃ jaññā vasalo iti. %\hfill\textcolor{gray}{\footnotesize 6}

\begin{enumerate}\item \textbf{对某物的欲求},即希求任何事物,乃至琐屑者。\textbf{路上的行人},即行走在道路上的任何女人或男人。\textbf{加以伤害,取走某物},即加以谋害、打击,拿走物品。\end{enumerate}

\subsection\*{\textbf{122}}

\textbf{若人因自、因他以及因财,\\}
\textbf{作为见证而说妄语,当知他是贱民。}

Attahetu parahetu, dhanahetu ca yo naro;\\
sakkhipuṭṭho musā brūti, taṃ jaññā vasalo iti. %\hfill\textcolor{gray}{\footnotesize 7}

\begin{enumerate}\item \textbf{因自},即出于自己活命之因,\textbf{因他}也同样。\textbf{因财},即出于自己或他人的财产之因。\textbf{以及}一词在一切处是可选之义。\textbf{作为见证},即被问道「你知道什么就说什么」。\textbf{说妄语},或知而说「我不知」,或不知而说「我知」,混淆所有者与非所有者。\end{enumerate}

\subsection\*{\textbf{123}}

\textbf{若现身于亲戚或者朋友的妻妾中,\\}
\textbf{以暴力或以亲昵,当知他是贱民。}

Yo ñātīnaṃ sakhīnaṃ vā, dāresu paṭidissati;\\
sāhasā sampiyena vā, taṃ jaññā vasalo iti. %\hfill\textcolor{gray}{\footnotesize 8}

\begin{enumerate}\item \textbf{亲戚},即有关系者。\textbf{朋友},即友人。\textbf{妻妾},即他人所拥有者。\textbf{现身},即因违逆被发现,行通奸时被发现之义。\textbf{以暴力},即以用强对待不从者。\textbf{以亲昵},即为彼等妻妾所希求且自身也希求者,即是说以双方的爱执。\end{enumerate}

\subsection\*{\textbf{124}}

\textbf{若对年老、青春已逝的父母,\\}
\textbf{堪能却不赡养\footnote{此颂前三句同\textbf{衰败经}第 98 颂。},当知他是贱民。}

Yo mātaraṃ pitaraṃ vā, jiṇṇakaṃ gatayobbanaṃ;\\
pahu santo na bharati, taṃ jaññā vasalo iti. %\hfill\textcolor{gray}{\footnotesize 9}

\begin{enumerate}\item \textbf{父母},如是以慈作为近因。\textbf{年老、青春已逝},如是以悲作为近因。\textbf{堪能却不赡养},即虽然具足财富、具足资具,却不养育。\end{enumerate}

\subsection\*{\textbf{125}}

\textbf{若对父母或兄弟、姐妹、岳母\\}
\textbf{加以伤害,以言语恼害,当知他是贱民。}

Yo mātaraṃ pitaraṃ vā, bhātaraṃ bhaginiṃ sasuṃ;\\
hanti roseti vācāya, taṃ jaññā vasalo iti. %\hfill\textcolor{gray}{\footnotesize 10}

\begin{enumerate}\item \textbf{伤害},即以掌、土块或任何其它东西击打。\textbf{以言语恼害},即以恶口令其生起忿怒。\end{enumerate}

\subsection\*{\textbf{126}}

\textbf{若被问及义利,却教授非义,\\}
\textbf{以隐语商讨,当知他是贱民。}

Yo atthaṃ pucchito santo, anattham anusāsati;\\
paṭicchannena manteti, taṃ jaññā vasalo iti. %\hfill\textcolor{gray}{\footnotesize 11}

\begin{enumerate}\item \textbf{义利},即现世、来世及第一义中的任一。\textbf{教授非义},即唯对他宣说无利益者。\textbf{以隐语商讨},即便在宣说义利时,也以不显明的文词、隐晦的言语商讨,好让他无法理解,或者握紧老师的拳头\footnote{老师的拳头:这是字面的直译,即秘笈之义。}、迁延时日后,有所保留地商讨。\end{enumerate}

\subsection\*{\textbf{127}}

\textbf{若作恶业后,希望「他莫发现我」,\\}
\textbf{行事隐密,当知他是贱民。}

Yo katvā pāpakaṃ kammaṃ, “mā maṃ jaññā” ti icchati;\\
yo paṭicchannakammanto, taṃ jaññā vasalo iti. %\hfill\textcolor{gray}{\footnotesize 12}

\begin{enumerate}\item \textbf{若作}等等,是说伪善之前分的恶欲,即凡如\begin{quoting}于此,有些人以身行恶行、以语行恶行、以意行恶行后,为隐密此,愿求恶欲,希望「他莫发现我」。(分别论·杂事分别第 894 段)\end{quoting}所及者。为了不让其他人知道而如是行事,且以不揭示所作而隐密其行事,为\textbf{行事隐密}。\end{enumerate}

\subsection\*{\textbf{128}}

\textbf{若到了别人家,享用了净妙的食物,\\}
\textbf{却不敬待来者,当知他是贱民。}

Yo ve parakulaṃ gantvā, bhutvāna sucibhojanaṃ;\\
āgataṃ na ppaṭipūjeti, taṃ jaññā vasalo iti. %\hfill\textcolor{gray}{\footnotesize 13}

\begin{enumerate}\item \textbf{别人家},即亲戚家或朋友家。\textbf{来者},即在其家藉以享用者。意即\textbf{不}以饮食等\textbf{敬待}来至自己家的人,或不给予,或给予劣质的食物。\end{enumerate}

\subsection\*{\textbf{129}}

\textbf{若对婆罗门、沙门或其他的乞食者\\}
\textbf{以妄语欺骗,当知他是贱民。}

Yo brāhmaṇaṃ samaṇaṃ vā, aññaṃ vā pi vanibbakaṃ;\\
musāvādena vañceti, taṃ jaññā vasalo iti. %\hfill\textcolor{gray}{\footnotesize 14}

\begin{enumerate}\item 此颂如衰败经(第 100 颂)中所述。\end{enumerate}

\subsection\*{\textbf{130}}

\textbf{若食时已到,对婆罗门或沙门\\}
\textbf{以言语恼害,且不布施,当知他是贱民。}

Yo brāhmaṇaṃ samaṇaṃ vā, bhattakāle upaṭṭhite;\\
roseti vācā na ca deti, taṃ jaññā vasalo iti. %\hfill\textcolor{gray}{\footnotesize 15}

\begin{enumerate}\item \textbf{食时已到},即食时来临。文本也作 upaṭṭhitaṃ,即在食时的来者之义。\textbf{以言语恼害,且不布施},即不思量「他是为了我的义利,前来促使我作福德」,而以不适当的粗恶语恼害,甚至不与他见面,遑论施食的意思。\end{enumerate}

\subsection\*{\textbf{131}}

\textbf{若于此出言不善,以愚弄裹挟,\\}
\textbf{企求某物,当知他是贱民。}

Asataṃ yo’dha pabrūti, mohena paliguṇṭhito;\\
kiñcikkhaṃ nijigīsāno, taṃ jaññā vasalo iti. %\hfill\textcolor{gray}{\footnotesize 16}

\begin{enumerate}\item \textbf{若于此出言不善},即于此,当诸相现起时,他便说如是不善人的言语「某天,你会有这事那事」。文本也作 asantaṃ,即不实之义。\textbf{出言},即说。以「在某村,我有如许家产,走!我们去那里!你来做我的女主人,我会给你这般那般」诳骗他人的妻子或女仆,如无赖一般。\textbf{企求},即追求,骗了她后,拿走任何东西,想要逃跑的意思。\end{enumerate}

\subsection\*{\textbf{132}}

\textbf{若赞叹自己,且蔑视他人,\\}
\textbf{以自身的慢而下劣,当知他是贱民。}

Yo c’attānaṃ samukkaṃse, pare ca mavajānāti;\\
nihīno sena mānena, taṃ jaññā vasalo iti. %\hfill\textcolor{gray}{\footnotesize 17}

\begin{enumerate}\item \textbf{赞叹},即以出身等赞叹,置身高位。\textbf{蔑视他人},即唯以此蔑视、贬低他人。\textbf{下劣},即减损功德的增长,或至最劣的状态。\textbf{以自身的慢},即以此被称为赞毁的自身的慢。\end{enumerate}

\subsection\*{\textbf{133}}

\textbf{恼害、贪婪,恶欲、悭吝、狡诈,\\}
\textbf{无惭、无愧,当知他是贱民。}

Rosako kadariyo ca, pāpiccho maccharī saṭho;\\
ahiriko anottappī, taṃ jaññā vasalo iti. %\hfill\textcolor{gray}{\footnotesize 18}

\begin{enumerate}\item \textbf{恼害},即以身、语恼害他人者。\textbf{贪婪},即强硬的守财奴,他遮止他人给其他人布施或做其它福德。\textbf{恶欲},即希望以不实的功德而受尊敬者。\textbf{悭吝},即于住处等的悭吝\footnote{于住处等的悭吝:即于住处、家族、利养、赞叹、法等五者的悭吝,见\textbf{分别论}·杂事分别第 940 段。}的相关者。\textbf{狡诈},即具足以彰显不实的功德为相的狡诈者,或非正语者,即便不想做,还说「我来做」等语。惭以不嫌厌其恶为相,愧以对其后的恐惧不悚惧为相,以此为\textbf{无惭、无愧}。\end{enumerate}

\subsection\*{\textbf{134}}

\textbf{若谤骂佛陀或他的弟子、\\}
\textbf{游行者或在家人,当知他是贱民。}

Yo buddhaṃ paribhāsati, atha vā tassa sāvakaṃ;\\
paribbājaṃ gahaṭṭhaṃ vā, taṃ jaññā vasalo iti. %\hfill\textcolor{gray}{\footnotesize 19}

\begin{enumerate}\item \textbf{佛陀},即正等正觉者。\textbf{谤骂},即以「非一切知者」等指责,且以「恶行道者」等指责\textbf{弟子}。\textbf{游行者或在家人},此唯指特殊的弟子:或为其出家弟子,或为资具施主的在家人之义。古人还如是理解这里的意思,即「他以不实的过失谤骂外道的游行者或任何在家人」。\end{enumerate}

\subsection\*{\textbf{135}}

\textbf{若实非阿罗汉,却自称阿罗汉,\\}
\textbf{在俱梵的世间作贼,他是最下劣的贱民,\\}
\textbf{上述这些贱民,我已向你阐明。}

Yo ve anarahaṃ santo, arahaṃ paṭijānāti;\\
coro sabrahmake loke, eso kho vasalādhamo;\\
ete kho vasalā vuttā, mayā ye te pakāsitā. %\hfill\textcolor{gray}{\footnotesize 20}

\begin{enumerate}\item \textbf{非阿罗汉},即非漏尽者。\textbf{自称阿罗汉},即自称「我是阿罗汉」,像这般发言、以身作态、以心希望、忍耐,好让人们知道「他是阿罗汉」。\textbf{贼},即盗。\textbf{俱梵的世间}是以最高贵者而说,即是说一切世间。因为在世间,人们称以开锁、抢劫、监护、设障等劫掠他人的财物者为贼,而在教内,则称以集会的成就等劫掠资具等者为贼,如说:\begin{quoting}诸比丘!世间存有这五种大贼。哪五种?于此,诸比丘!有些大贼想到「我何不带领百人或千人在村镇王城处徘徊,杀戮、教人杀戮,砍斫、教人砍斫,折磨、教人折磨」,他便于后时带领百人或千人在村镇王城处徘徊,杀戮……教人折磨。如是,诸比丘!于此,有些恶比丘想到「我何不在王城处游行,受到在家众与出家众的恭敬、尊重、奉事、供养、崇拜,得到衣等利养」,他便于后时带领百人或千人在村镇王城处游行……,诸比丘!这是世间存有的第一种大贼。\\复次,诸比丘!于此,有些恶比丘遍学了如来证得的法律,归于自己,诸比丘!这是世间的第二种。\\复次,诸比丘!于此,有些恶比丘以无根据的非梵行诽谤梵行清净、行遍净之梵行者,诸比丘!这是世间的第三种。\\复次,诸比丘!于此,有些恶比丘以任何僧伽的贵重物品、贵重资具摄受、引诱在家人,如僧园、僧园之地、寺庙、寺庙之地、床、椅、坐垫、枕、铜釜、铜器、铜瓶、铜盘、刀、斧、斤、锹、凿、蔓、竹、文阇草、灯芯草、草、粘土、木器、土器等,诸比丘!这是世间的第四种。\\诸比丘!在俱有天……天人的世间,这最上的大贼,即吹嘘不存在、不实的上人法者。(律藏·波罗夷第 195 段)\end{quoting}这里,世间的贼只盗取世间的财富、谷物等。而在教内所说的贼中,第一种仅仅是这般的衣等资具,第二种是圣典之法,第三种是他人的梵行,第四种是僧伽的贵重物品,第五种则是禅那、三摩地、等至、道、果等世间、出世间的功德之财,以及世间的衣等多种资具,如说:\begin{quoting}诸比丘!你们以盗取受用王国的食物。(同上引)\end{quoting}这里,世尊就第五种大贼而说「在俱梵的世间作贼」,因为他\begin{quoting}诸比丘!在俱有天……天人的世间,这最上的大贼,即吹嘘不存在、不实的上人法者。\end{quoting}如是以盗取世间、出世间之财而说为最上的大贼,所以,这里也以「俱梵的世间」此高贵的部分来阐明之。
\item \textbf{他是最下劣的贱民},此中的 kho 是强调之义,以此强调「他确实是最下劣的贱民,所有贱民中的低劣、最下者」。为什么?由在殊胜的事上落入盗法,且由只要不放弃此自称,便不离于成为贱民之法。
\item 「这些贱民」等。现在,上述三十三或三十四种贱民如是:第一颂以欠缺意乐有忿怒等五,或者分「恶且覆藏」为二则有六,第二颂以欠缺加行有一杀生,第三颂也以欠缺加行有一压迫村镇,第四颂以窃取有一,第五颂以债务欺骗有一,第六颂以强取有一路匪,第七颂以伪证有一,第八颂以害友有一,第九颂以不知恩有一,第十颂以作损害及恼害有一,第十一颂以欺心有一,第十二颂以行事隐密有二,第十三颂以不知恩有一,第十四颂以欺骗有一,第十五颂以恼害有一,第十六颂以欺骗有一,第十七颂以自赞毁他有二,第十八颂以欠缺加行及意乐有恼害等七,第十九颂以谤骂有二,第二十颂以最上大贼有一,为说明彼等而说「\textbf{上述这些贱民,我已向你阐明}」。
\item 其义为:我先前以「婆罗门!那你知道贱民……」略说的贱民,已详细地向你阐明。或者,凡是以人所说的,也以法向你阐明。或者,这些由圣者依业而非依出身所说的贱民,我已经以「忿怒、怨恨」等方法向你阐明。\end{enumerate}

\subsection\*{\textbf{136}}

\textbf{不由出生而成贱民,不由出生而成婆罗门,\\}
\textbf{由业而成贱民,由业而成婆罗门。}

Na jaccā vasalo hoti, na jaccā hoti brāhmaṇo;\\
kammunā vasalo hoti, kammunā hoti brāhmaṇo. %\hfill\textcolor{gray}{\footnotesize 21}

\begin{enumerate}\item 如是,世尊在显示了贱民后,现在,因为婆罗门极度执著于有身见,所以为遮止此见而说此颂。其义为:因为从第一义来说,\textbf{不由出生而成贱民,不由出生而成婆罗门},而是\textbf{由业而成贱民,由业而成婆罗门},由不遍净业的落入而成贱民,由以遍净的业排除不遍净而成婆罗门。或者,因为你们认为低劣者为贱民,高贵者为婆罗门,所以以低劣的业而成贱民,以高贵的业而成婆罗门,如是,为令了知此义而如是说。\end{enumerate}

\subsection\*{\textbf{137}}

\textbf{你们也可以此了知,好比我的这例子:\\}
\textbf{旃陀罗之子、贱民,以摩登伽著名者。}

Tad aminā pi jānātha, yathā me’daṃ nidassanaṃ;\\
caṇḍālaputto sopāko, Mātaṅgo iti vissuto. %\hfill\textcolor{gray}{\footnotesize 22}

\begin{enumerate}\item 现在,为了以例子成就此义,说了以下三颂,其中二颂各四句、一为六句。其义为:我所说的「不由出生而成贱民」等,\textbf{你们也可以此了知,好比我的这例子},你们也可以此方式了知,即是说以我的方式、以普遍而举此例。那么是什么例子?即「旃陀罗之子、贱民……投生到梵界」。
\item 为了自己吃而获取死狗并烹煮者为\textbf{贱民}\footnote{烹煮狗 \textit{sunakhe pacati}:这是从语源上解释「贱民 \textit{sopāka}」。}。\textbf{摩登伽},即如是之名\footnote{摩登伽:杂阿含作「须陀夷」,别译杂阿含作「须陀延」。}。\textbf{著名},即如是以低贱的出身、活命及名字而扬名。\end{enumerate}

\subsection\*{\textbf{138}}

\textbf{这摩登伽获得了极难得的至高的声誉,\\}
\textbf{许多刹帝利、婆罗门前往侍奉他。}

So yasaṃ paramaṃ patto, Mātaṅgo yaṃ sudullabhaṃ;\\
āgacchuṃ tass’upaṭṭhānaṃ, khattiyā brāhmaṇā bahū. %\hfill\textcolor{gray}{\footnotesize 23}

\begin{enumerate}\item 以\textbf{这}与前句相连,这摩登伽\textbf{获得了至高的声誉},获得了希有、最高、极其殊胜的声誉、称誉、赞叹。\textbf{极难得},即便以投生至高贵的家族也难得,而以投生至低贱的家族则极难得。且对获得了如是的声誉者,\textbf{许多刹帝利、婆罗门前往侍奉他},为敬事这摩登伽,刹帝利、婆罗门及其他许多吠舍、首陀罗等阎浮提人大多前来侍奉他之义。\end{enumerate}

\subsection\*{\textbf{139}}

\textbf{他登上了天乘、离尘的大路,\\}
\textbf{弃绝了对爱欲的贪染,便至梵界,\\}
\textbf{出身不能遮止他投生到梵界。}

Devayānaṃ abhiruyha, virajaṃ so mahāpathaṃ;\\
kāmarāgaṃ virājetvā, brahmalokūpago ahu;\\
na naṃ jāti nivāresi, brahmalokūpapattiyā. %\hfill\textcolor{gray}{\footnotesize 24}

\begin{enumerate}\item 如是,\textbf{他},具足侍奉的摩登伽,\textbf{登上了}由离去烦恼尘垢而为\textbf{离尘}、由为佛陀等大人所行道而为\textbf{大路}、由堪能在被称为梵界的天界存续而名为\textbf{天}界之\textbf{乘}的八等至之乘,以此行道,\textbf{弃绝了对爱欲的贪染},身坏后\textbf{便至梵界},这如是低贱的\textbf{出身不能遮止他投生到梵界}。
\item 而其义当知如是:据说\footnote{摩登伽本生,见\textbf{本生}第 15:1~23 颂。},在过去,当大人\footnote{大人:即菩萨。}以彼彼方法利益众生时,投生于以贱民活命的旃陀罗家。他名为摩登伽,容貌丑陋,在城外毛皮的棚屋居住,在城内乞食营生。于是某天,当此城酒节来临时,无赖们便与各自的随从嬉戏。某个婆罗门大财主的女儿,到了十五、十六岁的年纪,如天女一般,容貌可观而明净,想「我将按适合自己家族世系的方式来嬉戏」,把许多硬食、软食等嬉戏的用具装了几大车,上了通体白色的牝马驾驭的车乘,与大队随从去往园林之处。她名为「\textbf{见吉祥}」,据说,她不希望看见「形状丑陋、不祥之色」,因此她便得称为见吉祥。
\item 那时,这摩登伽刚好起来,著了布条的下衣,在手上绑了铜锣,器皿在手,进入城中,远远地一看见人群,就敲打起铜锣。见吉祥则被喊着「让开、让开」驱赶着前方下人的人们引领着,在城门中见到了摩登伽,便说:「这是谁?」「我是摩登伽旃陀罗。」她想「见到这样的人,去了又有什么好处」,便命人掉转车乘。人们激愤道「我们到了园林,能得到硬食、软食等,摩登伽对我们从中作梗」,便喊「捉住旃陀罗」,拿土块砸,以为死了,便提起脚扔到一边,用垃圾盖上后走开了。
\item 他苏醒后起来,便问人们:「先生们!这门是所有人共用的,还是只为婆罗门而建的?」人们便说:「对所有人共用的。」「如是,我进入为所有人共用的门,为生存乞食,见吉祥的人们便将这厄难带给我」,便徘徊于街道,向人们控诉,在婆罗门的家门口躺倒:「得不到见吉祥,我就不起!」
\item 婆罗门听说「摩登伽躺在门口」,便说:「给他一个硬币\footnote{硬币 \textit{kākaṇikā} 及下文的钱 \textit{māsaka}、钱币 \textit{kahāpaṇa} 都是货币单位,其中一钱币等于二十钱。}!用油涂了四肢,让他走!」他不要这些,仍说:「得不到见吉祥,我就不起!」随后,婆罗门说:「给两个硬币!一个硬币吃饼,一个硬币用油涂了四肢,让他走!」他不要这些,仍那样说。婆罗门听了,便命令「给一钱、五钱、十钱、一个钱币、二、三个」乃至一百钱币。他不要这些,仍那样说。如是请求时,太阳已落山。于是,婆罗门尼从楼阁下来,让人围好屏风,走向他后,请求道「亲爱的摩登伽!请宽恕见吉祥的罪过!拿着一千钱币!二千、三千」直至「拿着十万」。他仍默然躺着。
\item 如是便过了四五天,送了很多礼物也没得到见吉祥的刹帝利童子等人,让人在耳边告诉摩登伽:「所谓男子,就是经多年努力而圆满所求者,你别躺着了!事实上,再过二三天,你就会得到见吉祥!」他仍默然躺着。
\item 于是,在第七天,四周的邻居们便都出来说:「你们要么让摩登伽起来,要么给女儿,别毁了我们大家!」据说,他们持有此见「若在谁家门口这样躺着的旃陀罗死了,他家与四周的各七家住户都将成为旃陀罗」。
\item 随后,他们便让悲泣着的见吉祥著了蓝布条的下衣,给了勺和锅等,领到他跟前,给道:「带上女孩,起来走吧!」她站在一侧后便说:「请起!」他说:「用手扶我,帮我起来!」她便帮他起来。他坐下后说:「我们不能住在城内,走!带我到城外毛皮的棚屋!」她扶着他的手,便带到那里——本生诵者则说是「背在背上」。且带到后,为他用油涂了身体,用热水洗了澡,煮了粥后端给他。
\item 他想「莫毁了这婆罗门女孩」,便未混杂血统,仅半个月就恢复了体力,说「我去林间,你莫以『迁延太久』而期盼」,且命令家人们「别怠慢她」,即从家离开,出家为苦行者,在做了遍的预作后,没几天就令八等至及五神通生起,想「现在,我将对见吉祥变得中意」,从空中前往,在城门降落后,便派人到见吉祥的跟前。
\item 她听后,想着「我的某个亲戚出家人,我想,知道了我的痛苦,前来见我」而往,认出他后,投于双足便说:「你为何陷我于无祜?」大人说「见吉祥!你别悲伤!我要让整个阎浮提的人们都恭敬你」,并说:「去!你去宣扬『我的丈夫是大梵,不是摩登伽,他将在第七天劈开月宫,来到我的跟前』。」她说:「尊者!我曾是婆罗门大财主之女,以自身的恶业得此旃陀罗的身份,不能如是说。」大人说「你不知道摩登伽的威力」,为让她相信而显示了种种神变,仍这般命令她后,回到了自己的住处。她便照做了。
\item 人们讥嫌且大笑:「以自身的恶业成了旃陀罗,她又要怎样对待这大梵?」她却非常自信,每天徘徊于城中,宣扬着「从此第六天……第五……第四……第三……明天……今天,他就会来」。人们听了她自信的话语,想「到时也许会这样」,便在各自的门口让人搭了亭子,准备了屏风,装扮了正值青春的女孩「当大梵来时,我们将布施少女」,抬头望天而坐。
\item 于是,大人在满月的这天,当月亮在空中盈满时,便劈开月宫,在大众的瞩目下,以大梵的形象出现。大众想「变成两个月亮了」,随后看见他渐渐到来,便断言「见吉祥说的是真的,是大梵为了调御这见吉祥,先前以摩登伽出现」。如是,他在大众瞩目下,降落到见吉祥的住处。她那时正来月经,他便用拇指摩挲了她的肚脐,胎儿便以此触而住。随后,他对她说「你已怀胎,依这出生的孩子活命吧」,在大众的瞩目下,再次进入月宫。
\item 众婆罗门说「见吉祥是大梵的夫人,就是我们的生母」,便从各处而来。各处城门都被想要来恭敬她的人挤得水泄不通。他们将见吉祥置于金坛,澡浴、装扮后再送上车,以大恭敬载之绕城,并在城中搭了亭子,把她作为「大梵的夫人」置于高洁之处,让她居住:「直到盖好适合她的住所,就让她住在这里!」
\item 她便在亭中分娩了孩子。在清净之日,他们教人将她与孩子一起沐浴,因生于亭中,便称男孩之名为「\textbf{亭童子}」。且此后,众婆罗门便以「大梵之子」追随他而行。随后,数十万种礼物纷至沓来,那些婆罗门便为童子提供保护,来者不得轻易见到童子。
\item 童子渐渐长大,开始布施。他不向来到(布施)堂的穷苦旅人布施,唯布施给婆罗门。大人经转向「我的孩子布施给谁」,见到唯布施给婆罗门后,想「我要让他布施给所有人」,穿了衣、拿了钵,从空中前往,站在孩子的家门外。童子见到他,忿怒道「这衣装丑陋的贱民是从哪里来的」,说了此颂:\begin{quoting}你从哪里来?穿著简陋、贫贱的泥鬼,\\颈上缠着垃圾布,咄!你是谁,不值得供养者?\end{quoting}众婆罗门喊着「捉住、捉住」,捉了便打,让他受了厄难。他去到空中,住于城外。天人大怒,扼住童子的咽喉,脚朝上、头朝下地倒悬。他眼珠暴突,口吐白沫,扑哧扑哧呼着气,倍感痛苦。见吉祥听后,便问:「是不是有人来过?」「唯!有个出家人来过。」「去了哪里?」「如是而去。」她便去到那里,请求道:「尊者!请你饶恕自己的奴仆!」便投于他脚边的地上。
\item 尔时,大人乞食得了粥,正坐在那里喝粥,他便把剩下的一些粥给了见吉祥:「去!把这粥掺在水瓶里,洒在变形的眼、口、耳、鼻孔处,再喷在身上,这样就会变回来。」她便照做了。随后,当童子的身体恢复如初,说「来!亲爱的亭!我们去请求宽恕」,便让孩子和所有婆罗门倾倒在其足下,匍匐已,便请宽恕。
\item 他教诫道「应布施给所有人」,说了法论,仍去到自己的住处,思量:「女子之中,见吉祥已受调御而著称,男子之中,亭童子已著称,现在该调御谁?」
\item 随后,他看到苦行者「\textbf{生咒}」依止有亲城而住在有瓶河的岸边。他以「我出生殊胜,不可受用他人用过的水」,便在上游居住。大人在他的上游安置了住处,到用水的时间,便嚼了齿木,扔进水里。苦行者见到齿木随水漂流,想「这是谁扔的」,溯流而上,见到大人后,便说:「谁在此处?」「摩登伽旃陀罗,老师!」「离开!旃陀罗!不许住在上游!」大人说「善哉!老师」,便住在下游,齿木却仍逆流到了苦行者的跟前。苦行者又来说:「离开!旃陀罗!不许住在下游!仍去上游住!」大人说「善哉!老师」,便照做,情况却仍照旧。苦行者再次嗔怒「这样做」,诅咒大人说:「太阳升起时,让你的头裂成七分!」大人说「善哉!老师!但我不会让太阳升起」,便阻止了日出。
\item 随后,夜不转明,黑暗生起,恐惧的有亲城居民来到苦行者的跟前,问道:「老师!我们是否平安?」因为他们认为他是「阿罗汉」。他便向他们告知了一切。他们来到大人处,请求道:「尊者!请释放太阳!」大人说:「如果你们的阿罗汉来请我宽恕,我就放。」人们去到苦行者处说:「去!尊者!请摩登伽智者宽恕!别因你们的诤论,让我们毁灭!」他说:「我不去请求旃陀罗的宽恕!」人们说「你要让我们毁灭」,便捉了他的手脚,带到大人的跟前。
\item 大人说:「把肚子投于我的脚下,请我宽恕,我就宽恕。」人们说:「照做!」苦行者说:「我不顶礼旃陀罗。」人们说「按你的意愿,你不会顶礼」,便按住手脚、胡子、脖子等,让他在大人脚下卧倒。他说:「我宽恕他!而且正是出于对他的慈悲,我不会释放太阳,因为当太阳一升起,他的头将会裂成七分。」人们说:「尊者!现在该怎么办?」大人说:「那么,用水浸到他的咽喉,用粘土块盖住他的头,当阳光射到时,粘土块会裂成七分,裂开后,让他去别的地方!」他们便捉了苦行者的手脚照做。当太阳一被放出,粘土块便裂成七分而掉落,苦行者恐惧而逃。人们见后说「先生们!你们看沙门的威力」,便从丢齿木开始详说了全部,并对他净喜道:「再没有这样的沙门了!」
\item 从此以后,整个阎浮提的刹帝利、婆罗门、在家人、出家人等都前来侍奉摩登伽智者。他在寿尽身坏后,便投生到梵界。因此世尊说「你们也可以此了知……投生到梵界」。\end{enumerate}

\subsection\*{\textbf{140}}

\textbf{生于唱诵者之家、精通经典的婆罗门,\\}
\textbf{他们却常在众多恶业中现身。}

Ajjhāyakakule jātā, brāhmaṇā mantabandhavā;\\
te ca pāpesu kammesu, abhiṇham upadissare. %\hfill\textcolor{gray}{\footnotesize 25}

\begin{enumerate}\item 如是,在成就「不由出生而成贱民,由业而成贱民」后,现在,为成就「不由出生而成婆罗门,由业而成婆罗门」,说了以下二颂。
\item 这里,\textbf{生于唱诵者之家},即生于唱诵经典的婆罗门之家。文本也作 Ajjhāyakā kule jātā,即唱诵经典且生于无瑕疵的婆罗门之家之义。以经典为其眷属者为\textbf{精通经典},即是说为吠陀的眷属、以吠陀为皈依。
\item \textbf{他们却常在众多恶业中现身},即他们虽然生于如是家族且精通经典,仍然屡屡现身于杀生等恶业中。\end{enumerate}

\subsection\*{\textbf{141}}

\textbf{在现法即遭谴责,且在来世堕恶趣,\\}
\textbf{出身不能遮止他们堕恶趣或遭谴责。}

Diṭṭhe va dhamme gārayhā, samparāye ca duggati;\\
na ne jāti nivāreti, duggatyā garahāya vā. %\hfill\textcolor{gray}{\footnotesize 26}

\begin{enumerate}\item 于是,\textbf{在现法即遭谴责,且在来世堕恶趣},当他们如是现身时,即于此自体中,甚至受到父母的谴责「他们不是我们的孩子,这些恶生者是家族的炭种,应驱逐之」,受到众婆罗门的谴责「他们是长者,他们不是婆罗门,莫让他们参与追悼、献牲、笾豆等\footnote{追悼、献牲、笾豆等婆罗门的仪式,见\textbf{长部}·阿摩昼经。PED 对「笾豆 \textit{thālipāka}」的解释是一种在牛奶中煮大麦或米的供养,thāli 的本义也是一种器皿,故试作此译。},莫与他们交谈」,还受到其他众人的谴责「他们是恶业者,他们不是婆罗门」。且他们在来世堕地狱等类的恶趣,即他们在别的世间堕恶趣之义。文本也作「或在来世」。他们在别的世间的苦之趣为\textbf{恶趣},即唯至于苦之义。
\item \textbf{出身不能遮止他们堕恶趣或遭谴责},即你所从出的出身,即便如此高贵,也不能遮止这些现身于恶业中的婆罗门们如「且在来世堕恶趣」中所说的堕恶趣,或如「在现法即遭谴责」中所说的遭谴责。\end{enumerate}

\subsection\*{\textbf{142}}

\textbf{不由出生而成贱民,不由出生而成婆罗门,\\}
\textbf{由业而成贱民,由业而成婆罗门。}

Na jaccā vasalo hoti, na jaccā hoti brāhmaṇo;\\
kammunā vasalo hoti, kammunā hoti brāhmaṇo” ti. %\hfill\textcolor{gray}{\footnotesize 27}

\begin{enumerate}\item 如是,世尊显明了即便对生于唱诵者之家的婆罗门,由可谴责等业而于现法中沉沦的状态,并显明了由趣于恶趣,而在来世不得生为婆罗门,成就了「不由出生而成婆罗门,由业而成婆罗门」之义,现在,为总结两重意思而说:如是,婆罗门!\begin{quoting}不由出生而成贱民,不由出生而成婆罗门,\\由业而成贱民,由业而成婆罗门。\end{quoting}\end{enumerate}

\textbf{如是说已,事火婆罗豆婆遮婆罗门对世尊说:「希有!乔达摩君!……从今起,尽寿命,请乔达摩君受持我皈依!」}

Evaṃ vutte Aggikabhāradvājo brāhmaṇo Bhagavantaṃ etad avoca: “abhikkantaṃ, bho Gotama…pe… upāsakaṃ maṃ bhavaṃ Gotamo dhāretu ajjatagge pāṇupetaṃ saraṇaṃ gatan” ti.

\begin{enumerate}\item 其余则如耕田婆罗豆婆遮经中所述。或者个别地,此中以「能扶正被倾倒的」开始的部分当如是连结:好比有人\textbf{能扶正被倾倒的},如是以令忽略业而落入出身论的我从「以出身而为婆罗门、贱民」之见出起,好比\textbf{能揭示被遮蔽的},如是以揭示被出身论遮蔽的业论,好比\textbf{能给迷者指路},如是以给婆罗门与贱民性指以不混的直路,好比\textbf{能在黑暗中持油灯},如是以持以摩登伽之例的灯,由以这些方法的阐明,\textbf{乔达摩君}为我\textbf{以种种方法阐明法}。\end{enumerate}

\begin{center}\vspace{1em}贱民经第七\\Vasalasuttaṃ sattamaṃ.\end{center}

%\begin{flushright}癸卯十二月廿六二稿\end{flushright}