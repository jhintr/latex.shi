\chapter{術解第二十}

\subsection*{1}

\textbf{荀勖善解音聲,時論謂之闇解,遂調律呂、正雅樂,每至正會,殿庭作樂,自調宮商,無不諧韻,阮咸妙賞,時謂神解,每公會作樂,而心謂之不調,既無一言直勖,意忌之,遂出阮為始平太守,後有一田父耕於野,得周時玉尺,便是天下正尺,荀試以校己所治鐘鼓、金石、絲竹,皆覺短一黍,於是伏阮神識。}{\footnotesize \textbf{晉後略}曰鐘律之器,自周之未廢,而漢成、哀之間,諸儒修而治之,至後漢末,復隳矣,魏氏使協律知音者杜夔造之,不能考之典禮,徒依于時絲管之聲、時之尺寸而制之,甚乖失禮度,於是世祖命中書監荀勖依典制,定鐘律,既鑄律管,募求古器,得周時玉律數枚,比之不差,又諸郡舍倉庫,或有漢時故鐘,以律命之,皆不叩而應,聲響韻合,又若俱成。\textbf{晉諸公贊}曰律成,散騎侍郎阮咸謂「勖所造聲高,高則悲,夫『亡國之音哀以思,其民困』,今聲不合雅,懼非德政中和之音,必是古今尺有長短所致,然今鐘磬是魏時杜夔所造,不與勖律相應,音聲舒雅,而久不知夔所造,時人為之,不足改易」,勖性自矜,乃因事左遷咸為始平太守,而病卒,後得地中古銅尺,校度勖今尺,短四分,方明咸果解音,然無能正者。\textbf{干寶晉紀}曰荀勖始造正德、大象之舞,以魏杜夔所制律呂,校大樂本音不和,後漢至魏尺,長於古四分有餘,而夔據之,是以失韻,乃依周禮,積粟以起度量,以度古器,符于本銘,遂以為式,用之郊廟。}

\subsection*{2}

\textbf{荀勖嘗在晉武帝坐上食筍進飯,謂在坐人曰:「此是勞薪炊也。」坐者未之信,密遣問之,實用故車腳。}

\subsection*{3}

\textbf{人有相羊祜父墓,後應出受命君,祜惡其言,遂掘斷墓後,以壞其勢,相者立視之曰:「猶應出折臂三公。」俄而祜墜馬折臂,位果至公。}{\footnotesize \textbf{幽明錄}曰羊祜工騎乘,有一兒五六歲,端明可喜,掘墓之後,兒即亡,羊時為襄陽都督,因盤馬落地,遂折臂,于時士林咸歎其忠誠。}

\subsection*{4}

\textbf{王武子善解馬性,嘗乘一馬,著連錢障泥,前有水,終日不肯渡,王云:「此必是惜障泥。」使人解去,便徑渡。}{\footnotesize \textbf{語林}曰武子性愛馬,亦甚別之,故杜預道「王武子有馬癖,和長輿有錢癖」,武帝問杜預「卿有何癖」,對曰「臣有左傳癖」。}

\subsection*{5}

\textbf{陳述為大將軍掾,甚見愛重,及亡,郭璞往哭之,甚哀,乃呼曰:「嗣祖!焉知非福。」俄而大將軍作亂,如其所言。}{\footnotesize \textbf{陳氏譜}曰述,字嗣祖,潁川許昌人,有美名。}

\subsection*{6}

\textbf{晉明帝解占冢宅,聞郭璞為人葬,帝微服往看,因問主人:「何以葬龍角?此法當滅族。」主人曰:「郭云『此葬龍耳,不出三年,當致天子』。」帝問:「為是出天子邪?」答曰:「非出天子,能致天子問耳。」}{\footnotesize \textbf{青鳥子相冢書}曰葬龍之角,暴富貴,後當滅門。}

\subsection*{7}

\textbf{郭景純過江,居于暨陽,墓去水不盈百步,時人以為近水,景純曰:「將當為陸。」}{\footnotesize \textbf{璞別傳}曰璞少好經術,明解卜筮,永嘉中,海內將亂,璞投策歎曰「黔黎將同異類矣」,便結親暱十餘家,南渡江,居於暨陽。}\textbf{今沙漲,去墓數十里皆為桑田,其詩曰:「北阜烈烈,巨海混混,壘壘三墳,唯母與昆。」}

\subsection*{8}

\textbf{王丞相令郭璞試作一卦,卦成,郭意色甚惡,云:「公有震厄。」王問:「有可消伏理不?」郭曰:「命駕西出數里,得一柏樹,截斷如公長,置牀上常寢處,災可消矣。」王從其語,數日中,果震柏粉碎,子弟皆稱慶,}{\footnotesize \textbf{王隱晉書}曰璞消災轉禍,扶厄擇勝,時人咸言京、管不及。}\textbf{大將軍云:「君乃復委罪於樹木。」}

\subsection*{9}

\textbf{桓公有主簿善別酒,有酒輒令先嘗,好者謂「青州從事」,惡者謂「平原督郵」,青州有齊郡,平原有鬲縣,「從事」言到臍,「督郵」言在鬲上住。}

\subsection*{10}

\textbf{郗愔信道甚精勤,常患腹內惡,諸醫不可療,聞于法開有名,往迎之,既來,便脈云:「君侯所患,正是精進太過所致耳。」合一劑湯與之,一服,即大下,去數段許紙如拳大,剖看,乃先所服符也。}{\footnotesize \textbf{晉書}曰法開善醫術,嘗行,莫投主人,妻產,而兒積日不墮,法開曰「此易治耳」,殺一肥羊,食十餘臠而針之,須臾兒下,羊膋裹兒出,其精妙如此。}

\subsection*{11}

\textbf{殷中軍妙解經脈,中年都廢,有常所給使,忽叩頭流血,浩問其故,云:「有死事,終不可說。」詰問良久,乃云:「小人母年垂百歲,抱疾來久,若蒙官一脈,便有活理,訖就屠戮無恨。」浩感其至性,遂令舁來,為診脈處方,始服一劑湯便愈,於是悉焚經方。}