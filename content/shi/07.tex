\chapter{鄭緇衣詁訓傳第七}

\begin{quoting}\textbf{釋文}鄭者國名,周宣王母弟桓公友所封也,其地詩譜云「宗周圻內咸林之地,今京兆鄭縣是其都也」,漢書地理志云「京兆鄭縣,周宣王弟鄭桓公邑」是也,至桓公之子武公滑突隨平王東遷,遂滅虢鄶而居之,即「史伯所云十邑之地,右洛左濟,前華後河,食溱洧焉,今河南新鄭是也」,在滎陽宛陵縣西南。\end{quoting}

\section{緇衣}

%{\footnotesize 三章、章四句}

\textbf{緇衣,美武公也。父子並為周司徒,善於其職,國人宜之,故美其德,以明有國善善之功焉。}{\footnotesize 父,謂武公父桓公也。司徒之職掌十二教。善善者,治之有功也。鄭國之人皆謂桓公、武公居司徒之官正得其宜。}

\textbf{緇衣之宜兮,敝、予又改為兮。}{\footnotesize 緇,黑色,卿士聽朝之正服也。改,更也。有德君子宜世居卿士之位焉。箋云緇衣者,居私朝之服也,天子之朝服皮弁服也。}\textbf{適子之館兮,還、予授子之粲兮。}{\footnotesize 適之、館舍、粲餐也。諸侯入為天子卿士,受采祿。箋云卿士所之之館在天子之宮,如今之諸廬也,自館還在采地之都,我則設餐以授之,愛之,欲飲食之也。}

\begin{quoting}還,同旋。粲,新衣也,小雅大東「粲粲衣服」,毛傳「粲,鮮盛貌」。\end{quoting}

\textbf{緇衣之好兮,敝、予又改造兮。}{\footnotesize 好,猶宜也。箋云造,為也。}\textbf{適子之館兮,還、予授子之粲兮。}

\textbf{緇衣之蓆兮,敝、予又改作兮。}{\footnotesize 蓆,大也。箋云作,為也。}\textbf{適子之館兮,還、予授子之粲兮。}

\section{將仲子}

%{\footnotesize 三章、章八句}

\textbf{將仲子,刺莊公也。不勝其母,以害其弟,弟叔失道而公弗制,祭仲諫而公弗聽,小不忍以致大亂焉。}{\footnotesize 莊公之母,謂武姜,生莊公及弟叔段,段好勇而無禮,公不早為之所而使驕慢。}

\textbf{將仲子兮,無踰我里,無折我樹杞。}{\footnotesize 將,請也。仲子,祭仲也。踰越、里居也。二十五家為里。杞,木名也。折,言傷害也。箋云祭仲驟諫,莊公不能用其言,故言請固距之,無踰我里,喻言無干我親戚也,無折我樹杞,喻言無傷害我兄弟也。仲初諫曰「君將與之,臣請事之,君若不與,臣請除之」。}\textbf{豈敢愛之,畏我父母。}{\footnotesize 箋云段將為害,我豈敢愛之而不誅與,以父母之故,故不為也。}\textbf{仲可懷也,父母之言,亦可畏也。}{\footnotesize 箋云懷私曰懷。言仲子之言可私懷也,我迫於父母有言,不得從也。}

\textbf{將仲子兮,無踰我牆,無折我樹桑。}{\footnotesize 牆,垣也。桑,木之眾也。}\textbf{豈敢愛之,畏我諸兄。}{\footnotesize 諸兄,公族。}\textbf{仲可懷也,諸兄之言,亦可畏也。}

\textbf{將仲子兮,無踰我園,無折我樹檀。}{\footnotesize 園,所以樹木。檀,彊忍之木。}\textbf{豈敢愛之,畏人之多言。仲可懷也,人之多言,亦可畏也。}

\begin{quoting}\textbf{徐常吉}傳說彙纂曰由踰里而牆而園,仲之來也以漸而迫也,由父母而諸兄而眾人,女之畏也以漸而遠也。\end{quoting}

\section{叔于田}

%{\footnotesize 三章、章五句}

\textbf{叔于田,刺莊公也。叔處于京,繕甲治兵,以出于田,國人說而歸之。}{\footnotesize 繕之言善也。甲,鎧也。}

\textbf{叔于田,巷無居人。}{\footnotesize 叔,大叔段也。田,取禽也。巷,里塗也。箋云叔往田,國人注心于叔,似如無人處。}\textbf{豈無居人,不如叔也,洵美且仁。}{\footnotesize 箋云洵,信也。言叔信美好而又仁。}

\begin{quoting}說文「巷,里中道」。\end{quoting}

\textbf{叔于狩,巷無飲酒。}{\footnotesize 冬獵曰狩。箋云飲酒,謂燕飲也。}\textbf{豈無飲酒,不如叔也,洵美且好。}

\textbf{叔適野,巷無服馬。}{\footnotesize 箋云適,之也。郊外曰野。服馬,猶乘馬也。}\textbf{豈無服馬,不如叔也,洵美且武。}{\footnotesize 箋云武,有武節。}

\begin{quoting}\textbf{馬瑞辰}服者,犕之假借,易繫辭「服牛乘馬」,說文引作「犕牛乘馬」,玉篇「犕,猶服也,以鞍裝馬也」。\end{quoting}

\section{大叔于田}

%{\footnotesize 三章、章十句}

\textbf{大叔于田,刺莊公也。叔多才而好勇,不義而得眾也。}

\begin{quoting}釋文「叔于田,本或作大叔于田者誤」。\textbf{嚴粲}詩緝曰短篇者止曰叔于田,長篇者加大為別。\end{quoting}

\textbf{大叔于田,乘乘馬。}{\footnotesize 叔之從公田也。}\textbf{執轡如組,兩驂如舞。}{\footnotesize 驂之與服和諧中節。箋云如組者,如織組之為也。在旁曰驂。}\textbf{叔在藪,火烈具舉。}{\footnotesize 藪,澤,禽之府也。烈列、具俱也。箋云列人持火俱舉,言眾同心。}\textbf{襢裼暴虎,獻于公所。}{\footnotesize 襢裼,肉袒也。暴虎,空手以搏之。箋云獻于公所,進於君也。}\textbf{將叔無狃,戒其傷女。}{\footnotesize 狃,習也。箋云狃,復也。請叔無復者,愛也。}

\begin{quoting}孔疏「鄭有圃田,此言在藪,蓋圃田也」。烈,魯詩作列,迾古字,說文「迾,遮也」,火迾者,放火燒草以遮禽獸去路也。襢,齊詩韓詩作膻。\end{quoting}

\textbf{叔于田,乘乘黃。}{\footnotesize 四馬皆黃。}\textbf{兩服上襄,兩驂鴈行。}{\footnotesize 箋云兩服,中央夾轅者。襄,駕也。上駕者,言為眾馬之最良也。鴈行者,言與中服相次序。}\textbf{叔在藪,火烈具揚。}{\footnotesize 揚,揚光也。}\textbf{叔善射忌,又良御忌。}{\footnotesize 忌,辭也。箋云良,亦善也。忌,讀如彼己之子之己。}\textbf{抑磬控忌,抑縱送忌。}{\footnotesize 騁馬曰磬,止馬曰控。發矢曰縱,從禽曰送。}

\begin{quoting}襄,同驤,說文「驤,馬之低仰也」。抑,發語詞。\end{quoting}

\textbf{叔于田,乘乘鴇。}{\footnotesize 驪白雜毛曰鴇。}\textbf{兩服齊首,}{\footnotesize 馬首齊也。}\textbf{兩驂如手。}{\footnotesize 進止如御者之手。箋云如人左右手之相佐助也。}\textbf{叔在藪,火烈具阜。}{\footnotesize 阜,盛也。}\textbf{叔馬慢忌,叔發罕忌。}{\footnotesize 慢遲、罕希也。箋云田事且畢則其馬行遲、發矢希。}\textbf{抑釋掤忌,抑鬯弓忌。}{\footnotesize 掤,所以覆矢。鬯弓,弢弓。箋云射者蓋矢弢弓,言田事畢。}

\begin{quoting}掤 \texttt{bīng},箭筩蓋。\end{quoting}

\section{清人}

%{\footnotesize 三章、章四句}

\textbf{清人,刺文公也。高克好利而不顧其君,文公惡而欲遠之,不能使高克將兵而禦狄于竟,陳其師旅,翱翔河上,久而不召,眾散而歸,高克奔陳。公子素惡高克進之不以禮,文公退之不以道,危國亡師之本,故作是詩也。}{\footnotesize 好利不顧其君,注心於利也。禦狄于竟,時狄侵衛。}

\textbf{清人在彭,駟介旁旁。}{\footnotesize 清,邑也。彭,衛之河上,鄭之郊也。介,甲也。箋云清者,高克所帥眾之邑也。駟,四馬也。}\textbf{二矛重英,河上乎翱翔。}{\footnotesize 重英,矛有英飾也。箋云二矛,酋矛夷矛也,各有畫飾。}

\begin{quoting}左傳僖二十八年杜注「駟介,四馬被甲也」。旁旁,三家詩作騯騯,說文「騯騯,馬盛貌」。\end{quoting}

\textbf{清人在消,駟介麃麃。}{\footnotesize 消,河上地也。麃麃,武貌。}\textbf{二矛重喬,河上乎逍遙。}{\footnotesize 重喬,累荷也。箋云喬,矛矜近上及室題,所以縣毛羽也。}

\begin{quoting}喬,韓詩作鷮。文選南都賦注引韓詩「逍遙,遊也」。\end{quoting}

\textbf{清人在軸,駟介陶陶。}{\footnotesize 軸,河上地也。陶陶,驅馳之貌。}\textbf{左旋右抽,中軍作好。}{\footnotesize 左旋講兵,右抽抽矢以射,居軍中為容好。箋云左,左人,謂御者。右,車右也。中軍,謂將也。高克之為將,久不得歸日,使其御者習旋車,車右抽刃,自居中央,為軍之容好而已。兵車之法,將居鼓下,故御者在左。}

\begin{quoting}抽,三家詩作搯,說文「搯者,拔兵刃以習擊刺,詩曰左旋右搯」。\end{quoting}

\section{羔裘}

%{\footnotesize 三章、章四句}

\textbf{羔裘,刺朝也。言古之君子,以風其朝焉。}{\footnotesize 言,猶道也。鄭自莊公而賢者陵遲,朝無忠正之臣,故刺之。}

\textbf{羔裘如濡,洵直且侯。}{\footnotesize 如濡,潤澤也。洵均、侯君也。箋云緇衣、羔裘,諸侯之朝服也,言古朝廷之臣皆忠直且君也。君者,言正其衣冠,尊其瞻視,儼然人望而畏之。}\textbf{彼其之子,舍命不渝。}{\footnotesize 渝,變也。箋云舍,猶處也。之子,是子也。是子處命不變,謂守死善道,見危授命之等。}

\begin{quoting}洵,韓詩作恂。左傳昭元年「楚公子美矣君哉」,則古字訓君者多有美義。\end{quoting}

\textbf{羔裘豹飾,孔武有力。}{\footnotesize 豹飾,緣以豹皮也。孔,甚也。}\textbf{彼其之子,邦之司直。}{\footnotesize 司,主也。}

\begin{quoting}\textbf{馬瑞辰}上章云「洵直且侯」是君子之處己以直,此章「邦之司直」是言君子之能直人也。\end{quoting}

\textbf{羔裘晏兮,三英粲兮。}{\footnotesize 晏,鮮盛貌。三英,三德也。箋云三德,剛克、柔克、正直也。粲,眾意。}\textbf{彼其之子,邦之彥兮。}{\footnotesize 彥,士之美稱。}

\begin{quoting}爾雅「晏晏、溫溫,柔也」。\end{quoting}

\section{遵大路}

%{\footnotesize 二章、章四句}

\textbf{遵大路,思君子也。莊公失道,君子去之,國人思望焉。}

\textbf{遵大路兮,摻執子之袪兮。}{\footnotesize 遵循、路道、摻擥、袪袂也。箋云思望君子,於道中見之,則欲擥持其袂而留之。}\textbf{無我惡兮,不寁故也。}{\footnotesize 寁,速也。箋云子無惡我擥持子之袂,我乃以莊公不速於先君之道使我然。}

\begin{quoting}袪 \texttt{qū}。\textbf{馬瑞辰}寁 \texttt{zǎn} 字訓速,速當讀同孟子「可以速則速」之速,趙注孟子「速,速去也」,速對久言,久為遲留,故知速為速去。故,故人也。\end{quoting}

\textbf{遵大路兮,摻執子之手兮。}{\footnotesize 箋云言執手者,思望之甚。}\textbf{無我魗兮,不寁好也。}{\footnotesize 魗,棄也。箋云魗,亦惡也。好,猶善也。子無惡我,我乃以莊公不速於善道使我然。}

\begin{quoting}孔疏「魗與醜,古今字」。\end{quoting}

\section{女曰雞鳴}

%{\footnotesize 三章、章六句}

\textbf{女曰雞鳴,刺不說德也。陳古義以刺今不說德而好色也。}{\footnotesize 德,謂士大夫賓客有德者。}

\begin{quoting}\textbf{聞一多}風詩類鈔曰女曰雞鳴,樂新婚也。\end{quoting}

\textbf{女曰雞鳴,士曰昧旦。}{\footnotesize 箋云此夫婦相警覺以夙興,言不留色也。}\textbf{子興視夜,明星有爛。}{\footnotesize 言小星已不見也。箋云明星尚爛爛然,蚤於別色時。}\textbf{將翱將翔,弋鳧與鴈。}{\footnotesize 閒於政事,則翱翔習射。箋云弋,繳射也。言無事則往弋射鳧鴈,以待賓客為燕具。}

\begin{quoting}說文「昧,昧爽,且明也」。\end{quoting}

\textbf{弋言加之,與子宜之。}{\footnotesize 宜,肴也。箋云言,我也。子,謂賓客也。所弋之鳧鴈,我以為加豆之實,與君子共肴也。}\textbf{宜言飲酒,與子偕老。}{\footnotesize 箋云宜乎我燕樂賓客而飲酒,與之俱至老,親愛之言也。}\textbf{琴瑟在御,莫不靜好。}{\footnotesize 君子無故不徹琴瑟,賓主和樂,無不安好。}

\begin{quoting}言,語詞。\textbf{朱熹}加,中也,史記所謂「以弱弓微繳加諸鳧雁之上」是也。靜,同靖,爾雅釋詁「靖,善也」。\end{quoting}

\textbf{知子之來之,雜佩以贈之。}{\footnotesize 雜佩者,珩璜琚瑀衝牙之類。箋云贈,送也。我若知子之必來,我則豫儲雜佩,去則以送子也。與異國賓客燕時,雖無此物,猶言之,以致其厚意,其若有之,固將行之。士大夫以君命出使,主國之臣必以燕禮樂之,助君之歡。}\textbf{知子之順之,雜佩以問之。}{\footnotesize 問,遺也。箋云順,謂與己和順。}\textbf{知子之好之,雜佩以報之。}{\footnotesize 箋云好,謂與己同好。}

\begin{quoting}\textbf{王引之}來,讀為勞來之來,爾雅云「勞來,勤也」。\end{quoting}

\section{有女同車}

%{\footnotesize 二章、章六句}

\textbf{有女同車,刺忽也,鄭人刺忽之不昬于齊。太子忽嘗有功于齊,齊侯請妻之,齊女賢而不取,卒以無大國之助至於見逐,故國人刺之。}{\footnotesize 忽,鄭莊公世子,祭仲逐之而立突。}

\textbf{有女同車,顏如舜華。}{\footnotesize 親迎同車也。舜,木槿也。箋云鄭人刺忽不取齊女,親迎與之同車,故稱同車之禮,齊女之美。}\textbf{將翱將翔,佩玉瓊琚。}{\footnotesize 佩有琚瑀,所以納閒。}\textbf{彼美孟姜,洵美且都。}{\footnotesize 孟姜,齊長女。都,閑也。箋云洵,信也。言孟姜信美好且閑習婦禮。}

\begin{quoting}舜,魯詩作蕣。都,同奲 \texttt{duǒ},說文「奲,富奲奲貌」,\textbf{陳奐}奲合二字會意,奢,張也,單,大也,富奲奲,言容貌之美大也。\end{quoting}

\textbf{有女同行,顏如舜英。}{\footnotesize 行,行道也。英,猶華也。箋云女始乘車,壻御輪三周,御者代壻。}\textbf{將翱將翔,佩玉將將。}{\footnotesize 將將,鳴玉而後行。}\textbf{彼美孟姜,德音不忘。}{\footnotesize 箋云不忘者,後世傳道其德也。}

\begin{quoting}將將,魯詩作鏘鏘,同瑲瑲,玉聲也。\textbf{王引之}不忘,猶言德音不已。\end{quoting}

\section{山有扶蘇}

%{\footnotesize 二章、章四句}

\textbf{山有扶蘇,刺忽也。所美非美然。}{\footnotesize 言忽所美之人實非美人。}

\textbf{山有扶蘇,隰有荷華。}{\footnotesize 興也。扶蘇,扶胥,小木也。荷華,扶渠也,其華菡萏。言高下大小各得其宜也。箋云興者,扶胥之木生于山,喻忽置不正之人于上位也,荷華生于隰,喻忽置有美德者于下位,此言其用臣顛倒,失其所也。}\textbf{不見子都,乃見狂且。}{\footnotesize 子都,世之美好者也。狂,狂人也。且,辭也。箋云人之好美色不往覩子都,乃反往覩狂醜之人,以興忽好善不任用賢者,反任用小人,其意同。}

\begin{quoting}段注「扶疏謂大木枝柯四布,疏通作胥,亦作蘇」。孟子「至於子都,天下莫不知其姣也」。且 \texttt{jū}。\end{quoting}

\textbf{山有橋松,隰有游龍。}{\footnotesize 松,木也。龍,紅草也。箋云游龍,猶放縱也。橋松在山上,喻忽無恩澤於大臣也,紅草放縱枝葉於隰中,喻忽聽恣小臣。此又言養臣顛倒,失其所也。}\textbf{不見子充,乃見狡童。}{\footnotesize 子充,良人也。狡童,昭公也。箋云人之好忠良之人不往覩子充,乃反往覩狡童,狡童有貌而無實。}

\begin{quoting}橋,釋文「本亦作喬,王云高也」。龍,同蘢。\end{quoting}

\section{蘀兮}

%{\footnotesize 二章、章四句}

\textbf{蘀兮,刺忽也。君弱臣強,不倡而和也。}{\footnotesize 不倡而和,君臣各失其禮,不相倡和。}

\textbf{蘀兮蘀兮,風其吹女。}{\footnotesize 興也。蘀,槁也。人臣待君唱而後和。箋云槁,謂木葉也。木葉槁,待風乃落,興者,風喻號令也,喻君有政教,臣乃行之,言此者,刺今不然。}\textbf{叔兮伯兮,倡予和女。}{\footnotesize 叔伯,言群臣長幼也。君唱臣和也。箋云叔伯,群臣相謂也。群臣無其君而行,自以彊弱相服,女倡矣,我則將和之,言此者,刺其自專也。叔伯,兄弟之稱。}

\begin{quoting}說文「草木凡皮葉落陊地為蘀 \texttt{tuò}」。說文「倡,樂也」,段注「經傳皆用為唱字」。倡予和女,即予倡女和。\end{quoting}

\textbf{蘀兮蘀兮,風其漂女。}{\footnotesize 漂,猶吹也。}\textbf{叔兮伯兮,倡予要女。}{\footnotesize 要,成也。}

\begin{quoting}漂,釋文「本亦作飄」。要,通邀,莊子寓言「老聃西遊於秦,邀於郊」,釋文「邀,要也」。\end{quoting}

\section{狡童}

%{\footnotesize 二章、章四句}

\textbf{狡童,刺忽也。不能與賢人圖事,權臣擅命也。}{\footnotesize 權臣擅命,祭仲專也。}

\textbf{彼狡童兮,不與我言兮。}{\footnotesize 昭公有壯狡之志。箋云不與我言者,賢者欲與忽圖國之政事,而忽不能受之,故云然。}\textbf{維子之故,使我不能餐兮。}{\footnotesize 憂懼不遑餐也。}

\begin{quoting}\textbf{王引之}惟,猶以也,詩狡童曰「維子之故」。\end{quoting}

\textbf{彼狡童兮,不與我食兮。}{\footnotesize 不與賢人共食祿。}\textbf{維子之故,使我不能息兮。}{\footnotesize 憂不能息也。}

\section{褰裳}

%{\footnotesize 二章、章五句}

\textbf{褰裳,思見正也。狂童恣行,國人思大國之正己也。}{\footnotesize 狂童恣行,謂突與忽爭國,更出更入,而無大國正之。}

\textbf{子惠思我,褰裳涉溱。}{\footnotesize 惠,愛也。溱,水名也。箋云子者,斥大國之正卿。子若愛而思我,我國有突篡國之事而可征而正之,我則揭衣渡溱水往告難也。}\textbf{子不我思,豈無他人。}{\footnotesize 箋云言他人者,先鄉齊晉宋衛,後之荊楚。}\textbf{狂童之狂也且。}{\footnotesize 狂行童昏所化也。箋云狂童之人日為狂行,故使我言此也。}

\begin{quoting}\textbf{毛奇齡}女子曰,子思我,子當褰裳來,嗜山不顧高,嗜桃不顧毛也。國語韋注「童,無智」,\textbf{陳奐}童即狂也,童昏即狂行之狀,單言狂,累言狂童,無二義也。\end{quoting}

\textbf{子惠思我,褰裳涉洧。}{\footnotesize 洧,水名也。}\textbf{子不我思,豈無他士。}{\footnotesize 士,事也。箋云他士,猶他人也。大國之卿當天子之上士。}\textbf{狂童之狂也且。}

\begin{quoting}溱洧 \texttt{zhēn wěi} 皆鄭水名,在今河南密縣。\end{quoting}

\section{丰}

%{\footnotesize 四章、二章章三句、二章章四句}

\textbf{丰,刺亂也。昬姻之道缺,陽倡而陰不和,男行而女不隨。}{\footnotesize 婚姻之道,謂嫁取之禮。}

\textbf{子之丰兮,俟我乎巷兮,}{\footnotesize 丰,豐滿也。巷,門外也。箋云子,謂親迎者。我,我將嫁者。有親迎我者,面貌丰丰然豐滿善人也,出門而待我於巷中。}\textbf{悔予不送兮。}{\footnotesize 時有違而不至者。箋云悔乎我不送是子而去也。時不送則為異人之色,後不得耦而思之。}

\begin{quoting}\textbf{胡承珙}送,猶致也,荀子富國篇注「送,致女」,春秋言致女者,即以女授壻之謂,此女悔其不行,故託言於其家之不致,非自謂其不送男子也。\end{quoting}

\textbf{子之昌兮,俟我乎堂兮,}{\footnotesize 昌,盛壯貌。箋云堂,當為棖,棖,門梱上木近邊者。}\textbf{悔予不將兮。}{\footnotesize 將,行也。箋云將,亦送也。}

\textbf{衣錦褧衣,裳錦褧裳。}{\footnotesize 衣錦褧裳,嫁者之服。箋云褧,襌也,蓋以襌縠為之。中衣裳用錦,而上加襌縠焉,為其文之大著也。庶人之妻嫁服也,士妻䊷衣纁袡。}\textbf{叔兮伯兮,駕予與行。}{\footnotesize 叔伯,迎己者。箋云言此者,以前之悔,今則叔也伯也來迎己者,從之,志又易也。}

\begin{quoting}行,同「女子有行」之行,即下章之歸也。\end{quoting}

\textbf{裳錦褧裳,衣錦褧衣。叔兮伯兮,駕予與歸。}

\section{東門之墠}

%{\footnotesize 二章、章四句}

\textbf{東門之墠,刺亂也。男女有不待禮而相奔者也。}

\textbf{東門之墠,茹藘在阪。}{\footnotesize 東門,城東門也。墠,除地町町者。茹藘,茅蒐也。男女之際,近而易則如東門之墠,遠而難則茹藘在阪。箋云城東門之外有墠,墠邊有阪,茅蒐生焉,茅蒐之為難淺矣,易越而出。此女欲奔男之辭。}\textbf{其室則邇,其人甚遠。}{\footnotesize 邇,近也。得禮則近,不得禮則遠。箋云其室則近,謂所欲奔男之家,望其來迎己而不來,則為遠。}

\begin{quoting}墠 \texttt{shàn}。\end{quoting}

\textbf{東門之栗,有踐家室。}{\footnotesize 栗,行上栗也。踐,淺也。箋云栗而在淺家室之內,言易竊取。栗,人所㗖食而甘嗜,故女以自喻也。}\textbf{豈不爾思,子不我即。}{\footnotesize 即,就也。箋云我豈不思望女乎,女不就迎我而俱去耳。}

\begin{quoting}踐,韓詩作靖,\textbf{王先謙}韓踐作靖,云善也。\end{quoting}

\section{風雨}

%{\footnotesize 三章、章四句}

\textbf{風雨,思君子也。亂世則思君子不改其度焉。}

\textbf{風雨淒淒,雞鳴喈喈。}{\footnotesize 興也。風且雨淒淒然,雞猶守時而鳴喈喈然。箋云興者,喻君子雖居亂世,不變改其節度。}\textbf{既見君子,云胡不夷。}{\footnotesize 胡何、夷說也。箋云思而見之,云何而心不說。}

\textbf{風雨瀟瀟,雞鳴膠膠。}{\footnotesize 瀟瀟,暴疾也。膠膠,猶喈喈也。}\textbf{既見君子,云胡不瘳。}{\footnotesize 瘳,愈也。}

\begin{quoting}膠,三家詩作嘐。\end{quoting}

\textbf{風雨如晦,雞鳴不已。}{\footnotesize 晦,昏也。箋云已,止也。雞不為如晦而止不鳴。}\textbf{既見君子,云胡不喜。}

\begin{quoting}\textbf{陳奐}如,猶而也。說文「晦,月盡也」,段注「引伸為凡光盡之稱」。\end{quoting}

\section{子衿}

%{\footnotesize 三章、章四句}

\textbf{子衿,刺學校廢也。亂世則學校不脩焉。}{\footnotesize 鄭國謂學為校,言可以校正道藝。}

\textbf{靑靑子衿,悠悠我心。}{\footnotesize 青衿,青領也,學子之所服。箋云學子而俱在學校之中,己留彼去,故隨而思之耳。禮,父母在,衣純以青。}\textbf{縱我不往,子寧不嗣音。}{\footnotesize 嗣,習也。古者教以詩樂,誦之歌之,弦之舞之。箋云嗣,續也。女曾不傳聲問我,以恩責其忘己。}

\begin{quoting}衿,同䘳,亦作襟,顏氏家訓書證篇「古者斜領下連於衿,故謂領為衿」。嗣,魯詩韓詩作詒,韓詩「詒,寄也,曾不寄問也」。\end{quoting}

\textbf{靑靑子佩,悠悠我思。}{\footnotesize 佩,佩玉也。士佩瓀珉而青組綬。}\textbf{縱我不往,子寧不來。}{\footnotesize 不來者,言不一來也。}

\textbf{挑兮達兮,在城闕兮。}{\footnotesize 挑達,往來相見貌。乘城而見闕。箋云國亂,人廢學業,但好登高見於城闕,以候望為樂。}\textbf{一日不見,如三月兮。}{\footnotesize 言禮樂不可一日而廢。箋云君子之學,以文會友,以友輔仁,獨學而無友,則孤陋而寡聞,故思之甚。}

\begin{quoting}\textbf{胡承珙}大東「佻佻公子」,傳訓獨行,此挑達訓往來者,亦謂獨往獨來。\end{quoting}

\section{揚之水}

%{\footnotesize 二章、章六句}

\textbf{揚之水,閔無臣也。君子閔忽之無忠臣良士,終以死亡,而作是詩也。}

\textbf{揚之水,不流束楚。}{\footnotesize 揚,激揚也。激揚之水,可謂不能流漂束楚乎。箋云激揚之水,喻忽政教亂促,不流束楚,言其政不行於臣下。}\textbf{終鮮兄弟,維予與女。}{\footnotesize 箋云鮮,寡也。忽兄弟爭國,親戚相疑,後竟寡於兄弟之恩,獨我與女有耳。作此詩者,同姓臣也。}\textbf{無信人之言,人實迋女。}{\footnotesize 迋,誑也。}

\begin{quoting}終,既也。維,同惟。迋 \texttt{guàng}。\end{quoting}

\textbf{揚之水,不流束薪。終鮮兄弟,維予二人。}{\footnotesize 二人同心也。箋云二人者,我身與女忽。}\textbf{無信人之言,人實不信。}

\section{出其東門}

%{\footnotesize 二章、章六句}

\textbf{出其東門,閔亂也。公子五爭,兵革不息,男女相棄,民人思保其室家焉。}{\footnotesize 公子五爭者,謂突再也,忽、子亹、子儀各一也。}

\textbf{出其東門,有女如雲。}{\footnotesize 如雲,眾多也。箋云有女,謂諸見棄者也。如雲者,如雲從風東西南北,心無有定。}\textbf{雖則如雲,匪我思存。}{\footnotesize 思不存乎相救急。箋云匪,非也。此如雲者,皆非我思所存也。}\textbf{縞衣綦巾,聊樂我員。}{\footnotesize 縞衣,白色,男服也。綦巾,蒼艾色,女服也。願室家得相樂。箋云縞衣綦巾,所為作者之妻服也,時亦棄之,迫兵革之難,不能相畜,心不忍絕,故言且留樂我員。此思保其室家,窮困不得有其妻,而以衣巾言之,恩不忍斥之。綦,綦文也。}

\begin{quoting}\textbf{王先謙}鄭城西南門為溱洧二水所經,故以東門為遊人所集。綦 \texttt{qí}。\textbf{馬瑞辰}員,當讀如「婚姻孔云」之云,彼箋云「云,猶友也」,有與友同,詩言不相親者云「亦莫我有」,則言其相親友者宜曰「聊樂我員」矣。\end{quoting}

\textbf{出其闉闍,有女如荼。}{\footnotesize 闉,曲城也。闍,城臺也。荼,英荼也。言皆喪服也。箋云闍,讀當如彼都人士之都,謂國外曲城之中市里也。荼,茅秀,物之輕者,飛行無常。}\textbf{雖則如荼,匪我思且。}{\footnotesize 箋云匪我思且,猶非我思存也。}\textbf{縞衣茹藘,聊可與娛。}{\footnotesize 茹藘,茅蒐之染女服也。娛,樂也。箋云茅蒐,染巾也。聊可與娛,且可留與我為樂,心欲留之言也。}

\begin{quoting}說文「闉 \texttt{yīn},闉闍,城曲重門也,闍 \texttt{dū},闉闍」。且,同徂,爾雅「徂、在,存也」。\end{quoting}

\section{野有蔓草}

%{\footnotesize 二章、章六句}

\textbf{野有蔓草,思遇時也。君之澤不下流,民窮於蔓草,男女失時,思不期而會焉。}{\footnotesize 不期而會,謂不相與期而自俱會。}

\textbf{野有蔓草,零露漙兮。}{\footnotesize 興也。野,四郊之外。蔓,延也。漙漙然,盛多也。箋云零,落也。蔓草而有露,謂仲春之時,草始生,霜為露也,周禮「仲春之月,令會男女之無夫家者」。}\textbf{有美一人,清揚婉兮。邂逅相遇,適我願兮。}{\footnotesize 清揚,眉目之間婉然美也。邂逅,不期而會。適其時願。}

\begin{quoting}釋文「漙 \texttt{tuán},本又作團」。\end{quoting}

\textbf{野有蔓草,零露瀼瀼。}{\footnotesize 瀼瀼,盛貌。}\textbf{有美一人,婉如清揚。邂逅相遇,與子皆臧。}{\footnotesize 臧,善也。}

\begin{quoting}瀼 \texttt{ráng}。\textbf{朱熹}偕臧,言各得其所欲也。\end{quoting}

\section{溱洧}

%{\footnotesize 二章、章十二句}

\textbf{溱洧,刺亂也。兵革不息,男女相棄,淫風大行,莫之能救焉。}{\footnotesize 救,猶止也。亂者,士與女合會溱洧之上。}

\begin{quoting}太平御覽引韓詩章句曰當此盛流之時,士與女眾方執蘭,拂除邪惡,鄭國之俗,三月上巳之辰,于此兩水之上招魂續魄,除拂不祥。即修禊 \texttt{xì} 也。\end{quoting}

\textbf{溱與洧,方渙渙兮。}{\footnotesize 溱洧,鄭兩水名。渙渙,春水盛也。箋云仲春之時冰以釋,水則渙渙然。}\textbf{士與女,方秉蕳兮。}{\footnotesize 蕳,蘭也。箋云男女相棄,各無匹耦,感春氣並出,託采芬香之草,而為淫泆之行。}\textbf{女曰觀乎,士曰既且。}{\footnotesize 箋云女曰觀乎,欲與士觀於寬閒之處。既,已也。士曰已觀矣,未從之也。}\textbf{且往觀乎,洧之外,洵訏且樂。}{\footnotesize 訏,大也。箋云洵,信也。女情急,故勸男使往觀於洧之外,言其土地信寬大又樂也,於是男則往也。}\textbf{維士與女,伊其相謔,贈之以勺藥。}{\footnotesize 勺藥,香草。箋云伊,因也。士與女往觀,因相與戲謔,行夫婦之事,其別則送女以勺藥,結恩情也。}

\begin{quoting}渙,韓詩作洹。蕳 \texttt{jiān},菊科,亦名蘭,古人以沐浴佩身,拂除不祥。且,同徂。伊,同㗨,嘻笑貌。\end{quoting}

\textbf{溱與洧,瀏其清矣。}{\footnotesize 瀏,深貌。}\textbf{士與女,殷其盈矣。}{\footnotesize 殷,眾也。}\textbf{女曰觀乎,士曰既且。且往觀乎,洧之外,洵訏且樂。維士與女,伊其將謔,贈之以勺藥。}{\footnotesize 箋云將,大也。}

\begin{quoting}說文「瀏,流清貌」。\textbf{馬瑞辰}將謔,猶相謔也。\end{quoting}

%\begin{flushright}鄭國二十一篇、五十三章、二百八十三句\end{flushright}