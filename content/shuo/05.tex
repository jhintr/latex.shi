\chapter{方正第五}

\subsection*{1}

\textbf{陳太丘與友期行,期日中,過中不至,太丘舍去,去後乃至,元方時年七歲,門外戲,}{\footnotesize 陳寔及紀並已見。}\textbf{客問元方:「尊君在不?」答曰:「待君久不至,已去。」友人便怒曰:「非人哉!與人期行,相委而去。」元方曰:「君與家君期日中,日中不至,則是無信,對子罵父,則是無禮。」友人慙,下車引之,元方入門不顧。}

\subsection*{2}

\textbf{南陽宗世林,魏武同時,而甚薄其為人,不與之交,及魏武作司空,總朝政,從容問宗曰:「可以交未?」答曰:「松柏之志猶存。」世林既以忤旨見疏,位不配德,文帝兄弟毎造其門,皆獨拜牀下,其見禮如此。}{\footnotesize \textbf{楚國先賢傳}曰宗承,字世林,南陽安眾人,父資,有美譽,承少而修德雅正,确然不群,徵聘不就,聞德而至者如林,魏武弱冠,屢造其門,值賓客猥積,不能得言,乃伺承起,往要之,捉手請交,承拒而不納,帝後為司空,輔漢朝,乃謂承曰「卿昔不顧吾,今可為交未」,承曰「松柏之志猶存」,帝不說,以其名賢,猶敬禮之,敕文帝修子弟禮,就家拜漢中太守,武帝平冀州,從至鄴,陳群等皆為之拜,帝猶以舊情介意,薄其位而優其禮,就家訪以朝政,居賓客之右,文帝徵為直諫大夫,明帝欲引以為相,以老固辭。}

\subsection*{3}

\textbf{魏文帝受禪,陳群有慽容,帝問曰:「朕應天受命,卿何以不樂?」群曰:「臣與華歆服膺先朝,今雖欣聖化,猶義形於色。」}{\footnotesize \textbf{華嶠譜敘}曰魏受禪,朝臣三公以下並受爵位,華歆以形色忤時,徙為司空,不進爵,文帝久不懌,以問尚書令陳群曰「我應天受命,百辟莫不說喜,形於聲色,而相國及公獨有不怡者,何邪」,群起離席長跪曰「臣與相國曾事漢朝,心雖說喜,義干其色,亦懼陛下,實應見憎」,帝大說,歎息良久,遂重異之。}

\subsection*{4}

\textbf{郭淮作關中都督,甚得民情,亦屢有戰庸,}{\footnotesize \textbf{魏志}曰淮,字伯濟,太原陽曲人,建安中,除平原府丞,黃初元年,奉使賀文帝踐阼,而稽留不及,群臣歡會,帝正色責之曰「昔禹會諸侯於塗山,防風氏後至,便行大戮,今溥天同慶,而卿最留遲,何也」,淮曰「臣聞五帝先教,導民以德,夏后政衰,始用刑辟,今臣遭唐虞之世,是以知免防風氏之誅」,帝說之,擢為雍州刺史,遷征西將軍,淮在關中二十餘年,功績顯著,遷儀同三司,贈大將軍。}\textbf{淮妻,太尉王凌之妹,坐凌事當并誅,}{\footnotesize \textbf{魏略}曰凌,字彥雲,太原祁人,歷司空、太尉、征東將軍,密欲立楚王彪,司馬宣王自討之,凌自縛歸罪,遙謂太傅曰「卿直以折簡召我,我當不至邪」,太傅曰「以卿非肯逐折簡者也」,遂使人送至西,凌自知罪重,試索棺釘,以觀太傅意,太傅給之,凌行至項城,夜呼掾屬與決曰「行年八十,身名俱滅,命邪」,遂自殺。}\textbf{使者徵攝甚急,淮使戒裝,克日當發,州府文武及百姓勸淮舉兵,淮不許,至期,遣妻,百姓號泣追呼者數萬人,行數十里,淮乃命左右追夫人還,於是文武奔馳,如徇身首之急,既至,淮與宣帝書曰:「五子哀戀,思念其母,其母既亡,則無五子,五子若殞,亦復無淮。」宣帝乃表,特原淮妻。}{\footnotesize \textbf{世語}曰淮妻當從坐,侍御史往收,督將及羌胡渠帥數千人叩頭,請淮上表留妻,淮不從,妻上道,莫不流涕,人人扼腕,欲劫留之,淮五子叩頭流血請淮,淮不忍視,乃命追之,於是數千騎往追還,淮以書白司馬宣王曰「五子哀母,不惜其身,若無其母,是無五子,五子若亡,亦無淮也,今輒追還,若於法未通,當受罪於主者」,書至,宣王乃表原之。}

\subsection*{5}

\textbf{諸葛亮之次渭濱,關中震動,}{\footnotesize \textbf{蜀志}曰亮,字孔明,琅邪陽都人,客於荊州,躬耕隴畝,好為梁甫吟,長八尺,毎自比管仲、樂毅,時人莫之許也,惟博陵崔州平、潁川徐元直謂為信然,先主屯新野,徐庶見先主曰「諸葛孔明,臥龍也,將軍豈願見之乎」,先主曰「君與俱來」,庶曰「此人可就見,不可屈致也」,先主遂詣亮,謂關羽、張飛曰「孤之有孔明,猶魚之有水也」,累遷丞相、益州牧,率眾北征,卒於渭南。}\textbf{魏明帝深懼晉宣王戰,乃遣辛毗為軍司馬,}{\footnotesize \textbf{魏志}曰毗,字佐治,潁川陽翟人,累遷衛尉。}\textbf{宣王既與亮對渭而陳,亮設誘譎萬方,宣王果大忿,將欲應之以重兵,亮遣間諜覘之,還曰:「有一老夫,毅然仗黃鉞,當軍門立,軍不得出。」亮曰:「此必辛佐治也。」}{\footnotesize \textbf{晉陽秋}曰諸葛亮寇於郿,據渭水南原,詔使高祖拒之,亮善撫御,又戎政嚴明,且僑軍遠征,糧運艱澀,利在野戰,朝廷毎聞其出,欲以不戰屈之,高祖亦以為然,而擁大軍禦侮於外,不宜遠露怯弱之形以虧大勢,故秣馬坐甲,毎見吞併之威,亮雖挑戰,或遺高祖巾幗,巾幗,婦女之飾,欲以激怒,冀獲曹咎之利,朝廷慮高祖不勝忿憤,而衛尉辛毗骨鯁之臣,帝乃使毗仗節為高祖軍司馬,亮果復挑戰,高祖乃奮怒,將出應之,毗仗節中門而立,高祖乃止,將士聞見者益加勇鋭,識者以人臣雖擁眾千萬而屈於王人,大略深長,皆如此之類也。}

\subsection*{6}

\textbf{夏侯玄既被桎梏,}{\footnotesize \textbf{魏氏春秋}曰玄,字太初,譙國人,夏侯尚之子、大將軍前妻兄也,風格高朗,弘辯博暢,正始中,護軍,曹爽誅,徵為太常,內知不免,不交人事,不畜筆研,及太傅薨,許允謂玄曰「子無復憂矣」,玄歎曰「士宗,卿何不見事乎?此人猶能以通家年少遇我,子元、子上不吾容也」,後中書令李豐惡大將軍執政,遂謀以玄代之,大將軍聞其謀,誅豐,收玄送廷尉。\textbf{干寶晉紀}曰初,豐之謀也,使告玄,玄答曰「宜詳之爾」,不以聞也,故及於難。}\textbf{時鍾毓為廷尉,鍾會先不與玄相知,因便狎之,玄曰:「雖復刑餘之人,未敢聞命。」}{\footnotesize \textbf{世語}曰玄至廷尉,不肯下辭,廷尉鍾毓自臨履玄,玄正色曰「吾當何辭,為令史責人邪?卿便為吾作」,毓以玄名士,節高不可屈,而獄當竟,夜為作辭,令與事相附,流涕以示玄,玄視之曰「不當若是邪」,鍾會年少於玄,玄不與交,是日於毓坐狎玄,玄正色曰「鍾君,何得如是」。\textbf{名士傳}曰初,玄以鍾毓志趣不同,不與之交,玄被收時,毓為廷尉,執玄手曰「太初何至於此」,玄正色曰「雖復刑餘之人,不可得交」。\textbf{按}郭頒,西晉人,時世相近,為晉魏世語,事多詳覈,孫盛之徒皆採以著書,並云玄距鍾會,而袁宏名士傳最後出,不依前史,以為鍾毓,可謂謬矣。}\textbf{考掠初無一言,臨刑東市,顏色不異。}{\footnotesize \textbf{魏志}曰玄格量弘濟,臨斬,顏色不異,舉止自若。}

\subsection*{7}

\textbf{夏侯泰初與廣陵陳本善,本與玄在本母前宴飲,}{\footnotesize \textbf{世語}曰本,字休元,臨淮東陽人。\textbf{魏志}曰本,廣陵東陽人,父矯,司徒,本歷郡守、廷尉,所在操綱領、舉大體,能使群下自盡,有率御之才,不親小事,不讀法律,而得廷尉之稱,遷鎮北將軍。}\textbf{本弟騫}{\footnotesize \textbf{晉陽秋}曰騫,字休淵,司徒第二子,無謇諤風,滑稽而多智謀,仕至大司馬。}\textbf{行還,徑入,至堂戶,泰初因起曰:「可得同,不可得而雜。」}{\footnotesize \textbf{名士傳}曰玄以鄉黨貴齒,本不論德位,年長者必為拜,與陳本母前飲,騫來而出,其可得同,不可得而雜者也。}

\subsection*{8}

\textbf{高貴鄉公薨,內外諠譁,}{\footnotesize \textbf{魏志}曰高貴鄉公諱髦,字彥士,文帝孫、東海定王霖之子也,初封郯縣高貴鄉公,好學夙成,齊王廢,群臣迎之,即皇帝位。\textbf{漢晉春秋}曰自曹芳事後,魏人省徹宿衛,無復鎧甲,諸門戎兵,老弱而已,曹髦見威權日去,不勝其忿,召侍中王沈、尚書王經、散騎常侍王業謂曰「司馬昭之心,路人所知也,吾不能坐受廢辱,今日當與卿自出討之」,王經諫不聽,乃出懷中板令投地曰「行之決矣,正使死,何所恨,況不必死邪」,於是入白太后,沈業奔走告昭,昭為之備,髦遂率僮僕數百鼓譟而出,昭弟屯騎校尉伷入,遇髦於東止車門,左右訶之,伷眾奔走,中護軍賈充又逆髦,戰於南闕下,髦自用劍,眾欲退,太子舍人成濟問充曰「事急矣,當云何」,充曰「公畜汝等,正為今日,今日之事,無所問也」,濟即前刺髦,刃出於背。\textbf{魏氏春秋}曰帝將誅大將軍,詔有司復進位相國,加九錫,帝夜自將宂從僕射李昭、黃門從官焦伯等下陵雲臺,鎧仗授兵,欲因際會,遣使自出致討,會雨而卻,明日,遂見王經等,出黃素詔於懷曰「是可忍也,孰不可忍,今當決行此事」,帝遂拔劍升輦,率殿中宿衛倉頭官僮,撃戰鼓,出雲龍門,賈充自外而入,帝師潰散,帝猶稱天子,手劍奮撃,眾莫敢逼,充率厲將士,騎督成倅、弟濟以矛進,帝崩於師,時暴雨,雷電晦冥。}\textbf{司馬文王問侍中陳泰曰:}{\footnotesize \textbf{魏志}曰泰,字玄伯,司空群之子也。}\textbf{「何以靜之?」泰云:「唯殺賈充以謝天下。」文王曰:「可復下此不?」對曰:「但見其上,未見其下。」}{\footnotesize \textbf{干寶晉紀}曰高貴鄉公之殺,司馬文王召朝臣謀其故,太常陳泰不至,使其舅荀顗召之,告以可不,泰曰「世之論者,以泰方於舅,今舅不如泰也」,子弟內外咸共逼之,垂涕而入,文王待之曲室,謂曰「玄伯,卿何以處我」,對曰「可誅賈充以謝天下」,文王曰「為吾更思其次」,泰曰「惟有進於此,不知其次」,文王乃止。\textbf{漢晉春秋}曰曹髦之薨,司馬昭聞之,自投於地曰「天下謂我何」,於是召百官議其事,昭垂涕問陳泰曰「何以居我」,泰曰「公光輔數世,功蓋天下,謂當並迹古人,垂美於後,一旦有殺君之事,不亦惜乎,速斬賈充,猶可以自明也」,昭曰「公閭不可得殺也,卿更思餘計」,泰厲聲曰「意惟有進於此耳,餘無足委者也」,歸而自殺。\textbf{魏氏春秋}曰泰勸大將軍誅賈充,大將軍曰「卿更思其他」,泰曰「豈可使泰復發後言」,遂嘔血死。}

\subsection*{9}

\textbf{和嶠為武帝所親重,語嶠曰:「東宮頃似更成進,卿試往看。」還,問:「何如?」答云:「皇太子聖質如初。」}{\footnotesize \textbf{晉諸公贊}曰嶠,字長輿,汝南西平人,父逌,太常,知名,嶠少以雅量稱,深為賈充所知,毎向世祖稱之,歷尚書、太子少傅。\textbf{干寶晉紀}曰皇太子有醇古之風,美於信受,侍中和嶠數言於上曰「季世多偽,而太子尚信,非四海之主,憂太子不了陛下家事,願追思文武之祚」,上既重長適,又懷齊王,朋黨之論弗入也,後上謂嶠曰「太子近入朝,吾謂差進,卿可與荀侍中共往言」,及顗奉詔還,對上曰「太子明識弘新,有如明詔」,問嶠,嶠對曰「聖質如初」,上默然。\textbf{晉陽秋}曰世祖疑惠帝不可承繼大業,遣和嶠、荀勖往觀察之,既見,勖稱歎曰「太子德更進茂,不同於故」,嶠曰「皇太子聖質如初,此陛下家事,非臣所盡」,天下聞之,莫不稱嶠為忠,而欲灰滅勖也。\textbf{按}荀顗清雅,性不阿諛,校之二說,則孫盛為得也。}

\subsection*{10}

\textbf{諸葛靚後入晉,除大司馬,召不起,以與晉室有讎,常背洛水而坐,與武帝有舊,帝欲見之而無由,乃請諸葛妃呼靚,既來,帝就太妃間相見,禮畢,酒酣,帝曰:「卿故復憶竹馬之好不?」靚曰:「臣不能吞炭漆身,今日復覩聖顏。」因涕泗百行,帝於是慙悔而出。}{\footnotesize \textbf{晉諸公贊}曰吳亡,靚入洛,以父誕為太祖所殺,誓不見世祖,世祖叔母琅邪王妃,靚之姊也,帝後因靚在姊間,往就見焉,靚逃於廁中,於是以至孝發名,時嵇康亦被法,而康子紹死蕩陰之役,談者咸曰「觀紹、靚二人,然後知忠孝之道,區以別矣」。}

\subsection*{11}

\textbf{武帝語和嶠曰:「我欲先痛罵王武子,然後爵之。」嶠曰:「武子儁爽,恐不可屈。」帝遂召武子,苦責之,因曰:「知愧不?」}{\footnotesize \textbf{晉諸公贊}曰齊王當出藩,而王濟諫請無數,又累遣常山主與婦長廣公主共入稽顙,陳乞留之,世祖甚恚,謂王戎曰「我兄弟至親,今出齊王,自朕家計,而甄德、王濟連遣婦入來,生哭人邪?濟等尚爾,況餘者乎」,濟自此被責,左遷國子祭酒。}\textbf{武子曰:「尺布斗粟之謠,常為陛下恥之,}{\footnotesize \textbf{漢書}曰淮南厲王長,高祖少子也,有罪,文帝徙之於蜀,不食而死,民作歌曰「一尺布,尚可縫,一斗粟,尚可舂,兄弟二人,不能相容」。\textbf{瓚}注曰言一尺布帛,可縫而共衣,一斗米粟,可舂而共食,況以天下之屬而不相容也。}\textbf{它人能令疎親,臣不能使親疎,以此愧陛下。」}

\subsection*{12}

\textbf{杜預之荊州,頓七里橋,朝士悉祖,}{\footnotesize \textbf{王隱晉書}曰預,字元凱,京兆杜陵人,漢御史大夫延年十一世孫,祖畿,魏太保,父恕,幽州、荊州刺史,預智謀淵博,明於治亂,常稱立德者非所企及,立功、立言所庶幾也,累遷河南尹,為鎮南將軍,都督荊州諸軍事,鎮襄陽,以平吳勳封當陽侯,預無伎藝之能,身不跨馬,射不穿札,而毎有大事,輒在將帥之限,贈征南將軍、儀同三司。}\textbf{預少賤,好豪俠,不為物所許,楊濟既名氏,雄俊不堪,不坐而去,}{\footnotesize \textbf{八王故事}曰濟,字文通,弘農人,楊駿弟也,有才識,累遷太子太保,與駿同誅。}\textbf{須臾,和長輿來,問:「楊右衛何在?」客曰:「向來,不坐而去。」長輿曰:「必大夏門下盤馬。」往大夏門,果大閲騎,長輿抱內車,共載歸,坐如初。}

\subsection*{13}

\textbf{杜預拜鎮南將軍,朝士悉至,皆在連榻坐,}{\footnotesize \textbf{語林}曰中朝方鎮還,不與元凱共坐,預征吳還,獨榻,不與賓客共也。}\textbf{時亦有裴叔則,羊穉舒後至,曰:「杜元凱乃復連榻坐客。」不坐便去,}{\footnotesize \textbf{晉諸公贊}曰羊琇,字穉舒,泰山人,通濟有才幹,與世祖同年相善,謂世祖曰「後富貴時,見用作領護軍各十年」,世祖即位,累遷左將軍、特進。}\textbf{杜請裴追之,羊去數里住馬,既而俱還杜許。}

\subsection*{14}

\textbf{晉武帝時,荀勖為中書監,}{\footnotesize \textbf{虞預晉書}曰勖,字公曾,潁川潁陰人,漢司空爽曾孫也,十餘歲能屬文,外祖鍾繇曰「此兒當及其曾祖」,為安陽令,民生為立祠,累遷侍中、中書監。}\textbf{和嶠為令,故事,監、令由來共車,嶠性雅正,常疾勖諂諛,}{\footnotesize \textbf{王隱晉書}曰勖性佞媚,譽太子、出齊王,當時私議,損國害民,孫、劉之匹也,後世若有良史,當著佞倖傳。}\textbf{後公車來,嶠便登,正向前坐,不復容勖,勖方更覓車,然後得去,監、令各給車自此始。}{\footnotesize \textbf{曹嘉之晉紀}曰中書監、令常同車入朝,至和嶠為令,而荀勖為監,嶠意強抗,專車而坐,乃使監、令異車,自此始也。}

\subsection*{15}

\textbf{山公大兒著短帢,車中倚,武帝欲見之,山公不敢辭,問兒,兒不肯行,時論乃云勝山公。}{\footnotesize \textbf{晉諸公贊}曰山該,字伯倫,司徒濤長子也,雅有器識,仕至左衛將軍。}

\subsection*{16}

\textbf{向雄為河內主簿,有公事不及雄,而太守劉淮橫怒,遂與杖遣之,雄後為黃門郎,劉為侍中,初不交言,武帝聞之,敕雄復君臣之好,雄不得已,詣劉,再拜曰:「向受詔而來,而君臣之義絕,何如?」於是即去,武帝聞尚不和,乃怒問雄曰:「我令卿復君臣之好,何以猶絕?」}{\footnotesize \textbf{漢晉春秋}曰雄,字茂伯,河內人。\textbf{世語}曰雄有節槩,仕至黃門郎、護軍將軍。\textbf{按}王隱孫盛不與故君相聞議曰「昔在晉初,河內溫縣領校向雄,送御犧牛,不先呈郡,輒隨比送洛,值天大熱,郡送牛多暍死,臺法甚重,太守吳奮召雄與杖,雄不受杖,曰『郡牛者亦死也,呈牛者亦死也』,奮大怒,下雄獄,將大治之,會司隸辟雄都官從事,數年,為黃門侍郎,奮為侍中,同省,相避不相見,武帝聞之,給雄酒禮,使詣奮解,雄乃奉詔」,此則非劉淮也。\textbf{晉諸公贊}曰淮,字君平,沛國杼秋人,少以清正稱,累遷河內太守、侍中、尚書僕射、司徒。}\textbf{雄曰:「古之君子,進人以禮,退人以禮,今之君子,進人若將加諸厀,退人若將墜諸淵,臣於劉河內,不為戎首,亦已幸甚,安復為君臣之好?」武帝從之。}{\footnotesize \textbf{禮記}曰穆公問於子思曰「為舊君反服,古邪」,子思曰「古之君子,進人以禮,退人以禮,故有舊君反服之禮,今之君子,進人若將加諸厀,退人若將墜諸淵,無為戎首,不亦善乎,又何反服之有」。\textbf{鄭玄}曰為兵主來攻伐,故曰戎首也。}

\subsection*{17}

\textbf{齊王冏為大司馬輔政,}{\footnotesize \textbf{虞預晉書}曰冏,字景治,齊王攸子也,少聰惠,及長,謙約好施,趙王倫篡位,冏起義兵誅倫,拜大司馬,加九錫,政皆決之,而恣用群小,不復朝覲,遂為長沙王所誅。}\textbf{嵇紹為侍中,詣冏咨事,冏設宰會,召葛旟、}{\footnotesize \textbf{齊王官屬名}曰旟,字虛旟,齊王從事中郎。\textbf{晉陽秋}曰齊王起義,轉長史,既克趙王倫,與董艾等專執威權,冏敗,見誅。}\textbf{董艾等}{\footnotesize \textbf{八王故事}曰艾,字叔智,弘農人,祖遇,魏侍中,父緩,祕書監,艾少好功名,不修士檢,齊王起義,艾為新汲令,赴軍,用艾領右將軍,王敗,見誅。}\textbf{共論時宜,旟等白冏:「嵇侍中善於絲竹,公可令操之。」遂送樂器,紹推卻不受,冏曰:「今日共為歡,卿何卻邪?」紹曰:「公協輔皇室,令作事可法,紹雖官卑,職備常伯,操絲比竹,蓋樂官之事,不可以先王法服為伶人之業,今逼高命,不敢苟辭,當釋冠冕、襲私服,此紹之心也。」旟等不自得而退。}

\subsection*{18}

\textbf{盧志於眾坐,}{\footnotesize \textbf{世語}曰志,字子通,范陽人,尚書珽少子,少知名,起家鄴令,歷成都王長史、衛尉卿、尚書郎。}\textbf{問陸士衡:「陸遜、陸抗是君何物?」}{\footnotesize 抗已見。\textbf{吳書}曰遜,字伯言,吳郡人,世為冠族,初領海昌令,號神君,累遷丞相。}\textbf{答曰:「如卿於盧毓、盧珽。」}{\footnotesize \textbf{魏志}曰毓,字子家,涿人,父植,有名於世,累遷吏部郎、尚書,選舉,先性行而後言才,進司空。珽,咸熙中為泰山太守,字子笏,位至尚書。}\textbf{士龍失色,}{\footnotesize 雲別見。}\textbf{既出戶,謂兄曰:「何至如此,彼容不相知也?」士衡正色曰:「我父祖名播海內,甯有不知?鬼子敢爾。」}{\footnotesize \textbf{孔氏志怪}曰盧充者,范陽人,家西三十里有崔少府墓,充先冬至一日,出家西獵,見一麞,舉弓而射,即中之,麞倒而復起,充逐之,不覺遠,忽見一里門如府舍,門中一鈴下有唱家前,充問「此何府也」,答曰「少府府也」,充曰「我衣惡,那得見貴人」,即有人提襆新衣迎之,充著盡可體,便進見少府,展姓名,酒炙數行,崔曰「近得尊府君書,為君索小女婚,故相延耳」,即舉書示充,充,父亡時雖小,然已見父手迹,便歔欷無辭,崔即敕內,令女郎莊嚴,使充就東廊,充至,婦已下車,立席頭,共拜,為三日畢,還見崔,崔曰「君可歸矣,女有娠相,生男,當以相還,生女,當留自養」,敕外嚴車送客,崔送至門,執手零涕,離別之感,無異生人,復致衣一襲、被褥一副,充便上車,去如電逝,須臾至家,家人相見,悲喜推問,知崔是亡人而入其墓,追以懊惋,居四年,三月三日臨水戲,忽見一犢車乍浮乍沒,既上岸,充往開車後戶,見崔氏女與三歲男兒共載,充見之忻然,欲捉其手,女舉手指後車曰「府君見人」,即見少府,充往問訊,女抱兒還充,又與金盌,別,并贈詩曰「煌煌靈芝質,光麗何猗猗,華豔當時顯,嘉異表神奇,含英未及秀,中夏罹霜萎,榮曜長幽滅,世路永無施,不悟陰陽運,哲人忽來儀,會淺離別速,皆由靈與祇,何以贈余親,金盌可頤兒,愛恩從此別,斷絕傷肝脾」,充取兒、盌及詩,忽不見二車處,將兒還,四坐謂是鬼魅,僉遙唾之,形如故,問兒「誰是汝父」,兒逕就充懷,眾初怪惡,傳省其詩,慨然歎死生之玄通也,充詣市賣盌,高舉其價,不欲速售,冀有識者,欻有一老婢,問充得盌之由,還報其大家,即女姨也,遣視之,果是,謂充曰「我姨姊,崔少府女,未嫁而亡,家親痛之,贈一金盌著棺中,今視卿盌甚似,得盌本末可得聞不」,充以事對,即詣充家迎兒,兒有崔氏狀,又似充貌,姨曰「我舅甥三月末間産,父曰『春煗溫也,願休強也』,即字溫休,溫休,蓋幽婚也,其兆先彰矣」,兒遂成為令器,歷數郡二千石,皆著績,其後生植,為漢尚書,植子毓,為魏司空,冠蓋相承至今也。}\textbf{議者疑二陸優劣,謝公以此定之。}

\subsection*{19}

\textbf{羊忱性甚貞烈,趙王倫為相國,忱為太傅長史,乃版以參相國軍事,使者卒至,忱深懼豫禍,不暇被馬,於是帖騎而避,使者追之,忱善射,矢左右發,使者不敢進,遂得免。}{\footnotesize \textbf{文字志}曰忱,字長和,一名陶,泰山平陽人,世為冠族,父繇,車騎掾,忱歷太傅長史、揚州刺史,遷侍中,永嘉五年,遭亂被害,年五十餘。}

\subsection*{20}

\textbf{王太尉不與庾子嵩交,}{\footnotesize 王夷甫、庾敳。}\textbf{庾卿之不置,王曰:「君不得為爾。」庾曰:「卿自君我,我自卿卿,我自用我法,卿自用卿法。」}

\subsection*{21}

\textbf{阮宣子伐社樹,}{\footnotesize 阮脩已見。\textbf{春秋傳}曰共工氏有子曰句龍,為后土,后土為社。\textbf{風俗通}曰孝經稱「社者,土也」,廣博不可備敬,故封土以為社而祀之,報功也。然則社自祀句龍,非土之祭也。}\textbf{有人止之,宣子曰:「社而為樹,伐樹則社亡,樹而為社,伐樹則社移矣。」}

\subsection*{22}

\textbf{阮宣子論鬼神有無者,或以人死有鬼,宣子獨以為無,曰:「今見鬼者云,著生時衣服,若人死有鬼,衣服復有鬼邪?」}{\footnotesize \textbf{論衡}曰世謂人死為鬼,非也,人死不為鬼,無知,不能害人,如審鬼者死人精神,人見之宜從裸袒之形,無為見衣帶被服也,何則?衣無精神也,由此言之,見衣服象人,則形體亦象人,象人,知非死人之精神也,凡天地之間有鬼,非人死之精神也。}

\subsection*{23}

\textbf{元皇帝既登阼,以鄭后之寵,欲舍明帝而立簡文,時議者咸謂:「舍長立少,既於理非倫,且明帝以聰亮英斷,益宜為儲副。」周、王諸公並苦爭懇切,}{\footnotesize \textbf{中興書}曰鄭太后,字阿春,滎陽人,少孤,先嫁田氏,夫亡,依舅吳氏,時中宗敬后虞氏先崩,將納吳氏,后與吳氏女遊後園,有言之於中宗者,納為夫人,甚寵,生簡文,帝即位,尊之曰文宣太后。}\textbf{唯刁玄亮獨欲奉少主,以阿帝旨,元帝便欲施行,慮諸公不奉詔,於是先喚周侯、丞相入,然後欲出詔付刁,}{\footnotesize 刁協。}\textbf{周、王既入,始至階頭,帝逆遣傳詔,遏使就東廂,周侯未悟,即卻略下階,丞相披撥傳詔,徑至御牀前曰:「不審陛下何以見臣?」帝默然無言,乃探懷中黃紙詔裂擲之,由此皇儲始定,周侯方慨然愧歎曰:「我常自言勝茂弘,今始知不如也。」}{\footnotesize \textbf{中興書}曰元皇以明帝及琅邪王裒並非敬后所生,而謂裒有大成之度,勝於明帝,因從容問王導曰「立子以德不以年,今二子孰賢」,導曰「世子、宣城俱有爽明之德,莫能優劣,如此,故當以年」,於是更封裒為琅邪王。而此與世說互異,然法盛採摭典故,以何為實?且從容諷諫,理或可安,豈有登階一言,曾無奇說,便為之改計乎?}

\subsection*{24}

\textbf{王丞相初在江左,欲結援吳人,請婚陸太尉,對曰:「培塿無松柏,薰蕕不同器,}{\footnotesize \textbf{杜預左傳注}曰培塿,小阜,松柏,大木也,薰,香草,蕕,臭草。}\textbf{玩雖不才,義不為亂倫之始。」}{\footnotesize 玩已見。}

\subsection*{25}

\textbf{諸葛恢大女適太尉庾亮兒,}{\footnotesize \textbf{恢別傳}曰恢,字道明,琅邪陽都人,祖誕,司空,父靚,亦知名,恢少有令問,稱為明賢,避難江左,中宗召補主簿,累遷尚書令。\textbf{庾氏譜}曰庾亮子會,娶恢女,名文彪。庾會別見。}\textbf{次女適徐州刺史羊忱兒,}{\footnotesize \textbf{羊氏譜}曰羊楷,字道茂,祖繇,車騎掾,父忱,侍中,楷仕至尚書郎,娶諸葛恢次女。}\textbf{亮子被蘇峻害,改適江虨,}{\footnotesize 虨別見。}\textbf{恢兒娶鄧攸女,}{\footnotesize \textbf{諸葛氏譜}曰恢子衡,字峻文,仕至滎陽太守,娶河南鄧攸女。}\textbf{于時謝尚書求其小女婚,恢乃云:「羊鄧是世婚,江家我顧伊,庾家伊顧我,不能復與謝裒兒婚。」}{\footnotesize \textbf{永嘉流人名}曰裒,字幼儒,陳郡人,父衡,博士,裒歷侍中、吏部尚書、吳國內史。}\textbf{及恢亡,遂婚,}{\footnotesize \textbf{謝氏譜}曰裒子石,娶恢小女,名文熊。\textbf{中興書}曰石,字石奴,歷尚書令,聚斂無厭,取譏當世。}\textbf{於是王右軍往謝家看新婦,猶有恢之遺法,威儀端詳,容服光整,王歎曰:「我在遣女裁得爾耳。」}

\subsection*{26}

\textbf{周叔治作晉陵太守,周侯、仲智往別,叔治以將別,涕泗不止,仲智恚之曰:「斯人乃婦女,與人別惟啼泣。」便舍去,}{\footnotesize \textbf{鄧粲晉紀}曰周謨,字叔治,顗次弟也,仕至中護軍。嵩,字仲智,謨兄也,性絞直果俠,毎以才氣陵物,顗被害,王敦使人弔焉,嵩曰「亡兄,天下有義人,為天下無義人所殺,復何所弔」,敦甚銜之,猶取為從事中郎,因事誅嵩。\textbf{晉陽秋}曰嵩事佛,臨刑猶誦經。}\textbf{周侯獨留,與飲酒言話,臨別流涕,撫其背曰:「阿奴好自愛。」}{\footnotesize 阿奴,謨小字。}

\subsection*{27}

\textbf{周伯仁為吏部尚書,在省內夜疾危急,時刁玄亮為尚書令,營救備親好之至,良久小損,}{\footnotesize \textbf{虞預晉書}曰刁協,字玄亮,渤海饒安人,少好學,雖不研精,而多所博涉,中興制度,皆稟於協,累遷尚書令,中宗信重之,為王敦所忌,舉兵討之,奔至江南,敗死。}\textbf{明旦,報仲智,仲智狼狽來,始入戶,刁下牀對之大泣,說伯仁昨危急之狀,仲智手批之,刁為辟易於戶側,既前,都不問病,直云:「君在中朝,與和長輿齊名,那與佞人刁協有情?」逕便出。}

\subsection*{28}

\textbf{王含作廬江郡,貪濁狼籍,王敦護其兄,故於眾坐稱:「家兄在郡定佳,廬江人士咸稱之。」時何充為敦主簿,在坐,正色曰:「充即廬江人,所聞異於此。」敦默然,旁人為之反側,充晏然,神意自若。}{\footnotesize \textbf{中興書}曰王敦以震主之威,收羅賢儁,辟充為主簿,充知敦有異志,逡巡疏外,及敦稱含有惠政,一坐畏敦,撃節而已,充獨抗之,其時眾人為之失色,由是忤敦,出為東海王文學。}

\subsection*{29}

\textbf{顧孟著嘗以酒勸周伯仁,伯仁不受,顧因移勸柱,而語柱曰:「詎可便作棟梁自遇?」周得之欣然,遂為衿契。}{\footnotesize \textbf{徐廣晉紀}曰顧顯,字孟著,吳郡人,驃騎榮兄子,少有重名,泰興中為騎郎,蚤卒,時為悼惜之。}

\subsection*{30}

\textbf{明帝在西堂,會諸公飲酒,未大醉,帝問:「今名臣共集,何如堯舜時?」周伯仁為僕射,因厲聲曰:「今雖同人主,復那得等於聖治?」帝大怒,還內,作手詔滿一黃紙,遂付廷尉令收,因欲殺之,}{\footnotesize \textbf{按}明帝未即位,顗已為王敦所殺,此說非也。}\textbf{後數日,詔出周,群臣往省之,周曰:「近知當不死,罪不足至此。」}

\subsection*{31}

\textbf{王大將軍當下,時咸謂無緣爾,伯仁曰:「今主非堯舜,何能無過?且人臣安得稱兵以向朝廷?處仲狼抗剛愎,王平子何在?」}{\footnotesize \textbf{顗別傳}曰王敦討劉隗,時溫太真為東宮庶子,在承華門外,與顗相見,曰「大將軍此舉有在,義無有濫」,顗曰「君年少,希更事,未有人臣若此而不作亂,共相推戴數年而為此者乎?處仲狼抗而強忌,平子何在」。\textbf{晉陽秋}曰王澄為荊州,群賊並起,乃奔豫章,而恃其宿名,猶陵侮敦,敦使勇士路戎等搤而殺之。\textbf{裴子}曰平子從荊州下,大將軍伺欲殺之,而平子左右有二十人,甚健,皆持鐵楯馬鞭,平子恆持玉枕,大將軍乃犒荊州文武,二十人積飲食,皆不能動,乃借平子玉枕,便持下牀,平子手引大將軍帶絕,與力士鬬甚苦,乃得上屋上,久許而死。}

\subsection*{32}

\textbf{王敦既下,住船石頭,欲有廢明帝意,賓客盈坐,敦知帝聰明,欲以不孝廢之,毎言帝不孝之狀,而皆云「溫太真所說,溫嘗為東宮率,後為吾司馬,甚悉之」,須臾,溫來,敦便奮其威容,問溫曰:「皇太子作人何似?」溫曰:「小人無以測君子。」敦聲色並厲,欲以威力使從己,乃重問溫:「太子何以稱佳?」溫曰:「鉤深致遠,蓋非淺識所測,然以禮侍親,可稱為孝。」}{\footnotesize \textbf{劉謙之晉紀}曰敦欲廢明帝,言於眾曰「太子子道有虧,溫司馬昔在東宮悉其事」,嶠既正言,敦忿而愧焉。}

\subsection*{33}

\textbf{王大將軍既反,至石頭,周伯仁往見之,謂周曰:「卿何以相負?」對曰:「公戎車犯正,下官忝率六軍,而王師不振,以此負公。」}{\footnotesize \textbf{晉陽秋}曰王敦既下,六軍敗績,顗長史郝嘏及左右文武勸顗避難,顗曰「吾備位大臣,朝廷傾撓,豈可草間求活,投身胡虜邪」,乃與朝士詣敦,敦曰「近日戰有餘力不」,對曰「恨力不足,豈有餘邪」。}

\subsection*{34}

\textbf{蘇峻既至石頭,百僚奔散,}{\footnotesize \textbf{王隱晉書}曰峻,字子高,長廣掖人,少有才學,仕郡主簿,舉孝廉,值中原亂,招合流舊六千餘家,結壘本縣,宣示王化,收葬枯骨,遠近感其恩義,咸共宗焉,討王敦有功,封公,遷歷陽太守,峻外營將表曰「鼓自鳴」,峻自斫鼓曰「我鄉里時,有此則空城」,有頃,詔書徵峻,峻曰「臺下云我反,反豈得活邪?我寧山頭望廷尉,不能廷尉望山頭」,乃作亂。\textbf{晉陽秋}曰峻率眾二萬,濟自橫江,至於蔣山,王師敗績。}\textbf{唯侍中鍾雅獨在帝側,或謂鍾曰:「見可而進,知難而退,古之道也,君性亮直,必不容於寇讎,何不用隨時之宜,而坐待其弊邪?」鍾曰:「國亂不能匡,君危不能濟,而各遜遁以求免,吾懼董狐將執簡而進矣。」}

\subsection*{35}

\textbf{庾公臨去,顧語鍾後事,深以相委,鍾曰:「棟折榱崩,誰之責邪?」庾曰:「今日之事,不容復言,卿當期克復之效耳。」鍾曰:「想足下不愧荀林父耳。」}{\footnotesize \textbf{春秋傳}曰楚莊王圍鄭,晉使荀林父率師救鄭,與楚戰於邲,晉師敗績,桓子歸,請死,晉平公將許之,士貞子諫而止,後林父敗赤狄於曲梁,賞桓子狄臣千室,亦賞士伯以瓜衍之田,曰「吾獲狄田,子之功也,微子,吾喪伯氏矣」。}

\subsection*{36}

\textbf{蘇峻時,孔群在橫塘為匡術所逼,王丞相保存術,}{\footnotesize \textbf{會稽後賢記}曰群,字敬休,會稽山陰人,祖竺,吳豫章太守,父奕,全椒令,群有智局,仕至御史中丞。\textbf{晉陽秋}曰匡術為阜陵令,逃亡無行,庾亮征蘇峻,術勸峻誅亮,遂與峻同反,後以宛城降。}\textbf{因眾坐戲語,令術勸酒,以釋橫塘之憾,群答曰:「德非孔子,厄同匡人,}{\footnotesize \textbf{家語}曰孔子之宋,匡簡子以甲士圍之,子路怒,奮戟將戰,孔子止之曰「夫詩書之不講,禮樂之不習,是丘之過也,若述先王之道而為咎者,非丘罪也,命也夫!歌,予和汝」,子路彈劍,孔子和之,曲三終,匡人解甲罷。}\textbf{雖陽和布氣,鷹化為鳩,至於識者,猶憎其眼。」}{\footnotesize \textbf{禮記月令}曰仲春之月,鷹化為鳩。\textbf{鄭玄}曰鳩,播穀也。\textbf{夏小正}曰鷹則為鳩,鷹也者,其殺之時也,鳩也者,非殺之時也,善變而之仁,故具之。}

\subsection*{37}

\textbf{蘇子高事平,}{\footnotesize \textbf{靈鬼志謠徵}曰明帝初,有謠曰「高山崩,石自破」,高山,峻也,碩,峻弟也,後諸公誅峻,碩猶據石頭,潰散而逃,追斬之。}\textbf{王、庾諸公欲用孔廷尉為丹陽,}{\footnotesize 孔坦。}\textbf{亂離之後,百姓彫弊,孔慨然曰:「昔肅祖臨崩,諸君親升御牀,並蒙眷識,共奉遺詔,孔坦疏賤,不在顧命之列,既有艱難,則以微臣為先,今猶俎上腐肉,任人膾截耳。」於是拂衣而去,諸公亦止。}{\footnotesize \textbf{按}王隱晉書「蘇峻事平,陶侃欲將坦上,用為豫章太守,坦辭母老不行,臺以為吳郡,吳郡多名族,而坦年少,乃授吳興內史」,不聞尹京。}

\subsection*{38}

\textbf{孔車騎與中丞共行,}{\footnotesize \textbf{孔愉別傳}曰愉,字敬康,會稽山陰人,初辟中宗參軍,討華軼有功,封餘不亭侯,愉少時嘗得一龜,放於餘不溪中,龜於路左顧者數過,及後鑄印,而龜左顧,更鑄猶如此,印師以聞,愉悟,取而佩焉,累遷尚書左僕射,贈車騎將軍。中丞,孔群也。}\textbf{在御道逢匡術,賓從甚盛,因往與車騎共語,中丞初不視,直云:「鷹化為鳩,眾鳥猶惡其眼。」術大怒,便欲刃之,車騎下車,抱術曰:「族弟發狂,卿為我宥之。」始得全首領。}

\subsection*{39}

\textbf{梅頤嘗有惠於陶公,後為豫章太守,有事,王丞相遣收之,侃曰:「天子富於春秋,萬機自諸侯出,王公既得錄,陶公何為不可放?」乃遣人於江口奪之,}{\footnotesize \textbf{晉諸公贊}曰頤,字仲真,汝南西平人,少以學隱退,而才實進止。\textbf{永嘉流人名}曰頤,領軍司馬,頤弟陶,字叔真。\textbf{鄧粲晉紀}曰初,有譖侃於王敦者,乃以從弟廙代侃為荊州,左遷侃廣州,侃文武距廙而求侃,敦聞大怒,及侃將蒞廣州,過敦,敦陳兵欲害侃,敦咨議參軍梅陶諫敦,乃止,厚禮而遣之。\textbf{王隱晉書}亦同。\textbf{按}二書所敘,則有惠於陶是梅陶,非頤也。}\textbf{頤見陶公,拜,陶公止之,頤曰:「梅仲真厀,明日豈可復屈邪?」}

\subsection*{40}

\textbf{王丞相作女伎,施設牀席,蔡公先在坐,不說而去,王亦不留。}{\footnotesize \textbf{蔡司徒別傳}曰謨,字道明,濟陽考城人,博學有識,避地江左,歷左光祿、錄尚書事、揚州刺史,薨,贈司空。}

\subsection*{41}

\textbf{何次道、庾季堅二人並為元輔,}{\footnotesize \textbf{晉陽秋}曰庾冰,字季堅,太尉亮之弟也,少有檢操,兄亮常器之,曰「吾家晏平仲」,累遷車騎將軍、江州刺史。}\textbf{成帝初崩,於時嗣君未定,何欲立嗣子,庾及朝議以外寇方強,嗣子沖幼,乃立康帝,}{\footnotesize \textbf{中興書}曰帝諱岳,字世同,成帝同母弟也,成帝崩,即位,年二十二。}\textbf{康帝登阼,會群臣,謂何曰:「朕今所以承大業,為誰之議?」何答曰:「陛下龍飛,此是庾冰之功,非臣之力,于時用微臣之議,今不覩盛明之世。」}{\footnotesize \textbf{晉陽秋}曰初,顯宗臨崩,庾冰議立長君,何充謂宜奉皇子,爭之不得,充不自安,求處外任,及冰出鎮武昌,充自京馳還,言於帝曰「冰不宜出,昔年陛下龍飛,使晉德再隆者,冰之勳也,臣無與焉」。}\textbf{帝有慙色。}

\subsection*{42}

\textbf{江僕射年少,王丞相呼與共棊,王手嘗不如兩道許,而欲敵道戲,試以觀之,江不即下,王曰:「君何以不行?」江曰:「恐不得爾。」}{\footnotesize \textbf{徐廣晉紀}曰江虨,字思玄,陳留人,博學知名,兼善弈,為中興之冠,累遷尚書左僕射、護軍將軍。}\textbf{傍有客曰:「此年少戲迺不惡。」王徐舉首曰:「此年少非惟圍棊見勝。」}{\footnotesize \textbf{范汪棊品}曰虨與王恬等,棊第一品,導第五品。}

\subsection*{43}

\textbf{孔君平疾篤,庾司空為會稽,省之,}{\footnotesize 庾冰。}\textbf{相問訊甚至,為之流涕,庾既下牀,孔慨然曰:「大丈夫將終,不問安國寧家之術,迺作兒女子相問。」庾聞,迴謝之,請其話言。}{\footnotesize \textbf{王隱晉書}曰坦方直而有雅望。}

\subsection*{44}

\textbf{桓大司馬詣劉尹,臥不起,桓彎彈彈劉枕,丸迸碎牀褥間,劉作色而起曰:「使君!如馨地寧可鬬戰求勝?」}{\footnotesize \textbf{中興書}曰溫曾為徐州刺史。沛國屬徐州,故呼溫使君。鬬戰者,以溫為將也。}\textbf{桓甚有恨容。}{\footnotesize 劉尹,真長,已見。}

\subsection*{45}

\textbf{後來年少多有道深公者,深公謂曰:「黃吻年少,勿為評論宿士,昔嘗與元明二帝、王庾二公周旋。」}{\footnotesize \textbf{高逸沙門傳}曰晉元、明二帝游心玄虛,託情道味,以賓友禮待法師,王公、庾公傾心側席,好同臭味也。}

\subsection*{46}

\textbf{王中郎年少時,}{\footnotesize 坦之已見。}\textbf{江虨為僕射,領選,欲擬之為尚書郎,有語王者,王曰:「自過江來,尚書郎正用第二人,何得擬我?」江聞而止。}{\footnotesize \textbf{按}王彪之別傳曰「彪之從伯導謂彪之曰『選曹舉汝為尚書郎,幸可作諸王佐邪』」,此知郎官,寒素之品也。}

\subsection*{47}

\textbf{王述轉尚書令,事行便拜,文度曰:「故應讓杜許。」藍田云:「汝謂我堪此不?」文度曰:「何為不堪?但克讓自是美事,恐不可闕。」藍田慨然曰:「既云堪,何為復讓?人言汝勝我,定不如我。」}{\footnotesize \textbf{述別傳}曰述常以為人之處世,當先量己而後動,義無虛讓,是以應辭便當固執,其貞正不踰皆此類。}

\subsection*{48}

\textbf{孫興公作庾公誄,文多託寄之辭,}{\footnotesize \textbf{綽集}載誄文曰咨予與公,風流同歸,擬量託情,視公猶師,君子之交,相與無私,虛中納是,吐誠悔非,雖實不敏,敬佩弦韋,永戢話言,口誦心悲。}\textbf{既成,示庾道恩,庾見,慨然送還之,曰:「先君與君,自不至於此。」}{\footnotesize 道恩,庾羲小字。\textbf{徐廣晉紀}曰羲,字叔和,太傅亮第三子,拔尚率到,位建威將軍、吳國內史。}

\subsection*{49}

\textbf{王長史求東陽,撫軍不用,}{\footnotesize 簡文。}\textbf{後疾篤,臨終,撫軍哀歎曰:「吾將負仲祖於此。」命用之,長史曰:「人言會稽王癡,真癡。」}{\footnotesize 王濛已見。}

\subsection*{50}

\textbf{劉簡作桓宣武別駕,後為東曹參軍,}{\footnotesize \textbf{劉氏譜}曰簡,字仲約,南陽人,祖喬,豫州刺史,父挺,潁川太守,簡仕至大司馬參軍。}\textbf{頗以剛直見疏,嘗聽訊,簡都無言,宣武問:「劉東曹何以不下意?」答曰:「會不能用。」宣武亦無怪色。}

\subsection*{51}

\textbf{劉真長、王仲祖共行,日旰未食,有相識小人貽其餐,肴案甚盛,真長辭焉,仲祖曰:「聊以充虛,何苦辭?」真長曰:「小人都不可與作緣。」}{\footnotesize 孔子稱「惟女子與小人為難養,近之則不遜,遠之則怨」,劉尹之意,蓋從此言也。}

\subsection*{52}

\textbf{王脩齡嘗在東山,甚貧乏,}{\footnotesize 司州已見。}\textbf{陶胡奴為烏程令,}{\footnotesize 胡奴,陶範小字也。\textbf{陶侃別傳}曰範,字道則,侃第十子也,侃諸子中最知名,歷尚書、祕書監。\textbf{何法盛}以為第九子。}\textbf{送一船米遺之,卻不肯取,直答語:「王脩齡若飢,自當就謝仁祖索食,不須陶胡奴米。」}

\subsection*{53}

\textbf{阮光祿}{\footnotesize 阮裕已見。}\textbf{赴山陵,至都,不往殷、劉許,過事便還,諸人相與追之,阮亦知時流必當逐己,乃遄疾而去,至方山不相及,}{\footnotesize \textbf{中興書}曰裕終日頽然,無所錯綜,而物自宗之。}\textbf{劉尹時索會稽,乃歎曰:「我入,當泊安石渚下耳,不敢復近思曠傍,伊便能捉杖打人,不易。」}

\subsection*{54}

\textbf{王、劉與桓公共至覆舟山看,酒酣後,劉牽腳加桓公頸,桓公甚不堪,舉手撥去,既還,王長史語劉曰:「伊詎可以形色加人不?」}{\footnotesize \textbf{溫別傳}曰溫有豪邁風氣也。}

\subsection*{55}

\textbf{桓公問桓子野:「謝安石料萬石必敗,何以不諫?」}{\footnotesize 子野,桓伊小字也。\textbf{續晉陽秋}曰伊,字叔夏,譙國銍人,父景,護軍將軍,伊少有才藝,又善聲律,加以標悟省率,為王濛、劉惔所知,累遷豫州刺史,贈右將軍。}\textbf{子野答曰:「故當出於難犯耳。」桓作色曰:「萬石撓弱凡才,有何嚴顏難犯?」}

\subsection*{56}

\textbf{羅君章曾在人家,主人令與坐上客共語,答曰:「相識已多,不煩復爾。」}{\footnotesize \textbf{羅府君別傳}曰含,字君章,桂陽耒陽人,蓋楚熊姓之後,啓土羅國,遂氏族焉,後寓湘境,故為桂陽人,含,臨海太守彥曾孫、滎陽太守緩少子也,桓宣武辟為別駕,以官廨諠擾,於城西池小洲上立茅茨,伐木為牀,織葦為席,布衣蔬食,晏若有餘,桓公嘗謂眾坐曰「此自江左之清秀,豈惟荊楚而已」,累遷散騎常侍、廷尉、長沙相,致仕中散大夫,門施行馬,含自在官舍,有一白雀棲集堂宇,及致仕還家,階庭忽蘭菊挺生,豈非至行之徵邪。}

\subsection*{57}

\textbf{韓康伯病,拄杖前庭消搖,}{\footnotesize 韓伯已見。}\textbf{見諸謝皆富貴,轟隱交路,歎曰:「此復何異王莽時?」}{\footnotesize \textbf{漢書}曰王莽宗族凡十侯、五大司馬,外戚莫盛焉。}

\subsection*{58}

\textbf{王文度為桓公長史,桓為兒求王女,王許咨藍田。}{\footnotesize 王坦之、王述並已見。}\textbf{既還,藍田愛念文度,雖長大猶抱著厀上,文度因言桓求己女婚,藍田大怒,排文度下厀,曰:「惡見文度已復癡,畏桓溫面?兵,那可嫁女與之?」文度還報云:「下官家中先得婚處。」桓公曰:「吾知矣,此尊府君不肯耳。」後桓女遂嫁文度兒。}{\footnotesize \textbf{王氏譜}曰坦之子愷,娶桓溫第二女,字伯子。\textbf{中興書}曰愷,字茂仁,歷吳國內史、丹陽尹,贈太常。}

\subsection*{59}

\textbf{王子敬數歲時,嘗看諸門生樗蒲,見有勝負,因曰:「南風不競。」}{\footnotesize \textbf{春秋傳}曰楚伐鄭,師曠曰「不害,吾驟歌南風,南風不競,多死聲,楚必無功」。\textbf{杜預}曰歌者吹律,以詠八風,南風音微,故曰不競也。}\textbf{門生輩輕其小兒,迺曰:「此郎亦管中窺豹,時見一斑。」子敬瞋目曰:「遠慙荀奉倩,近愧劉真長。」遂拂衣而去。}{\footnotesize 荀、劉已見。}

\subsection*{60}

\textbf{謝公聞羊綏佳,致意令來,終不肯詣,}{\footnotesize \textbf{羊氏譜}曰綏,字仲彥,太山人,父楷,尚書郎,綏仕至中書侍郎。}\textbf{後綏為太學博士,因事見謝公,公即取以為主簿。}

\subsection*{61}

\textbf{王右軍與謝公詣阮公,}{\footnotesize 阮思曠也。}\textbf{至門語謝:「故當共推主人。」謝曰:「推人正自難。」}

\subsection*{62}

\textbf{太極殿始成,}{\footnotesize \textbf{徐廣晉紀}曰孝武寧康二年,尚書令王彪之等啓改作新宮,太元三年二月,內外軍六千人始營築,至七月而成,太極殿高八丈,長二十七丈,廣十丈,尚書謝萬監視,賜爵關內侯,大匠毛安之,關中侯。}\textbf{王子敬時為謝公長史,謝送版,使王題之,王有不平色,語信云:「可擲著門外。」謝後見王曰:「題之上殿何若?昔魏朝韋誕諸人,亦自為也。」王曰:「魏阼所以不長。」謝以為名言。}{\footnotesize \textbf{宋明帝文章志}曰太元中,新宮成,議者欲屈王獻之題榜,以為萬代寶,謝安與王語次,因及魏時起陵雲閣忘題榜,乃使韋仲將縣梯上題之,比下,鬚髮盡白,裁餘氣息,還語子弟云「宜絕楷法」,安欲以此風動其意,王解其旨,正色曰「此奇事,韋仲將魏朝大臣,寧可使其若此?有以知魏德之不長」,安知其心,迺不復逼之。}

\subsection*{63}

\textbf{王恭欲請江盧奴為長史,晨往詣江,江猶在帳中,王坐,不敢即言,良久乃得及,江不應,}{\footnotesize 盧奴,江敳小字也。\textbf{晉安帝紀}曰敳,字仲凱,濟陽人,祖正,散騎常侍,父虨,僕射,並以義正器素,知名當世,敳歷位內外,簡退著稱,歷黃門侍郎、驃騎咨議。}\textbf{直喚人取酒,自飲一盌,又不與王,王且笑且言:「那得獨飲?」江云:「卿亦復須邪?」更使酌與王,王飲酒畢,因得自解去,未出戶,江歎曰:「人自量,固為難。」}{\footnotesize \textbf{宋書}曰敳即湘州江夷之父也,夷,字茂遠,湘州刺史。}

\subsection*{64}

\textbf{孝武問王爽:「卿何如卿兄?」王答曰:「風流秀出,臣不如恭,忠孝亦何可以假人。」}{\footnotesize \textbf{中興書}曰爽忠孝正直,烈宗崩,王國寶夜開門入,為遺詔,爽為黃門郎,距之曰「大行晏駕,太子未立,敢有先入者斬」,國寶懼,乃止。}

\subsection*{65}

\textbf{王爽與司馬太傅飲酒,太傅醉,呼王為小子,王曰:「亡祖長史,與簡文皇帝為布衣之交,亡姑亡姊,伉儷二宮,何小子之有?」}{\footnotesize \textbf{中興書}曰王濛女諱穆之,為哀帝皇后,王蘊女諱法惠,為孝武皇后。}

\subsection*{66}

\textbf{張玄與王建武先不相識,}{\footnotesize 張玄已見。建武,王忱也。\textbf{晉安帝紀}曰忱初作荊州刺史,後為建武將軍。}\textbf{後遇於范豫章許,范令二人共語,}{\footnotesize 范甯已見。}\textbf{張因正坐斂衽,王孰視良久,不對,張大失望,便去,范苦譬留之,遂不肯住,范是王之舅,}{\footnotesize \textbf{王氏譜}曰王坦之娶順陽郡范汪女,名蓋,即甯妹也,生忱。}\textbf{乃讓王曰:「張玄,吳士之秀,亦見遇於時,而使至於此,深不可解。」王笑曰:「張祖希若欲相識,自應見詣。」范馳報張,張便束帶造之,遂舉觴對語,賓主無愧色。}