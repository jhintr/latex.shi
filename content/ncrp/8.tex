\chapter{Lesson 8}

\section*{Reading 1}

Atha kho Venāgapurikā brāhmaṇagahapatikā yena Bhagavā ten’upasaṅkamiṃsu, upasaṅkamitvā app’ekacce Bhagavantaṃ abhivādetvā ekamantaṃ nisīdiṃsu, app’ekacce Bhagavatā saddhiṃ sammodiṃsu ekamantaṃ nisīdiṃsu, app’ekacce nāmagottaṃ sāvetvā ekamantaṃ nisīdiṃsu, app’ekacce tuṇhībhūtā ekamantaṃ nisīdiṃsu. Ekamantaṃ nisinno kho Venāgapuriko Vacchagotto brāhmaṇo Bhagavantaṃ etadavoca

“Acchariyaṃ, bho Gotama, abbhutaṃ, bho Gotama. Yāvañ c’idaṃ bhoto Gotamassa vippasannāni indriyāni, parisuddho chavivaṇṇo pariyodāto. Seyyathāpi, bho Gotama, sāradaṃ badarapaṇḍuṃ parisuddhaṃ hoti pariyodātaṃ, evameva bhoto Gotamassa vippasannāni indriyāni parisuddho chavivaṇṇo pariyodāto. Seyyathāpi, bho Gotama, tālapakkaṃ sampati bandhanā pamuttaṃ parisuddhaṃ hoti pariyodātaṃ, evameva bhoto Gotamassa vippasannāni indriyāni parisuddho chavivaṇṇo pariyodāto.” (A 3.7.3)

\section*{Reading 2}

Tena kho pana samayena Uggatasarīrassa brāhmaṇassa mahāyañño upakkhaṭo hoti. Pañca usabhasatāni thūṇ’ūpanītāni honti yaññatthāya, pañca vacchatarasatāni thūṇ’ūpanītāni honti yaññatthāya, pañca vacchatarisatāni thūṇ’ūpanītāni honti yaññatthāya, pañca ajasatāni thūṇ’ūpanītāni honti yaññatthāya, pañca urabbhasatāni thūṇ’ūpanītāni honti yaññatthāya. Atha kho Uggatasarīro brāhmaṇo yena Bhagavā ten’upasaṅkami, upasaṅkamitvā Bhagavatā saddhiṃ sammodi, ekamantaṃ nisīdi. Ekamantaṃ nisinno kho Uggatasarīro brāhmaṇo Bhagavantaṃ etadavoca

“Sutaṃ m’etaṃ, bho Gotama, aggissa ādānaṃ yūpassa ussāpanaṃ mahapphalaṃ hoti mahānisaṃsan” ti.

“Mayā pi kho etaṃ, brāhmaṇa, sutaṃ aggissa ādānaṃ yūpassa ussāpanaṃ mahapphalaṃ hoti mahānisaṃsan” ti.

Dutiyam pi kho Uggatasarīro brāhmaṇo… pe… tatiyam pi kho Uggatasarīro brāhmaṇo Bhagavantaṃ etadavoca “Sutaṃ m’etaṃ, bho Gotama, aggissa ādānaṃ yūpassa ussāpanaṃ mahapphalaṃ hoti mahānisaṃsan” ti.

“Mayā pi kho etaṃ, brāhmaṇa, sutaṃ aggissa ādānaṃ yūpassa ussāpanaṃ mahapphalaṃ hoti mahānisaṃsan” ti.

“Tayidaṃ, bho Gotama, sameti bhoto c’eva Gotamassa amhākaṃ ca, yadidaṃ sabbena sabbaṃ.”

Evaṃ vutte āyasmā Ānando Uggatasarīraṃ brāhmaṇaṃ etadavoca “Na kho, brāhmaṇa, Tathāgatā evaṃ pucchitabbā ‘sutaṃ mahapphalaṃ hoti mahānisaṃsan’ ti. Evaṃ kho, brāhmaṇa, tathāgatā pucchitabbā ‘ahañ hi, bhante, aggiṃ ādātukāmo, yūpaṃ ussāpetukāmo. Ovadatu maṃ, bhante, bhagavā. Anusāsatu maṃ, bhante, bhagavā yaṃ mama assa dīgharattaṃ hitāya sukhāyā’” ti. (A. 7.5.4)

\section*{Reading 3}

Dunniggahassa lahuno,\\
yatthakāmanipātino,\\
cittassa damatho sādhu,\\
cittaṃ dantaṃ sukhāvahaṃ.

Sududdasaṃ sunipuṇaṃ,\\
yatthakāmanipātinaṃ,\\
cittaṃ rakkhetha medhāvī,\\
cittaṃ guttaṃ sukhāvahaṃ.

Anavaṭṭhitacittassa,\\
saddhammaṃ avijānato,\\
pariplavapasādassa,\\
paññā na paripūrati. (Dhp 3)

Yāvajīvam pi ce bālo,\\
paṇḍitaṃ payirupāsati,\\
na so dhammaṃ vijānāti,\\
dabbī sūparasaṃ yathā.

Muhuttam api ce viññū,\\
paṇḍitaṃ payirupāsati,\\
khippaṃ dhammaṃ vijānāti,\\
jivhā sūparasaṃ yathā.

Na taṃ kammaṃ kataṃ sādhu,\\
yaṃ katvā anutappati,\\
yassa assumukho rodaṃ,\\
vipākaṃ paṭisevati.

Taṃ ca kammaṃ kataṃ sādhu,\\
yaṃ katvā nānutappati,\\
yassa patīto sumano,\\
vipākaṃ paṭisevati. (Dhp 5)

Attānam eva paṭhamaṃ,\\
patirūpe nivesaye,\\
ath’aññam anusāseyya,\\
na kilisseyya paṇḍito. (Dhp 12)

\section*{Further Reading 1}

Ekaṃ samayaṃ Bhagavā Vesāliyaṃ viharati Mahāvane Kūṭāgārasālāyaṃ. Atha kho Sīho senāpati yena Bhagavā ten’upasaṅkami, upasaṅkamitvā Bhagavantaṃ abhivādetvā ekamantaṃ nisīdi. Ekamantaṃ nisinno kho Sīho senāpati Bhagavantaṃ etadavoca “Sakkā nu kho, bhante, Bhagavā sandiṭṭhikaṃ dānaphalaṃ paññāpetuṃ” ti?

“Sakkā, Sīhā” ti bhagavā avoca “dāyako, Sīha, dānapati bahuno janassa piyo hoti manāpo. Yaṃ pi, Sīha, dāyako dānapati bahuno janassa piyo hoti manāpo, idaṃ pi sandiṭṭhikaṃ dānaphalaṃ.”

“Puna ca paraṃ, Sīha, dāyakaṃ dānapatiṃ santo sappurisā bhajanti. Yaṃ pi, Sīha, dāyakaṃ dānapatiṃ santo sappurisā bhajanti, idaṃ pi sandiṭṭhikaṃ dānaphalaṃ.”

“Puna ca paraṃ, Sīha, dāyakassa dānapatino kalyāṇo kittisaddo abbhuggacchati. Yaṃ pi, Sīha, dāyakassa dānapatino kalyāṇo kittisaddo abbhuggacchati, idaṃ pi sandiṭṭhikaṃ dānaphalaṃ.”

“Puna ca paraṃ, Sīha, dāyako dānapati yaṃ yadeva parisaṃ upasaṅkamati yadi khattiyaparisaṃ yadi brāhmaṇaparisaṃ yadi gahapatiparisaṃ yadi samaṇaparisaṃ visārado upasaṅkamati amaṅkubhūto. Yaṃ pi, Sīha, dāyako dānapati yaṃ yadeva parisaṃ upasaṅkamati yadi khattiyaparisaṃ yadi brāhmaṇaparisaṃ yadi gahapatiparisaṃ yadi samaṇaparisaṃ visārado upasaṅkamati amaṅkubhūto, idaṃ pi sandiṭṭhikaṃ dānaphalaṃ.”

“Puna ca paraṃ, Sīha, dāyako dānapati kāyassa bhedā paraṃ maraṇā sugatiṃ saggaṃ lokaṃ upapajjati. Yaṃ pi, Sīha, dāyako dānapati kāyassa bhedā paraṃ maraṇā sugatiṃ saggaṃ lokaṃ upapajjati, idaṃ samparāyikaṃ dānaphalaṃ” ti. (A 5.4.4 Sīhasenāpatisuttaṃ)

\section*{Further Reading 2}

Ekaṃ samayaṃ Bhagavā Vesāliyaṃ viharati Mahāvane Kūṭāgārasālāyaṃ. Atha kho Mahāli Licchavi yena Bhagavā ten’upasaṅkami, upasaṅkamitvā Bhagavantaṃ abhivādetvā ekamantaṃ nisīdi. Ekamantaṃ nisinno kho Mahāli Licchavi Bhagavantaṃ etadavoca “Ko nu kho, bhante, hetu, ko paccayo pāpassa kammassa kiriyāya, pāpassa kammassa pavattiyā” ti?

“Lobho kho, Mahāli, hetu, lobho paccayo pāpassa kammassa kiriyāya, pāpassa kammassa pavattiyā. Doso kho, Mahāli, hetu, doso paccayo pāpassa kammassa kiriyāya pāpassa kammassa pavattiyā. Moho kho, Mahāli, hetu, moho paccayo pāpassa kammassa kiriyāya pāpassa kammassa pavattiyā. Ayoniso manasikāro kho, Mahāli, hetu, ayonisomanasikāro paccayo pāpassa kammassa kiriyāya pāpassa kammassa pavattiyā. Micchāpaṇihitaṃ kho, Mahāli, cittaṃ hetu, micchāpaṇihitaṃ cittaṃ paccayo pāpassa kammassa kiriyāya pāpassa kammassa pavattiyā ti. Ayaṃ kho, mahāli, hetu, ayaṃ paccayo pāpassa kammassa kiriyāya pāpassa kammassa pavattiyā” ti. (A 10.5.7 Mahālisuttaṃ)

\section*{Further Reading 3}

Akkodhano ’nupanāhī,\\
amāyo rittapesuṇo,\\
sa ve tādisako bhikkhu,\\
evaṃ pecca na socati.

Akkodhano ’nupanāhī,\\
amāyo rittapesuṇo,\\
guttadvāro sadā bhikkhu,\\
evaṃ pecca na socati.

Akkodhano ’nupanāhī,\\
amāyo rittapesuṇo,\\
kalyāṇasīlo so bhikkhu,\\
evaṃ pecca na socati.

Akkodhano ’nupanāhī,\\
amāyo rittapesuṇo,\\
kalyāṇamitto so bhikkhu,\\
evaṃ pecca na socati.

Akkodhano ’nupanāhī,\\
amāyo rittapesuṇo,\\
kalyāṇapañño so bhikkhu,\\
evaṃ pecca na socati. (Thg 8.2 Sirimittattheragāthā)

\section*{Further Reading 4}

Rājā āha “Bhante Nāgasena, yo idha kālaṅkato Brahmaloke uppajjeyya, yo ca idha kālaṅkato Kasmīre uppajjeyya, ko cirataraṃ ko sīghataran” ti?

“Samakaṃ, mahārājā” ti.

“Opammaṃ karohī” ti.

“Kuhiṃ pana, mahārāja, tava jātanagaran” ti?

“Atthi, bhante, Kalasigāmo nāma, tatthāhaṃ jāto” ti.

“Kīva dūro, mahārāja, ito Kalasigāmo hotī” ti.

“Dvimattāni, bhante, yojanasatānī” ti.

“Kīva dūraṃ, mahārāja, ito Kasmīraṃ hotī” ti?

“Dvādasa, bhante, yojanānī” ti.

“Iṅgha, tvaṃ mahārāja, Kalasigāmaṃ cintehī” ti.

“Cintito, bhante” ti.

“Iṅgha, tvaṃ mahārāja, Kasmīraṃ cintehī” ti.

“Cintitaṃ bhante” ti.

“Katamaṃ nu kho, mahārāja, cirena cintitaṃ, katamaṃ sīghataran” ti?

“Samakaṃ bhante” ti.

“Evameva kho, mahārāja, yo idha kālaṅkato Brahmaloke uppajjeyya, yo ca idha kālaṅkato Kasmīre uppajjeyya, samakaṃ yeva uppajjantī” ti.

“Bhiyyo opammaṃ karohī” ti.

“Taṃ kiṃ maññasi, mahārāja, dve sakuṇā ākāsena gaccheyyuṃ, tesu eko ucce rukkhe nisīdeyya, eko nīce rukkhe nisīdeyya, tesaṃ samakaṃ patiṭṭhitānaṃ katamassa chāyā paṭhamataraṃ pathaviyaṃ patiṭṭhaheyya, katamassa chāyā cirena pathaviyaṃ patiṭṭhaheyyā” ti?

“Samakaṃ, bhante” ti.

“Evameva kho, mahārāja, yo idha kālaṅkato Brahmaloke uppajjeyya, yo ca idha kālaṅkato Kasmīre uppajjeyya, samakaṃ yeva uppajjantī” ti.

(Miln III.7.5 Dvinnaṃlokuppannānaṃamakabhāvapañho)