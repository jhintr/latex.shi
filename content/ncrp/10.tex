\chapter{Lesson 10}

\section*{Reading 1}

Ekasmiṃ samaye satthā gaṇaṃ pahāya ekako va ekaṃ vanaṃ pāvisi. Pārileyyakanāmo eko hatthirājāpi hatthigaṇaṃ pahāya taṃ vanaṃ pavisitvā, bhagavantaṃ ekassa rukkhassa mūle nisinnaṃ disvā, pādena paharanto rukkhamūlaṃ sodhetvā soṇḍāya sākhaṃ gahetvā sammajji. Tato paṭṭhāya divase divase soṇḍāya ghaṭaṃ gahetvā pānīyaparibhojanīyaṃ udakaṃ āharati upaṭṭhāpeti, uṇhodakena atthe sati uṇhodakaṃ paṭiyādeti. Kathaṃ? Kaṭṭhāni ghaṃsitvā aggiṃ pāteti, tattha dārūni pakkhipanto aggiṃ jāletvā tattha tattha pāsāṇe pacitvā, dārukkhaṇḍakena pavaṭṭetvā khuddakasoṇḍiyaṃ khipati. Tato hatthaṃ otāretvā udakassa tattabhāvaṃ jānitvā gantvā satthāraṃ vandati. Satthā tattha gantvā nahāyati. Atha nānāvidhāni phalāni āharitvā deti.

Yadā pana satthā gāmaṃ piṇḍāya pavisati, tadā satthu pattacīvaram ādāya kumbhe ṭhapetvā satthārā saddhiṃ yeva gacchati, rattiṃ vāḷamiganivāraṇatthaṃ mahantaṃ daṇḍaṃ soṇḍāya gahetvā yāva aruṇ’uggamanā vanasaṇḍe vicarati. (Rasv)

\section*{Reading 2}

Atīte kira bārāṇasiyaṃ sālittakasippe nipphattiṃ patto eko pīṭhasappi ahosi. So nagaradvāre ekassa vaṭarukkhassa heṭṭhā nisinno sakkharāni khipitvā tassa paṇṇāni chindanto “hatthirūpakaṃ no dassehi, assarūpakaṃ no dassehī” ti gāmadārakehi vuccamāno icchiticchitāni rūpāni dassetvā tesaṃ santikā khādanīyādīni labhati.

Ath’ekadivasaṃ rājā uyyānaṃ gacchanto taṃ padesaṃ pāpuṇi. Dārakā pīṭhasappiṃ pāroh’antare katvā palāyiṃsu. Rañño ṭhitamajjhantike rukkhamūlaṃ paviṭṭhassa chiddacchāyā sarīraṃ phari. So “kiṃ nu kho etan” ti uddhaṃ olokento rukkhapaṇṇesu hatthirūpakādīni disvā “kass’etaṃ kamman” ti pucchitvā “pīṭhasappino” ti sutvā taṃ pakkosāpetvā āha “mayhaṃ purohito atimukharo appamattake pi vutte bahuṃ bhaṇanto maṃ upaddavati, sakkhissasi tassa mukhe nāḷimattā ajalaṇḍikā khipitun” ti? “Sakkhissāmi, deva. Ajalaṇḍikā āharāpetvā purohitena saddhiṃ tumhe antosāṇiyaṃ nisīdatha, aham ettha kattabbaṃ jānissāmī” ti.

Rājā tathā kāresi. Itaro pi kattariy’aggena sāṇiyaṃ chiddaṃ katvā, purohitassa raññā saddhiṃ kathentassa mukhe vivaṭamatte ekekaṃ ajalaṇḍikaṃ khipi. Purohito mukhaṃ paviṭṭhaṃ paviṭṭhaṃ gili. Pīṭhasappī khīṇāsu ajalaṇḍikāsu sāṇiṃ cālesi. Rājā tāya saññāya ajalaṇḍikānaṃ khīṇabhāvaṃ ñatvā āha “ācariya, ahaṃ tumhehi saddhiṃ kathento kathaṃ nittharituṃ na sakkhissāmi. Tumhe atimukharatāya nāḷimattā ajalaṇḍikā gilantā pi tuṇhībhāvaṃ nāpajjathā” ti.

Brāhmaṇo maṅkubhāvaṃ āpajjitvā tato paṭṭhāya mukhaṃ vivaritvā raññā saddhiṃ sallapituṃ nāsakkhi. Rājā pīṭhasappiṃ pakkosāpetvā “taṃ nissāya me sukhaṃ laddhan” ti tuṭṭho tassa sabbaṭṭhakaṃ nāma dhanaṃ datvā nagarassa catūsu disāsu cattāro varagāme adāsi. (DhpA II.70)

\section*{Reading 3}

Yathāgāraṃ ducchannaṃ,\\
vuṭṭhī samativijjhati,\\
evaṃ abhāvitaṃ cittaṃ,\\
rāgo samativijjhati.

Yathāgāraṃ suchannaṃ,\\
vuṭṭhī na samativijjhati,\\
evaṃ subhāvitaṃ cittaṃ,\\
rāgo na samativijjhati.

Idha socati pecca socati,\\
pāpakārī ubhayattha socati,\\
so socati so vihaññati,\\
disvā kammakiliṭṭham attano.

Idha modati pecca modati,\\
katapuñño ubhayattha modati,\\
so modati so pamodati,\\
disvā kammavisuddhim attano.

Idha tappati pecca tappati,\\
pāpakārī ubhayattha tappati,\\
“pāpaṃ me katan” ti tappati,\\
bhiyyo tappati duggatiṃ gato.

Idha nandati pecca nandati,\\
katapuñño ubhayattha nandati,

“puññaṃ me katan” ti nandati,\\
bhiyyo nandati suggatiṃ gato. (Dhp 1)

\section*{Further Reading 1}

Ath’eko makkaṭo taṃ hatthiṃ divase divase tathāgatassa upaṭṭhānaṃ karontaṃ disvā “aham pi kiñcideva karissāmī” ti vicaranto ekadivasaṃ nimmakkhikaṃ daṇḍakamadhuṃ disvā daṇḍakaṃ bhañjitvā daṇḍaken’eva saddhiṃ madhupaṭalaṃ satthu santikaṃ āharitvā kadalipattaṃ chinditvā tattha ṭhapetvā adāsi. Satthā gaṇhi. Makkaṭo “karissati nu kho paribhogaṃ, na karissatī” ti olokento gahetvā nisinnaṃ disvā “kinnu kho” ti cintetvā daṇḍakoṭiyaṃ gahetvā parivattetvā olokento aṇḍakāni disvā tāni saṇikaṃ apanetvā adāsi. Satthā paribhogam akāsi. So tuṭṭhamānaso taṃ taṃ sākhaṃ gahetvā naccanto aṭṭhāsi. Tassa gahitasākhā pi akkantasākhā pi bhijji. So ekasmiṃ khāṇumatthake patitvā nibbiddhagatto satthari pasannena cittena kālaṅkatvā tāvatiṃsabhavane nibbatti. (Rasv)

\section*{Further Reading 2}

Atīte eko vejjo gāmanigamesu caritvā vejjakammaṃ karonto ekaṃ cakkhudubbalaṃ itthiṃ disvā pucchi “Kiṃ te aphāsukan” ti? “Akkhīhi na passāmī” ti. “Bhesajjaṃ te karomī” ti? “Karohi, sāmī” ti. “Kiṃ me dassasī” ti? “Sace me akkhīni pākatikāni kātuṃ sakkhissasi, ahaṃ te puttadhītāhi saddhiṃ dāsī bhavissāmī” ti. So bhesajjaṃ saṃvidahi. Ekabhesajjene va akkhīni pākatikāni ahesuṃ. Sā cintesi “ahaṃ etassa puttadhītāhi saddhiṃ dāsī bhavissāmī” ti paṭijāniṃ, “vañcessāmi nan” ti. Sā vejjena “kīdisaṃ, bhadde?” ti puṭṭhā “pubbe me akkhīni thokaṃ rujiṃsu, idāni atirekataraṃ rujantī” ti āha. (Rasv)

\section*{Further Reading 3}

Atīte kir’eko vejjo vejjakammatthāya gāmaṃ vicaritvā kiñci kammaṃ alabhitvā chātajjhatto nikkhamitvā gāmadvāre sambahule kumārake kīḷante disvā “ime sappena ḍasāpetvā tikicchitvā āhāraṃ labhissāmī” ti ekasmiṃ rukkhabile sīsaṃ niharitvā nipannaṃ sappaṃ dassetvā, “ambho, kumārakā, eso sāḷikapotako, gaṇhatha nan” ti āha. Ath’eko kumārako sappaṃ gīvāyaṃ daḷhaṃ gahetvā nīharitvā tassa sappabhāvaṃ ñatvā viravanto avidūre ṭhitassa vejjassa matthake khipi. Sappo vejjassa khandhaṭṭhikaṃ parikkhipitvā daḷhaṃ ḍasitvā tatth’eva jīvitakkhayaṃ pāpesi. (DhpA Kokasunakhaluddakavatthu)

\section*{Further Reading 4}

Atīte Bārāṇasiyaṃ Brahmadatte rajjaṃ kārente bodhisatto Bārāṇasiyaṃ vāṇijakule nibbatti. Nāmaggahaṇadivase c’assa “Paṇḍito” ti nāmaṃ akaṃsu. So vayappatto aññena vāṇijena saddhiṃ ekato hutvā vaṇijjaṃ karoti, tassa “atipaṇḍito” ti nāmaṃ ahosi. Te Bārāṇasito pañcahi sakaṭasatehi bhaṇḍaṃ ādāya janapadaṃ gantvā vaṇijjaṃ katvā laddhalābhā puna Bārāṇasiṃ āgamiṃsu. Atha tesaṃ bhaṇḍabhājanakāle Atipaṇḍito āha “Mayā dve koṭṭhāsā laddhabbā” ti. “Kiṃ kāraṇā” ti? “Tvaṃ Paṇḍito, ahaṃ Atipaṇḍito. Paṇḍito ekaṃ laddhuṃ arahati, atipaṇḍito dve” ti. “Nanu amhākaṃ dvinnaṃ bhaṇḍamūlakam pi goṇādayo pi samasamā yeva, kasmā tvaṃ dve koṭṭhāse laddhuṃ arahasī” ti? “Atipaṇḍitabhāvenā” ti. Evaṃ te kathaṃ vaḍḍhetvā kalahaṃ akaṃsu. Tato atipaṇḍito “atth’eko upāyo” ti cintetvā attano pitaraṃ ekasmiṃ susirarukkhe pavesetvā “tvaṃ amhesu āgatesu ‘atipaṇḍito dve koṭṭhāse laddhuṃ arahatī’ ti vadeyyāsī” ti vatvā bodhisattaṃ upasaṅkamitvā “samma, mayhaṃ dvinnaṃ koṭṭhāsānaṃ yuttabhāvaṃ vā ayuttabhāvaṃ vā esā rukkhadevatā jānāti, ehi, taṃ pucchissāmā” ti taṃ tattha netvā “ayye rukkhadevate, amhākaṃ aṭṭaṃ pacchindā” ti āha. Ath’assa pitā saraṃ parivattetvā “tena hi kathethā” ti āha. “Ayye, ayaṃ Paṇḍito, ahaṃ Atipaṇḍito. Amhehi ekato vohāro kato, tattha kena kiṃ laddhabban” ti. “Paṇḍitena eko koṭṭhāso, Atipaṇḍitena dve laddhabbā” ti. Bodhisatto evaṃ vinicchitaṃ aṭṭaṃ sutvā “idāni devatābhāvaṃ vā adevatābhāvaṃ vā jānissāmī” ti palālaṃ āharitvā susiraṃ pūretvā aggiṃ adāsi, atipaṇḍitassa pitā jālāya phuṭṭhakāle aḍḍhajjhāmena sarīrena upari āruyha sākhaṃ gahetvā olambanto bhūmiyaṃ patitvā imaṃ gāthaṃ āha “Sādhu kho Paṇḍito nāma, natveva atipaṇḍito” ti. (JaA 1.1.98 Kūṭavāṇijajātakavaṇṇanā)