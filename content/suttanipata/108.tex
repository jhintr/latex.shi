\section{慈经}

\begin{center}Metta Sutta\end{center}\vspace{1em}

\begin{enumerate}\item 缘起为何?据说,在雪山腹地,一群为天人滋扰的比丘去到舍卫国世尊的跟前。世尊为了护卫及业处,向他们说了此经。这只是略说。
\item 而其详说为:一时,世尊在将要开始雨安居时住舍卫国。尔时,众多来自各处异域的比丘去到世尊处,想要在世尊跟前获取业处后,去往各处入雨安居。于此,世尊以此方法讲述八万四千种随适于性行的业处:为贪行者以有识、无识讲述十一种不净业处,为嗔行者讲述四种慈等业处,为痴行者讲述死念业处等,为寻行者讲述入出息念、地遍等,为信行者讲述佛随念业处等,为觉行者讲述四界差别等。
\item 于是,五百比丘在世尊跟前取了业处后,即寻求适宜的坐卧处与行处村落,渐次而行,便在边地看见一座与雪山连属的山,岩石如同蓝水晶,缀以有着清凉密荫的青林,地面铺以细沙,如裹以银箔的珠网一般,围绕以洁净悦意而清凉的湖泊。
\item 于是,这些比丘在那里住了一夜,在拂晓时分洗漱后,便在距其不远的某村行乞。村内人群聚集,可有千户,且这里的人们有信、净喜,他们因在边地而难得见到出家人,一见比丘,即生起喜悦,令比丘们饮食,请求道「尊者!请在此处住三月」,教人修建了五百个精勤寮房后,在那里准备了床、椅、饮用水与洗用水瓶等一切器具。
\item 比丘们在第二天到别的村内行乞。那里的人们也如是护持,请求雨安居。比丘们以「不存在障难」而同意,进入密林后,整日整夜地发起精进,在敲过时分的揵槌后,总是如理作意而住,去到树下安坐。
\item 树神们的光彩为具戒比丘们的光彩所压制,从各自的宫中下来,带着孩子们四处游荡。就好比当村民家中的场所被来至村中的王或王大臣所占据时,家中之人便弃家而去,住于别处,从远处观望「他们何时能离开」,如是,天人丢开了各自的宫殿,四处游荡,从远处观望「大德们何时能离开」。随后,他们如是筹划:「初入安居的比丘们必然将居住三月,而我们将从此长久地带着孩子躲避,不得安住,噫!让我们向比丘们示现恐怖的所缘!」
\item 夜里,在行沙门法之时,他们化作恐怖的夜叉形象,站在各个比丘的面前,并发出可怕的声响。比丘们看到这些形象、听到这声响,便心惊胆战,且脸色难看,泛起蜡黄。因此,他们不能一境其心。此等不能一境之心由一再怖畏而悚惧,念便忘失。随后,他们向那些失念者施以恶臭的所缘。他们的脑髓好像被这恶臭挤榨,生起剧烈的头痛,且他们未对彼此告知此事。
\item 于是,一天,在给侍僧伽长老时,当全体聚集后,僧伽长老便问:「朋友!你们刚进入这密林的几天极其遍净,肤色洁白,诸根明净,但如今憔悴不堪,脸色难看,泛起蜡黄,你们于此有什么不适宜吗?」随后,一比丘说:「尊者!我在夜里看到、听到这般这般可怕的所缘,且闻到这般的气味,因此我心不得等持。」以此方式,大家都告知了此事。僧伽长老说:「朋友!世尊施设了两个雨安居的开端,且我们的坐卧处不适宜,朋友!我们去到世尊跟前,到了之后,问取其它适宜的坐卧处。」「善哉!尊者!」诸比丘答复长老后,全都收拾了坐卧处,持了衣钵,由不染著于俗家,便未通知任何人,即往舍卫国进发。渐次到了舍卫国后,去到世尊跟前。
\item 世尊见到这些比丘后,说:「诸比丘!我施设了学处『安居期间不应游行』,你们为何游行?」他们便向世尊告知了一切。世尊经转向,在整个阎浮提中,连四足椅子之量的适宜彼等的坐卧处都未得见。于是,便对彼比丘众说:「诸比丘!没有其它适宜你们的坐卧处,你们唯住在那里方能证得漏尽。去吧!诸比丘!就依彼坐卧处而住!若欲不畏惧天人,则应受持此护卫,因为这将是你们的护卫及业处。」便说了此经。
\item 而其他人则说,在说了「去吧!诸比丘!就依彼坐卧处而住」后,世尊说:「而且,林居者应知护卫。此即是,早晚作两遍慈、两遍护卫、两遍不净、两遍死念,以及转向于八大悚惧事。八大悚惧事,即生、老、病、死及四恶趣之苦。或者,生老病死为四,恶趣之苦为第五,过去源于流转之苦,未来源于流转之苦,现在源于觅食之苦。」如是,世尊在宣说了护卫后,为了慈、护卫及作为观的基础的禅那,向这些比丘说了此经。\end{enumerate}

\subsection\*{\textbf{143}}

\textbf{在证得寂静境地后,善巧于义利者应当\\}
\textbf{堪能、正直、极正直,且应易语、柔和、不傲慢,}

Karaṇīyam atthakusalena, yanta santaṃ padaṃ abhisamecca;\\
sakko ujū ca suhujū ca, sūvaco c’assa mudu anatimānī. %\hfill\textcolor{gray}{\footnotesize 1}

\begin{enumerate}\item 这里,先是此初颂的释词。\textbf{应当},即应作、当作之义。\textbf{义利},即行道,或者举凡任何自身的利益,这一切由应予从事而被称为义利,应予从事,即应予承担。\textbf{善巧于义利者},即是说熟练于义利者。\textbf{Yaṃ} 为关系代词之体格,\textbf{taṃ} 为指示代词之业格,或者 \textbf{yaṃ taṃ} 两者都是体格\footnote{原文中的 \textbf{yanta} 或 yan taṃ 都是代词,通常可译作「此」,考虑到义注给出了两种解释,这里的译文未予译出,详见义注的下文及后两条注。}。\textbf{寂静境地}是业格。这里,由相为寂静,由可证得为境地,即涅槃的同义语。\textbf{证得},即完全掌握。以能够为\textbf{堪能},即是说有力、胜任。\textbf{正直},即与诚实相应。以非常正直为\textbf{极正直}。以对其易于言谈为\textbf{易语}。\textbf{柔和},即与温柔相应。
\item 而其释义为:此中,首先有应当、有不应当。这里,略说三学为\textbf{应当},而如戒的欠缺、见的欠缺、正行的欠缺、活命的欠缺等为\textbf{不应当}。同样,有善巧于义利者、有不善巧于义利者。
\item 这里,若于此教法内出家已,未正当地律己而破损了戒,依二十一种邪求而营生。此即是:以竹布施、以叶布施、以花布施、以果布施、以齿木布施、以洗脸水布施、以沐浴布施、以肥皂粉布施、以粘土布施、谄谀、豆汤语、抚爱、走使传讯、从医、遣使、送信、交换食物、风水、占星、相术等,以及行于六种非行处,即娼妓处,寡妇、老处女、阉人、比丘尼、酒肆处等。且与国王、王大臣、外道、外道弟子以不适当的俗事交际而住。或者,凡是对比丘……优婆夷无信、不净喜、不供水、骂詈指责、不欲其义利、不欲其利益安乐及离轭安稳的家庭,他与这样的家庭亲近、结交、承事。此即\textbf{不善巧于义利者}。
\item 若于此教法内出家已,正当地律己,舍弃了邪求,欲住立于四遍净戒而圆满了以信为首的别解脱律仪、以念为首的根律仪、以精进为首的活命遍净、以慧为首的资具受用,此即\textbf{善巧于义利者}。
\item 或者,若以净化七种违犯来净化别解脱律仪,以于撞击六门的所缘不生起贪等来净化根律仪,以避免邪求及以为智者所赞叹、佛及佛弟子所称赞的方式受用资具来净化活命遍净,以所说的省察来净化资具受用,且当转换四威仪时,以省察其目的来净化正知,此亦为善巧于义利者。
\item 或者,好比靠盐水洁净染污的衣服,靠灰烬洁净镜子,靠坩埚洁净金子,如是,若了知到「靠智净化戒」,则以智水清洗,令戒清洁。且好比松鸦守护蛋,牦牛守护尾,独子的妇人守护喜爱的独子,独眼的人守护独眼,如是极不放逸者守护自身的戒蕴,早晚省察也不见微细的罪过,此亦为善巧于义利者。
\item 或者,若住立于起无后悔作用的戒已,策励于镇伏烦恼的行道,策励此已,作遍的预作,作遍的预作已,生起等至,此亦为善巧于义利者。
\item 或者,若从等持出起,把握诸行已,得证阿罗汉,此为善巧于义利者之最上。
\item 这里,凡是从以住立于起无后悔作用的戒,或从以策励于镇伏烦恼的行道、以道果所说明的善巧于义利者,他们是此义中「善巧于义利者」的意思,而这些比丘即是此类。因此,世尊就这些比丘,以基于一人的开示而说「善巧于义利者应当」。
\item 随后,就他们生起的疑惑「应当什么」而说「\textbf{在证得寂静境地后}」。此中的意趣为:以通达而证得此 \textit{taṃ} 由佛及随佛所说明的寂静之涅槃境地后,欲住者应当如此 \textit{yaṃ}。且此中,此句开头所说的「此 \textit{yaṃ}」由统摄而与「应当」一致,而「在证得此 \textit{taṃ} 寂静境地后」,因其文义不完整,所以当知即是所说的「欲住者」。\footnote{本段展开说明「yaṃ 为关系代词之体格,taṃ 为指示代词之业格」:yaṃ 指「应当 \textit{karaṇīyam}」,taṃ 指「寂静境地 \textit{santaṃ padaṃ}」,则前两句可译作「在证得此寂静境地后,善巧于义利者应当如此」,且认为「应当」的施动者已证得涅槃。}
\item 或者,「在证得寂静境地后」即以传闻等的世间慧了知涅槃境地为寂静后,由统摄与「欲得证此者应当如此 \textit{yaṃ taṃ}」一致,如是当知此中的意趣即「善巧于义利者应当如此」。
\item 或者,当说「善巧于义利者应当」时,对思索「什么」者而说「在证得寂静境地后」。如是当知其意趣为:以世间慧证得寂静为境地后,应当者即是此,即是说「应作者即是应当、唯此当作」。
\item 那么「此」是什么?除了证得此的方法,还能是什么?而这自然是开头所说的以当作之义所显明的三学。正因如此,我们在其释义中说「有应当、有不应当,这里,略说三学为应当」。\footnote{以上三段展开说明「yaṃ taṃ 两者都是体格」:即两者都指「应当 \textit{karaṇīyam}」,则前两句可译作「在证得寂静为境地后,善巧于义利者应当如此」,且认为「应当」的施动者只是欲证涅槃者,而颂中的「证得」则作「了知」解。}
\item 但由开示太过简略,这些比丘中有些能了别,有些则不能。随后,为了让那些不能了别者能够了别,而详述林居的比丘尤当作者,先说了「\textbf{堪能、正直、极正直,且应易语、柔和、不傲慢}」这半颂。
\item 这说的是什么?欲住者在证得寂静境地后,或以世间慧证得此后,为得证此而行道的林居比丘,以具足第二、第四精勤支\footnote{精勤支:有多种说法,这里应指\textbf{增支部}第 5:53 经中所说的五精勤支,略说即信、少病、不狡诈及不欺瞒、精进、具慧等。}而不顾身命,他将堪能为通达谛而行道,同样,于遍的预作、受持义务等,及于修补自己的衣钵等,于任何对同梵行者的种种应作及于其它类似者,他将堪能、熟练、不怠惰、有力。
\item 当堪能,且以具足第三精勤支,则能正直。当正直,且不以一次正直而满足,以尽寿命、一再地不松弛,则能非常正直。或者,以不狡诈为正直,以不欺瞒为极正直,以舍弃身语的邪曲为正直,以舍弃意的邪曲为极正直,以不展示不实的功德为正直,以不承受由不实之功德所得的利养为极正直。如是,以省虑所缘及相,以初二学及第三学,以加行与意乐的清净,则能正直且极正直。
\item 不仅是正直且极正直,而且还\textbf{应易语}。因为若人被告知「此不应做」,却说「你何所见,你何所闻,对我说话的人是谁,是和尚、阿阇黎、相知还是相交」,或者以沉默恼害,或者领受却不照做,他便距证得殊胜尚远。然而,若被教诫时,说「善哉!尊者!所说甚善,自己的罪过总难得见,若再见我如此,请出于怜悯告知!我从您跟前受教已很久了」,并如所教授而行道,他便距证得殊胜不远。所以,如是领受他人的话语而行者,便为易语。
\item 且如同易语,如是应\textbf{柔和}。柔和,即当被在家人怂恿于走使送信等时,于此不应柔和而应坚决,于义务、行道及整个梵行中则应柔和,如善锻的黄金一般,处处泰然于劝教。或者,柔和即不皱眉、语言明了、易于交谈、如好的渡口般习于承迎、易于融入。
\item 不仅是柔和,而且还应\textbf{不傲慢},不应以出身、族姓等傲慢之事蔑视他人,而应如舍利弗长老一般,以旃陀罗男孩似的心而住\footnote{舍利弗长老以旃陀罗男孩似的心而住,见\textbf{增支部}第 9:11 经。}。\end{enumerate}

\subsection\*{\textbf{144}}

\textbf{知足,易养,少事务,生活简朴,\\}
\textbf{诸根寂静,贤明,不鲁莽,不贪求于俗家。}

Santussako ca subharo ca, appakicco ca sallahukavutti;\\
santindriyo ca nipako ca, appagabbho kulesv ananugiddho. %\hfill\textcolor{gray}{\footnotesize 2}

\begin{enumerate}\item 如是,世尊对已证得寂静境地的欲住者,或对为得证此的行道者,尤其是林居比丘,说了一些应作后,欲更多说,而说了第二颂。
\item 这里,以「知足、知恩」中所说的十二种满足\footnote{「知足、知恩」中所说的十二种满足,见\textbf{吉祥经}第 268 颂注。}而知足为\textbf{知足}。或者以满足于自身、满足于现有、满足于平等为知足。这里,\textbf{自身}即如在具足戒坛上所指出的「依于团饭之食」及由自己领受的各种四资具。在受纳及受用时,于其善妙或不善妙、恭敬或不恭敬地被给予,不显现差别而存活,被称为「满足于自身」。\textbf{现有}即自己现在的所得,唯满足于此现有而不希求更多,舍弃过度的希望,被称为「满足于现有」。\textbf{平等}即舍弃对诸可意、不可意的随贪与嗔恚,于一切所缘以此平等而满足,被称为「满足于平等」。
\item 易被养活为\textbf{易养},即是说易于养育。因为若比丘在钵里装满了稻、肉、饭等,仍于所施面露难色、心有不满,或者当着他们的面不喜欢这食物「你们布施了什么」,给了沙弥与在家人等,他为难养。人们见到他后,便远远地避开「不堪养育难养的比丘」。但若无论得了精粗多少都心满意足、脸色明净地存活,他为易养。人们见到他后,都极其信赖「我们的大德易养,甚至满足于少许,我们要养育他」,给予认可而养育。如此即是此处「易养」之意。
\item \textbf{少事务},即不因喜欢工作、喜欢闲谈、喜欢聚会等而忙碌于种种事务。或者,即是说在整个寺庙中,没有新建、僧产及沙弥与园工的管理等事务,在做好自己修剪发甲、缝补衣钵等后,以沙门法之事务为首要。
\item \textbf{生活简朴},好比有些比丘物品饶多,在去往某方时,让大众以头戴腰负等抬走众多衣钵、铺盖、油糖等而去,若不如是而资具寡少,则在去往某方时,仅携带衣钵等八沙门资具而去,如鸟一般,只身前往。如此即是此处「生活简朴」之意。
\item \textbf{诸根寂静},即是说于可意的所缘等,不因贪等而掉举诸根。
\item \textbf{贤明},即有智、聪明、具慧,具足守护戒之慧、筹划衣等之慧、知晓住处等七种适宜\footnote{住处等七种适宜,见\textbf{清净道论}·说地遍品第 35 段及以下。}之慧的意思。
\item \textbf{不鲁莽},即无有八处身鲁莽、四处语鲁莽以及多处意鲁莽之义。\textbf{八处身鲁莽},即在僧团、众人、个人、食堂、桑拿房、洗浴场、行乞路、进入室内中,身体的不当行为。比如,于此,有人在僧团中或抱膝而坐,或翘腿等等。在众人中也如此,众人,即四众集会。在年长的个人处也如此。而在食堂中,不给年长者让座,阻止新来者入座。在桑拿房处也如此,且于此,未经年长者许可,即行烧火等。而在洗浴场,置「不以长幼为则,而按先来后到洗浴」的规定于不顾,晚来便入水,且逼恼年长者及新来者。在行乞路上,为了最上之坐、最上之水、最上之食,胳膊挤着胳膊,行于年长者之前。进入室内时,较年长者抢先进入,与幼者身体嬉戏,如是等等。
\item \textbf{四处语鲁莽},即在僧团、众人、个人、室内中发出不当的语言。比如,于此,有人在僧团中,未经许可即说法。在先前所说的众人及年长的个人处也如此,此处,当被人们问了问题,未问年长者即予解答。而在室内,如说「某人在吗?什么粥或硬食、软食?你会给我什么?今天嚼什么、吃什么、喝什么」等等。
\item \textbf{多处意鲁莽},即于彼彼处虽未犯身语的过行,但以意起欲寻等种种品类的不当之寻。
\item \textbf{不贪求于俗家},即是说在所到的俗家中,不因渴爱资具或以与俗家不当的交际而贪求,不与同忧,不与同喜,不于乐处而乐,不于苦处而苦,或不置自身于发生的事务。
\item 且此颂中,于「应易语」中所说的「应」字,当与所有词相连,如「应知足、应易养」等。\end{enumerate}

\subsection\*{\textbf{145}}

\textbf{且不应做其他智者会谴责的任何小事!\\}
\textbf{愿他们快乐、安稳!愿一切有情幸福!}

Na ca khuddam ācare kiñci, yena viññū pare upavadeyyuṃ;\\
sukhino va khemino hontu, sabbasattā bhavantu sukhitattā. %\hfill\textcolor{gray}{\footnotesize 3}

\begin{enumerate}\item 如是,世尊对已证得寂静境地的欲住者,或对为得证此的欲行道者,尤其是林居比丘,宣说了更多的应作后,现在,还欲宣说不应作而说了这半颂。
\item 其义为:如是作此应作者,\textbf{且不应做}凡是被称为低劣的身语意恶行的\textbf{小事},即是说不仅不做粗重者,而是不应做\textbf{任何},连少许、微量者也不应做。随后,向他显示做此的现世过患:\textbf{其他智者会谴责}。且此中,因为其他的非智者并非标准,他们或行无过,或行有过,或有小过,或有大过,而唯有智者才是标准,他们经省思、彻查之后,不赞叹不值得赞叹者,而赞叹值得赞叹者,所以说「其他智者」。
\item 如是,世尊以此二颂半对已证得寂静境地的欲住者,或对为得证此的欲行道者,尤其是林居比丘,以及对以林居者为首的所有已获取业处的欲住者,在说了应作、不应作等类的业处的近行后,现在,为防御天人的怖畏,对这些比丘以「愿他们快乐、安稳」等方法,为了护卫及作为观的基础禅那的业处,开始说慈论。
\item 这里,\textbf{快乐},即具有乐。\textbf{安稳},即具有安稳,即是说无有怖畏、免于祸害。\textbf{一切},即无余。\textbf{有情},即生类。\textbf{幸福},即心之乐。且此中,当知或以身之乐为快乐,以意为幸福,而以离于两者所有的怖畏及祸害为安稳。
\item 但为什么这么说?为了显示慈修习的方式。因为慈当如是修习:「愿一切有情快乐」,或「愿他们安稳」,或「愿他们幸福」。\end{enumerate}

\subsection\*{\textbf{146}}

\textbf{举凡是呼吸的生命,弱者或强者,皆无遗漏,\\}
\textbf{长者或大者、中者、短者、细者、粗者,}

Ye keci pāṇabhūt’atthi, tasā vā thāvarā v’anavasesā;\\
dīghā vā ye va mahantā, majjhimā rassakā aṇukathūlā. %\hfill\textcolor{gray}{\footnotesize 4}

\begin{enumerate}\item 如是,在略示了从近行到安止之顶的慈修习后,现在为详示此而说了这二颂。或者,因为心惯习于多样的所缘,不能一开始就安立于同一,但在对所缘进行分解后,则可次第安立,所以,为其安立而逐步分解所缘为弱强等两分及三分,说了这二颂。或者,因为若所缘对其明了,则其心于此易住,所以,欲令这些比丘的心安立于对其明了的所缘,而说了这显明弱强等两分及三分之分解所缘的二颂。
\item 此中,以弱强两分、可见不可见两分、远近两分、已出生将出生两分为四种两分,且为了以「长」等六字形成三分,「中」字现于三处,「细」字现于二处,显明长短中三分、大细中三分、粗细中三分等三种三分。
\item 这里,\textbf{举凡},即无余之语。有呼吸的生命为\textbf{呼吸的生命}。或者,以「呼吸」为呼吸者,以此摄依赖于入息出息的五蕴有之有情,以「存在」为生命,以此摄一蕴有、四蕴有之有情。\textbf{是},即存在、现有。
\item 如是,以「举凡是呼吸的生命」之语统一地显示了应以两分、三分归纳的一切有情之后,现在,以「弱者或强者,皆无遗漏」之两分归纳彼等一切而显示之。
\item 这里,以害怕为\textbf{弱者},即有所渴爱、有所怖畏的同义语。以站立为\textbf{强者},即舍弃了渴爱、怖畏之阿罗汉的同义语。\textbf{皆无遗漏},即是说全体。且在下颂最后所说的当与所有两分、三分相连结:「举凡是呼吸的生命,弱者或强者,皆无遗漏,愿这一切有情幸福」,如是乃至「已出生者、将出生者,也愿这一切有情幸福」。
\item 现在,在显明长短中等三种三分之中,「长者」等六字中的\textbf{长者},即龙、鱼、蜥蜴等的体长者。因为大海之中,龙的身体有数百寻长,而鱼、蜥蜴等的身体也有数由旬长。\textbf{大者},即体大者,如水中的鱼龟等、陆上的象等、非人中的魔鬼等,如说「罗睺为体量之最上」,因为其身体高四千八百由旬,臂长一千二百由旬,眉间五十由旬,指间亦然,手掌二百由旬。\textbf{中者},即马、牛、水牛、猪等的身体。\textbf{短者},即生于彼等中的侏儒,除大者、中者以外的体量劣小的有情。\textbf{细者},即非肉眼的行处,而为天眼的境界,如水等中出生的体量细微的有情,或虱子等。且生于彼等中,除了大者中者及粗者中者以外的体量劣小的有情,当知也为细者。\textbf{粗者},即体量圆胖的有情,如鱼、龟、牡蛎、贝类等。\end{enumerate}

\subsection\*{\textbf{147}}

\textbf{可见者或不可见者,住于远方或非远方,\\}
\textbf{已出生者、将出生者,愿一切有情幸福!}

Diṭṭhā vā ye va adiṭṭhā, ye va dūre vasanti avidūre;\\
bhūtā va sambhavesī va, sabbasattā bhavantu sukhitattā. %\hfill\textcolor{gray}{\footnotesize 5}

\begin{enumerate}\item 如是在以三种三分无余地显示了有情后,现在,再以「可见者或不可见者」等三种两分归纳而显示之。
\item 这里,\textbf{可见者},即以来至自己眼睛视域的先前所见。\textbf{不可见者},即存在于其它的海、其它的山、其它的轮围之中者。\textbf{住于远方或非远方},以此两分显示相对于自己的身体住于远方及非远方的有情,彼等当以相对而知。因为住于自身的有情为非远方,于身外者为远方。同样,住于邻近之内为非远方,住于邻近之外为远方,住于自己的寺庙、村庄、国土、洲岛、轮围者为非远方,住于其它轮围者则被称为住于远方。
\item \textbf{已出生者},即已生者、已转生者。凡只是已出生但将不再存在者归于此中,此即彼等漏尽者的同义语。以寻求生成为\textbf{将出生者},此即由未舍弃有结而仍寻求未来之生成的众有学、凡夫的同义语。或者,在四种出生中,卵生及胎生的有情,只要蛋壳及胎膜未破,便为将出生者,蛋壳及胎膜破后出于外者,为已出生者。湿生及化生者,在第一心刹那为将出生者,从第二心刹那起为已出生者。或者,以某种威仪出生者,只要未采取其它威仪,便为将出生者,此后则为已出生者。\end{enumerate}

\subsection\*{\textbf{148}}

\textbf{愿无人欺骗他人!愿不在任何场合轻贱任何人!\\}
\textbf{愿不以忿怒、嗔恚想而希望彼此受苦!}

Na paro paraṃ nikubbetha, nātimaññetha katthaci na kañci;\\
byārosanā paṭighasañña, nāññamaññassa dukkham iccheyya. %\hfill\textcolor{gray}{\footnotesize 6}

\begin{enumerate}\item 如是,世尊以「愿他们快乐」等二颂半,从种种品类,以希求趋向利益与乐的方式,为这些比丘显示了对有情的慈修习,现在,又为以希求离于不利与苦的方式显示此,说了此颂。此(paraṃ ni°)系古本,而今本作 paraṃ hi 者不妥。
\item 这里,\textbf{任何场合},即任何空间、村镇、国土或亲属中、团体中等等。\textbf{任何人},即任何刹帝利、婆罗门、在家人、出家人或幸者、不幸者等等。\textbf{忿怒、嗔恚想},即以身语扰乱之忿怒,及以意扰乱之嗔恚想。
\item 这说的是什么?不应仅以「愿他们快乐、安稳」等的作意培育慈,还应如是作意而培育:「哎!任何人不应以诳骗等的欺诈欺骗任何人,且不应以出身等九慢事在任何场所轻贱任何人,且不应以忿怒或嗔恚想希望彼此受苦!」\end{enumerate}

\subsection\*{\textbf{149}}

\textbf{好比母亲对自己的孩子,会以生命保护独子,\\}
\textbf{如是,也应对一切生命培育无量的心意!}

Mātā yathā niyaṃ puttam, āyusā ekaputtam anurakkhe;\\
evam pi sabbabhūtesu, mānasaṃ bhāvaye aparimāṇaṃ. %\hfill\textcolor{gray}{\footnotesize 7}

\begin{enumerate}\item 如是,以希求离于不利与苦的方式从意义上显示了慈修习后,现在,为举例说明而说了此颂。
\item 其义为:\textbf{好比母亲对自己的孩子},对自身所生、所养的孩子,且\textbf{会以生命保护}此\textbf{独子},为防除其趋向于苦,甚至会舍弃自己的生命去保护他,\textbf{如是,也应对一切生命培育}、再再生起、增长此慈\textbf{意},且应以无量的有情为所缘的方式,或对一处的有情聚以无余遍满的方式来培育\textbf{无量}的心意。\end{enumerate}

\subsection\*{\textbf{150}}

\textbf{且在所有世间,应培育无量的慈意!\\}
\textbf{对上方、下方及四旁,无障碍、无怨恨、无敌对。}

Mettañ ca sabbalokasmi, mānasaṃ bhāvaye aparimāṇaṃ;\\
uddhaṃ adho ca tiriyañ ca, asambādhaṃ averam asapattaṃ. %\hfill\textcolor{gray}{\footnotesize 8}

\begin{enumerate}\item 如是,以一切行相显示慈修习后,现在,唯为显示其增长,说了此颂。
\item 这里,以「被润泽及庇护」为朋友,即为期望利益所爱及守护不利的到来之义,朋友之状态为\textbf{慈}。\textbf{所有}即无余,\textbf{世间}即有情世间。以存在于意中为\textbf{意},因其与心相应,故如是说。\textbf{培育},即增长。\textbf{无量},以无量有情为所缘,故如是说。\textbf{上方},即在上,以此摄无色有。\textbf{下方},即在下,以此摄欲有。\textbf{四旁},即中间,以此摄色有。
\item \textbf{无障碍},即是说无有障碍、打破边界。怨敌被称为边界,即对其亦能转起之义。\textbf{无怨恨},即是说连片刻怨恨之思的表露也无有之义。\textbf{无敌对},因为慈住之人为人喜爱,为非人喜爱,无人是其怨敌,因此,其心意由离于怨敌,故被称为无敌对,因为怨敌与敌对为近义词。这是逐字的释义。
\item 而此中其旨趣的释义为:即此所说的「如是,也应对一切生命培育无量的心意」,还应在所有世间培育、增长此无量的慈意,以至增长广大。如何?\textbf{对上方、下方及四旁},上至有顶,下至无间,四旁至其余方向,或上至无色,下至欲界,四旁遍满其余的色界。且当如是培育此时,如无障碍、怨恨、敌对而行,以至无障碍、无怨恨、无敌对。或者,凡是已达修习之成就者,以于一切处获得许可为\textbf{无障碍},以调伏自己对他人的嫌恨为\textbf{无怨恨},以调伏他人对自己的嫌恨为\textbf{无敌对},此无障碍、无怨恨、无敌对的无量慈意,对上方、下方及四旁等三部分的一切世间,应培养、增长之。\end{enumerate}

\subsection\*{\textbf{151}}

\textbf{站着、走着、坐着或躺着,只要离于睡眠,\\}
\textbf{就应决意此念,他们说这就是此世的梵住。}

Tiṭṭhaṃ caraṃ nisinno va, sayāno yāvatāssa vitamiddho;\\
etaṃ satiṃ adhiṭṭheyya, brahmam etaṃ vihāram idha māhu. %\hfill\textcolor{gray}{\footnotesize 9}

\begin{enumerate}\item 如是,已显示了慈修习的增长,现在,为向从事而住者显示此修习不限于威仪,说了此颂。其义为:当如是修习此慈意时,他不必限于如「结跏趺坐,端身正愿」般的威仪,可随所好地选择去除妨碍的某种威仪,\textbf{站着、走着、坐着或躺着,只要离于睡眠,就应决意此}慈禅之\textbf{念}。
\item 或者,在如是显示了慈修习的增长后,现在为显示自在的状态而说此颂。因为得自在者,无论站着、走着、坐着或躺着,只要以威仪,便可成为欲决意此慈禅之念者。或者,对他而言,「站着或坐着」的站立等不构成为障碍,且他欲决意此慈禅之念,只要离于睡眠,便可决意,他于此无有迟钝,因此说「站着、走着、坐着或躺着,只要离于睡眠,就能决意\footnote{决意的祈愿语气 \textit{adhiṭṭheyya} 可译为「应决意」或「能决意」,这里在义注解释作「为显示自在」时译作「能」。}此念」。
\item 其旨趣为:对于「且在所有世间,应培育慈意」中所说者,应如是培育之:只要以站立等中的威仪,或不顾及\footnote{Anādiyitvā 在这里译作「不顾及」,见菩提比丘注 713 及 DOP 的解释。}站立等,只要欲决意此慈禅之念,就能离于睡眠,决意此念。
\item 如是,为显示慈修习的自在,在以「就能决意此念」敦促此慈住后,现在为赞叹此住而说「他们说这就是此世的梵住」。
\item 其义为:这从「愿他们安稳、快乐」开始,直至「就应决意此念」所说明的慈住,\textbf{他们说这}在天、梵、圣、威仪等四种住中,由无过失及由为自己与他人带来义利之故,\textbf{就是此世}圣者法律中\textbf{的梵住}、最胜之住。因此,无论站着、走着、坐着或躺着,只要离于睡眠,就应常常、连续、不断地决意此念。\end{enumerate}

\subsection\*{\textbf{152}}

\textbf{且不再执取见,具戒,具足知见,\\}
\textbf{调伏了对爱欲的贪求,他就决不再入母胎。}

Diṭṭhiñ ca anupaggamma, sīlavā dassanena sampanno;\\
kāmesu vinaya gedhaṃ, na hi jātuggabbhaseyya punar etī ti. %\hfill\textcolor{gray}{\footnotesize 10}

\begin{enumerate}\item 如是,世尊对这些比丘从种种行相显示了慈修习,现在,因为慈以有情为所缘之故而近于我见,所以,以遮止见的执取为首,显示即以此慈禅为基础便可得至圣地,对这些比丘说了此颂。以此颂终结了开示。
\item 其义为:从以「他们说这就是此世的梵住」所说明的慈禅之住出起后,把握了此处的寻伺等法与其依处等相伴的色法,以如\begin{quoting}这是单纯的行的积聚,\\于此无有情可得。(相应部第 5:10 经)\end{quoting}所说的名色差别而\textbf{不再执取见},渐次以出世间戒而成\textbf{具戒},\textbf{具足}唯与出世间戒相应的被称为须陀洹道之正见的\textbf{知见}。随后,他仍未舍弃贪求事欲的烦恼欲,便以斯陀含道之减薄与阿那含道之无余舍断而\textbf{调伏了}、止息了\textbf{对爱欲的贪求},\textbf{他就决不再入母胎},绝对不再入母胎,唯转生于净居天中,即于彼处证得阿罗汉而般涅槃。
\item 如是,世尊终结了开示,便对这些比丘说:「去吧!诸比丘!就住于那密林中!且在每月八次闻法等场合中,在敲过揵槌后,应称扬、宣法、讨论、随喜此经,于此业处则应修习、培育、多作,彼等非人将不再对你们显示可怕的所缘,而将欲求义利、欲求利益。」他们答复世尊「善哉」,即从坐起,礼敬了世尊,右绕后,去到那里,如是而行。而诸天人生起喜悦「大德们是欲求我们的义利、利益」,亲自打扫坐卧处,准备热水,涂抹背与足,安排守卫。那些比丘即如是修习慈已,以之为基础,开始作观,全都在这三月之内证得了最上的阿罗汉果,便于大自恣中举行了清净的自恣。\begin{quoting}如是,善巧于义利者,于如来、\\法自在天所说的应作的义利\\作已,体验到最胜的心之寂静,\\慧究竟者得证寂静的境地。\end{quoting}\begin{quoting}所以,于此不死、希有、圣者悦意的\\寂静境地,已证得且欲住的\\有智之士,于无垢的戒定慧\\等应作的义利,应当常常去做。\end{quoting}\end{enumerate}

\begin{center}\vspace{1em}慈经第八\\Mettasuttaṃ aṭṭhamaṃ.\end{center}

%\begin{flushright}甲辰正月晦日二稿\end{flushright}