\chapter{詩譜}

\section*{詩譜序}

詩之興也,諒不於上皇之世。大庭軒轅,逮於高辛,其時有亡載籍,亦蔑云焉。虞書曰「詩言志,歌永言,聲依永,律和聲」,然則詩之道放於此乎?有夏承之,篇章泯棄,靡有孑遺。邇及商王,不風不雅,何者?論功頌德,所以將順其美,刺過譏失,所以匡救其惡,各於其黨,則為法者彰顯,為戒者著明。周自后稷播種百穀,黎民阻饑,茲旹乃粒,自傳於此名也。陶唐之末,中葉公劉亦世脩其德,以明民共財。至於大王王季,克堪顧天,文武之德,光熙前緒,以集大命於厥身,遂為天下父母,使民有政有居,其時詩風有周南召南,雅有鹿鳴文王之屬,及成王周公致太平,制禮作樂而有頌聲興焉,盛之至也,本之由此風雅而來,故皆錄之,謂之詩之正經。

後王稍更凌遲,懿王始受譖,亨齊哀公,夷身失禮之後,邶不尊賢。自是而下,厲也幽也,政教尤衰,周室大壞,十月之交、民勞、板、蕩勃爾俱作,眾國紛然,刺怨相尋。五霸之末,上無天子,下無方伯,善者誰賞,惡者誰罰,紀綱絕矣。故孔子錄懿王夷王時詩,訖於陳靈公淫亂之事,謂之變風變雅,以為勤民恤功、昭事上帝則受頌聲,弘福如彼,若違而弗用則被劫殺,大禍如此,吉凶之所由,憂娛之萌漸,昭昭在斯,足作後王之鑒,於是止矣。

夷厲已上,年數不明,大史年表自共和始,歷宣幽平王而得春秋次第,以立斯譜。欲知源流清濁之所處,則循其上下而省之,欲知風化芳臭氣澤之所及,則傍行而觀之,此詩之大綱也,舉一綱而萬目張,解一卷而眾篇明,於力則鮮,於思則寡,其諸君子,亦有樂於是與?

\section*{周南召南譜}

周召者,禹貢雍州岐山之陽地名,今屬右扶風美陽縣,地形險阻而原田肥美。周之先公曰大王者避狄難,自豳始遷焉,而脩德建王業,商王帝乙之初,命其子王季為西伯,至紂,又命文王典治南國江漢汝旁之諸侯。於時三分天下有其二,以服事殷,故雍梁荊豫徐揚之人咸被其德而從之。文王受命,作邑於豐,乃分岐邦周召之地為周公旦、召公奭之采地,施先公之教於己所職之國。武王伐紂定天下,巡守述職,陳誦諸國之詩以觀民風俗,六州者得二公之德教尤純,故獨錄之,屬之大師,分而國之。其得聖人之化者,謂之周南,得賢人之化者,謂之召南,言二公之德教自岐而行於南國也。乃棄其餘,謂此為風之正經。

初,古公亶父「聿來胥宇,爰及姜女」,其後大任「思媚周姜」,「大姒嗣徽音」,歷世有賢妃之助,以致其治,文王「刑于寡妻,至于兄弟,以御于家邦」,是故二國之詩以后妃夫人之德為首,終以麟趾、騶虞,言后妃夫人有斯德,興助其君子,皆可以成功,至于獲嘉瑞。風之始,所以風化天下而正夫婦焉,故周公作樂,用之鄉人焉,用之邦國焉。或謂之房中之樂者,后妃夫人侍御於其君子,女史歌之,以節義序故耳。

射禮「天子以騶虞、諸侯以貍首、大夫以采蘋、士以采蘩為節」,今無貍首,周衰,諸侯並僭而去之,孔子錄詩不得也,為禮樂之記者從後存之,遂不得其次序。周公封魯,死諡曰文公,召公封燕,死諡曰康公,元子世之。其次子亦世守采地,在王官,春秋時周公召公是也。

問者曰,周南召南之詩為風之正經則然矣,自此之後,南國諸侯政之興衰,何以無變風?答曰,陳諸國之詩者,將以知其缺失,省方設教為黜陟,時徐及吳楚僭號稱王,不承天子之風,今棄其詩,夷狄之也,其餘江黃六蓼之屬,既驅陷於彼俗,又亦小國,猶邾滕紀莒之等夷,其詩蔑而不得列於此。

\section*{邶鄘衛譜}

邶鄘衛者,商紂畿內方千里之地。其封域在禹貢冀州大行之東,北踰衡漳,東及兗州桑土之野。周武王伐紂,以京師封紂子武庚,為殷後,庶殷頑民被紂化日久,未可以建諸侯,乃三分其地,置三監,使管叔、蔡叔、霍叔尹而教之,自紂城而北謂之邶,南謂之鄘,東謂之衛。

武王既喪,管叔及其群弟見周公將攝政,乃流言於國曰「公將不利於孺子」。周公避之,居東都二年,秋,大熟未穫,有雷電疾風之異,乃後成王悦而迎之,反而遂居攝。三監導武庚叛,成王既黜殷命,殺武庚,復伐三監,更於此三國建諸侯,以殷餘民封康叔於衛,使為之長。後世子孫稍并彼二國,混而名之。七世至頃侯,當周夷王時,衛國政衰,變風始作。故作者各有所傷,從其國本而異之,為邶鄘衛之詩焉。

\section*{王城譜}

王城者,周東都王城畿內方六百里之地。其封域在禹貢豫州太華外方之閒,北得河陽,漸冀州之南。始,武王作邑於鎬京,謂之宗周,是為西都,周公攝政五年,成王在豐,欲宅洛邑,使召公先相宅,既成,謂之王城,是為東都,今河南是也,召公既相宅,周公往營成周,今洛陽是也。成王居洛邑,遷殷頑民於成周,復還歸處西都。至於夷厲,政教尤衰,十一世,幽王嬖褒姒,生伯服,廢申后,大子宜咎犇申。申侯與犬戎攻宗周,殺幽王於戲。晉文侯、鄭武公迎宜咎于申而立之,是為平王,以亂故,徙居東都王城。於是王室之尊與諸侯無異,其詩不能復雅,故貶之,謂之王國之變風。

\section*{鄭譜}

初,宣王封母弟友於宗周畿內咸林之地,是為鄭桓公,今京兆鄭縣是其都也。又云為幽王大司徒,甚得周眾與東土之人,問於史伯曰:「王室多故,余愳及焉,其何所可以逃死?」史伯曰:「其濟洛河潁之間乎!是其子男之國,虢鄶為大,虢叔恃勢,鄶仲恃險,皆有驕侈怠慢之心,加之以貪冒,君若以周難之故,寄帑與賄,不敢不許,是驕而貪,必將背君,君以成周之眾奉辭罰罪,無不克矣,若克二邑,鄢蔽補丹依疇歷華,君之土也,脩典刑以守之,惟是可以少固。」桓公從之,言「然之」。後三年,幽王為犬戎所殺,桓公死之,其子武公與晉文侯定平王於東都王城,卒取史伯所云十邑之地,右洛左濟,前華後河,食溱洧焉,今河南新鄭是也。武公又作卿士,國人宜之,鄭之變風又作。

\section*{齊譜}

齊者,古少皥之世爽鳩氏之墟。周武王伐紂,封太師呂望於齊,是謂齊太公,地方百里,都營丘。周公致太平,敷定九畿,復夏禹之舊制。成王用周公之法制,廣大邦國之境,而齊受上公之地,更方五百里。其封域東至于海,西至于河,南至于穆陵,北至于无棣。在禹貢青州岱山之陰,濰淄之野。其子丁公嗣位於王官。後五世,哀公政衰,荒淫怠慢,紀侯譖之於周懿王,使烹焉,齊人變風始作。

\section*{魏譜}

魏者,虞舜夏禹所都之地,在禹貢冀州雷首之北,析城之西,周以封同姓焉。其封域南枕河曲,北涉汾水。昔舜耕於歷山,陶於河濱,禹菲飲食而致孝乎鬼神,惡衣服而致美乎黻冕,卑宮室而盡力乎溝洫,此一帝一王儉約之化於時猶存,及今魏君嗇且褊急,不務廣脩德於民,教以義方。其與秦晉鄰國,日見侵削,國人憂之。當周平桓之世,魏之變風始作。至春秋魯閔公元年,晉獻公竟滅之,以其地賜大夫畢萬,自爾而後,晉有魏氏。

\section*{唐譜}

唐者,帝堯舊都之地,今曰太原晉陽是,堯始居此,後乃遷河東平陽。成王封母弟叔虞於堯之故墟,曰唐侯,南有晉水,至子燮改為晉侯。其封域在禹貢冀州太行恆山之西,太原太岳之野。至曾孫成侯南徙居曲沃,近平陽焉。昔堯之末,洪水九年,下民其咨,萬國不粒,於時殺禮以救艱厄,其流乃被於今。當周公召公共和之時,成侯曾孫僖侯甚嗇愛物,儉不中禮,國人閔之,唐之變風始作。其孫穆侯又徙於絳云。

\section*{秦譜}

秦者,隴西谷名,於禹貢近雍州鳥鼠之山。堯時有伯翳者,實臯陶之子,佐禹治水,水土既平,舜命作虞官,掌上下草木鳥獸,賜姓曰贏。歷夏商興衰,亦世有人焉。周孝王使其末孫非子養馬於汧渭之間,孝王為伯翳能知禽獸之言,子孫不絕,故封非子為附庸,邑之於秦谷。至曾孫秦仲,宣王又命作大夫,始有車馬禮樂侍御之好,國人美之,翳之變風始作。秦仲之孫襄公,平王之初,興兵討西戎以救周,平王東遷王城,乃以岐豐之地賜之,始列為諸侯。遂橫有周西都宗周畿內八百里之地。其封域東至迆山,在荊岐終南惇物之野。至玄孫德公又徙於雍云。

\section*{陳譜}

陳者,大皥虙戲氏之墟。帝舜之冑有虞閼父者,為周武王陶正,武王賴其利器用,與其神明之後,封其子媯滿於陳,都於宛丘之側,是曰陳胡公,以備三恪,妻以元女太姬。其封域在禹貢豫州之東,其地廣平,無名山大澤,西望外方,東不及明豬。太姬無子,好巫覡禱祈鬼神歌舞之樂,民俗化而為之。五世至幽公,當厲王時,政衰,大夫淫荒,所為無度,國人傷而刺之,陳之變風作矣。

\section*{檜譜}

檜者,古高辛氏火正祝融之墟。檜國在禹貢豫州外方之北,滎波之南,居溱洧之間。祝融氏名黎,其後八姓,唯妘姓檜者處其地焉。周夷王厲王之時,檜公不務政事而好絜衣服,大夫去之,於是檜之變風始作。其國北鄰於虢。

\section*{曹譜}

曹者,禹貢兗州陶丘之北地名。周武王既定天下,封弟叔振鐸於曹,今曰濟陰定陶是也。其封域在雷夏菏澤之野。昔堯嘗遊成陽,死而葬焉,舜漁於雷澤,民俗始化,其遺風重厚,多君子,務稼穡,薄衣食,以致蓄積。夾於魯衛之間,又寡於患難,末時富而無教,乃更驕侈。十一世,當周惠王時,政衰,昭公好奢而任小人,曹之變風始作。

\section*{豳譜}

豳者,后稷之曾孫曰公劉者自邰而出所徙戎狄之地名,今屬右扶風栒邑。公劉以夏后大康時失其官守,竄於此地,猶脩后稷之業,勤恤愛民,民咸歸之而國成焉。其封域在禹貢雍州岐山之北,原隰之野。至商之末世,大王又避戎狄之難而入處於岐陽,民又歸之。公劉之出,大王之入,雖有其異,由有事難之故,皆能守后稷之教,不失其德。成王之時,周公避流言之難,出居東都二年,思公劉大王居豳之職,憂念民事至苦之功,以比序己志。後成王迎而反之,攝政致太平,其出入也一德不回,純似於公劉大王之所為。大師大述其志,主意於豳公之事,故別其詩,以為豳國變風焉。

\section*{小大雅譜}

小雅大雅者,周室居西都豐鎬之時詩也。始祖后稷由神氣而生,有播種之功於民,公劉至于大王王季,歷及千載、越異代而列世載其功業,為天下所歸。文王受命,武王遂定天下,盛德之隆,大雅之初,起自文王,至于文王有聲,據盛隆而推原天命,上述祖考之美。小雅自鹿鳴至於魚麗,先其文所以治內,後其武所以治外。此二雅逆順之次,要於極賢聖之情,著天道之助,如此而已矣。

又大雅生民下及卷阿,小雅南有嘉魚下及菁菁者莪,周公成王之時詩也。傳曰「文王基之,武王鑿之,周公內之」,謂其道同,終始相成,比而合之,故大雅十八篇、小雅十六篇為正經。其用於樂,國君以小雅,天子以大雅,然而饗賓或上取,燕或下就。何者?天子饗元侯,歌肆夏、合文王,諸侯歌文王、合鹿鳴,諸侯於鄰國之君與天子於諸侯同,天子諸侯燕群臣及聘問之賓皆歌鹿鳴、合鄉樂。此其著略大校,見在書籍,禮樂崩壞,不可得詳。

大雅民勞、小雅六月之後,皆謂之變雅,美惡各以其時,亦顯善懲過,正之次也。

問者曰,常棣閔管蔡之失道,何故列於文王之詩?曰閔之,閔之者,閔其失兄弟相承順之道,至於被誅,若在成王周公之詩,則是彰其罪,非閔之,故為隱,推而上之,因文王有親兄弟之義。

又問曰,小雅之臣何也獨無刺厲王?曰有焉,十月之交、雨無正、小旻、小宛之詩是也,漢興之初,師移其第耳。亂甚焉,既移文,改其目,義順上下,刺幽王亦過矣。

\section*{周頌譜}

周頌者,周室成功致太平德洽之詩,其作在周公攝政、成王即位之初。頌之言容,天子之德,光被四表,格于上下,無不覆燾,無不持載,此之謂容,於是和樂興焉,頌聲乃作。

禮運曰「政也者,君之所以藏身也,是故夫政必本於天,殽以降命,命降於社之謂殽地,降於祖廟之謂仁義,降於山川之謂興作,降於五祀之謂制度」,又曰「祭帝於郊,所以定天位,祀社於國,所以列地利,祖廟所以本仁,山川所以儐鬼神,五祀所以本事」,又曰「禮行於郊而百神受職焉,禮行於社而百貨可極焉,禮行於祖廟而孝慈服焉,禮行於五祀而正法則焉,故自郊社祖廟山川五祀,義之脩、禮之藏也」,功大如此,可不美報乎?故人君必絜其牛羊,馨其黍稷,齊明而薦之,歌之舞之,所以顯神明、昭至德也。

\section*{魯頌譜}

魯者,少皥摯之墟也,國中有大庭氏之庫,則大庭氏亦居茲乎?在周公歸政,成王封其元子伯禽於魯。其封域在禹貢徐州大野蒙羽之野。自後政衰,國事多廢。十九世至僖公,當周惠王襄王時,而遵伯禽之法,養四種之馬,牧於坰野,尊賢祿士,脩泮宮,崇禮教,僖十六年冬,會諸侯于淮上,謀東略,公遂伐淮夷,僖二十年,新作南門,又脩姜嫄之廟,至於復魯舊制,未徧而薨。國人美其功,季孫行父請命於周而作其頌。文公十三年,太室屋壞。初,成王以周公有太平制典法之勳,命魯郊祭天,三望,如天子之禮,故孔子錄其詩之頌,同於王者之後。

問者曰,列國作詩,未有請於周者,行父請之,何也?曰周尊魯,巡守述職,不陳其詩,至於臣頌君功,樂周室之聞,是以行父請焉。周之不陳其詩者,為優耳,其有大罪,侯伯監之,行人書之,亦示覺焉。

\section*{商頌譜}

商者,契所封之地。有娀氏之女名簡狄者,吞鳦卵而生契,堯之末年,舜舉為司徒,有五教之功,乃賜姓而封之。世有官守,十四世至湯,則受命代夏桀,定天下。後世有中宗者,嚴恭寅畏,天命自度,治民祇懼,不敢荒寧。後有高宗者,舊勞於外,爰洎小人,作其即位,乃或諒闇,三年不言,言乃雍,不敢荒寧,嘉靜殷邦,至于小大,無時或怨。此三王有受命中興之功,時有作詩頌之者。商德之壞,武王伐紂,乃以陶唐氏火正閼伯之墟封紂兄微子啟為宋公,代武庚為商後。其封域在禹貢徐州泗濱,西及豫州盟豬之野。自後政衰,散亡商之禮樂。七世至戴公,時當宣王,大夫正考父者,校商之名頌十二篇於周大師,以那為首,歸以祀其先王。孔子錄詩之時,則得五篇而已,乃列之以備三頌,著為後王之義,監三代之成功,法莫大於是矣。

問者曰,列國政衰則變風作,宋何獨無乎?曰有焉,乃不錄之,王者之後,時王所客也,巡守述職,不陳其詩,亦示無貶黜客之義也。

又問曰,周大師何由得商頌?曰周用六代之樂,故有之。