\chapter{南有嘉魚之什詁訓傳第十七}

\section{南有嘉魚}

%{\footnotesize 四章、章四句}

\textbf{南有嘉魚,樂與賢也。太平之君子至誠,樂與賢者共之也。}{\footnotesize 樂得賢者,與共立於朝,相燕樂也。}

\begin{quoting}\textbf{釋文}自此至菁菁者莪六篇並亡篇三,是成王周公之小雅,成王有雅名,公有雅德,二人協佐以致太平,故亦並為正也。\end{quoting}

\textbf{南有嘉魚,烝然罩罩。}{\footnotesize 江漢之間,魚所產也。罩罩,籗也。箋云烝,塵也,塵然,猶言久如也。言南方水中有善魚,人將久如而俱罩之,遲之也,喻天下有賢者,在位之人將久如而並求致之於朝,亦遲之也,遲之者,謂至誠也。}\textbf{君子有酒,嘉賓式燕以樂。}{\footnotesize 箋云君子,斥時在位者也。式,用也,用酒與賢者燕飲而樂也。}

\begin{quoting}\textbf{馬瑞辰}罩罩、汕汕蓋皆眾魚游水之貌。\end{quoting}

\textbf{南有嘉魚,烝然汕汕。}{\footnotesize 汕汕,樔也。箋云樔者,今之撩罟也。}\textbf{君子有酒,嘉賓式燕以衎。}{\footnotesize 衎,樂也。}

\textbf{南有樛木,甘瓠纍之。}{\footnotesize 興也。纍,蔓也。箋云君子下其臣,故賢者歸往也。}\textbf{君子有酒,嘉賓式燕綏之。}{\footnotesize 箋云綏,安也,與嘉賓燕飲而安之,鄉飲酒曰「賓以我安」。}

\begin{quoting}之,語詞,與下章「思」字同。\end{quoting}

\textbf{翩翩者鵻,烝然來思。}{\footnotesize 鵻,壹宿之鳥。箋云壹宿者,壹意於其所宿之木也,喻賢者有專壹之意於我,我將久如而來,遲之也。}\textbf{君子有酒,嘉賓式燕又思。}{\footnotesize 箋云又,復也,以其壹意,欲復與燕,加厚之。}

\begin{quoting}\textbf{馬瑞辰}又,即今之右字,古右與侑、宥並通用,彤弓詩毛傳「右,勸也」,又當即侑之假借,猶侑可通作右與宥耳。\end{quoting}

\section{南山有臺}

%{\footnotesize 五章、章六句}

\textbf{南山有臺,樂得賢也。得賢則能為邦家立太平之基矣。}{\footnotesize 人君得賢,則其德廣大堅固,如南山之有基址。}

\textbf{南山有臺,北山有萊。}{\footnotesize 興也。臺,夫須也。萊,草也。箋云興者,山之有草木以自覆蓋成其高大,喻人君有賢臣以自尊顯。}\textbf{樂只君子,邦家之基。樂只君子,萬壽無期。}{\footnotesize 基,本也。箋云只之言是也。人君既得賢者,置之於位,又尊敬以禮樂樂之,則能為國家之本,得壽考之福。}

\begin{quoting}孔疏引陸璣義疏云「舊說夫須,莎草也,可為蓑笠」,都人士「臺笠緇撮」。只,語詞,左傳襄二十四年及昭十三年引詩作「樂旨君子」。\end{quoting}

\textbf{南山有桑,北山有楊。樂只君子,邦家之光。樂只君子,萬壽無疆。}{\footnotesize 箋云光,明也,政教明,有榮曜。}

\textbf{南山有杞,北山有李。樂只君子,民之父母。樂只君子,德音不已。}{\footnotesize 箋云已,止也,不止者,言長見稱頌也。}

\textbf{南山有栲,北山有杻。}{\footnotesize 栲,山樗。杻,檍也。}\textbf{樂只君子,遐不眉壽。樂只君子,德音是茂。}{\footnotesize 眉壽,秀眉也。箋云遐,遠也,遠不眉壽者,言其近眉壽也。茂,盛也。}

\begin{quoting}\textbf{經傳釋詞}遐,何也,遐不,何不也。\end{quoting}

\textbf{南山有枸,北山有楰。}{\footnotesize 枸,枳枸。楰,鼠梓。}\textbf{樂只君子,遐不黃耇。樂只君子,保艾爾後。}{\footnotesize 黃,黃髮也。耇老、艾養、保安也。}

\begin{quoting}枸 \texttt{jǔ},又名枳枸。說文「耇,老人面凍黎若垢」。\textbf{馬瑞辰}據毛傳先艾後保,似經文原作「艾保爾後」。\end{quoting}

\section{由庚·崇丘·由儀}

\textbf{由庚,萬物得由其道也。崇丘,萬物得極其高大也。由儀,萬物之生各得其宜也。有其義而亡其辭。}{\footnotesize 此三篇者,鄉飲酒、燕禮亦用焉,曰「乃間歌魚麗,笙由庚,歌南有嘉魚,笙崇丘,歌南山有臺,笙由儀」,亦遭世亂而亡之。燕禮又有「升歌鹿鳴,下管新宮」,新宮亦詩篇名也,辭義皆亡,無以知其篇第之處。}

\begin{quoting}\textbf{釋文}此三篇義與南陔等同,依六月序,由庚在南有嘉魚前,崇丘在南山有臺前,今同在此者,以其俱亡,使相從耳。\end{quoting}

\section{蓼蕭}

%{\footnotesize 四章、章六句}

\textbf{蓼蕭,澤及四海也。}{\footnotesize 九夷、八狄、七戎、六蠻謂之四海,同在九州之外,雖有大者,爵不過子,虞書曰「州有十二師,外薄四海,咸建五長」。}

\textbf{蓼彼蕭斯,零露湑兮。}{\footnotesize 興也。蓼,長大貌。蕭,蒿也。湑湑然,蕭上露貌。箋云興者,蕭,香物之微者,喻四海之諸侯,亦國君之賤者。露者,天所以潤萬物,喻王者恩澤不為遠國則不及也。}\textbf{既見君子,我心寫兮。}{\footnotesize 輸寫其心也。箋云既見君子者,遠國之君朝見於天子也。我心寫者,舒其情意,無留恨也。}\textbf{燕笑語兮,是以有譽處兮。}{\footnotesize 箋云天子與之燕而笑語,則遠國之君各得其所,是以稱揚德美,使聲譽常處天子。}

\begin{quoting}說文段注「按凡傾吐曰寫,故作字、作畫皆曰寫」,\textbf{陳奐}輸亦寫也。譽處,安樂,蘇轍詩集傳「譽、豫通,凡詩之譽皆言樂也」,禮記檀弓「何以處我」,鄭注「處,安也」。或曰譽同與,與處為古人常語,\textbf{于省吾}二詩皆言相見之後,情孚意愜,無寂寞之憂,故云是以有與處兮。\end{quoting}

\textbf{蓼彼蕭斯,零露瀼瀼。}{\footnotesize 瀼瀼,露蕃貌。}\textbf{既見君子,為龍為光。}{\footnotesize 龍,寵也。箋云為寵為光,言天子恩澤光耀,被及己也。}\textbf{其德不爽,壽考不忘。}{\footnotesize 爽,差也。}

\begin{quoting}忘,同亡。\end{quoting}

\textbf{蓼彼蕭斯,零露泥泥。}{\footnotesize 泥泥,霑濡也。}\textbf{既見君子,孔燕豈弟。}{\footnotesize 豈樂、弟易也。箋云孔甚、燕安也。}\textbf{宜兄宜弟,令德壽豈。}{\footnotesize 為兄亦宜,為弟亦宜。}

\begin{quoting}豈弟,同愷悌,和樂平易。豈,同愷。\end{quoting}

\textbf{蓼彼蕭斯,零露濃濃。}{\footnotesize 濃濃,厚貌。}\textbf{既見君子,鞗革沖沖。和鸞雝雝,萬福攸同。}{\footnotesize 鞗,轡也。革,轡首也。沖沖,垂飾貌。在軾曰和,在鑣曰鸞。箋云此說天子之車飾者,諸侯燕見天子,天子必乘車迎于門,是以云然。攸,所也。}

\begin{quoting}\textbf{陳奐}鞗 \texttt{tiáo},當作鋚,革,古文勒,說文云「鋚,轡首銅也,勒,馬頭絡銜也,銜,馬勒口中也」,是轡之絡馬首者謂之勒,勒關馬口者謂之銜,勒以革為之,故字從革,勒絡馬首所垂之轡其上飾謂之鋚,鋚以金為之。\textbf{賈誼}新書容經篇曰登車則馬行,馬行則鸞鳴,鸞鳴則和應,聲曰和,和則敬,故詩曰「和鸞雝雝,萬福攸同」,言動有紀度則萬福之所聚也。\end{quoting}

\section{湛露}

%{\footnotesize 四章、章四句}

\textbf{湛露,天子燕諸侯也。}{\footnotesize 燕,謂與之燕飲酒也。諸侯朝覲會同,天子與之燕,所以示慈惠。}

\begin{quoting}左傳文四年寧武子曰「昔諸侯朝正于王,王宴樂之,于是賦湛露,則天子當陽,諸侯用命也」。\end{quoting}

\textbf{湛湛露斯,匪陽不晞。}{\footnotesize 興也。湛湛,露茂盛貌。陽,日也。晞,乾也。露雖湛湛然,見陽則乾。箋云興者,露之在物湛湛然,使物柯葉低垂,喻諸侯受燕爵,其儀有似醉之貌,諸侯旅酬之則猶然,唯天子賜爵則貌變,肅敬承命,有似露見日則晞。}\textbf{厭厭夜飲,不醉無歸。}{\footnotesize 厭厭,安也。夜飲,私燕也。宗子將有事則族人皆侍,不醉而出,是不親也,醉而不出,是渫宗也。箋云天子燕諸侯之禮亡,此假宗子與族人燕為說爾。族人猶群臣也,其醉不出,不醉而出,猶諸侯之儀也,飲酒至夜,猶云「不醉無歸」,此天子於諸侯之儀。燕飲之禮,宵則兩階及庭門皆設燭焉。}

\begin{quoting}陽,通暘,說文「暘,日出也」。夜飲,古人稱燕私,\textbf{孔疏}楚茨「備言燕私」,傳曰「燕而盡其私恩」,明夜飲者,亦君留而盡私恩之義,故言燕私也。\end{quoting}

\textbf{湛湛露斯,在彼豐草。厭厭夜飲,在宗載考。}{\footnotesize 豐,茂也。夜飲必於宗室。箋云豐草,喻同姓諸侯也。載之言則也。考,成也。夜飲之禮,在宗室同姓諸侯則成之,於庶姓其讓之則止。昔者陳敬仲飲桓公酒而樂,桓公命以火繼之,敬仲曰「臣卜其晝,未卜其夜」,於是乃止,此之謂不成也。}

\begin{quoting}\textbf{姚際恆}宗,宗廟也,大雅鳧鷖亦云「既燕于宗」,聘、享皆于廟,則燕亦在廟也。\textbf{林義光}考,祭享也,彝器言享孝者亦作享考,此詩「在宗載考」,即享考宗室之義。\end{quoting}

\textbf{湛湛露斯,在彼杞棘。顯允君子,莫不令德。}{\footnotesize 箋云杞也棘也異類,喻庶姓諸侯也。令,善也,無不善其德,言飲酒不至於醉。}

\begin{quoting}\textbf{孔疏}顯允,明信之君子。\textbf{于省吾}允應讀作駿,訓大,凡典籍中的駿字,金文通作㽙。\end{quoting}

\textbf{其桐其椅,其實離離。豈弟君子,莫不令儀。}{\footnotesize 離離,垂也。箋云桐也椅也同類而異名,喻二王之後也。其實離離,喻其薦俎禮物多於諸侯也。飲酒不至於醉,徒善其威儀而已,謂陔節也。}

\begin{quoting}離離,猶今言累累。\end{quoting}

\section{彤弓}

%{\footnotesize 三章、章六句}

\textbf{彤弓,天子錫有功諸侯也。}{\footnotesize 諸侯敵王所愾而獻其功,王饗禮之,於是賜彤弓一、彤矢百、玈弓矢千。凡諸侯,賜弓矢,然後專征伐。}

\begin{quoting}左傳文四年寧武子聘魯,文公與之宴,為賦彤弓。\textbf{朱熹}引東萊呂氏曰「受言藏之」言其重也,弓人所獻,藏之王府以待有功,不敢輕與人也,「中心貺之」言其誠也,中心實欲貺之,非由外也,「一朝饗之」言其速也,以王府寶藏之弓,一朝舉以畀人,未嘗有遲留顧惜之意也。\end{quoting}

\textbf{彤弓弨兮,受言藏之。}{\footnotesize 彤弓,朱弓也,以講德習射。弨,弛貌。言,我也。箋云言者,謂王策命也,王賜朱弓,必策其功以命之,受出藏之,乃反入也。}\textbf{我有嘉賓,中心貺之。}{\footnotesize 貺,賜也。箋云貺者,欲加恩惠也,王意殷勤於賓,故歌序之。}\textbf{鐘鼓既設,一朝饗之。}{\footnotesize 箋云大飲賓曰饗。一朝,猶早朝。}

\begin{quoting}嚴粲詩緝「賜弓不張」。說文無貺字,古通況,爾雅釋詁「況,賜也」。\textbf{馬瑞辰}廣韻「況,喜也」,貺亦訓喜,與下文「中心喜之、中心好之」同義。\textbf{陳奐}一朝,猶終朝也。\end{quoting}

\textbf{彤弓弨兮,受言載之。}{\footnotesize 載以歸也。箋云出載之車也。}\textbf{我有嘉賓,中心喜之。}{\footnotesize 喜,樂也。}\textbf{鐘鼓既設,一朝右之。}{\footnotesize 右,勸也。箋云右之者,「主人獻之,賓受爵,奠于薦右,既祭俎,乃席末坐,卒爵」之謂也。}

\begin{quoting}\textbf{胡承珙}上言鐘鼓既設,則右、醻明是饗時之事,右之、醻之,當主侑幣、醻幣為義,左傳莊十八年「虢公、晉侯朝王,王饗醴,命之侑,皆賜玉五瑴、馬三匹」,僖二十五年「晉侯朝王,王饗醴,命之侑」,僖二十八年「晉侯獻楚俘于王,王饗醴,命晉侯宥」,是則饗醴本有侑幣,王禮或更有玉與馬。\end{quoting}

\textbf{彤弓弨兮,受言櫜之。}{\footnotesize 櫜,韜也。}\textbf{我有嘉賓,中心好之。}{\footnotesize 好,說也。}\textbf{鐘鼓既設,一朝醻之。}{\footnotesize 醻,報也。箋云飲酒之禮,主人獻賓,賓酢主人,主人又飲而酌賓,謂之醻,醻,猶厚也、勸也。}

\begin{quoting}櫜 \texttt{gāo},放入弓袋。\end{quoting}

\section{菁菁者莪}

%{\footnotesize 四章、章四句}

\textbf{菁菁者莪,樂育材也。君子能長育人材,則天下喜樂之矣。}{\footnotesize 樂育材者,歌樂人君教學國人秀士,選士俊士,造士進士,養之以漸,至於官之。}

\textbf{菁菁者莪,在彼中阿。}{\footnotesize 興也。菁菁,盛貌。莪,蘿蒿也。中阿,阿中也,大陵曰阿。君子能長育人材,如阿之長莪菁菁然。箋云長育之者,既教學之,又不征役也。}\textbf{既見君子,樂且有儀。}{\footnotesize 箋云既見君子者,官爵之而得見也,見則心既喜樂,又以禮儀見接。}

\begin{quoting}菁菁,韓詩作蓁蓁。莪,今名茵陳。\textbf{嚴粲}詩中既見君子二十有二,見於九詩,其接句皆述喜之情,謂見君子者喜,非所見者喜也。\end{quoting}

\textbf{菁菁者莪,在彼中沚。}{\footnotesize 中沚,沚中也。}\textbf{既見君子,我心則喜。}{\footnotesize 喜,樂也。}

\textbf{菁菁者莪,在彼中陵。}{\footnotesize 中陵,陵中也。}\textbf{既見君子,錫我百朋。}{\footnotesize 箋云古者貨貝,五貝為朋,賜我百朋,得祿多,言得意也。}

\textbf{汎汎楊舟,載沈載浮。}{\footnotesize 楊木為舟,載沉亦沉,載浮亦浮。箋云舟者,沉物亦載,浮物亦載,喻人君用士,文亦用,武亦用,於人之材無所廢。}\textbf{既見君子,我心則休。}{\footnotesize 箋云休者,休休然。}

\begin{quoting}國語周語「為晉休戚」,韋昭注「休,喜也」。\textbf{陳奐}末章又以舟之載物,興君子之用人材。\end{quoting}

\section{六月}

%{\footnotesize 六章、章八句}

\textbf{六月,宣王北伐也。鹿鳴廢則和樂缺矣,四牡廢則君臣缺矣,皇皇者華廢則忠信缺矣,常棣廢則兄弟缺矣,伐木廢則朋友缺矣,天保廢則福祿缺矣,采薇廢則征伐缺矣,出車廢則功力缺矣,杕杜廢則師眾缺矣,魚麗廢則法度缺矣,南陔廢則孝友缺矣,白華廢則廉耻缺矣,華黍廢則蓄積缺矣,由庚廢則陰陽失其道理矣,南有嘉魚廢則賢者不安、下不得其所矣,崇丘廢則萬物不遂矣,南山有臺廢則為國之基隊矣,由儀廢則萬物失其道理矣,蓼蕭廢則恩澤乖矣,湛露廢則萬國離矣,彤弓廢則諸夏衰矣,菁菁者莪廢則無禮儀矣,小雅盡廢,則四夷交侵,中國微矣。}{\footnotesize 六月言周室微而復興,美宣王之北伐也。}

\begin{quoting}\textbf{釋文}從此至無羊十四篇,是宣王之變小雅。\end{quoting}

\textbf{六月棲棲,戎車既飭。四牡騤騤,載是常服。}{\footnotesize 棲棲,簡閱貌。飭,正也。日月為常。服,戎服也。箋云記六月者,盛夏出兵,明其急也。戎車,革輅之等也,其等有五。戎車之常服,韋弁服也。}\textbf{玁狁孔熾,我是用急。}{\footnotesize 熾,盛也。箋云此序吉甫之意也,北狄來侵甚熾,故王以是急遣我。}\textbf{王于出征,以匡王國。}{\footnotesize 箋云于曰、匡正也。王曰「今女出征玁狁,以正王國之封畿」。}

\begin{quoting}文選李注「棲遑,不安居之意也」。是用,即因此。\end{quoting}

\textbf{比物四驪,閑之維則。}{\footnotesize 物,毛物也。則,法也。言先教戰而後用師。}\textbf{維此六月,既成我服。我服既成,于三十里。}{\footnotesize 師行三十里。箋云王既成我戎服,將遣之,戒之曰「日行三十里,可以舍息」。}\textbf{王于出征,以佐天子。}{\footnotesize 出征以佐其為天子也。箋云王曰「今女出征伐,以佐助我天子之事」,禦北狄也。}

\begin{quoting}毛物,周禮夏官校人鄭注「毛,馬齊其色,物,馬齊其力」。閑,訓練。則,本義為籌劃物,說文段注「籌劃物者,定其差等而各為介畫也,今云科則是也,介畫之,故從刀,引申之為法則,假借之語詞」。于,往。\end{quoting}

\textbf{四牡脩廣,其大有顒。}{\footnotesize 脩長、廣大也。顒,大貌。}\textbf{薄伐玁狁,以奏膚公。}{\footnotesize 奏為、膚大、公功也。}\textbf{有嚴有翼,共武之服。}{\footnotesize 嚴,威嚴也。翼,敬也。箋云服,事也。言今師之群帥有威嚴者,有恭敬者,而共典是兵事,言文武之人備。}\textbf{共武之服,以定王國。}{\footnotesize 箋云定,安也。}

\begin{quoting}\textbf{馬瑞辰}「共武之服」即言敬武之事,正承上「有嚴有翼」言之,嚴、翼皆恭也。\end{quoting}

\textbf{玁狁匪茹,整居焦穫。侵鎬及方,至于涇陽。}{\footnotesize 焦穫,周地接于玁狁者。箋云匪非、茹度也。鎬也方也皆北方地名。言玁狁之來侵,非其所當度為也,乃自整齊而處周之焦穫,來侵至涇水之北,言其大恣也。}\textbf{織文鳥章,白旆央央。}{\footnotesize 鳥章,錯革鳥為章也。白旆,繼旐者也。央央,鮮明貌。箋云織,徽織也。鳥章,鳥隼之文章,將帥以下衣皆著焉。}\textbf{元戎十乘,以先啟行。}{\footnotesize 元,大也。夏后氏曰鉤車,先正也,殷曰寅車,先疾也,周曰元戎,先良也。箋云鉤,鉤鞶,行曲直有正也。寅,進也。二者及元戎皆可以先前啟突敵陳之前行,其制之同異未聞。}

\begin{quoting}廣雅「茹,弱也」。整,整隊。織,同識,織文即徽號。爾雅釋天孫炎注「錯,置也,革,急也,畫急疾之鳥于縿也」。白,魯詩作帛。\end{quoting}

\textbf{戎車既安,如輊如軒。四牡既佶,既佶且閑。}{\footnotesize 輊摯、佶正也。箋云戎車之安,從後視之如摯,從前視之如軒,然後適調也。佶,壯健之貌。}\textbf{薄伐玁狁,至于大原。}{\footnotesize 言逐出之而已。}\textbf{文武吉甫,萬邦為憲。}{\footnotesize 吉甫,尹吉甫也,有文有武。憲,法也。箋云吉甫,此時大將也。}

\begin{quoting}\textbf{胡承珙}上二句言車之善,下二句言馬之善,車以平均調適為善,馬以整齊馴習為善,佶者整齊,閑者馴習。大原,顧炎武謂在今甘肅平涼,胡渭謂在今寧夏固原附近,陳奐謂在平涼北、固原東。\end{quoting}

\textbf{吉甫燕喜,既多受祉。}{\footnotesize 祉,福也。箋云吉甫既伐玁狁而歸,天子以燕禮樂之,則歡喜矣,又多受賞賜也。}\textbf{來歸自鎬,我行永久。飲御諸友,炰鱉膾鯉。}{\footnotesize 御,進也。箋云御,侍也。王以吉甫遠從鎬地來,又日月長久,今飲之酒,使其諸友恩舊者侍之,又加其珍美之饌,所以極勸之也。}\textbf{侯誰在矣,張仲孝友。}{\footnotesize 侯,維也。張仲,賢臣也。善父母為孝,善兄弟為友,使文武之臣征伐,與孝友之臣處內。箋云張仲,吉甫之友,其性孝友。}

\begin{quoting}我行永久,言歸途之遠。炰,同缹,烝也。膾,細切肉也。\end{quoting}

\section{采芑}

%{\footnotesize 四章、章十二句}

\textbf{采芑,宣王南征也。}

\textbf{薄言采芑,于彼新田,于此菑畝。}{\footnotesize 興也。芑,菜也。田一歲曰菑,二歲曰新田,三歲曰畬。宣王能新美天下之士,然後用之。箋云興者,新美之,喻和治其家、養育其身也。士,軍士也。}\textbf{方叔涖止,其車三千,師干之試。}{\footnotesize 方叔,卿士也,受命而為將也。涖臨、師眾、干扞、試用也。箋云方叔臨視此戎車三千乘,其士卒皆有佐師扞敵之用爾。司馬法,兵車一乘,甲士三人,步卒七十二人。宣王承亂,羨卒盡起。}\textbf{方叔率止,乘其四騏,四騏翼翼。}{\footnotesize 箋云率者,率此戎車士卒而行也。翼翼,壯健貌。}\textbf{路車有奭,簟茀魚服,鉤膺鞗革。}{\footnotesize 奭,赤貌。鉤膺,樊纓也。箋云茀之言蔽也,車之蔽飾,象席文也。魚服,矢服也。鞗革,轡首垂也。}

\begin{quoting}\textbf{陳奐}案新、菑為休耕之田,至畬而出耕,新田菑畝中得有芑菜可采,以喻國家人材養蓄之以待足用,凡軍士起於田畝,故詩人假以為興。\textbf{馬瑞辰}此師干當讀干戈之干,謂盾也。奭,同赫,說文「赫,火赤貌」。\textbf{陳奐}人之纓結頷下,馬之纓結胸前,小戎傳「膺,馬帶也」,纓即馬帶,䋣下垂,其上有鉤金以為飾。\end{quoting}

\textbf{薄言采芑,于彼新田,于此中鄉。}{\footnotesize 鄉,所也。箋云中鄉,美地名。}\textbf{方叔涖止,其車三千,旂旐央央。}{\footnotesize 箋云交龍為旂,龜蛇為旐。此言軍眾將帥之車皆備。}\textbf{方叔率止,約軝錯衡,八鸞瑲瑲。}{\footnotesize 軝,長轂之軝也,朱而約之。錯衡,文衡也。瑲瑲,聲也。}\textbf{服其命服,朱芾斯皇,有瑲葱珩。}{\footnotesize 朱芾,黃朱芾也。皇,猶煌煌也。瑲,珩聲也。葱,蒼也。三命葱珩,言周室之強、車服之美也。言其強美,斯劣矣。箋云命服者,命為將,受王命之服也。天子之服韋弁服,朱衣裳也。}

\begin{quoting}\textbf{馬瑞辰}廣雅「所,居也」,古者公田為居,廬舍在內,環廬舍種桑麻雜菜,小雅所云「中田有廬」也,中鄉當指中田有廬言之,傳訓鄉為所,亦以所為居也。芾,同韍,蔽膝,斯干鄭箋「天子純朱,諸侯黃朱」。\end{quoting}

\textbf{鴥彼飛隼,其飛戾天,亦集爰止。}{\footnotesize 戾,至也。箋云隼,急疾之鳥也,飛乃至天,喻士卒勁勇,能深攻入敵也。爰,於也。亦集於其所止,喻士卒須命乃行也。}\textbf{方叔涖止,其車三千,師干之試。}{\footnotesize 箋云三稱此者,重師也。}\textbf{方叔率止,鉦人伐鼓,陳師鞠旅。}{\footnotesize 伐,擊也。鉦以靜之,鼓以動之。鞠,告也。箋云鉦也鼓也各有人焉,言鉦人伐鼓,互言爾。二千五百人為師,五百人為旅。此言將戰之日,陳列其師旅,誓告之也。陳師告旅,亦互言之。}\textbf{顯允方叔,伐鼓淵淵,振旅闐闐。}{\footnotesize 淵淵,鼓聲也。入曰振旅,復長幼也。箋云伐鼓淵淵,謂戰時進士眾也,至戰止將歸,又振旅伐鼓闐闐然。振,猶止也。旅,眾也。春秋傳曰「出曰治兵,入曰振旅,其禮一也」。}

\textbf{蠢爾蠻荊,大邦為讎。}{\footnotesize 蠢,動也。蠻荊,荊州之蠻也。箋云大邦,列國之大也。}\textbf{方叔元老,克壯其猶。}{\footnotesize 元,大也。五官之長出於諸侯,曰天子之老。壯大、猶道也。箋云猶,謀也,謀,兵謀也。}\textbf{方叔率止,執訊獲醜。}{\footnotesize 箋云方叔率其士眾,執將可言問所獲敵人之眾以還歸也。}\textbf{戎車嘽嘽,嘽嘽焞焞,如霆如雷。}{\footnotesize 嘽嘽,眾也。焞焞,盛貌。箋云言戎車既眾盛,其威又如雷霆,言雖久在外,無罷勞也。}\textbf{顯允方叔,征伐玁狁,蠻荊來威。}{\footnotesize 箋云方叔先與吉甫征伐玁狁,今特往伐蠻荊,皆使來服於宣王之威,美其功之多也。}

\begin{quoting}猶,韓詩、魯詩作猷。嘽 \texttt{tān},兵車行軍聲。焞 \texttt{tūn},魯詩作推。來,語中助詞,表倒裝,如谷風「伊余來塈」,四牡「將母來諗」等。\end{quoting}

\section{車攻}

%{\footnotesize 八章、章四句}

\textbf{車攻,宣王復古也。宣王能內修政事,外攘夷狄,復文武之竟土,修車馬,備器械,復會諸侯於東都,因田獵而選車徒焉。}{\footnotesize 東都,王城也。}

\textbf{我車既攻,我馬既同。}{\footnotesize 攻堅、同齊也。宗廟齊豪,尚純也,戎事齊力,尚強也,田獵齊足,尚疾也。}\textbf{四牡龐龐,駕言徂東。}{\footnotesize 龐龐,充實也。東,洛邑也。}

\begin{quoting}說文「龐,高屋也」,段注「謂屋之高者也,故字從广,引申之凡高大之稱」。\end{quoting}

\textbf{田車既好,四牡孔阜。東有甫草,駕言行狩。}{\footnotesize 甫,大也。田者,大艾草以為防,或舍其中,褐纏旃以為門,裘纏質以為槸,間容握,驅而入,擊則不得入,左者之左,右者之右,然後焚而射焉,天子發然後諸侯發,諸侯發然後大夫士發,天子發抗大綏,諸侯發抗小綏,獻禽於其下,故戰不出頃,田不出防,不逐奔走,古之道也。箋云甫草者,甫田之草也,鄭有甫田。}

\begin{quoting}爾雅「火田為狩」,郭注「放火燒草獵亦為狩」。\end{quoting}

\textbf{之子于苗,選徒嚻嚻。}{\footnotesize 之子,有司也。夏獵曰苗。嚻嚻,聲也。維數車徒者為有聲也。箋云于,曰也。}\textbf{建旐設旄,薄狩于敖。}{\footnotesize 敖,地名。箋云獸,田獵搏獸也。敖,鄭地,今近滎陽。}

\begin{quoting}選,通算,清點。嚻 \texttt{áo}。薄狩有作搏獸者,誤,文選東京賦、水經濟水注、後漢書安帝紀注等引詩皆作薄狩。\end{quoting}

\textbf{駕彼四牡,四牡奕奕。}{\footnotesize 言諸侯來會也。}\textbf{赤芾金舄,會同有繹。}{\footnotesize 諸侯赤芾金舄,舄,達屨也。時見曰會,殷見曰同。繹,陳也。箋云金舄,黃朱色也。}

\begin{quoting}說文「䮸,馬行徐而疾,詩曰四牡䮸䮸」,又巧言傳、韓奕傳「奕奕,大貌」。釋名「複其下曰舄 \texttt{xì},舄,腊也,行禮久立,地或泥濕,故複其末下使乾腊也」。\end{quoting}

\textbf{決拾既佽,弓矢既調。}{\footnotesize 決,鉤弦也。拾,遂也。佽,利也。箋云佽,謂手指相次比也。調,謂弓強弱與矢輕重相得。}\textbf{射夫既同,助我舉柴。}{\footnotesize 柴,積也。箋云既同,已射,同復將射之位也,雖不中,必助中者舉積禽也。}

\begin{quoting}周禮繕人鄭注「詩家說,或謂決謂引弦彄也,拾謂韝扞也,玄謂決,挾矢時所以持弦飾也,著右手巨指,韝扞,著左臂裏,以韋為之」。\textbf{陳奐}同,猶合也,既同,言已合耦也。舉,取。柴 \texttt{zì},魯詩作胔,齊詩、韓詩作㧘。\end{quoting}

\textbf{四黃既駕,兩驂不猗。}{\footnotesize 言御者之良也。}\textbf{不失其馳,舍矢如破。}{\footnotesize 言習於射御法也。箋云御者之良,得舒疾之中,射者之工,矢發則中,如椎破物也。}

\begin{quoting}猗,同倚,偏斜。不失其馳,孟子滕文公引詩,趙注「言御者不失其驅馳之法」。\textbf{經傳釋詞}如破,而破也。\end{quoting}

\textbf{蕭蕭馬鳴,悠悠旆旌。}{\footnotesize 言不讙譁也。}\textbf{徒御不驚,大庖不盈。}{\footnotesize 徒,輦也。御,御馬也。不驚,驚也。不盈,盈也。一曰乾豆,二曰賓客,三日充君之庖,故自左膘而射之,達于右腢,為上殺,射右耳本,次之,射左髀,達于右䯚,為下殺。面傷不獻,踐毛不獻,不成禽不獻。禽雖多,擇取三十焉,其餘以與大夫士。以習射於澤宮,田雖得禽,射不中不得取禽,田雖不得禽,射中則得取禽。古者以辭讓取,不以勇力取。箋云「不驚,驚也,不盈,盈也」,反其言,美之也。「射右耳本」,射當為達。三十者,每禽三十也。}

\begin{quoting}徒御,亦名輦,說文「輦,人輓車也」,\textbf{王念孫}廣雅疏證曰輦之言連也,連者引也,引之而行曰輦,以其徒行而引車,故亦曰徒御。不,語詞。驚,同警,孔疏「相警戒也」。\end{quoting}

\textbf{之子于征,有聞無聲。}{\footnotesize 有善聞而無諠譁之聲。箋云晉人伐鄭,陳成子救之,舍於柳舒之上,去穀七里,穀人不知,可謂有聞無聲。}\textbf{允矣君子,展也大成。}{\footnotesize 箋云允信、展誠也。大成,謂致太平也。}

\begin{quoting}爾雅釋詁「允、展,信也,展、允,誠也」。\end{quoting}

\section{吉日}

%{\footnotesize 四章、章六句}

\textbf{吉日,美宣王田也。能慎微接下,無不自盡以奉其上焉。}

\begin{quoting}\textbf{陳奐}昭三年左傳「鄭伯如楚,子產相,楚子享之,賦吉日,既享,子產乃具田備」,案此吉日為出田之證,車攻會諸侯而遂田獵,吉日則專美宣王田也,一在東都,一在西都。\end{quoting}

\textbf{吉日維戊,既伯既禱。}{\footnotesize 維戊,順類乘牡也。伯,馬祖也。重物慎微,將用馬力,必先為之禱其祖。禱,禱獲也。箋云戊,剛日也,故乘牡為順類也。}\textbf{田車既好,四牡孔阜。升彼大阜,從其群醜。}{\footnotesize 箋云醜,眾也,田而升大阜,從禽獸之群眾也。}

\begin{quoting}\textbf{朱熹}以下章推之,是日也其為戊辰與。伯,同禡 \texttt{mà},師祭,說文「師所行止,恐有慢其神,下而祀之曰禡」。禱,同禂 \texttt{dǎo},說文「禂,禱牲,馬祭也」。\end{quoting}

\textbf{吉日庚午,既差我馬。}{\footnotesize 外事以剛日。差,擇也。}\textbf{獸之所同,麀鹿麌麌。}{\footnotesize 鹿牝曰麀。麌麌,眾多也。箋云同,猶聚也。麕牡曰麌,麌復麌,言多也。}\textbf{漆沮之從,天子之所。}{\footnotesize 漆沮之水,麀鹿所生也,從漆沮驅禽而至天子之所。}

\begin{quoting}麀 \texttt{yōu},引申為凡牝之稱。麌 \texttt{yǔ}。\end{quoting}

\textbf{瞻彼中原,其祁孔有。}{\footnotesize 祁,大也。箋云祁,當作麎,麎,麋牝也,中原之野甚有之。}\textbf{儦儦俟俟,或群或友。}{\footnotesize 趨則儦儦,行則俟俟。獸三曰群,二曰友。}\textbf{悉率左右,以燕天子。}{\footnotesize 驅禽之左右以安待天子。箋云率,循也,悉驅禽順其左右之宜,以安待王之射也。}

\begin{quoting}儦 \texttt{biāo},俟 \texttt{sì}。\textbf{胡承珙}率有驅義,六朝人每以驅率連文。\end{quoting}

\textbf{既張我弓,既挾我矢。發彼小豝,殪此大兕。}{\footnotesize 殪,壹發而死,言能中微而制大也。箋云豕牡曰豝。}\textbf{以御賓客,且以酌醴。}{\footnotesize 饗醴,天子之飲酒也。箋云御賓客者,給賓客之御也,賓客謂諸侯也。酌醴,酌而飲群臣,以為俎實也。}

\begin{quoting}儀禮鄉射「凡挾矢,於二指之間橫之」,鄭注「二指謂左右手之第二指,此以食指、將指挾之」。騶虞傳「豕牝曰豝」。醴,甜酒,段注「如今江東人家之白酒」。\end{quoting}

%\begin{flushright}南有嘉魚之什十篇、四十六章、二百七十二句\end{flushright}