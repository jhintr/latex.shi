\chapter{雅量第六}

\subsection*{1}

\textbf{豫章太守顧劭,}{\footnotesize \textbf{環濟吳紀}曰劭,字孝則,吳郡人,年二十七起家為豫章太守,舉善以教民,風化大行。}\textbf{是雍之子,劭在郡卒,雍盛集僚屬,自圍棊,}{\footnotesize \textbf{江表傳}曰雍,字元歎,曾就蔡伯喈,伯喈賞異之,以其名與之。\textbf{吳志}曰雍累遷尚書令,封陽遂鄉侯,拜侯還第,家人不知,為人不飲酒,寡言語,孫權嘗曰「顧侯在坐,令人不樂」,位至丞相。}\textbf{外啓信至,而無兒書,雖神氣不變,而心了其故,以爪掐掌,血流霑褥,賓客既散,方歎曰:「已無延陵之高,豈可有喪明之責?」}{\footnotesize \textbf{禮記}曰延陵季子適齊,及其反也,其長子死,葬於嬴、博之間,孔子曰「延陵季子,吳之習於禮者也」,往而觀其葬焉,其坎深不至於泉,其斂以時服,既葬而封,廣輪掩坎,其高可隱也,既封,左袒,右還其封,且號者三,曰「骨肉歸復於土,命也,若魂氣則無不之也」,而遂行,孔子曰「延陵季子之於禮也,其合矣乎」。子夏哭其子而喪其明,曾子弔之,曰「朋友喪明則哭之」,曾子哭,子夏亦哭,曰「天乎,予之無罪也」,曾子怒曰「商,汝何無罪也?吾與汝事夫子於洙泗之間,退而老於西河之上,使西河之民疑汝於夫子,爾罪一也,喪爾親,使民未有聞焉,爾罪二也,喪爾子,喪爾明,爾罪三也」,子夏投其杖而拜曰「吾過矣,吾過矣」。}\textbf{於是豁情散哀,顏色自若。}

\subsection*{2}

\textbf{嵇中散臨刑東市,神氣不變,索琴彈之,奏廣陵散,曲終,曰:「袁孝尼嘗請學此散,吾靳固未與,廣陵散於今絕矣。」}{\footnotesize \textbf{晉陽秋}曰初,康與東平呂安親善,安嫡兄遜淫安妻徐氏,安欲告遜遣妻,以咨於康,康喻而抑之,遜內不自安,陰告安撾母,表求徙邊,安當徙,訴自理,辭引康。\textbf{文士傳}曰呂安罹事,康詣獄以明之,鍾會庭論康,曰「今皇道開明,四海風靡,邊鄙無詭隨之民,街巷無異口之議,而康上不臣天子,下不事王侯,輕時傲世,不為物用,無益於今,有敗於俗,昔太公誅華士,孔子戮少正卯,以其負才亂群惑眾也,今不誅康,無以清絜王道」,於是錄康閉獄,臨死,而兄弟親族咸與共別,康顏色不變,問其兄曰「向以琴來不邪」,兄曰「以來」,康取調之,為太平引,曲成,歎曰「太平引於今絕也」。}\textbf{太學生三千人上書,請以為師,不許,文王亦尋悔焉。}{\footnotesize \textbf{王隱晉書}曰康之下獄,太學生數千人請之,于時豪俊皆隨康入獄,悉解喻,一時散遣,康竟與安同誅。}

\subsection*{3}

\textbf{夏侯太初嘗倚柱作書,時大雨,霹靂破所倚柱,衣服燋然,神色無變,書亦如故,賓客左右,皆跌蕩不得住。}{\footnotesize 見顧愷之書贊。\textbf{語林}曰太初從魏帝拜陵,陪列於松柏下,時暴雨霹靂,正中所立之樹,冠冕焦壞,左右覩之皆伏,太初顏色不改。\textbf{臧榮緒}又以為諸葛誕也。}

\subsection*{4}

\textbf{王戎七歲,嘗與諸小兒遊,看道邊李樹多子折枝,諸兒競走取之,惟戎不動,人問之,答曰:「樹在道邊而多子,此必苦李。」取之,信然。}{\footnotesize \textbf{名士傳}曰戎由是幼有神理之稱也。}

\subsection*{5}

\textbf{魏明帝於宣武場上斷虎爪牙,縱百姓觀之,王戎七歲,亦往看,虎承間攀欄而吼,其聲震地,觀者無不辟易顛仆,戎湛然不動,了無恐色。}{\footnotesize \textbf{竹林七賢論}曰明帝自閣上望見,使人問戎姓名而異之。}

\subsection*{6}

\textbf{王戎為侍中,南郡太守劉肇遺筒中箋布五端,戎雖不受,厚報其書。}{\footnotesize \textbf{晉陽秋}曰司隸校尉劉毅奏「南郡太守劉肇以布五十疋雜物遺前豫州刺史王戎,請檻車徵付廷尉治罪,除名終身」,戎以書未達,不坐。\textbf{竹林七賢論}曰戎報肇書,議者僉以為譏,世祖患之,乃發口詔曰「以戎之為士,義豈懷私」,議者乃息,戎亦不謝。}

\subsection*{7}

\textbf{裴叔則被收,神氣無變,舉止自若,求紙筆作書,書成,救者多,乃得免,後位儀同三司。}{\footnotesize \textbf{晉諸公贊}曰楷息瓚,取楊駿女,駿誅,以楷婚黨,收付廷尉,侍中傅祗證楷素意,由此得免。\textbf{名士傳}曰楚王之難,李肇惡楷名重,收將害之,楷神色不變,舉動自若,諸人請救,得免。\textbf{晉陽秋}曰楷與王戎俱加儀同三司。}

\subsection*{8}

\textbf{王夷甫嘗屬族人事,經時未行,遇於一處飲燕,因語之曰:「近屬尊事,那得不行?」族人大怒,便舉樏擲其面,夷甫都無言,盥洗畢,牽王丞相臂,與共載去,在車中照鏡語丞相曰:「汝看我眼光,迺出牛背上。」}{\footnotesize 王夷甫蓋自謂風神英俊,不至與人校。}

\subsection*{9}

\textbf{裴遐在周馥所,馥設主人,}{\footnotesize \textbf{鄧粲晉紀}曰馥,字祖宣,汝南人,代劉淮為鎮東將軍,鎮壽陽,移檄四方,欲奉迎天子,元皇使甘卓攻之,馥出奔,道卒。}\textbf{遐與人圍棊,馥司馬行酒,遐正戲,不時為飲,司馬恚,因曳遐墜地,遐還坐,舉止如常,顏色不變,復戲如故,王夷甫問遐:「當時何得顏色不異?」答曰:「直是闇當故耳。」}{\footnotesize 一作闇故當耳,一作真是鬬將故耳。}

\subsection*{10}

\textbf{劉慶孫在太傅府,于時人士多為所構,唯庾子嵩縱心事外,無跡可間,後以其性儉家富,說太傅令換千萬,冀其有吝,於此可乘,}{\footnotesize \textbf{晉陽秋}曰劉輿,字慶孫,中山人,有豪俠才算,善交結,為范陽王虓所暱,虓薨,太傅召之,大相委仗,用為長史。\textbf{八王故事}曰司馬越,字元超,高密王泰長子,少尚布衣之操,為中外所歸,累遷司空、太傅。}\textbf{太傅於眾坐中問庾,庾時頽然已醉,幘墜几上,以頭就穿取,徐答云:「下官家故可有兩娑千萬,隨公所取。」於是乃服,後有人向庾道此,庾曰:「可謂以小人之慮,度君子之心。」}

\subsection*{11}

\textbf{王夷甫與裴景聲志好不同,景聲惡欲取之,卒不能回,乃故詣王,肆言極罵,要王答己,欲以分謗,王不為動色,徐曰:「白眼兒遂作。」}{\footnotesize \textbf{晉諸公贊}曰邈,字景聲,河東聞喜人,少有通才,從兄頠器賞之,毎與清言,終日達曙,自謂理構多知,輒毎謝之,然未能出也,歷太傅從事中郎、左司馬、監東海王軍事,少為文士,而經事為將,雖非其才,而以罕重稱也。}

\subsection*{12}

\textbf{王夷甫長裴成公四歲,不與相知,時共集一處,皆當時名士,謂王曰:「裴令令望何足計?」王便卿裴,裴曰:「自可全君雅志。」}{\footnotesize 裴頠已見。}

\subsection*{13}

\textbf{有往來者云,庾公有東下意,或謂王公:「可潛稍嚴,以備不虞。」王公曰:「我與元規雖俱王臣,本懷布衣之好,若其欲來,吾角巾徑還烏衣,}{\footnotesize \textbf{丹陽記}曰烏衣之起,吳時烏衣營處所也,江左初立,琅邪諸王所居。}\textbf{何所稍嚴?」}{\footnotesize \textbf{中興書}曰於是風塵自消,內外緝穆。}

\subsection*{14}

\textbf{王丞相主簿欲檢校帳下,公語主簿:「欲與主簿周旋,無為知人几案間事。」}

\subsection*{15}

\textbf{祖士少好財,阮遙集好屐,並恆自經營,同是一累,而未判其得失,}{\footnotesize \textbf{祖約別傳}曰約,字士少,范陽遒人,累遷平西將軍、豫州刺史,鎮壽陽,與蘇峻反,峻敗,約投石勒,約本幽州冠族,賓客填門,勒登高望見車騎,大驚,又使佔奪鄉里先人田地,地主多恨,勒惡之,遂誅約。\textbf{晉陽秋}曰阮孚,字遙集,陳留人,咸第二子也,少有智調,而無儁異,累遷侍中、吏部尚書、廣州刺史。}\textbf{人有詣祖,見料視財物,客至,屏當未盡,餘兩小簏著背後,傾身障之,意未能平,或有詣阮,見自吹火蠟屐,因歎曰:「未知一生當著幾量屐?」神色閑暢,於是勝負始分。}{\footnotesize \textbf{孚別傳}曰孚風韻疏誕,少有門風。}

\subsection*{16}

\textbf{許侍中、顧司空俱作丞相從事,爾時已被遇,遊宴集聚,略無不同,}{\footnotesize \textbf{晉百官名}曰許璪,字思文,義興陽羡人。\textbf{許氏譜}曰璪祖艶,字子良,永興長,父裴,字季顯,烏程令,璪仕至吏部侍郎。}\textbf{嘗夜至丞相許戲,二人歡極,丞相便命使入己帳眠,顧至曉回轉,不得快熟,許上牀便咍臺大鼾,丞相顧諸客曰:「此中亦難得眠處。」}{\footnotesize 顧和,字君孝,少知名,族人顧榮曰「此吾家騏驥也,必興吾宗」,仕至尚書令,五子,治、隗、淳、履、之。}

\subsection*{17}

\textbf{庾太尉風儀偉長,不輕舉止,時人皆以為假,亮有大兒數歲,雅重之質,便自如此,人知是天性,溫太真嘗隱幔怛之,此兒神色恬然,乃徐跪曰:「君侯何以為此?」論者謂不減亮,蘇峻時遇害。}{\footnotesize \textbf{庾氏譜}曰會,字會宗,太尉亮長子,年十九,咸和六年遇害。}\textbf{或云:「見阿恭,知元規非假。」}{\footnotesize 阿恭,會小字也。}

\subsection*{18}

\textbf{褚公於章安令遷太尉記室參軍,}{\footnotesize \textbf{按}庾亮啓參佐名,裒時直為參軍,不掌記室也。}\textbf{名字已顯而位微,人未多識,公東出,乘估客船,送故吏數人投錢唐亭住,}{\footnotesize \textbf{錢唐縣記}曰縣近海,為潮漂沒,縣諸豪姓,斂錢雇人,輦土為塘,因以為名也。}\textbf{爾時吳興沈充為縣令,}{\footnotesize 未詳。}\textbf{當送客過浙江,客出,亭吏驅公移牛屋下,潮水至,沈令起彷徨,問:「牛屋下是何物人?」吏云:「昨有一傖父來寄亭中,}{\footnotesize \textbf{晉陽秋}曰吳人以中州人為傖。}\textbf{有尊貴客,權移之。」令有酒色,因遙問:「傖父欲食䴵不?姓何等?可共語。」褚因舉手答曰:「河南褚季野。」遠近久承公名,令於是大遽,不敢移公,便於牛屋下修刺詣公,更宰殺為饌具,於公前鞭撻亭吏,欲以謝慙,公與之酌宴,言色無異,狀如不覺,令送公至界。}

\subsection*{19}

\textbf{郗太傅在京口,遣門生與王丞相書,求女壻,丞相語郗信:「君往東廂,任意選之。」門生歸,白郗曰:「王家諸郎亦皆可嘉,聞來覓壻,咸自矜持,惟有一郎,在東牀上坦腹臥,如不聞。」郗公云:「正此好。」訪之,乃是逸少,因嫁女與焉。}{\footnotesize \textbf{王氏譜}曰逸少,羲之小字,羲之妻,太傅郗鑒女,名璿,字子房。}

\subsection*{20}

\textbf{過江初,拜官,輿飾供饌,羊曼拜丹陽尹,客來蚤者,並得佳設,日晏漸罄,不復及精,隨客早晩,不問貴賤,}{\footnotesize \textbf{曼別傳}曰曼,字延祖,泰山南城人,父暨,陽平太守,曼頽縱宏任,飲酒誕節,與陳留阮放等號兗州八達,累遷丹陽尹,為蘇峻所害。}\textbf{羊固拜臨海,竟日皆美供,雖晩至,亦獲盛饌,時論以固之豐華,不如曼之真率。}{\footnotesize \textbf{明帝東宮僚屬名}曰固,字道安,太山人。\textbf{文字志}曰固父坦,車騎長史,固善草行,著名一時,避亂渡江,累遷黃門侍郎,褒其清儉,贈大鴻臚。}

\subsection*{21}

\textbf{周仲智飲酒醉,瞋目還面謂伯仁曰:「君才不如弟,而橫得重名。」須臾,舉蠟燭火擲伯仁,伯仁笑曰:「阿奴火攻,固出下策耳。」}{\footnotesize \textbf{孫子兵法}曰火攻有五,一曰火人,二曰火積,三曰火車,四曰火軍,五曰火隊,凡軍必知五火之變,故以火攻者,明也。}

\subsection*{22}

\textbf{顧和始為揚州從事,月旦當朝,未入頃,停車州門外,周侯詣丞相,歷和車邊,}{\footnotesize \textbf{語林}曰周侯飲酒已醉,著白祫,憑兩人來詣丞相。}\textbf{和覓蝨,夷然不動,周既過,反還,指顧心曰:「此中何所有?」顧搏蝨如故,徐應曰:「此中最是難測地。」周侯既入,語丞相曰:「卿州吏中有一令僕才。」}{\footnotesize \textbf{中興書}曰和有操量,弱冠知名。}

\subsection*{23}

\textbf{庾太尉與蘇峻戰,敗,率左右十餘人,乘小船西奔,}{\footnotesize \textbf{晉陽秋}曰蘇峻作逆,詔亮都督征討,戰於建陽門外,王師敗績,亮於陳攜二弟奔溫嶠。}\textbf{亂兵相剝掠,射誤中柂工,應弦而倒,舉船上咸失色分散,亮不動容,徐曰:「此手那可使著賊。」眾迺安。}

\subsection*{24}

\textbf{庾小征西嘗出未還,婦母阮是劉萬安妻,}{\footnotesize \textbf{劉氏譜}曰劉綏妻,陳留阮蕃女,字幼娥。綏別見。}\textbf{與女上安陵城樓上,俄頃翼歸,策良馬,盛輿衛,阮語女:「聞庾郎能騎,我何由得見?」婦告翼,}{\footnotesize \textbf{庾氏傳}曰翼娶高平劉綏女,字靜女。}\textbf{翼便為於道開鹵簿盤馬,始兩轉,墜馬墮地,意色自若。}

\subsection*{25}

\textbf{宣武}{\footnotesize 桓溫。}\textbf{與簡文、太宰}{\footnotesize 武陵王晞。}\textbf{共載,密令人在輿前後鳴鼓大叫,鹵簿中驚擾,太宰惶怖求下輿,顧看簡文,穆然清恬,宣武語人曰:「朝廷間故復有此賢。」}{\footnotesize \textbf{續晉陽秋}曰帝性溫深,雅有局鎮,嘗與桓溫、太宰武陵王晞同乘,至板橋,溫密敕令無因鳴角鼓譟,部伍並驚馳,溫陽駭異,晞大震,帝舉止自若,音顏無變,溫毎以此稱其德量,故論者謂溫服憚也。}

\subsection*{26}

\textbf{王劭、王薈共詣宣武,}{\footnotesize \textbf{劭、薈別傳}曰劭,字敬倫,丞相導第五子,清貴簡素,研味玄賾,大司馬桓溫稱為鳳鶵,累遷尚書僕射、吳國內史。薈,字敬文,丞相最小子,有清譽,夷泰無競,仕至鎮軍將軍。}\textbf{正值收庾希家,}{\footnotesize \textbf{中興書}曰希,字始彥,司空冰長子,累遷徐、兗二州刺史,希兄弟貴盛,桓溫忌之,諷免希官,遂奔於暨陽,初,郭璞筮冰子孫必有大禍,惟固三陽可以有後,故希求鎮山陽,弟友為東陽,希自家暨陽,及溫誅希弟柔、倩,聞希難,逃於海陵,後還京口聚眾,事敗,為溫所誅。}\textbf{薈不自安,逡巡欲去,劭堅坐不動,待收信還,得不定迺出,論者以劭為優。}

\subsection*{27}

\textbf{桓宣武與郗超議芟夷朝臣,條牒既定,其夜同宿,}{\footnotesize \textbf{續晉陽秋}曰超謂溫雄武,當樂推之運,遂深自委結,溫亦深相器重,故潛謀密計,莫不預焉。}\textbf{明晨起,呼謝安、王坦之入,擲疏示之,郗猶在帳內,謝都無言,王直擲還,云「多」,宣武取筆欲除,郗不覺竊從帳中與宣武言,謝含笑曰:「郗生可謂入幕賓也。」}{\footnotesize 帳,一作帷。}

\subsection*{28}

\textbf{謝太傅盤桓東山時,與孫興公諸人汎海戲,}{\footnotesize \textbf{中興書}曰安先居會稽,與支道林、王羲之、許詢共遊處,出則漁弋山水,入則談說屬文,未嘗有處世意也。}\textbf{風起浪涌,孫、王諸人色並遽,便唱使還,太傅神情方王,吟嘯不言,舟人以公貌閑意說,猶去不止,既風轉急,浪猛,諸人皆諠動不坐,公徐云:「如此,將無歸。」眾人即承響而回,於是審其量,足以鎮安朝野。}

\subsection*{29}

\textbf{桓公伏甲設饌,廣延朝士,因此欲誅謝安、王坦之,}{\footnotesize \textbf{晉安帝紀}曰簡文晏駕,遺詔桓溫依諸葛亮、王導故事,溫大怒,以為黜其權,謝安、王坦之所建也,入赴山陵,百官拜於道側,在位望者,戰慄失色。或云自此欲殺王、謝。}\textbf{王甚遽,問謝曰:「當作何計?」謝神意不變,謂文度曰:「晉阼存亡,在此一行。」相與俱前,王之恐狀,轉見於色,謝之寬容,愈表於貌,望階趨席,方作洛生詠,諷「浩浩洪流」,桓憚其曠遠,乃趣解兵,}{\footnotesize \textbf{按}宋明帝文章志曰「安能作洛下書生詠,而少有鼻疾,語音濁,後名流多斅其詠,弗能及,手掩鼻而吟焉,桓溫止新亭,大陳兵衛,呼安及坦之,欲於坐害之,王入失措,倒執手版,汗流霑衣,安神姿舉動不異於常,舉目徧歷溫左右衛士,謂溫曰『安聞諸侯有道,守在四鄰,明公何須壁間著阿堵輩』,溫笑曰『正自不能不爾』,於是矜莊之心頓盡,命卻左右,促燕行觴,笑語移日」。}\textbf{王、謝舊齊名,於此始判優劣。}

\subsection*{30}

\textbf{謝太傅與王文度共詣郗超,日旰未得前,王便欲去,謝曰:「不能為性命忍俄頃?」}{\footnotesize 超得寵桓溫,專殺生之威。}

\subsection*{31}

\textbf{支道林還東,}{\footnotesize \textbf{高逸沙門傳}曰遁為哀帝所迎,遊京邑久,心在故山,乃拂衣王都,還就巖穴。}\textbf{時賢並送於征虜亭,}{\footnotesize \textbf{丹陽記}曰太安中,征虜將軍謝安立此亭,因以為名。}\textbf{蔡子叔前至,坐近林公,}{\footnotesize \textbf{中興書}曰蔡糸,字子叔,濟陽人,司徒謨第二子,有文理,仕至撫軍長史。}\textbf{謝萬石後來,坐小遠,蔡暫起,謝移就其處,蔡還,見謝在焉,因合褥舉謝擲地,自復坐,謝冠幘傾脫,乃徐起振衣就席,神意甚平,不覺瞋沮,坐定,謂蔡曰:「卿奇人,殆壞我面。」蔡答曰:「我本不為卿面作計。」其後二人俱不介意。}

\subsection*{32}

\textbf{郗嘉賓欽崇釋道安德問,}{\footnotesize \textbf{安和上傳}曰釋道安者,常山薄柳人,本姓衛,年十二作沙門,神性聰敏而貌至陋,佛圖澄甚重之,值石氏亂,於陸渾山木食修學,為慕容俊所逼,乃住襄陽,以佛法東流,經籍錯謬,更為條章,標序篇目,為之注解,自支道林等皆宗其理,無疾卒。}\textbf{餉米千斛,修書累紙,意寄殷勤,道安答直云:「損米。」愈覺有待之為煩。}

\subsection*{33}

\textbf{謝安南免吏部尚書還東,}{\footnotesize \textbf{晉百官名}曰謝奉,字弘道,會稽山陰人。\textbf{謝氏譜}曰奉祖端,散騎常侍,父鳳,丞相主簿,奉歷安南將軍、廣州刺史、吏部尚書。}\textbf{謝太傅赴桓公司馬出西,相遇破岡,既當遠別,遂停三日共語,太傅欲慰其失官,安南輒引以它端,雖信宿中塗,竟不言及此事,太傅深恨在心未盡,謂同舟曰:「謝奉故是奇士。」}

\subsection*{34}

\textbf{戴公從東出,謝太傅往看之,謝本輕戴,見但與論琴書,戴既無吝色,而談琴書愈妙,謝悠然知其量。}{\footnotesize \textbf{晉安帝紀}曰戴逵,字安道,譙國人,少有清操,恬和通任,為劉真長所知,性甚快暢,泰於娛生,好鼓琴,善屬文,尤樂遊燕,多與高門風流者游,談者許其通隱,屢辭徵命,遂著高尚之稱。}

\subsection*{35}

\textbf{謝公與人圍棊,俄而謝玄淮上信至,看書竟,默然無言,徐向局,客問淮上利害,答曰:「小兒輩大破賊。」意色舉止,不異於常。}{\footnotesize \textbf{續晉陽秋}曰初,苻堅南寇,京師大震,謝安無懼色,方命駕出墅,與兄子玄圍棊,夜還乃處分,少日皆辦,破賊又無喜容,其高量如此。\textbf{謝車騎傳}曰氐賊苻堅傾國大出,眾號百萬,朝廷遣諸軍距之,凡八萬,堅進屯壽陽,玄為前鋒都督,與從弟琰等選精鋭決戰,射傷堅,俘獲數萬計,得偽輦及雲母車,寶器山積,錦罽萬端,牛、馬、驢、騾、駝十萬頭。}

\subsection*{36}

\textbf{王子猷、子敬曾俱坐一室,上忽發火,子猷遽走避,不惶取屐,}{\footnotesize \textbf{晉百官名}曰王徽之,字子猷。\textbf{中興書}曰徽之,羲之第五子,卓犖不羈,欲為傲達,仕至黃門侍郎。}\textbf{子敬神色恬然,徐喚左右,扶憑而出,不異平常,}{\footnotesize \textbf{續晉陽秋}曰獻之雖不脩常貫,而容止不妄。}\textbf{世以此定二王神宇。}

\subsection*{37}

\textbf{苻堅遊魂近境,}{\footnotesize 堅別見。}\textbf{謝太傅謂子敬曰:「可將當軸,了其此處。」}

\subsection*{38}

\textbf{王僧彌、謝車騎共王小奴許集,}{\footnotesize 王珉、謝玄並已見。小奴,王薈小字也。}\textbf{僧彌舉酒勸謝云:「奉使君一觴。」謝曰:「可爾。」}{\footnotesize 謝玄曾為徐州,故云使君。}\textbf{僧彌勃然起,作色曰:「汝故是吳興溪中釣碣耳,何敢譸張?」}{\footnotesize 玄叔父安曾為吳興,玄少時從之遊,故珉云然。}\textbf{謝徐撫掌而笑曰:「衛軍,僧彌殊不肅省,乃侵陵上國也。」}

\subsection*{39}

\textbf{王東亭為桓宣武主簿,既承藉,有美譽,公甚欲其人地為一府之望,初,見謝失儀,而神色自若,坐上賓客即相貶笑,公曰:「不然,觀其情貌,必自不凡,吾當試之。」後因月朝閣下伏,公於內走馬直出突之,左右皆宕仆,而王不動,名價於是大重,咸云是公輔器也。}{\footnotesize \textbf{續晉陽秋}曰珣初辟大司馬掾,桓溫至重之,常稱「王掾必為黑頭公,未易才也」。}

\subsection*{40}

\textbf{太元末,長星見,孝武心甚惡之,}{\footnotesize \textbf{徐廣晉紀}曰泰元二十年九月,有蓬星如粉絮,東南行,歷須女,至哭星。\textbf{按}泰元末,惟有此妖,不聞長星也,且漢文八年,有長星出東方,文穎注曰「長星有光芒,或竟天,或長十丈,或二三丈,無常也」,此星見,多為兵革事,此後十六年,文帝乃崩,蓋知長星非關天子,世說虛也。}\textbf{夜,華林園中飲酒,舉桮屬星云:「長星,勸爾一桮酒,自古何時有萬歲天子?」}

\subsection*{41}

\textbf{殷荊州有所識,作賦,是束皙慢戲之流,}{\footnotesize \textbf{文士傳}曰晳,字廣微,陽平元城人,漢太子太傅踈廣後也,王莽末,廣曾孫孟達自東海避難元城,改姓,去踈之足以為束氏,晳博學多識,問無不對,元康中,有人自嵩高山下得竹簡一枚,上兩行科斗書,司空張華以問晳,晳曰「此明帝顯節陵中策文也」,檢校果然,曾為䴵賦諸文,文甚俳謔,三十九歲卒,元城為之廢市。}\textbf{殷甚以為有才,語王恭:「適見新文,甚可觀。」便於手巾函中出之,王讀,殷笑之不自勝,王看竟,既不笑,亦不言好惡,但以如意帖之而已,殷悵然自失。}

\subsection*{42}

\textbf{羊綏第二子孚,少有儁才,與謝益壽相好,}{\footnotesize 益壽,謝混小字也。}\textbf{嘗蚤往謝許,未食,俄而王齊、王睹來,}{\footnotesize 王睹已見。齊,王熙小字也。\textbf{中興書}曰熙,字叔和,恭次弟,尚鄱陽公主,太子洗馬,早卒。}\textbf{既先不相識,王向席有不說色,欲使羊去,羊了不眄,惟腳委几上,詠矚自若,謝與王敘寒溫數語畢,還與羊談賞,王方悟其奇,乃合共語,須臾食下,二王都不得餐,惟屬羊不暇,羊不大應對之,而盛進食,食畢便退,遂苦相留,羊義不住,直云:「向者不得從命,中國尚虛。」二王是孝伯兩弟。}