\chapter{尤悔第三十三}

\subsection*{1}

\textbf{魏文帝忌弟任城王驍壯,因在卞太后閤共圍棊,並噉棗,文帝以毒置諸棗蔕中,自選可食者而進,王弗悟,遂雜進之,既中毒,太后索水救之,帝預敕左右毀缾罐,太后徒跣趨井,無以汲,須臾,遂卒,}{\footnotesize \textbf{魏略}曰任城威王彰,字子文,太祖卞太后弟二子,性剛勇而黃鬚,北討代郡,獨與麾下百餘人突虜而走,太祖聞曰「我黃鬚兒可用也」。\textbf{魏志春秋}曰黃初三年,彰來朝,初,彰問璽綬,將有異志,故來朝不即得見,有此忿懼而暴薨。}\textbf{復欲害東阿,太后曰:「汝已殺我任城,不得復殺我東阿。」}{\footnotesize \textbf{魏志方伎傳}曰文帝問占夢周宣「吾夢磨錢文,欲滅而愈更明,何謂」,宣悵然不對,帝固問之,宣曰「陛下家事,雖欲爾,而太后不聽,是以欲滅更明耳」,帝欲治弟植之罪,逼於太后,但加貶爵。}

\subsection*{2}

\textbf{王渾後妻,琅邪顏氏女,王時為徐州刺史,交禮拜訖,王將答拜,觀者咸曰:「王侯州將,新婦州民,恐無由答拜。」王乃止,武子以其父不答拜,不成禮,恐非夫婦,不為之拜,謂為顏妾,顏氏恥之,以其門貴,終不敢離。}{\footnotesize 婚姻之禮,人道之大,豈由一不拜而遂為妾媵者乎?世說之言,於是乎紕繆。}

\subsection*{3}

\textbf{陸平原河橋敗,為盧志所讒,被誅,}{\footnotesize \textbf{王隱晉書}曰成都王穎討長沙王乂,使陸為都督前鋒諸軍事。\textbf{機別傳}曰成都王長史盧志,與機弟雲趣舍不同,又黃門孟玖求為邯鄲令於穎,穎教付雲,雲時為左司馬,曰「刑餘之人,不可以君民」,玖聞此怨雲,與志讒構日至,及機於七里澗大敗,玖誣機謀反所致,穎乃使牽秀斬機,先是,夕夢黑幔繞車,手決不開,惡之,明旦,秀兵奄至,機解戎服,著衣幍見秀,容貌自若,遂見害,時年四十三,軍士莫不流涕,是日天地霧合,大風折木,平地尺雪。\textbf{干寶晉紀}曰初,陸抗誅步闡,百口皆盡,有識尤之,及機、雲見害,三族無遺。}\textbf{臨刑歎曰:「欲聞華亭鶴唳,可復得乎?」}{\footnotesize \textbf{八王故事}曰華亭,吳由拳縣郊外墅也,有清泉茂林,吳平後,陸機兄弟共游於此十餘年。\textbf{語林}曰機為河北都督,聞警角之聲,謂孫丞曰「聞此不如華亭鶴唳」,故臨刑而有此歎。}

\subsection*{4}

\textbf{劉琨善能招延,而拙於撫御,一日雖有數千人歸投,其逃散而去亦復如此,所以卒無所建。}{\footnotesize \textbf{鄧粲晉紀}曰琨為并州牧,糺合齊盟,驅率戎旅,而內不撫其民,遂至喪軍失士,無成功也。\textbf{敬胤}按琨以永嘉元年為并州,于時晉陽空城,寇盜四攻,而能收合士眾,抗行淵、勒,十年之中,敗而能振,不能撫御,其得如此乎?凶荒之日,千里無煙,豈一日有數千人歸之?若一日數千人去之,又安得一紀之間以對大難乎?}

\subsection*{5}

\textbf{王平子始下,丞相語大將軍:「不可復使羌人東行。」平子面似羌。}{\footnotesize \textbf{按}王澄自為王敦所害,丞相名德,豈應有斯言也。}

\subsection*{6}

\textbf{王大將軍起事,丞相兄弟詣闕謝,周侯深憂諸王,始入,甚有憂色,丞相呼周侯曰:「百口委卿。」周直過不應,既入,苦相存救,既釋,周大說,飲酒,及出,諸王故在門,周曰:「今年殺諸賊奴,當取金印如斗大繫肘後。」大將軍至石頭,問丞相曰:「周侯可為三公不?」丞相不答,又問:「可為尚書令不?」又不應,因云:「如此,唯當殺之耳。」復默然,逮周侯被害,丞相後知周侯救己,歎曰:「我不殺周侯,周侯由我而死,幽冥中負此人。」}{\footnotesize \textbf{虞預晉書}曰敦克京邑,參軍呂漪說敦曰「周顗、戴淵,皆有名望,足以惑眾,視近日之言,無慙懼之色,若不除之,役將未歇也」,敦即然之,遂害淵、顗,初,漪為臺郎,淵既上官,素有高氣,以漪小器待之,故售其說焉。}

\subsection*{7}

\textbf{王導、溫嶠俱見明帝,帝問溫前世所以得天下之由,溫未答,頃,王曰:「溫嶠年少未諳,臣為陛下陳之。」王迺具敘宣王創業之始,誅夷名族,寵樹同己,及文王之末,高貴鄉公事,}{\footnotesize 宣王創業,誅曹爽、任蔣濟之流者是也。高貴鄉公之事已見上。}\textbf{明帝聞之,覆面著牀曰:「若如公言,祚安得長?」}

\subsection*{8}

\textbf{王大將軍於眾坐中曰:「諸周由來未有作三公者。」有人答曰:「唯周侯邑五馬領頭而不克。」大將軍曰:「我與周洛下相遇,一面頓盡,值世紛紜,遂至於此。」因為流涕。}{\footnotesize \textbf{鄧粲晉紀}曰王敦參軍有於敦坐樗蒱,臨當成者,馬頭被殺,因謂曰「周家奕世令望,而位不至三公,伯仁垂作而不果,有似下官此馬」,敦慨然流涕曰「伯仁總角時,與於東宮相遇,一面披衿,便許之三司,何圖不幸,王法所裁,悽愴之深,言何能盡」。}

\subsection*{9}

\textbf{溫公初受劉司空使勸進,母崔氏固駐之,嶠絕裾而去,}{\footnotesize \textbf{溫氏譜}曰嶠父襜,娶清河崔參女。}\textbf{迄於崇貴,鄉品猶不過也,每爵皆發詔。}{\footnotesize \textbf{虞預晉書}曰元帝即位,以溫嶠為散騎侍郎,嶠以母亡,逼賊,不得往臨葬,固辭,詔曰「嶠以未葬,朝議又頗有異同,故不拜,其令人坐議,吾將折其衷」。}

\subsection*{10}

\textbf{庾公欲起周子南,子南執辭愈固,庾每詣周,庾從南門入,周從後門出,庾嘗一往奄至,周不及去,相對終日,庾從周索食,周出蔬食,庾亦彊飯,極歡,并語世故,約相推引,同佐世之任,既仕,至將軍二千石,}{\footnotesize \textbf{尋陽記}曰周邵,字子南,與南陽翟湯隱於尋陽廬山,庾亮臨江州,聞翟、周之風,束帶躡履而詣焉,聞庾至,轉避之,亮後密往,值邵彈鳥於林,因前與語,還,便云「此人可起」,即拔為鎮蠻護軍、西陽太守。其集載\textbf{與邵書}曰西陽一郡,戶口差實,非履道真純,何以鎮其流遁?詢之朝野,僉曰足下,今具上表,請足下臨之無讓。}\textbf{而不稱意,中宵慨然曰:「大丈夫乃為庾元規所賣。」一歎,遂發背而卒。}

\subsection*{11}

\textbf{阮思曠奉大法,敬信甚至,大兒年未弱冠,忽被篤疾,}{\footnotesize \textbf{阮氏譜}曰牖,字彥倫,裕長子也,仕至州主簿。}\textbf{兒既是偏所愛重,為之祈請三寶,晝夜不懈,謂至誠有感者,必當蒙祐,而兒遂不濟,於是結恨釋氏,宿命都除。}{\footnotesize 以阮公智識,必無此弊,脫此非謬,何其惑歟?夫文王期盡,聖子不能駐其年,釋種誅夷,神力無以延其命,故業有定限,報不可移,若請禱而望其靈,匪驗而忽其道,固陋之徒耳,豈可與言神明之智者哉?}

\subsection*{12}

\textbf{桓宣武對簡文帝,不甚得語,廢海西後,宜自申敘,乃豫撰數百語,陳廢立之意,既見簡文,簡文便泣下數十行,宣武矜愧,不得一言。}

\subsection*{13}

\textbf{桓公臥語曰:「作此寂寂,將為文、景所笑。」既而屈起坐曰:「既不能流芳後世,亦不足復遺臭萬載邪?」}{\footnotesize \textbf{續晉陽秋}曰桓溫既以雄武專朝,任兼將相,其不臣之心形于音迹,曾臥對親僚,撫枕而起曰「為爾寂寂,為文、景所笑」,眾莫敢對。}

\subsection*{14}

\textbf{謝太傅於東船行,小人引船,或遲或速,或停或待,又放船從橫,撞人觸岸,公初不呵譴,人謂公常無嗔喜,曾送兄征西葬還,}{\footnotesize 征西,謝奕。}\textbf{日莫雨,駛小人皆醉,不可處分,公乃於車中,手取車柱撞馭人,聲色甚厲,夫以水性沈柔,入隘奔激,方之人情,固知迫隘之地,無得保其夷粹。}{\footnotesize \textbf{孟子}曰湍水,決之東則東,決之西則西,搏而躍之,可使過顙,激而行之,可使在山,豈水之性哉?人可使為不善,性亦猶是也。}

\subsection*{15}

\textbf{簡文見田稻不識,問是何草,左右答是稻,簡文還,三日不出,云:「寧有賴其末,而不識其本?」}{\footnotesize 文公種菜,曾子牧羊,縱不識稻,何所多悔?此言必虛。}

\subsection*{16}

\textbf{桓車騎在上明畋獵,東信至,傳淮上大捷,語左右云:「群謝年少,大破賊。」因發病薨,談者以為此死,賢於讓揚之荊。}{\footnotesize \textbf{續晉陽秋}曰桓沖本以將相異宜,才用不同,忖己德量,不及謝安,故解揚州以讓安,自謂少經軍鎮,及為荊州,聞苻堅自出淮、肥,深以根本為慮,遣其隨身精兵三千人赴京師,時安已遣諸軍,且欲外示閒暇,因令沖軍還,沖大驚曰「謝安乃有廟堂之量,不閑將略,吾量賊必破襄陽,而并力淮、肥,今大敵果至,方遊談示暇,遣諸不經事年少,而實寡弱,天下誰知?吾其左衽矣」,俄聞大勳克舉,慚慨而薨。}

\subsection*{17}

\textbf{桓公初報破殷荊州,}{\footnotesize \textbf{周祗隆安記}曰仲堪以人情注於玄,疑朝廷欲以玄代己,遣道人竺僧愆齎寶物遺相王寵幸媒尼左右,以罪狀玄,玄知其謀而擊滅之。}\textbf{曾講論語,至「富與貴,是人之所欲,不以其道得之,不處」,}{\footnotesize \textbf{孔安國}注曰不以其道得富貴,則仁者不處。}\textbf{玄意色甚惡。}