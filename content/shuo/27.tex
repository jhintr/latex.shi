\chapter{假譎第二十七}

\subsection*{1}

\textbf{魏武少時,嘗與袁紹好為游俠,觀人新婚,因潛入主人園中,夜叫呼云:「有偷兒賊。」青廬中人皆出觀,魏武乃入,抽刃劫新婦與紹還出,失道,墜枳棘中,紹不能得動,復大叫云:「偷兒在此。」紹遑迫自擲出,遂以俱免。}{\footnotesize \textbf{曹瞞傳}曰操,小字阿瞞,少好譎詐,遊放無度。\textbf{孫盛雜語}云武王少好俠,放蕩不修行業,嘗私入常侍張讓宅中,讓乃手戟於庭,踰垣而出,有絕人力,故莫之能害也。}

\subsection*{2}

\textbf{魏武行役,失汲道,軍皆渴,乃令曰:「前有大梅林,饒子,甘酸,可以解渴。」士卒聞之,口皆出水,乘此得及前源。}

\subsection*{3}

\textbf{魏武常言:「人欲危己,己輒心動。」因語所親小人曰:「汝懷刃密來我側,我必說心動,執汝使行刑,汝但勿言其使,無他,當厚相報。」執者信焉,不以為懼,遂斬之,此人至死不知也,左右以為實,謀逆者挫氣矣。}{\footnotesize \textbf{曹瞞傳}曰操在軍,廩穀不足,私語主者曰「何如」,主者云「可以小斛足之」,操曰「善」,後軍中言操欺眾,操題其主者背以徇曰「行小斛,盜軍穀」,遂斬之,仍云「特當借汝死以厭眾心」,其變詐皆此類也。}

\subsection*{4}

\textbf{魏武常云:「我眠中不可妄近,近便斫人,亦不自覺,左右宜深慎此。」後陽眠,所幸一人竊以被覆之,因便斫殺,自爾每眠,左右莫敢近者。}

\subsection*{5}

\textbf{袁紹年少時,曾遣人夜以劍擲魏武,少下,不著,魏武揆之,其後來必高,因帖臥牀上,劍至果高。}{\footnotesize \textbf{按}袁、曹後由鼎跱,迹始攜貳,自斯以前,不聞讐隟,有何意故而剚之以劍也。}

\subsection*{6}

\textbf{王大將軍既為逆,頓軍姑孰,晉明帝以英武之才,猶相猜憚,乃著戎服,騎巴賨馬,齎一金馬鞭,陰察軍形勢,未至十餘里,有一客姥,居店賣食,帝過愒之,謂姥曰:「王敦舉兵圖逆,猜害忠良,朝廷駭懼,社稷是憂,故劬勞晨夕,用相覘察,恐形迹危露,或致狼狽,追迫之日,姥其匿之。」便與客姥馬鞭而去,行敦營帀而出,軍士覺,曰:「此非常人也。」敦臥心動,曰:「此必黃鬚鮮卑奴來。」命騎追之,已覺多許里,追士因問向姥:「不見一黃鬚人騎馬度此邪?」姥曰:「去已久矣,不可復及。」於是騎人息意而反。}{\footnotesize \textbf{異苑}曰帝躬往姑孰,敦時晝寢,卓然驚悟曰「營中有黃頭鮮卑奴來,何不縛取」,帝所生母荀氏,燕國人,故貌類焉。}

\subsection*{7}

\textbf{王右軍年裁十歲時,大將軍甚愛之,恆置帳中眠,大將軍嘗先出,右軍猶未起,須臾,錢鳳入,屏人論事,}{\footnotesize \textbf{晉陽秋}曰鳳,字世儀,吳嘉興尉子也,姦慝好利,為敦鎧曹參軍,知敦有不臣心,因進說,後敦敗,見誅。}\textbf{都忘右軍在帳中,便言逆節之謀,右軍覺,既聞所論,知無活理,乃陽吐汙頭面被褥,詐熟眠,敦論事造半,方憶右軍未起,相與大驚曰:「不得不除之。」及開帳,乃見吐唾從橫,信其實熟眠,於是得全,于時稱其有智。}{\footnotesize \textbf{按}諸書皆云王允之事,而此言羲之,疑謬。}

\subsection*{8}

\textbf{陶公自上流來,赴蘇峻之難,令誅庾公,謂必戮庾,可以謝峻,}{\footnotesize \textbf{晉陽秋}曰是時成帝在襁褓,太后臨朝,中書令庾亮以元舅輔政,欲以風軌格政,繩御四海,而峻擁兵近甸,為逋逃藪,亮圖召峻,王導、卞壺並不欲,亮曰「蘇峻豺狼,終為禍亂,晁錯所謂削亦反,不削亦反」,遂下優詔,以大司農徵之,峻怒曰「庾亮欲誘殺我也」,遂克京邑,平南溫嶠聞亂,號泣登舟,遣參軍王愆期推征西陶侃為盟主,俱赴京師,時亮敗績奔嶠,人皆尤而少之,嶠愈相崇重,分兵以配給之。}\textbf{庾欲奔竄,則不可,欲會,恐見執,進退無計,溫公勸庾詣陶,曰:「卿但遙拜,必無它,我為卿保之。」庾從溫言詣陶,至便拜,陶自起止之,曰:「庾元規何緣拜陶士衡?」畢,又降就下坐,陶又自要起同坐,坐定,庾乃引咎責躬,深相遜謝,陶不覺釋然。}

\subsection*{9}

\textbf{溫公喪婦,從姑劉氏,家值亂離散,唯有一女,甚有姿慧,姑以屬公覓婚,公密有自婚意,答云:「佳壻難得,但如嶠比云何?」姑云:「喪敗之餘,乞粗存活,便足慰吾餘年,何敢希汝比?」卻後少日,公報姑云:「已覓得婚處,門地粗可,壻身名宦,盡不減嶠。」因下玉鏡臺一枚,姑大喜,既婚,交禮,女以手披紗扇,撫掌大笑曰:「我固疑是老奴,果如所卜。」}{\footnotesize \textbf{按}溫氏譜「嶠初取高平李暅女,中取琅琊王詡女,後取廬江何邃女」,都不聞取劉氏,便為虛謬。\textbf{谷口}云劉氏,政謂其姑爾,非指其女姓劉也,孝標之注,亦未為得。}\textbf{玉鏡臺,是公為劉越石長史,北征劉聰所得。}{\footnotesize \textbf{王隱晉書}曰建興二年,嶠為劉琨假守左司馬,都督上前鋒諸軍事,討劉聰。\textbf{晉陽秋}曰聰,一名載,字玄明,屠各人,父淵,因亂起兵,死,聰嗣業。}

\subsection*{10}

\textbf{諸葛令女,庾氏婦,既寡,誓云不復重出,此女性甚正彊,無有登車理,}{\footnotesize 即庾亮子會妻文彪,已見上。}\textbf{恢既許江思玄婚,乃移家近之,初,誑女云「宜徙於是」,家人一時去,獨留女在後,比其覺,已不復得出,江郎莫來,女哭詈彌甚,積日漸歇,江虨暝入宿,恆在對牀上,後觀其意轉帖,虨乃詐厭,良久不悟,聲氣轉急,女乃呼婢云:「喚江郎覺。」江於是躍來就之曰:「我自是天下男子,厭,何預卿事而見喚邪?既爾相關,不得不與人語。」女默然而慙,情義遂篤。}{\footnotesize 葛令之清英,江君之茂識,必不背聖人之正典,習蠻夷之穢行,康王之言,所輕多矣。}

\subsection*{11}

\textbf{愍度道人始欲過江,與一傖道人為侶,謀曰:「用舊義往江東,恐不辦得食。」便共立「心無義」,既而此道人不成渡,愍度果講義積年,}{\footnotesize \textbf{名德沙門題目}曰支愍度才鑒清出。\textbf{孫綽愍度贊}曰支度彬彬,好是拔新,俱稟昭見,而能越人,世重秀異,咸競爾珍,孤桐嶧陽,浮磬泗濱。}\textbf{後有傖人來,先道人寄語云:「為我致意愍度,無義那可立?}{\footnotesize 舊義者曰「種智有是,而能圓照,然則萬累斯盡,謂之空無,常住不變,謂之妙有」,而無義者曰「種智之體,豁如太虛,虛而能知,無而能應,居宗至極,其唯無乎」。}\textbf{治此計,權救饑爾,無為遂負如來也。」}

\subsection*{12}

\textbf{王文度弟阿智,惡乃不翅,當年長而無人與婚,孫興公有一女,亦僻錯,又無嫁娶理,因詣文度,求見阿智,既見,便陽言:「此定可,殊不如人所傳,那得至今未有婚處?我有一女,乃不惡,但吾寒士,不宜與卿計,欲令阿智娶之。」文度欣然而啓藍田云:「興公向來,忽言欲與阿智婚。」藍田驚喜,既成婚,女之頑嚚,欲過阿智,方知興公之詐。}{\footnotesize 阿智,王虔之小字。虔之,字文將,辟州別駕,不就,娶太原孫綽女,字阿恆。}

\subsection*{13}

\textbf{范玄平為人,好用智數,而有時以多數失會,嘗失官居東陽,桓大司馬在南州,故往投之,桓時方欲招起屈滯,以傾朝廷,且玄平在京,素亦有譽,桓謂遠來投己,喜躍非常,比入至庭,傾身引望,語笑歡甚,顧謂袁虎曰:「范公且可作太常卿。」范裁坐,桓便謝其遠來意,范雖實投桓,而恐以趨時損名,乃曰:「雖懷朝宗,會有亡兒瘞在此,故來省視。」桓悵然失望,向之虛佇,一時都盡。}{\footnotesize \textbf{中興書}曰初,桓溫請范汪為征西長史,復表為江州,並不就,還都,因求為東陽太守,溫甚恨之,汪後為徐州,溫北伐,令汪出梁國,失期,溫挾憾奏汪為庶人,汪居吳,後至姑孰見溫,溫語其下曰「玄平乃來見,當以護軍起之」,汪數日辭歸,溫曰「卿適來,何以便去」,汪曰「數歲小兒喪,往年經亂,權瘞此境,故來迎之,事竟去耳」,溫愈怒之,竟不屑意。}

\subsection*{14}

\textbf{謝遏年少時,好著紫羅香囊,垂覆手,太傅患之,而不欲傷其意,乃譎與賭,得即燒之。}{\footnotesize 遏,謝玄小字。}