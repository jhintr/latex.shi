\section{牟尼经}

\begin{center}Muni Sutta\end{center}\vspace{1em}

\subsection\*{\textbf{209} {\footnotesize 〔PTS 207〕}}

\textbf{从亲密生出怖畏,从居所生出尘垢,\\}
\textbf{无居所,无亲密,这才是牟尼知见。}

Santhavāto bhayaṃ jātaṃ, niketā jāyate rajo;\\
aniketam asanthavaṃ, etaṃ ve munidassanaṃ. %\hfill\textcolor{gray}{\footnotesize 1}

\begin{enumerate}\item 缘起为何?全经不止一个缘起,且于此是这头四颂的缘起。据说,当世尊住舍卫国时,村落中某个苦命的女人,死了丈夫,将儿子出家为比丘后,自己也出家为比丘尼。他二人都在舍卫国度安居,便经常想见彼此。母亲得了什么,就带给儿子,儿子对母亲也是。如是早晚聚会,分享所得,彼此问候,嘘寒问暖,便无疑虑。因二人如是经常的相见,便生起交际,从交际而亲厚,从亲厚而陷落,而对陷入贪染的心,出家想与母子想便淡却了。随后,便越过界限,从事于不善法,且由不名誉而还俗,住于俗家间。
\item 诸比丘便告知了世尊。世尊在以\begin{quoting}诸比丘!那愚人在想什么:没有母亲会执著儿子,或儿子之于母亲?\end{quoting}责备后,又以「诸比丘!我不见另有一色……」等余下的经\footnote{自缘起之初至此的一段,见\textbf{增支部}第 5:55 经。}警示诸比丘,在说罢\begin{quoting}因此,诸比丘!——\\一如猛烈的毒药,一如沸腾的油,\\如熔化的铜,应当回避女人!\end{quoting}后,又为给诸比丘开示法,说了这指涉自身的四颂。
\item 这里,\textbf{亲密},即先前以爱、见、朋友等所说的三种\footnote{三种亲密,见\textbf{犀牛角经}第 37 颂义注。},于此是指爱、见的亲密,世尊据此而说:「看!诸比丘!正如这愚人,从亲密生出怖畏。」因为他从想要经常见面等生出强烈烦恼之怖畏,因之不能安立,而对母亲行邪,或因对自身非难的大怖畏,弃置教法而还俗。\textbf{居所},即以\begin{quoting}流连、束缚于色相的居所,长者!被称为「流于居所」。(相应部第 22:3 经)\end{quoting}等方法所说的各类所缘。\textbf{生出尘垢},即生出贪嗔痴的尘垢。这说的是什么?他不仅从亲密生出怖畏,而且现在,更由这以烦恼住处之义被称为「居所」的有漏所缘的防护被破坏、界限被越过,由此从居所生出尘垢,杂染之心以此将遭厄难。或者,当如是连结这两句:看!诸比丘!正如这愚人,从亲密生出怖畏,又正如一切凡夫,从居所生出尘垢。
\item 而于一切处,世尊在以前半句谴责了凡夫的知见后,为赞叹自己的知见,便说后半句。这里,当知以拒斥如上所说的居所为\textbf{无居所},并以拒斥亲密为\textbf{无亲密},这两者都是涅槃的同义语。\textbf{这才是牟尼知见},即这无居所、无亲密为佛牟尼所见之义。这里,「才」当视为惊叹之义的不变词。且因此得成此义:正是在母子因居所、亲密而行邪中的无居所、无亲密,是为牟尼所见,哎!希有!或者,牟尼之知见为牟尼知见,而知见即认可、好乐,他认可且好乐之义。\end{enumerate}

\subsection\*{\textbf{210} {\footnotesize 〔PTS 208〕}}

\textbf{若切断了已生者,不再培植正生者,不再滋益它,\\}
\textbf{他们说他是独行的牟尼,这大仙得见寂静的境地。}

Yo jātam ucchijja na ropayeyya, jāyantam assa nānuppavecche;\\
tam āhu ekaṃ muninaṃ carantaṃ, addakkhi so santipadaṃ mahesi. %\hfill\textcolor{gray}{\footnotesize 2}

\begin{enumerate}\item 在第二颂中,\textbf{若切断了已生者},即他对在任何依处中已生、已成、已发生的烦恼,正如舍断已生起的不善,如是精进,以于此依处不再发生而切断之。而未来的烦恼,由在如是样类的缘汇集时发生而成为现前,在现在的临近,以现在为相,而被称为「正生者」,\textbf{不再培植}此\textbf{正生者},正如不生起未生的不善,如是精进,令不发生之义。且如何令不发生?\textbf{不再滋益它},对以之令生的缘,不再滋益,不令汇集,如是以资粮的欠缺,便不能培植此正生者。
\item 或者,因为道的修习,过去的烦恼以无有将来的异熟而被切断,现在的以无此而不被培植,未来的以发生势能的破坏而不容进入心相续,所以,此中当知如是连结:他因圣道的修习,切断了已生者,不再培植正生者,不再为其生起而滋益未来,他们说他是独行的牟尼,且这大仙得见寂静的境地。
\item 以一向离于烦恼为\textbf{独},或以最胜之义为独。\textbf{行},即以一切行相圆满的世间义利之行及以其余之行而行。\textbf{这},即这如切断了已生者,且为不滋益增长而「不再培植正生者,不再滋益它」所说的佛牟尼。\textbf{寂静的境地},即寂静之处,在六十二见、毗婆舍那、涅槃等类的共许寂静、彼分寂静、究竟寂静三者中,当知大仙在如是未止息的世间得见究竟寂静之义。\end{enumerate}

\subsection\*{\textbf{211} {\footnotesize 〔PTS 209〕}}

\textbf{省察了依处,碾碎了种子,不再以爱执滋益它,\\}
\textbf{他实是见生之尽头的牟尼,舍弃了寻思,不可得名。}

Saṅkhāya vatthūni pamāya bījaṃ, sineham assa nānuppavecche;\\
sa ve munī jātikhayantadassī, takkaṃ pahāya na upeti saṅkhaṃ. %\hfill\textcolor{gray}{\footnotesize 3}

\begin{enumerate}\item 在第三颂中,\textbf{省察},即考虑、确定、考察、如实了知,以遍知苦而遍知之义。\textbf{依处},即此世间执著于其中的蕴、处、界等类的烦恼处。\textbf{碾碎了种子},即碾碎、伤害、阻碍了作为这些依处的种子的行作识\footnote{行作识:据菩提比丘注 879,\textbf{增支部}疏云「行作识,即业俱生之识」,而因其在新有之初生起新蕴等,被称为「种子」。},以正断断而舍弃之义。\textbf{不再以爱执滋益它},即这种子为爱、见之爱执所执,以将来的结生,能生长先前所说的依处之谷物,不再以爱执滋益它,以对治彼的道修习不再滋益它之义。
\item \textbf{他实是见生之尽头的牟尼},即他这样的佛牟尼,以证得涅槃,由得见作为生死尽头的涅槃而为见生之尽头者。\textbf{舍弃了寻思,不可得名},以此四谛之修习,舍弃了九不善寻\footnote{九不善寻,见\textbf{蛇经}第 7 颂义注。},在得证有余依涅槃界后,行世间的义利,渐次地,在最后识灭去、得证无余依涅槃界时,不可得名为天或人。或者,正如未涅槃者由未舍弃欲寻等的寻,得名为「这人是贪者」或「嗔者」,如是舍弃了寻思而不可得名,当如是视此中之义。\end{enumerate}

\subsection\*{\textbf{212} {\footnotesize 〔PTS 210〕}}

\textbf{了知了一切住处,也不欲求其中的某个,\\}
\textbf{他实是牟尼,离贪而无求,不再追逐,因为已到彼岸。}

Aññāya sabbāni nivesanāni, anikāmayaṃ aññataram pi tesaṃ;\\
sa ve munī vītagedho agiddho, nāyūhatī pāragato hi hoti. %\hfill\textcolor{gray}{\footnotesize 4}

\begin{enumerate}\item 在第四颂中,\textbf{了知},即以无常等方法了知。\textbf{一切},即无余。\textbf{住处},即欲有等的有。因为有情在其中居住,所以称为住处。\textbf{也不欲求其中的某个},即如是由所见的过患,连这些住处中一个也不希求。\textbf{他}这样的佛\textbf{牟尼},以道修习之力,由离于渴爱的贪求而\textbf{离贪},并由离贪而\textbf{无求},不像一众未离贪者却自称「我们无求」。\textbf{不再追逐},即不造作令生彼彼住处的善或不善。什么原因?\textbf{因为已到彼岸},因为这样的人已到一切住处的涅槃彼岸之义。
\item 如是,在第一颂中,指责了凡夫的知见,赞叹自己的知见,在第二颂中,以无有凡夫以之不止息的烦恼,赞叹自己得证寂静的境地,在第三颂中,在凡夫于彼等依处不舍寻思而得称如此如此之处,以于彼处修习四谛,舍弃了寻思,赞叹自己不可得名,在第四颂中,于欲求之凡夫以有、爱而追逐的将来住处,以于彼处无有渴爱,赞叹自己的不追逐,即用四颂,以阿罗汉为顶点,完成了一处缘起的开示。\end{enumerate}

\subsection\*{\textbf{213} {\footnotesize 〔PTS 211〕}}

\textbf{征服一切,知晓一切,善慧,于一切法不染,\\}
\textbf{舍弃一切,于渴爱尽处解脱,智者们知晓他实是牟尼。}

Sabbābhibhuṃ sabbaviduṃ sumedhaṃ, sabbesu dhammesu anūpalittaṃ;\\
sabbañjahaṃ taṇhakkhaye vimuttaṃ, taṃ vāpi dhīrā muni vedayanti. %\hfill\textcolor{gray}{\footnotesize 5}

\begin{enumerate}\item 缘起为何?大人在行了大出离,渐次证得一切知后,为转法轮前往波罗奈,在菩提座和伽耶之间,与活命者优波迦相遇,当以\begin{quoting}朋友!你诸根明净……(中部·圣寻经)\end{quoting}等方式被问及时,便说了「征服一切」等。优波迦在说了「朋友!兴许是」后,摇头择了旁路,便即离开,并渐渐到了孟加拉国的某个猎人村庄。猎人的头领见后,想「哎!少欲的沙门,甚至都没著下衣,他是世间的阿罗汉」,带至家中,施以肉味,和儿女一起礼拜,邀请道「尊者!就住在这里,我会供给资具」,建了住处后便施与。他即住在那里。
\item 在热季,当兽群为逐清凉俱水之处而离远时,猎人为前往那里,便命女儿差婆「恭敬地给侍我们的阿罗汉」,与儿兄们一起前去。他的女儿容貌可观,体态丰满。第二天,优波迦来到家中,见到少女作了一切准备,前来施食,便为贪染征服,乃至不能饮食,以器皿盛了饭食回到住处后,将饭食丢在一边「若得了差婆,我就活,要不然,我就死」,饭也不吃就躺倒。
\item 第七天,猎人回来后,便问女儿优波迦的经过。她说:「只来了一天,之后再也没来过。」猎人想「我就穿回来的衣装前去问他」,即刻前往,按摩着双足便问:「尊者!哪里不舒服?」优波迦呻吟着,只是辗转反侧。他便说:「说吧!尊者!只要是我能做的,我都做!」优波迦便说:「若得了差婆,我就活,要不然,还是死了好。」「那么,尊者!你会什么技艺吗?」「我不会。」「尊者!不会任何技艺,难以安立俗家生活。」他便说:「我什么技艺也不会,不过,我可以为你挑肉卖肉。」猎人想「有这我就知足了」,给了上衣,带回家中,便许了女儿。他们同居后便生下一子,为其取名作「善贤」。差婆用哄孩子的歌谣嘲笑优波迦,他不堪,便说「夫人!我去『无尽胜者』的跟前」,往中国出发。
\item 尔时,世尊住舍卫国祇林大寺。于是,世尊事先命令众比丘:「诸比丘!若有前来询问『无尽胜者』的,你们带来见我!」优波迦渐次来到舍卫国,站在寺里,便问:「我的朋友『无尽胜者』在这寺里,他住在哪里?」众比丘便领他到世尊跟前。世尊为他作了随适的开示。他便在开示的终了,住于阿那含果。众比丘听闻他先前的经历,便发起议论:「世尊最初曾为不祥的裸行沙门开示法。」世尊了知到发起了议论,便从香房出来,以随适于此时的神变坐于佛座,告诸比丘:「诸比丘!你们现今为何议论而共坐?」他们便说了一切。随后,世尊说「诸比丘!如来非无因缘而开示法,如来之开示法无有尘垢,从中不可得见过失,因此,诸比丘!依于法的开示,优波迦现今成了阿那含」,为显明自己开示之无垢,说了此颂。
\item 其义为:于有漏的一切蕴处界,以舍弃欲贪而不为彼等所胜,并由自身征服彼等一切法,为\textbf{征服一切}。且由以一切行相知晓彼等及其余一切法为\textbf{知晓一切}。由具足堪能开示一切法的净慧为\textbf{善慧}。以爱、见的沾染之力而于有漏的蕴等类的一切法沾染,由于彼等无此沾染,为\textbf{于一切法不染}。且以于彼等一切法无有欲贪,由舍弃彼等一切法而住,为\textbf{舍弃一切}。以倾向于远离依持之心,由在渴爱尽处的涅槃个殊地解脱,为\textbf{于渴爱尽处解脱},即是说信解\footnote{这里,个殊地解脱 \textit{visesena mutta}、信解 \textit{adhimutta} 等都是从语源上解释「解脱 \textit{vimutta}」。}。
\item \textbf{智者们知晓他实是牟尼},即有智的有情们知晓、了知他也是牟尼。以「看!这牟尼多奇特,他的开示何处有垢」阐明自身,而其中的「实」字是阐明之义。而有人解释道:众比丘发起议论「优波迦那时见到如来,并不信『这是佛牟尼』」,随后,世尊为显示「不论他信不信,智者们知晓他是牟尼」,说了此颂。\end{enumerate}

\subsection\*{\textbf{214} {\footnotesize 〔PTS 212〕}}

\textbf{有慧力,具足戒行,等持,乐于禅那,具念,\\}
\textbf{解脱于执著,无荒秽,无漏,智者们知晓他实是牟尼。}

Paññābalaṃ sīlavatūpapannaṃ, samāhitaṃ jhānarataṃ satīmaṃ;\\
saṅgā pamuttaṃ akhilaṃ anāsavaṃ, taṃ vāpi dhīrā muni vedayanti. %\hfill\textcolor{gray}{\footnotesize 6}

\begin{enumerate}\item 缘起为何?此颂是就离婆多长老而说。这里,当如「于村落林间」一颂\footnote{即\textbf{法句}·阿罗汉品第 98 颂,离婆多即舍利弗的幼弟,事见\textbf{法句}义注。}所述而知离婆多长老最初的出家、既出家而住于金合欢林、住于彼处而证殊胜,以及世尊往还于彼处。而当世尊返回时,一个老年比丘忘了鞋子,回去取时,见被系在金合欢树上,等到了舍卫国,优婆夷毗舍佉问众比丘「尊者!离婆多长老的住处是否惬意」,他便斥责那些赞叹的众比丘,说:「优婆夷!这些人妄说,场地并不善妙,就是极粗虐的金合欢林\footnote{这里的金合欢当是阿拉伯金合欢,多刺,老年比丘去取系在树上的鞋时,恐多不便,故出怨言。}。」
\item 他吃了毗舍佉的客饭,饭后对聚集在圆亭的众比丘嫌恨道:「朋友!你们见到离婆多长老的坐卧处很惬意吗?」世尊了知后,便从香房出来,以随适于此时的神变到达集会中间,坐于佛座,告诸比丘:「诸比丘!你们现今为何议论而共坐?」他们便说:「尊者!就离婆多发起谈论:『这样的建造者,何时能行沙门法?』」便说:「诸比丘!离婆多不是建造者,离婆多是阿罗汉、漏尽者。」为给这些比丘作法的开示,说了此颂。
\item 其义为:由具足舍弃引致弱力之烦恼的成就,或由具足变化、决意等类的慧力,为\textbf{有慧力}。由具足四遍净戒及头陀支行为\textbf{具足戒行}。由道定、果定及威仪路的定为\textbf{等持}。由以近行、安止等类的禅那而乐,或于禅那而乐,为\textbf{乐于禅那}。由得达念的广大为\textbf{具念}。由解脱于贪染等的执著为\textbf{解脱于执著}。以无有五种心的荒秽\footnote{五种心的荒秽,见\textbf{有财者经}第 19 颂义注。}及四漏为\textbf{无荒秽}、\textbf{无漏}。\textbf{智者们知晓他实是牟尼},如是与慧等德相应、与执著等过失相违者,有智的有情们知晓他也是牟尼。以此阐明离婆多长老:看!这漏尽牟尼多奇特,如何能被说是「建造者」或「何时能行沙门法」?因为他以慧力完成了住处,而非以建造之功,他已作应作,现在已不用再行沙门法。而其中的「实」字是阐明之义。\end{enumerate}

\subsection\*{\textbf{215} {\footnotesize 〔PTS 213〕}}

\textbf{独行的牟尼不放逸,不为毁誉所动,\\}
\textbf{好比狮子不惊怖于声响,好比清风不羁绊于罗网,好比莲花不著于水\footnote{以上三句也见于\textbf{犀牛角经}第 71 颂。},\\}
\textbf{引领他人,而非被他人引领,智者们知晓他实是牟尼。}

Ekaṃ carantaṃ munim appamattaṃ, nindāpasaṃsāsu avedhamānaṃ;\\
sīhaṃ va saddesu asantasantaṃ, vātaṃ va jālamhi asajjamānaṃ,\\
\makebox[2em]{} padmaṃ va toyena alippamānaṃ;\\
netāram aññesam anaññaneyyaṃ, taṃ vāpi dhīrā muni vedayanti. %\hfill\textcolor{gray}{\footnotesize 7}

\begin{enumerate}\item 缘起为何?当世尊从菩提座出发,渐次到达迦毗罗卫,父子相聚时,被净饭王以「尊者!你在居家之时,穿在香箧中熏过的迦尸国等的布,现在怎么能著截断的粪扫衣」等问及,为引导国王,说道:\begin{quoting}亲爱的!你对我说的丝织、麻织的迦尸布,\\粪扫衣较之更好,这才是我所希求的。\end{quoting}等等,为显示自己不为世间法所动,说了这七句之颂,以向国王开示法。
\item 其义为:以被称为出家等为\textbf{独},以威仪路等之行为\textbf{行}。以具足寂默之法为\textbf{牟尼}。由在一切处无有放逸为\textbf{不放逸}。在骂詈、指责等类的毁,以及赞美、称赏等类的誉等\textbf{毁誉}中,\textbf{不}因对抗、随从而\textbf{为所动}。且当知此中以毁誉为首而说八世间法\footnote{八世间法:即得失 \textit{lābhālābha},荣辱 \textit{yasāyasa},毁誉 \textit{nindāpasaṃsā},苦乐 \textit{sukhadukkha} 等。}。
\item \textbf{好比狮子不惊怖于}鼓声等的\textbf{声响},于八世间法以未致天性的变化,或于边鄙的坐卧处以无有恐惧而不惊怖。\textbf{好比清风不羁绊于}线织等类的\textbf{罗网},于爱、见的罗网以四道而不羁绊,或于八世间法不以对抗、随从而羁绊。\textbf{好比莲花不著于水},虽生于世间,由舍弃众有情以之染著世间的爱、见之染著,而不著于世间。
\item 令生起趣向涅槃之道,以此道\textbf{引领}其他天人,但由自己不被其他任何人显示道而引领\textbf{而非被他人引领}。\textbf{智者们知晓他实是牟尼},以「他们知晓是佛牟尼」阐明自身。其余仍如前述。\end{enumerate}

\subsection\*{\textbf{216} {\footnotesize 〔PTS 214〕}}

\textbf{当别人极端地说话时,如浴场的柱子般不动,\\}
\textbf{离贪,善等持诸根,智者们知晓他实是牟尼。}

Yo ogahaṇe thambhor ivābhijāyati, yasmiṃ pare vācā pariyantaṃ vadanti;\\
taṃ vītarāgaṃ susamāhitindriyaṃ, taṃ vāpi dhīrā muni vedayanti. %\hfill\textcolor{gray}{\footnotesize 8}

\begin{enumerate}\item 缘起为何?世尊初现等觉,而有四阿僧祇又十万劫所圆满的十波罗蜜、十小波罗蜜、十第一义波罗蜜等的志向功德,圆满波罗蜜后,在兜率天居处的转生功德,于此居住的功德,大观察的功德,入胎、住胎、出胎、错步、观察方向、梵吼、大出离、大精进、现等觉、转法轮,四种道智果智、于八众不动智、十力之智、四生区分智、五趣区分智、包括六种不共智及八种与声闻共通之佛智的十四种佛智、十八佛德区分智、十九种省察智等七十七种智的依处,依于如是等的百千功德,而获大利养恭敬,众外道不堪忍受,便遣散女学童罗望子,以「违犯一乘法」一颂\footnote{即\textbf{法句}·世品第 176 颂。 }所述之事,在四众之中令对世尊生起不名誉,以此缘由,众比丘便发起议论:「即便发生这样的不名誉,世尊的心也不转。」
\item 世尊了知后,便从香房出来,以随适于此时的神变到达集会中间,坐于佛座,告诸比丘:「诸比丘!你们现今为何议论而共坐?」他们便说了一切。随后,世尊说「诸比丘!诸佛者,于八世间法如如也」,为向这些比丘开示法,说了此颂。
\item 其义为:正好比在人们洗澡的浴场中,为摩擦肢体而植入四方或八角的柱子,不论由上等家族还是下等家族来摩擦肢体,柱子不会因此而高举或倾俯,如是,\textbf{当别人极端地说话时,如浴场的柱子般不动}。这说的是什么?别的外道或者他人于某事以最上的赞美或最下的不赞美极端地说话时,则于此事不陷入随从或对抗,而以如如的状态,如浴场的柱子般。
\item \textbf{离贪,善等持诸根},即于可意的所缘,以无有贪染而离贪,以及于不可意的所缘,以无有嗔痴而善等持诸根,或是说,经善加汇集而安置诸根、守护诸根、保护诸根。\textbf{智者们知晓他实是牟尼},以「他们知晓是佛牟尼,他的心如何会转」阐明自身。其余仍如前述。\end{enumerate}

\subsection\*{\textbf{217} {\footnotesize 〔PTS 215〕}}

\textbf{坚定,如梭子般正直,嫌厌于恶业,\\}
\textbf{审视邪正,智者们知晓他实是牟尼。}

Yo ve ṭhitatto tasaraṃ va ujju, jigucchati kammehi pāpakehi;\\
vīmaṃsamāno visamaṃ samañ ca, taṃ vāpi dhīrā muni vedayanti. %\hfill\textcolor{gray}{\footnotesize 9}

\begin{enumerate}\item 缘起为何?据说,在舍卫国,某商人之女从楼阁下来,去到楼阁下的织布堂,看到人们在走梭,因其正直,便得了与之相似的相:「哎!一切有情舍弃了身语意的邪曲后,心便如梭子般正直!」她又上到楼阁,再再转向于此相而坐。且以如是的行道,不久无常相便成明了,又据此而有苦、无我相。于是,三有便如炽燃般现起。世尊了知到她如是修观,便坐在香房里放光。她见后,转向于「这是什么」,见到世尊仿佛坐在身旁,便起来合掌而立。于是,世尊知晓其顺适,以法的开示说了此颂。
\item 其义为:以一境心与不动解脱而无有增减,并由灭尽生与轮回而不入下一有,故\textbf{坚定}。由舍弃身语意的邪曲,或由不行于非道,故\textbf{如梭子般正直}。由具足惭愧而\textbf{嫌厌于恶业},即是说嫌厌、惭耻恶业如粪便一般。以连结的分开,而在语法上用具格得成业格之义。\textbf{审视邪正},即依舍弃与修习的作用成就,以道慧审视、考察身邪等的邪与身正等的正。\textbf{智者们知晓他实是}漏尽的\textbf{牟尼}。
\item 这说的是什么?以如上所述的道慧审视邪正的坚定者,他如是如梭子般正直,不作任何违犯,嫌厌于恶业,智者们知晓他实是牟尼。为显示「因为作为这样的人」为漏尽牟尼,以阿罗汉为顶点开示了此颂。在开示终了,商人之女即住于须陀洹果。且此中,当视「实」字为可选或合并之义。\end{enumerate}

\subsection\*{\textbf{218} {\footnotesize 〔PTS 216〕}}

\textbf{自制,当少年及中年时不作恶,克己,牟尼\\}
\textbf{不可使怒,他不激怒任何人,智者们知晓他实是牟尼。}

Yo saññatatto na karoti pāpaṃ, daharo majjhimo ca muni yatatto;\\
arosaneyyo na so roseti kañci, taṃ vāpi dhīrā muni vedayanti. %\hfill\textcolor{gray}{\footnotesize 10}

\begin{enumerate}\item 缘起为何?当世尊住在旷野时,旷野城内的某个织布工命七岁的女儿:「姑娘!昨天残存的梭子不多了,走完梭就快点来织布堂,别耽搁!」她便领命「善哉」,去到堂里,站着理线。
\item 这天,世尊从大悲等至出起,观察世间,见到这女孩须陀洹果的近依,以及在开示终了,八万四千生类法的现观,便清早料理完身体,持了衣钵进城。人们见到世尊,想「看来今天有人要被摄受了,世尊进来得早」,便追随世尊。世尊站在这女孩去到父亲跟前所经的路边。城民们便洒扫了此地,呈上鲜花,扎了华盖,设好坐处。世尊在设好的坐处坐下,大众围绕而立。
\item 这女孩到达此地,见到大众围绕着世尊,便五体投地礼拜。世尊便问她:「女孩!从哪里来?」「我不知道,世尊!」「往哪里去?」「我不知道,世尊!」「你不知道?」「我知道,世尊!」「你知道?」「我不知道,世尊!」众人听后,便讥嫌道:「看!先生!这女孩明明从家里来,被世尊问,便说不知,又明明往织布堂去,被问也说不知,说『你不知道』便说知道,说『你知道』便说不知道,一切都是反着来。」
\item 为向众人明示其义,世尊便问她:「我问的是什么?你说的是什么?」她便说:「尊者!没人不知道我『从家里来、往织布堂去』,但你是以结生来问我『从哪里来』,以死来问『往哪里去』,我都不知道,因为我不知道『我从哪里来,地狱还是天界?我将往哪里去,地狱还是天界』,所以我便说『我不知道』,随后世尊就死来问我『你不知道』,我知道『所有人都会死』,因此便说『我知道』,随后世尊就死的时间来问我『你知道』,我不知道『何时会死?今天抑或明天』,因此便说『我不知道』。」
\item 世尊便随喜她的解答道:「善哉!善哉!」大众也鼓掌千遍:「这女孩真是智者!」于是,世尊知晓女孩的随适,为开示法,说了此颂:\begin{quoting}此世界盲暝,能得见者少,\\如鸟脱罗网,鲜有升天者。(法句·世品第 174 颂)\end{quoting}她在偈颂的终了便住于须陀洹果,八万四千生类得了法的现观。
\item 她礼拜了世尊后,便去到父亲跟前。父亲见后,怒道「这么久才来」,猛地把线扔向织机。它弹了出来,竟刺破女孩的小腹,她便死在那里。他见后,审视道「我没打我的女儿,而是从这织机猛地弹了出来,刺破了她小腹,她还活着没」,见到已死,便想:「人们得知我害死了女儿,便会责难,国王也会因此裁以重罚,噫!我还是趁早逃吧!」
\item 他因害怕惩罚逃到世尊跟前,取了业处,便到了在林间居住的众比丘的住处,且往这些比丘处请求出家。他们度其出家后,更授了皮五法的业处。他取了业处而精进,不久便证得了阿罗汉,而他的那些阿阇黎与和尚也是。于是,所有人为了大自恣都去到世尊跟前:「我们将行清净的自恣!」世尊自恣后,出了安居,为比丘僧团所随从,便游行于村镇间,渐次到了旷野。
\item 那里,人们邀请世尊,在作布施等时见到那比丘,便说「害死了女儿,现在还来害谁」等等予以嘲讽。众比丘听后,在给侍时便前往告知世尊此事。世尊说「诸比丘!这比丘并未害死女儿,她因自己的业而死」,为阐明此比丘难为众人了知的漏尽牟尼的状态,向众比丘开示法,说了此颂。
\item 其义为:于三业门以戒自制为\textbf{自制},\textbf{不}以身语意\textbf{作}杀生等的\textbf{恶},且\textbf{少年}住于少年时,或\textbf{中年}住于中年时,同理,或老年住于暮年时,无论何时都不作。什么原因?\textbf{克己},即是说因为以无上的戒离,心已止息一切恶。
\item 现在,对「牟尼不可使怒,他不激怒任何人」这些词,其连结与旨趣为:这漏尽\textbf{牟尼不可使怒},不应以「害女者」或「织工」等方式的身语令怒、刺激、逼恼,因为\textbf{他}也\textbf{不激怒任何人},不以「我没害我的女儿,你害的,或你这样的害的」等激怒、刺激、逼恼任何人,而唯应如\begin{quoting}且置龙象!莫刺激龙象!请礼敬龙象!(中部·蚁垤经)\end{quoting}等所说,予以礼敬。\textbf{智者们知晓他实是牟尼},当如是分析词语:即此中唯有智者们知晓他实是牟尼。且此中的旨趣为:这些愚人不知道他「不可使怒」而激怒之,而唯有智者们知晓他也是牟尼,即了知「这是漏尽牟尼」。\end{enumerate}

\subsection\*{\textbf{219} {\footnotesize 〔PTS 217〕}}

\textbf{靠他人布施的活命者,无论从上、中或从余处得到食物,\\}
\textbf{不去赞美,也不贬低,智者们知晓他实是牟尼。}

Yad aggato majjhato sesato vā, piṇḍaṃ labhetha paradattūpajīvī;\\
nālaṃ thutuṃ no pi nipaccavādī, taṃ vāpi dhīrā muni vedayanti. %\hfill\textcolor{gray}{\footnotesize 11}

\begin{enumerate}\item 缘起为何?据说,在舍卫国,有婆罗门名「五上施者」。他在谷物成熟时,以最上之田、最上之聚、最上之仓、最上之釜、最上之食等五种最上而布施。此处,让人拿来最初刚成熟的稻、大麦、小麦的穗,备好糜、粥、面等,以「有智者施于最上,他便获得最上」之见,施与以佛陀为首的比丘僧团,此即其\textbf{最上之田}的布施。而当谷物已熟,且经收割、脱粒后,以最好的谷物如上布施,此即其\textbf{最上之聚}的布施。再者,让人将这些谷物装满粮仓后,当开启第一个粮仓时,以最初取出的谷物如上布施,此即其\textbf{最上之仓}的布施。再者,凡在其家烹饪的最上者,在未施与来到的出家人时,甚至都不与孩子等的任何人,此即其\textbf{最上之釜}的布施。再者,当自己饮食时,对最初供奉的食物,在饭前时间未施与僧团,饭后时间未与到来的乞者,若无此时,甚至未与狗则不食,此即其\textbf{最上之食}的布施。如是,他便以「五上施者」得称。
\item 于是,某天,世尊在黎明时分以佛眼观察世间,见到这婆罗门和婆罗门尼须陀洹道的近依,便料理完身体,早早进入香房。众比丘见到香房的门紧闭,便了知「世尊今天想独自入村」,到了行乞之时,便在右绕香房后,入而乞食。世尊也在婆罗门用餐的时间出来,入舍卫国。人们见到世尊如是,了知到「看来今天有某个有情要被摄受了,因为世尊这样独自进入」,便未前来邀请。
\item 世尊渐次到达婆罗门的家门而立。此时,婆罗门正端着食物而坐,婆罗门尼则持扇而立。她见到世尊,想「如果婆罗门看到,就会取了钵施与所有食物,然后我就要再去煮饭」,生起不喜与悭吝,用多罗扇遮蔽,好让婆罗门看不到世尊。世尊了知后,便放出身光。婆罗门见到金光,巡视着「这是什么」,便看到世尊站在门外。婆罗门尼想「他看到世尊了」,便立刻放下多罗扇,走到世尊处,五体投地而拜。且在其礼拜而起时,已知晓其随适,说了此颂:\begin{quoting}若于名与色,不著我我所,\\非有故无忧,彼实称比丘。(法句·比丘品第 367 颂)\end{quoting}她即在偈颂的终了住立于须陀洹果。婆罗门则请世尊进入室内,令坐于最上之坐,施与供养与水,手授以自己的饮食:「尊者!你是俱有天的世间最上的应供,善哉!请允许我在你的钵里盛上这食物!」世尊为摄受他,便受纳而享用。食事已毕,知晓了婆罗门的随适,说了此颂。
\item 其义为:由从釜中最初所得为\textbf{从上},从剩余的半釜所得为从\textbf{中},从剩余一二匙之量的釜中所得为\textbf{从余处得到食物}。\textbf{靠他人布施的活命者},即出家人。因为他除了水与齿木,其余唯靠他人所施而活命,所以称为「靠他人布施的活命者」。\textbf{不去赞美,也不贬低},即对从上之所得,由舍弃随贪,不应赞美其或施主,对从余处所得,由舍弃嗔恨,也不以「他给的是什么」等方式贬低施主,并说不喜之语。\textbf{智者们知晓他实是牟尼},即智者们知晓他舍弃了随贪与嗔恨,实是牟尼,以阿罗汉为顶点,对婆罗门开示了此颂。在偈颂的终了,婆罗门便住立于须陀洹果。\end{enumerate}

\subsection\*{\textbf{220} {\footnotesize 〔PTS 218〕}}

\textbf{牟尼游行,戒离交媾,青春之时不束缚于任何处,\\}
\textbf{戒离㤭慢与放逸,解脱,智者们知晓他实是牟尼。}

Muniṃ carantaṃ virataṃ methunasmā, yo yobbane nopanibajjhate kvaci;\\
madappamādā virataṃ vippamuttaṃ, taṃ vāpi dhīrā muni vedayanti. %\hfill\textcolor{gray}{\footnotesize 12}

\begin{enumerate}\item 缘起为何?据说,在舍卫国,某个商人之子随季节在三个楼阁中为一切成就所敬事,即在孩童,已欲出家,向父母请求,如犀牛角经「爱欲实在多彩」一颂\footnote{即\textbf{犀牛角经}第 50 颂。}的缘起中所述,三次出家及还俗,第四番时,便证得了阿罗汉。众比丘因其先前的习惯,对他说:「朋友!是时候还俗了。」他便说:「朋友!我现在不能还俗。」众比丘听后,便告知了世尊。世尊说「如是,诸比丘!他现在不能还俗」,为解释其漏尽沙门的状态,说了此颂。
\item 其义为:以具足寂默之法为\textbf{牟尼},以独自居住,或以先前所说品类的任一所行而\textbf{游行},心不似先前一般行于交媾法,以无上的戒离而\textbf{戒离交媾}。第二句的情形为:若问什么样的牟尼游行,戒离交媾?\textbf{青春之时不束缚于任何处}者,即便正当美好青春之时,也不像先前那样,因交媾的贪染,束缚于任何女人之处。或者,此中之义为,于任何处,当自己或他人的青春之时,不以「我正青春,或他正青春,让我先受用爱欲」,因贪染而束缚。
\item 且不仅戒离交媾,还\textbf{戒离}出身㤭慢等类的\textbf{㤭慢},\textbf{与}被称为离念于种种爱欲的\textbf{放逸},且由戒离㤭慢与放逸而从一切烦恼束缚中\textbf{解脱}。或者,非如以世间之戒离而戒离,而是\textbf{解脱戒离},即由从一切烦恼束缚中解脱,以出世间之戒离而戒离之义。\textbf{智者们知晓他实是牟尼},即显示「智者们知晓他也是牟尼,而你们却不知晓,因此这样说他」。\end{enumerate}

\subsection\*{\textbf{221} {\footnotesize 〔PTS 219〕}}

\textbf{了知世间,见第一义,度过暴流、大海而如如,\\}
\textbf{切断系缚,无所依,无漏,智者们知晓他实是牟尼。}

Aññāya lokaṃ paramatthadassiṃ, oghaṃ samuddaṃ atitariya tādiṃ;\\
taṃ chinnaganthaṃ asitaṃ anāsavaṃ, taṃ vāpi dhīrā muni vedayanti. %\hfill\textcolor{gray}{\footnotesize 13}

\begin{enumerate}\item 缘起为何?世尊住迦毗罗卫。尔时,他们正为难陀做璎珞、灌顶、婚礼等三件喜事。世尊于此也受到邀请,与五百比丘一起到了那里,吃完离开时,便把钵放到难陀的手上。见其离开,倾国难陀便说:「主人!你要快点回来!」他出于对世尊的尊重,不便说「噫!世尊!钵」,只能去到寺里。世尊站在香房的隔间说「难陀!拿钵来」,接了后便说:「你出家吗?」他出于对世尊的尊重,不便拒绝,便说:「我出家!世尊!」世尊便度他出家。但他反复忆念着倾国难陀的话,便生烦躁。众比丘告知了世尊。世尊为除遣难陀的不乐,便说:「难陀!你之前去过三十三天的居处吗?」难陀说:「尊者!我之前没去过。」
\item 随后,世尊以自身的威神,将他带至三十三天的居处,站在最胜殿的门前。得知世尊前来,帝释为众天女所随从,从殿上下来。她们全都布施过迦叶世尊的弟子们以涂足油,便有斑鸠似的足。于是,世尊便告难陀:「难陀!有没看见这五百天女有斑鸠似的足?」详述了一切。在整个佛语中,都没有应把握女人的相与随形好的,而在这里,世尊以方便善巧,如医生欲祛病,先以美食令恼患者呕出毒素般,欲祛病而先令难陀呕出贪染,作为无上调御丈夫,允许他把握相与随形好。
\item 随后,世尊看到难陀因天女而乐于梵行,便命众比丘:「你们应以雇工指责难陀!」他为众人指责而羞耻,经如理作意而行道,不久便得证阿罗汉。住在其经行道一端的树上的天人告知了世尊此事,世尊便生起智。而众比丘不知,仍如是指责尊者,世尊便说「诸比丘!现在不要再这样指责难陀了」,为显明其漏尽牟尼的状态,向这些比丘开示法,说了此颂。
\item 其义为:以行苦谛之差别而\textbf{了知}、知道、确定蕴等\textbf{世间},以证得灭谛而\textbf{见第一义},以舍断集而\textbf{度过}、超越四种\textbf{暴流},且由集的舍断,堪忍色之㤭慢\footnote{色之㤭慢 \textit{rūpamada} 费解。菩提比丘在注 916 中说,据\textbf{相应部}第 35:228 经:Yo taṃ rūpamayaṃ vegaṃ sahati, ayaṃ vuccati, bhikkhave, atari cakkhusamuddaṃ……,应作「色之所造 \textit{rūpamaya}」,当是。}等的冲激,度过眼等处的\textbf{大海},以道的修习,以「彼义释如如\footnote{彼义释如如 \textit{taṃniddesā tādī},见\textbf{大义释},即阿罗汉于戒、信、精进、念、定、慧、明、神通等名称如如。如如,即不偏之义,详见\textbf{最上八颂经}第 810 颂的译注。}」得至如如之相而\textbf{如如}。或者,此爱欲之贪染等的烦恼聚,以向下裹挟之义为暴流,以转向卑劣之趣、以鼓荡之义为大海,以舍断集而度过暴流与大海,由已度暴流,现在即便被你们如是说,也以不至混乱而如如,当知如是为此中之义与旨趣。
\item \textbf{切断系缚,无所依,无漏},这是对其赞美之语,即是说他以四道的修习,由切断四种系缚为\textbf{切断系缚},由不依任何见或爱为\textbf{无所依},以无有四漏为\textbf{无漏}。\textbf{智者们知晓他实是牟尼},即显示:智者们知晓他也是漏尽牟尼,而你们尚不知,故作是说。\end{enumerate}

\subsection\*{\textbf{222} {\footnotesize 〔PTS 220〕}}

\textbf{两者不同,住处与行为差远,在家人养育妻子,善行者无我所,\\}
\textbf{在家人伤害别的生命而不自制,克己的牟尼总是保护生命。}

Asamā ubho dūravihāravuttino, gihī dāraposī amamo ca subbato;\\
parapāṇarodhāya gihī asaññato, niccaṃ munī rakkhati pāṇine yato. %\hfill\textcolor{gray}{\footnotesize 14}

\begin{enumerate}\item 缘起为何?某位比丘在㤭萨罗国土,依于边鄙的村庄,住于林野。在此村中,有狩猎人去到这比丘的住所捕兽。当他进入林野时,看到长老也入村乞食,从林野出来时,又看到他从村里离开,如是因常常见面,便对长老生出爱执。当他得了许多肉时,也施与长老美味的食物。人们讥嫌道:「这比丘告知猎人『野兽在某处站着、走着、喝水』,随后猎人就杀了野兽,因此两人一起营生。」于是,世尊在游行国土时,便到了此地。众比丘入村乞食时,听闻这经过,便告知了世尊。世尊为显示不与猎人相同活命的成就,与此比丘漏尽牟尼的状态,向这些比丘开示法,说了此颂。
\item 其义为:诸比丘!比丘与猎人,这\textbf{两者不同},人们说「相同活命」是错的。什么原因?\textbf{住处与行为差远},他们的住处与行为相差甚远,住处即住所,这比丘在林野,而猎人在村内,行为即活命,这比丘在村内次第行乞,而猎人在林野猎杀鸟兽。复次,\textbf{在家人养育妻子},这猎人以其业抚养妻儿。\textbf{善行者无我所},这漏尽比丘于妻儿无爱、见、我所,由清净行及善妙行为善行者。复次,\textbf{在家人伤害别的生命而不自制},这在家的猎人斩杀其它生命的命根,不以身语心等自制。\textbf{克己的牟尼总是保护生命},而另一漏尽牟尼总是以身语心等自制、克己,保护生命。如是情形下,他们如何能有相同的活命呢?\end{enumerate}

\subsection\*{\textbf{223} {\footnotesize 〔PTS 221〕}}

\textbf{好比青颈的孔雀飞在空中,永远也赶不上天鹅的速度,\\}
\textbf{如是,在家人也无法效仿比丘,那在林中独处禅修的牟尼。}

Sikhī yathā nīlagīvo vihaṅgamo, haṃsassa nopeti javaṃ kudācanaṃ;\\
evaṃ gihī nānukaroti bhikkhuno, munino vivittassa vanamhi jhāyato ti. %\hfill\textcolor{gray}{\footnotesize 15}

\begin{enumerate}\item 缘起为何?当世尊住迦毗罗卫时,众释氏发起谈论:「先入流者在法上较后证入流者为长,所以后入流的比丘应向先入流的在家人行问讯等。」某位行乞的比丘听了这谈论,便告知世尊。世尊就「这与出生不同,应予尊敬的依处乃是形相」而说「诸比丘!在家人即便是阿那含,也应对当天出家的沙弥行问讯等」,又为显示即便是后入流的比丘,较之先入流的在家人也有极大的殊胜,向众比丘开示法,说了此颂。
\item 其义为:以天生顶有羽冠为\textbf{孔雀},以与摩尼杖相似的颈为\textbf{青颈},以此来说\textbf{飞在空中}的孔雀。好比在雏天鹅、棕天鹅、白天鹅、黑天鹅、老天鹅、金天鹅中的金天鹅,\textbf{赶不上}此\textbf{天鹅的速度}的十六分之一。因为金天鹅在须臾间能行一千由旬,而其它则连一由旬也不能够。但从可观而言,两者都很可观。\textbf{如是,在家人}虽然先入流,以见道而可观,但他\textbf{也无法效仿}后入流、以见道而同等可观的\textbf{比丘}的速度。
\item 什么速度?上分道的观智的速度。因为在家人的智由为妻儿等的结所缚而迟钝,而对比丘,由已解此结故锐利。世尊以「\textbf{在林中独处禅修的牟尼}」一句来显明此义。因为这有学牟尼的比丘以身心的远离而独处,且总是在林中以省虑相与所缘\footnote{省虑相与所缘:见\textbf{犀牛角经}第 69 颂义注。}而禅修。此中的旨趣为:在家人如何能有这样的独处与禅那呢?\end{enumerate}

\begin{center}\vspace{1em}牟尼经第十二\\Munisuttaṃ dvādasamaṃ.\end{center}

其总颂曰:

\begin{quoting}蛇、有财者、角,以及耕田,\\纯陀、衰败、贱民、慈修习,\\七岳、旷野、胜利,及牟尼,\\这十二经,被称为「蛇品」。\end{quoting}

Tass’uddānaṃ —

\begin{quoting}Urago Dhaniyo c’eva, Visāṇañ ca tathā Kasi;\\Cundo Parābhavo c’eva, Vasalo Mettabhāvanā;\\Sātāgiro Āḷavako, Vijayo ca tathā Muni;\\dvādas’etāni suttāni, Uragavaggo ti vuccatī ti.\end{quoting}

\begin{center}\vspace{1em}蛇品第一\\Uragavaggo paṭhamo.\end{center}

%\begin{flushright}甲辰六月初九二稿\end{flushright}