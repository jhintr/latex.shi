\chapter{寵禮第二十二}

\subsection*{1}

\textbf{元帝正會,引王丞相登御牀,王公固辭,中宗引之彌苦,王公曰:「使太陽與萬物同暉,臣下何以瞻仰?」}{\footnotesize \textbf{中興書}曰元帝登尊號,百官陪位,詔王導升御坐,固辭然後止。}

\subsection*{2}

\textbf{桓宣武嘗請參佐入宿,袁宏、伏滔相次而至,蒞名,府中復有袁參軍,彥伯疑焉,令傳教更質,傳教曰:「參軍是袁、伏之袁,復何所疑?」}

\subsection*{3}

\textbf{王珣、郗超並有奇才,為大司馬所眷拔,珣為主簿,超為記室參軍,超為人多鬚,珣狀短小,于時荊州為之語曰:「髯參軍,短主簿,能令公喜,能令公怒。」}{\footnotesize \textbf{續晉陽秋}曰超有才能,珣有器望,並為溫所暱。}

\subsection*{4}

\textbf{許玄度停都一月,劉尹無日不往,乃歎曰:「卿復少時不去,我成輕薄京尹。」}{\footnotesize \textbf{語林}曰玄度出都,真長九日十一詣之,曰「卿尚不去,使我成薄德二千石」。}

\subsection*{5}

\textbf{孝武在西堂會,伏滔預坐,還,下車呼其兒,}{\footnotesize 兒,即系也。\textbf{丘淵之文章錄}曰系,字敬魯,仕至光祿大夫。}\textbf{語之曰:「百人高會,臨坐未得他語,先問『伏滔何在,在此不』,此故未易得,為人作父如此,何如?」}

\subsection*{6}

\textbf{卞範之為丹陽尹,羊孚南州暫還,往卞許,云:「下官疾動不堪坐。」卞便開帳拂褥,羊徑上大牀,入被須枕,卞回坐傾睞,移晨達莫,羊去,卞語曰:「我以第一理期卿,卿莫負我。」}{\footnotesize \textbf{丘淵之文章錄}曰範之,字敬祖,濟陰冤句人,祖㟪,下邳太守,父循,尚書郎,桓玄輔政,範之遷丹陽尹,玄敗,伏誅。}