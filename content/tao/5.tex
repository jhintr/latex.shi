\chapter{卷之五\hspace{1ex}賦辭}

\section{感士不遇賦\hspace{1ex}{\footnotesize 并序}}

\begin{quoting}昔董仲舒作士不遇賦,司馬子長又為之,余嘗以三餘之日,講習之暇,讀其文,慨然惆悵。夫履信思順,生人之善行,抱朴守靜,君子之篤素,自真風告逝,大偽斯興,閭閻懈廉退之節,市朝驅易進之心,懷正志道之士,或潛玉於當年,潔己清操之人,或沒世以徒勤,故夷皓有安歸之歎,三閭發已矣之哀。悲夫!寓形百年而瞬息已盡,立行之難而一城莫賞,此古人所以染翰慷慨,屢伸而不能已者也,夫導達意氣,其惟文乎?撫卷躊躇,遂感而賦之。\end{quoting}

\textbf{咨大塊之受氣,何斯人之獨靈,禀神智以藏照,秉三五\footnote{三五:三才五常。}而垂名。或擊壤以自歡,或大濟於蒼生,靡潛躍之非分,常傲然以稱情。世流浪而遂徂,物群分以相形,密網裁而魚駭,宏羅制而鳥驚。彼達人之善覺,乃逃祿而歸耕,山嶷嶷而懷影,川汪汪而藏聲,望軒唐而永歎,甘貧賤以辭榮。}

\textbf{淳源汩以長分,美惡作以異途,原百行之攸貴,莫為善之可娛,奉上天之成命,師聖人之遺書,發忠孝於君親,生信義於鄉閭,推誠心而獲顯,不矯然而祈譽。}

\textbf{嗟乎!雷同毀異,物惡其上\footnote{晉書袁宏傳:人惡其上,世不容哲。},妙算者謂迷,直道者云妄,坦至公而無猜,卒蒙恥以受謗,雖懷瓊而握蘭,徒芳潔而誰亮\footnote{亮:同諒,信也。}?}

\textbf{哀哉士之不遇,已不在炎帝帝魁\footnote{張平子東京賦:仰不睹炎帝帝魁之美。注:帝魁,神農後也。}之世,獨祗\footnote{祗 \texttt{zhī}:敬也。}修以自勤,豈三省之或廢,庶進德以及時,時既至而不惠。無爰生之晤言,念張季之終蔽\footnote{爰生、張季:袁盎、張釋之也。}。愍馮叟於郎署,賴魏守以納計\footnote{馮叟、魏守:馮唐、魏尚也。}。}

\textbf{雖僅然於必知,亦苦心而曠歲,審夫市之無虎,眩三夫之獻說\footnote{韓非子:三人言,成市虎。}。悼賈傅之秀朗,紆遠轡於促界\footnote{二句言賈誼事。促,狹也。}。悲董相之淵致,屢乘危而幸濟\footnote{二句言董仲舒事。}。感哲人之無偶,淚淋浪以灑袂。}

\textbf{承前王之清誨,曰天道之無親,澄得一以作鑒,恆\footnote{恆:對澄言,易曰「天地之道恆久而不已也」。}輔善而佑仁。夷投老以長飢,回早夭而又貧,傷請車以備槨,悲茹薇而殞身,雖好學與行義,何死生之苦辛。疑報德之若茲,懼斯言之虛陳。}

\textbf{何曠世之無才,罕無路之不澀,伊古人之慷慨,病奇名之不立。廣結髮以從政,不愧賞於萬邑,屈雄志於戚豎,竟尺土之莫及,留誠信於身後,動眾人之悲泣\footnote{六句言李廣事。}。商盡規以拯弊,言始順而患入\footnote{二句言王商事。}。奚良辰之易傾,胡害勝其乃急。蒼旻遐緬,人事無已,有感有昧\footnote{昧:無感也。},疇\footnote{爾雅釋詁:疇,誰也。}測其理?寧固窮以濟意,不委曲而累己,既軒冕之非榮,豈縕袍之為恥\footnote{論語子罕:衣敝縕袍,與衣狐貉者立而不恥者,其由也與?}?誠謬會以取拙,且欣然而歸止,擁孤襟以畢歲,謝良價於朝市。}

\section{閑情賦\hspace{1ex}{\footnotesize 并序}}

\begin{quoting}初張衡作定情賦,蔡邕作靜情賦,檢逸辭而宗澹泊,始則蕩以思慮而終歸閑正,將以抑流宕之邪心,諒有助於諷諫,綴文之士奕代繼作,並固觸類,廣其辭義,余園閭多暇,復染翰為之,雖文妙不足,庶不謬作者之意乎?\end{quoting}

\textbf{夫何瓌逸之令姿,獨曠世以秀群,表傾城之艷色,期有德於傳聞,佩鳴玉以比潔,齊幽蘭以爭芬,淡柔情於俗內,負雅志於高雲。}

\textbf{悲晨曦之易夕,感人生之長勤,同一盡於百年,何歡寡而愁殷。褰朱幃而正坐,泛清瑟以自欣,送纖指之餘好,攘皓袖之繽紛,瞬美目以流眄,含言笑而不分。}

\textbf{曲調將半,景落西軒,悲商叩林,白雲依山,仰睇天路,俯促鳴絃,神儀嫵媚,舉止詳妍。激清音以感余,願接膝以交言,欲自往以結誓,懼冒禮之為諐\footnote{諐:即愆字。},待鳳鳥以致辭,恐他人之我先,意惶惑而靡寧,魂須臾而九遷。}

\textbf{願在衣而為領,承華首之餘芳,悲羅襟之宵離,怨秋夜之未央。}

\textbf{願在裳而為帶,束窈窕之纖身,嗟溫涼之異氣,或脫故而服新。}

\textbf{願在髮而為澤,刷玄鬢於頹肩,悲佳人之屢沐,從白水以枯煎。}

\textbf{願在眉而為黛,隨瞻視以閑揚,悲脂粉之尚鮮,或取毀於華粧。}

\textbf{願在莞\footnote{莞:草也,可以作席,詩小雅斯干「下莞上簟」。}而為席,安弱體於三秋,悲文茵\footnote{文茵:虎皮也,詩秦風小戎「文茵暢轂」。}之代御,方經年而見求。}

\textbf{願在絲而為履,附素足以周旋,悲行止之有節,空委棄於牀前。}

\textbf{願在晝而為影,常依形而西東,悲高樹之多蔭,慨有時而不同。}

\textbf{願在夜而為燭,照玉容於兩楹,悲扶桑之舒光,奄滅景而藏明。}

\textbf{願在竹而為扇,含淒飇於柔握,悲白露之晨零,顧襟袖以緬邈。}

\textbf{願在木而為桐,作膝上之鳴琴,悲樂極以哀來,終推我而輟音。}

\textbf{考所願而必違,徒契闊以苦心,擁勞情而罔訴,步容與於南林。栖木蘭之遺露,翳青松之餘陰,儻行行之有覿,交欣懼於中襟,竟寂寞而無見,獨悁\footnote{詩陳風澤陂:中心悁悁。}想以空尋。}

\textbf{斂輕裾以復路,瞻夕陽而流歎,步徙倚以忘趣,色慘悽而矜顏。葉燮燮以去條,氣淒淒而就寒,日負影以偕沒,月媚景於雲端。鳥悽聲以孤歸,獸索偶而不還,悼當年之晚暮,恨茲歲之欲殫。思宵夢以從之,神飄颻而不安,若憑舟之失棹,譬緣崖而無攀。}

\textbf{于時畢昴盈軒,北風淒淒,耿耿不寐,眾念徘徊,起攝帶以伺晨,繁霜粲於素階。雞斂翅而未鳴,笛流遠以清哀,始妙密以閑和,終寥亮而藏摧,意夫人之在茲,託行雲以送懷。}

\textbf{行雲逝而無語,時奄冉而就過,徒勤思以自悲,終阻山而滯河。迎清風以祛累,寄弱志於歸波,尤蔓草之為會,誦邵南之餘歌,坦萬慮以存誠,憩遙情於八遐\footnote{八遐:八方。}。}

\section{歸去來兮辭\hspace{1ex}{\footnotesize 并序}}

\begin{quoting}余家貧,耕植不足以自給,幼稚盈室,缾無儲粟,生生所資,未見其術,親故多勸余為長吏,脫然有懷,求之靡途。會有四方之事,諸侯以惠愛為德,家叔以余貧苦,遂見用為小邑,于時風波未靜,心憚遠役,彭澤去家百里,公田之利足以為酒,故便求之。及少日,眷然有歸歟之情,何則?質性自然,非矯勵所得,飢凍雖切,違己交病,嘗從人事,皆口腹自役,於是悵然慷慨,深愧平生之志,猶望一稔當斂裳宵逝,尋程氏妹喪于武昌,情在駿奔,自免去職,仲秋至冬,在官八十餘日,因事順心,命篇曰歸去來兮,乙巳歲十一月也。\end{quoting}

\textbf{歸去來兮,田園將蕪胡不歸?既自以心為形役,奚惆悵而獨悲,悟已往之不諫,知來者之可追,實迷途其未遠,覺今是而昨非。}

\textbf{舟遙遙以輕颺,風飄飄而吹衣,問征夫以前路,恨晨光之熹微。乃瞻衡宇,載欣載奔,僮僕歡迎,稚子候門。三徑就荒,松菊猶存,攜幼入室,有酒盈罇。}

\textbf{引壺觴以自酌,眄庭柯以怡顏,倚南窗以寄傲,審容膝\footnote{韓詩外傳:北郭先生妻曰「今結駟列騎,所安不過容膝」。}之易安,園日涉以成趣\footnote{爾雅:堂上謂之行,堂下謂之步,門外謂之趨,中庭謂之走。郭注:此皆行步趨走之處,因以名。},門雖設而常關,策扶老以流憩,時矯首而遐觀,雲無心以出岫,鳥倦飛而知還,景翳翳以將入,撫孤松而盤桓。}

\textbf{歸去來兮,請息交以絕遊,世與我而相遺,復駕言\footnote{詩邶風泉水:駕言出遊。}兮焉求,悅親戚之情話,樂琴書以消憂,農人告余以春及,將有事於西疇,或命巾車,或棹孤舟,既窈窕以尋壑,亦崎嶇而經丘,木欣欣以向榮,泉涓涓而始流,善萬物之得時,感吾生之行休。}

\textbf{已矣乎!寓形宇內復幾時,曷不委心任去留,胡為乎遑遑欲何之,富貴非吾願,帝鄉不可期。懷良辰以孤往,或植杖而耘耔,登東皐以舒嘯,臨清流而賦詩,聊乘化以歸盡,樂夫天命復奚疑。}

