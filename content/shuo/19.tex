\chapter{賢媛第十九}

\subsection*{1}

\textbf{陳嬰者,東陽人,少脩德行,著稱鄉黨,秦末大亂,東陽人欲奉嬰為主,母曰:「不可!自我為汝家婦,少見貧賤,一旦富貴,不祥,不如以兵屬人,事成,少受其利,不成,禍有所歸。」}{\footnotesize \textbf{史記}曰嬰故東陽令史,居縣素信,為長者,東陽人欲立長,乃請嬰,嬰母諫之,乃以兵屬項梁,梁以嬰為上柱國。}

\subsection*{2}

\textbf{漢元帝宮人既多,乃令畫工圖之,欲有呼者,輒披圖召之,其中常者,皆行貨賂,王明君姿容甚麗,志不苟求,工遂毀為其狀,後匈奴來和,求美女於漢帝,帝以明君充行,既召見而惜之,但名字已去,不欲中改,於是遂行。}{\footnotesize \textbf{漢書匈奴傳}曰竟寧元年,呼韓邪單于來朝,自言願壻漢氏以自親,元帝以後宮良家子王嬙字明君賜之,單于懽喜,上書願保塞。\textbf{文穎}曰昭君本蜀郡秭歸人也。\textbf{琴操}曰王昭君者,齊國王穰女也,年十七,儀形絕麗,以節聞國中,長者求之者,王皆不許,乃獻漢元帝,帝造次不能別房帷,昭君恚怒之,會單于遣使,帝令宮人裝出,使者請一女,帝乃謂宮中曰「欲至單于者起」,昭君喟然越席而起,帝視之,大驚悔,是時使者並見,不得止,乃賜單于,單于大說,獻諸珍物,昭君有子曰世違,單于死,世違繼立,凡為胡者,父死妻母,昭君問世違曰「汝為漢也、為胡也」,世違曰「欲為胡耳」,昭君乃吞藥自殺。\textbf{石季倫}曰昭以觸文帝諱,故改為明。}

\subsection*{3}

\textbf{漢成帝幸趙飛燕,飛燕讒班婕妤祝詛,於是考問,辭曰:「妾聞死生有命,富貴在天,脩善尚不蒙福,為邪欲以何望?若鬼神有知,不受邪佞之訴,若其無知,訴之何益?故不為也。」}{\footnotesize \textbf{漢書外戚傳}曰成帝趙皇后,本長安宮人,初生,父母不舉,三日不死,乃收養之,及壯,屬陽阿主家學歌舞,號曰飛燕,帝微行過主,見而說之,召入宮,大得幸,立為后。班婕妤者,雁門人,成帝初選入宮,大得幸,立為婕妤,帝遊後庭,嘗欲與同輦,婕妤辭之,趙飛燕譖許皇后及婕妤,婕妤對有辭致,上憐之,賜黃金百斤,飛燕嬌妒,婕妤恐見危中,求供養太后於長信宮,帝崩,婕妤充奉園陵,薨,葬園中。}

\subsection*{4}

\textbf{魏武帝崩,文帝悉取武帝宮人自侍,及帝病困,卞后出看疾,太后入戶,見直侍並是昔日所愛幸者,太后問:「何時來邪?」云:「正伏魄時過。」因不復前而歎曰:「狗鼠不食汝餘,死故應爾。」至山陵,亦竟不臨。}{\footnotesize \textbf{魏書}曰武宣卞皇后,琅邪開陽人,以漢延熹三年生齊郡白亭,有黃氣滿室移日,父敬侯怪之,以問卜者王越,越曰「此吉祥也」,年二十,太祖納於譙,性約儉,不尚華麗,有母儀德行。}

\subsection*{5}

\textbf{趙母嫁女,女臨去,敕之曰:「慎勿為好。」女曰:「不為好,可為惡邪?」母曰:「好尚不可為,其況惡乎?」}{\footnotesize \textbf{列女傳}曰趙姬者,桐鄉令東郡虞韙妻,潁川趙氏女也,才敏多覽,韙既沒,文皇帝敬其文才,詔入宮省,上欲自征公孫淵,姬上疏以諫,作列女傳解,號趙母注,賦數十萬言,赤烏六年卒。\textbf{淮南子}曰人有嫁其女而教之者,曰「爾為善,善人疾之」,對曰「然則當為不善乎」,曰「善尚不可為,而況不善乎」。\textbf{景獻羊皇后}曰此言雖鄙,可以命世人。}

\subsection*{6}

\textbf{許允婦是阮衛尉女、德如妹,}{\footnotesize \textbf{魏略}曰允,字士宗,高陽人,少與清河崔贊俱發名於冀州,仕至領軍將軍。\textbf{陳留志名}曰阮共,字伯彥,尉氏人,清真守道,動以禮讓,仕魏,至衛尉卿,少子侃,字德如,有俊才,而飭以名理,風儀雅潤,與嵇康為友,仕至河內太守。}\textbf{奇醜,交禮竟,允無復入理,家人深以為憂,會允有客至,婦令婢視之,還,答曰:「是桓郎。」桓郎者,桓範也,}{\footnotesize \textbf{魏略}曰範,字允明,沛郡人,仕至大司農,為宣王所誅。}\textbf{婦云:「無憂,桓必勸入。」桓果語許云:「阮家既嫁醜女與卿,故當有意,卿宜察之。」許便回入內,既見婦,即欲出,婦料其此出無復入理,便捉裾停之,許因謂曰:「婦有四德,卿有其幾?」}{\footnotesize \textbf{周禮}九嬪掌婦學之法,以教九御,婦德、婦言、婦容、婦功。\textbf{鄭注}曰德謂貞順,言謂辭令,容謂婉娩,功謂絲枲。}\textbf{婦曰:「新婦所乏唯容爾,然士有百行,君有幾?」許云:「皆備。」婦曰:「夫百行以德為首,君好色不好德,何謂皆備?」允有慚色,遂相敬重。}

\subsection*{7}

\textbf{許允為吏部郎,多用其鄉里,魏明帝遣虎賁收之,其婦出誡允曰:「明主可以理奪,難以情求。」既至,帝覈問之,允對曰:「『舉爾所知』,臣之鄉人,臣所知也,陛下檢校為稱職與不,若不稱職,臣受其罪。」既檢校,皆官得其人,於是乃釋,允衣服敗壞,詔賜新衣。初,允被收,舉家號哭,阮新婦自若,云:「勿憂,尋還。」作粟粥待,頃之,允至。}{\footnotesize \textbf{魏氏春秋}曰初,允為吏部,選遷郡守,明帝疑其所用非次,將加其罪,允妻阮氏跣出,謂曰「明主可以理奪,不可以情求」,允頷之而入,帝怒詰之,允對曰「某郡太守雖限滿,文書先至,年限在後,日限在前」,帝前取事視之,乃釋然,遣出,望其衣敗,曰「清吏也」。}

\subsection*{8}

\textbf{許允為晉景王所誅,門生走入告其婦,婦正在機中,神色不變,曰:「蚤知爾耳。」}{\footnotesize \textbf{魏志}曰初,領軍與夏侯玄、李豐親善,有詐作尺一詔書,以玄為大將軍,允為太尉,共錄尚書事,無何,有人天未明乘馬以詔版付允門吏,曰「有詔」,因便驅走,允投書燒之,不以關呈景王。\textbf{魏略}曰明年,李豐被收,允欲往見大將軍,已出門,允回遑不定,中道還取袴,大將軍聞而怪之曰「我自收李豐,士大夫何為悤悤乎」,會鎮北將軍劉靜卒,以允代靜,大將軍與允書曰「鎮北雖少事,而都典一方,念足下震華鼓,建朱節,歷本州,此所謂著繡晝行也」,會有司奏允前擅以廚錢穀,乞諸俳及其官屬,減死徙邊,道死。\textbf{魏氏春秋}曰允之為鎮北,喜謂其妻曰「吾知免矣」,妻曰「禍見於此,何免之有」。\textbf{晉諸公贊}曰允有正情,與文帝不平,遂幽殺之。\textbf{婦人集}載阮氏與允書,陳允禍患所起,辭甚酸愴,文多不錄。}\textbf{門人欲藏其兒,婦曰:「無豫諸兒事。」後徙居墓所,景王遣鍾會看之,若才流及父,當收,兒以咨母,母曰:「汝等雖佳,才具不多,率胸懷與語,便無所憂,不須極哀,會止便止,又可少問朝事。」兒從之,會反以狀對,卒免。}{\footnotesize \textbf{世語}曰允二子,奇,字子太,猛,字子豹,並有治理。\textbf{晉諸公贊}曰奇,泰始中為太常丞,世祖嘗祠廟,奇應行事,朝廷以奇受害之門,不令接近,出為長史,世祖下詔,述允宿望,又稱奇才,擢為尚書祠部郎,猛禮學儒博,加有才識,為幽州刺史。}

\subsection*{9}

\textbf{王公淵娶諸葛誕女,入室,言語始交,王謂婦曰:「新婦神色卑下,殊不似公休。」婦曰:「大丈夫不能仿佛彥雲,而令婦人比蹤英傑。」}{\footnotesize \textbf{魏氏春秋}曰王廣,字公淵,王淩子也,有風量才學,名重當世,與傅嘏等論才性同異,行於世。\textbf{魏志}曰廣有志尚學行,淩誅,并死。臣謂王廣名士,豈以妻父為戲,此言非也。}

\subsection*{10}

\textbf{王經少貧苦,仕至二千石,母語之曰:「汝本寒家子,仕至二千石,此可以止乎?」經不能用,為尚書,助魏,不忠於晉,被收,涕泣辭母曰:「不從母敕,以至今日。」母都無慽容,語之曰:「為子則孝,為臣則忠,有孝有忠,何負吾邪?」}{\footnotesize \textbf{世語}曰經,字彥偉,清河人,高貴鄉公之難,王沈、王業馳告文王,經以正直不出,因沈、業申意,後誅經及其母。\textbf{晉諸公贊}曰沈、業將出,呼經,不從,曰「吾子行矣」。\textbf{漢晉春秋}曰初,曹髦將自討司馬昭,經諫曰「昔魯昭不忍季氏,敗走失國,為天下笑,今權在其門久矣,朝廷四方皆為之致死,不顧逆順之理,非一日也,且宿衛空闕,寸刃無有,陛下何所資用?而一旦如此,無乃欲除疾而更深之邪」,髦不聽,後殺經,并及其母,將死,垂泣謝母,母顏色不變,笑而謂曰「人誰不死,往所以止汝者,恐不得其所也,以此并命,何恨之有」。\textbf{干寶晉紀}曰經正直,不忠於我,故誅之。\textbf{按}傅暢、干寶所記,則是經實忠貞於魏,而世語既謂其正直,復云因沈、業申意,何其相反乎?故二家之言深得之。}

\subsection*{11}

\textbf{山公與嵇、阮一面,契若金蘭,山妻韓氏,覺公與二人異於常交,問公,公曰:「我當年可以為友者,唯此二生耳。」妻曰:「負羈之妻亦親觀狐、趙,意欲窺之,可乎?」他日,二人來,妻勸公止之宿,具酒肉,夜穿墉以視之,達旦忘反,公入曰:「二人何如?」妻曰:「君才致殊不如,正當以識度相友耳。」公曰:「伊輩亦常以我度為勝。」}{\footnotesize \textbf{晉陽秋}曰濤雅素恢達,度量弘遠,心存事外,而與時俛仰,嘗與阮籍、嵇康諸人著忘言之契,至于群子,屯蹇於世,濤獨保浩然之度。\textbf{王隱晉書}曰韓氏有才識,濤未仕時,戲之曰「忍寒,我當作三公,不知卿堪為夫人不耳」。}

\subsection*{12}

\textbf{王渾妻鍾氏生女令淑,}{\footnotesize \textbf{虞預晉書}曰渾,字玄沖,太原晉陽人,魏司徒昶子,仕至司徒。}\textbf{武子為妹求簡美對而未得,有兵家子,有儁才,欲以妹妻之,乃白母,}{\footnotesize \textbf{王氏譜}曰鍾夫人名琰之,太傅繇之孫。}\textbf{曰:「誠是才者,其地可遺,然要令我見。」武子乃令兵兒與群小雜處,使母帷中察之,既而,母謂武子曰:「如此衣形者,是汝所擬者非邪?」武子曰:「是也。」母曰:「此才足以拔萃,然地寒,不有長年,不得申其才用,觀其形骨,必不壽,不可與婚。」武子從之,兵兒數年果亡。}

\subsection*{13}

\textbf{賈充前婦是李豐女,豐被誅,離婚徙邊,}{\footnotesize \textbf{婦人集}曰充妻李氏,名婉,字淑文,豐誅,徙樂浪。}\textbf{後遇赦得還,充先已取郭配女,}{\footnotesize \textbf{賈氏譜}曰郭氏名玉璜,即廣宣君也。}\textbf{武帝特聽置左右夫人,李氏別住外,不肯還充舍,}{\footnotesize \textbf{晉諸公贊}曰世祖踐阼,李氏赦還,而齊獻王妃欲令充遣郭氏,更納其母,充不許,為李氏築宅而不往來,充母柳氏將亡,充問所欲言者,柳曰「我教汝迎李新婦尚不肯,安問他事」。}\textbf{郭氏語充:「欲就省李。」充曰:「彼剛介有才氣,卿往不如不去。」}{\footnotesize \textbf{充別傳}曰李氏有淑性令才也。}\textbf{郭氏於是盛威儀,多將侍婢,既至,入戶,李氏起迎,郭不覺腳自屈,因跪再拜,既反,語充,充曰:「語卿道何物?」}{\footnotesize \textbf{按}晉諸公贊曰「世祖以李豐得罪晉室,又郭氏是太子妃母,無離絕之理,乃下詔勅斷,不得往還」,而王隱晉書亦云「充既與李絕婚,更取城陽太守郭配女,名槐,李禁錮解,詔充置左右夫人,充母柳亦勅充迎李,槐怒,攘臂責充曰『刊定律令,為佐命之功,我有其分,李那得與我並』,充乃架屋永年里中以安李,槐晚乃知,充出,輒使人尋充,詔許充置左右夫人,充答詔以謙讓不敢當盛禮」,晉贊既云世祖下詔不遣李還,而王隱晉書及充別傳並言詔聽置立左右夫人,充憚郭氏,不敢迎李,三家之說並不同,未詳孰是,然李氏不還,別有餘故,而世說云自不肯還,謬矣,且郭槐彊狠,豈能就李而為之拜乎?皆為虛也。}

\subsection*{14}

\textbf{賈充妻李氏作女訓,行於世,李氏女,齊獻王妃,郭氏女,惠帝后,充卒,李、郭女各欲令其母合葬,經年不決,賈后廢,李氏乃祔葬,遂定。}{\footnotesize \textbf{晉諸公贊}曰李氏有才德,世稱「李夫人訓」者,生女合,亦才明,即齊王妃。\textbf{婦人集}曰李氏至樂浪,遺二女典式八篇。\textbf{王隱晉書}曰賈后,字南風,為趙王所誅。}

\subsection*{15}

\textbf{王汝南少無婚,自求郝普女,}{\footnotesize \textbf{郝氏譜}曰普,字道匡,太原襄城人,仕至洛陽太守。}\textbf{司空以其癡,會無婚處,任其意,便許之,}{\footnotesize \textbf{魏氏志}曰王昶,字文舒,仕至司空。}\textbf{既婚,果有令姿淑德,生東海,遂為王氏母儀,或問汝南:「何以知之?」曰:「嘗見井上取水,舉動容止不失常,未嘗忤觀,以此知之。」}{\footnotesize \textbf{汝南別傳}曰襄城郝仲將,門至孤陋,非其所偶也,君嘗見其女,便求聘焉,果高朗英邁,母儀冠族,其通識餘裕,皆此類。}

\subsection*{16}

\textbf{王司徒婦,鍾氏女、太傅曾孫,}{\footnotesize \textbf{王氏譜}曰夫人,黃門侍郎鍾琰女。}\textbf{亦有俊才女德,}{\footnotesize \textbf{婦人集}曰夫人有文才,其詩賦頌誄行於世。}\textbf{鍾、郝為娣姒,雅相親重,鍾不以貴陵郝,郝亦不以賤下鍾,東海家內,則郝夫人之法,京陵家內,範鍾夫人之禮。}

\subsection*{17}

\textbf{李平陽,秦州子,}{\footnotesize 李重已見。\textbf{永嘉流人名}曰康,字玄冑,江夏人,魏秦州刺史。}\textbf{中夏名士,于時以比王夷甫,孫秀初欲立威權,咸云:「樂令民望不可殺,減李重者又不足殺。」}{\footnotesize \textbf{晉諸公贊}曰孫秀,字俊忠,琅邪人,初,趙王倫封琅邪,秀給為近職小吏,倫數使秀作書疏,文才稱倫意,倫封趙,秀徙戶為趙人,用為侍郎,信任之。\textbf{晉陽秋}曰倫篡位,秀為中書令,事皆決於秀,為齊王所誅。}\textbf{遂逼重自裁,初,重在家,有人走從門入,出髻中疏示重,重看之色動,入內示其女,女直叫「絕」,了其意,出則自裁。}{\footnotesize \textbf{按}諸書皆云「重知趙王倫作亂,有疾不治,遂以致卒」,而此書乃言自裁,甚乖謬,且倫、秀兇虐,動加誅夷,欲立威權,自當顯戮,何為逼令自裁?}\textbf{此女甚高明,重每咨焉。}

\subsection*{18}

\textbf{周浚作安東時,行獵,值暴雨,過汝南李氏,李氏富足,而男子不在,有女名絡秀,聞外有貴人,與一婢於內宰豬羊,作數十人飲食,事事精辦,不聞有人聲,密覘之,獨見一女子,狀貌非常,浚因求為妾,父兄不許,絡秀曰:「門戶殄瘁,何惜一女?若連姻貴族,將來或大益。」父兄從之,}{\footnotesize \textbf{八王故事}曰浚,字開林,汝南安城人,少有才名,太康初,平吳,自御史中丞出為揚州刺史,元康初,加安東將軍。}\textbf{遂生伯仁兄弟,絡秀語伯仁等:「我所以屈節為汝家作妾,門戶計耳,}{\footnotesize \textbf{按}周氏譜「浚取同郡李伯宗女」,此云為妾,妄耳。}\textbf{汝若不與吾家作親親者,吾亦不惜餘年。」伯仁等悉從命,由此李氏在世,得方幅齒遇。}

\subsection*{19}

\textbf{陶公少有大志,家酷貧,與母湛氏同居,同郡范逵素知名,舉孝廉,}{\footnotesize 逵未詳。}\textbf{投侃宿,于時冰雪積日,侃室如懸磬,而逵馬僕甚多,侃母湛氏語侃曰:「汝但出外留客,吾自為計。」湛頭髮委地,下為二髲,}{\footnotesize 一作髢。}\textbf{賣得數斛米,斫諸屋柱,悉割半為薪,剉諸薦以為馬草,日夕,遂設精食,從者皆無所乏,逵既歎其才辯,又深愧其厚意,明旦去,侃追送不已,且百里許,逵曰:「路已遠,君宜還。」侃猶不返,逵曰:「卿可去矣,至洛陽,當相為美談。」侃迺返,逵及洛,遂稱之於羊晫、顧榮諸人,大獲美譽。}{\footnotesize \textbf{晉陽秋}曰侃父丹,娶新淦湛氏女,生侃,湛虔恭有智算,以陶氏貧賤,紡績以資給侃,使交結勝己,侃少為尋陽吏,鄱陽孝廉范逵嘗過侃宿,時大雪,侃家無草,湛徹所臥薦剉給,陰截髮,賣以供調,逵聞之歎息,逵去,侃追送之,逵曰「豈欲仕乎」,侃曰「有仕郡意」,逵曰「當相談致」,過廬江,向太守張夔稱之,召補吏,舉孝廉,除郎中,時豫章顧榮或責羊晫曰「君奈何與小人同輿」,晫曰「此寒俊也」。\textbf{王隱晉書}曰侃母既截髮供客,聞者歎曰「非此母不生此子」,乃進之於張夔,羊晫亦簡之,後晫為十郡中正,舉侃為鄱陽小中正,始得上品也。}

\subsection*{20}

\textbf{陶公少時,作魚梁吏,嘗以坩鮓餉母,母封鮓付使,反書責侃曰:「汝為吏,以官物見餉,非唯不益,乃增吾憂也。」}{\footnotesize \textbf{侃別傳}曰母湛氏,賢明有法訓,侃在武昌,與佐吏從容飲燕,常有飲限,或勸猶可少進,侃悽然良久曰「昔年少,曾有酒失,二親見約,故不敢踰限」,及侃丁母憂,在墓下,忽有二客來弔,不哭而退,儀服鮮異,知非常人,遣隨視之,但見雙鶴沖天而去。\textbf{幽明錄}曰陶公在尋陽西南一塞取魚,自謂其池曰鶴門。\textbf{按}吳司徒孟宗為雷池監,以鮓餉母,母不受,非侃也,疑後人因孟假為此說。}

\subsection*{21}

\textbf{桓宣武平蜀,以李勢妹為妾,甚有寵,常著齋後,主始不知,既聞,與數十婢拔白刃襲之,}{\footnotesize \textbf{續晉陽秋}曰溫尚明帝女南康長公主。}\textbf{正值李梳頭,髮委藉地,膚色玉曜,不為動容,徐曰:「國破家亡,無心至此,今日若能見殺,乃是本懷。」主慚而退。}{\footnotesize \textbf{妒記}曰溫平蜀,以李勢女為妾,郡主兇妒,不即知之,後知,乃拔刃往李所,因欲斫之,見李在窗梳頭,姿貌端麗,徐徐結髮,斂手向主,神色閑正,辭甚悽惋,主於是擲刀前,抱之曰「阿子!我見汝亦憐,何況老奴」,遂善之。}

\subsection*{22}

\textbf{庾玉臺,希之弟也,希誅,將戮玉臺,}{\footnotesize 希已見。玉臺,庾友小字。\textbf{庾氏譜}曰友,字惠彥,司空冰第三子,歷中書郎、東陽太守。}\textbf{玉臺子婦,宣武弟桓豁女也,}{\footnotesize \textbf{庾氏譜}曰友,字弘之,長子宣,娶宣武弟桓豁之女,字女幼。}\textbf{徒跣求進,閽禁不內,女厲聲曰:「是何小人?我伯父門,不聽我前。」因突入,號泣請曰:「庾玉臺常因人,腳短三寸,當復能作賊不?」宣武笑曰:「壻故自急。」遂原玉臺一門。}{\footnotesize \textbf{中興書}曰桓溫殺庾希弟倩,希聞難而逃,希弟友當伏誅,子婦桓氏女,請溫,得宥。}

\subsection*{23}

\textbf{謝公夫人幃諸婢,使在前作伎,使太傅暫見,便下幃,太傅索更開,夫人云:「恐傷盛德。」}{\footnotesize 劉夫人已見。}

\subsection*{24}

\textbf{桓車騎不好著新衣,浴後,婦故送新衣與,}{\footnotesize \textbf{桓氏譜}曰沖娶琅耶王恬女,字女宗。}\textbf{車騎大怒,催使持去,婦更持還,傳語云:「衣不經新,何由而故?」桓公大笑,著之。}

\subsection*{25}

\textbf{王右軍郗夫人謂二弟司空、中郎曰:}{\footnotesize 司空,愔,已見。\textbf{郗曇別傳}曰曇,字重熙,鑒少子,性韻方質,和正沈簡,累遷丹陽尹、北中郎將、徐兗二州刺史。}\textbf{「王家見二謝,傾筐倒庋,}{\footnotesize 二謝,安、萬。}\textbf{見汝輩來,平平爾,汝可無煩復往。」}

\subsection*{26}

\textbf{王凝之謝夫人既往王氏,大薄凝之,既還謝家,意大不說,太傅慰釋之曰:「王郎,逸少之子,人材亦不惡,汝何以恨乃爾?」答曰:「一門叔父,則有阿大、中郎,群從兄弟,則有封、胡、遏、末,}{\footnotesize 封胡,謝韶小字,遏末,謝淵小字。韶,字穆度,萬子,車騎司馬。淵,字叔度,奕第二子,義興太守,時人稱其尤彥秀者。或曰封、胡、遏、末,封謂韶、胡謂朗、遏謂玄、末謂淵,一作胡謂淵、遏謂玄、末謂韶也。}\textbf{不意天壤之中,乃有王郎。」}

\subsection*{27}

\textbf{韓康伯母,隱古几毀壞,卞鞠見几惡,欲易之,}{\footnotesize 鞠,卞範之,母之外孫也。}\textbf{答曰:「我若不隱此,汝何以得見古物?」}

\subsection*{28}

\textbf{王江州夫人語謝遏曰:「汝何以都不復進,}{\footnotesize 夫人,玄之妹。}\textbf{為是塵務經心,天分有限?」}

\subsection*{29}

\textbf{郗嘉賓喪,婦兄弟欲迎妹還,終不肯歸,}{\footnotesize \textbf{郗氏譜}曰超娶汝南周閔女,名馬頭。}\textbf{曰:「生縱不得與郗郎同室,死寧不同穴?」}{\footnotesize \textbf{毛詩}曰穀則異室,死則同穴。\textbf{鄭玄}注曰穴,謂壙中墟也。}

\subsection*{30}

\textbf{謝遏絕重其姊,張玄常稱其妹,欲以敵之,有濟尼者,並遊張、謝二家,人問其優劣,答曰:「王夫人神情散朗,故有林下風氣,顧家婦清心玉映,自是閨房之秀。」}

\subsection*{31}

\textbf{王尚書惠嘗看王右軍夫人,}{\footnotesize \textbf{宋書}曰惠,字令明,瑯邪人,歷吏部尚書,贈太常卿。}\textbf{問:「眼耳未覺惡不?」}{\footnotesize \textbf{婦人集}載謝表曰「妾年九十,孤骸獨存,願蒙哀矜,賜其鞠養」。}\textbf{答曰:「髮白齒落,屬乎形骸,至於眼耳,關於神明,那可便與人隔?」}

\subsection*{32}

\textbf{韓康伯母殷,隨孫繪之之衡陽,}{\footnotesize \textbf{韓氏譜}曰繪之,字季倫,父康伯,太常卿,繪之仕至衡陽太守。}\textbf{於闔廬洲中逢桓南郡,卞鞠是其外孫,時來問訊,謂鞠曰:「我不死,見此豎二世作賊。」在衡陽數年,繪之遇桓景真之難也,}{\footnotesize \textbf{續晉陽秋}曰桓亮,字景真,大司馬溫之孫,父濟,給事中,叔父玄,篡逆見誅,亮聚眾於長沙,自號湘州刺史,殺太宰甄恭、衡陽前太守韓繪之等十餘人,為劉毅軍人郭珍斬之。}\textbf{殷撫屍哭曰:「汝父昔罷豫章,徵書朝至夕發,汝去郡邑數年,為物不得動,遂及於難,夫復何言?」}