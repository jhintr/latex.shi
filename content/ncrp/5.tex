\chapter{Lesson 5}

\section*{Reading 1}

“Jāneyya nu kho, bho Gotama, asappuriso asappurisaṃ ‘asappuriso ayaṃ bhavaṃ’” ti?

“Aṭṭhānaṃ kho etaṃ, brāhmaṇa, anavakāso yaṃ asappuriso asappurisaṃ jāneyya ‘asappuriso ayaṃ bhavaṃ’” ti.

“Jāneyya pana, bho Gotama, asappuriso sappurisaṃ ‘sappuriso ayaṃ bhavaṃ’” ti?

“Etam pi kho, brāhmaṇa, aṭṭhānaṃ anavakāso yaṃ asappuriso sappurisaṃ jāneyya ‘sappuriso ayaṃ bhavaṃ’” ti.

“Jāneyya nu kho, bho Gotama, sappuriso sappurisaṃ ‘sappuriso ayaṃ bhavaṃ’” ti?

“Ṭhānaṃ kho etaṃ, brāhmaṇa, vijjati yaṃ sappuriso sappurisaṃ jāneyya ‘sappuriso ayaṃ bhavaṃ’” ti.

“Jāneyya pana, bho Gotama, sappuriso asappurisaṃ ‘asappuriso ayaṃ bhavaṃ’” ti?

“Etam pi kho, brāhmaṇa, ṭhānaṃ vijjati yaṃ sappuriso asappurisaṃ jāneyya ‘asappuriso ayaṃ bhavaṃ’” ti. (A 4.19.7)

\section*{Reading 2}

Yo hi koci manussesu,\\
gorakkhaṃ upajīvati,\\
evaṃ, Vāseṭṭha, jānāhi,\\
‘kassako’ so, na brāhmaṇo.

Yo hi koci manussesu,\\
puthusippena jīvati,\\
evaṃ, Vāseṭṭha, jānāhi,\\
‘sippiko’ so, na brāhmaṇo.

Yo hi koci manussesu,\\
vohāraṃ upajīvati,\\
evaṃ, Vāseṭṭha, jānāhi,\\
‘vāṇijo’ so, na brāhmaṇo.

Yo hi koci manussesu,\\
parapessena jīvati,\\
evaṃ, Vāseṭṭha, jānāhi,\\
‘pessiko’ so, na brāhmaṇo.

Yo hi koci manussesu,\\
adinnaṃ upajīvati,\\
evaṃ, Vāseṭṭha, jānāhi,\\
‘coro’ eso, na brāhmaṇo.

Yo hi koci manussesu,\\
issatthaṃ upajīvati,\\
evaṃ, Vāseṭṭha, jānāhi,\\
‘yodhājīvo’, na brāhmaṇo.

Yo hi koci manussesu,\\
porohiccena jīvati,\\
evaṃ, Vāseṭṭha, jānāhi,\\
‘yājako’ so, na brāhmaṇo.

Yo hi koci manussesu,\\
gāmaṃ raṭṭhañ ca bhuñjati,\\
evaṃ, Vāseṭṭha, jānāhi,\\
‘rājā’ eso, na brāhmaṇo.

Na cāhaṃ ‘brāhmaṇaṃ’ brūmi,\\
yonijaṃ mattisambhavaṃ,\\
‘Bhovādi’ nāma so hoti,\\
sace hoti sakiñcano,\\
akiñcanaṃ anādānaṃ,\\
tam ahaṃ brūmi ‘brāhmaṇaṃ’.

Sabbasaṃyojanaṃ chetvā,\\
so ve na paritassati,\\
saṅgātigaṃ, visaṃyuttaṃ,\\
tam ahaṃ brūmi ‘brāhmaṇaṃ’. (Sn 3.9)

\section*{Reading 3}

Appamādena maghavā,\\
devānaṃ seṭṭhataṃ gato,\\
appamādaṃ pasaṃsanti,\\
pamādo garahito sadā. (Dhp 2)

Yathā pi ruciraṃ pupphaṃ,\\
vaṇṇavantaṃ agandhakaṃ,\\
evaṃ subhāsitā vācā,\\
aphalā hoti akubbato.

Yathā pi ruciraṃ pupphaṃ,\\
vaṇṇavantaṃ sugandhakaṃ,\\
evaṃ subhāsitā vācā,\\
saphalā hoti kubbato. (Dhp 4)

Dīghā jāgarato ratti,\\
dīghaṃ santassa yojanaṃ,\\
dīgho bālānaṃ saṃsāro,\\
saddhammaṃ avijānataṃ. (Dhp 5)

\section*{Reading 4}

Asevanā ca bālānaṃ,\\
paṇḍitānañ ca sevanā,\\
pūjā ca pūjanīyānaṃ,\\
etaṃ maṅgalamuttamaṃ.

Bāhusaccaṃ ca sippañ ca,\\
vinayo ca susikkhito,\\
subhāsitā ca yā vācā,\\
etaṃ maṅgalamuttamaṃ.

Dānañ ca dhammacariyā ca,\\
ñātakānaṃ ca saṅgaho,\\
anavajjāni kammāni,\\
etaṃ maṅgalamuttamaṃ.

Ārati virati pāpā,\\
majjapānā ca saṃyamo,\\
appamādo ca dhammesu,\\
etaṃ maṅgalamuttamaṃ.

Gāravo ca nivāto ca,\\
santuṭṭhi ca kataññutā,\\
kālena dhammasavanaṃ,\\
etaṃ maṅgalamuttamaṃ.

Khantī ca sovacassatā,\\
samaṇānañ ca dassanaṃ,\\
kālena dhammasākacchā,\\
etaṃ maṅgalamuttamaṃ. (Sn 2.4 Maṅgalasuttaṃ)

\section*{Further Reading 1}

Chahi, bhikkhave, dhammehi samannāgato bhikkhu āhuneyyo hoti pāhuneyyo dakkhiṇeyyo añjalikaraṇīyo anuttaraṃ puññakkhettaṃ lokassa.

Katamehi chahi?

Idha, bhikkhave, bhikkhu cakkhunā rūpaṃ disvā n’eva sumano hoti na dummano, upekkhako viharati sato sampajāno.

Sotena saddaṃ sutvā n’eva sumano hoti na dummano, upekkhako viharati sato sampajāno.

Ghānena gandhaṃ ghāyitvā n’eva sumano hoti na dummano, upekkhako viharati sato sampajāno.

Jivhāya rasaṃ sāyitvā n’eva sumano hoti na dummano, upekkhako viharati sato sampajāno.

Kāyena phoṭṭhabbaṃ phusitvā n’eva sumano hoti na dummano, upekkhako viharati sato sampajāno.

Manasā dhammaṃ viññāya n’eva sumano hoti na dummano, upekkhako viharati sato sampajāno.

Imehi kho, bhikkhave, chahi dhammehi samannāgato bhikkhu āhuneyyo hoti pāhuneyyo dakkhiṇeyyo añjalikaraṇīyo anuttaraṃ puññakkhettaṃ lokassā ti. (A 6.1.1.1 Paṭhamaāhuneyyasuttaṃ)

\section*{Further Reading 2}

“Tena hi, Sīvaka, taññev’ettha paṭipucchāmi. Yathā te khameyya tathā naṃ byākareyyāsi. Taṃ kiṃ maññasi, Sīvaka, santaṃ vā ajjhattaṃ lobhaṃ ‘atthi me ajjhattaṃ lobho’ ti pajānāsi, asantaṃ vā ajjhattaṃ lobhaṃ ‘natthi me ajjhattaṃ lobho’ ti pajānāsī” ti?

“Evaṃ, bhante.”

“Yaṃ kho tvaṃ, Sīvaka, santaṃ vā ajjhattaṃ lobhaṃ ‘atthi me ajjhattaṃ lobho’ ti pajānāsi, asantaṃ vā ajjhattaṃ lobhaṃ ‘natthi me ajjhattaṃ lobho’ ti pajānāsi, evam pi kho, Sīvaka, sandiṭṭhiko dhammo hoti… pe….

“Taṃ kiṃ maññasi, Sīvaka, santaṃ vā ajjhattaṃ dosaṃ… pe… santaṃ vā ajjhattaṃ mohaṃ… pe… santaṃ vā ajjhattaṃ lobhadhammaṃ… pe… santaṃ vā ajjhattaṃ dosadhammaṃ… pe… santaṃ vā ajjhattaṃ mohadhammaṃ ‘atthi me ajjhattaṃ mohadhammo’ ti pajānāsi, asantaṃ vā ajjhattaṃ mohadhammaṃ ‘natthi me ajjhattaṃ mohadhammo’ ti pajānāsī” ti?

“Evaṃ, bhante.”

“Yaṃ kho tvaṃ, Sīvaka, santaṃ vā ajjhattaṃ mohadhammaṃ ‘atthi me ajjhattaṃ mohadhammo’ ti pajānāsi, asantaṃ vā ajjhattaṃ mohadhammaṃ ‘natthi me ajjhattaṃ mohadhammo’ ti pajānāsi, evaṃ kho, Sīvaka, sandiṭṭhiko dhammo hoti.”

“Abhikkantaṃ, bhante, abhikkantaṃ, bhante… pe… upāsakaṃ maṃ, bhante, bhagavā dhāretu ajjatagge pāṇ’upetaṃ saraṇaṃ gatan” ti. (A 6.5.5 Paṭhamasandiṭṭhikasuttaṃ)

\section*{Further Reading 3}

Rājā āha “Bhante Nāgasena, yo jānanto pāpakammaṃ karoti, yo ajānanto pāpakammaṃ karoti, kassa bahutaraṃ apuññan” ti?

Thero āha “yo kho, mahārāja, ajānanto pāpakammaṃ karoti, tassa bahutaraṃ apuññan” ti.

“Tena hi, bhante Nāgasena, yo amhākaṃ rājaputto vā rājamahāmatto vā ajānanto pāpakammaṃ karoti, taṃ mayaṃ diguṇaṃ daṇḍemā” ti.

“Taṃ kiṃ maññasi, mahārāja, tattaṃ ayoguḷaṃ ādittaṃ sampajjalitaṃ sajotibhūtaṃ eko jānanto gaṇheyya, eko ajānanto gaṇheyya, katamo balikataraṃ ḍayheyyā” ti?

“Yo kho, bhante, ajānanto gaṇheyya, so balikataraṃ ḍayheyyā” ti.

“Evameva kho, mahārāja, yo ajānanto pāpakammaṃ karoti, tassa bahutaraṃ apuññan” ti.

“Kallo’si, bhante Nāgasenā” ti.

(Miln III.7.8 Jānantājānantapāpakaraṇapañho)

\section*{Further Reading 4}

“Taṃ kiṃ maññatha, bhikkhave, rūpaṃ niccaṃ vā aniccaṃ vā” ti?

“Aniccaṃ, bhante.”

“Yaṃ panāniccaṃ, dukkhaṃ vā taṃ sukhaṃ vā” ti?

“Dukkhaṃ, bhante.”

“Yaṃ panāniccaṃ dukkhaṃ vipariṇāmadhammaṃ, kallaṃ nu taṃ samanupassituṃ ‘etaṃ mama, eso’ham asmi, eso me attā’” ti?

“No h’etaṃ, bhante.”

“Vedanā… saññā… saṅkhārā… viññāṇaṃ niccaṃ vā aniccaṃ vā” ti?

“Aniccaṃ, bhante.”

“Yaṃ panāniccaṃ dukkhaṃ vā taṃ sukhaṃ vā” ti?

“Dukkhaṃ, bhante.”

“Yaṃ panāniccaṃ dukkhaṃ vipariṇāmadhammaṃ, kallaṃ nu taṃ samanupassituṃ ‘etaṃ mama, eso’ham asmi, eso me attā’” ti?

“No h’etaṃ, bhante.” (S 3.1.7 Anattalakkhaṇasuttaṃ)