\chapter{任誕第二十三}

\subsection*{1}

\textbf{陳留阮籍、譙國嵇康、河內山濤,三人年皆相比,康年少亞之,預此契者,沛國劉伶、陳留阮咸、河內向秀、琅邪王戎,七人常集于竹林之下,肆意酣暢,故世謂「竹林七賢」。}{\footnotesize \textbf{晉陽秋}曰于時風譽扇于海內,至于今詠之。}

\subsection*{2}

\textbf{阮籍遭母喪,在晉文王坐進酒肉,司隸何曾亦在坐,}{\footnotesize \textbf{晉諸公贊}曰何曾,字穎考,陳郡陽夏人,父夔,魏太僕,曾以高雅稱,加性仁孝,累遷司隸校尉,用心甚正,朝廷師之,仕晉至太宰。}\textbf{曰:「明公方以孝治天下,而阮籍以重喪,顯於公坐飲酒食肉,宜流之海外,以正風教。」文王曰:「嗣宗毀頓如此,君不能共憂之,何謂?且有疾而飲酒食肉,固喪禮也。」籍飲噉不輟,神色自若。}{\footnotesize \textbf{干寶晉紀}曰何曾嘗謂阮籍曰「卿恣情任性,敗俗之人也,今忠賢執政,綜核名實,若卿之徒,何可長也」,復言之於太祖,籍飲噉不輟,故魏晉之間,有被髮夷傲之事,背死忘生之人,反謂行禮者,籍為之也。\textbf{魏氏春秋}曰籍性至孝,居喪雖不率常禮,而毀幾滅性,然為文俗之士何曾等深所讐疾,大將軍司馬昭愛其通偉,而不加害也。}

\subsection*{3}

\textbf{劉伶病酒,渴甚,從婦求酒,婦捐酒毀器,涕泣諫曰:「君飲太過,非攝生之道,必宜斷之。」伶曰:「甚善!我不能自禁,唯當祝鬼神,自誓斷之耳,便可具酒肉。」婦曰:「敬聞命。」供酒肉於神前,請伶祝誓,伶跪而祝曰:「天生劉伶,以酒為名,一飲一斛,五斗解酲,}{\footnotesize \textbf{毛公}注曰酒病曰酲。}\textbf{婦人之言,慎不可聽。」便引酒進肉,隗然已醉矣。}{\footnotesize 見竹林七賢論。}

\subsection*{4}

\textbf{劉公榮與人飲酒,雜穢非類,人或譏之,答曰:「勝公榮者,不可不與飲,不如公榮者,亦不可不與飲,是公榮輩者,又不可不與飲。」故終日共飲而醉。}{\footnotesize \textbf{劉氏譜}曰昶,字公榮,沛國人。\textbf{晉陽秋}曰昶為人通達,仕至兗州刺史。}

\subsection*{5}

\textbf{步兵校尉缺,廚中有貯酒數百斛,阮籍乃求為步兵校尉。}{\footnotesize \textbf{文士傳}曰籍放誕有傲世情,不樂仕宦,晉文帝親愛籍,恆與談戲,任其所欲,不迫以職事,籍常從容曰「平生曾遊東平,樂其土風,願得為東平太守」,文帝說,從其意,籍便騎驢徑到郡,皆壞府舍諸壁障,使內外相望,然後教令清寧,十餘日,便復騎驢去,後聞步兵廚中有酒三百石,忻然求為校尉,於是入府舍,與劉伶酣飲。\textbf{竹林七賢論}又云籍與伶共飲步兵廚中,並醉而死。此好事者為之言,籍景元中卒,而劉伶太始中猶在。}

\subsection*{6}

\textbf{劉伶恆縱酒放達,或脫衣裸形在屋中,人見譏之,伶曰:「我以天地為棟宇,屋室為褌衣,諸君何為入我褌中?」}{\footnotesize \textbf{鄧粲晉紀}曰客有詣伶,值其裸袒,伶笑曰「吾以天地為宅舍,以屋宇為褌衣,諸君自不當入我褌中,又何惡乎」,其自任若是。}

\subsection*{7}

\textbf{阮籍嫂嘗還家,籍見與別,或譏之,}{\footnotesize \textbf{曲禮}嫂叔不通問。故譏之。}\textbf{籍曰:「禮豈為我輩設也?」}

\subsection*{8}

\textbf{阮公鄰家婦有美色,當壚酤酒,阮與王安豐常從婦飲酒,阮醉,便眠其婦側,夫始殊疑之,伺察,終無他意。}{\footnotesize \textbf{王隱晉書}曰籍鄰家處子有才色,未嫁而卒,籍與無親,生不相識,往哭,盡哀而去,其達而無檢,皆此類也。}

\subsection*{9}

\textbf{阮籍當葬母,蒸一肥豚,飲酒二斗,然後臨訣,直言「窮矣」,都得一號,因吐血,廢頓良久。}{\footnotesize \textbf{鄧粲晉紀}曰籍,母將死,與人圍棊如故,對者求止,籍不肯,留與決賭,既而飲酒三斗,舉聲一號,嘔血數升,廢頓久之。}

\subsection*{10}

\textbf{阮仲容、}{\footnotesize 咸也。}\textbf{步兵居道南,諸阮居道北,北阮皆富,南阮貧,七月七日,北阮盛曬衣,皆紗羅錦綺,仲容以竿掛大布犢鼻褌於中庭,人或怪之,答曰:「未能免俗,聊復爾耳。」}{\footnotesize \textbf{竹林七賢論}曰諸阮前世皆儒學,善居室,唯咸一家尚道棄事,好酒而貧,舊俗,七月七日,法當曬衣,諸阮庭中,爛然錦綺,咸時總角,乃豎長竿,掛犢鼻褌也。}

\subsection*{11}

\textbf{阮步兵}{\footnotesize 籍也。}\textbf{喪母,裴令公}{\footnotesize 楷也。}\textbf{往弔之,阮方醉,散髮坐牀,箕踞不哭,裴至,下席於地,哭弔喭畢,便去,或問裴:「凡弔,主人哭,客乃為禮,阮既不哭,君何為哭?」裴曰:「阮方外之人,故不崇禮制,我輩俗中人,故以儀軌自居。」時人歎為兩得其中。}{\footnotesize \textbf{名士傳}曰阮籍喪親,不率常禮,裴楷往弔之,遇籍方醉,散髮箕踞,旁若無人,楷哭泣盡哀而退,了無異色,其安同異如此。\textbf{戴逵}論之曰若裴公之致弔,欲冥外以護內,有達意也,有弘防也。}

\subsection*{12}

\textbf{諸阮皆能飲酒,仲容至宗人間共集,不復用常桮斟酌,以大甕盛酒,圍坐,相向大酌,時有群豬來飲,直接去上,便共飲之。}

\subsection*{13}

\textbf{阮渾長成,風氣韻度似父,亦欲作達,步兵曰:「仲容已預之,卿不得復爾。」}{\footnotesize \textbf{竹林七賢論}曰籍之抑渾,蓋以渾未識己之所以為達也,後咸兄子簡,亦以曠達自居,父喪,行遇大雪,寒凍,遂詣浚儀令,令為它賓設黍臛,簡食之,以致清議,廢頓幾三十年,是時竹林諸賢之風雖高,而禮教尚峻,迨元康中,遂至放蕩越禮,樂廣譏之曰「名教中自有樂地,何至於此」,樂令之言有旨哉,謂彼非玄心,徒利其縱恣而已。}

\subsection*{14}

\textbf{裴成公婦,王戎女,王戎晨往裴許,不通徑前,裴從牀南下,女從北下,相對作賓主,了無異色。}{\footnotesize \textbf{裴氏家傳}曰頠取戎長女。}

\subsection*{15}

\textbf{阮仲容先幸姑家鮮卑婢,及居母喪,姑當遠移,初云當留婢,既發,迺將去,仲容借客驢著重服自追之,累騎而返,曰:「人種不可失。」即遙集之母也。}{\footnotesize \textbf{竹林七賢論}曰咸既追婢,於是世議紛然,自魏末沈淪閭巷,逮晉咸寧中,始登王途。\textbf{阮孚別傳}曰咸與姑書曰「胡婢遂生胡兒」,姑答書曰「魯靈光殿賦曰『胡人遙集於上楹』,可字曰遙集也」,故孚字遙集。}

\subsection*{16}

\textbf{任愷既失權勢,不復自檢括,或謂和嶠曰:「卿何以坐視元裒敗而不救?」和曰:「元裒如北夏門,拉攞自欲壞,非一木所能支。」}{\footnotesize \textbf{晉諸公贊}曰愷,字元裒,樂安博昌人,有雅識國幹,萬機大小多綜之,與賈充不平,充乃啓愷掌吏部,又使有司奏愷用御食器,坐免官,世祖情遂薄焉。}

\subsection*{17}

\textbf{劉道真少時,常漁草澤,善歌嘯,聞者莫不留連,有一老嫗,識其非常人,甚樂其歌嘯,乃殺豚進之,道真食豚盡,了不謝,嫗見不飽,又進一豚,食半餘半,迺還之,後為吏部郎,嫗兒為小令史,道真超用之,不知所由,問母,母告之,於是齎牛酒詣道真,道真曰:「去,去!無可復用相報。」}{\footnotesize 劉寶已見。}

\subsection*{18}

\textbf{阮宣子常步行,以百錢挂杖頭,至酒店,便獨酣暢,雖當世貴盛,不肯詣也。}{\footnotesize \textbf{名士傳}曰脩性簡任。}

\subsection*{19}

\textbf{山季倫為荊州,時出酣暢,人為之歌曰:「山公時一醉,徑造高陽池,日莫倒載歸,茗艼無所知,復能乘駿馬,倒著白接籬,舉手問葛彊,何如并州兒?」高陽池在襄陽,彊是其愛將,并州人也。}{\footnotesize \textbf{襄陽記}曰漢侍中習郁於峴山南,依范蠡養魚法作魚池,池邊有高隄,種竹及長楸,芙蓉蔆芡覆水,是遊燕名處也,山簡每臨此池,未嘗不大醉而還,曰「此是我高陽池也」,襄陽小兒歌之。}

\subsection*{20}

\textbf{張季鷹縱任不拘,時人號為江東步兵,或謂之曰:「卿乃可縱適一時,獨不為身後名邪?」答曰:「使我有身後名,不如即時一桮酒。」}{\footnotesize \textbf{文士傳}曰翰任性自適,無求當世,時人貴其曠達。}

\subsection*{21}

\textbf{畢茂世云:「一手持蟹螯,一手持酒桮,拍浮酒池中,便足了一生。」}{\footnotesize \textbf{晉中興書}曰畢卓,字茂世,新蔡人,少傲達,為胡毋輔之所知,太興末,為吏部郎,嘗飲酒廢職,比舍郎釀酒熟,卓因醉,夜至其罋間取飲之,主者謂是盜,執而縛之,知為吏部也,釋之,卓遂引主人燕罋側,取醉而去,溫嶠素知愛卓,請為平南長史,卒。}

\subsection*{22}

\textbf{賀司空入洛赴命,為太孫舍人,經吳閶門,在船中彈琴,張季鷹本不相識,先在金閶亭,聞絃甚清,下船就賀,因共語,便大相知說,問賀:「卿欲何之?」賀曰:「入洛赴命,正爾進路。」張曰:「吾亦有事北京,因路寄載。」便與賀同發,初不告家,家追問迺知。}

\subsection*{23}

\textbf{祖車騎過江時,公私儉薄,無好服玩,王、庾諸公共就祖,忽見裘袍重疊,珍飾盈列,諸公怪問之,祖曰:「昨夜復南塘一出。」祖于時恆自使健兒鼓行劫鈔,在事之人,亦容而不問。}{\footnotesize \textbf{晉陽秋}曰逖性通濟,不拘小節,又賓從多是桀黠勇士,逖待之皆如子弟,永嘉中,流民以萬數,揚土大饑,賓客攻剽,逖輒擁護全衛,談者以此少之,故久不得調。}

\subsection*{24}

\textbf{鴻臚卿孔群好飲酒,王丞相語云:「卿何為恆飲酒?不見酒家覆瓿布,日月糜爛?」群曰:「不爾,不見糟肉,乃更堪久?」群嘗書與親舊:「今年田得七百斛秫米,不了麴糵事。」}{\footnotesize 群已見上。}

\subsection*{25}

\textbf{有人譏周僕射與親友言戲,穢雜無檢節,}{\footnotesize \textbf{鄧粲晉紀}曰王導與周顗及朝士詣尚書紀瞻觀伎,瞻有愛妾,能為新聲,顗於眾中欲通其妾,露其醜穢,顏無怍色,有司奏免顗官,詔特原之。}\textbf{周曰:「吾若萬里長江,何能不千里一曲?」}

\subsection*{26}

\textbf{溫太真位未高時,屢與揚州、淮中估客樗蒱,與輒不競,嘗一過,大輸物,戲屈,無因得反,與庾亮善,於舫中大喚亮曰:「卿可贖我。」庾即送直,然後得還,經此數四。}{\footnotesize \textbf{中興書}曰嶠有儁朗之目,而不拘細行。}

\subsection*{27}

\textbf{溫公喜慢語,卞令禮法自居,}{\footnotesize \textbf{卞壼別傳}曰壼正色立朝,百寮嚴憚,貴遊子弟,莫不祗肅。}\textbf{至庾公許,大相剖擊,溫發口鄙穢,庾公徐曰:「太真終日無鄙言。」}{\footnotesize 重其達也。}

\subsection*{28}

\textbf{周伯仁風德雅重,深達危亂,過江積年,恆大飲酒,嘗經三日不醒,時人謂之「三日僕射」。}{\footnotesize \textbf{晉陽秋}曰初,顗以雅望獲海內盛名,後屢以酒失,庾亮曰「周侯末年,可謂鳳德之衰也」。\textbf{語林}曰伯仁正有姊喪,三日醉,姑喪,二日醉,大損資望,每醉,諸公常共屯守。}

\subsection*{29}

\textbf{衛君長為溫公長史,溫公甚善之,每率爾提酒脯就衛,箕踞相對彌日,衛往溫許亦爾。}{\footnotesize 衛永已見。}

\subsection*{30}

\textbf{蘇峻亂,諸庾逃散,庾冰時為吳郡,單身奔亡,民吏皆去,唯郡卒獨以小船載冰出錢塘口,蘧篨覆之,時峻賞募覓冰,屬所在搜檢甚急,卒捨船市渚,因飲酒醉還,舞棹向船曰:「何處覓庾吳郡?此中便是。」冰大惶怖,然不敢動,監司見船小裝狹,謂卒狂醉,都不復疑,自送過淛江,寄山陰魏家,得免,}{\footnotesize \textbf{中興書}曰冰為吳郡,蘇峻作逆,遣軍伐冰,冰棄郡奔會稽。}\textbf{後事平,冰欲報卒,適其所願,卒曰:「出自廝下,不願名器,少苦執鞭,恆患不得快飲酒,使其酒足餘年畢矣,無所復須。」冰為起大舍,市奴婢,使門內有百斛酒,終其身,時謂此卒非唯有智,且亦達生。}

\subsection*{31}

\textbf{殷洪喬作豫章郡,}{\footnotesize \textbf{殷氏譜}曰羡,字洪喬,陳郡人,父識,鎮東司馬,羡仕至豫章太守。}\textbf{臨去,都下人因附百許函書,既至石頭,悉擲水中,因祝曰:「沈者自沈,浮者自浮,殷洪喬不能作致書郵。」}

\subsection*{32}

\textbf{王長史、謝仁祖同為王公掾,}{\footnotesize \textbf{王濛別傳}曰丞相王導辟名士時賢,協贊中興,旌命所加,必延俊乂,辟濛為掾。}\textbf{長史云:「謝掾能作異舞。」謝便起舞,神意甚暇,}{\footnotesize \textbf{晉陽秋}曰尚性通任,善音樂。\textbf{語林}曰謝鎮西酒後,於槃案間為洛市肆工鴝鵒舞,甚佳。}\textbf{王公熟視,謂客曰:「使人思安豐。」}{\footnotesize 戎性通任,尚類之。}

\subsection*{33}

\textbf{王、劉共在杭南,酣宴於桓子野家,}{\footnotesize 伊已見。}\textbf{謝鎮西往尚書墓還,葬後三日反哭,諸人欲要之,初遣一信,猶未許,然已停車,重要,便回駕,諸人門外迎之,把臂便下,裁得脫幘,著帽酣宴,半坐,乃覺未脫衰。}{\footnotesize 尚書,謝裒,尚叔也,已見。\textbf{宋明帝文章志}曰尚性輕率,不拘細行,兄葬後,往墓還,王濛、劉惔共遊新亭,濛欲招尚,先以問惔曰「計仁祖正當不為異同耳」,惔曰「仁祖韻中自應來」,乃遣要之,尚初辭,然已無歸意,及再請,即回軒焉,其率如此。}

\subsection*{34}

\textbf{桓宣武少家貧,戲大輸,債主敦求甚切,思自振之方,莫知所出,陳郡袁躭,俊邁多能,}{\footnotesize \textbf{袁氏家傳}曰躭,字彥道,陳郡陽夏人,魏郎中令渙曾孫也,魁梧爽朗,高風振邁,少有異才,倜儻不羈,士人多歸之,仕至司徒從事中郎。}\textbf{宣武欲求救於躭,躭時居艱,恐致疑,試以告焉,應聲便許,略無慊吝,遂變服懷布帽隨溫去,與債主戲,躭素有蓺名,債主就局,曰:「汝故當不辦作袁彥道邪?」遂共戲,十萬一擲,直上百萬數,投馬絕叫,傍若無人,探布帽擲對人曰:「汝竟識袁彥道不?」}{\footnotesize \textbf{郭子}曰桓公樗蒱,失數百斛米,求救於袁躭,躭在艱中,便云「大快,我必作采,卿但大喚」,即脫其衰,共出門去,覺頭上有布帽,擲去,著小帽,既戲,袁形勢呼袒,擲必盧雉,二人齊叫,敵家頃刻失數百萬也。}

\subsection*{35}

\textbf{王光祿云:「酒,正使人人自遠。」}{\footnotesize 光祿,王蘊也。\textbf{續晉陽秋}曰蘊素嗜酒,末年尤甚,及在會稽,略少醒日。}

\subsection*{36}

\textbf{劉尹云:「孫承公狂士,每至一處,賞翫累日,或回至半路卻返。」}{\footnotesize \textbf{中興書}曰承公少誕任不羈,家於會稽,性好山水,及求鄞縣,遺心細務,縱意遊肆,名阜盛川,靡不歷覽。}

\subsection*{37}

\textbf{袁彥道有二妹,一適殷淵源,一適謝仁祖,}{\footnotesize \textbf{袁氏譜}曰躭大妹名女皇,適殷浩,小妹名女正,適謝尚。}\textbf{語桓宣武云:「恨不更有一人配卿。」}

\subsection*{38}

\textbf{桓車騎在荊州,張玄為侍中,使至江陵,路經陽岐村,}{\footnotesize 村臨江,去荊州二百里。}\textbf{俄見一人,持半小籠生魚,徑來造船云:「有魚,欲寄作膾。」張乃維舟而納之,問其姓字,稱是劉遺民,}{\footnotesize \textbf{中興書}曰劉驎之,一字遺民。已見。}\textbf{張素聞其名,大相忻待,劉既知張銜命,問:「謝安、王文度並佳不?」張甚欲話言,劉了無停意,既進膾,便去,云:「向得此魚,觀君船上當有膾具,是故來耳。」於是便去,張乃追至劉家,為設酒,殊不清旨,張高其人,不得已而飲之,方共對飲,劉便先起,云:「今正伐荻,不宜久廢。」張亦無以留之。}

\subsection*{39}

\textbf{王子猷詣郗雍州,}{\footnotesize \textbf{中興書}曰郗恢,字道胤,高平人,父曇,北中郎將,恢長八尺,美髭髯,風神魁梧,烈宗器之,以為蕃伯之望,自太子左率擢為雍州刺史。}\textbf{雍州在內,見有毾㲪,云:「阿乞那得此物?」}{\footnotesize 阿乞,恢小字。}\textbf{令左右送還家,郗出覓之,王曰:「向有大力者負之而趨。」}{\footnotesize \textbf{莊子}曰夫藏舟於壑,藏山於澤,謂之固矣,然有大力者負之而走,昧者不知也。}\textbf{郗無忤色。}

\subsection*{40}

\textbf{謝安始出西戲,失車牛,便杖策步歸,道逢劉尹,語曰:「安石將無傷?」謝乃同載而歸。}

\subsection*{41}

\textbf{襄陽羅友有大韻,少時多謂之癡,嘗伺人祠,欲乞食,往太蚤,門未開,主人迎神出見,問以非時,何得在此,答曰:「聞卿祠,欲乞一頓食耳。」遂隱門側,至曉,得食便退,了無怍容。為人有記功,從桓宣武平蜀,按行蜀城闕觀宇,內外道陌廣狹,植種果竹多少,皆默記之,後宣武漂洲與簡文集,友亦預焉,共道蜀中事,亦有所遺忘,友皆名列,曾無錯漏,宣武驗以蜀城闕簿,皆如其言,坐者歎服,謝公云:「羅友詎減魏陽元。」後為廣州刺史,當之鎮,刺史桓豁語令莫來宿,答曰:「民已有前期,主人貧,或有酒饌之費,見與甚有舊,請別日奉命。」征西密遣人察之,至夕,乃往荊州門下書佐家,處之怡然,不異勝達。在益州語兒云:「我有五百人食器。」家中大驚,其由來清,而忽有此物,定是二百五十沓烏樏。}{\footnotesize \textbf{晉陽秋}曰友,字宅仁,襄陽人,少好學,不持節檢,性嗜酒,當其所遇,不擇士庶,又好伺人祠,往乞餘食,雖復營署壚肆,不以為羞,桓溫常責之云「君太不逮,須食,何不就身求?乃至於此」,友傲然不屑,答曰「就公乞食,今乃可得,明日已復無」,溫大笑之,始仕荊州,後在溫府,以家貧乞祿,溫雖以才學遇之,而謂其誕肆,非治民才,許而不用,後同府人有得郡者,溫為席赴別,友至尤晚,問之,友答曰「民性飲道嗜味,昨奉教旨,乃是首旦出門,於中路逢一鬼,大見揶揄,云『我只見汝送人作郡,何以不見人送汝作郡』,民始怖終慚,回還以解,不覺成淹緩之罪」,溫雖笑其滑稽,而心頗愧焉,後以為襄陽太守,累遷廣、益二州刺史,在藩舉其宏綱,不存小察,甚為吏民所安說,薨於益州。}

\subsection*{42}

\textbf{桓子野每聞清歌,輒喚「奈何」,謝公聞之曰:「子野可謂一往有深情。」}

\subsection*{43}

\textbf{張湛好於齋前種松柏,}{\footnotesize \textbf{晉東宮官名}曰湛,字處度,高平人。\textbf{張氏譜}曰湛祖嶷,正員郎,父曠,鎮軍司馬,湛仕至中書郎。}\textbf{時袁山松出遊,每好令左右作挽歌,}{\footnotesize 山松別見。\textbf{續晉陽秋}曰袁山松善音樂,北人舊歌有行路難曲,辭頗疎質,山松好之,乃為文其章句,婉其節制,每因酒酣,從而歌之,聽者莫不流涕,初,羊曇善唱樂,桓尹能挽歌,及山松以行路難繼之,時人謂之三絕。今云挽歌,未詳。}\textbf{時人謂:「張屋下陳屍,袁道上行殯。」}{\footnotesize \textbf{裴啓語林}曰張湛好於齋前種松,養鴝鵒,袁山松出遊,好令左右作挽歌,時人云云。}

\subsection*{44}

\textbf{羅友作荊州從事,桓宣武為王車騎集別,}{\footnotesize 車騎,王洽,別見。}\textbf{友進,坐良久,辭出,宣武曰:「卿向欲咨事,何以便去?」答曰:「友聞白羊肉美,一生未曾得喫,故冒求前耳,無事可咨,今已飽,不復須駐。」了無慚色。}

\subsection*{45}

\textbf{張驎酒後挽歌甚悽苦,桓車騎曰:「卿非田橫門人,何乃頓爾至致?」}{\footnotesize 驎,張湛小字也。\textbf{譙子法訓}云有喪而歌者,或曰「彼為樂喪也,有不可乎」,譙子曰「書云『四海遏密八音』,何樂喪之有」,曰「今喪有挽歌者,何以哉」,譙子曰「周聞之,蓋高帝召齊田橫至于戶鄉亭,自刎奉首,從者挽至於宮,不敢哭而不勝哀,故為歌以寄哀音,彼則一時之為也,鄰有喪,舂不相引,挽人銜枚,孰樂喪者邪」。\textbf{按}莊子曰「紼謳所生,必於斥苦」,司馬彪注曰「紼,引柩索也,斥,疏緩也,苦,用力也,引紼所以有謳歌者,為人有用力不齊,故促急之也」,春秋左氏傳曰「魯哀公會吳伐齊,其將公孫夏命歌虞殯」,杜預曰「虞殯,送葬歌,示必死也」,史記絳侯世家曰「周勃以吹簫樂喪」,然則挽歌之來久矣,非始起於田橫也,然譙氏引禮之文,頗有明據,非固陋者所能詳聞,疑以傳疑,以俟通博。}

\subsection*{46}

\textbf{王子猷嘗暫寄人空宅住,便令種竹,或問:「暫住何煩爾?」王嘯詠良久,直指竹曰:「何可一日無此君?」}{\footnotesize \textbf{中興書}曰徽之卓犖不羈,欲為傲達,放肆聲色頗過度,時人欽其才、穢其行也。}

\subsection*{47}

\textbf{王子猷居山陰,夜大雪,眠覺,開室,命酌酒,四望皎然,因起仿偟,詠左思招隱詩,}{\footnotesize \textbf{中興書}曰徽之任性放達,棄官東歸,居山陰也。\textbf{左詩}曰杖策招隱士,荒塗橫古今,巖穴無結構,丘中有鳴琴,白雪停陰岡,丹葩曜陽林。}\textbf{忽憶戴安道,時戴在剡,即便夜乘小船就之,經宿方至,造門不前而返,人問其故,王曰:「吾本乘興而行,興盡而返,何必見戴?」}

\subsection*{48}

\textbf{王衛軍云:「酒正自引人著勝地。」}{\footnotesize 王薈已見。}

\subsection*{49}

\textbf{王子猷出都,尚在渚下,舊聞桓子野善吹笛,}{\footnotesize \textbf{續晉陽秋}曰左將軍桓伊善音樂,孝武飲燕,謝安侍坐,帝命伊吹笛,伊神色無忤,既吹一弄,乃放笛云「臣於箏乃不如笛,然自足以韻合歌管,臣有一奴,善吹笛,且相便串,請進之」,帝賞其放率,聽召奴,奴既至,吹笛,伊撫箏而歌怨詩,因以為諫也。}\textbf{而不相識,遇桓於岸上過,王在船中,客有識之者,云是桓子野,王便令人與相聞云:「聞君善吹笛,試為我一奏。」桓時已貴顯,素聞王名,即便回下車,踞胡牀,為作三調,弄畢,便上車去,客主不交一言。}

\subsection*{50}

\textbf{桓南郡被召作太子洗馬,}{\footnotesize \textbf{玄別傳}曰玄初拜太子洗馬,時朝廷以溫有不臣之迹,故抑玄為素官。}\textbf{船泊荻渚,王大服散後已小醉,往看桓,桓為設酒,不能冷飲,頻語左右:「令溫酒來。」桓乃流涕嗚咽,王便欲去,桓以手巾掩淚,因謂王曰:「犯我家諱,何預卿事?」}{\footnotesize \textbf{晉安帝紀}曰玄哀樂過人,每歡戚之發,未嘗不至嗚咽。}\textbf{王歎曰:「靈寶故自達。」}{\footnotesize 靈寶,玄小字也。\textbf{異苑}曰玄生而有光照室,善占者云「此兒生有奇耀,宜字為天人」,宣武嫌其三文,復言為神靈寶,猶復用三,既難重前,卻減神一字,名曰靈寶。\textbf{語林}曰玄不立忌日,止立忌時,其達而不拘,皆此類。}

\subsection*{51}

\textbf{王孝伯問王大:「阮籍何如司馬相如?」王大曰:「阮籍胸中壘塊,故須酒澆之。」}{\footnotesize 言阮皆同相如,而飲酒異耳。}

\subsection*{52}

\textbf{王佛大歎言:「三日不飲酒,覺形神不復相親。」}{\footnotesize \textbf{晉安帝紀}曰忱少慕達,好酒,在荊州轉甚,一飲或至連日不醒,遂以此死。\textbf{宋明帝文章志}曰忱嗜酒,醉輒經日,自號上頓,世喭以大飲為上頓,起自忱也。}

\subsection*{53}

\textbf{王孝伯言:「名士不必須奇才,但使常得無事,痛飲酒,熟讀離騷,便可稱名士。」}

\subsection*{54}

\textbf{王長史登茅山,大慟哭曰:「琅邪王伯輿,終當為情死。」}{\footnotesize \textbf{王氏譜}曰廞,字伯輿,琅邪人,父薈,衛將軍,廞歷司徒長史。\textbf{周祗隆安記}曰初,王恭將唱義,使喻三吳,廞居喪,拔以為吳國內史,國寶既死,恭罷兵,令廞反喪服,廞大怒,即日據吳都以叛,恭使司馬劉牢之討廞,廞敗,不知所在。}