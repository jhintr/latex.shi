\chapter{棲逸第十八}

\subsection*{1}

\textbf{阮步兵嘯,聞數百步,蘇門山中,忽有真人,樵伐者咸共傳說,阮籍往觀,見其人擁厀巖側,籍登嶺就之,箕踞相對,籍商略終古,上陳黃農玄寂之道,下考三代盛德之美,以問之,仡然不應,復敘有為之教、棲神導氣之術以觀之,彼猶如前,凝矚不轉,籍因對之長嘯,良久,乃笑曰:「可更作。」籍復嘯,意盡,退,還半嶺許,聞上酋然有聲,如數部鼓吹,林谷傳響,顧看,迺向人嘯也。}{\footnotesize \textbf{魏氏春秋}曰阮籍常率意獨駕,不由徑路,車跡所窮,輒慟哭而反,嘗遊蘇門山,有隱者莫知姓名,有竹實數斛,杵臼而已,籍聞而從之,談太古無為之道,論五帝三皇之義,蘇門先生翛然曾不眄之,籍乃嘐然長嘯,韻響寥亮,蘇門先生乃逌爾而笑,籍既降,先生喟然高嘯,有如鳳音,籍素知音,乃假蘇門先生之論以寄所懷,其歌曰「日沒不周西,月出丹淵中,陽精晦不見,陰光代為雄,亭亭在須臾,厭厭將復隆,富貴俛仰間,貧賤何必終」。\textbf{竹林七賢論}曰籍歸,遂著大人先生論,所言皆胸懷間本趣,大意謂先生與己不異也,觀其長嘯相和,亦近乎目擊道存矣。}

\subsection*{2}

\textbf{嵇康遊於汲郡山中,遇道士孫登,遂與之遊,康臨去,登曰:「君才則高矣,保身之道不足。」}{\footnotesize \textbf{康集序}曰孫登者,不知何許人,無家,於汲郡北山土窟住,夏則編草為裳,冬則被髮自覆,好讀易,鼓一絃琴,見者皆親樂之。\textbf{魏氏春秋}曰登性無喜怒,或沒諸水,出而觀之,登復大笑,時時出入人間,所經家設衣食者,一無所辭,去皆舍去。\textbf{文士傳}曰嘉平中,汲縣民共入山中,見一人,所居懸巖百仞,叢林鬱茂,而神明甚察,自云「孫姓,登名,字公和」,康聞,乃從遊三年,問其所圖,終不答,然神謀所存良妙,康每薾然歎息,將別,謂曰「先生竟無言乎」,登乃曰「子識火乎?生而有光,而不用其光,果然在於用光,人生有才,而不用其才,果然在於用才,故用光在乎得薪,所以保其曜,用才在乎識物,所以全其年,今子才多識寡,難乎免於今之世矣,子無多求」,康不能用,及遭呂安事,在獄為詩自責云「昔慚下惠,今愧孫登」。\textbf{王隱晉書}曰孫登即阮籍所見者也,嵇康執弟子禮而師焉,魏晉去就,易生嫌疑,貴賤並沒,故登或默也。}

\subsection*{3}

\textbf{山公將去選曹,欲舉嵇康,康與書告絕。}{\footnotesize \textbf{康別傳}曰山巨源為吏部郎,遷散騎常侍,舉康,康辭之,並與山絕,豈不識山之不以一官遇己情邪?亦欲標不屈之節,以杜舉者之口耳,乃答濤書,自說不堪流俗,而非薄湯武,大將軍聞而惡之。}

\subsection*{4}

\textbf{李廞是茂曾第五子,清貞有遠操,而少羸病,不肯婚宦,居在臨海,住兄侍中墓下,既有高名,王丞相欲招禮之,故辟為府掾,廞得牋命,笑曰:「茂弘乃復以一爵假人。」}{\footnotesize \textbf{文字志}曰廞,字宗子,江夏鍾武人,祖康,秦州刺史,父重,平陽太守,世有名望,廞好學,善草隸,與兄式齊名,躄疾不能行坐,常仰臥彈琴,讀誦不輟,河間王辟太尉掾,以疾不赴,後避難,隨兄南渡,司徒王導復辟之,廞曰「茂弘乃復以一爵加人」,永和中卒,廞嘗為二府辟,故號李公府也。式,字景則,廞長兄也,思理儒隱,有平素之譽,渡江,累遷臨海太守、侍中,年五十四而卒。}

\subsection*{5}

\textbf{何驃騎弟以高情避世,而驃騎勸之令仕,答曰:「予第五之名,何必減驃騎?」}{\footnotesize \textbf{中興書}曰何準,字幼道,廬江灊人,驃騎將軍充第五弟也,雅好高尚,徵聘一無所就,充位居宰相,權傾人主,而準散帶衡門,不及世事,于時名德皆稱之,年四十七卒,有女,為穆帝皇后,贈光祿大夫,子恢讓不受。}

\subsection*{6}

\textbf{阮光祿在東山,蕭然無事,常內足於懷,}{\footnotesize \textbf{阮裕別傳}曰裕居會稽剡山,志存肥遁。}\textbf{有人以問王右軍,右軍曰:「此君近不驚寵辱,}{\footnotesize \textbf{老子}曰寵辱若驚,得之若驚,失之若驚。}\textbf{雖古之沈冥,何以過此?」}{\footnotesize \textbf{楊子}曰蜀莊沈冥。\textbf{李軌}注曰沈冥,猶玄寂,泯然无迹之貌。}

\subsection*{7}

\textbf{孔車騎少有嘉遁意,年四十餘,始應安東命,未仕宦時,常獨寢,歌吹自箴誨,自稱孔郎,遊散名山,}{\footnotesize \textbf{孔愉別傳}曰永嘉大亂,愉入臨海山中,不求聞達,中宗命為參軍。}\textbf{百姓謂有道術,為生立廟,今猶有孔郎廟。}

\subsection*{8}

\textbf{南陽劉驎之,高率善史傳,隱於陽岐,于時符堅臨江,荊州刺史桓沖將盡訏謨之益,徵為長史,遣人船往迎,贈貺甚厚,驎之聞命,便升舟,悉不受所餉,緣道以乞窮乏,比至上明亦盡,一見沖,因陳無用,翛然而退,居陽岐積年,衣食有無常與村人共,值己匱乏,村人亦如之,甚厚為鄉閭所安。}{\footnotesize \textbf{鄧粲晉紀}曰驎之,字子驥,南陽安眾人,少尚質素,虛退寡欲,好遊山澤間,志存遁逸,桓沖嘗至其家,驎之方條桑,謂沖「使君既枉駕光臨,宜先詣家君」,沖遂詣其父,父命驎之,然後乃還,拂短褐與沖言,父使驎之自持濁酒葅菜供賓,沖敕人代之,父辭曰「若使官人,則非野人之意也」,沖為慨然,至昏乃退,因請為長史,固辭,居陽岐,去道斥近,人士往來,必投其家,驎之身自供給,贈致無所受,去家百里,有孤嫗疾將死,謂人曰「唯有劉長史當埋我耳」,驎之身往候之,疾終,為治棺殯,其仁愛皆如此,以壽卒。}

\subsection*{9}

\textbf{南陽翟道淵與汝南周子南少相友,共隱于尋陽,庾太尉說周以當世之務,周遂仕,翟秉志彌固,其後周詣翟,翟不與語。}{\footnotesize \textbf{晉陽秋}曰翟湯,字道淵,南陽人,漢方進之後也,篤行任素,義讓廉潔,饋贈一無所受,值亂多寇,聞湯名德,皆不敢犯。\textbf{尋陽記}曰初,庾亮臨江州,聞翟湯之風,束帶躡屐而詣焉,亮禮甚恭,湯曰「使君直敬其枯木朽株耳」,亮稱其能言,表薦之,徵國子博士,不赴,主簿張玄曰「此君臥龍,不可動也」,終于家。}

\subsection*{10}

\textbf{孟萬年及弟少孤,居武昌陽新縣,萬年遊宦,有盛名當世,少孤未嘗出,京邑人士思欲見之,乃遣信報少孤,云兄病篤,狼狽至都,時賢見之者,莫不嗟重,因相謂曰:「少孤如此,萬年可死。」}{\footnotesize \textbf{袁宏孟處士銘}曰處士名陋,字少孤,武昌陽新人,吳司空孟宗後也,少而希古,布衣蔬食,棲遲蓬蓽之下,絕人間之事,親族慕其孝,大將軍命會稽王辟之,稱疾不至,相府歷年虛位,而澹然無悶,卒不降志,時人奇之。}

\subsection*{11}

\textbf{康僧淵在豫章,去郭數十里,立精舍,旁連嶺,帶長川,芳林列於軒庭,清流激於堂宇,乃閒居研講,希心理味,庾公諸人多往看之,觀其運用吐納,風流轉佳,加處之怡然,亦有以自得,聲名乃興,後不堪,遂出。}{\footnotesize 僧淵已見。}

\subsection*{12}

\textbf{戴安道既厲操東山,}{\footnotesize \textbf{續晉陽秋}曰逵不樂當世,以琴書自娛,隱會稽剡山,國子博士徵,不就。}\textbf{而其兄欲建式遏之功,}{\footnotesize \textbf{戴氏譜}曰逯,字安丘,譙國人,祖碩,父綏,有名位,逯以武勇顯,有功,封廣陵侯,仕至大司農。}\textbf{謝太傅曰:「卿兄弟志業,何其太殊?」戴曰:「下官不堪其憂,家弟不改其樂。」}

\subsection*{13}

\textbf{許玄度隱在永興南幽穴中,每致四方諸侯之遺,或謂許曰:「嘗聞箕山人似不爾耳。」許曰:「筐篚苞苴,故當輕於天下之寶耳。」}{\footnotesize \textbf{鄭玄禮記注}云苞苴,裹肉也,或以葦,或以茅。此言許由尚致堯帝之讓,筐篚之遺,豈非輕邪?}

\subsection*{14}

\textbf{范宣未嘗入公門,韓康伯與同載,遂誘俱入郡,范便於車後趨下。}{\footnotesize \textbf{續晉陽秋}曰宣少尚隱遁,家于豫章,以清潔自立。}

\subsection*{15}

\textbf{郗超每聞欲高尚隱退者,輒為辦百萬資,并為造立居宇,在剡為戴公起宅,甚精整,戴始往舊居,與所親書曰:「近至剡,如官舍。」郗為傅約亦辦百萬資,傅隱事差互,故不果遺。}{\footnotesize 約,瓊小字。}

\subsection*{16}

\textbf{許掾好遊山水,而體便登陟,時人云:「許非徒有勝情,實有濟勝之具。」}

\subsection*{17}

\textbf{郗尚書與謝居士善,常稱:「謝慶緒識見雖不絕人,可以累心處都盡。」}{\footnotesize 尚書,郗恢也,別見。\textbf{檀道鸞續晉陽秋}曰謝敷,字慶緒,會稽人,崇信釋氏,初入太平山中十餘年,以長齋供養為業,招引同事,化納不倦,以母老還南山若邪中,內史郗愔表薦之,徵博士,不就,初,月犯少微星,一名處士星,占云「以處士當之」,時戴逵居剡,既美才藝而交遊貴盛,先敷著名,時人憂之,俄而敷死,會稽人士以嘲吳人云「吳中高士,便是求死不得」。}