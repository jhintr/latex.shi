\chapter{清廟之什詁訓傳第二十六}

\begin{quoting}\textbf{釋文}周頌三十一篇,皆是周室太平德洽著成功之樂歌也,名之曰頌,頌者誦也容也,歌誦盛德,序太平之形容,以此至美吿於神明,皆成王周公時作也。\end{quoting}

\section{清廟}

%{\footnotesize 一章、八句}

\textbf{清廟,祀文王也。周公既成洛邑,朝諸侯,率以祀文王焉。}{\footnotesize 清廟者,祭有清明之德者之宮也,謂祭文王也。天德清明,文王象焉,故祭之而歌此詩也。廟之言貌也,死者精神不可得而見,但以生時之居立宮室象貌為之耳。成洛邑,居攝五年時。}

\textbf{於穆清廟,肅雝顯相。}{\footnotesize 於,歎辭也。穆美、肅敬、雝和、相助也。箋云顯,光也見也。於乎美哉,周公之祭清廟也,其禮儀敬且和,又諸侯有光明著見之德者來助祭。}\textbf{濟濟多士,秉文之德,對越在天。}{\footnotesize 執文德之人也。箋云對配、越於也。濟濟之眾士皆執行文王之德,文王精神已在天矣,猶配順其素如生存。}\textbf{駿奔走在廟,不顯不承,無射於人斯。}{\footnotesize 駿,長也。顯於天矣,見承於人矣,不見厭於人矣。箋云駿,大也。諸侯與眾士於周公祭文王俱奔走而來,在廟中助祭,是不光明文王之德與,言其光明之也,是不承順文王志意與,言其承順之也,此文王之德人無厭之。}

\begin{quoting}對,報答。越,宣揚。爾雅釋詁「駿,速也」,孔疏「廟中奔走以疾為敬」。不,同丕,發語詞。射,齊詩作斁,厭棄也。\end{quoting}

\section{維天之命}

%{\footnotesize 一章、八句}

\textbf{維天之命,太平告文王也。}{\footnotesize 告太平者,居攝五年之末也。文王受命,不卒而崩,今天下太平,故承其意而告之,明六年制禮作樂。}

\begin{quoting}\textbf{陳奐}書雒誥大傳云「周公攝政,六年制禮作樂,七年致政」,維天之命制禮也,維清作樂也,烈文致政也,三詩並列,正與大傳節次合。\end{quoting}

\textbf{維天之命,於穆不已。}{\footnotesize 孟仲子曰「大哉天命之無極而美周之禮也」。箋云命,猶道也。天之道於乎美哉,動而不止,行而不已。}\textbf{於乎不顯,文王之德之純。假以溢我,我其收之,駿惠我文王。}{\footnotesize 純大、假嘉、溢慎、收聚也。箋云純,亦不已也。溢,盈溢之言也。於乎不光明與,文王之施德教之無倦已,美其與天同功也,以嘉美之道饒衍與我,我其聚斂之以制法度,以大順我文王之意,謂為周禮六官之職也,書曰「考朕昭子刑,乃單文祖德」。}\textbf{曾孫篤之。}{\footnotesize 成王能厚行之也。箋云曾,猶重也,自孫之子而下事先祖皆稱曾孫。是言曾孫,欲使後王皆厚行之,非維今也。}

\begin{quoting}釋文引韓詩云「維,念也」。\textbf{陳奐}假以溢我,言以嘉美之道戒慎於我也。\textbf{馬瑞辰}惠,順也,駿,當為馴之假借,馴亦順也,駿惠二字平列,皆為順。\end{quoting}

\section{維清}

%{\footnotesize 一章、五句}

\textbf{維清,奏象舞也。}{\footnotesize 象舞,象用兵時刺伐之舞,武王制焉。}

\begin{quoting}\textbf{戴震}辭彌少而意旨極深遠。\end{quoting}

\textbf{維清緝熙,文王之典。}{\footnotesize 典,法也。箋云緝熙,光明也。天下之所以無敗亂之政而清明者,乃文王有征伐之法故也。文王受命,七年五伐也。}\textbf{肇禋。}{\footnotesize 肇始、禋祀也。箋云文王受命,始祭天而征伐也。周禮,以禋祀祀昊天上帝。}\textbf{迄用有成,維周之禎。}{\footnotesize 迄至、禎祥也。箋云文王造此征伐之法,至今用之而有成功,謂伐紂克勝也,征伐之法乃周家得天下之祥。}

\begin{quoting}\textbf{陳奐}肇始、迄至,文義相對,言文王始行禋祀,至武王伐紂,用能有此成功也。\textbf{戴震}言此天下澄清光昭於無窮者,文王之法典實開始禋祀昊天盛禮,以迄於今而有成,是周有天下之祥如此也。\end{quoting}

\section{烈文}

%{\footnotesize 一章、十三句}

\textbf{烈文,成王即政,諸侯助祭也。}{\footnotesize 新王即政,必以朝享之禮祭於祖考,告嗣位也。}

\textbf{烈文辟公,錫茲祉福,惠我無疆,子孫保之。}{\footnotesize 烈,光也。文王錫之。箋云惠,愛也。光文百辟卿士及天下諸侯者,天錫之以此祉福也,又長愛之無有期竟,子孫得傳世,安而居之,謂文王武王以純德受命定天位。}\textbf{無封靡于爾邦,維王其崇之。念茲戎功,繼序其皇之。}{\footnotesize 封,大也。靡,累也。崇,立也。戎大、皇美也。箋云崇,厚也。皇,君也。無大累於女國,謂諸侯治國無罪惡也,王其厚之,增其爵土也,念此大功,勤事不廢,謂卿大夫能守其職,得繼世在位,以其次序其君之者,謂有大功,王則出而封之。}\textbf{無競維人,四方其訓之。不顯維德,百辟其刑之。於乎前王不忘。}{\footnotesize 競彊、訓道也。前王,武王也。箋云無彊乎維得賢人也,得賢人則國家彊矣,故天下諸侯順其所為也,不勤明其德乎,勤明之也,故卿大夫法其所為也,於乎先王,文王武王,其於此道,人稱頌之不忘。}

\begin{quoting}\textbf{馬瑞辰}烈文二字平列,烈言其功,文言其德。序,通敘,訓為緒,繼序即繼承。刑,通型,模範。\end{quoting}

\section{天作}

%{\footnotesize 一章、七句}

\textbf{天作,祀先王先公也。}{\footnotesize 先王,謂大王已下,先公,諸盩至不窋。}

\textbf{天作高山,大王荒之。}{\footnotesize 作生、荒大也。天生萬物於高山,大王行道,能大天之所作也。箋云高山,謂岐山也,書曰「道岍及岐,至于荊山」。天生此高山,使興雲雨,以利萬物,大王自豳遷焉,則能尊大之,廣其德澤,居之一年成邑,二年成都,三年五倍其初。}\textbf{彼作矣,文王康之。彼徂矣,岐有夷之行。}{\footnotesize 夷,易也。箋云彼,彼萬民也。徂往、行道也。彼萬民居岐邦者皆築作宮室,以為常居,文王則能安之,後之往者又以岐邦之君有佼易之道故也。易曰「乾以易知,坤以簡能,易則易知,簡則易從,易知則有親,易從則有功,有親則可久,有功則可大,可久則賢人之德,可大則賢人之業」,以此訂大王文王之道,卓爾與天地合其德。}\textbf{子孫保之。}

\begin{quoting}荒,治也,說文「荒,蕪也」,治乃其反訓義。康,同賡,續也。矣,後漢書西南夷傳引作者。\textbf{楊樹達}天作高山,太王墾闢其蕪穢,彼為其始,文王賡繼為之,是以雖彼險阻之岐山亦有平易之道路也,夫先人創業之難如此,子孫其善保之哉。\end{quoting}

\section{昊天有成命}

%{\footnotesize 一章、七句}

\textbf{昊天有成命,郊祀天地也。}

\begin{quoting}國語叔向引此詩而言曰「是道成王之德也,成王能明文昭、定武烈者也」。\end{quoting}

\textbf{昊天有成命,二后受之。成王不敢康,夙夜基命宥密。}{\footnotesize 二后,文武也。基始、命信、宥寬、密寧也。箋云昊天,天大號也。有成命者,言周自后稷之生而已有王命也。文王武王受其業,施行道德,成此王功,不敢自安逸,早夜始信順天命,不敢解倦,行寬仁安靜之政以定天下。寬仁,所以止苛刻也,安靜,所以息暴亂也。}\textbf{於緝熙,單厥心,肆其靖之。}{\footnotesize 緝明、熙廣、單厚、肆固、靖和也。箋云廣,當為光,固,當為故,字之誤也。於美乎,此成王之德也,既光明矣,又能厚其心矣,為之不解倦,故於其功終能和安之,謂夙夜自勤,至於天下太平。}

\begin{quoting}\textbf{馬瑞辰}古文明、成二字同義,成命猶言明命。康,安逸。爾雅釋詁「基,謀也」。單,國語引作亶,厚也。\end{quoting}

\section{我將}

%{\footnotesize 一章、十句}

\textbf{我將,祀文王於明堂也。}

\begin{quoting}此又一大武也。\end{quoting}

\textbf{我將我享,維羊維牛,維天其右之。}{\footnotesize 將大、享獻也。箋云將,猶奉也。我奉養我享祭之羊牛,皆充盛肥腯,有天氣之力助,言神饗其德而右助之。}\textbf{儀式刑文王之典,日靖四方。伊嘏文王,既右饗之。}{\footnotesize 儀善、刑法、典常、靖謀也。箋云靖,治也。受福曰嘏。我儀則式象法行文王之常道,以日施政于天下,維受福於文王,文王既右而饗之,言受而福之。}\textbf{我其夙夜,畏天之威,于時保之。}{\footnotesize 箋云于於、時是也。早夜敬天,於是得安文王之道。}

\begin{quoting}\textbf{朱熹}儀、式、刑皆法也。嘏 \texttt{jiǎ},同假,大也。\end{quoting}

\section{時邁}

%{\footnotesize 一章、十五句}

\textbf{時邁,巡守告祭柴望也。}{\footnotesize 巡守告祭者,天子巡行邦國,至于方嶽之下而封禪也,書曰「歲二月,東巡守至于岱宗,柴,望秩于山川,徧于群神」。}

\textbf{時邁其邦,昊天其子之,實右序有周。薄言震之,莫不震疊。懷柔百神,及河喬嶽,允王維后。}{\footnotesize 邁行、震動、疊懼、懷來、柔安、喬高也。高嶽,岱宗也。箋云薄,猶甫也,甫,始也。允,信也。武王既定天下,時出行其邦國,謂巡守也,天其子愛之,右助次序其事,謂多生賢知,使為之臣也,其兵所征伐,甫動之以威,則莫不動懼而服者,言其威武,又見畏也,王行巡守,其至方岳之下,來安群神,望于山川,皆以尊卑祭之,信哉武王之宜為君,美之也。}\textbf{明昭有周,式序在位。}{\footnotesize 明矣,知未然也。昭然,不疑也。箋云昭,見也。王巡守而明見天之子有周家也,以其有俊乂,用次第處位。言此者,著天其子愛之,右序之效也。}\textbf{載戢干戈,載櫜弓矢。}{\footnotesize 戢聚、櫜韜也。箋云載之言則也。王巡守而天下咸服,兵不復用,此又著震疊之效也。}\textbf{我求懿德,肆于時夏。}{\footnotesize 夏,大也。箋云懿美、肆陳也。我武王求有美德之士而任用之,故陳其功於是夏而歌之。樂歌大者稱夏。}\textbf{允王保之。}{\footnotesize 箋云允,信也。信哉武王之德,能長保此時夏之美。}

\begin{quoting}時,發語詞。\textbf{吳闓生}右、序皆助也。疊,通懾。櫜 \texttt{gāo}。\textbf{朱熹}夏,中國也,言求懿美之德以布陳於中國。\end{quoting}

\section{執競}

%{\footnotesize 一章、十四句}

\textbf{執競,祀武王也。}

\begin{quoting}\textbf{朱熹}此祭武王、成王、康王之詩。\end{quoting}

\textbf{執競武王,無競維烈。不顯成康,上帝是皇。}{\footnotesize 無競,競也。烈,業也。不顯乎其成大功而安之也。顯,光也。皇,美也。箋云競,彊也。能持彊道者,維有武王耳,不彊乎其克商之功業,言其彊也,不顯乎其成安祖考之道,言其又顯也,天以是故美之,予之福祿。}\textbf{自彼成康,奄有四方,斤斤其明。}{\footnotesize 自彼成康,用彼成安之道也。奄,同也。斤斤,明察也。箋云四方,謂天下也。武王用成安祖考之道,故受命伐紂,定天下,為周明察之君,斤斤如也。}\textbf{鍾鼓喤喤,磬筦將將,降福穰穰。降福簡簡,威儀反反。既醉既飽,福祿來反。}{\footnotesize 喤喤,和也。將將,集也。穰穰,眾也。簡簡,大也。反反,難也。反,復也。箋云反反,順習之貌。武王既定天下,祭祖考之廟,奏樂而八音克諧,神與之福又眾大,謂如嘏辭也,群臣醉飽,禮無違者,以重得福祿也。}

\begin{quoting}\textbf{馬瑞辰}釋文引韓詩云「執,服也」,蓋以執競為能執服彊禦。斤斤,即昕昕。喤,三家詩作鍠。筦,魯詩作管。反反,釋文引韓詩作昄昄,\textbf{胡承珙}說文「反,覆也」,凡言反覆者皆慎重之意。\end{quoting}

\section{思文}

%{\footnotesize 一章、八句}

\textbf{思文,后稷配天也。}

\textbf{思文后稷,克配彼天。立我烝民,莫匪爾極。}{\footnotesize 極,中也。箋云克,能也。立,當作粒。烝,眾也。周公思先祖有文德者,后稷之功能配天,昔堯遭洪水,黎民阻飢,后稷播殖百穀,烝民乃粒,萬邦作乂,天下之人無不於女時得其中者,言反其性。}\textbf{貽我來牟,帝命率育。無此疆爾界,陳常于時夏。}{\footnotesize 牟麥、率用也。箋云貽遺、率循、育養也。武王渡孟津,白魚躍入于舟,出涘以燎,後五日,火流為烏,五至,以穀俱來,此謂遺我來牟,天命以是循存后稷養天下之功而廣大其子孫之國,無此封竟於女今之經界,乃大有天下也,用是故,陳其久常之功,於是夏而歌之。夏之屬有九。書說烏以穀俱來,云穀紀后稷之德。}

\begin{quoting}配,祔祭。說文「配,酒色也」,段注「本義如是,後人借為妃字而本義廢矣,妃者匹也」。粒,養育。\textbf{馬瑞辰}牟麥為雙聲,來麥為疊韻,合牟來則為麥,焦氏循曰「麥為牟來之合聲,猶終葵之為錐,牟來倒為來牟,方音相轉,往往倒稱」,其說是也。\end{quoting}

%\begin{flushright}清廟之什十篇、十章、九十五句\end{flushright}