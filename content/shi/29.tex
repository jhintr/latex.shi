\chapter{駉詁訓傳第二十九}

\begin{quoting}\textbf{釋文}魯者,周公之子伯禽所封之國也。周公有大勳勞於天下,成王留之輔相,而封伯禽焉。其封域在禹貢徐州蒙羽之野,十七世至僖公,當周惠王襄王之時,能遵伯禽之法,外征淮夷,內修德教,國人美之,於是國卿季文子請周,而使史克作頌四篇。夫子刪詩錄之者,以周公有致太平之勳,成王命魯郊祭用天子禮樂,故取魯頌而同於王者之後焉。\end{quoting}

\section{駉}

%{\footnotesize 四章、章八句}

\textbf{駉,頌僖公也。僖公能遵伯禽之法,儉以足用,寬以愛民,務農重穀,牧于坰野,魯人尊之,於是季孫行父請命于周,而史克作是頌。}{\footnotesize 季孫行父,季文子也。史克,魯史也。}

\textbf{駉駉牡馬,在坰之野。}{\footnotesize 駉駉,良馬腹榦肥張也。坰,遠野也,邑外曰郊,郊外曰野,野外曰林,林外曰坰。箋云必牧於坰野者,辟民居與良田也,周禮曰「以官田、牛田、賞田、牧田任遠郊之地」。}\textbf{薄言駉者,有驈有皇,有驪有黃,以車彭彭。}{\footnotesize 牧之坰野則駉駉然。驪馬白跨曰驈,黃白曰皇,純黑曰驪,黃騂曰黃。諸侯六閑,馬四種,有良馬,有戎馬,有田馬,有駑馬。彭彭,有力有容也。箋云坰之牧地水草既美,牧人又良,飲食得其時,則自肥健耳。}\textbf{思無疆,思馬斯臧。}{\footnotesize 箋云臧,善也。僖公之思遵伯禽之法,反覆思之,無有竟已,乃至於思馬斯善,多其所及廣博。}

\begin{quoting}驈 \texttt{yù}。\textbf{王先謙}思無疆者,言僖公思慮深微,無有疆畔,即牧馬之法亦皆盡善,致斯蕃庶,與定之方中詩美衛文公「匪直也人,秉心塞淵,騋牝三千」同意。又曰上思,思慮,下思,語詞。\end{quoting}

\textbf{駉駉牡馬,在坰之野。薄言駉者,有騅有駓,有騂有騏,以車伾伾。}{\footnotesize 蒼白雜毛曰騅,黃白雜毛曰駓,赤黃曰騂,蒼祺曰騏。伾伾,有力也。}\textbf{思無期,思馬斯才。}{\footnotesize 才,多材也。}

\begin{quoting}伾 \texttt{pī}。\end{quoting}

\textbf{駉駉牡馬,在坰之野。薄言駉者,有驒有駱,有駵有雒,以車繹繹。}{\footnotesize 青驪驎曰驒,白馬黑鬣曰駱,赤身黑鬣曰駵,黑身白鬣曰雒。繹繹,善走也。}\textbf{思無斁,思馬斯作。}{\footnotesize 作,始也。箋云斁,厭也。思遵伯禽之法,無厭倦也。作,謂牧之使可乘駕也。}

\begin{quoting}驒 \texttt{tuó}。\textbf{朱熹}作,奮起也。\end{quoting}

\textbf{駉駉牡馬,在坰之野。薄言駉者,有駰有騢,有驔有魚,以車祛祛。}{\footnotesize 陰白雜毛曰駰,彤白雜毛曰騢,豪骭曰驔,二目白曰魚。祛祛,彊健也。}\textbf{思無邪,思馬斯徂。}{\footnotesize 箋云徂,猶行也。思遵伯禽之法,專心無復邪意也,牧馬使可走行。}

\begin{quoting}驔 \texttt{diàn}。\textbf{王先謙}思無邪者,思之真正無有邪曲。\end{quoting}

\section{有駜}

%{\footnotesize 三章、章九句}

\textbf{有駜,頌僖公君臣之有道也。}{\footnotesize 有道者,以禮義相與之謂也。}

\textbf{有駜有駜,駜彼乘黃。}{\footnotesize 駜,馬肥彊貌,馬肥彊則能升高進遠,臣彊力則能安國。箋云此喻僖公之用臣必先致其祿食,祿食足而臣莫不盡其忠。}\textbf{夙夜在公,在公明明。}{\footnotesize 箋云夙,早也。言時臣憂念君事,早起夜寐在於公之所,在於公之所,但明義明德也,禮記曰「大學之道,在明明德」。}\textbf{振振鷺,鷺于下。鼓咽咽,醉言舞。于胥樂兮。}{\footnotesize 振振,群飛貌。鷺,白鳥也,以興絜白之士咽咽鼓節也。箋云于於、胥皆也。僖公之時,君臣無事則相與明義明德而已,絜白之士群集於君之朝,君以禮樂與之飲酒,以鼓節之,咽咽然至於無筭爵,則又舞燕樂以盡其歡,君臣於是則皆喜樂也。}

\begin{quoting}駜 \texttt{bì}。\textbf{馬瑞辰}明、勉一聲之轉,明明即勉勉之假借,謂其在公盡力也。\textbf{朱熹}鷺羽,舞者所持,或坐或伏,如鷺之下也。見宛丘注。于,同曰,語詞。\textbf{朱熹}咽與淵同,鼓聲之深長也。\end{quoting}

\textbf{有駜有駜,駜彼乘牡。夙夜在公,在公飲酒。}{\footnotesize 言臣有餘敬,而君有餘惠。}\textbf{振振鷺,鷺于飛。鼓咽咽,醉言歸。于胥樂兮。}{\footnotesize 箋云飛,喻群臣飲酒醉欲退也。}

\begin{quoting}\textbf{朱熹}舞者振作鷺羽如飛也。\end{quoting}

\textbf{有駜有駜,駜彼乘駽。}{\footnotesize 青驪曰駽。}\textbf{夙夜在公,在公載燕。}{\footnotesize 箋云載之言則也。}\textbf{自今以始,歲其有。君子有穀,詒孫子。于胥樂兮。}{\footnotesize 歲其有,豐年也。箋云穀善、詒遺也。君臣安樂,則陰陽和而有豐年,其善道則可以遺子孫也。}

\begin{quoting}駽 \texttt{xuān}。\end{quoting}

\section{泮水}

%{\footnotesize 八章、章八句}

\textbf{泮水,頌僖公能修泮宮也。}

\textbf{思樂泮水,薄采其芹。}{\footnotesize 泮水,泮宮之水也,天子辟雝,諸侯泮宮。言水則采取其芹,宮則采取其化。箋云芹,水菜也。言己思樂僖公之修泮宮之水,復伯禽之法,而往觀之,采其芹也。辟雝者,築土雝水之外,圓如璧,四方來觀者均也。泮之言半也,半水者,蓋東西門以南通水,北無也,天子諸侯宮異制,因形然。}\textbf{魯侯戾止,言觀其旂。其旂茷茷,鸞聲噦噦。無小無大,從公于邁。}{\footnotesize 戾來、止至也。言觀其旗,言法則其文章也。茷茷,言有法度也。噦噦,言有聲也。箋云于往、邁行也。我采泮水之芹,見僖公來至于泮宮,我則觀其旂茷茷然,鸞和之聲噦噦然,臣無尊卑,皆從君行而來稱。言此者,僖公賢君,人樂見之。}

\begin{quoting}泮 \texttt{pàn},水名。茷 \texttt{pèi},同旆,見出車注。\end{quoting}

\textbf{思樂泮水,薄采其藻。魯侯戾止,其馬蹻蹻。其馬蹻蹻,其音昭昭。}{\footnotesize 其馬蹻蹻,言強盛也。箋云其音昭昭,僖公之德音。}\textbf{載色載笑,匪怒伊敎。}{\footnotesize 色溫潤也。箋云僖公之至泮宮,和顏色而笑語,非有所怒,於是有所教化也。}

\textbf{思樂泮水,薄采其茆。}{\footnotesize 茆,鳧葵也。}\textbf{魯侯戾止,在泮飲酒。既飲旨酒,永錫難老。}{\footnotesize 箋云在泮飲酒者,徵先生君子與之行飲酒之禮而因以謀事也,已飲美酒而長賜其難使老,難使老者,最壽考也,長賜之者,如王制所云「八十月告存,九十日有秩」者與。}\textbf{順彼長道,屈此群醜。}{\footnotesize 屈收、醜眾也。箋云順從長遠,屈治醜惡也。是時淮夷叛逆,既謀之於泮宮,則從彼遠道往伐之,治此群為惡之人。}

\textbf{穆穆魯侯,敬明其德。敬慎威儀,維民之則。允文允武,昭假烈祖。}{\footnotesize 假,至也。箋云則,法也。僖公之行,民之所法俲也,僖公信文矣,為修泮宮也,信武矣,為伐淮夷也,其聦明乃至於美祖之德,謂遵伯禽之法。}\textbf{靡有不孝,自求伊祜。}{\footnotesize 箋云祜,福也。國人無不法俲之者,皆庶幾力行,自求福祿。}

\begin{quoting}\textbf{王引之}靡有不孝,謂僖公無事不法俲其祖,非謂國人俲僖公也。\end{quoting}

\textbf{明明魯侯,克明其德。既作泮宮,淮夷攸服。}{\footnotesize 箋云克能、攸所也。言僖公能明其德,修泮宮而德化行,於是伐淮夷,所以能服也。}\textbf{矯矯虎臣,在泮獻馘。淑問如臯陶,在泮獻囚。}{\footnotesize 囚,拘也。箋云矯矯,武貌。馘,所格者之左耳。淑,善也。囚,所虜獲者。僖公既伐淮夷而反,在泮宮使武臣獻馘,又使善聽獄之吏如臯陶者獻囚,言伐有功,所任得其人。}

\textbf{濟濟多士,克廣德心。桓桓于征,狄彼東南。}{\footnotesize 桓桓,威武貌。箋云多士,謂虎臣及如臯陶之屬。征,征伐也。狄,當作剔,剔,治也。東南,斥淮夷。}\textbf{烝烝皇皇,不吳不揚。不告于訩,在泮獻功。}{\footnotesize 烝烝,厚也。皇皇,美也。揚,傷也。箋云烝烝,猶進進也。皇皇,當作暀暀,暀暀,猶往往也。吳,譁也。訩,訟也。言多士之於伐淮夷,皆勸之,有進進往往之心,不讙譁,不大聲,僖公還在泮宮,又無以爭訟之事告於治訟之官者,皆自獻其功。}

\begin{quoting}告,同鞫,窮治罪人也,\textbf{陳奐}不告于訩,言不窮治其惡,唯在柔服之而已。\end{quoting}

\textbf{角弓其觩,束矢其搜。戎車孔博,徒御無斁。既克淮夷,孔淑不逆。}{\footnotesize 觩,㢮貌。五十矢為束。搜,眾意也。箋云角弓觩然,言持弦急也,束矢搜然,言勁疾也。博,當作傅,甚傅致者,言安利也。徒行者、御車者皆敬其事,又無厭倦也,僖公以此兵眾伐淮夷而勝之,其士卒甚順軍法而善,無有為逆者,謂堙井刊木之類。}\textbf{式固爾猶,淮夷卒獲。}{\footnotesize 箋云式用、猶謀也。用堅固女軍謀之故,故淮夷盡可獲服也。謀,謂度己之德,慮彼之罪,以出兵也。}

\begin{quoting}\textbf{陳奐}淑,善也,不逆,言率從也。\end{quoting}

\textbf{翩彼飛鴞,集于泮林。食我桑黮,懷我好音。}{\footnotesize 翩,飛貌。鴞,惡聲之鳥也。黮,桑實也。箋云懷,歸也。言鴞恆惡鳴,今來止於泮水之木上食其桑黮,為此之故,故改其鳴,歸就我以善音,喻人感於恩則化也。}\textbf{憬彼淮夷,來獻其琛。元龜象齒,大賂南金。}{\footnotesize 憬,遠行貌。琛,寶也。元龜尺二寸。賂,遺也。南,謂荊楊也。箋云大,猶廣也。廣賂者,賂君及卿大夫也。荊楊之州,貢金三品。}

\begin{quoting}文選引韓詩「獷彼淮夷」云「獷,覺悟之貌」,說文「憬,覺悟也」。\textbf{俞樾}賂,當讀為璐,大璐,猶尚書顧命篇大玉耳,从玉、貝之字古或相通,說文「玩,弄也,重文貦」。\end{quoting}

\section{閟宮}

%{\footnotesize 八章、二章章十七句、一章十二句、一章三十八句、二章章八句、二章章十句}

\textbf{閟宮,頌僖公能復周公之宇也。}{\footnotesize 宇,居也。}

\begin{quoting}此毛詩最長之一篇也。\textbf{王安石}周頌之詞約,約所以為嚴,盛德故也,魯頌之詞侈,侈所以為夸,德不足故也。案,重分此詩為九章,三章章十七句、一章十六句、一章十七句、二章章八句、二章章十句。\end{quoting}

\textbf{閟宮有侐,實實枚枚。}{\footnotesize 閟,閉也。先妣姜嫄之廟在周常閉而無事,孟仲子曰「是禖宮也」。侐,清靜也。實實,廣大也。枚枚,礱密也。箋云閟,神也,姜嫄神所依,故廟曰神宮。}\textbf{赫赫姜嫄,其德不回。上帝是依,無災無害,彌月不遲。}{\footnotesize 上帝是依,依其子孫也。箋云依,依其身也。彌,終也。赫赫乎顯著姜嫄也,其德貞正不回邪,天用是馮依而降精氣,其任之又無災害,不坼不副,終人道十月而生子,不遲晚。}\textbf{是生后稷,降之百福。黍稷重穋,稙稺菽麥。奄有下國,俾民稼穡。}{\footnotesize 先種曰稙,後種曰稺。箋云奄,猶覆也。姜嫄用是而生子后稷,天神多予之福,以五穀終覆蓋天下,使民知稼穡之道,言其不空生也。后稷生而名棄,長大,堯登用之,使居稷官,民賴其功,後雖作司馬,天下猶以后稷稱焉。}\textbf{有稷有黍,有稻有秬。奄有下土,纘禹之緒。}{\footnotesize 緒,業也。箋云秬,黑黍也。緒,事也。堯時洪水為災,民不粒食,天神多予后稷以五穀,禹平水土,乃教民播種之,於是天下大有,故云纘禹之事也。美之,故申說以明之。}

\begin{quoting}閟 \texttt{bì} 宮,神廟也。枚枚,釋文「閒暇無人之貌也」。稙稺 \texttt{zhí zhì}。\textbf{朱熹}奄有下國,封於邰也。\end{quoting}

\textbf{后稷之孫,實維大王。居岐之陽,實始翦商。}{\footnotesize 翦,齊也。箋云翦,斷也。大王自豳徙居岐陽,四方之民咸歸往之,於時而有王迹,故云是始斷商。}\textbf{至于文武,纘大王之緒。致天之屆,于牧之野。無貳無虞,上帝臨女。}{\footnotesize 虞,誤也。箋云屆極、虞度也。文王武王繼大王之事,至受命,致天所罰,極紂於商郊牧野,其時之民皆樂武王之如是,故戒之云,無有二心也,無復計度也,天視護女,至則克勝。}\textbf{敦商之旅,克咸厥功。}{\footnotesize 箋云敦治、旅眾、咸同也。武王克殷而治商之臣民,使得其所,能同其功於先祖也。后稷、大王、文王亦周公之祖考也,伐紂,周公又與焉,故述之以美大魯。}\textbf{王曰叔父,建爾元子,俾侯于魯,大啟爾宇,為周室輔。}{\footnotesize 王,成王也。元首、宇居也。箋云叔父,謂周公也。成王告周公曰「叔父,我立女首子,使為君於魯」,謂欲封伯禽也,封魯公以為周公後,故云大開女居,以為我周家之輔,謂封以方七百里,欲其彊於眾國。}

\begin{quoting}\textbf{馬瑞辰}此詩敦亦當讀屯,屯,聚也,蓋自聚其師旅為聚,俘虜敵之士眾,亦為屯聚之也,克咸厥功,猶云克備厥功,亦即克成厥功也。\end{quoting}

\textbf{乃命魯公,俾侯于東。錫之山川,土田附庸。}{\footnotesize 箋云東,東藩,魯國也。既告周公以封伯禽之意,乃策命伯禽,使為君於東,加賜之以山川土田及附庸,令專統之,王制曰「名山大川不以封諸侯,附庸則不得專臣也」。}\textbf{周公之孫,莊公之子。龍旂承祀,六轡耳耳。春秋匪解,享祀不忒。}{\footnotesize 周公之孫,莊公之子,謂僖公也。耳耳然,至盛也。箋云交龍為旂。承祀,謂視祭事也。四馬故六轡。春秋,猶言四時也。忒,變也。}\textbf{皇皇后帝,皇祖后稷。享以騂犧,是饗是宜,降福既多。}{\footnotesize 騂赤、犧純也。箋云皇皇后帝,謂天也。成王以周公功大,命魯郊祭天,亦配之以君祖后稷,其牲用赤牛純色,與天子同也,天亦饗之宜之,多予之福。}\textbf{周公皇祖,亦其福女。}{\footnotesize 箋云此皇祖謂伯禽也。}

\begin{quoting}饗、宜皆祭名,\textbf{馬瑞辰}按宜本祭社之名,爾雅釋天「起大事、動大眾必先有事乎社而後出,謂之宜」,孫炎注「宜,求見福佑也」是也。\end{quoting}

\textbf{秋而載嘗,夏而楅衡。白牡騂剛,犧尊將將。毛炰胾羹,籩豆大房。萬舞洋洋,孝孫有慶。}{\footnotesize 諸侯夏禘則不礿,秋祫則不嘗,唯天子兼之。楅衡,設牛角以楅之也。白牡,周公牲也。騂剛,魯公牲也。犧尊,有沙飾也。毛炰,豚也。胾,肉也。羹,大羹、鉶羹也。大房,半體之俎也。洋洋,眾多也。箋云載,始也。秋將嘗祭,於夏則養牲,楅衡其牛角,為其觸觝人也。秋嘗而言始者,秋物新成,尚之也。大房,玉飾俎也,其制足間有橫,下有拊,似乎堂後有房然。萬舞,干舞也。}\textbf{俾爾熾而昌,俾爾壽而臧。保彼東方,魯邦是常。不虧不崩,不震不騰。三壽作朋,如岡如陵。}{\footnotesize 震,動也。騰,乘也。壽,考也。箋云此皆慶孝孫之辭也。俾使、臧善、保安、常守也。虧、崩皆謂毀壞也。震、騰皆謂僭踰相侵犯也。三壽,三卿也。岡、陵取堅固也。}

\begin{quoting}楅 \texttt{bì} 衡,說文「衡,牛觸衡大木」,段注「是闌閈之謂」。炰 \texttt{páo}。\textbf{朱熹}胾 \texttt{zì},切肉也。孔疏「震、騰以川喻」。三壽,上中下壽也。\end{quoting}

\textbf{公車千乘,朱英綠縢,二矛重弓。}{\footnotesize 大國之賦千乘。朱英,矛飾也。縢,繩也。重弓,重於鬯中也。箋云二矛重弓,備折壞也。兵車之法,左人持弓,右人持矛,中人御。}\textbf{公徒三萬,貝胄朱綅,烝徒增增。}{\footnotesize 貝胄,貝飾也。朱綅,以朱綅綴之。增增,眾也。箋云萬二千五百人為軍,大國三軍,合三萬七千五百人,言三萬者,舉成數也。烝,進也。徒進行增增然。}\textbf{戎狄是膺,荊舒是懲,則莫我敢承。}{\footnotesize 膺當、承止也。箋云懲,艾也。僖公與齊桓舉義兵,北當戎與狄,南艾荊及群舒,天下無敢禦之。}\textbf{俾爾昌而熾,俾爾壽而富。黃髮台背,壽胥與試。}{\footnotesize 箋云此慶僖公勇於用兵,討有罪也。黃髮台背,皆壽徵也。胥,相也。壽而相與試,謂講氣力,不衰倦。}\textbf{俾爾昌而大,俾爾耆而艾。萬有千歲,眉壽無有害。}{\footnotesize 箋云此又慶僖公勇於用兵,討有罪也。中時魯微弱,為鄰邦所侵削,今乃復其故,故喜而重慶之。俾爾,猶使女也。眉壽,秀眉亦壽徵。}

\begin{quoting}\textbf{馬瑞辰}古制蓋以五百乘為一軍,此詩公車千乘,謂次國二軍也。又曰按朱綅 \texttt{qīn} 承貝胄言,段玉裁言毛意謂以朱綫綴貝於冑是也。膺,魯詩作應。\textbf{馬瑞辰}試,猶式也,字通作視,廣雅「視,比也」,比之言比儗也,壽胥與試,承黃髮台背言,猶云壽相與比耳。\end{quoting}

\textbf{泰山巖巖,魯邦所詹。奄有龜蒙,遂荒大東。至于海邦,淮夷來同。莫不率從,魯侯之功。}{\footnotesize 詹,至也。龜,山也。蒙,山也。荒,有也。箋云奄覆、荒奄也。大東,極東。海邦,近海之國也。來同,為同盟也。率從,相率從於中國也。魯侯,謂僖公。}

\begin{quoting}詹,同瞻。\textbf{馬瑞辰}來,語詞,淮夷來同,猶大雅「徐方既同」也,同,亦朝會之通名。率,順也。\end{quoting}

\textbf{保有鳧繹,遂荒徐宅,至于海邦。淮夷蠻貊,及彼南夷,莫不率從。莫敢不諾,魯侯是若。}{\footnotesize 鳧,山也。繹,山也。宅,居也。淮夷蠻貊,如夷行也。南夷,荊楚也。若,順也。箋云諾,應辭也,是若者,是僖公所謂順也。}

\textbf{天錫公純嘏,眉壽保魯。居常與許,復周公之宇。}{\footnotesize 常、許,魯南鄙、西鄙。箋云純,大也。受福曰嘏。許,許田也,魯朝宿之邑也。常,或作嘗,在薛之旁,春秋魯莊公三十一年「築臺于薛」是與,周公有嘗邑,所由未聞也,六國時齊有孟嘗君,食邑於薛。}\textbf{魯侯燕喜,令妻壽母。宜大夫庶士,邦國是有。既多受祉,黃髮兒齒。}{\footnotesize 箋云燕,燕飲也。令,善也。僖公燕飲於內寢,則善其妻、壽其母,謂為之祝慶也,與群臣燕,則欲與之相宜,亦祝慶也。是有,猶常有也。兒齒,亦壽徵。}

\begin{quoting}兒,同齯 \texttt{ní},說文「老人齒也」。\end{quoting}

\textbf{徂來之松,新甫之柏,是斷是度,是尋是尺。}{\footnotesize 徂來,山也。新甫,山也。八尺曰尋。}\textbf{松桷有舄,路寢孔碩,新廟奕奕,奚斯所作。}{\footnotesize 桷,榱也。舄,大貌。路寢,正寢也。新廟,閔公廟也,有大夫公子奚斯者作是廟也。箋云孔甚、碩大也。奕奕,佼美也。修舊曰新,新者,姜嫄廟也。僖公承衰亂之政,修周公伯禽之教,故治正寢,上新姜嫄之廟,姜嫄之廟,廟之先也,奚斯作者,教護屬功課章程也。至文公之時,大室屋壞。}\textbf{孔曼且碩,萬民是若。}{\footnotesize 曼,長也。箋云曼,修也廣也。且,然也。國人謂之順也。}

\begin{quoting}新甫,梁父也。\textbf{陳奐}路寢居宮之中央,左社稷而右宗廟,故經言路寢必連及新廟也。新廟,魯詩、齊詩作寢廟。\textbf{馬瑞辰}猶崧高詩「其詩孔碩,其風肆好」也。\end{quoting}

%\begin{flushright}駉四篇、二十三章、二百四十三句\end{flushright}