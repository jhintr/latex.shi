\chapter{惑溺第三十五}

\subsection*{1}

\textbf{魏甄后惠而有色,先為袁熙妻,甚獲寵,曹公之屠鄴也,令疾召甄,左右白:「五官中郎已將去。」公曰:「今年破賊正為奴。」}{\footnotesize \textbf{魏略}曰建安中,袁紹為中子熙娶甄會女,紹死,熙出任幽州,甄留侍姑,及鄴城破,五官將從而入紹舍,甄驚怖,以頭伏姑厀上,五官將謂紹妻袁夫人「扶甄令舉頭」,見其色非凡,稱歎之,太祖聞其意,遂為迎娶,擅室數歲。\textbf{世語}曰太祖下鄴,文帝先入袁尚府,見婦人被髮垢面垂涕,立紹妻劉後,文帝問,知是熙妻,使令攬髮,以袖拭面,姿貌絕倫,既過,劉謂甄曰「不復死矣」,遂納之,有寵。\textbf{魏氏春秋}曰五官將納熙妻也,孔融與太祖書曰「武王伐紂,以妲己賜周公」,太祖以融博學,真謂書傳所記,後見融問之,對曰「以今度古,想其然也」。}

\subsection*{2}

\textbf{荀奉倩與婦至篤,冬月婦病熱,乃出中庭自取冷,還以身熨之,婦亡,奉倩後少時亦卒,以是獲譏於世,}{\footnotesize \textbf{粲別傳}曰粲常以婦人才智不足論,自宜以色為主,驃騎將軍曹洪女有色,粲於是聘焉,容服帷帳甚麗,專房燕婉,歷年後婦病亡,未殯,傅嘏往喭粲,粲不明而神傷,嘏問曰「婦人才色並茂為難,子之聘也,遺才存色,非難遇也,何哀之甚」,粲曰「佳人難再得,顧逝者不能有傾城之異,然未可易遇也」,痛悼不能已已,歲餘亦亡,亡時年二十九,粲簡貴,不與常人交接,所交者一時俊傑,至葬夕,赴期者裁十餘人,悉同年相知名士也,哭之,感慟路人,粲雖褊隘,以燕婉自喪,然有識猶追惜其能言。}\textbf{奉倩曰:「婦人德不足稱,當以色為主。」裴令聞之曰:「此乃是興到之事,非盛德言,冀後人未昧此語。」}{\footnotesize \textbf{何劭}論粲曰仲尼稱「有德者有言」,而荀粲減於是,內顧所言有餘,而識不足。}

\subsection*{3}

\textbf{賈公閭}{\footnotesize \textbf{充別傳}曰充父逵,晚有子,故名曰充,字公閭,言後必有充閭之異。}\textbf{後妻郭氏酷妒,有男兒名黎民,生載周,充自外還,乳母抱兒在中庭,兒見充喜踊,充就乳母手中嗚之,郭遙望見,謂充愛乳母,即殺之,兒悲思啼泣,不飲它乳,遂死,郭後終無子。}{\footnotesize \textbf{晉諸公贊}云郭氏即賈后母也,為性高朗,知后無子,甚憂愛愍懷,每勸厲之,臨亡,誨賈后,令盡意於太子,言甚切至,趙充華及賈謐母並勿令出入宮中,又曰「此皆亂汝事」,后不能用,終至誅夷。\textbf{臣按}傅暢此言,則郭氏賢明婦人也,向令賈后撫愛愍懷,豈當縱其妒悍,自斃其子,然則物我不同,或老壯情異乎?}

\subsection*{4}

\textbf{孫秀降晉,晉武帝厚存寵之,}{\footnotesize \textbf{太原郭氏錄}曰秀,字彥才,吳郡吳人,為下口督,甚有威恩,孫皓憚欲除之,遣將軍何定溯江而上,辭以捕鹿三千口供廚,秀豫知謀,遂來歸化,世祖喜之,以為驃騎將軍、交州牧。}\textbf{妻以姨妹蒯氏,室家甚篤,妻嘗妒,乃罵秀為「貉子」,}{\footnotesize \textbf{晉陽秋}曰蒯氏,襄陽人,祖良,吏部尚書,父鈞,南陽太守。}\textbf{秀大不平,遂不復入,蒯氏大自悔責,請救於帝,時大赦,群臣咸見,既出,帝獨留秀,從容謂曰:「天下曠蕩,蒯夫人可得從其例不?」秀免冠而謝,遂為夫婦如初。}

\subsection*{5}

\textbf{韓壽美姿容,賈充辟以為掾,充每聚會,賈女於青璅中看,見壽,說之,恆懷存想,發於吟詠,後婢往壽家,具述如此,并言女光麗,壽聞之心動,遂請婢潛修音問,及期往宿,壽蹻捷絕人,踰牆而入,家中莫知,}{\footnotesize \textbf{晉諸公贊}曰壽,字德真,南陽赭陽人,曾祖暨,魏司徒,有高行。壽敦家風,性忠厚,豈有若斯之事?諸書無聞,唯見世說,自未可信。}\textbf{自是充覺女盛自拂拭,說暢有異於常,後會諸吏,聞壽有奇香之氣,是外國所貢,一著人,則歷月不歇,}{\footnotesize \textbf{十洲記}曰漢武帝時,西域月氏國王遣使獻香四兩,大如雀卵,黑如桑椹,燒之,芳氣經三月不歇。蓋此香也。}\textbf{充計武帝唯賜己及陳騫,餘家無此香,疑壽與女通,而垣牆重密,門閤急峻,何由得爾?乃託言有盜,令人修牆,使反曰:「其餘無異,唯東北角如有人跡,而牆高,非人所踰。」充乃取女左右婢考問,即以狀對,充秘之,以女妻壽。}{\footnotesize 郭子謂與韓壽通者,乃是陳騫女,即以妻壽,未婚而女亡,壽因娶賈氏,故世因傳是充女。}

\subsection*{6}

\textbf{王安豐婦常卿安豐,安豐曰:「婦人卿壻,於禮為不敬,後勿復爾。」婦曰:「親卿愛卿,是以卿卿,我不卿卿,誰當卿卿?」遂恆聽之。}

\subsection*{7}

\textbf{王丞相有幸妾姓雷,頗預政事納貨,蔡公謂之「雷尚書」。}{\footnotesize \textbf{語林}曰雷有寵,生恬、洽。}