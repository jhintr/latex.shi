\chapter{汰侈第三十}

\subsection*{1}

\textbf{石崇每要客燕集,常令美人行酒,客飲酒不盡者,使黃門交斬美人,王丞相與大將軍嘗共詣崇,丞相素不能飲,輒自勉彊,至於沈醉,每至大將軍,固不飲,以觀其變,已斬三人,顏色如故,尚不肯飲,丞相讓之,大將軍曰:「自殺伊家人,何預卿事?」}{\footnotesize \textbf{王隱晉書}曰石崇為荊州刺史,劫奪殺人,以致巨富。\textbf{王丞相德音記}曰丞相素為諸父所重,王君夫問王敦「聞君從弟佳人,又解音律,欲一作妓,可與共來」,遂往,吹笛人有小忘,君夫聞,使黃門階下打殺之,顏色不變,丞相還,曰「恐此君處世,當有如此事」。兩說不同,故詳錄。}

\subsection*{2}

\textbf{石崇廁,常有十餘婢侍列,皆麗服藻飾,置甲煎粉、沈香汁之屬,無不畢備,又與新衣著令出,客多羞不能如廁,王大將軍往,脫故衣,著新衣,神色傲然,群婢相謂曰:「此客必能作賊。」}{\footnotesize \textbf{語林}曰劉寔詣石崇,如廁,見有絳紗帳大牀,茵蓐甚麗,兩婢持錦香囊,寔遽反走,即謂崇曰「向誤入卿室內」,崇曰「是廁耳」。}

\subsection*{3}

\textbf{武帝嘗降王武子家,武子供饌,並用瑠璃器,婢子百餘人,皆綾羅絝𧟌,以手擎飲食,烝㹠肥美,異於常味,帝怪而問之,答曰:「以人乳飲㹠。」帝甚不平,食未畢便去,王、石所未知作。}{\footnotesize 𧟌,一作襬。}

\subsection*{4}

\textbf{王君夫以飴糒澳釜,石季倫用蠟燭作炊,君夫作紫絲布步障碧綾裏四十里,石崇作錦步障五十里以敵之,石以椒為泥,王以赤石脂泥壁。}{\footnotesize \textbf{晉諸公贊}曰王愷,字君夫,東海人,王肅子也,雖無檢行,而少以才力見名,有在公之稱,既自以外戚,晉氏政寬,又性至豪,舊制,鴆不得過江,為其羽櫟酒中,必殺人,愷為翊軍時,得鴆於石崇而養之,其大如鵝,喙長尺餘,純食蛇虺,司隸奏按愷、崇,詔悉原之,即燒於都街,愷肆其意色,無所忌憚,為後軍將軍,卒,諡曰醜。}

\subsection*{5}

\textbf{石崇為客作豆粥,咄嗟便辦,恆冬天得韭蓱䪢,又牛形狀氣力不勝王愷牛,而與愷出遊,極晚發,爭入洛城,崇牛數十步後,迅若飛禽,愷牛絕走不能及,每以此三事為搤腕,乃密貨崇帳下都督及御車人,問所以,都督曰:「豆至難煮,唯豫作熟末,客至,作白粥以投之,韭蓱䪢是搗韭根,雜以麥苗爾。」復問馭人牛所以駛,馭人云:「牛本不遲,由將車人不及制之爾,急時聽偏轅,則駛矣。」愷悉從之,遂爭長,石崇後聞,皆殺告者。}{\footnotesize \textbf{晉諸公贊}曰崇性好俠,與王愷競相誇衒也。}

\subsection*{6}

\textbf{王君夫有牛,名「八百里駮」,常瑩其蹄角,王武子語君夫:「我射不如卿,今指賭卿牛,以千萬對之。」君夫既恃手快,且謂駿物無有殺理,便相然可,令武子先射,武子一起便破的,卻據胡牀,叱左右「速探牛心來」,須臾,炙至,一臠便去。}{\footnotesize \textbf{相牛經}曰牛經出甯戚,傳百里奚,漢世河西薛公得其書,以相牛,千百不失,本以負重致遠,未服輜軿,故文不傳,至魏世,高堂生又傳以與晉宣帝,其後王愷得其書焉。臣按其相經云「陰虹屬頸,千里」,注曰「陰虹者,雙筋自尾骨屬頸,甯戚所飯者也」,愷之牛亦有陰虹也,甯戚經曰「棰頭欲得高,百體欲得緊,大膁疎肋難齝,龍頭突目好跳,又角欲得細,身欲促,形欲得如卷」。}

\subsection*{7}

\textbf{王君夫嘗責一人無服餘衵,因直內著曲閤重閨裏,不聽人將出,遂饑經日,迷不知何處去,後因緣相為,垂死,迺得出。}

\subsection*{8}

\textbf{石崇與王愷爭豪,並窮綺麗,以飾輿服,}{\footnotesize \textbf{續文章志}曰崇資產累巨萬金,宅室輿馬,僭擬王者,庖膳必窮水陸之珍,後房百數,皆曳紈繡、珥金翠,而絲竹之蓺,盡一世之選,築榭開沼,殫極人巧,與貴戚羊琇、王愷之徒競相高以侈靡,而崇為居最之首,琇等每愧羡,以為不及也。}\textbf{武帝,愷之甥也,每助愷,嘗以一珊瑚樹高二尺許賜愷,枝柯扶疎,世罕其比,愷以示崇,崇視訖,以鐵如意擊之,應手而碎,愷既惋惜,又以為疾己之寶,聲色甚厲,崇曰:「不足恨,今還卿。」乃命左右悉取珊瑚樹,有三尺四尺、條幹絕世、光彩溢目者六七枚,如愷許比甚眾,愷惘然自失。}{\footnotesize \textbf{南州異物志}曰珊瑚生大秦國,有洲在漲海中,距其國七八百里,名珊瑚樹洲,底有盤石,水深二十餘丈,珊瑚生於石上,初生白,軟弱似菌,國人乘大船,載鐵網,先沒在水下,一年便生網目中,其色尚黃,枝柯交錯,高三四尺,大者圍尺餘,三年色赤,便以鐵鈔發其根,繫鐵網於船,絞車舉網,還,裁鑿恣意所作,若過時不鑿,便枯索蟲蠱,其大者輸之王府,細者賣之。\textbf{廣志}曰珊瑚大者,可為車軸。}

\subsection*{9}

\textbf{王武子被責,移第北邙下,}{\footnotesize \textbf{晉諸公贊}曰濟與從兄恬不平,濟為河南尹,未拜,行過王宮,吏不時下道,濟於車前鞭之,有司奏免官,論者以濟為不長者,尋轉太僕,而王恬已見委任,濟遂斥外。}\textbf{于時人多地貴,濟好馬射,買地作埒,編錢帀地竟埒,時人號曰「金溝」。}{\footnotesize 溝,一作埒。}

\subsection*{10}

\textbf{石崇每與王敦入學戲,見顏、原象,}{\footnotesize \textbf{家語}曰顏回,字子淵,魯人,少孔子二十九歲而髮白,三十二歲蚤死。原憲已見。}\textbf{而歎曰:「若與同升孔堂,去人何必有間?」王曰:「不知餘人云何,子貢去卿差近。」}{\footnotesize \textbf{史記}曰端木賜,字子貢,衛人,嘗相魯,家累千金,終於齊。}\textbf{石正色云:「士當令身名俱泰,何至以甕牖語人?」}{\footnotesize 原憲以甕為巨牖。}

\subsection*{11}

\textbf{彭城王有快牛,至愛惜之,}{\footnotesize \textbf{朱鳳晉書}曰彭城穆王權,字子輿,宣帝弟馗子,太始元年封。}\textbf{王太尉與射,賭得之,彭城王曰:「君欲自乘則不論,若欲噉者,當以二十肥者代之,既不廢噉,又存所愛。」王遂殺噉。}

\subsection*{12}

\textbf{王右軍少時,在周侯末坐,割牛心噉之,於此改觀。}{\footnotesize 俗以牛心為貴,故羲之先飡之。}